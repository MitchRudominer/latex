\documentclass[oneside,12pt]{amsart}

\usepackage{amsmath,amssymb,latexsym,eucal,amsthm}

%%%%%%%%%%%%%%%%%%%%%%%%%%%%%%%%%%%%%%%%%%%%%
% Common Set Theory Constructs
%%%%%%%%%%%%%%%%%%%%%%%%%%%%%%%%%%%%%%%%%%%%%

\newcommand{\setof}[2]{\left\{ \, #1 \, \left| \, #2 \, \right.\right\}}
\newcommand{\lsetof}[2]{\left\{\left. \, #1 \, \right| \, #2 \,  \right\}}
\newcommand{\bigsetof}[2]{\bigl\{ \, #1 \, \bigm | \, #2 \,\bigr\}}
\newcommand{\Bigsetof}[2]{\Bigl\{ \, #1 \, \Bigm | \, #2 \,\Bigr\}}
\newcommand{\biggsetof}[2]{\biggl\{ \, #1 \, \biggm | \, #2 \,\biggr\}}
\newcommand{\Biggsetof}[2]{\Biggl\{ \, #1 \, \Biggm | \, #2 \,\Biggr\}}
\newcommand{\dotsetof}[2]{\left\{ \, #1 \, : \, #2 \, \right\}}
\newcommand{\bigdotsetof}[2]{\bigl\{ \, #1 \, : \, #2 \,\bigr\}}
\newcommand{\Bigdotsetof}[2]{\Bigl\{ \, #1 \, \Bigm : \, #2 \,\Bigr\}}
\newcommand{\biggdotsetof}[2]{\biggl\{ \, #1 \, \biggm : \, #2 \,\biggr\}}
\newcommand{\Biggdotsetof}[2]{\Biggl\{ \, #1 \, \Biggm : \, #2 \,\Biggr\}}
\newcommand{\sequence}[2]{\left\langle \, #1 \,\left| \, #2 \, \right. \right\rangle}
\newcommand{\lsequence}[2]{\left\langle\left. \, #1 \, \right| \,#2 \,  \right\rangle}
\newcommand{\bigsequence}[2]{\bigl\langle \,#1 \, \bigm | \, #2 \, \bigr\rangle}
\newcommand{\Bigsequence}[2]{\Bigl\langle \,#1 \, \Bigm | \, #2 \, \Bigr\rangle}
\newcommand{\biggsequence}[2]{\biggl\langle \,#1 \, \biggm | \, #2 \, \biggr\rangle}
\newcommand{\Biggsequence}[2]{\Biggl\langle \,#1 \, \Biggm | \, #2 \, \Biggr\rangle}
\newcommand{\singleton}[1]{\left\{#1\right\}}
\newcommand{\angles}[1]{\left\langle #1 \right\rangle}
\newcommand{\bigangles}[1]{\bigl\langle #1 \bigr\rangle}
\newcommand{\Bigangles}[1]{\Bigl\langle #1 \Bigr\rangle}
\newcommand{\biggangles}[1]{\biggl\langle #1 \biggr\rangle}
\newcommand{\Biggangles}[1]{\Biggl\langle #1 \Biggr\rangle}


\newcommand{\force}[1]{\Vert\!\underset{\!\!\!\!\!#1}{\!\!\!\relbar\!\!\!%
\relbar\!\!\relbar\!\!\relbar\!\!\!\relbar\!\!\relbar\!\!\relbar\!\!\!%
\relbar\!\!\relbar\!\!\relbar}}
\newcommand{\longforce}[1]{\Vert\!\underset{\!\!\!\!\!#1}{\!\!\!\relbar\!\!\!%
\relbar\!\!\relbar\!\!\relbar\!\!\!\relbar\!\!\relbar\!\!\relbar\!\!\!%
\relbar\!\!\relbar\!\!\relbar\!\!\relbar\!\!\relbar\!\!\relbar\!\!\relbar\!\!\relbar}}
\newcommand{\nforce}[1]{\Vert\!\underset{\!\!\!\!\!#1}{\!\!\!\relbar\!\!\!%
\relbar\!\!\relbar\!\!\relbar\!\!\!\relbar\!\!\relbar\!\!\relbar\!\!\!%
\relbar\!\!\not\relbar\!\!\relbar}}
\newcommand{\forcein}[2]{\overset{#2}{\Vert\underset{\!\!\!\!\!#1}%
{\!\!\!\relbar\!\!\!\relbar\!\!\relbar\!\!\relbar\!\!\!\relbar\!\!\relbar\!%
\!\relbar\!\!\!\relbar\!\!\relbar\!\!\relbar\!\!\relbar\!\!\!\relbar\!\!%
\relbar\!\!\relbar}}}

\newcommand{\pre}[2]{{}^{#2}{#1}}

\newcommand{\restr}{\!\!\upharpoonright\!}

%%%%%%%%%%%%%%%%%%%%%%%%%%%%%%%%%%%%%%%%%%%%%
% Set-Theoretic Connectives
%%%%%%%%%%%%%%%%%%%%%%%%%%%%%%%%%%%%%%%%%%%%%

\newcommand{\intersect}{\cap}
\newcommand{\union}{\cup}
\newcommand{\Intersection}[1]{\bigcap\limits_{#1}}
\newcommand{\Union}[1]{\bigcup\limits_{#1}}
\newcommand{\adjoin}{{}^\frown}
\newcommand{\forces}{\Vdash}

%%%%%%%%%%%%%%%%%%%%%%%%%%%%%%%%%%%%%%%%%%%%%
% Miscellaneous
%%%%%%%%%%%%%%%%%%%%%%%%%%%%%%%%%%%%%%%%%%%%%
\newcommand{\defeq}{=_{\text{def}}}
\newcommand{\Turingleq}{\leq_{\text{T}}}
\newcommand{\Turingless}{<_{\text{T}}}
\newcommand{\lexleq}{\leq_{\text{lex}}}
\newcommand{\lexless}{<_{\text{lex}}}
\newcommand{\Turingequiv}{\equiv_{\text{T}}}
\newcommand{\isomorphic}{\cong}

%%%%%%%%%%%%%%%%%%%%%%%%%%%%%%%%%%%%%%%%%%%%%
% Constants
%%%%%%%%%%%%%%%%%%%%%%%%%%%%%%%%%%%%%%%%%%%%%
\newcommand{\R}{\mathbb{R}}
\renewcommand{\P}{\mathbb{P}}
\newcommand{\Q}{\mathbb{Q}}
\newcommand{\Z}{\mathbb{Z}}
\newcommand{\Zpos}{\Z^{+}}
\newcommand{\Znonneg}{\Z^{\geq 0}}
\newcommand{\C}{\mathbb{C}}
\newcommand{\N}{\mathbb{N}}
\newcommand{\B}{\mathbb{B}}
\newcommand{\Bairespace}{\pre{\omega}{\omega}}
\newcommand{\LofR}{L(\R)}
\newcommand{\JofR}[1]{J_{#1}(\R)}
\newcommand{\SofR}[1]{S_{#1}(\R)}
\newcommand{\JalphaR}{\JofR{\alpha}}
\newcommand{\JbetaR}{\JofR{\beta}}
\newcommand{\JlambdaR}{\JofR{\lambda}}
\newcommand{\SalphaR}{\SofR{\alpha}}
\newcommand{\SbetaR}{\SofR{\beta}}
\newcommand{\Pkl}{\mathcal{P}_{\kappa}(\lambda)}
\DeclareMathOperator{\con}{con}
\DeclareMathOperator{\ORD}{OR}
\DeclareMathOperator{\Ord}{OR}
\DeclareMathOperator{\WO}{WO}
\DeclareMathOperator{\OD}{OD}
\DeclareMathOperator{\HOD}{HOD}
\DeclareMathOperator{\HC}{HC}
\DeclareMathOperator{\WF}{WF}
\DeclareMathOperator{\wfp}{wfp}
\DeclareMathOperator{\HF}{HF}
\newcommand{\One}{I}
\newcommand{\ONE}{I}
\newcommand{\Two}{II}
\newcommand{\TWO}{II}
\newcommand{\Mladder}{M^{\text{ld}}}

%%%%%%%%%%%%%%%%%%%%%%%%%%%%%%%%%%%%%%%%%%%%%
% Commutative Algebra Constants
%%%%%%%%%%%%%%%%%%%%%%%%%%%%%%%%%%%%%%%%%%%%%
\DeclareMathOperator{\dottimes}{\dot{\times}}
\DeclareMathOperator{\dotminus}{\dot{-}}
\DeclareMathOperator{\Spec}{Spec}

%%%%%%%%%%%%%%%%%%%%%%%%%%%%%%%%%%%%%%%%%%%%%
% Theories
%%%%%%%%%%%%%%%%%%%%%%%%%%%%%%%%%%%%%%%%%%%%%
\DeclareMathOperator{\ZFC}{ZFC}
\DeclareMathOperator{\ZF}{ZF}
\DeclareMathOperator{\AD}{AD}
\DeclareMathOperator{\ADR}{AD_{\R}}
\DeclareMathOperator{\KP}{KP}
\DeclareMathOperator{\PD}{PD}
\DeclareMathOperator{\CH}{CH}
\DeclareMathOperator{\GCH}{GCH}
\DeclareMathOperator{\HPC}{HPC} % HOD pair capturing
%%%%%%%%%%%%%%%%%%%%%%%%%%%%%%%%%%%%%%%%%%%%%
% Iteration Trees
%%%%%%%%%%%%%%%%%%%%%%%%%%%%%%%%%%%%%%%%%%%%%

\newcommand{\pred}{\text{-pred}}

%%%%%%%%%%%%%%%%%%%%%%%%%%%%%%%%%%%%%%%%%%%%%%%%
% Operator Names
%%%%%%%%%%%%%%%%%%%%%%%%%%%%%%%%%%%%%%%%%%%%%%%%
\DeclareMathOperator{\Det}{Det}
\DeclareMathOperator{\dom}{dom}
\DeclareMathOperator{\ran}{ran}
\DeclareMathOperator{\range}{ran}
\DeclareMathOperator{\image}{image}
\DeclareMathOperator{\crit}{crit}
\DeclareMathOperator{\card}{card}
\DeclareMathOperator{\cf}{cf}
\DeclareMathOperator{\cof}{cof}
\DeclareMathOperator{\rank}{rank}
\DeclareMathOperator{\ot}{o.t.}
\DeclareMathOperator{\ords}{o}
\DeclareMathOperator{\ro}{r.o.}
\DeclareMathOperator{\rud}{rud}
\DeclareMathOperator{\Powerset}{\mathcal{P}}
\DeclareMathOperator{\length}{lh}
\DeclareMathOperator{\lh}{lh}
\DeclareMathOperator{\limit}{lim}
\DeclareMathOperator{\fld}{fld}
\DeclareMathOperator{\projection}{p}
\DeclareMathOperator{\Ult}{Ult}
\DeclareMathOperator{\ULT}{Ult}
\DeclareMathOperator{\Coll}{Coll}
\DeclareMathOperator{\Col}{Col}
\DeclareMathOperator{\Hull}{Hull}
\DeclareMathOperator*{\dirlim}{dir lim}
\DeclareMathOperator{\Scale}{Scale}
\DeclareMathOperator{\supp}{supp}
\DeclareMathOperator{\trancl}{tran.cl.}
\DeclareMathOperator{\trace}{Tr}
\DeclareMathOperator{\diag}{diag}
\DeclareMathOperator{\spn}{span}
\DeclareMathOperator{\sgn}{sgn}
%%%%%%%%%%%%%%%%%%%%%%%%%%%%%%%%%%%%%%%%%%%%%
% Logical Connectives
%%%%%%%%%%%%%%%%%%%%%%%%%%%%%%%%%%%%%%%%%%%%%
\newcommand{\IImplies}{\Longrightarrow}
\newcommand{\SkipImplies}{\quad\Longrightarrow\quad}
\newcommand{\Ifff}{\Longleftrightarrow}
\newcommand{\iimplies}{\longrightarrow}
\newcommand{\ifff}{\longleftrightarrow}
\newcommand{\Implies}{\Rightarrow}
\newcommand{\Iff}{\Leftrightarrow}
\renewcommand{\implies}{\rightarrow}
\renewcommand{\iff}{\leftrightarrow}
\newcommand{\AND}{\wedge}
\newcommand{\OR}{\vee}
\newcommand{\st}{\text{ s.t. }}
\newcommand{\Or}{\text{ or }}

%%%%%%%%%%%%%%%%%%%%%%%%%%%%%%%%%%%%%%%%%%%%%
% Function Arrows
%%%%%%%%%%%%%%%%%%%%%%%%%%%%%%%%%%%%%%%%%%%%%

\newcommand{\injection}{\xrightarrow{\text{1-1}}}
\newcommand{\surjection}{\xrightarrow{\text{onto}}}
\newcommand{\bijection}{\xrightarrow[\text{onto}]{\text{1-1}}}
\newcommand{\cofmap}{\xrightarrow{\text{cof}}}
\newcommand{\map}{\rightarrow}

%%%%%%%%%%%%%%%%%%%%%%%%%%%%%%%%%%%%%%%%%%%%%
% Mouse Comparison Operators
%%%%%%%%%%%%%%%%%%%%%%%%%%%%%%%%%%%%%%%%%%%%%
\newcommand{\initseg}{\trianglelefteq}
\newcommand{\properseg}{\lhd}
\newcommand{\notinitseg}{\ntrianglelefteq}
\newcommand{\notproperseg}{\ntriangleleft}

%%%%%%%%%%%%%%%%%%%%%%%%%%%%%%%%%%%%%%%%%%%%%
%           calligraphic letters
%%%%%%%%%%%%%%%%%%%%%%%%%%%%%%%%%%%%%%%%%%%%%
\newcommand{\cA}{\mathcal{A}}
\newcommand{\cB}{\mathcal{B}}
\newcommand{\cC}{\mathcal{C}}
\newcommand{\cD}{\mathcal{D}}
\newcommand{\cE}{\mathcal{E}}
\newcommand{\cF}{\mathcal{F}}
\newcommand{\cG}{\mathcal{G}}
\newcommand{\cH}{\mathcal{H}}
\newcommand{\cI}{\mathcal{I}}
\newcommand{\cJ}{\mathcal{J}}
\newcommand{\cK}{\mathcal{K}}
\newcommand{\cL}{\mathcal{L}}
\newcommand{\cM}{\mathcal{M}}
\newcommand{\cN}{\mathcal{N}}
\newcommand{\cO}{\mathcal{O}}
\newcommand{\cP}{\mathcal{P}}
\newcommand{\cQ}{\mathcal{Q}}
\newcommand{\cR}{\mathcal{R}}
\newcommand{\cS}{\mathcal{S}}
\newcommand{\cT}{\mathcal{T}}
\newcommand{\cU}{\mathcal{U}}
\newcommand{\cV}{\mathcal{V}}
\newcommand{\cW}{\mathcal{W}}
\newcommand{\cX}{\mathcal{X}}
\newcommand{\cY}{\mathcal{Y}}
\newcommand{\cZ}{\mathcal{Z}}


%%%%%%%%%%%%%%%%%%%%%%%%%%%%%%%%%%%%%%%%%%%%%
%          Primed Letters
%%%%%%%%%%%%%%%%%%%%%%%%%%%%%%%%%%%%%%%%%%%%%
\newcommand{\aprime}{a^{\prime}}
\newcommand{\bprime}{b^{\prime}}
\newcommand{\cprime}{c^{\prime}}
\newcommand{\dprime}{d^{\prime}}
\newcommand{\eprime}{e^{\prime}}
\newcommand{\fprime}{f^{\prime}}
\newcommand{\gprime}{g^{\prime}}
\newcommand{\hprime}{h^{\prime}}
\newcommand{\iprime}{i^{\prime}}
\newcommand{\jprime}{j^{\prime}}
\newcommand{\kprime}{k^{\prime}}
\newcommand{\lprime}{l^{\prime}}
\newcommand{\mprime}{m^{\prime}}
\newcommand{\nprime}{n^{\prime}}
\newcommand{\oprime}{o^{\prime}}
\newcommand{\pprime}{p^{\prime}}
\newcommand{\qprime}{q^{\prime}}
\newcommand{\rprime}{r^{\prime}}
\newcommand{\sprime}{s^{\prime}}
\newcommand{\tprime}{t^{\prime}}
\newcommand{\uprime}{u^{\prime}}
\newcommand{\vprime}{v^{\prime}}
\newcommand{\wprime}{w^{\prime}}
\newcommand{\xprime}{x^{\prime}}
\newcommand{\yprime}{y^{\prime}}
\newcommand{\zprime}{z^{\prime}}
\newcommand{\Aprime}{A^{\prime}}
\newcommand{\Bprime}{B^{\prime}}
\newcommand{\Cprime}{C^{\prime}}
\newcommand{\Dprime}{D^{\prime}}
\newcommand{\Eprime}{E^{\prime}}
\newcommand{\Fprime}{F^{\prime}}
\newcommand{\Gprime}{G^{\prime}}
\newcommand{\Hprime}{H^{\prime}}
\newcommand{\Iprime}{I^{\prime}}
\newcommand{\Jprime}{J^{\prime}}
\newcommand{\Kprime}{K^{\prime}}
\newcommand{\Lprime}{L^{\prime}}
\newcommand{\Mprime}{M^{\prime}}
\newcommand{\Nprime}{N^{\prime}}
\newcommand{\Oprime}{O^{\prime}}
\newcommand{\Pprime}{P^{\prime}}
\newcommand{\Qprime}{Q^{\prime}}
\newcommand{\Rprime}{R^{\prime}}
\newcommand{\Sprime}{S^{\prime}}
\newcommand{\Tprime}{T^{\prime}}
\newcommand{\Uprime}{U^{\prime}}
\newcommand{\Vprime}{V^{\prime}}
\newcommand{\Wprime}{W^{\prime}}
\newcommand{\Xprime}{X^{\prime}}
\newcommand{\Yprime}{Y^{\prime}}
\newcommand{\Zprime}{Z^{\prime}}
\newcommand{\alphaprime}{\alpha^{\prime}}
\newcommand{\betaprime}{\beta^{\prime}}
\newcommand{\gammaprime}{\gamma^{\prime}}
\newcommand{\Gammaprime}{\Gamma^{\prime}}
\newcommand{\deltaprime}{\delta^{\prime}}
\newcommand{\epsilonprime}{\epsilon^{\prime}}
\newcommand{\kappaprime}{\kappa^{\prime}}
\newcommand{\lambdaprime}{\lambda^{\prime}}
\newcommand{\rhoprime}{\rho^{\prime}}
\newcommand{\Sigmaprime}{\Sigma^{\prime}}
\newcommand{\tauprime}{\tau^{\prime}}
\newcommand{\xiprime}{\xi^{\prime}}
\newcommand{\thetaprime}{\theta^{\prime}}
\newcommand{\Omegaprime}{\Omega^{\prime}}
\newcommand{\cMprime}{\cM^{\prime}}
\newcommand{\cNprime}{\cN^{\prime}}
\newcommand{\cPprime}{\cP^{\prime}}
\newcommand{\cQprime}{\cQ^{\prime}}
\newcommand{\cRprime}{\cR^{\prime}}
\newcommand{\cSprime}{\cS^{\prime}}
\newcommand{\cTprime}{\cT^{\prime}}

%%%%%%%%%%%%%%%%%%%%%%%%%%%%%%%%%%%%%%%%%%%%%
%          bar Letters
%%%%%%%%%%%%%%%%%%%%%%%%%%%%%%%%%%%%%%%%%%%%%
\newcommand{\abar}{\bar{a}}
\newcommand{\bbar}{\bar{b}}
\newcommand{\cbar}{\bar{c}}
\newcommand{\ibar}{\bar{i}}
\newcommand{\jbar}{\bar{j}}
\newcommand{\nbar}{\bar{n}}
\newcommand{\xbar}{\bar{x}}
\newcommand{\ybar}{\bar{y}}
\newcommand{\zbar}{\bar{z}}
\newcommand{\pibar}{\bar{\pi}}
\newcommand{\phibar}{\bar{\varphi}}
\newcommand{\psibar}{\bar{\psi}}
\newcommand{\thetabar}{\bar{\theta}}
\newcommand{\nubar}{\bar{\nu}}

%%%%%%%%%%%%%%%%%%%%%%%%%%%%%%%%%%%%%%%%%%%%%
%          star Letters
%%%%%%%%%%%%%%%%%%%%%%%%%%%%%%%%%%%%%%%%%%%%%
\newcommand{\phistar}{\phi^{*}}
\newcommand{\Mstar}{M^{*}}

%%%%%%%%%%%%%%%%%%%%%%%%%%%%%%%%%%%%%%%%%%%%%
%          dotletters Letters
%%%%%%%%%%%%%%%%%%%%%%%%%%%%%%%%%%%%%%%%%%%%%
\newcommand{\Gdot}{\dot{G}}

%%%%%%%%%%%%%%%%%%%%%%%%%%%%%%%%%%%%%%%%%%%%%
%         check Letters
%%%%%%%%%%%%%%%%%%%%%%%%%%%%%%%%%%%%%%%%%%%%%
\newcommand{\deltacheck}{\check{\delta}}
\newcommand{\gammacheck}{\check{\gamma}}


%%%%%%%%%%%%%%%%%%%%%%%%%%%%%%%%%%%%%%%%%%%%%
%          Formulas
%%%%%%%%%%%%%%%%%%%%%%%%%%%%%%%%%%%%%%%%%%%%%

\newcommand{\formulaphi}{\text{\large $\varphi$}}
\newcommand{\Formulaphi}{\text{\Large $\varphi$}}


%%%%%%%%%%%%%%%%%%%%%%%%%%%%%%%%%%%%%%%%%%%%%
%          Fraktur Letters
%%%%%%%%%%%%%%%%%%%%%%%%%%%%%%%%%%%%%%%%%%%%%

\newcommand{\fa}{\mathfrak{a}}
\newcommand{\fb}{\mathfrak{b}}
\newcommand{\fc}{\mathfrak{c}}
\newcommand{\fk}{\mathfrak{k}}
\newcommand{\fp}{\mathfrak{p}}
\newcommand{\fq}{\mathfrak{q}}
\newcommand{\fr}{\mathfrak{r}}
\newcommand{\fA}{\mathfrak{A}}
\newcommand{\fB}{\mathfrak{B}}
\newcommand{\fC}{\mathfrak{C}}
\newcommand{\fD}{\mathfrak{D}}

%%%%%%%%%%%%%%%%%%%%%%%%%%%%%%%%%%%%%%%%%%%%%
%          Bold Letters
%%%%%%%%%%%%%%%%%%%%%%%%%%%%%%%%%%%%%%%%%%%%%
\newcommand{\ba}{\mathbf{a}}
\newcommand{\bb}{\mathbf{b}}
\newcommand{\bc}{\mathbf{c}}
\newcommand{\bd}{\mathbf{d}}
\newcommand{\be}{\mathbf{e}}
\newcommand{\bu}{\mathbf{u}}
\newcommand{\bv}{\mathbf{v}}
\newcommand{\bw}{\mathbf{w}}
\newcommand{\bx}{\mathbf{x}}
\newcommand{\by}{\mathbf{y}}
\newcommand{\bz}{\mathbf{z}}
\newcommand{\bSigma}{\boldsymbol{\Sigma}}
\newcommand{\bPi}{\boldsymbol{\Pi}}
\newcommand{\bDelta}{\boldsymbol{\Delta}}
\newcommand{\bdelta}{\boldsymbol{\delta}}
\newcommand{\bgamma}{\boldsymbol{\gamma}}
\newcommand{\bGamma}{\boldsymbol{\Gamma}}

%%%%%%%%%%%%%%%%%%%%%%%%%%%%%%%%%%%%%%%%%%%%%
%         Bold numbers
%%%%%%%%%%%%%%%%%%%%%%%%%%%%%%%%%%%%%%%%%%%%%
\newcommand{\bzero}{\mathbf{0}}

%%%%%%%%%%%%%%%%%%%%%%%%%%%%%%%%%%%%%%%%%%%%%
% Projective-Like Pointclasses
%%%%%%%%%%%%%%%%%%%%%%%%%%%%%%%%%%%%%%%%%%%%%
\newcommand{\Sa}[2][\alpha]{\Sigma_{(#1,#2)}}
\newcommand{\Pa}[2][\alpha]{\Pi_{(#1,#2)}}
\newcommand{\Da}[2][\alpha]{\Delta_{(#1,#2)}}
\newcommand{\Aa}[2][\alpha]{A_{(#1,#2)}}
\newcommand{\Ca}[2][\alpha]{C_{(#1,#2)}}
\newcommand{\Qa}[2][\alpha]{Q_{(#1,#2)}}
\newcommand{\da}[2][\alpha]{\delta_{(#1,#2)}}
\newcommand{\leqa}[2][\alpha]{\leq_{(#1,#2)}}
\newcommand{\lessa}[2][\alpha]{<_{(#1,#2)}}
\newcommand{\equiva}[2][\alpha]{\equiv_{(#1,#2)}}


\newcommand{\Sl}[1]{\Sa[\lambda]{#1}}
\newcommand{\Pl}[1]{\Pa[\lambda]{#1}}
\newcommand{\Dl}[1]{\Da[\lambda]{#1}}
\newcommand{\Al}[1]{\Aa[\lambda]{#1}}
\newcommand{\Cl}[1]{\Ca[\lambda]{#1}}
\newcommand{\Ql}[1]{\Qa[\lambda]{#1}}

\newcommand{\San}{\Sa{n}}
\newcommand{\Pan}{\Pa{n}}
\newcommand{\Dan}{\Da{n}}
\newcommand{\Can}{\Ca{n}}
\newcommand{\Qan}{\Qa{n}}
\newcommand{\Aan}{\Aa{n}}
\newcommand{\dan}{\da{n}}
\newcommand{\leqan}{\leqa{n}}
\newcommand{\lessan}{\lessa{n}}
\newcommand{\equivan}{\equiva{n}}

\newcommand{\SigmaOneOmega}{\Sigma^1_{\omega}}
\newcommand{\SigmaOneOmegaPlusOne}{\Sigma^1_{\omega+1}}
\newcommand{\PiOneOmega}{\Pi^1_{\omega}}
\newcommand{\PiOneOmegaPlusOne}{\Pi^1_{\omega+1}}
\newcommand{\DeltaOneOmegaPlusOne}{\Delta^1_{\omega+1}}

%%%%%%%%%%%%%%%%%%%%%%%%%%%%%%%%%%%%%%%%%%%%%
% Linear Algebra
%%%%%%%%%%%%%%%%%%%%%%%%%%%%%%%%%%%%%%%%%%%%%
\newcommand{\transpose}[1]{{#1}^{\text{T}}}
\newcommand{\norm}[1]{\lVert{#1}\rVert}
\newcommand\aug{\fboxsep=-\fboxrule\!\!\!\fbox{\strut}\!\!\!}

%%%%%%%%%%%%%%%%%%%%%%%%%%%%%%%%%%%%%%%%%%%%%
% Number Theory
%%%%%%%%%%%%%%%%%%%%%%%%%%%%%%%%%%%%%%%%%%%%%
\newcommand{\av}[1]{\lvert#1\rvert}
\DeclareMathOperator{\divides}{\mid}
\DeclareMathOperator{\ndivides}{\nmid}
\DeclareMathOperator{\lcm}{lcm}
\DeclareMathOperator{\sign}{sign}
\newcommand{\floor}[1]{\left\lfloor{#1}\right\rfloor}
\DeclareMathOperator{\ConCl}{CC}
\DeclareMathOperator{\ord}{ord}


%%%%%%%%%%%%%%%%%%%%%%%%%%%%%%%%%%%%%%%%%%%%%%%%%%%%%%%%%%%%%%%%%%%%%%%%%%%
%%  Theorem-Like Declarations
%%%%%%%%%%%%%%%%%%%%%%%%%%%%%%%%%%%%%%%%%%%%%%%%%%%%%%%%%%%%%%%%%%%%%%%%%%

\newtheorem{theorem}{Theorem}[section]
\newtheorem{lemma}[theorem]{Lemma}
\newtheorem{corollary}[theorem]{Corollary}
\newtheorem{proposition}[theorem]{Proposition}


\theoremstyle{definition}

\newtheorem{definition}[theorem]{Definition}
\newtheorem{conjecture}[theorem]{Conjecture}
\newtheorem{remark}[theorem]{Remark}
\newtheorem{remarks}[theorem]{Remarks}
\newtheorem{notation}[theorem]{Notation}

\theoremstyle{remark}

\newtheorem*{note}{Note}
\newtheorem*{warning}{Warning}
\newtheorem*{question}{Question}
\newtheorem*{example}{Example}
\newtheorem*{fact}{Fact}


\newenvironment*{subproof}[1][Proof]
{\begin{proof}[#1]}{\renewcommand{\qedsymbol}{$\diamondsuit$} \end{proof}}

\newenvironment*{case}[1]
{\textbf{Case #1.  }\itshape }{}

\newenvironment*{claim}[1][Claim]
{\textbf{#1.  }\itshape }{}


\pagestyle{plain}

\begin{document}

\title{Notes On Quaternions}
\author{Mitch Rudominer}

\maketitle

\tableofcontents

%%%%%%%%%%%%%%%%%%%%%%%%%%%%%%%%%%%%%%%%%%%%%%%%%%%%%%%%%%%%%%%%%%%%%%%%%%%%%%%%%%%%%%%%
In these notes we will study the quaternions and their relationship to the double covering $\pi:\SU(2)\to \SO(3)$.

\section{Quaternions}
\emph{The quaternions} is an algebra obtained from $\R^4$ by adding an associative multiplication operation. There are several natural ways to define the multiplcation. We give three and show they are equivalent.

\subsection{The Quaternions as a Matrix Algebra}

\begin{definition}
Let $\H_M \subset M(2; \C)$ be defined by 
$$
\H_M = \Bigsetof{ \begin{pmatrix}
\alpha & \beta \\
-\betabar & \alphabar
\end{pmatrix} 
}{\alpha,\beta \in \C}.
$$
\end{definition}

\begin{lemma}
\label{lemma:det_of_A}
Let
$$A= \begin{pmatrix}
\alpha & \beta \\
-\betabar & \alphabar
\end{pmatrix}
\in\H_M.$$
Then $\det(A) = |\alpha|^2 + |\beta|^2$ and $\det(A)>=0$ and $\det(A)=0$ iff $A=0$.
\end{lemma} 
\begin{proof}
Immediate.
\end{proof}


\begin{note}
$\setof{A\in \H_M}{\det(A) = 1} = \SU(2).$
\end{note}


\begin{lemma} $\H_M$  is a real sub-algebra of $M(2; \C)$.
\end{lemma}
\begin{proof}
Because complex conjugation is linear over $\R$, $\H_M$ is closed under addition and real scalar multiplication.
To see that $\H_M$ is closed under multiplication we compute 

\begin{equation}
\label{eq:matrixmult}
\begin{pmatrix}
\alpha & \beta \\
-\betabar & \alphabar
\end{pmatrix}
\times
\begin{pmatrix}
\gamma & \delta \\
-\deltabar & \gammabar
\end{pmatrix} 
= 
\begin{pmatrix}
\alpha\gamma - \beta\deltabar & \alpha\delta + \beta\gammabar \\
-\betabar\gamma -\alphabar\deltabar & -\betabar\delta + \alphabar\gammabar
\end{pmatrix} .
\end{equation}
\end{proof}

\begin{lemma}  $\H_M$ is a real division algebra.
\end{lemma}
\begin{proof}
Let $A\in \H_M$, $A\not=0$. We need to show that $A$ is invertible and $A^{-1}\in \H_M$.
Write $A = \begin{pmatrix}
\alpha & \beta \\
-\betabar & \alphabar
\end{pmatrix}$. Then

$\det(A) = |\alpha|^2 + |\beta|^2 \not = 0$ so $A$ is invertible.
Also $\det(A)\in\R$ so

$$A^{-1} = \det(A)^{-1} \begin{pmatrix}
\alphabar & -\beta \\
\betabar & \alpha
\end{pmatrix} \in \H_M.$$

\end{proof}

\begin{lemma}
$H_M$ is closed under Hermitian adjoint.
\end{lemma}
\begin{proof}
\begin{equation}
\label{eq:quaternion_matrix_adjoint}
\begin{pmatrix}
\alpha & \beta \\
-\betabar & \alphabar
\end{pmatrix}^{*}=
\begin{pmatrix}
\alphabar & -\beta \\
\betabar & \alpha
\end{pmatrix}\in \H_M.
\end{equation}
\end{proof}

\begin{lemma}
\label{lemma:a_times_a_star}
For $A\in \H_M$, $AA^* = \det(A) I$.
\end{lemma}
\begin{proof}
Let $A= \begin{pmatrix}\alpha & \beta \\-\betabar & \alphabar\end{pmatrix}$.
Let $d=\det(A) = |\alpha|^2+|\beta|^2$. then
$$
AA^* = \begin{pmatrix}
\alpha & \beta \\
-\betabar & \alphabar
\end{pmatrix}
\begin{pmatrix}
\alphabar & -\beta \\
\betabar & \alpha
\end{pmatrix}
=\begin{pmatrix}d & 0\\ 0 & d\end{pmatrix}.
$$
\end{proof}


For $\alpha,\beta \in \C$ let 
$$Q(\alpha,\beta) = \begin{pmatrix}
\alpha & \beta \\
-\bar{\beta} & \bar{\alpha}
\end{pmatrix}.$$

For $x,y\in\R$ let $\alpha(x,y) = x + iy\in\C$.

For $x,y,z,w\in\R$ let $Q(x,y,z,w) = Q(\alpha(x,y), \alpha(z,w))$.

For $v\in\R^4$ let $Q(v) = Q(v^0, v^1, v^2, v^3)$.

This last $Q$ is clearly a bijection from $\R^4$ to the matrix algebra $\H_M$ defined above. 

\begin{definition}
We
define quaternionic multiplication on $\R^4$ using this mapping.
For $v,w\in\R^4$, define
$$vw = Q^{-1}(Q(v)Q(w)).$$

With this multiplication, in addition to the
standard vector space structure,
$\R^4$ becomes a real associative algebra we call the quaternions.
We use the symbol $\H$ to refer to the quaternions.
\end{definition}

\begin{remark}
$Q:\H\to \H_M$ is an algebra isomorphism. This is trivial
as we defined multiplication in $\H$ so as to make this so.
\end{remark}

\begin{definition}
$0,1\in\H$ are defined by
$$0=\begin{pmatrix}0\\0\\0\\0\end{pmatrix}\text{   }
1=\begin{pmatrix}1\\0\\0\\0\end{pmatrix}.$$

Since $Q(0)$ is the zero matrix and $Q(1)$ is the identity matrix,
$0$ is the zero element of $\H$ and $1$ is the identity element
of $\H$.

We can tell from context when the symbols 0 and 1 refer to
quaternions instead of numbers.
\end{definition}

\begin{definition}
We define quaternionic inverses using the mapping $Q$.
For $v\in\H$, $v\not=0$,
$$v^{-1} = Q^{-1}(Q(v)^{-1}).$$

With this inverse $\H$ is a division algebra.
\end{definition}

\begin{definition}
A \emph{scalar} quaternion is a $q\in\H$ of the form
$q=\begin{pmatrix}c\\0\\0\\0\end{pmatrix}$ for $c\in\R$,
and in this case we say that $c$ is the \emph{scalar part}
of $q$.
Equivalently $q$ is a scalar quaternion iff $Q(q)$ is
a scalar matrix. $0,1\in\H$ are examples of scalar quaternions.
\end{definition}

\begin{lemma}
Let $c\in\R$ and $q\in\H$.
Then $cq = \begin{pmatrix}c\\0\\0\\0\end{pmatrix} q$. That is,
scalar multiplication by $c$ is the same as quaternionic
multiplication by the scalar quaternion with scalar part $c$.
\end{lemma}
\begin{proof}
This is true becuase the analogous fact is true about scalar matrices.
\end{proof}

\begin{lemma}
\label{corollary:scalars_commute}
A scalar quaternion commutes with all quaternions.
Also real scalar multiplication commutes with quaternionic multiplication.
\end{lemma}
\begin{proof}
This is true because the analogous fact is true about
scalar matrices and matrix multiplication.
\end{proof}

\begin{remark}
If $c\in \R$ and $1$ represents the unit quaternion, then $c 1$ is the scalar quaternion
with scalar part $c$. We will henceforth simplify this and write $c 1$ as $c$.
Thus we are identifying the real numbers and the scalar quaternions. One can tell from
context which kind of object a symbol refers to.
\end{remark}


\begin{definition}
For $q\in\H$ let $|q|$ denote the Euclidean norm of $q$ in $\R^4$.
\end{definition}

\begin{lemma}
\label{lemma:norm_is_determinant}
For $q\in\H$, $|q|^2 = \det(Q(q))$.
\end{lemma}
\begin{proof}
By Lemma \ref{lemma:det_of_A},
$\det(Q(q)) = |q^0 + iq^1|^2 + |q^2 + iq^3|^2 = |q|^2$.
\end{proof}

\begin{lemma}
For all $q_1,q_2 \in \H$, $|q_1 q_2| = |q_1| |q_2|$.
\end{lemma}
\begin{proof}
$|q_1 q_2|^2 = \det(Q(q_1 q_2)) = \det(Q(q_1) Q(q_2)) = \det(Q(q_1))\det(Q(q_2)) = |q_1|^2 |q_2|^2.$
The result follows since norms are non-negative.
\end{proof}

\begin{definition}
For $q\in\H$ we define its \emph{conjugate} to be $\bar{q} = Q^{-1}(Q(q)^{*})$.
That is, the conjugate of $q$ corresponds to the Hermitian adjoint of $Q(q)$.
\end{definition}

\begin{corollary}
\label{corollary:q_times_qbar}
 For $q\in\H$,  $q \qbar = |q|^2 $. (We are using our convention of conflating a real number with its scalar quaternion.)
 \end{corollary}
 \begin{proof}  $Q(q \qbar) = Q(q) Q(q)^* = \det(Q(q))^2 I$ (by Lemma \ref{lemma:a_times_a_star})
 $= |q|^2 I$ (by Lemma \ref{lemma:norm_is_determinant}) $ = Q(|q|^2 1)$.
 \end{proof}

 \begin{corollary}
 \label{corollary:q_inverse_by_qbar}
 For $q\in\H$, 
 $$q^{-1} = \frac{1}{|q|^2} \qbar.$$
 \end{corollary}
 \begin{proof}
 By Corollary \ref{corollary:q_times_qbar}
 $$\frac{1}{|q|^2} q \qbar = 1.$$ 
 \end{proof}

 \begin{lemma}
 For $q\in\H$, $\bar{\qbar} = q$.
 \end{lemma}
 \begin{proof}
 Immediate because for $A$ a matrix $(A^{*})^{*} = A$.
 \end{proof}

\subsection{The Quaternions in terms of vector operations}

We can write an element of $\R^4$ as
$\begin{pmatrix}
c\\
\vec{v}
\end{pmatrix}$ with $c\in \R$ and $\vec{v}\in\R^3$. When
we do this we call $c$ the \emph{real} or \emph{scalar}
part of the quaternion and $\vec{v}$ the \emph{pure} part. 

\begin{definition}
A \emph{pure} quaternion is one in which the real part is zero.
A \emph{scalar} quaternion is one in which the pure part is zero.
(This is a restatement of the definition from the previous subsection.)
\end{definition}

\begin{lemma}
\label{lemma:in_terms_of_vector_products}
With quaternionic multiplication as defined in the previous subsection and the vector notation of the current subsection we have
$$
\begin{pmatrix}
c\\
\vec{v}
\end{pmatrix}
\begin{pmatrix}
d\\
\vec{w}
\end{pmatrix}
=
\begin{pmatrix}
cd - \vec{v} \cdot \vec{w} \\
c\vec{w} + d\vec{v} + \vec{v}\times \vec{w}
\end{pmatrix}.
$$
\end{lemma}
\begin{proof}
Let $\alpha = c + iv_0$, $\beta = v_1 + i v_2$,
$\gamma = d + iw_0$, $\delta = w_1 + iw_2$.

Let 

$$
\begin{pmatrix}
e\\
\vec{u}
\end{pmatrix}
=
\begin{pmatrix}
c\\
\vec{v}
\end{pmatrix}
\begin{pmatrix}
d\\
\vec{w}
\end{pmatrix}.
$$

By equation \ref{eq:matrixmult} we have that
\begin{itemize}
\item $e=\re(\alpha\gamma - \beta\deltabar) = (cd -v_0 w_0) - (v_1 w_1 + v_2 w_2)$
\item $u_0 = \im(\alpha\gamma - \beta\deltabar) = (c w_0 + d v_0) - (- v_1 w2 + w_1 v_2)$
\item $u_1 = \re(\alpha\delta + \beta\gammabar) = (c w_1 - v_0 w_2) + (d v_1 + v_2 w_0)$
\item $u_2 = \im(\alpha\delta + \beta\gammabar) = (c w_2 + v_0 w_1) + (-v_1 w_0 + d v_2)$
\end{itemize}

So

$$
\begin{pmatrix}
e\\
\vec{u}
\end{pmatrix}
=
\begin{pmatrix}
cd - \vec{v} \cdot \vec{w} \\
c w_0 + d v_0 + ( v_1 w2 -  v_2 w_1) \\
c w_1 + d v_1 + (v_2 w_0 - v_0 w_2))\\
c w_2 + d v_2 + (v_0 w_1 - v_1 w_0)
\end{pmatrix}
=
\begin{pmatrix}
cd - \vec{v} \cdot \vec{w} \\
c\vec{w} + d\vec{v} + \vec{v}\times \vec{w}
\end{pmatrix}.
$$
\end{proof}

\begin{note}
We could have made Lemma \ref{lemma:in_terms_of_vector_products} the definition of quaternion
multiplication. But then we would have needed to prove that it is associative and
a division algebra.
\end{note}

\begin{corollary}
\label{corollary:pure_mult}
Multiplication of pure quaternions is given by the vector dot and cross products:
$$
\begin{pmatrix}
0\\
\vec{v}
\end{pmatrix}
\begin{pmatrix}
0\\
\vec{w}
\end{pmatrix}
=
\begin{pmatrix}
\vec{v} \cdot \vec{w} \\
\vec{v}\times \vec{w}
\end{pmatrix}.
$$
\end{corollary}

\begin{lemma}
\label{lemma:commute_criterion}
Let $q_1, q_2 \in \H$. Then $q_1 q_2 = q_2 q_1$ iff 
the pure parts of $q_1$ and $q_2$ are colinear.
\end{lemma}
\begin{proof}
Let $\vec{v}$ and $\vec{w}$ be the pure parts of $q_1$ and $q_2$ respectively.
By Lemma \ref{lemma:in_terms_of_vector_products}, $q_1$ and $q_2$ commute
iff $v_1 \times v_2$ = $v_2 \times v_1$ and this is so iff
$v_1 \times v_2 = 0$ iff $v_1$ and $v_2$ are colinear.
\end{proof}

\begin{lemma}
\label{lemma:qbar_for_vectors}
 For $q\in\H$, with $q=\begin{pmatrix}c \\ \vec{v}\end{pmatrix}$,
 $$\qbar = \begin{pmatrix}c \\ -\vec{v}\end{pmatrix}.$$
 \end{lemma}
 \begin{proof}
 Write $q=\begin{pmatrix}x \\ y \\ z \\ w\end{pmatrix}$.
 Let $\alpha = x + iy$, $\beta=z+iw$.
 Then $\alphabar = x - iy$ and $-\beta = -z - iw$.
 By equation \ref{eq:quaternion_matrix_adjoint},
 $$
 \qbar=Q^{-1}(Q(\alpha,\beta)^*)=Q^{-1}(Q(\alphabar,-\beta))=
 \begin{pmatrix}
 x \\ -y \\ -z \\ -w
 \end{pmatrix}.
 $$
 \end{proof}

 \begin{lemma}
Let $q\in\H$, $q\not=0$.
Write $q=\begin{pmatrix}c\\ \vec{v}\end{pmatrix}$. Then
$$q^{-1} = \frac{1}{|q|^2}\begin{pmatrix}c\\ -\vec{v}\end{pmatrix}.$$
\end{lemma}
\begin{proof}
By Corollary \ref{corollary:q_inverse_by_qbar}, 
$$q^{-1} = \frac{1}{|q|^2}\qbar$$ so this lemma
follws from Lemma \ref{lemma:qbar_for_vectors}.
\end{proof}


\subsection{The Quaternions in terms of canonical basis elements}

We define four standard basis quaternions:
$$
1 = \begin{pmatrix}1\\0\\0\\0\end{pmatrix},\text{  }
i =  \begin{pmatrix}0\\1\\0\\0\end{pmatrix},\text{  }
j =  \begin{pmatrix}0\\0\\1\\0\end{pmatrix},\text{  }
k =  \begin{pmatrix}0\\0\\0\\1\end{pmatrix},\text{.}
$$

\begin{note}
We have already defined the quaternion 1 in an earlier subsection.
As with 1, we are overloading the symbol $i$ to refer to a quaternion in addition
to a complex number. One can tell from context which object $i$ refers to.
\end{note}

\begin{remark}
Let $\vec{i}$, $\vec{j}$, $\vec{k}$ be the standard basis vectors of three-dimensional vector calculus.
Then
$i = \begin{pmatrix}0\\\vec{i}\end{pmatrix}$, 
$j = \begin{pmatrix}0\\\vec{j}\end{pmatrix}$, 
$k = \begin{pmatrix}0\\\vec{k}\end{pmatrix}$.
\end{remark}

\begin{note}
So in addition to $i$ as a complex number and $i$ as a quaternion we also have $\vec{i}$
as a vector in $\R^3$. We remark below that all three objects are connected.
\end{note}

\begin{lemma} We record the following basic equations about the basis elements.
\label{lemma:basic_equations}
\begin{enumerate}
\item $1$ is the multiplicative identity in the quaternions.
\item $i^2=j^2=k^2 = -1$.
\item $ij = -ji = k$.
\item $jk = -kj = i$.
\item $ki = -ik = j$.
\end{enumerate}
Here $-1$ is a scalar quaternion.
\end{lemma}
\begin{proof}
(1) $Q(1)$ is the identity matrix. We have already seen this
earlier in these notes.

(2) Note that in the equation below we use the letter $i$ to mean two different things
on the left and right side of the equation.

$$Q(i^2) = \begin{pmatrix}i & 0 \\ 0 & -i\end{pmatrix}^2 = \begin{pmatrix}-1 & 0 \\ 0 & -1\end{pmatrix} = Q(-1).$$

$$Q(j^2) = \begin{pmatrix}0 & 1 \\ -1 & 0\end{pmatrix}^2 = \begin{pmatrix}-1 & 0 \\ 0 & -1\end{pmatrix} = Q(-1).$$

$$Q(k^2) = \begin{pmatrix}0 & i \\ i & 0\end{pmatrix}^2 = \begin{pmatrix}-1 & 0 \\ 0 & -1\end{pmatrix} = Q(-1).$$

(3),(4),(5) follow from Lemma \ref{lemma:in_terms_of_vector_products} because the identities are true when we
 replace $i,j,k$ with $\vec{i},\vec{j},\vec{k}$, and replace quaternionic multiplication with the vector cross product.
 \end{proof}

 \begin{lemma}
 Quaternionic multiplication as we have defined it earlier is the unique multiplication on $\R^4$ 
that distributes over addition, commutes with scalar multiplication,
 and satisfies the equations in Lemma \ref{lemma:basic_equations}. Thus we could have defined the quaternions
 using the equations in Lemma \ref{lemma:basic_equations}.
 \end{lemma}
 \begin{proof}
 A binary operation on $\R^4$ that distributes over addition and commutes with scalar multiplication
 is the same thing as a bilinear transformation from $\R^4 \times \R^4$ to $\R^4$ and such a transformation
 is completely determined by its values on all ordered pairs from a fixed basis.
 \end{proof}

 \begin{remarks} $\R$, $\R^3$ and $\C$ all naturally embed into $\H$ and going forward we will often
 tacitly conflate all of these with their images in $\H$.
 \begin{enumerate}
 \item As previously mentioned, we will identify $\R$ with the scalar quaternions via the identification $x\mapsto x 1$ where $x$ is a
 real number and $1$ is the unit quaternion.
 \item Define the \emph{complex} quaternions to be the two-dimensional subspace of $\H$ spanned by 1 and $i$.
 It is easy to see that the complex quaternions are a sub-algebra of $\H$.
 We will identify $\C$ with this sub-algebra by identifying the number 1 with the quaternion 1 and the
 number $i$ with the quaternion $i$.
 \item We will identify $\R^3$ with the pure quaternions by identifying $\vec{i}$ with the quaternion $i$,
 $\vec{j}$ with the quaternion $j$, and $\vec{k}$ with the quaternion $k$.
 \item It is pleasing that these identifications unify all three uses of the letter $i$. In particular it is
 not obvious to the student who first encounters vector calculus that the vector $\vec{i}$ has anything
 to do with the imaginary unit $i$.
 \end{enumerate}
 \end{remarks}

 \section{Lie Group and Lie Algebra Structures for the Quaternions}

 Because $\H$ and $\H_M$ are associative algebras, they naturally have a commutator bracket
 making them into Lie algebras. For $\H_M$ this is the usual matrix commutator and so
 $\H_M$ is a Lie subalgebra of $M(2; \C)$. We have that $Q:\H\map\H_M$ is a Lie algebra isomorphism.

\begin{lemma}
\label{lemma:bracket_in_terms_of_vector_products}
The Lie bracket on $\H$ may be computed as
$$
\left[
\begin{pmatrix}
c\\
\vec{v}
\end{pmatrix},
\begin{pmatrix}
d\\
\vec{w}
\end{pmatrix}
\right]
=
\begin{pmatrix}
0 \\
2\vec{v}\times \vec{w}
\end{pmatrix}.
$$
\end{lemma}
\begin{proof}
Using Lemma \ref{lemma:in_terms_of_vector_products} we compute
$$
\begin{pmatrix}
cd - \vec{v} \cdot \vec{w} \\
c\vec{w} + d\vec{v} + \vec{v}\times \vec{w}
\end{pmatrix}
-
\begin{pmatrix}
dc - \vec{w} \cdot \vec{v} \\
d\vec{v} + c\vec{w} + \vec{w}\times \vec{v}
\end{pmatrix}
=
\begin{pmatrix}
0 \\
 \vec{v}\times \vec{w} - (-\vec{v}\times \vec{w})
\end{pmatrix}
$$
\end{proof}

\begin{definition}
Let $\H_0$ be the space of pure quaternions. For $\vec{v}\in\R^3$, let $q(\vec{v})\in\H_0$
be the pure quaternion whose pure part is $\vec{v}$.
\end{definition}

\begin{corollary}
$\H_0$ with the commutator bracket is a three-dimensional Lie algebra, isomorphic to $\R^3$ with
the cross product:
$$\left(\H_0,[\cdot]\right) \isomorphic \left(\R^3, \times \right).$$
\end{corollary}
\begin{proof}
$\H_0$ is a three dimensional subspace closed under the the bracket, so it is a Lie
subalgebra of $\H$.
Let $\pi:\H_0\map\R^3$ be defined by $\pi(q(\vec{v}))=2\vec{v}$. This is a linear isomorphism
and we claim that for $v_1,v_2\in\H_0$, $\pi([q_1,q_2]) = \pi(q_1) \times \pi(q_2)$. To see
this fix $q_1=q(\vec{v}_1),q_2=q(\vec{v}_2)$. Then
$\pi([q_1,q_2]) = \pi(q(2 \vec{v}_1\times\vec{v}_2)) = 4 \vec{v}_1\times\vec{v}_2
= 2\vec{v}_1\times 2\vec{v}_2 = \pi(q_1) \times \pi(q_2)$.
\end{proof}

\begin{remark}
We need to be careful because we now have two different natural isomorphisms between $\H_0$ and $\R^{3}$.
We have the isomorphism that maps $i$ to $\vec{i}$ which is useful when we want to consider $\R^3$ as a subset of
$\H_0$ and which identifies the multiple
uses of the letter $i$, and we have the isomorphism that maps $i$ to $2\vec{i}$ which is a Lie algebra isomrphism.
\end{remark}

\begin{lemma}
\label{lemma:quaternion_lie_group}
$\H -\singleton{0}$ is a four-dimensional Lie group.
\end{lemma}
\begin{proof}
$M(2; \C)$ is a real vector space and so an 8-dimensional real manifold. 

$\H_M$ is a 4-dimensional vector subspace and so a 4-dimensional embedded submanifold of $M(2; \C)$.

$\H_M- \singleton{0}$ is an open submanifold of $\H_M$ and so a a 4-dimensional embedded submanifold of $M(2; \C)$.


$\H_M- \singleton{0}\subset \GL(2,\C)$ so $\H_M- \singleton{0}$ is a 4-dimensional embedded submanifold of $GL(2,\C)$.


$\H_M- \singleton{0}$ is also a subgroup of $\GL(2,\C)$ and so it is a Lie subgroup.

$Q:\H\map\H_M$ is a vector space isomorphism and so a diffeomorphism.

So $Q:\H - \singleton{0}\map\H_M- \singleton{0}$ a diffeomorphism.

$Q:\H- \singleton{0}\map\H_M- \singleton{0}$ is also a group isomorphism.

So $\H$ is also a four-dimensional Lie group and $Q$ is a Lie group isomorphism.
\end{proof}

\begin{corollary}
The set of unit quaternions is a three-dimensional Lie group, isomorphic to $\SU(2)$.
\end{corollary}
\begin{proof}
The proof of Lemma \ref{lemma:quaternion_lie_group} shows that $Q:\H\map\H_M$
is a Lie group isomorphism and
$Q$ maps the set of unit quaternions one-to-one and onto $\SU(2)$.
\end{proof}


\bibliographystyle{amsalpha}

\bibliography{math}

\end{document}