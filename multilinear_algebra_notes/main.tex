\documentclass[oneside,12pt]{amsart}

\usepackage{amsmath,amssymb,latexsym,eucal,amsthm}

%%%%%%%%%%%%%%%%%%%%%%%%%%%%%%%%%%%%%%%%%%%%%
% Common Set Theory Constructs
%%%%%%%%%%%%%%%%%%%%%%%%%%%%%%%%%%%%%%%%%%%%%

\newcommand{\setof}[2]{\left\{ \, #1 \, \left| \, #2 \, \right.\right\}}
\newcommand{\lsetof}[2]{\left\{\left. \, #1 \, \right| \, #2 \,  \right\}}
\newcommand{\bigsetof}[2]{\bigl\{ \, #1 \, \bigm | \, #2 \,\bigr\}}
\newcommand{\Bigsetof}[2]{\Bigl\{ \, #1 \, \Bigm | \, #2 \,\Bigr\}}
\newcommand{\biggsetof}[2]{\biggl\{ \, #1 \, \biggm | \, #2 \,\biggr\}}
\newcommand{\Biggsetof}[2]{\Biggl\{ \, #1 \, \Biggm | \, #2 \,\Biggr\}}
\newcommand{\dotsetof}[2]{\left\{ \, #1 \, : \, #2 \, \right\}}
\newcommand{\bigdotsetof}[2]{\bigl\{ \, #1 \, : \, #2 \,\bigr\}}
\newcommand{\Bigdotsetof}[2]{\Bigl\{ \, #1 \, \Bigm : \, #2 \,\Bigr\}}
\newcommand{\biggdotsetof}[2]{\biggl\{ \, #1 \, \biggm : \, #2 \,\biggr\}}
\newcommand{\Biggdotsetof}[2]{\Biggl\{ \, #1 \, \Biggm : \, #2 \,\Biggr\}}
\newcommand{\sequence}[2]{\left\langle \, #1 \,\left| \, #2 \, \right. \right\rangle}
\newcommand{\lsequence}[2]{\left\langle\left. \, #1 \, \right| \,#2 \,  \right\rangle}
\newcommand{\bigsequence}[2]{\bigl\langle \,#1 \, \bigm | \, #2 \, \bigr\rangle}
\newcommand{\Bigsequence}[2]{\Bigl\langle \,#1 \, \Bigm | \, #2 \, \Bigr\rangle}
\newcommand{\biggsequence}[2]{\biggl\langle \,#1 \, \biggm | \, #2 \, \biggr\rangle}
\newcommand{\Biggsequence}[2]{\Biggl\langle \,#1 \, \Biggm | \, #2 \, \Biggr\rangle}
\newcommand{\singleton}[1]{\left\{#1\right\}}
\newcommand{\angles}[1]{\left\langle #1 \right\rangle}
\newcommand{\bigangles}[1]{\bigl\langle #1 \bigr\rangle}
\newcommand{\Bigangles}[1]{\Bigl\langle #1 \Bigr\rangle}
\newcommand{\biggangles}[1]{\biggl\langle #1 \biggr\rangle}
\newcommand{\Biggangles}[1]{\Biggl\langle #1 \Biggr\rangle}


\newcommand{\force}[1]{\Vert\!\underset{\!\!\!\!\!#1}{\!\!\!\relbar\!\!\!%
\relbar\!\!\relbar\!\!\relbar\!\!\!\relbar\!\!\relbar\!\!\relbar\!\!\!%
\relbar\!\!\relbar\!\!\relbar}}
\newcommand{\nforce}[1]{\Vert\!\underset{\!\!\!\!\!#1}{\!\!\!\relbar\!\!\!%
\relbar\!\!\relbar\!\!\relbar\!\!\!\relbar\!\!\relbar\!\!\relbar\!\!\!%
\relbar\!\!\not\relbar\!\!\relbar}}
\newcommand{\forcein}[2]{\overset{#2}{\Vert\underset{\!\!\!\!\!#1}%
{\!\!\!\relbar\!\!\!\relbar\!\!\relbar\!\!\relbar\!\!\!\relbar\!\!\relbar\!%
\!\relbar\!\!\!\relbar\!\!\relbar\!\!\relbar\!\!\relbar\!\!\!\relbar\!\!%
\relbar\!\!\relbar}}}

\newcommand{\pre}[2]{{}^{#2}\!{#1}}

\newcommand{\restr}{\!\!\upharpoonright\!}

%%%%%%%%%%%%%%%%%%%%%%%%%%%%%%%%%%%%%%%%%%%%%
% Set-Theoretic Connectives
%%%%%%%%%%%%%%%%%%%%%%%%%%%%%%%%%%%%%%%%%%%%%

\newcommand{\intersect}{\cap}
\newcommand{\union}{\cup}
\newcommand{\Intersection}[1]{\bigcap\limits_{#1}}
\newcommand{\Union}[1]{\bigcup\limits_{#1}}
\newcommand{\adjoin}{{}^\frown}
\newcommand{\forces}{\Vdash}

%%%%%%%%%%%%%%%%%%%%%%%%%%%%%%%%%%%%%%%%%%%%%
% Miscellaneous
%%%%%%%%%%%%%%%%%%%%%%%%%%%%%%%%%%%%%%%%%%%%%
\newcommand{\defeq}{=_{\text{def}}}
\newcommand{\Turingleq}{\leq_{\text{T}}}
\newcommand{\Turingless}{<_{\text{T}}}
\newcommand{\lexleq}{\leq_{\text{lex}}}
\newcommand{\lexless}{<_{\text{lex}}}
\newcommand{\Turingequiv}{\equiv_{\text{T}}}

%%%%%%%%%%%%%%%%%%%%%%%%%%%%%%%%%%%%%%%%%%%%%
% Constants
%%%%%%%%%%%%%%%%%%%%%%%%%%%%%%%%%%%%%%%%%%%%%
\newcommand{\R}{\mathbb{R}}
\renewcommand{\P}{\mathbb{P}}
\newcommand{\Q}{\mathbb{Q}}
\newcommand{\Z}{\mathbb{Z}}
\newcommand{\C}{\mathbb{C}}
\newcommand{\N}{\mathbb{N}}
\newcommand{\B}{\mathbb{B}}
\newcommand{\LofR}{L(\R)}
\newcommand{\JofR}[1]{J_{#1}(\R)}
\newcommand{\SofR}[1]{S_{#1}(\R)}
\newcommand{\JalphaR}{\JofR{\alpha}}
\newcommand{\JbetaR}{\JofR{\beta}}
\newcommand{\JlambdaR}{\JofR{\lambda}}
\newcommand{\SalphaR}{\SofR{\alpha}}
\newcommand{\SbetaR}{\SofR{\beta}}
\newcommand{\Pkl}{\mathcal{P}_{\kappa}(\lambda)}
\DeclareMathOperator{\con}{con}
\DeclareMathOperator{\ORD}{OR}
\DeclareMathOperator{\Ord}{OR}
\DeclareMathOperator{\WO}{WO}
\DeclareMathOperator{\OD}{OD}
\DeclareMathOperator{\HOD}{HOD}
\DeclareMathOperator{\HC}{HC}
\DeclareMathOperator{\WF}{WF}
\DeclareMathOperator{\HF}{HF}
\newcommand{\One}{I}
\newcommand{\ONE}{I}
\newcommand{\Two}{II}
\newcommand{\TWO}{II}

%%%%%%%%%%%%%%%%%%%%%%%%%%%%%%%%%%%%%%%%%%%%%
% Commutative Algebra Constants
%%%%%%%%%%%%%%%%%%%%%%%%%%%%%%%%%%%%%%%%%%%%%
\DeclareMathOperator{\dottimes}{\dot{\times}}

%%%%%%%%%%%%%%%%%%%%%%%%%%%%%%%%%%%%%%%%%%%%%
% Theories
%%%%%%%%%%%%%%%%%%%%%%%%%%%%%%%%%%%%%%%%%%%%%
\DeclareMathOperator{\ZFC}{ZFC}
\DeclareMathOperator{\ZF}{ZF}
\DeclareMathOperator{\AD}{AD}
\DeclareMathOperator{\ADR}{AD_{\R}}
\DeclareMathOperator{\KP}{KP}
\DeclareMathOperator{\PD}{PD}
\DeclareMathOperator{\CH}{CH}
\DeclareMathOperator{\HPC}{HPC} % HOD pair capturing
%%%%%%%%%%%%%%%%%%%%%%%%%%%%%%%%%%%%%%%%%%%%%
% Iteration Trees
%%%%%%%%%%%%%%%%%%%%%%%%%%%%%%%%%%%%%%%%%%%%%

\newcommand{\pred}{\text{-pred}}

%%%%%%%%%%%%%%%%%%%%%%%%%%%%%%%%%%%%%%%%%%%%%%%%
% Operator Names
%%%%%%%%%%%%%%%%%%%%%%%%%%%%%%%%%%%%%%%%%%%%%%%%
\DeclareMathOperator{\Det}{Det}
\DeclareMathOperator{\dom}{dom}
\DeclareMathOperator{\ran}{ran}
\DeclareMathOperator{\range}{ran}
\DeclareMathOperator{\image}{image}
\DeclareMathOperator{\crit}{crit}
\DeclareMathOperator{\card}{card}
\DeclareMathOperator{\cf}{cf}
\DeclareMathOperator{\cof}{cof}
\DeclareMathOperator{\rank}{rank}
\DeclareMathOperator{\ot}{o.t.}
\DeclareMathOperator{\ords}{o}
\DeclareMathOperator{\ro}{r.o.}
\DeclareMathOperator{\rud}{rud}
\DeclareMathOperator{\Powerset}{\mathcal{P}}
\DeclareMathOperator{\length}{lh}
\DeclareMathOperator{\lh}{lh}
\DeclareMathOperator{\limit}{lim}
\DeclareMathOperator{\fld}{fld}
\DeclareMathOperator{\projection}{p}
\DeclareMathOperator{\Ult}{Ult}
\DeclareMathOperator{\ULT}{Ult}
\DeclareMathOperator{\Coll}{Coll}
\DeclareMathOperator{\Col}{Col}
\DeclareMathOperator{\Hull}{Hull}
\DeclareMathOperator*{\dirlim}{dir lim}
\DeclareMathOperator{\Scale}{Scale}
\DeclareMathOperator{\supp}{supp}
\DeclareMathOperator{\trancl}{tran.cl.}
\DeclareMathOperator{\trace}{Tr}
\DeclareMathOperator{\diag}{diag}
\DeclareMathOperator{\spn}{span}
\DeclareMathOperator{\sgn}{sgn}
%%%%%%%%%%%%%%%%%%%%%%%%%%%%%%%%%%%%%%%%%%%%%
% Logical Connectives
%%%%%%%%%%%%%%%%%%%%%%%%%%%%%%%%%%%%%%%%%%%%%
\newcommand{\IImplies}{\Longrightarrow}
\newcommand{\SkipImplies}{\quad\Longrightarrow\quad}
\newcommand{\Ifff}{\Longleftrightarrow}
\newcommand{\iimplies}{\longrightarrow}
\newcommand{\ifff}{\longleftrightarrow}
\newcommand{\Implies}{\Rightarrow}
\newcommand{\Iff}{\Leftrightarrow}
\renewcommand{\implies}{\rightarrow}
\renewcommand{\iff}{\leftrightarrow}
\newcommand{\AND}{\wedge}
\newcommand{\OR}{\vee}
\newcommand{\st}{\text{ s.t. }}
\newcommand{\Or}{\text{ or }}

%%%%%%%%%%%%%%%%%%%%%%%%%%%%%%%%%%%%%%%%%%%%%
% Function Arrows
%%%%%%%%%%%%%%%%%%%%%%%%%%%%%%%%%%%%%%%%%%%%%

\newcommand{\injection}{\xrightarrow{\text{1-1}}}
\newcommand{\surjection}{\xrightarrow{\text{onto}}}
\newcommand{\bijection}{\xrightarrow[\text{onto}]{\text{1-1}}}
\newcommand{\cofmap}{\xrightarrow{\text{cof}}}
\newcommand{\map}{\rightarrow}

%%%%%%%%%%%%%%%%%%%%%%%%%%%%%%%%%%%%%%%%%%%%%
% Mouse Comparison Operators
%%%%%%%%%%%%%%%%%%%%%%%%%%%%%%%%%%%%%%%%%%%%%
\newcommand{\initseg}{\trianglelefteq}
\newcommand{\properseg}{\lhd}
\newcommand{\notinitseg}{\ntrianglelefteq}
\newcommand{\notproperseg}{\ntriangleleft}

%%%%%%%%%%%%%%%%%%%%%%%%%%%%%%%%%%%%%%%%%%%%%
%           calligraphic letters
%%%%%%%%%%%%%%%%%%%%%%%%%%%%%%%%%%%%%%%%%%%%%
\newcommand{\cA}{\mathcal{A}}
\newcommand{\cB}{\mathcal{B}}
\newcommand{\cC}{\mathcal{C}}
\newcommand{\cD}{\mathcal{D}}
\newcommand{\cE}{\mathcal{E}}
\newcommand{\cF}{\mathcal{F}}
\newcommand{\cG}{\mathcal{G}}
\newcommand{\cH}{\mathcal{H}}
\newcommand{\cI}{\mathcal{I}}
\newcommand{\cJ}{\mathcal{J}}
\newcommand{\cK}{\mathcal{K}}
\newcommand{\cL}{\mathcal{L}}
\newcommand{\cM}{\mathcal{M}}
\newcommand{\cN}{\mathcal{N}}
\newcommand{\cO}{\mathcal{O}}
\newcommand{\cP}{\mathcal{P}}
\newcommand{\cQ}{\mathcal{Q}}
\newcommand{\cR}{\mathcal{R}}
\newcommand{\cS}{\mathcal{S}}
\newcommand{\cT}{\mathcal{T}}
\newcommand{\cU}{\mathcal{U}}
\newcommand{\cV}{\mathcal{V}}
\newcommand{\cW}{\mathcal{W}}
\newcommand{\cX}{\mathcal{X}}
\newcommand{\cY}{\mathcal{Y}}
\newcommand{\cZ}{\mathcal{Z}}


%%%%%%%%%%%%%%%%%%%%%%%%%%%%%%%%%%%%%%%%%%%%%
%          Primed Letters
%%%%%%%%%%%%%%%%%%%%%%%%%%%%%%%%%%%%%%%%%%%%%
\newcommand{\aprime}{a^{\prime}}
\newcommand{\bprime}{b^{\prime}}
\newcommand{\cprime}{c^{\prime}}
\newcommand{\dprime}{d^{\prime}}
\newcommand{\eprime}{e^{\prime}}
\newcommand{\fprime}{f^{\prime}}
\newcommand{\gprime}{g^{\prime}}
\newcommand{\hprime}{h^{\prime}}
\newcommand{\iprime}{i^{\prime}}
\newcommand{\jprime}{j^{\prime}}
\newcommand{\kprime}{k^{\prime}}
\newcommand{\lprime}{l^{\prime}}
\newcommand{\mprime}{m^{\prime}}
\newcommand{\nprime}{n^{\prime}}
\newcommand{\oprime}{o^{\prime}}
\newcommand{\pprime}{p^{\prime}}
\newcommand{\qprime}{q^{\prime}}
\newcommand{\rprime}{r^{\prime}}
\newcommand{\sprime}{s^{\prime}}
\newcommand{\tprime}{t^{\prime}}
\newcommand{\uprime}{u^{\prime}}
\newcommand{\vprime}{v^{\prime}}
\newcommand{\wprime}{w^{\prime}}
\newcommand{\xprime}{x^{\prime}}
\newcommand{\yprime}{y^{\prime}}
\newcommand{\zprime}{z^{\prime}}
\newcommand{\Aprime}{A^{\prime}}
\newcommand{\Bprime}{B^{\prime}}
\newcommand{\Cprime}{C^{\prime}}
\newcommand{\Dprime}{D^{\prime}}
\newcommand{\Eprime}{E^{\prime}}
\newcommand{\Fprime}{F^{\prime}}
\newcommand{\Gprime}{G^{\prime}}
\newcommand{\Hprime}{H^{\prime}}
\newcommand{\Iprime}{I^{\prime}}
\newcommand{\Jprime}{J^{\prime}}
\newcommand{\Kprime}{K^{\prime}}
\newcommand{\Lprime}{L^{\prime}}
\newcommand{\Mprime}{M^{\prime}}
\newcommand{\Nprime}{N^{\prime}}
\newcommand{\Oprime}{O^{\prime}}
\newcommand{\Pprime}{P^{\prime}}
\newcommand{\Qprime}{Q^{\prime}}
\newcommand{\Rprime}{R^{\prime}}
\newcommand{\Sprime}{S^{\prime}}
\newcommand{\Tprime}{T^{\prime}}
\newcommand{\Uprime}{U^{\prime}}
\newcommand{\Vprime}{V^{\prime}}
\newcommand{\Wprime}{W^{\prime}}
\newcommand{\Xprime}{X^{\prime}}
\newcommand{\Yprime}{Y^{\prime}}
\newcommand{\Zprime}{Z^{\prime}}
\newcommand{\alphaprime}{\alpha^{\prime}}
\newcommand{\betaprime}{\beta^{\prime}}
\newcommand{\gammaprime}{\gamma^{\prime}}
\newcommand{\Gammaprime}{\Gamma^{\prime}}
\newcommand{\deltaprime}{\delta^{\prime}}
\newcommand{\epsilonprime}{\epsilon^{\prime}}
\newcommand{\kappaprime}{\kappa^{\prime}}
\newcommand{\lambdaprime}{\lambda^{\prime}}
\newcommand{\rhoprime}{\rho^{\prime}}
\newcommand{\Sigmaprime}{\Sigma^{\prime}}
\newcommand{\tauprime}{\tau^{\prime}}
\newcommand{\xiprime}{\xi^{\prime}}
\newcommand{\thetaprime}{\theta^{\prime}}
\newcommand{\Omegaprime}{\Omega^{\prime}}
\newcommand{\cMprime}{\cM^{\prime}}
\newcommand{\cNprime}{\cN^{\prime}}
\newcommand{\cPprime}{\cP^{\prime}}
\newcommand{\cQprime}{\cQ^{\prime}}
\newcommand{\cRprime}{\cR^{\prime}}
\newcommand{\cSprime}{\cS^{\prime}}
\newcommand{\cTprime}{\cT^{\prime}}

%%%%%%%%%%%%%%%%%%%%%%%%%%%%%%%%%%%%%%%%%%%%%
%          bar Letters
%%%%%%%%%%%%%%%%%%%%%%%%%%%%%%%%%%%%%%%%%%%%%
\newcommand{\abar}{\bar{a}}
\newcommand{\bbar}{\bar{b}}
\newcommand{\zbar}{\bar{z}}
\newcommand{\phibar}{\bar{\varphi}}
\newcommand{\psibar}{\bar{\psi}}
\newcommand{\thetabar}{\bar{\theta}}
\newcommand{\nubar}{\bar{\nu}}

%%%%%%%%%%%%%%%%%%%%%%%%%%%%%%%%%%%%%%%%%%%%%
%          star Letters
%%%%%%%%%%%%%%%%%%%%%%%%%%%%%%%%%%%%%%%%%%%%%
\newcommand{\phistar}{\phi^{*}}


%%%%%%%%%%%%%%%%%%%%%%%%%%%%%%%%%%%%%%%%%%%%%
%          Formulas
%%%%%%%%%%%%%%%%%%%%%%%%%%%%%%%%%%%%%%%%%%%%%

\newcommand{\formulaphi}{\text{\large $\varphi$}}
\newcommand{\Formulaphi}{\text{\Large $\varphi$}}


%%%%%%%%%%%%%%%%%%%%%%%%%%%%%%%%%%%%%%%%%%%%%
%          Fraktur Letters
%%%%%%%%%%%%%%%%%%%%%%%%%%%%%%%%%%%%%%%%%%%%%

\newcommand{\fa}{\mathfrak{a}}
\newcommand{\fb}{\mathfrak{b}}
\newcommand{\fc}{\mathfrak{c}}
\newcommand{\fk}{\mathfrak{k}}
\newcommand{\fp}{\mathfrak{p}}
\newcommand{\fq}{\mathfrak{q}}
\newcommand{\fr}{\mathfrak{r}}
\newcommand{\fA}{\mathfrak{A}}
\newcommand{\fB}{\mathfrak{B}}
\newcommand{\fC}{\mathfrak{C}}
\newcommand{\fD}{\mathfrak{D}}

%%%%%%%%%%%%%%%%%%%%%%%%%%%%%%%%%%%%%%%%%%%%%
%          Bold Letters
%%%%%%%%%%%%%%%%%%%%%%%%%%%%%%%%%%%%%%%%%%%%%
\newcommand{\ba}{\mathbf{a}}
\newcommand{\bb}{\mathbf{b}}
\newcommand{\bc}{\mathbf{c}}
\newcommand{\bd}{\mathbf{d}}
\newcommand{\be}{\mathbf{e}}
\newcommand{\bu}{\mathbf{u}}
\newcommand{\bv}{\mathbf{v}}
\newcommand{\bw}{\mathbf{w}}
\newcommand{\bx}{\mathbf{x}}
\newcommand{\by}{\mathbf{y}}
\newcommand{\bz}{\mathbf{z}}
\newcommand{\bSigma}{\boldsymbol{\Sigma}}
\newcommand{\bPi}{\boldsymbol{\Pi}}
\newcommand{\bDelta}{\boldsymbol{\Delta}}
\newcommand{\bdelta}{\boldsymbol{\delta}}
\newcommand{\bgamma}{\boldsymbol{\gamma}}
\newcommand{\bGamma}{\boldsymbol{\Gamma}}

%%%%%%%%%%%%%%%%%%%%%%%%%%%%%%%%%%%%%%%%%%%%%
%         Bold numbers
%%%%%%%%%%%%%%%%%%%%%%%%%%%%%%%%%%%%%%%%%%%%%
\newcommand{\bzero}{\mathbf{0}}

%%%%%%%%%%%%%%%%%%%%%%%%%%%%%%%%%%%%%%%%%%%%%
% Projective-Like Pointclasses
%%%%%%%%%%%%%%%%%%%%%%%%%%%%%%%%%%%%%%%%%%%%%
\newcommand{\Sa}[2][\alpha]{\Sigma_{(#1,#2)}}
\newcommand{\Pa}[2][\alpha]{\Pi_{(#1,#2)}}
\newcommand{\Da}[2][\alpha]{\Delta_{(#1,#2)}}
\newcommand{\Aa}[2][\alpha]{A_{(#1,#2)}}
\newcommand{\Ca}[2][\alpha]{C_{(#1,#2)}}
\newcommand{\Qa}[2][\alpha]{Q_{(#1,#2)}}
\newcommand{\da}[2][\alpha]{\delta_{(#1,#2)}}
\newcommand{\leqa}[2][\alpha]{\leq_{(#1,#2)}}
\newcommand{\lessa}[2][\alpha]{<_{(#1,#2)}}
\newcommand{\equiva}[2][\alpha]{\equiv_{(#1,#2)}}


\newcommand{\Sl}[1]{\Sa[\lambda]{#1}}
\newcommand{\Pl}[1]{\Pa[\lambda]{#1}}
\newcommand{\Dl}[1]{\Da[\lambda]{#1}}
\newcommand{\Al}[1]{\Aa[\lambda]{#1}}
\newcommand{\Cl}[1]{\Ca[\lambda]{#1}}
\newcommand{\Ql}[1]{\Qa[\lambda]{#1}}

\newcommand{\San}{\Sa{n}}
\newcommand{\Pan}{\Pa{n}}
\newcommand{\Dan}{\Da{n}}
\newcommand{\Can}{\Ca{n}}
\newcommand{\Qan}{\Qa{n}}
\newcommand{\Aan}{\Aa{n}}
\newcommand{\dan}{\da{n}}
\newcommand{\leqan}{\leqa{n}}
\newcommand{\lessan}{\lessa{n}}
\newcommand{\equivan}{\equiva{n}}

%%%%%%%%%%%%%%%%%%%%%%%%%%%%%%%%%%%%%%%%%%%%%
% Linear Algebra
%%%%%%%%%%%%%%%%%%%%%%%%%%%%%%%%%%%%%%%%%%%%%
\newcommand{\transpose}[1]{{#1}^{\text{T}}}
\newcommand{\norm}[1]{\lVert{#1}\rVert}
\newcommand\aug{\fboxsep=-\fboxrule\!\!\!\fbox{\strut}\!\!\!}

%%%%%%%%%%%%%%%%%%%%%%%%%%%%%%%%%%%%%%%%%%%%%
% Number Theory
%%%%%%%%%%%%%%%%%%%%%%%%%%%%%%%%%%%%%%%%%%%%%
\DeclareMathOperator{\Spec}{Spec}
\newcommand{\av}[1]{\lvert#1\rvert}
\DeclareMathOperator{\divides}{\mid}
\DeclareMathOperator{\ndivides}{\nmid}

%%%%%%%%%%%%%%%%%%%%%%%%%%%%%%%%%%%%%%%%%%%%%%%%%%%%%%%%%%%%%%%%%%%%%%%%%%%
%%  Theorem-Like Declarations
%%%%%%%%%%%%%%%%%%%%%%%%%%%%%%%%%%%%%%%%%%%%%%%%%%%%%%%%%%%%%%%%%%%%%%%%%%

\newtheorem{theorem}{Theorem}[section]
\newtheorem{lemma}[theorem]{Lemma}
\newtheorem{corollary}[theorem]{Corollary}
\newtheorem{proposition}[theorem]{Proposition}


\theoremstyle{definition}

\newtheorem{definition}[theorem]{Definition}
\newtheorem{conjecture}[theorem]{Conjecture}
\newtheorem{remark}[theorem]{Remark}
\newtheorem{remarks}[theorem]{Remarks}
\newtheorem{notation}[theorem]{Notation}

\newtheorem{homework}[theorem]{Exercise}
\newtheorem{numbered_example}[theorem]{Example}

\theoremstyle{remark}

\newtheorem*{aside}{Aside}
\newtheorem*{note}{Note}
\newtheorem*{observation}{Observation}
\newtheorem*{warning}{Warning}
\newtheorem*{question}{Question}
\newtheorem*{example}{Example}
\newtheorem*{in_class_example}{In-Class Example}
\newtheorem*{exercise}{Exercise}
\newtheorem*{fact}{Fact}


\newenvironment*{subproof}[1][Proof]
{\begin{proof}[#1]}{\renewcommand{\qedsymbol}{$\diamondsuit$} \end{proof}}

\newenvironment*{case}[1]
{\textbf{Case #1.  }\itshape }{}

\newenvironment*{claim}[1][Claim]
{\textbf{#1.  }\itshape }{}


\pagestyle{plain}

\begin{document}

\title{Multilinear Algebra Notes}
\author{Mitch Rudominer}

\maketitle

\tableofcontents

%%%%%%%%%%%%%%%%%%%%%%%%%%%%%%%%%%%%%%%%%%%%%%%%%%%%%%%%%%%%%%%%%%%%%%%%%%%%%%%%%%%%%%%%

\section{Covariant and Contravariant Tensors}

We study covariant and contravariant tensors and explain the meanings of those terms.

\subsection{Index Notation}

Let $V$ and $W$ be finite-dimensional vector spaces, 
$\dim(V)=n$, $\dim(W) = m$. Let $\pi:V\map W$ be linear.

Let $\cB = \singleton{v_1,\cdots,v_n}$ be a basis for $V$ and
$\cC=\singleton{w_1,\cdots,w_m}$ a basis for $W$.

We will use \emph{index notation} or \emph{Einstein notation}
from physics in which variables representing an indexed set of numbers or other objects will
be decorated with indices that may be either upper (superscript) or lower (subscript) and
an expression
like $a^i_j b^j$ means either the $m$-tuple 
$\sequence{\sum_{j=1}^n(a^i_j b^j)}{i=1\cdots m}$
or the number $\sum_{j=1}^n(a^i_j b^j)$ depending on the context. 
And $n$ and $m$ are also derived from the context.

\begin{itemize}
\item Paired upper and lower variables are bound and indicate summation over an implicit range of indices
\item Unpaired variables are free and indicate either a single index or an implicit range of indices
\end{itemize}

The letter $\delta$ with two indices, upper or lower, will always mean the Kronecker delta. So for example $\delta^i_j$ equals 1 if $i=j$ and equals 0 if $i\neq j$. 
As above ``$\delta^i_j$" might refer to a single number or a 2-dimensional array
of numbers indexed by $i$ an $j$, depending on the context. In the case that
$\delta^i_j$ refers to a square matrix, then, it is the identity matrix.


\begin{definition}
 For any matrix $A$ we write $A=(a^i_j)$ to mean that $a^i_j$ is the
 element of $A$ in row $i$, column $j$. 
\end{definition}

\begin{remark}
 So the upper index is the row and the lower index is the column. Commensurate with this
 when an $n$-tuple of numbers or objects is written with an upper index like $a^i$ we
 will think of it as a \emph{column} vector and when written with a lower index like $a_j$
 we will think of it as a \emph{row} vector.

Different types of one-dimensional arrays of objects will be written with either upper or
lower indices. We have seen two examples already above and we will see more below. We
summarize some of them here.
\begin{itemize}
\item Basis vectors are indexed with lower indices like $v_i$ and $w_i$. 
\item The $n$-tuple of components of a single vector in a given basis will be written using upper indices like $a^i$.
Therefore we are thinking of each vector as a column and the list of all basis vectors as a row.
\item When a vector space is fixed by the context then we will think of the elements of its dual space as \emph{covectors}.
Basis covectors are indexed with upper indices like $v^i$ and $w^i$.
\item The $n$-tuple of components of a single covector in a given basis will be written using lower indices like $a_i$.
Therefore we are thinking of each covector as a row and the list of all basis covectors as a column.
\end{itemize}
\end{remark}


\begin{definition}
$M^{\pi}_{\cB,\cC}$ is the $m\times n$ matrix of $\pi$ 
relative to the two bases:

 Letting $M^{\pi}_{\cB,\cC}=A=(a^i_j)$,
$a^i_j$ is defined by
$$\pi(c^jv_j) = a^i_j c^j w_i$$
for $1\leq i \leq m$, $1\leq j \leq n$.

In particular we have
$$\pi(v_j) = a^i_j  w_i$$
for $1\leq i \leq m$, $1\leq j \leq n$,
so that $a^i_j$ is the $i$-th component in the $\cC$-representation of $\pi(v_j)$.
\end{definition}

\begin{definition}
$V^*$, $W^*$ are the dual-spaces of $V$ and $W$ respectively.
$\pi^*:W^*\map V^*$ is the \emph{adjoint} of $\pi$ defined by
$\pi^*(\varphi) \cdot v = \varphi \cdot \pi(v)$.
\end{definition}

Note that above and going forward, if
$\varphi$ is a covector and $v$ is a vector we write $\varphi(v)$
as $\varphi \cdot v$.

Let
$\hat{\cB} = \singleton{v^1,\cdots,v^n}$ and
$\hat{\cC} = \singleton{w^1,\cdots,w^m}$ be the corresponding
dual bases for $V^*$ and $W^*$ respectively. 
So $v^i \cdot v_j = \delta^i_j$ and $w^i \cdot w_j = \delta^i_j$.

\begin{lemma}
Let $A = M^{\pi}_{\cB,\cC} =(a^i_j)$.

Then $M^{\pi^*}_{\hat{\cC},\hat{\cB}}  = \transpose{A}$.

But we must be careful with the notation here!

Because we write vectors as column vectors and multiply them by matrices on the left
and we write co-vectors as row-vectors and multiply them by matrices on the right, it
is actually $A$ itself and not $\transpose{A}$ that we will use to represent
$M^{\pi^*}_{\hat{\cC},\hat{\cB}}$.


In index notation, we write the components of vectors using upper indices and the components
of covectors using lower indices. So when representing $\pi$, $A$ acts as
$c^j \to a^i_jc^j$ but when representing $\pi^*$, $A$ acts as $d_k \to a^k_j d_k$.
\end{lemma}
\begin{proof}
Let $v=c^j v_j$ and let $\varphi = d_k w^k$. Then $\pi^*(\varphi)\cdot v = \varphi \cdot \pi(v)
= d_k w^k \cdot a^i_j c^j w_i = d_k a^k_j c ^ j$. Therefore $\pi^*(\varphi) = d_k a^k_j v^j$.
\end{proof}


\begin{corollary}
Suppose $n=m$ and $\pi$ is a linear isomorphism. Then
$(\pi^*)^{-1}:V^* \map W^*$ is defined
and letting $A = M^{\pi}_{\cB,\cC}$ we have that
$M^{(\pi^*)^{-1}}_{\cB,\cC} = (\transpose{A})^{-1}$.

Again, because we write covectors as row vectors (equivalently we write the components of covectors using lower indices), it is actually $A^{-1}$ we will use to represent $(\pi^*)^{-1}.$

\end{corollary}

We will sometimes use the following notation:
If $A =(a^i_j)$ is an invertable square matrix, we will write $(\atwiddle^i_j)$ to implicitly
mean $A^{-1} = (\atwiddle^i_j)$.

Thus we have $a^i_j \atwiddle^j_k = \delta^i_k$.

So the above corollary can be stated like this:

If $\pi(c^j v_j) = a^i_j c^j w_i$, then
$(\pi^*)^{-1} (c_i v^i) = \atwiddle^i_j c_i w^j$.

\begin{remarks}
The above results are summarized here:
\begin{enumerate}
\item $v_j \mapsto a^i_j w_i$ (single vector) defines $\pi$ on  $\cB$ and $\cC$.
\item $c^j \mapsto a^i_j c ^j$ ($n$-tuple of numbers) ($\cB$ - $\cC$)-matrix of $\pi$.
\item $w^i \mapsto a^i_j v^j$ (single covector) defines $\pi^*$ on $\hat{\cC}$ and $\hat{\cB}$.
\item $d_i \mapsto a^i_j d_i$ ($m$-tuple of numbers) ($\hat{\cC}$ - $\hat{\cB}$)-matrix of $\pi^*$.
\item $c_i \mapsto \atwiddle^i_j c_i$ ($n=m$-tuple of numbers) ($\hat{\cB}$ - $\hat{\cC}$)-matrix of $(\pi^*)^{-1}$.
\end{enumerate}
\end{remarks}



\begin{corollary}
Suppose that not only is $\pi$ a linear isomorphism but that it preserves
the inner products induced by the two bases. For example we might have
started with $V$ and $W$ being inner product spaces and $\cB$ and $\cC$
being orthonormal bases.

Letting $A = M^{\pi}_{\cB,\cC}$ we have that
$M^{(\pi^*)^{-1}}_{\cB,\cC} = A$.
\end{corollary}
\begin{proof}
$\pi$ preserves the inner product iff $\transpose{A} = A^{-1}$.
\end{proof}

\begin{note}
So when $\pi$ preserves inner product then
the dual map in the dual bases has the same matrix as the
regular map in the regular bases.
Awkwardly, because we represent covectors using row vectors we would
need to use $\transpose{A}$ instead of $A$ to represent the dual map
and then we would need to introduce another letter for the
components of $\transpose{A}$.
To avoid this awkwardness we might stop making the distinction between
the spaces and their duals, using the natural isomorphisms given by
the inner products, and identify $(\pi^*)^{-1}$ with $\pi$.


In this case if $v,w \in V$, $v=(c^i)$, $w=(e^i)$ in the basis $\cB$,
then 
$c_i e^i$ is interpretted as the dot product computation of the inner product $v \cdot w $,
so that implicitly we are defining $c_i = c^i$.

The following computation is related to the above corrolary: Let $v,w\in\R^n$. Then
$$\transpose{(Av)}Aw = \transpose{v}\transpose{A}Aw = \transpose{v}w$$
since $\transpose{A}A = I$.

It is akward to try to write the above computation using index notation. One is
tempted to write $a^j_i$ to mean the transpose of $a^i_j$, but this is a bad
idea because the choice of dummy variables should not have significance.

\end{note}

\subsection{Change of Basis}

We now specialize the above discussion to the case of a \emph{change of basis.}

Let $\pi: V \map V$ be a linear automorphism. Let $\cB^{\prime} = \singleton{u_1,\cdots,u_n}$
where $u_i = \pi(v_i)$. We will call $\cB$ the \emph{old basis} and $\cB^{\prime}$ the
\emph{new basis.}

Let $\id:V\map V$ be the identity function.

\begin{definition}
We temporarily introduce the following terminology.

\begin{itemize}
\item $M^{\id}_{\cB,\cB^{\prime}}$ is the \emph{forward-translation matrix}.
\item $M^{\id}_{\cB^{\prime},\cB}$ is the \emph{backwards-translation matrix}.
\item $M^{\pi}_{\cB,\cB}$ is the \emph{change-of-basis matrix}.
\end{itemize}
\end{definition}

\begin{lemma}
$M^{\pi}_{\cB,\cB} = M^{\id}_{\cB^{\prime},\cB}$. That is, the change-of-basis
matrix is also the backwards-translation matrix.
\end{lemma}
\begin{proof}
Clearly $M^{\pi}_{\cB,\cB^{\prime}} = \mathrm{I}$.
Since $\pi = \pi \circ \id$ we have
$$M^{\pi}_{\cB,\cB} = M^{\pi}_{\cB,\cB^{\prime}} \times M^{\id}_{\cB^{\prime},\cB} 
= \mathrm{I} \times M^{\id}_{\cB^{\prime},\cB} =  M^{\id}_{\cB^{\prime},\cB}.$$
\end{proof}

\begin{corollary}
$M^{\id}_{\cB,\cB^{\prime}} = \left(M^{\pi}_{\cB,\cB}\right)^{-1}$. That is, the forward-translation matrix
is the inverse of the change-of-basis matrix.
\end{corollary}
\begin{proof}
Clearly $M^{\id}_{\cB,\cB^{\prime}} = \left(M^{\id}_{\cB^{\prime},\cB}\right)^{-1}$.
\end{proof}

Let $(\pi^*)^{-1}: V^* \map V^*$ be the dual change-of-basis transformation. 
Let $\hat{\cB}^{\prime} = \singleton{u^1,\cdots,u^n}$
where $u^i = (\pi^*)^{-1}(v^i)$. We will call $\hat{\cB}$ the \emph{old dual basis} and $\hat{\cB}^{\prime}$ the
\emph{new dual basis.}

\begin{remarks}
\label{main-remarks}
The the results of this and the previous subsection are summarized here.
Let $A = (a^i_j) = M^{\pi}_{\cB,\cB}$ be the change-of-basis matrix. Let $v = c^i v_i \in V$.
\begin{enumerate}
\item $u_j = a^i_j v_i$ is $\pi$, the change-of-basis transformation. [$\transpose{A}$]
\item $d^i = a^i_j c^j$ gives $\pi$ in old coordinates. [$A$]
\item $c^i = a^i_j e^j$ is the backward-translation from new coordinates to old. [$A$]
\item $e^i = \atwiddle^i_j c^j$ is the forward-translation from old coordinates to new. [$A^{-1}$]

\item $u^i = \atwiddle^i_j v^j$ is $(\pi^*)^{-1}$, the dual change-of-basis transformation. $[A^{-1}$]
\item $d_j = \atwiddle^i_j c_i$ gives $(\pi^*)^{-1}$ in old coordinates [$(\transpose{A})^{-1}$]
\item $e_j = a^i_j c_i$ is the forward-dual-translation. [$\transpose{A}$]
\end{enumerate}
\end{remarks}

\begin{remark}
We will use the above remarks to explain the origin of the terminology \emph{covariant} and \emph{contravariant}.
\begin{itemize}
\item In general, objects whose components are indexed with lower indices are called \emph{covariant} and
objects whose components are indexed with upper indices are called \emph{contravariant.}
\item The origin of this terminology seems to be items (1), (4), and (7) above. 
\item Comparing item (1) and item (7)
we can say that ``the components of covectors transform in the same way as the basis vectors" so that
covectors ``co-vary with the change of basis". 
\item Comparing item (1) and item (4), people sometimes say that
``the components of vectors transform in the opposite way as the basis vectors" so that vectors are ``contra-variant."
\item The above bullet is of course imprecise as it is hard to see why $A^{-1}$ is the ``opposite" of  $\transpose{A}$.
A more precise opposite is comparing item (2) and item (4). It is easy to see why 
$A^{-1}$ is the ``opposite" of $A$.
\item For example,
if the basis vectors rotate counter-clockwise by 45 degrees, then the coordinates of $v$ rotate clockwise by 45 degrees.
If the basis vectors are doubled in length, then the coordinates of $v$ are cut in half.
\item Notice that the translation matrix for co-vectors is the inverse transpose of the translation matrix
for vectors.
\item It follows that if the change of basis is orthogonal, then the translation matrix for
vectors and covectors is the same. For example if the basis vectors are rotated 45 degrees counter-clockwise then
then the coordinates of a dual vector, just like the coordinates of a vector, are rotated 45 degrees clockwise.
\item So focusing on orthogonal transformations is not a good way to motivate the terms
covariant and contravaraint.
\item But consider instead symmetric transformations where $\transpose{A} = A$. For example consider
the change of basis in $\R^2$ in which the first basis vector is multiplied by 3 and the second
basis vector is multiplied by 5. Consider what happens to the coordinates of vectors and covectors.
For vectors: the first coordinate is divided by 3 and the second coordinate is divided by 5.
For covectors: the first coordinate is multiplied by 3 and the second coordinate is multiplied by 5.
This example helps motivate the terms covariant and contravariant.
\end{itemize}
\end{remark}

\subsection{The Chain Rule}
Let $U_1$ and $U_2$ be two open subsets of $\R^n$. Let $F:U_1 \to U_2$ be a diffeomorphism.
We write $(y^1,\cdots y^n) = F(x^1, \cdots, x^n)$.
We will think of this as the change-of-variable transformation between two different coordinate patches
on a manifold $M$. Fix a point $p\in M$ in the intersection of the domain of the two patches. Let 
$p_1 \in U_1$ and $p_2 = F(p_1) \in U_2$ be the corresponding points in $U_1$ and $U_2$.
We will focus on the derivative $DF$ at $p_1$. 
We will think of $DF:\R^n\map\R^n$ as the change-of-basis transformation for
$T_p(M)$, the tangent space at $p$. In this section we demonstrate that the
chain rule for partial derivatives yields the same results as the previous section.

Let $f$ be a smooth function on $M$. We will use the same name $f$ for the corresponding functions on 
$U_1$ and $U_2$.

The chain rule gives: 
$$\frac{\partial f}{\partial y^j} = \frac{\partial f}{\partial x^i}\frac{\partial x^i}{\partial y^j}.$$

Here we are extending the index notation to expressions involving derivatives and considering indices in the numerator as upper indices and indices in the denominator as lower indices.

Since the above is true for arbitrary $f$ we conclude that

\begin{equation}
\label{chain-rule-1}
\partial y_j = \frac{\partial x^i}{\partial y^j} \partial x_i.
\end{equation}

Here we are writing $\partial y_j$ as an abbreviation for $\frac{\partial}{\partial y^j}$,
and $\partial x_i$ as an abbreviation for $\frac{\partial}{\partial x^i}$.
$\singleton{\partial x_i}$ and $\singleton{\partial y_i}$ are the old and new basis
vectors for $T_p(M)$. We are extending the index notation to these abbreviations
and considering the indices on $\partial x_i$ and $\partial y_j$ as lower indices (which
is consistent with them being the indices of basis vectors.)

Notice then that equation \ref{chain-rule-1} corresponds to item (1) in Remarks \ref{main-remarks}:

\begin{itemize}
\item $u_j = a^i_j v_i$ is $\pi$, the change-of-basis transformation. [$\transpose{A}$].
\end{itemize}

Thus we using $A = (a^i_j)$ where $a^i_j = \frac{\partial x^i}{\partial y_j}$.

Notice that $DF = (\frac{\partial y^i}{\partial x^j}) = A^{-1}$.

By (4) from Remarks \ref{main-remarks}:
\begin{itemize}
\item $e^i = \atwiddle^i_j c^j$ is the forward-translation from old coordinates to new. [$A^{-1}$]
\end{itemize}

we have that if $v = c^j \partial x_j$ is a vector in $T_p(M)$ then
$v = e^i \partial y_i$ where $e^i = \frac{\partial y^i}{\partial x^j} c^j$.

\begin{remarks}
The the results of this subsection are summarized here. The items are numbered in parallel with
the corresponding items in Remarks \ref{main-remarks}.
\begin{enumerate}
\item $\partial y_j = \frac{\partial x^i}{\partial y^j} \partial x_i$, the change-of-basis transformation for tangent vectors
\item -- omitted --
\item $c^i = \frac{\partial x^i}{\partial y^j} e^j$ is the backward-translation from new tangent vector coordinates to old.
\item $e^i = \frac{\partial y^i}{\partial x^j} c^j$ is the forward-translation from old tangent coordinates to new.

\item $\deriv y^i = \frac{\partial y^i}{\partial x^j} \deriv x^j$ is the dual change-of-basis transformation for one forms. 
\item -- omitted --
\item $e_j = \frac{\partial x^i}{\partial y^j} c_i$ is the forward-translation from old one form coordinates
to new.
\end{enumerate}
\end{remarks}


 
\bibliographystyle{amsalpha}

\bibliography{math}

\end{document}
