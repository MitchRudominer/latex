\documentclass[oneside,12pt]{amsart}

\usepackage{amsmath,amssymb,latexsym,eucal,amsthm}

%%%%%%%%%%%%%%%%%%%%%%%%%%%%%%%%%%%%%%%%%%%%%
% Common Set Theory Constructs
%%%%%%%%%%%%%%%%%%%%%%%%%%%%%%%%%%%%%%%%%%%%%

\newcommand{\setof}[2]{\left\{ \, #1 \, \left| \, #2 \, \right.\right\}}
\newcommand{\lsetof}[2]{\left\{\left. \, #1 \, \right| \, #2 \,  \right\}}
\newcommand{\bigsetof}[2]{\bigl\{ \, #1 \, \bigm | \, #2 \,\bigr\}}
\newcommand{\Bigsetof}[2]{\Bigl\{ \, #1 \, \Bigm | \, #2 \,\Bigr\}}
\newcommand{\biggsetof}[2]{\biggl\{ \, #1 \, \biggm | \, #2 \,\biggr\}}
\newcommand{\Biggsetof}[2]{\Biggl\{ \, #1 \, \Biggm | \, #2 \,\Biggr\}}
\newcommand{\dotsetof}[2]{\left\{ \, #1 \, : \, #2 \, \right\}}
\newcommand{\bigdotsetof}[2]{\bigl\{ \, #1 \, : \, #2 \,\bigr\}}
\newcommand{\Bigdotsetof}[2]{\Bigl\{ \, #1 \, \Bigm : \, #2 \,\Bigr\}}
\newcommand{\biggdotsetof}[2]{\biggl\{ \, #1 \, \biggm : \, #2 \,\biggr\}}
\newcommand{\Biggdotsetof}[2]{\Biggl\{ \, #1 \, \Biggm : \, #2 \,\Biggr\}}
\newcommand{\sequence}[2]{\left\langle \, #1 \,\left| \, #2 \, \right. \right\rangle}
\newcommand{\lsequence}[2]{\left\langle\left. \, #1 \, \right| \,#2 \,  \right\rangle}
\newcommand{\bigsequence}[2]{\bigl\langle \,#1 \, \bigm | \, #2 \, \bigr\rangle}
\newcommand{\Bigsequence}[2]{\Bigl\langle \,#1 \, \Bigm | \, #2 \, \Bigr\rangle}
\newcommand{\biggsequence}[2]{\biggl\langle \,#1 \, \biggm | \, #2 \, \biggr\rangle}
\newcommand{\Biggsequence}[2]{\Biggl\langle \,#1 \, \Biggm | \, #2 \, \Biggr\rangle}
\newcommand{\singleton}[1]{\left\{#1\right\}}
\newcommand{\angles}[1]{\left\langle #1 \right\rangle}
\newcommand{\bigangles}[1]{\bigl\langle #1 \bigr\rangle}
\newcommand{\Bigangles}[1]{\Bigl\langle #1 \Bigr\rangle}
\newcommand{\biggangles}[1]{\biggl\langle #1 \biggr\rangle}
\newcommand{\Biggangles}[1]{\Biggl\langle #1 \Biggr\rangle}


\newcommand{\force}[1]{\Vert\!\underset{\!\!\!\!\!#1}{\!\!\!\relbar\!\!\!%
\relbar\!\!\relbar\!\!\relbar\!\!\!\relbar\!\!\relbar\!\!\relbar\!\!\!%
\relbar\!\!\relbar\!\!\relbar}}
\newcommand{\longforce}[1]{\Vert\!\underset{\!\!\!\!\!#1}{\!\!\!\relbar\!\!\!%
\relbar\!\!\relbar\!\!\relbar\!\!\!\relbar\!\!\relbar\!\!\relbar\!\!\!%
\relbar\!\!\relbar\!\!\relbar\!\!\relbar\!\!\relbar\!\!\relbar\!\!\relbar\!\!\relbar}}
\newcommand{\nforce}[1]{\Vert\!\underset{\!\!\!\!\!#1}{\!\!\!\relbar\!\!\!%
\relbar\!\!\relbar\!\!\relbar\!\!\!\relbar\!\!\relbar\!\!\relbar\!\!\!%
\relbar\!\!\not\relbar\!\!\relbar}}
\newcommand{\forcein}[2]{\overset{#2}{\Vert\underset{\!\!\!\!\!#1}%
{\!\!\!\relbar\!\!\!\relbar\!\!\relbar\!\!\relbar\!\!\!\relbar\!\!\relbar\!%
\!\relbar\!\!\!\relbar\!\!\relbar\!\!\relbar\!\!\relbar\!\!\!\relbar\!\!%
\relbar\!\!\relbar}}}

\newcommand{\pre}[2]{{}^{#2}{#1}}

\newcommand{\restr}{\!\!\upharpoonright\!}

%%%%%%%%%%%%%%%%%%%%%%%%%%%%%%%%%%%%%%%%%%%%%
% Set-Theoretic Connectives
%%%%%%%%%%%%%%%%%%%%%%%%%%%%%%%%%%%%%%%%%%%%%

\newcommand{\intersect}{\cap}
\newcommand{\union}{\cup}
\newcommand{\Intersection}[1]{\bigcap\limits_{#1}}
\newcommand{\Union}[1]{\bigcup\limits_{#1}}
\newcommand{\adjoin}{{}^\frown}
\newcommand{\forces}{\Vdash}

%%%%%%%%%%%%%%%%%%%%%%%%%%%%%%%%%%%%%%%%%%%%%
% Miscellaneous
%%%%%%%%%%%%%%%%%%%%%%%%%%%%%%%%%%%%%%%%%%%%%
\newcommand{\defeq}{=_{\text{def}}}
\newcommand{\Turingleq}{\leq_{\text{T}}}
\newcommand{\Turingless}{<_{\text{T}}}
\newcommand{\lexleq}{\leq_{\text{lex}}}
\newcommand{\lexless}{<_{\text{lex}}}
\newcommand{\Turingequiv}{\equiv_{\text{T}}}
\newcommand{\isomorphic}{\cong}

%%%%%%%%%%%%%%%%%%%%%%%%%%%%%%%%%%%%%%%%%%%%%
% Constants
%%%%%%%%%%%%%%%%%%%%%%%%%%%%%%%%%%%%%%%%%%%%%
\newcommand{\R}{\mathbb{R}}
\renewcommand{\P}{\mathbb{P}}
\newcommand{\Q}{\mathbb{Q}}
\newcommand{\Z}{\mathbb{Z}}
\newcommand{\Zpos}{\Z^{+}}
\newcommand{\Znonneg}{\Z^{\geq 0}}
\newcommand{\C}{\mathbb{C}}
\newcommand{\N}{\mathbb{N}}
\newcommand{\B}{\mathbb{B}}
\newcommand{\Bairespace}{\pre{\omega}{\omega}}
\newcommand{\LofR}{L(\R)}
\newcommand{\JofR}[1]{J_{#1}(\R)}
\newcommand{\SofR}[1]{S_{#1}(\R)}
\newcommand{\JalphaR}{\JofR{\alpha}}
\newcommand{\JbetaR}{\JofR{\beta}}
\newcommand{\JlambdaR}{\JofR{\lambda}}
\newcommand{\SalphaR}{\SofR{\alpha}}
\newcommand{\SbetaR}{\SofR{\beta}}
\newcommand{\Pkl}{\mathcal{P}_{\kappa}(\lambda)}
\DeclareMathOperator{\con}{con}
\DeclareMathOperator{\ORD}{OR}
\DeclareMathOperator{\Ord}{OR}
\DeclareMathOperator{\WO}{WO}
\DeclareMathOperator{\OD}{OD}
\DeclareMathOperator{\HOD}{HOD}
\DeclareMathOperator{\HC}{HC}
\DeclareMathOperator{\WF}{WF}
\DeclareMathOperator{\wfp}{wfp}
\DeclareMathOperator{\HF}{HF}
\newcommand{\One}{I}
\newcommand{\ONE}{I}
\newcommand{\Two}{II}
\newcommand{\TWO}{II}
\newcommand{\Mladder}{M^{\text{ld}}}

%%%%%%%%%%%%%%%%%%%%%%%%%%%%%%%%%%%%%%%%%%%%%
% Commutative Algebra Constants
%%%%%%%%%%%%%%%%%%%%%%%%%%%%%%%%%%%%%%%%%%%%%
\DeclareMathOperator{\dottimes}{\dot{\times}}
\DeclareMathOperator{\dotminus}{\dot{-}}
\DeclareMathOperator{\Spec}{Spec}

%%%%%%%%%%%%%%%%%%%%%%%%%%%%%%%%%%%%%%%%%%%%%
% Theories
%%%%%%%%%%%%%%%%%%%%%%%%%%%%%%%%%%%%%%%%%%%%%
\DeclareMathOperator{\ZFC}{ZFC}
\DeclareMathOperator{\ZF}{ZF}
\DeclareMathOperator{\AD}{AD}
\DeclareMathOperator{\ADR}{AD_{\R}}
\DeclareMathOperator{\KP}{KP}
\DeclareMathOperator{\PD}{PD}
\DeclareMathOperator{\CH}{CH}
\DeclareMathOperator{\GCH}{GCH}
\DeclareMathOperator{\HPC}{HPC} % HOD pair capturing
%%%%%%%%%%%%%%%%%%%%%%%%%%%%%%%%%%%%%%%%%%%%%
% Iteration Trees
%%%%%%%%%%%%%%%%%%%%%%%%%%%%%%%%%%%%%%%%%%%%%

\newcommand{\pred}{\text{-pred}}

%%%%%%%%%%%%%%%%%%%%%%%%%%%%%%%%%%%%%%%%%%%%%%%%
% Operator Names
%%%%%%%%%%%%%%%%%%%%%%%%%%%%%%%%%%%%%%%%%%%%%%%%
\DeclareMathOperator{\Det}{Det}
\DeclareMathOperator{\dom}{dom}
\DeclareMathOperator{\ran}{ran}
\DeclareMathOperator{\range}{ran}
\DeclareMathOperator{\image}{image}
\DeclareMathOperator{\crit}{crit}
\DeclareMathOperator{\card}{card}
\DeclareMathOperator{\cf}{cf}
\DeclareMathOperator{\cof}{cof}
\DeclareMathOperator{\rank}{rank}
\DeclareMathOperator{\ot}{o.t.}
\DeclareMathOperator{\ords}{o}
\DeclareMathOperator{\ro}{r.o.}
\DeclareMathOperator{\rud}{rud}
\DeclareMathOperator{\Powerset}{\mathcal{P}}
\DeclareMathOperator{\length}{lh}
\DeclareMathOperator{\lh}{lh}
\DeclareMathOperator{\limit}{lim}
\DeclareMathOperator{\fld}{fld}
\DeclareMathOperator{\projection}{p}
\DeclareMathOperator{\Ult}{Ult}
\DeclareMathOperator{\ULT}{Ult}
\DeclareMathOperator{\Coll}{Coll}
\DeclareMathOperator{\Col}{Col}
\DeclareMathOperator{\Hull}{Hull}
\DeclareMathOperator*{\dirlim}{dir lim}
\DeclareMathOperator{\Scale}{Scale}
\DeclareMathOperator{\supp}{supp}
\DeclareMathOperator{\trancl}{tran.cl.}
\DeclareMathOperator{\trace}{Tr}
\DeclareMathOperator{\diag}{diag}
\DeclareMathOperator{\spn}{span}
\DeclareMathOperator{\sgn}{sgn}
%%%%%%%%%%%%%%%%%%%%%%%%%%%%%%%%%%%%%%%%%%%%%
% Logical Connectives
%%%%%%%%%%%%%%%%%%%%%%%%%%%%%%%%%%%%%%%%%%%%%
\newcommand{\IImplies}{\Longrightarrow}
\newcommand{\SkipImplies}{\quad\Longrightarrow\quad}
\newcommand{\Ifff}{\Longleftrightarrow}
\newcommand{\iimplies}{\longrightarrow}
\newcommand{\ifff}{\longleftrightarrow}
\newcommand{\Implies}{\Rightarrow}
\newcommand{\Iff}{\Leftrightarrow}
\renewcommand{\implies}{\rightarrow}
\renewcommand{\iff}{\leftrightarrow}
\newcommand{\AND}{\wedge}
\newcommand{\OR}{\vee}
\newcommand{\st}{\text{ s.t. }}
\newcommand{\Or}{\text{ or }}

%%%%%%%%%%%%%%%%%%%%%%%%%%%%%%%%%%%%%%%%%%%%%
% Function Arrows
%%%%%%%%%%%%%%%%%%%%%%%%%%%%%%%%%%%%%%%%%%%%%

\newcommand{\injection}{\xrightarrow{\text{1-1}}}
\newcommand{\surjection}{\xrightarrow{\text{onto}}}
\newcommand{\bijection}{\xrightarrow[\text{onto}]{\text{1-1}}}
\newcommand{\cofmap}{\xrightarrow{\text{cof}}}
\newcommand{\map}{\rightarrow}

%%%%%%%%%%%%%%%%%%%%%%%%%%%%%%%%%%%%%%%%%%%%%
% Mouse Comparison Operators
%%%%%%%%%%%%%%%%%%%%%%%%%%%%%%%%%%%%%%%%%%%%%
\newcommand{\initseg}{\trianglelefteq}
\newcommand{\properseg}{\lhd}
\newcommand{\notinitseg}{\ntrianglelefteq}
\newcommand{\notproperseg}{\ntriangleleft}

%%%%%%%%%%%%%%%%%%%%%%%%%%%%%%%%%%%%%%%%%%%%%
%           calligraphic letters
%%%%%%%%%%%%%%%%%%%%%%%%%%%%%%%%%%%%%%%%%%%%%
\newcommand{\cA}{\mathcal{A}}
\newcommand{\cB}{\mathcal{B}}
\newcommand{\cC}{\mathcal{C}}
\newcommand{\cD}{\mathcal{D}}
\newcommand{\cE}{\mathcal{E}}
\newcommand{\cF}{\mathcal{F}}
\newcommand{\cG}{\mathcal{G}}
\newcommand{\cH}{\mathcal{H}}
\newcommand{\cI}{\mathcal{I}}
\newcommand{\cJ}{\mathcal{J}}
\newcommand{\cK}{\mathcal{K}}
\newcommand{\cL}{\mathcal{L}}
\newcommand{\cM}{\mathcal{M}}
\newcommand{\cN}{\mathcal{N}}
\newcommand{\cO}{\mathcal{O}}
\newcommand{\cP}{\mathcal{P}}
\newcommand{\cQ}{\mathcal{Q}}
\newcommand{\cR}{\mathcal{R}}
\newcommand{\cS}{\mathcal{S}}
\newcommand{\cT}{\mathcal{T}}
\newcommand{\cU}{\mathcal{U}}
\newcommand{\cV}{\mathcal{V}}
\newcommand{\cW}{\mathcal{W}}
\newcommand{\cX}{\mathcal{X}}
\newcommand{\cY}{\mathcal{Y}}
\newcommand{\cZ}{\mathcal{Z}}


%%%%%%%%%%%%%%%%%%%%%%%%%%%%%%%%%%%%%%%%%%%%%
%          Primed Letters
%%%%%%%%%%%%%%%%%%%%%%%%%%%%%%%%%%%%%%%%%%%%%
\newcommand{\aprime}{a^{\prime}}
\newcommand{\bprime}{b^{\prime}}
\newcommand{\cprime}{c^{\prime}}
\newcommand{\dprime}{d^{\prime}}
\newcommand{\eprime}{e^{\prime}}
\newcommand{\fprime}{f^{\prime}}
\newcommand{\gprime}{g^{\prime}}
\newcommand{\hprime}{h^{\prime}}
\newcommand{\iprime}{i^{\prime}}
\newcommand{\jprime}{j^{\prime}}
\newcommand{\kprime}{k^{\prime}}
\newcommand{\lprime}{l^{\prime}}
\newcommand{\mprime}{m^{\prime}}
\newcommand{\nprime}{n^{\prime}}
\newcommand{\oprime}{o^{\prime}}
\newcommand{\pprime}{p^{\prime}}
\newcommand{\qprime}{q^{\prime}}
\newcommand{\rprime}{r^{\prime}}
\newcommand{\sprime}{s^{\prime}}
\newcommand{\tprime}{t^{\prime}}
\newcommand{\uprime}{u^{\prime}}
\newcommand{\vprime}{v^{\prime}}
\newcommand{\wprime}{w^{\prime}}
\newcommand{\xprime}{x^{\prime}}
\newcommand{\yprime}{y^{\prime}}
\newcommand{\zprime}{z^{\prime}}
\newcommand{\Aprime}{A^{\prime}}
\newcommand{\Bprime}{B^{\prime}}
\newcommand{\Cprime}{C^{\prime}}
\newcommand{\Dprime}{D^{\prime}}
\newcommand{\Eprime}{E^{\prime}}
\newcommand{\Fprime}{F^{\prime}}
\newcommand{\Gprime}{G^{\prime}}
\newcommand{\Hprime}{H^{\prime}}
\newcommand{\Iprime}{I^{\prime}}
\newcommand{\Jprime}{J^{\prime}}
\newcommand{\Kprime}{K^{\prime}}
\newcommand{\Lprime}{L^{\prime}}
\newcommand{\Mprime}{M^{\prime}}
\newcommand{\Nprime}{N^{\prime}}
\newcommand{\Oprime}{O^{\prime}}
\newcommand{\Pprime}{P^{\prime}}
\newcommand{\Qprime}{Q^{\prime}}
\newcommand{\Rprime}{R^{\prime}}
\newcommand{\Sprime}{S^{\prime}}
\newcommand{\Tprime}{T^{\prime}}
\newcommand{\Uprime}{U^{\prime}}
\newcommand{\Vprime}{V^{\prime}}
\newcommand{\Wprime}{W^{\prime}}
\newcommand{\Xprime}{X^{\prime}}
\newcommand{\Yprime}{Y^{\prime}}
\newcommand{\Zprime}{Z^{\prime}}
\newcommand{\alphaprime}{\alpha^{\prime}}
\newcommand{\betaprime}{\beta^{\prime}}
\newcommand{\gammaprime}{\gamma^{\prime}}
\newcommand{\Gammaprime}{\Gamma^{\prime}}
\newcommand{\deltaprime}{\delta^{\prime}}
\newcommand{\epsilonprime}{\epsilon^{\prime}}
\newcommand{\kappaprime}{\kappa^{\prime}}
\newcommand{\lambdaprime}{\lambda^{\prime}}
\newcommand{\rhoprime}{\rho^{\prime}}
\newcommand{\Sigmaprime}{\Sigma^{\prime}}
\newcommand{\tauprime}{\tau^{\prime}}
\newcommand{\xiprime}{\xi^{\prime}}
\newcommand{\thetaprime}{\theta^{\prime}}
\newcommand{\Omegaprime}{\Omega^{\prime}}
\newcommand{\cMprime}{\cM^{\prime}}
\newcommand{\cNprime}{\cN^{\prime}}
\newcommand{\cPprime}{\cP^{\prime}}
\newcommand{\cQprime}{\cQ^{\prime}}
\newcommand{\cRprime}{\cR^{\prime}}
\newcommand{\cSprime}{\cS^{\prime}}
\newcommand{\cTprime}{\cT^{\prime}}

%%%%%%%%%%%%%%%%%%%%%%%%%%%%%%%%%%%%%%%%%%%%%
%          bar Letters
%%%%%%%%%%%%%%%%%%%%%%%%%%%%%%%%%%%%%%%%%%%%%
\newcommand{\abar}{\bar{a}}
\newcommand{\bbar}{\bar{b}}
\newcommand{\cbar}{\bar{c}}
\newcommand{\ibar}{\bar{i}}
\newcommand{\jbar}{\bar{j}}
\newcommand{\nbar}{\bar{n}}
\newcommand{\xbar}{\bar{x}}
\newcommand{\ybar}{\bar{y}}
\newcommand{\zbar}{\bar{z}}
\newcommand{\pibar}{\bar{\pi}}
\newcommand{\phibar}{\bar{\varphi}}
\newcommand{\psibar}{\bar{\psi}}
\newcommand{\thetabar}{\bar{\theta}}
\newcommand{\nubar}{\bar{\nu}}

%%%%%%%%%%%%%%%%%%%%%%%%%%%%%%%%%%%%%%%%%%%%%
%          star Letters
%%%%%%%%%%%%%%%%%%%%%%%%%%%%%%%%%%%%%%%%%%%%%
\newcommand{\phistar}{\phi^{*}}
\newcommand{\Mstar}{M^{*}}

%%%%%%%%%%%%%%%%%%%%%%%%%%%%%%%%%%%%%%%%%%%%%
%          dotletters Letters
%%%%%%%%%%%%%%%%%%%%%%%%%%%%%%%%%%%%%%%%%%%%%
\newcommand{\Gdot}{\dot{G}}

%%%%%%%%%%%%%%%%%%%%%%%%%%%%%%%%%%%%%%%%%%%%%
%         check Letters
%%%%%%%%%%%%%%%%%%%%%%%%%%%%%%%%%%%%%%%%%%%%%
\newcommand{\deltacheck}{\check{\delta}}
\newcommand{\gammacheck}{\check{\gamma}}


%%%%%%%%%%%%%%%%%%%%%%%%%%%%%%%%%%%%%%%%%%%%%
%          Formulas
%%%%%%%%%%%%%%%%%%%%%%%%%%%%%%%%%%%%%%%%%%%%%

\newcommand{\formulaphi}{\text{\large $\varphi$}}
\newcommand{\Formulaphi}{\text{\Large $\varphi$}}


%%%%%%%%%%%%%%%%%%%%%%%%%%%%%%%%%%%%%%%%%%%%%
%          Fraktur Letters
%%%%%%%%%%%%%%%%%%%%%%%%%%%%%%%%%%%%%%%%%%%%%

\newcommand{\fa}{\mathfrak{a}}
\newcommand{\fb}{\mathfrak{b}}
\newcommand{\fc}{\mathfrak{c}}
\newcommand{\fk}{\mathfrak{k}}
\newcommand{\fp}{\mathfrak{p}}
\newcommand{\fq}{\mathfrak{q}}
\newcommand{\fr}{\mathfrak{r}}
\newcommand{\fA}{\mathfrak{A}}
\newcommand{\fB}{\mathfrak{B}}
\newcommand{\fC}{\mathfrak{C}}
\newcommand{\fD}{\mathfrak{D}}

%%%%%%%%%%%%%%%%%%%%%%%%%%%%%%%%%%%%%%%%%%%%%
%          Bold Letters
%%%%%%%%%%%%%%%%%%%%%%%%%%%%%%%%%%%%%%%%%%%%%
\newcommand{\ba}{\mathbf{a}}
\newcommand{\bb}{\mathbf{b}}
\newcommand{\bc}{\mathbf{c}}
\newcommand{\bd}{\mathbf{d}}
\newcommand{\be}{\mathbf{e}}
\newcommand{\bu}{\mathbf{u}}
\newcommand{\bv}{\mathbf{v}}
\newcommand{\bw}{\mathbf{w}}
\newcommand{\bx}{\mathbf{x}}
\newcommand{\by}{\mathbf{y}}
\newcommand{\bz}{\mathbf{z}}
\newcommand{\bSigma}{\boldsymbol{\Sigma}}
\newcommand{\bPi}{\boldsymbol{\Pi}}
\newcommand{\bDelta}{\boldsymbol{\Delta}}
\newcommand{\bdelta}{\boldsymbol{\delta}}
\newcommand{\bgamma}{\boldsymbol{\gamma}}
\newcommand{\bGamma}{\boldsymbol{\Gamma}}

%%%%%%%%%%%%%%%%%%%%%%%%%%%%%%%%%%%%%%%%%%%%%
%         Bold numbers
%%%%%%%%%%%%%%%%%%%%%%%%%%%%%%%%%%%%%%%%%%%%%
\newcommand{\bzero}{\mathbf{0}}

%%%%%%%%%%%%%%%%%%%%%%%%%%%%%%%%%%%%%%%%%%%%%
% Projective-Like Pointclasses
%%%%%%%%%%%%%%%%%%%%%%%%%%%%%%%%%%%%%%%%%%%%%
\newcommand{\Sa}[2][\alpha]{\Sigma_{(#1,#2)}}
\newcommand{\Pa}[2][\alpha]{\Pi_{(#1,#2)}}
\newcommand{\Da}[2][\alpha]{\Delta_{(#1,#2)}}
\newcommand{\Aa}[2][\alpha]{A_{(#1,#2)}}
\newcommand{\Ca}[2][\alpha]{C_{(#1,#2)}}
\newcommand{\Qa}[2][\alpha]{Q_{(#1,#2)}}
\newcommand{\da}[2][\alpha]{\delta_{(#1,#2)}}
\newcommand{\leqa}[2][\alpha]{\leq_{(#1,#2)}}
\newcommand{\lessa}[2][\alpha]{<_{(#1,#2)}}
\newcommand{\equiva}[2][\alpha]{\equiv_{(#1,#2)}}


\newcommand{\Sl}[1]{\Sa[\lambda]{#1}}
\newcommand{\Pl}[1]{\Pa[\lambda]{#1}}
\newcommand{\Dl}[1]{\Da[\lambda]{#1}}
\newcommand{\Al}[1]{\Aa[\lambda]{#1}}
\newcommand{\Cl}[1]{\Ca[\lambda]{#1}}
\newcommand{\Ql}[1]{\Qa[\lambda]{#1}}

\newcommand{\San}{\Sa{n}}
\newcommand{\Pan}{\Pa{n}}
\newcommand{\Dan}{\Da{n}}
\newcommand{\Can}{\Ca{n}}
\newcommand{\Qan}{\Qa{n}}
\newcommand{\Aan}{\Aa{n}}
\newcommand{\dan}{\da{n}}
\newcommand{\leqan}{\leqa{n}}
\newcommand{\lessan}{\lessa{n}}
\newcommand{\equivan}{\equiva{n}}

\newcommand{\SigmaOneOmega}{\Sigma^1_{\omega}}
\newcommand{\SigmaOneOmegaPlusOne}{\Sigma^1_{\omega+1}}
\newcommand{\PiOneOmega}{\Pi^1_{\omega}}
\newcommand{\PiOneOmegaPlusOne}{\Pi^1_{\omega+1}}
\newcommand{\DeltaOneOmegaPlusOne}{\Delta^1_{\omega+1}}

%%%%%%%%%%%%%%%%%%%%%%%%%%%%%%%%%%%%%%%%%%%%%
% Linear Algebra
%%%%%%%%%%%%%%%%%%%%%%%%%%%%%%%%%%%%%%%%%%%%%
\newcommand{\transpose}[1]{{#1}^{\text{T}}}
\newcommand{\norm}[1]{\lVert{#1}\rVert}
\newcommand\aug{\fboxsep=-\fboxrule\!\!\!\fbox{\strut}\!\!\!}

%%%%%%%%%%%%%%%%%%%%%%%%%%%%%%%%%%%%%%%%%%%%%
% Number Theory
%%%%%%%%%%%%%%%%%%%%%%%%%%%%%%%%%%%%%%%%%%%%%
\newcommand{\av}[1]{\lvert#1\rvert}
\DeclareMathOperator{\divides}{\mid}
\DeclareMathOperator{\ndivides}{\nmid}
\DeclareMathOperator{\lcm}{lcm}
\DeclareMathOperator{\sign}{sign}
\newcommand{\floor}[1]{\left\lfloor{#1}\right\rfloor}
\DeclareMathOperator{\ConCl}{CC}
\DeclareMathOperator{\ord}{ord}


%%%%%%%%%%%%%%%%%%%%%%%%%%%%%%%%%%%%%%%%%%%%%%%%%%%%%%%%%%%%%%%%%%%%%%%%%%%
%%  Theorem-Like Declarations
%%%%%%%%%%%%%%%%%%%%%%%%%%%%%%%%%%%%%%%%%%%%%%%%%%%%%%%%%%%%%%%%%%%%%%%%%%

\newtheorem{theorem}{Theorem}[section]
\newtheorem{lemma}[theorem]{Lemma}
\newtheorem{corollary}[theorem]{Corollary}
\newtheorem{proposition}[theorem]{Proposition}


\theoremstyle{definition}

\newtheorem{definition}[theorem]{Definition}
\newtheorem{conjecture}[theorem]{Conjecture}
\newtheorem{remark}[theorem]{Remark}
\newtheorem{remarks}[theorem]{Remarks}
\newtheorem{notation}[theorem]{Notation}

\theoremstyle{remark}

\newtheorem*{note}{Note}
\newtheorem*{warning}{Warning}
\newtheorem*{question}{Question}
\newtheorem*{example}{Example}
\newtheorem*{fact}{Fact}


\newenvironment*{subproof}[1][Proof]
{\begin{proof}[#1]}{\renewcommand{\qedsymbol}{$\diamondsuit$} \end{proof}}

\newenvironment*{case}[1]
{\textbf{Case #1.  }\itshape }{}

\newenvironment*{claim}[1][Claim]
{\textbf{#1.  }\itshape }{}


\pagestyle{plain}

\begin{document}

\title{Commutative Algebra Notes}
\author{Mitch Rudominer}

\maketitle

\tableofcontents

%%%%%%%%%%%%%%%%%%%%%%%%%%%%%%%%%%%%%%%%%%%%%%%%%%%%%%%%%%%%%%%%%%%%%%%%%%%%%%%%%%%%%%%%

\section{Rings and Ideals}


\subsection{Exercises}
These are exercises from Chapter 1 of \cite{AtiyahBook}.

\begin{exercise}[exercise 20]
(iv) If $A$ is a ring and $X=\Spec(A)$, then the irredecible components of $X$ are
the closed sets $V(\fp)$ where $\fp$ is a minimal prime ideal.
\end{exercise}
\begin{proof}
In any toplogical space the closure of a point is irreducible and $V(\fp)$ is the closure of $\singleton{\fp}$
so it is irreducible. An arbitrary closed set in $\Spec(A)$ is of the form $V(\fa)$ for
some ideal $\fa$ such that $\fa=r(\fa)$. If $\fa$ is not prime then there are two points
$a,b\notin\fa$ (so that $X_a\intersect V(\fa)\not=\emptyset$ and
$X_b\intersect V(\fa) \not=\emptyset$) such that $ab \in \fa$ (so that
$X_a\intersect X_b \intersect \fa = \emptyset$) so $V(\fa)$ is not irreducible.
In other words the only closed irreducible sets are of the form $V(\fp)$ for
$\fp$ prime.

In any topological space if a point $x$ is in the closure of a different point
$y$ then the closure of $\singleton{x}$ is not a maximal irreducible set so
if $\fp$ is not a minimal prime ideal then $V(\fp)$ is not an irreducible
component.

Finally suppose $\fp$ a prime ideal and $V(\fp)$ is not a maximal irreducible
set. Then by the discussion above there must be a prime ideal $\fq\subset\fp$
in order to yield a closed irreducible set $V(\fq)\supset V(\fp)$, and so
$\fp$ is not a minimal prime ideal.
\end{proof}

\begin{exercise}[exercise 21]Let $\phi:A\map B$ be a ring homomorhpishm. Let $X=\Spec(A)$
and $Y=\Spec(B)$. Notice that if $\fq\in Y$, then $\phi^{-1}[\fq]\in X$ so $\phi$ induces a mapping
$\phistar:Y\map X$.
\begin{enumerate}
\item[(iii)] If $\fV$ is any ideal of $B$ then $\overline{\phistar[V(\fV)]} = V(\fV^c)$.
\end{enumerate}
\end{exercise}
\begin{proof}
(iii) Let $\fp\in \phistar[V(\fV)]$. Then there is $\fq\in V(\fV)$ s.t.
$\fp=\phi^{-1}[\fq]$. So $\fp\in V(\fV^c)$. Since $V(\fV^c)$ is closed we have
$\overline{\phistar[V(\fV)]} \subseteq V(\fV^c)$.

Suppose towards a contradiction that the containment were proper.
Then there is an $E\subseteq A$
such that $\overline{\phistar[V(\fV)]} = V(E)$ and $r(\fV^c) \subsetneqq E$
and $E\subseteq\fp$ for all $\fp$ in $\phistar[V(\fV)]$.

Let $a\in E - r(\fV^c)$. Then $\phi(a)\in\fq$ for all $\fq\in V(\fV)$. But this means that
$\phi(a)\in r(\fV)$ and so $a\in r(\fV^c)$, contradiction.
\end{proof}

\begin{exercise}[exercise 22] Let $A = A_1 \otimes A_2$ be a direct product of rings.
Then $\Spec(A)$ is the disjoint union of two clopen sets $\Spec(A) = U_1 \union U_2$
where $U_1$ is canonically homeomorphic to $\Spec(A_1)$ and $U_2$ is canonically
homeomorphic to $\Spec(A_2)$.
\end{exercise}
\begin{proof}
Let $U_1=X_{(1,0)}$ and $U_2=X_{(0, 1)}$. These sets are disjoint because no prime ideal
can omit both $(0, 1)$ and $(1,0)$ because then it would omit their product $(0, 0)$.
The sets cover $\Spec(A)$ because no prime ideal can contain both $(0, 1)$ and
$(1, 0)$ because then it would contain their sum $(1, 1)$.

Let $\fp\in X_{(1, 0)}$.
Then $(0, 1) \in \fp$ which means that $\fp = \fq \times A_2$ for some $\fq \subseteq A_1$ and then
it is easy to see that in fact $\fq$ is a prime ideal, $\fq \in \Spec(A_1)$.
Conversely let $\fq \in \Spec(A_1)$. Then $\fq \times A_2$ is a prime ideal of $A$.
Thus there is a canonical bijection $\pi:\Spec(A_1) \map U_1$ given by
$\pi(\fq) = \fq \times A_2$.

To see that $\pi$ is a homeomorphism notice that if $a\in A_1$ then $\pi[X_a]=X_{(a, 0)}$ so
$\pi^{-1}$ is continuous. Conversely for any $(a, b)$, $\pi^{-1}[X_{(a,b)}] = X_a$, so $\pi$  is continuous.

Similarly $U_2$ is canonically homeomorphic to $\Spec(A_2)$.

As an aside we note that $X_{(a, b)} \subseteq U_1$ iff $b$ is nilpotent. pf. $b$ is nilpotent iff
for all $\fr\in\Spec(A_2)$, $(a,b)\in A_1 \times \fr$ iff $U_2 \intersect X_{(a,b)}=\emptyset$.
\end{proof}

%%%%%%%%%%%%%%%%%%%%%%%%%%%%%%%%%%%%%%%%%%%%%%%%%%%%%%%%%%%%%%%%%%%%%%%%%%%%%%%%%%%%%%%%

\section{Algebraic Geometry}


\subsection{Exercises}
These are exercises from Chapter 1 of \cite{Kunz_Book}.

\begin{exercise}[Section 1, exercise 5, page 9]
Let $L/K$ be a normal field extension. Two points $(x_1,\cdots,x_n)$ and $(y_1,\cdots,y_n)$ in
$\A^n(L)$ are called \emph{conjugate} over $K$ if there is a $K$-automorphism $\sigma$ of $L$
such that $(\sigma(x_1),\cdots,\sigma(x_n))=(y_1,\cdots,y_n)$.

(a) Any $K$-Variety $V\subset\A^n(L)$ that contains $x$ also contains all conjugates of $x$.

(b) If $V\subset\A^n(L)$ is a finite set of points with the property that if $V$ contains $x$
then it also contains all conjugates of $x$ over $K$, then $V$ is a $K$-variety.
\end{exercise}
\begin{proof}
(a) If $y$ is a conjugate of $x$ via the $K$-automorphism $\sigma$, and $f\in K[x_1,\cdots,x_n]$,
and $f(x)=0$, then $f(y)=0$, because
because the coefficients of $f$ are fixed by $\sigma$. $\qed$.

(b) Given a point $a\in\A^n(L)$, let $V_a$ be the set of all conjugates of $a$. It suffices to see that
for all $a$, $V_a$ is a $K$-variety. This is because every $V\subset\A^n(L)$ as in the statement of the
exercise is the union of $V_a$ for $a\in V$, and the union of finitely many varieties is a variety.

Since we don't yet know that $K[x_1,\cdots,x_n]$ is Noetherian, we need the following

\begin{claim}[Claim 1]
For all $a\in\A^n(L)$, $\fI(\singleton{a})$ is finitely-generated. 
\end{claim}

We will prove Claim 1 below. For now let's assume it.

Fix $a=(a_1,\cdots,a_n)\in\A^n(L)$. We will show that $V_a = \fV(\fI(\singleton{a}))$
and so $V_a$ is a $K$-variety.


By part (a), $V_a\subseteq \fV(\fI(\singleton{a}))$.  So it suffices to see that if
$b\in\fV(\fI(\singleton{a}))$, then $b$ is a conjugate of $a$.

For $b=(b_1,\cdots,b_n)\in\A^n(L)$, let $\pi_b:K[x_1,\cdots,x_n]\map K(b_1,\cdots,b_n)$ be given by $\pi_b(f(x_1,\cdots x_n)) = f(b_1,\cdots,b_n)$.
Then $\pi_b$ is a surjective homomorphism. Let $I_b=\ker(\pi_b)$. Then $I_b$ is a maximal ideal,
since $K(b_1,\cdots,b_n)$ is a field. Clearly $I_b=\fI(\singleton{b})$.
Let 
$\bar{\pi_b}:(K[x_1,\cdots,x_n]/I_b)\map K(b_1,\cdots,b_n)$ be the induced isomorphism. 
For $f\in K[x_1,\cdots,x_n]$, $\bar{\pi_b}([f]_{I_b}) = f(b_1,\cdots,b_n)$, where we have written
$[f]_{I_b}$ to mean the coset of $f$ in $K[x_1,\cdots,x_n]/I_b$.

\begin{claim}[Claim 2]
If $b\in\fV(\fI(\singleton{a}))$, then $I_b=I_a$.
\end{claim}
\begin{subproof}
Let $f\in I_a = \fI(\singleton{a})$. Then $f(b)=0$ so $f\in\fI(\singleton{b})=I_b$.
So $I_a\subseteq I_b$. Since $I_a$ and $I_b$ are both maximal ideals, $I_a=I_b$.
\end{subproof}

\begin{claim}[Claim 3]
If $b\in\fV(\fI(\singleton{a}))$, then $b$ is a conjugate of $a$.
\end{claim}
\begin{subproof}
Let $b=(b_1,\cdots,b_n)\in\fV(\fI(\singleton{a}))$. 
By Claim 2 and the paragraph preceding it, $\bar{\pi}_b\circ (\bar{\pi}_a)^{-1}$ is an isomorphism of $K(a_1,\cdots,a_n)$
onto $K(b_1,\cdots,b_n)$ that fixes $K$ and such that 
$\bar{\pi}_b\circ (\bar{\pi}_a)^{-1}(f(a_1,\cdots,a_n))=\bar{\pi}_b([f]_{I_a})=\bar{\pi}_b([f]_{I_b}) =f(b_1,\cdots,b_n)$ for all $f\in K[x_1,\cdots,x_n]$.
Note that $\bar{\pi}_b\circ (\bar{\pi}_a)^{-1}(a_i)=b_i$ for $i=1,\cdots, n$.

By Theorem 2.8 (section 2 of Chapter VII) on page 275 of \cite{Lang_Book}, there is an extension $\sigma$
of $\bar{\pi}_b\circ (\bar{\pi}_a)^{-1}$ to an embedding of $L$ into the algebraic closure of $L$. Since
$L/K$ is normal, $\sigma\restr L$ is an automorphism.
\end{subproof}

Finally we must prove Claim 1. Note that this would not be necessary if we had already proved the Hilbert Basis Theorem, but this exercise occurs in the textbook prior to that proof. The proof of Claim 1 is of independent interest.

\begin{subproof}[Proof of Claim 1]
Fix $a=(a_1,\cdots,a_n)\in\A^n(L)$. We will find $f_1,\cdots,f_n\in K[x_1,\cdots,x_n]$ such that
$\fI(\singleton{a}) = (f_1,\cdots,f_n)$.

Since $L/K$ is assumed to be normal, by definition each of the $a_i$ are algebraic over $K$. 


Let $f_1 \in K[x_1]$ be the minimal polynomial of $a_1$ over $K$. Let $f_2\in K[x_1,x_2]$ be such that
$\fbar_2(x_2)=f_2(a_1,x_2)$ is the minimal polynomial of $a_2$ over $K(a_1)$. In general, for $i=2,\cdots,n$,
let $f_i\in K[x_1,\cdots, x_i]$ be such that $\fbar_i(x_i)=f_i(a_1,\cdots,a_{i-1},x_i)$ is the minimal polynomial of $a_i$
over $K(a_1,\cdots,a_{i-1})$. 

We claim that $\fI(\singleton{a}) = (f_1,\cdots,f_n)$. We prove this by induction on $n$. For $n=1$, it is
standard that $K[x_1]$ is a PID and $\fI(\singleton{a_1}) = (f_1)$.
Suppose it is true for $n$ and we will show it is true for $n+1$.
Let $\pi:K[x_1,\cdots,x_n]\map K(a_1,\cdots, a_n)$ be given by $\pi(f(x_1,\cdots,x_n)) = f(a_1,\cdots,a_n)$.
By induction, $\ker(\pi) = (f_1,\cdots,f_n)$. 

We extend $\pi$ in the natural way to 
$\pi^*:K[x_1,\cdots,x_n,x_{n+1}]\map K(a_1,\cdots, a_n)[x_{n+1}].$ Notice that  $\ker(\pi^*) = (f_1,\cdots,f_n)$ but now thinking of $(f_1,\cdots,f_n)$ as an
ideal in $K[x_1,\cdots,x_n,x_{n+1}]$. To see this notice first that for $i=1,\cdots,n$,
 $f_i\in\ker(\pi^*)$ since $f_i\in\ker(\pi)$. Conversely, suppose $f\in\ker(\pi^*)$.
If we think of $f$ as a polynomial in $x_{n+1}$
with coefficients in $K[x_1,\cdots,x_n]$, then each of the coefficients of $f$ is
in $\ker(\pi)$ so each of the coefficients of $f$ are in $(f_1,\cdots,f_n)$, so $f$ is too.


Let $\rho : K(a_1,\cdots,a_n)[x_{n+1}] \map K(a_1,\cdots,a_n,a_{n+1})$ be given by 
$\rho(f(a_1,\cdots,a_n,x_{n+1})) = f(a_1,\cdots,a_n,a_{n+1})$. Then
$\ker(\rho) = (\fbar_{n+1})$ where $\fbar_{n+1}=\pi^*(f_{n+1})$ is the minimal polynomial
of $a_{n+1}$ over $K(a_1,\cdots,a_n)$. So $\ker(\rho\circ\pi^*)$ is the pullback
under $\pi^*$ of $(\fbar_{n+1})$ which is $(f_1,\cdots,f_n,f_{n+1})$.
\end{subproof}



\end{proof}

%%%%%%%%%%%%%%%%%%%%%%%%%%%%%%%%%%%%%%%%%%%%%%%%%%%%%%%%%%%%%%%%%%%%%%%%%%%%%%%%%%%%%%%%

\bibliographystyle{amsalpha}

\bibliography{math}

\end{document}
