\documentclass[oneside,12pt]{amsart}

\usepackage{amsmath,amssymb,latexsym,eucal,amsthm}

%%%%%%%%%%%%%%%%%%%%%%%%%%%%%%%%%%%%%%%%%%%%%
% Common Set Theory Constructs
%%%%%%%%%%%%%%%%%%%%%%%%%%%%%%%%%%%%%%%%%%%%%

\newcommand{\setof}[2]{\left\{ \, #1 \, \left| \, #2 \, \right.\right\}}
\newcommand{\lsetof}[2]{\left\{\left. \, #1 \, \right| \, #2 \,  \right\}}
\newcommand{\bigsetof}[2]{\bigl\{ \, #1 \, \bigm | \, #2 \,\bigr\}}
\newcommand{\Bigsetof}[2]{\Bigl\{ \, #1 \, \Bigm | \, #2 \,\Bigr\}}
\newcommand{\biggsetof}[2]{\biggl\{ \, #1 \, \biggm | \, #2 \,\biggr\}}
\newcommand{\Biggsetof}[2]{\Biggl\{ \, #1 \, \Biggm | \, #2 \,\Biggr\}}
\newcommand{\dotsetof}[2]{\left\{ \, #1 \, : \, #2 \, \right\}}
\newcommand{\bigdotsetof}[2]{\bigl\{ \, #1 \, : \, #2 \,\bigr\}}
\newcommand{\Bigdotsetof}[2]{\Bigl\{ \, #1 \, \Bigm : \, #2 \,\Bigr\}}
\newcommand{\biggdotsetof}[2]{\biggl\{ \, #1 \, \biggm : \, #2 \,\biggr\}}
\newcommand{\Biggdotsetof}[2]{\Biggl\{ \, #1 \, \Biggm : \, #2 \,\Biggr\}}
\newcommand{\sequence}[2]{\left\langle \, #1 \,\left| \, #2 \, \right. \right\rangle}
\newcommand{\lsequence}[2]{\left\langle\left. \, #1 \, \right| \,#2 \,  \right\rangle}
\newcommand{\bigsequence}[2]{\bigl\langle \,#1 \, \bigm | \, #2 \, \bigr\rangle}
\newcommand{\Bigsequence}[2]{\Bigl\langle \,#1 \, \Bigm | \, #2 \, \Bigr\rangle}
\newcommand{\biggsequence}[2]{\biggl\langle \,#1 \, \biggm | \, #2 \, \biggr\rangle}
\newcommand{\Biggsequence}[2]{\Biggl\langle \,#1 \, \Biggm | \, #2 \, \Biggr\rangle}
\newcommand{\singleton}[1]{\left\{#1\right\}}
\newcommand{\angles}[1]{\left\langle #1 \right\rangle}
\newcommand{\bigangles}[1]{\bigl\langle #1 \bigr\rangle}
\newcommand{\Bigangles}[1]{\Bigl\langle #1 \Bigr\rangle}
\newcommand{\biggangles}[1]{\biggl\langle #1 \biggr\rangle}
\newcommand{\Biggangles}[1]{\Biggl\langle #1 \Biggr\rangle}


\newcommand{\force}[1]{\Vert\!\underset{\!\!\!\!\!#1}{\!\!\!\relbar\!\!\!%
\relbar\!\!\relbar\!\!\relbar\!\!\!\relbar\!\!\relbar\!\!\relbar\!\!\!%
\relbar\!\!\relbar\!\!\relbar}}
\newcommand{\nforce}[1]{\Vert\!\underset{\!\!\!\!\!#1}{\!\!\!\relbar\!\!\!%
\relbar\!\!\relbar\!\!\relbar\!\!\!\relbar\!\!\relbar\!\!\relbar\!\!\!%
\relbar\!\!\not\relbar\!\!\relbar}}
\newcommand{\forcein}[2]{\overset{#2}{\Vert\underset{\!\!\!\!\!#1}%
{\!\!\!\relbar\!\!\!\relbar\!\!\relbar\!\!\relbar\!\!\!\relbar\!\!\relbar\!%
\!\relbar\!\!\!\relbar\!\!\relbar\!\!\relbar\!\!\relbar\!\!\!\relbar\!\!%
\relbar\!\!\relbar}}}

\newcommand{\pre}[2]{{}^{#2}\!{#1}}

\newcommand{\restr}{\!\!\upharpoonright\!}

%%%%%%%%%%%%%%%%%%%%%%%%%%%%%%%%%%%%%%%%%%%%%
% Set-Theoretic Connectives
%%%%%%%%%%%%%%%%%%%%%%%%%%%%%%%%%%%%%%%%%%%%%

\newcommand{\intersect}{\cap}
\newcommand{\union}{\cup}
\newcommand{\Intersection}[1]{\bigcap\limits_{#1}}
\newcommand{\Union}[1]{\bigcup\limits_{#1}}
\newcommand{\adjoin}{{}^\frown}
\newcommand{\forces}{\Vdash}

%%%%%%%%%%%%%%%%%%%%%%%%%%%%%%%%%%%%%%%%%%%%%
% Miscellaneous
%%%%%%%%%%%%%%%%%%%%%%%%%%%%%%%%%%%%%%%%%%%%%
\newcommand{\defeq}{=_{\text{def}}}
\newcommand{\Turingleq}{\leq_{\text{T}}}
\newcommand{\Turingless}{<_{\text{T}}}
\newcommand{\lexleq}{\leq_{\text{lex}}}
\newcommand{\lexless}{<_{\text{lex}}}
\newcommand{\Turingequiv}{\equiv_{\text{T}}}

%%%%%%%%%%%%%%%%%%%%%%%%%%%%%%%%%%%%%%%%%%%%%
% Constants
%%%%%%%%%%%%%%%%%%%%%%%%%%%%%%%%%%%%%%%%%%%%%
\newcommand{\R}{\mathbb{R}}
\renewcommand{\P}{\mathbb{P}}
\newcommand{\Q}{\mathbb{Q}}
\newcommand{\Z}{\mathbb{Z}}
\newcommand{\C}{\mathbb{C}}
\newcommand{\N}{\mathbb{N}}
\newcommand{\B}{\mathbb{B}}
\newcommand{\LofR}{L(\R)}
\newcommand{\JofR}[1]{J_{#1}(\R)}
\newcommand{\SofR}[1]{S_{#1}(\R)}
\newcommand{\JalphaR}{\JofR{\alpha}}
\newcommand{\JbetaR}{\JofR{\beta}}
\newcommand{\JlambdaR}{\JofR{\lambda}}
\newcommand{\SalphaR}{\SofR{\alpha}}
\newcommand{\SbetaR}{\SofR{\beta}}
\newcommand{\Pkl}{\mathcal{P}_{\kappa}(\lambda)}
\DeclareMathOperator{\con}{con}
\DeclareMathOperator{\ORD}{OR}
\DeclareMathOperator{\Ord}{OR}
\DeclareMathOperator{\WO}{WO}
\DeclareMathOperator{\OD}{OD}
\DeclareMathOperator{\HOD}{HOD}
\DeclareMathOperator{\HC}{HC}
\DeclareMathOperator{\WF}{WF}
\DeclareMathOperator{\HF}{HF}
\newcommand{\One}{I}
\newcommand{\ONE}{I}
\newcommand{\Two}{II}
\newcommand{\TWO}{II}

%%%%%%%%%%%%%%%%%%%%%%%%%%%%%%%%%%%%%%%%%%%%%
% Commutative Algebra Constants
%%%%%%%%%%%%%%%%%%%%%%%%%%%%%%%%%%%%%%%%%%%%%
\DeclareMathOperator{\dottimes}{\dot{\times}}

%%%%%%%%%%%%%%%%%%%%%%%%%%%%%%%%%%%%%%%%%%%%%
% Theories
%%%%%%%%%%%%%%%%%%%%%%%%%%%%%%%%%%%%%%%%%%%%%
\DeclareMathOperator{\ZFC}{ZFC}
\DeclareMathOperator{\ZF}{ZF}
\DeclareMathOperator{\AD}{AD}
\DeclareMathOperator{\ADR}{AD_{\R}}
\DeclareMathOperator{\KP}{KP}
\DeclareMathOperator{\PD}{PD}
\DeclareMathOperator{\CH}{CH}
\DeclareMathOperator{\HPC}{HPC} % HOD pair capturing
%%%%%%%%%%%%%%%%%%%%%%%%%%%%%%%%%%%%%%%%%%%%%
% Iteration Trees
%%%%%%%%%%%%%%%%%%%%%%%%%%%%%%%%%%%%%%%%%%%%%

\newcommand{\pred}{\text{-pred}}

%%%%%%%%%%%%%%%%%%%%%%%%%%%%%%%%%%%%%%%%%%%%%%%%
% Operator Names
%%%%%%%%%%%%%%%%%%%%%%%%%%%%%%%%%%%%%%%%%%%%%%%%
\DeclareMathOperator{\Det}{Det}
\DeclareMathOperator{\dom}{dom}
\DeclareMathOperator{\ran}{ran}
\DeclareMathOperator{\range}{ran}
\DeclareMathOperator{\image}{image}
\DeclareMathOperator{\crit}{crit}
\DeclareMathOperator{\card}{card}
\DeclareMathOperator{\cf}{cf}
\DeclareMathOperator{\cof}{cof}
\DeclareMathOperator{\rank}{rank}
\DeclareMathOperator{\ot}{o.t.}
\DeclareMathOperator{\ords}{o}
\DeclareMathOperator{\ro}{r.o.}
\DeclareMathOperator{\rud}{rud}
\DeclareMathOperator{\Powerset}{\mathcal{P}}
\DeclareMathOperator{\length}{lh}
\DeclareMathOperator{\lh}{lh}
\DeclareMathOperator{\limit}{lim}
\DeclareMathOperator{\fld}{fld}
\DeclareMathOperator{\projection}{p}
\DeclareMathOperator{\Ult}{Ult}
\DeclareMathOperator{\ULT}{Ult}
\DeclareMathOperator{\Coll}{Coll}
\DeclareMathOperator{\Col}{Col}
\DeclareMathOperator{\Hull}{Hull}
\DeclareMathOperator*{\dirlim}{dir lim}
\DeclareMathOperator{\Scale}{Scale}
\DeclareMathOperator{\supp}{supp}
\DeclareMathOperator{\trancl}{tran.cl.}
\DeclareMathOperator{\trace}{Tr}
\DeclareMathOperator{\diag}{diag}
\DeclareMathOperator{\spn}{span}
\DeclareMathOperator{\sgn}{sgn}
%%%%%%%%%%%%%%%%%%%%%%%%%%%%%%%%%%%%%%%%%%%%%
% Logical Connectives
%%%%%%%%%%%%%%%%%%%%%%%%%%%%%%%%%%%%%%%%%%%%%
\newcommand{\IImplies}{\Longrightarrow}
\newcommand{\SkipImplies}{\quad\Longrightarrow\quad}
\newcommand{\Ifff}{\Longleftrightarrow}
\newcommand{\iimplies}{\longrightarrow}
\newcommand{\ifff}{\longleftrightarrow}
\newcommand{\Implies}{\Rightarrow}
\newcommand{\Iff}{\Leftrightarrow}
\renewcommand{\implies}{\rightarrow}
\renewcommand{\iff}{\leftrightarrow}
\newcommand{\AND}{\wedge}
\newcommand{\OR}{\vee}
\newcommand{\st}{\text{ s.t. }}
\newcommand{\Or}{\text{ or }}

%%%%%%%%%%%%%%%%%%%%%%%%%%%%%%%%%%%%%%%%%%%%%
% Function Arrows
%%%%%%%%%%%%%%%%%%%%%%%%%%%%%%%%%%%%%%%%%%%%%

\newcommand{\injection}{\xrightarrow{\text{1-1}}}
\newcommand{\surjection}{\xrightarrow{\text{onto}}}
\newcommand{\bijection}{\xrightarrow[\text{onto}]{\text{1-1}}}
\newcommand{\cofmap}{\xrightarrow{\text{cof}}}
\newcommand{\map}{\rightarrow}

%%%%%%%%%%%%%%%%%%%%%%%%%%%%%%%%%%%%%%%%%%%%%
% Mouse Comparison Operators
%%%%%%%%%%%%%%%%%%%%%%%%%%%%%%%%%%%%%%%%%%%%%
\newcommand{\initseg}{\trianglelefteq}
\newcommand{\properseg}{\lhd}
\newcommand{\notinitseg}{\ntrianglelefteq}
\newcommand{\notproperseg}{\ntriangleleft}

%%%%%%%%%%%%%%%%%%%%%%%%%%%%%%%%%%%%%%%%%%%%%
%           calligraphic letters
%%%%%%%%%%%%%%%%%%%%%%%%%%%%%%%%%%%%%%%%%%%%%
\newcommand{\cA}{\mathcal{A}}
\newcommand{\cB}{\mathcal{B}}
\newcommand{\cC}{\mathcal{C}}
\newcommand{\cD}{\mathcal{D}}
\newcommand{\cE}{\mathcal{E}}
\newcommand{\cF}{\mathcal{F}}
\newcommand{\cG}{\mathcal{G}}
\newcommand{\cH}{\mathcal{H}}
\newcommand{\cI}{\mathcal{I}}
\newcommand{\cJ}{\mathcal{J}}
\newcommand{\cK}{\mathcal{K}}
\newcommand{\cL}{\mathcal{L}}
\newcommand{\cM}{\mathcal{M}}
\newcommand{\cN}{\mathcal{N}}
\newcommand{\cO}{\mathcal{O}}
\newcommand{\cP}{\mathcal{P}}
\newcommand{\cQ}{\mathcal{Q}}
\newcommand{\cR}{\mathcal{R}}
\newcommand{\cS}{\mathcal{S}}
\newcommand{\cT}{\mathcal{T}}
\newcommand{\cU}{\mathcal{U}}
\newcommand{\cV}{\mathcal{V}}
\newcommand{\cW}{\mathcal{W}}
\newcommand{\cX}{\mathcal{X}}
\newcommand{\cY}{\mathcal{Y}}
\newcommand{\cZ}{\mathcal{Z}}


%%%%%%%%%%%%%%%%%%%%%%%%%%%%%%%%%%%%%%%%%%%%%
%          Primed Letters
%%%%%%%%%%%%%%%%%%%%%%%%%%%%%%%%%%%%%%%%%%%%%
\newcommand{\aprime}{a^{\prime}}
\newcommand{\bprime}{b^{\prime}}
\newcommand{\cprime}{c^{\prime}}
\newcommand{\dprime}{d^{\prime}}
\newcommand{\eprime}{e^{\prime}}
\newcommand{\fprime}{f^{\prime}}
\newcommand{\gprime}{g^{\prime}}
\newcommand{\hprime}{h^{\prime}}
\newcommand{\iprime}{i^{\prime}}
\newcommand{\jprime}{j^{\prime}}
\newcommand{\kprime}{k^{\prime}}
\newcommand{\lprime}{l^{\prime}}
\newcommand{\mprime}{m^{\prime}}
\newcommand{\nprime}{n^{\prime}}
\newcommand{\oprime}{o^{\prime}}
\newcommand{\pprime}{p^{\prime}}
\newcommand{\qprime}{q^{\prime}}
\newcommand{\rprime}{r^{\prime}}
\newcommand{\sprime}{s^{\prime}}
\newcommand{\tprime}{t^{\prime}}
\newcommand{\uprime}{u^{\prime}}
\newcommand{\vprime}{v^{\prime}}
\newcommand{\wprime}{w^{\prime}}
\newcommand{\xprime}{x^{\prime}}
\newcommand{\yprime}{y^{\prime}}
\newcommand{\zprime}{z^{\prime}}
\newcommand{\Aprime}{A^{\prime}}
\newcommand{\Bprime}{B^{\prime}}
\newcommand{\Cprime}{C^{\prime}}
\newcommand{\Dprime}{D^{\prime}}
\newcommand{\Eprime}{E^{\prime}}
\newcommand{\Fprime}{F^{\prime}}
\newcommand{\Gprime}{G^{\prime}}
\newcommand{\Hprime}{H^{\prime}}
\newcommand{\Iprime}{I^{\prime}}
\newcommand{\Jprime}{J^{\prime}}
\newcommand{\Kprime}{K^{\prime}}
\newcommand{\Lprime}{L^{\prime}}
\newcommand{\Mprime}{M^{\prime}}
\newcommand{\Nprime}{N^{\prime}}
\newcommand{\Oprime}{O^{\prime}}
\newcommand{\Pprime}{P^{\prime}}
\newcommand{\Qprime}{Q^{\prime}}
\newcommand{\Rprime}{R^{\prime}}
\newcommand{\Sprime}{S^{\prime}}
\newcommand{\Tprime}{T^{\prime}}
\newcommand{\Uprime}{U^{\prime}}
\newcommand{\Vprime}{V^{\prime}}
\newcommand{\Wprime}{W^{\prime}}
\newcommand{\Xprime}{X^{\prime}}
\newcommand{\Yprime}{Y^{\prime}}
\newcommand{\Zprime}{Z^{\prime}}
\newcommand{\alphaprime}{\alpha^{\prime}}
\newcommand{\betaprime}{\beta^{\prime}}
\newcommand{\gammaprime}{\gamma^{\prime}}
\newcommand{\Gammaprime}{\Gamma^{\prime}}
\newcommand{\deltaprime}{\delta^{\prime}}
\newcommand{\epsilonprime}{\epsilon^{\prime}}
\newcommand{\kappaprime}{\kappa^{\prime}}
\newcommand{\lambdaprime}{\lambda^{\prime}}
\newcommand{\rhoprime}{\rho^{\prime}}
\newcommand{\Sigmaprime}{\Sigma^{\prime}}
\newcommand{\tauprime}{\tau^{\prime}}
\newcommand{\xiprime}{\xi^{\prime}}
\newcommand{\thetaprime}{\theta^{\prime}}
\newcommand{\Omegaprime}{\Omega^{\prime}}
\newcommand{\cMprime}{\cM^{\prime}}
\newcommand{\cNprime}{\cN^{\prime}}
\newcommand{\cPprime}{\cP^{\prime}}
\newcommand{\cQprime}{\cQ^{\prime}}
\newcommand{\cRprime}{\cR^{\prime}}
\newcommand{\cSprime}{\cS^{\prime}}
\newcommand{\cTprime}{\cT^{\prime}}

%%%%%%%%%%%%%%%%%%%%%%%%%%%%%%%%%%%%%%%%%%%%%
%          bar Letters
%%%%%%%%%%%%%%%%%%%%%%%%%%%%%%%%%%%%%%%%%%%%%
\newcommand{\abar}{\bar{a}}
\newcommand{\bbar}{\bar{b}}
\newcommand{\zbar}{\bar{z}}
\newcommand{\phibar}{\bar{\varphi}}
\newcommand{\psibar}{\bar{\psi}}
\newcommand{\thetabar}{\bar{\theta}}
\newcommand{\nubar}{\bar{\nu}}

%%%%%%%%%%%%%%%%%%%%%%%%%%%%%%%%%%%%%%%%%%%%%
%          star Letters
%%%%%%%%%%%%%%%%%%%%%%%%%%%%%%%%%%%%%%%%%%%%%
\newcommand{\phistar}{\phi^{*}}


%%%%%%%%%%%%%%%%%%%%%%%%%%%%%%%%%%%%%%%%%%%%%
%          Formulas
%%%%%%%%%%%%%%%%%%%%%%%%%%%%%%%%%%%%%%%%%%%%%

\newcommand{\formulaphi}{\text{\large $\varphi$}}
\newcommand{\Formulaphi}{\text{\Large $\varphi$}}


%%%%%%%%%%%%%%%%%%%%%%%%%%%%%%%%%%%%%%%%%%%%%
%          Fraktur Letters
%%%%%%%%%%%%%%%%%%%%%%%%%%%%%%%%%%%%%%%%%%%%%

\newcommand{\fa}{\mathfrak{a}}
\newcommand{\fb}{\mathfrak{b}}
\newcommand{\fc}{\mathfrak{c}}
\newcommand{\fk}{\mathfrak{k}}
\newcommand{\fp}{\mathfrak{p}}
\newcommand{\fq}{\mathfrak{q}}
\newcommand{\fr}{\mathfrak{r}}
\newcommand{\fA}{\mathfrak{A}}
\newcommand{\fB}{\mathfrak{B}}
\newcommand{\fC}{\mathfrak{C}}
\newcommand{\fD}{\mathfrak{D}}

%%%%%%%%%%%%%%%%%%%%%%%%%%%%%%%%%%%%%%%%%%%%%
%          Bold Letters
%%%%%%%%%%%%%%%%%%%%%%%%%%%%%%%%%%%%%%%%%%%%%
\newcommand{\ba}{\mathbf{a}}
\newcommand{\bb}{\mathbf{b}}
\newcommand{\bc}{\mathbf{c}}
\newcommand{\bd}{\mathbf{d}}
\newcommand{\be}{\mathbf{e}}
\newcommand{\bu}{\mathbf{u}}
\newcommand{\bv}{\mathbf{v}}
\newcommand{\bw}{\mathbf{w}}
\newcommand{\bx}{\mathbf{x}}
\newcommand{\by}{\mathbf{y}}
\newcommand{\bz}{\mathbf{z}}
\newcommand{\bSigma}{\boldsymbol{\Sigma}}
\newcommand{\bPi}{\boldsymbol{\Pi}}
\newcommand{\bDelta}{\boldsymbol{\Delta}}
\newcommand{\bdelta}{\boldsymbol{\delta}}
\newcommand{\bgamma}{\boldsymbol{\gamma}}
\newcommand{\bGamma}{\boldsymbol{\Gamma}}

%%%%%%%%%%%%%%%%%%%%%%%%%%%%%%%%%%%%%%%%%%%%%
%         Bold numbers
%%%%%%%%%%%%%%%%%%%%%%%%%%%%%%%%%%%%%%%%%%%%%
\newcommand{\bzero}{\mathbf{0}}

%%%%%%%%%%%%%%%%%%%%%%%%%%%%%%%%%%%%%%%%%%%%%
% Projective-Like Pointclasses
%%%%%%%%%%%%%%%%%%%%%%%%%%%%%%%%%%%%%%%%%%%%%
\newcommand{\Sa}[2][\alpha]{\Sigma_{(#1,#2)}}
\newcommand{\Pa}[2][\alpha]{\Pi_{(#1,#2)}}
\newcommand{\Da}[2][\alpha]{\Delta_{(#1,#2)}}
\newcommand{\Aa}[2][\alpha]{A_{(#1,#2)}}
\newcommand{\Ca}[2][\alpha]{C_{(#1,#2)}}
\newcommand{\Qa}[2][\alpha]{Q_{(#1,#2)}}
\newcommand{\da}[2][\alpha]{\delta_{(#1,#2)}}
\newcommand{\leqa}[2][\alpha]{\leq_{(#1,#2)}}
\newcommand{\lessa}[2][\alpha]{<_{(#1,#2)}}
\newcommand{\equiva}[2][\alpha]{\equiv_{(#1,#2)}}


\newcommand{\Sl}[1]{\Sa[\lambda]{#1}}
\newcommand{\Pl}[1]{\Pa[\lambda]{#1}}
\newcommand{\Dl}[1]{\Da[\lambda]{#1}}
\newcommand{\Al}[1]{\Aa[\lambda]{#1}}
\newcommand{\Cl}[1]{\Ca[\lambda]{#1}}
\newcommand{\Ql}[1]{\Qa[\lambda]{#1}}

\newcommand{\San}{\Sa{n}}
\newcommand{\Pan}{\Pa{n}}
\newcommand{\Dan}{\Da{n}}
\newcommand{\Can}{\Ca{n}}
\newcommand{\Qan}{\Qa{n}}
\newcommand{\Aan}{\Aa{n}}
\newcommand{\dan}{\da{n}}
\newcommand{\leqan}{\leqa{n}}
\newcommand{\lessan}{\lessa{n}}
\newcommand{\equivan}{\equiva{n}}

%%%%%%%%%%%%%%%%%%%%%%%%%%%%%%%%%%%%%%%%%%%%%
% Linear Algebra
%%%%%%%%%%%%%%%%%%%%%%%%%%%%%%%%%%%%%%%%%%%%%
\newcommand{\transpose}[1]{{#1}^{\text{T}}}
\newcommand{\norm}[1]{\lVert{#1}\rVert}
\newcommand\aug{\fboxsep=-\fboxrule\!\!\!\fbox{\strut}\!\!\!}

%%%%%%%%%%%%%%%%%%%%%%%%%%%%%%%%%%%%%%%%%%%%%
% Number Theory
%%%%%%%%%%%%%%%%%%%%%%%%%%%%%%%%%%%%%%%%%%%%%
\DeclareMathOperator{\Spec}{Spec}
\newcommand{\av}[1]{\lvert#1\rvert}
\DeclareMathOperator{\divides}{\mid}
\DeclareMathOperator{\ndivides}{\nmid}

%%%%%%%%%%%%%%%%%%%%%%%%%%%%%%%%%%%%%%%%%%%%%%%%%%%%%%%%%%%%%%%%%%%%%%%%%%%
%%  Theorem-Like Declarations
%%%%%%%%%%%%%%%%%%%%%%%%%%%%%%%%%%%%%%%%%%%%%%%%%%%%%%%%%%%%%%%%%%%%%%%%%%

\newtheorem{theorem}{Theorem}[section]
\newtheorem{lemma}[theorem]{Lemma}
\newtheorem{corollary}[theorem]{Corollary}
\newtheorem{proposition}[theorem]{Proposition}


\theoremstyle{definition}

\newtheorem{definition}[theorem]{Definition}
\newtheorem{conjecture}[theorem]{Conjecture}
\newtheorem{remark}[theorem]{Remark}
\newtheorem{remarks}[theorem]{Remarks}
\newtheorem{notation}[theorem]{Notation}

\newtheorem{homework}[theorem]{Exercise}
\newtheorem{numbered_example}[theorem]{Example}

\theoremstyle{remark}

\newtheorem*{aside}{Aside}
\newtheorem*{note}{Note}
\newtheorem*{observation}{Observation}
\newtheorem*{warning}{Warning}
\newtheorem*{question}{Question}
\newtheorem*{example}{Example}
\newtheorem*{in_class_example}{In-Class Example}
\newtheorem*{exercise}{Exercise}
\newtheorem*{fact}{Fact}


\newenvironment*{subproof}[1][Proof]
{\begin{proof}[#1]}{\renewcommand{\qedsymbol}{$\diamondsuit$} \end{proof}}

\newenvironment*{case}[1]
{\textbf{Case #1.  }\itshape }{}

\newenvironment*{claim}[1][Claim]
{\textbf{#1.  }\itshape }{}


\pagestyle{plain}

\begin{document}

\title{Set Theory Notes}

\maketitle

\tableofcontents

\section{Elementary Embeddings}

Throughout this section let $j:M \map N$ be an elementary embedding with $\crit(j)=\kappa$, where $M$ and $N$ are transitive models of a sufficient amount of $\ZFC$.

\begin{lemma}
\label{inductiveLemma}
If $j(A) = A$ and $j\restr A = id$ then $j\restr \Powerset(A) = id$.
\end{lemma}
\begin{proof}
Let $S\subseteq A$. Then $j(S) \subseteq j(A) = A$ and for all $x\in A$,
$x \in S$ iff $j(x) \in j(S)$ iff $x\in j(S)$. So $S = j(S)$.
\end{proof}


\begin{corollary}
\label{subsetLemma}
$j(A) = A$ whenever $A\subseteq \nu < \kappa$.
\end{corollary}
\begin{proof}
This follows from Lemma \ref{inductiveLemma} since $j(\nu) = \nu$ and
$j \restr \nu = id$.
\end{proof}

\begin{lemma}
\label{crossProductLemma}
$j(A) = A$ whenever $A\subseteq \nu \times \nu $ and $\nu < \kappa$.
\end{lemma}
\begin{proof}
This follows from Lemma \ref{inductiveLemma} since $j(\nu \times \nu) = \nu \times \nu$ and
$j \restr \nu \times \nu = id$.
\end{proof}

\begin{lemma}
$j\restr H^{M}_{\kappa} = id$ and $H^{M}_{\kappa}\subseteq H^{N}_{\kappa}$.
\end{lemma}
\begin{proof}
Let $f:\nu\bijection A$, $f\in M$, $A$ transitive, $\nu<\kappa$. Let
$B=\setof{\angles{\alpha, \beta}}{f(\alpha) \in f(\beta)}$. Then $j(B) = B$
and $j(f):\nu\bijection j(A)$ so $j(A)$ is a transitive set isomorphic to $A$
so $j(A)=A$.
\end{proof}

\begin{lemma}
\label{lemma0}
$\Powerset^{M}(\kappa) \subseteq N$.
\end{lemma}
\begin{proof}
$A=j(A)\intersect\kappa$.
\end{proof}

\begin{lemma}
\label{lemma1}
Suppose $j(A) = A$ and $j\restr A = id$ and $f\in M$ is a function with $\dom(f) = A$. Then $j(f) = j \circ f$.
\end{lemma}
\begin{proof}
$\dom(j(f)) = A$ and for  $x \in A$, if $y = f(x)$ then $j(y) = j(f)(x)$.
\end{proof}

\begin{lemma}
\label{lemma2}
Suppose $\nu<\kappa$ and $f\in M$, $f:\nu \map A$ where $A\in M$ is such that $f\restr A = id$.
Then $j(f) = f$.
\end{lemma}
\begin{proof}
Follows from \ref{lemma1}.
\end{proof}

\begin{lemma}
$M\models$ ``$\kappa$ is a regular cardinal.''
\end{lemma}
\begin{proof}
Suppose $f:\nu\cofmap\kappa$, $\nu < \kappa$, $f\in M$. Then $j(f) : \nu \cofmap j(\kappa)$.
But by \ref{lemma2} $j(f) = f$ is not cofinal in $j(\kappa) > \kappa$.
\end{proof}

\begin{lemma}
\label{equivalence1}
The following are equivalent:
\begin{itemize}
\item[(a)] $M\models$ ``$\kappa$ is an inaccessible cardinal.''

\item[(b)] $j\restr V^{M}_{\kappa} = id$.

\item[(c)] $V^{M}_{\kappa} = V^{N}_{\kappa}$.

\item[(d)] $\Powerset^{M}(\alpha) = \Powerset^{N}(\alpha)$ for all $\alpha < \kappa$.
\end{itemize}
\end{lemma}
\begin{proof}
(a) $\Implies$ (b). By induction on $\alpha < \kappa$ we show that $j\restr V^{M}_{\alpha} = id$.
The limit step is trivial so suppose $j\restr V^{M}_{\alpha} = id$ and we will show
$j\restr V^{M}_{\alpha + 1} = id$. Let $A\subseteq V^{M}_{\alpha}$ with $A\in M$.
Since $\kappa$ is inaccessible in $M$ let $f:\nu\bijection A$, with $\nu < \kappa$ and
$f\in M$. By lemma \ref{lemma2} and induction, $j(f) = f$ so $j(A) = A$.

(b) $\Implies$ (a). Suppose $A\in V^{M}_{\kappa}$ and $f\in M$, $f:A \cofmap \kappa$. Then
by (b) and lemma \ref{lemma2}, $j(f) = f$. But this contradicts the fact that $j(f)$
is cofinal in $j(\kappa) > \kappa$.

(b) $\Implies$ (c). For $\alpha < \kappa$,  $V^{N}_{\alpha} = j(V^{M}_{\alpha}) = V^{M}_{\alpha}$.

(c) $\Implies$ (b). By induction on $\alpha < \kappa$ we show that $j\restr V^{M}_{\alpha} = id$.
The limit step is trivial so suppose $j\restr V^{M}_{\alpha} = id$ and we will show
$j\restr V^{M}_{\alpha + 1} = id$. Let $A\subseteq V^{M}_{\alpha}$ with $A\in M$.
Let $x \in V^{M}_{\alpha} = V^{N}_{\alpha}$. Then $x\in A$ iff $x = j(x) \in j(A)$.

(c) $\Implies$ (d). Immediate.

(d) $\Implies$ (a). Assume (d) and suppose $2^\alpha >= \kappa$ in $M$ for some
$\alpha < \kappa$ and let $f:\Powerset^{M}(\alpha) \surjection \kappa$. By (d)
$j(\Powerset^{M}(\alpha)) = (\Powerset^{M}(\alpha)$ and by Lemma \ref{subsetLemma},
$j\restr \Powerset^{M}(\alpha) = id$  so $j(f) = j \circ f$
by Lemma \ref{lemma1}. But this is a contradiciton because $j(f)$ is onto $j(k)$
whereas $\kappa \not\in\ran(j)$.
\end{proof}

The conditions (a) - (d) in the previous lemma all follow in particular from the
hypotheses $\Powerset^{M}(\kappa) = \Powerset^{N}(\kappa)$.

\begin{lemma}
\label{equivalence2}
The following are equivalent, and in case they are true
$\kappa$ is a weakly compact cardinal with stationarily many weakly
compact cardinals below it in both $M$ and $N$.
\begin{itemize}
\item[(a)]  $\Powerset^{M}(\kappa) = \Powerset^{N}(\kappa)$.

\item[(b)] $V^{M}_{\kappa + 1} = V^{N}_{\kappa + 1}$.
\end{itemize}
\end{lemma}
\begin{proof}
(a) $\Implies$ (b). Assuming (a) $\kappa$ is inaccessible in $M$
and $V^{M}_{\kappa} = V^{N}_{\kappa}$, by the previous lemma. It follows
that $\kappa$ is a strong limit cardinal in $N$ and using (a) again it follwos
that $\kappa$ is regualr in $N$. So $\kappa$ is inacessible in $M$ and $N$ and
so (b) follows from the fact that $|V_{\kappa}| = \kappa$ in $M$ and $N$.

Now let $T\in M$ be a tree on $\kappa$ of height $\kappa$. Then $j(T) \in N$ is
a tree on $j(\kappa)$ of height $j(\kappa)$ so $j(T)\intersect V^{N}_{\kappa}$
has a cofinal branch. But $j(T)\intersect V^{N}_{\kappa} = T$ so $T$ has a cofinal
branch  $b\in N$ and by (a) $b\in M$. Thus $\kappa$ has the tree property in
$M$ and so it is weakly compact. But then $\kappa$ is also weakly compact in
$N$. Now let $C\subseteq\kappa$ be club with $C\in M$. Then $\kappa\in j(C)$
so $N\models$ ``there is a weakly compact cardinal in $j(C)$" so
$M\models$ ``there is a weakly compact cardinal in $C$". So there are stationarily
many weakly compact cardinals below $\kappa$ in $M$ and so also in $N$.

(b) $\Implies$ (a). Trivial.
\end{proof}

\begin{example}
It is possible for $\kappa$ to be inaccessible in $M$ and $N$ but
$\Powerset^{M}(\kappa) \not= \Powerset^{N}(\kappa)$. For example suppose
$0^{\#}$ exists and let $\kappa < \kappa_2 < \kappa_3$ be indiscernibles,
with $\kappa$ a limit indisernible.
Working in $L$, let $X$ be the hull in
$L_{\kappa_3}$ of $\kappa \union \singleton{\kappa_2}$.
Notice that $\kappa \notin X$ and $|X| = \kappa$. Let $j : M \bijection X$
be the inverse of the transitive collapse. Then
$j:M \prec L_{\kappa_3}$ is an elementary embedding with critical point
$\kappa$, $\kappa$ is inaccessible
in $M$ and in $L_{\kappa_3}$ but $M$ does not have every subset of $\kappa$.
\end{example}

\section{Ultrapowers}

\begin{definition}
Let $M$ be a transitive model of a sufficient amount of $\ZFC$ and let $\kappa$ be
a cardinal of $M$. An $M$-ultrafilter $U$ is \em{weakly amenable} to $M$ iff whenever
$\sequence{X_{\alpha}}{\alpha<\kappa} \in M$ wtih each $X_{\alpha} \subseteq \kappa$,
then $\setof{\alpha < \kappa}{X_{\alpha} \in U} \in M$.
\end{definition}

Of course if $U\in M$ then $U$ is weakly amenable to $M$.

\begin{lemma}
Let $j:M \map N$ be an elementary embedding with $\crit(j)=\kappa$, where $M$ and $N$ are transitive models of a sufficient amount of $\ZFC$.
Let $U = \setof{X \subseteq \kappa}{\kappa \in j(X)}$. If
$\Powerset^{M}(\kappa) = \Powerset^{N}(\kappa)$ then $U$ is weakly amenable to $M$.
\end{lemma}
\begin{proof}
$$\setof{\alpha < \kappa}{X_{\alpha} \in U} =
\setof{\alpha < \kappa}{\kappa\in j(X_{\alpha})} =
S \intersect \kappa $$
where S = $\setof{\alpha < j(\kappa)}{\kappa\in Y_{\alpha}}$
where $\sequence{Y_{\alpha}}{\alpha<j(\kappa)} = j(\sequence{X{\alpha}}{\alpha<\kappa})$.
Since $S\in N$, $S \intersect \kappa \in M$.
\end{proof}

\begin{lemma}
\label{SameNextLevel}
Let $M$ be a transitive model of a sufficient amount of $\ZFC$, let $\kappa$ be
a cardinal of $M$, and let $U$ be a normal
 $M$-ultrafilter over $\kappa$. Suppose
$N = \Ult(M, U)$ is wellfounded. Then the following are equivalent:
\begin{itemize}
\item[(a)] $\Powerset^{M}(\kappa) = \Powerset^{N}(\kappa)$

\item[(b)] $U$ is weakly amenable to $M$.
\end{itemize}
\end{lemma}
\begin{proof}
(a) $\Implies$ (b) follows from the previous lemmma.

(b) $\Implies$ (a). Let $A\in\Powerset^{N}(\kappa)$  and let $f\in M$ represent
$A$, with $f(\beta) \subseteq \beta$ for all $\beta < \kappa$. For each $\alpha<\kappa$
let $X_{\alpha} = \setof{\beta < \kappa}{\alpha \in f(\beta)}$. Then
$A = \setof{\alpha<\kappa}{X_{\alpha} \in U}$. So $A\in M$ by weak amenability.
\end{proof}

From lema \ref{equivalence2} in case (a) or (b) holds we have
$V^{M}_{\kappa+1} = V^{N}_{\kappa+1}$ and $\kappa$ is inaccessible
in $M$ and $N$.

\section{Iterated Ultrapowers}

\begin{lemma}
Suppose $\sequence{M_{\alpha}, \kappa_{\alpha}, U_{\alpha}, i_{(\alpha, \beta)}}{\alpha<\beta}$
is an iterated ultrapower with non-decreasing critical points $\kappa_{\alpha}$ and weakly amenable
ultrafilters $U_{\alpha}$. Then $\kappa_{\alpha}$ is an inaccessible cardinal in
$M_{\beta}$ and $V^{M_{\alpha}}_{\kappa_{\alpha} + 1} = V^{M_{\beta}}_{\kappa_{\alpha} + 1}$
for $\beta>=\alpha$.
\end{lemma}
\begin{proof}
By induction. The successor step follows from lemma \ref{SameNextLevel}. The limit step is trivial.
\end{proof}

\section{Mitchell Models}

\begin{note}
Lemma 2.10 of \cite{BeginningInnerModelTheory} and Theorem 19.37 of \cite{Jech_Book2}
are incorrect as written.
\end{note}
\begin{proof}
On page 1469 of \cite{BeginningInnerModelTheory} we see the following defintion:

\begin{quote}
Call a sequence $\cU$ \emph{weakly coherent} if it satisfies conditions 1 and 2 of Definition 2.6, together with the following weakened coherence condition: if $(\kappa,\beta) \in \dom(\cU)$ and
$U = \cU(\kappa, \beta)$ then $o^{V}(U)=\beta$.
\end{quote}

Then Lemma 2.10 claims:

\begin{quote}
Suppse $\cU$ and $\cW$ are weakly coherent sequences of measures in $V$ with the same
domain. Then $L[\cU] = L[\cW]$ and $\cU(\kappa, \beta) \intersect L[\cU] =
\cW(\kappa, \beta) \intersect L[\cW]$ for every $(\kappa, \beta)$ in their common
domain.
\end{quote}

Here is a simple example to see that this is not true. Suppose $o(\kappa) = 2$
and there exists two different normal measures on $\kappa$ of order 1:
$U_{1} \not= W_{1}$. Let $A\subseteq \kappa$ be in $U_{1}$ but not $U_{2}$ and consist
of only measurable cardinals.
Let the domains of $\cU$ and $\cW$ be $\setof{(\alpha,0)}{\alpha\in A} \union
\singleton{(\kappa,0), (\kappa, 1)}$.
For $\alpha\in A$ let $\cU(\alpha,0) = \cW(\alpha, 0)$ = any normal measure on
$\alpha$ of order 0. Let $\cU(\kappa,0) = \cW(\kappa, 0)$  = any normal measure
on $\kappa$ of order 0. Let $\cU(\kappa,1) = U_{1}$,
$\cW(\kappa,1) = W_{1}$. Then $\cU$ and $\cW$ are weakly coherent according to
the definition given above. Suppose $L[\cU] = L[\cW]$. The since $A\in L[\cU]$
and $A\in U_1$ and $A \notin W_1$, it is clear that
$U_1 \intersect L[\cU] \not= W_1 \intersect L[\cW]$.

The same example contradicts Theorem 19.37 of \cite{Jech_Book2}.
\end{proof}

To correct the problem described above we must modify the definition of \emph{weakly coherent}.
To use the language from \cite{Jech_Book2} and from \cite{Mitchell-Revisited},
if $(\kappa,\beta) \in \dom(\cU)$  and $U = \cU(\kappa, \beta)$ we want not that
$o^{V}(U)=\beta$ but rather that $o^{\cU}(U)=\beta$.

In \cite{Mitchell-Revisited} Mitchell gives the following definition of \emph{weakly coherent}:
If $(\kappa,\beta) \in \dom(\cU)$  and $U = \cU(\kappa, \beta)$ then
then  $[\alpha \mapsto o^{\cU}(\alpha)]_{U} = \beta$. This is equivalent to the condition
$o^{\cU}(U)=\beta$. In \cite{Mitchell-Revisited} Mitchell then goes on to prove Theorem 5 which states
the same thing as Lemma 2.10 from \cite{BeginningInnerModelTheory} that was quoted above.
With the correct definition of weakly coherent the proof of Theorem 5 from
\cite{Mitchell-Revisited} works.

It seems that in \cite{BeginningInnerModelTheory} Mitchell thought he could
simplify the definition of \emph{weakly coherent}, but the simpler definition does not work.

It is interesting to consider why the proof of Theorem 5 from \cite{Mitchell-Revisited} fails using
the incorrect definition of weakly coherent from \cite{BeginningInnerModelTheory}. The answer I think
is that with the incorrect definition of weakly coherent the comparison iteration from the proof
of Theorem 5 does not necessarily use strictly increasing critical points. For example if you consider
the comparison of the $L[\cU]$ and $L[\cW]$ from the counter example given above, the critical point
$\kappa$ will end up being used more than once. This invalidates the argument that attempts to show
that the measures on $\kappa$ used on the two sides of the comparison must be equal. What is curious
is that in the proof sketch of Lemma 2.10 from \cite{BeginningInnerModelTheory}, Mitchell points
out that:
\begin{quote}
Unlike [an earlier proof], the sequence of ordinals $\kappa_{\nu}$ need not be strictly increasing;
however the sequence is nondecreasing and the fact that $\beta_{\nu+1} <\beta_{\nu}$ whenever
$\kappa_{\nu+1} = \kappa_{\nu}$ implies that for each $\nu$ there is an $n < \omega$ such that
$\kappa_{\nu + n} > \kappa_{\nu}$. This, together with the weak coherence of $\cU$ and $\cW$ is
enough to show that the comparison termates at some stage $\theta$.
\end{quote}

I find this statement confusing because we don't need any hypothesis to know that
there is an $n < \omega$ such that $\kappa_{\nu + n} > \kappa_{\nu}$. This follows from the fact that
the iterated ultrapower is well-founded. Also I don't know what Mitchell had in mind for how to
finish the proof without strictly increasing critical points.

It is also interesting to note that something similar to the incorrect definition of weakly coherent
from \cite{BeginningInnerModelTheory} is correst. In \cite{Mitchell-Revisited} Mitchell points out
that:

\begin{quote}
A simple example of a weakly coherent sequence $\cF$ ... is obtained by letting, for $\delta<o(\alpha)$,
$\cF(\alpha,\delta)$ be any measure $U$ on $\alpha$ such that $o(U) = \delta$.
\end{quote}

This is correct and sounds very similar to the incorrect definition of weakly coherent from
\cite{BeginningInnerModelTheory}. The difference is that here we are adding measures at \emph{every}
$\alpha$, not just $\alpha$ in an arbitrary domain.



\bibliographystyle{amsalpha}

\bibliography{math}

\end{document}
