% This is the file correct.tex

\skipbig

\section{Inductive Correctness}

\label{section:correctness}

In this section we will complete the proof that $\Ahyp$ is a mouse
set. We will show that if $\cM$ is iterable and fairly big,
then $\Ahyp\subseteq\R\intersect\cM$.
We will need to use Woodin's stationary tower forcing. See chapter 9  of
\cite{Martin_Book} for a thorough treatment of stationary tower forcing.

\begin{definition}
Let $\delta$ be a limit ordinal. Then $\Q_{\delta}$ is the
``$\Q$-version''  of the stationary
tower forcing up to $\delta$.  More specifically, $x\in\Q_{\delta}$ iff
$x\in V_{\delta}$ and  $x$ is a stationary set and every element of $x$
is countable. The ordering on $\Q_{\delta}$ is  described in chapter 9
of \cite{Martin_Book}.
\end{definition}

We will need to use the following technical fact about the
$\Q_{\delta}$.

\begin{lemma}[Woodin]
\label{TechFact}
Let $\delta$ be a Woodin cardinal, and let $\rho>\delta$ be a
limit ordinal. There is a condition $\hat{S}\in\Q_{\rho}$ such that
for all $T\in\Q_{\delta}$, $\hat{S}$ is compatible with $T$ in
$\Q_{\rho}$. Furthermore, if $G$ is $V$-generic over $\Q_{\rho}$ with
$\hat{S}\in G$, then $G\intersect\Q_{\delta}$ is $V$-generic
over $\Q_{\delta}$.
\end{lemma}
\begin{proof}
(Sketch.) We will assume that the reader is familiar with the technical
details of chapter 9 of \cite{Martin_Book}. Let $\hat{S}$ be the set of all
$X\in\cP_{\aleph_1}(V_{\delta+1})$ such that $X$ captures every
$\mathbf{A}\in X$ that is predense in $\Q_{\delta}$. It is shown in
\cite{Martin_Book} that if $G$ is $V$-generic over $\Q_{\rho}$ with
$\hat{S}\in G$, then $G\intersect\Q_{\delta}$ is $V$-generic
over $\Q_{\delta}$. So it only remains to show that for all
$T\in\Q_{\delta}$, $\hat{S}$ is compatible with $T$ in
$\Q_{\rho}$.  Fix $T\in\Q_{\delta}$. Let $T^*$ be the set of
$X\in\cP_{\aleph_1}(V_{\delta+1})$ such that $X\intersect\bigcup T\in T$
and such that $X$ captures every $\mathbf{A}\in X$ that is predense in
$\Q_{\delta}$. It is shown in \cite{Martin_Book} that $T^*$ is stationary
in $\cP_{\aleph_1}(V_{\delta+1})$. Thus $T^*\in\Q_{\rho}$.
Clearly $T^*$ is below both $\hat{S}$
and $T$. Thus $\hat{S}$ is compatible with $T$ in
$\Q_{\rho}$.
\end{proof}

Using the above-mentioned fact, we will now prove another
 technical lemma about the $\Q_{\delta}$. Suppose that
$\delta_1<\delta_2$ are Woodin cardinals. The next lemma says, roughly,
that a $\Q_{\delta_1}$-generic object can be extended to a
$\Q_{\delta_2}$-generic object in such a way as to absorb some other
given generic object.

\begin{lemma}
\label{Extend_Stat_Tower_Filter}
Let $M$ be a transitive inner model of some sufficiently large fragment
of $\ZFC$. Let $\delta_1<\delta_2$ be countable ordinals such that
in $M$,  $\delta_1$ and $\delta_2$ are both Woodin cardinals.
Let $\Q_1=\Q_{\delta_1}^{M}$ and  $\Q_2=\Q_{\delta_2}^{M}$.
Let $\P\in M$  be any partial order such that $\card(\P)^{\cM}<\delta_2$.
Let $G_1$ and $H$ be sets such that $G_1$ is $M$-generic over $\Q_1$
and $H$ is $M[G_1]$-generic over $\P$. Then there exists a $G_2$ such that
\begin{itemize}
\item $G_2$ is $M$-generic over $\Q_2$.
\item $G_2\intersect\Q_1=G_1$.
\item $H\in M[G_2]$.
\end{itemize}
\end{lemma}
\begin{proof}
By Lemma \ref{TechFact} above, there is a condition
$\hat{S}\in\Q_2$ such that
for all $T\in\Q_1$, $\hat{S}$ is compatible with $T$ in
$\Q_2$. Furthermore, if $G$ is $M$-generic over $\Q_2$ with
$\hat{S}\in G$, then
$G\intersect\Q_1$ is $M$-generic
over $\Q_1$. Fix such a condition $\hat{S}$. Let $\Q_2^{\prime}$ be the
collection of elements $T\in\Q_2$ such that $T$ is compatible
with $\hat{S}$. So we have that $\Q_1\subset\Q_2^{\prime}\subset\Q_2$.
Below we
will use the notion of a \emph{complete} embedding from one partial
order into another. (See Definition 7.1 on page 218 of \cite{Kunen_Book}.)

\begin{claim}[Claim 1]
$\Q_1$ is a complete suborder of $\Q_2^{\prime}$.
\end{claim}
\begin{subproof}[Proof of Claim 1]
Let $G$ be $M$-generic over $\Q_2^{\prime}$. Then $G$ is $M$-generic
over $\Q_2$, and $\hat{S}\in G$. So $G\intersect\Q_1$ is $M$-generic
over $\Q_1$.  In summary we have that whenever $G$ is generic over
$\Q_2^{\prime}$, $G\intersect\Q_1$ is generic over $\Q_1$. It is an easy
exercise to see that this condition is equivalent to the fact that
$\Q_1$  is a complete suborder of $\Q_2^{\prime}$.
\end{subproof}

Now fix $G_1$ and $H$ so that $G_1$ is $M$-generic over $\Q_1$
and $H$ is $M[G_1]$-generic over $\P$. We must find a $G_2$ as in
the statement of the lemma. Working in $M[G_1]$, set
$$\Q=\setof{p\in\Q_2^{\prime}}{p \text{ is compatible with every element
of } G_1}.$$

\begin{claim}[Claim 2]
In $M[G_1]$,  $\ro(\P)$ can be completely embedded into $\ro(\Q)$.
\end{claim}
\begin{subproof}[Proof of Claim 2]
Work in $\cM[G_1]$. Let $\gamma=\card\bigl(\cP(\ro(\P))\bigr)$.
As $\delta_2$ is inaccessible in $M[G_1]$,
 $\gamma<\delta_2$. We claim that $\Q$ collapses $\gamma$ to
be countable. For let $G$ be $\cM[G_1]$-generic over $\Q$.
By exercise (D4) on page 244 of \cite{Kunen_Book}, $G$ is
$M$-generic over $\Q_2^{\prime}$, and $M[G_1][G]=M[G]$. It then
follows that $G$ is $M$-generic over $\Q_2$. By chapter 9 of
\cite{Martin_Book}, $\delta_2=(\omega_1)^{\cM[G]}$. So $\gamma$ is
countable in $M[G]=M[G_1][G]$. In other words,  $\Q$ collapses $\gamma$ to
be countable. Our conclusion easily follows from this.
For let $\dot{H}$ be a $\Q$-name for a generic for $\ro(\P)$. If
$b\in\ro(\P)$, let $\pi(b)$ be the boolean value in $\ro(\Q)$
that $\check{b}\in\dot{H}$. As in the proof of Lemma 25.5 from
\cite{Jech_Book}, it is easy to see that $\pi$ is a complete embedding.
\end{subproof}

It follows  that there
is a $G_2$ such that $G_2$ is $M[G_1]$-generic over $\Q$, with
$H\in M[G_1][G_2]$. Fix such a $G_2$. By exercise (D4) on page
244 of \cite{Kunen_Book}, $G_2$ is $M$-generic over $\Q_2^{\prime}$
and $M[G_2]=M[G_1][G_2]$. It follows that $G_2$ is $M$-generic
over $\Q_2$ and $\hat{S}\in G_2$. Thus $G_2\intersect\Q_1=G_1$ and
our lemma is proved.
\end{proof}


We shall use heavily a result due to H. Woodin concerning genericity over
$L[\vec{E}]$ models. This result is often referred  to as ``iterating
to make a real generic.'' We state the result here in the form in which
we shall need it. The following is also Theorem 4.3 from
\cite{Many_Woodins}.

\begin{lemma}[Woodin]
\label{MakeRealGeneric}
Let $\cM$ be a countable, realizable,  tame premouse.  Suppose
$\gamma<\delta<\ORD^{\cM}$. Suppose
$\cM\models\text{``}\delta$ is a Woodin cardinal.''
Let $\P\subseteq J^{\cM}_{\gamma}$ be a partial
order in $\cM$ and let $G$ be $\cM$-generic over $\P$. Then there
is a partial order $\Q\subseteq J^{\cM}_{\delta}$, with
$\Q\in \cM$, such that for any real $w$, there is an
iteration tree $\cT$ on $\cM$ of countable length $\theta+1$ such that
\begin{itemize}
\item[(a)] $\cM^{\cT}_{\theta}$ is realizable, and
\item[(b)] $D^{\cT}=\emptyset$ so that $i^{\cT}_{0,\theta}$ is defined,
and
\item[(c)] $\crit(E^{\cT}_{\xi}) > \gamma$ for all $\xi<\theta$
(so $G$ is $\cM^{\cT}_{\theta}$-generic over $\P$), and
\item[(d)] $w$ is $\cM^{\cT}_{\theta}[G]$-generic over
$i^{\cT}_{0,\theta}(\Q)$.
\end{itemize}
\end{lemma}

For an idea of how to prove this lemma, see the exercises at the
end of chapters 6 and 7 in \cite{Martin_Book}.

Our next two results will say that if
a premouse $\cM$ has $\omega$ Woodin cardinals cofinal in its ordinals,
then $\cM$ is, in
a certain sense, $\Sigma_1(\JofR{\kappa})$-correct.
The  following is the main technical lemma in this section. Our
proof of this lemma is very similar to, and owes its main idea to
Steel's proof of Corollary 4.7 in \cite{Many_Woodins}. In that proof,
steel iterates a premouse $\cM$, yielding a premouse $\cM^{\prime}$,
with the property that every real in $V$ is generic over $\cM^{\prime}$.
Steel then takes a generic ultrapower of $\cM^{\prime}$ yielding an
embedding $j:\cM^{\prime}\map\Ult$, with the property that
$\R^{\Ult}=\R^{V}$.
Now $\Ult$ will in general not
be wellfounded, but because of the large cardinals in $\cM^{\prime}$
Steel can arrange that the ordinal height of the wellfounded part
of $\Ult$ is as large as he wants.
(Steel assumes that $\cM$ has $\omega$ Woodin cardinals, \emph{plus}
another extender above the $\omega$ Woodin cardinals.) This allows
Steel to reflect any $\Sigma_1$ truth in $\LofR$ down to $\cM^{\prime}$,
and hence down to $\cM$. In our setting, we are only assuming that
$\cM$ has $\omega$ Woodin cardinals cofinal in its ordinals. We will
therefore not be able to arrange that the ordinal height of
the wellfounded part of $\Ult$ is as large as we want. Instead, we
will only be able to quote abstract admissibility theory to conclude
that the ordinal height of the wellfounded part of $\Ult$ is at
least $\kappa^{\R}$. This will allow us to reflect $\Sigma_1$ truth
from $\JofR{\kappa}$ down to $\cM^{\prime}$, and hence down to $\cM$.

\begin{lemma}
Let $\cM$ be a countable, realizable, tame premouse. Suppose that
$\delta_1<\delta_2<\delta_3<\cdots$ is an increasing $\omega$-sequence of
ordinals such that each $\delta_i$ is a Woodin cardinal of $\cM$,
and the $\delta_i$ are cofinal in the ordinals of $\cM$.
Let $\P$ be any partial order in $J^{\cM}_{\delta_1}$ and suppose that
 $H$ is $\cM$-generic over $\P$, with $H\in V$.
Let $\R^*=\R\intersect\cM[H]$.
Then there is an ordinal $\alpha<\delta_1$
such that whenever $x\in\R^*$ and $\formulaphi$ is a $\Sigma_1$ formula
and $\JofR{\kappa}\models\formulaphi[x,\R]$, then
 $J_{\alpha}(\R^*)\models\formulaphi[x,\R^*]$.
\end{lemma}
\begin{proof}
 Fix an ordinal $\mu<\delta_1$
such that $\P\in J^{\cM}_{\mu}$.
Let $\Col(\omega,2^{\aleph_0})$ be the partial order which collapses
$2^{\aleph_0}$ to be countable. For the rest of the proof we work
in a generic extension of $V$ via this partial order.
 Let $x_1,x_2,x_3,\dots$ be an
$\omega$-enumeration  of the reals of $V$.

We are going to iterate $\cM$ to yield a new premouse $\cM^{\prime}$
with the property that every real of $V$ is generic over $\cM^{\prime}[H]$.
More specifically, we are going to define
a sequence of mice: $\cM_0, \cM_1, \cM_2, \dots$ with $\cM_0=\cM$,
and a commutative system of embeddings $j_{n,m}:\cM_n\map\cM_m$,
with the property that $\crit(j_{0,n})>\mu$ so that $H$ is $\cM_n$-generic
over $\P$, and
a sequence of partial orders $\Q_1,\Q_2,\Q_3,\dots$ such that
for each  $n\geq1$, $\Q_n\in\cM_n[H]$, and a sequence of generic objects
$G_1,G_2,G_3,\dots$ such that for each $n\geq1$, $G_n$ is
$\cM_n[H]$-generic over $\Q_n$.  We will arrange that for each $n\geq 1$,
$x_n\in\cM_n[H][G_n]$.  We will maintain
inductively that every finite initial segment of  our construction lives in
$V$. The whole construction will not live in $V$.  Then we will define
$\cM^{\prime}$ to be the direct limit of the $\cM_n$.

Given such a construction, let $\bar{\delta}_{2n}=j_{0,n}(\delta_{2n})$.
We will also arrange that:
\begin{itemize}
\item $\Q_n\subseteq V_{\bar{\delta}_{2n}}^{\cM_n[H]}$, and
\item $\crit(j_{n,m})>\bar{\delta}_{2n}$.
\end{itemize}
Thus if $m>n$ then we will have that $G_n$ is $\cM_m[H]$-generic
over $\Q_n$. Finally, we will arrange that if $m>n$ then
$G_m\intersect \Q_n = G_n$.

We begin by setting $\cM_0=\cM$. Now, working  in $V$ and
using Lemma \ref{MakeRealGeneric} above, we can find a partial order
$\hat{\P}\subseteq J^{\cM_0}_{\delta_1}$, with $\hat{\P}\in \cM_0$, and an
iteration tree $\cT$ on $\cM_0$ of countable length $\theta+1$ such that
\begin{itemize}
\item[(a)] $\cM^{\cT}_{\theta}$ is realizable, and
\item[(b)] $D^{\cT}=\emptyset$ so that $i^{\cT}_{0,\theta}$ is defined,
and
\item[(c)] $\crit(E^{\cT}_{\xi}) > \mu$ for all $\xi<\theta$
(so $H$ is $\cM^{\cT}_{\theta}$-generic over $\P$), and
\item[(d)] $x_1$ is $\cM^{\cT}_{\theta}[H]$-generic over
$i^{\cT}_{0,\theta}(\hat{\P})$.
\end{itemize}
Set $\cM_1=\cM^{\cT}_{\theta}$ and set $j_{0,1}=i^{\cT}_{0,\theta}$. Now
let $\bar{\delta}_2=j_{0,1}(\delta_2)$, and let
$\Q_1=\Q_{\bar{\delta}_2}^{\cM_1[H]}$.
Finally, again working in $V$ and using (part of the statement of)
Lemma \ref{Extend_Stat_Tower_Filter}, let $G_1$ be an $\cM_1[H]$-generic
filter over $\Q_1$ such that $x_1\in\cM_1[H][G_1]$.

Now we describe the second step of our construction. Working in $V$ again,
and using  Lemma \ref{MakeRealGeneric}, we can find a partial order
$\hat{\P}\subseteq J^{\cM_1}_{j_{0,1}(\delta_3)}$, with $\hat{\P}\in
\cM_1$, and an iteration tree $\cT$ on $\cM_1$ of countable length
$\theta+1$ such that
\begin{itemize}
\item[(a)] $\cM^{\cT}_{\theta}$ is realizable, and
\item[(b)] $D^{\cT}=\emptyset$ so that $i^{\cT}_{0,\theta}$ is defined,
and
\item[(c)] $\crit(E^{\cT}_{\xi}) > \bar{\delta_2}$ for all $\xi<\theta$
(so $H$ is $\cM^{\cT}_{\theta}$-generic over $\P$ and $G_1$ is
$\cM^{\cT}_{\theta}[H]$-generic over $\Q_1$), and
\item[(d)] $x_2$ is $\cM^{\cT}_{\theta}[H][G_1]$-generic over
$i^{\cT}_{0,\theta}(\hat{\P})$.
\end{itemize}
 Set $\cM_2=\cM^{\cT}_{\theta}$ and set
$j_{1,2}=i^{\cT}_{0,\theta}$ and set $j_{0,2}=j_{0,1}\circ j_{1,2}$. Now
let $\bar{\delta}_4=i_{0,2}(\delta_4)$, and let
$\Q_2=\Q_{\bar{\delta}_4}^{\cM_2[H]}$.
Finally, again working in $V$ and using
Lemma \ref{Extend_Stat_Tower_Filter}, let $G_2$ be an $\cM_2[H]$-generic
filter over $\Q_2$ such that $x_2\in\cM_2[H][G_2]$, and
$G_2\intersect\Q_1=G_1$.

The $n$th step of the construction for $n>2$ is similar to the
second step of the construction. It is easy to see how to continue the
construction so as to obtain a sequence of mice
$\cM_0, \cM_1, \cM_2, \dots$,  a commutative system of embeddings
$j_{n,m}:\cM_n\map\cM_m$, a sequence of partial orders
$\Q_1, \Q_2, \Q_3, \dots$, and a sequence of generic filters
$G_1, G_2, G_3, \dots$ as described above. Letting
$\bar{\delta}_{2n}=j_{0,n}(\delta_{2n})$ we have, in summary, the following:
\begin{itemize}
\item[(1)] $\crit(j_{0,n})>\mu$ so $H$ is $\cM_n$-generic over $\P$.
\item[(2)] $\Q_n = \Q_{\bar{\delta}_{2n}}^{\cM_n[H]}$ for $n\geq 1$.
\item[(3)] $G_n$ is $\cM_n[H]$-generic over $\Q_n$, for $n\geq 1$.
\item[(4)] $x_n\in\cM_n[H][G_n]$, for $n\geq 1$.
\item[(5)] For $1\leq n < m$, $\crit(j_{n,m})>\bar{\delta}_{2n}$ so
           $G_n$ is $\cM_m[H]$-generic over $\Q_n$.
\item[(6)] For $1\leq n < m$, $G_m\intersect \Q_n= G_n$.
\item[(7)] For $n\geq 0$, $\cM_n[H][G_n]\in V$.
\end{itemize}

Let $\cM^{\prime}$ be the direct limit of the $\cM_n$ under the
$j_{n,m}$. It is obvious that $\cM^{\prime}$ is wellfounded, because
every thread in the direct limit system is eventually constant.
Let $j:\cM\map\cM^{\prime}$ be the direct limit map.
The
following facts follow easily from (1) through (7) above:
\begin{itemize}
\item[(a)] $\bar{\delta}_{2n} = j(\delta_{2n})$.
\item[(b)] $H$ is $\cM^\prime$-generic over $\P$.
\item[(c)] $\Q_n=\Q_{\bar{\delta}_{2n}}^{\cM^{\prime}[H]}$.
\item[(d)] $G_n$ is $\cM^{\prime}[H]$-generic over $\Q_n$.
\item[(e)] $x_n\in\cM^{\prime}[H][G_n]$.
\item[(f)] $\R\intersect\cM^{\prime}[H][G_n]=
            \R\intersect\cM_n[H][G_n]\in V$, for $n\geq 1$.
\end{itemize}
It follows from (e) and (f) above that
$\Union{n}\R\intersect\cM^{\prime}[H][G_n]=\R^{V}$.

Our next step is to form the generic ultrapower of $\cM^{\prime}[H]$ via
the $G_n$s.
Let $\cP_n=\Ult(\cM^{\prime}[H],G_n)$. Since $\bar{\delta}_{2n}$ is a Woodin
cardinal of $\cM^{\prime}[H]$, by Chapter 9 of \cite{Martin_Book},
$\cP_n$ is wellfounded.
(While $\cM^{\prime}[H]$ is not a model of ZFC, and so the results of
Chapter 9 of \cite{Martin_Book} do no literally apply to $\cM^{\prime}[H]$,
cofinally many of the rank initial segment of $\cM^{\prime}[H]$ are models
of ZFC. Thus cofinally many of the rank initial segments of the ultrapower
are wellfounded, and so the ultrapower is wellfounded.)
Also, by by Chapter 9 from \cite{Martin_Book},
$\R\intersect\cP_n=\R\intersect\cM^{\prime}[H][G_n]$.
Let \mbox{$i_n:\cM^{\prime}[H]\map\cP_n$} be the ultrapower embedding. Then
$i_n$ is a cofinal, $\Sigma_0$-embedding.
 For $1\leq n<m$, let $k_{n,m}:\cP_n\map\cP_m$
be the canonical embedding. That is:
$$k_{n,m}\left([f]_{G_n}\right)=[f]_{G_m}.$$
Then the $k_{n,m}$ form a commutative system of embeddings.
Also, for each $n<m$, $i_m=k_{n,m}\circ i_n$.
Let $\cP$ be the direct limit of the $\cP_n$ under the $k_{n,m}$ embeddings.
Let $k_n:\cP_n\map\cP$ be the direct limit map.
Also let $i:\cM^{\prime}[H]\map\cP$ be the induced map.
We may think of $\cP$ as
being the ultrapower of $\cM^{\prime}[H]$ via $\Union{n} G_n$, and we may
think of $i$ as the ultrapower embedding.

Now $\cP$ will, in general, not be wellfounded. Let us identify the
wellfounded part of $\cP$ with the transitive set to which it is
isomorphic. It is easy to see that $\R^{\cP}=\Union{n}\R\intersect\cP_n
=\R^{V}$.
It follows from general admissibility
theory that the rank of the
wellfounded part of $\cP$ is at least $\kappa^{\R}$.

Now fix a real $x\in\R^*$, and fix a $\Sigma_1$ formula $\formulaphi$
such that $\JofR{\kappa}\models\formulaphi[x,\R]$. Then there is
a $\gamma<\kappa^{\R}$ such that $\JofR{\gamma}\models\formulaphi[x,\R]$.
Since $\gamma$ is in the wellfounded part of $\cP$, $\cP$ satisfies the
following $\Sigma_1$ sentence:
\begin{quote}
There is an ordinal $\gamma$ such that
$\JofR{\gamma}\models\formulaphi[x,\R]$ and for all $\eta\leq\gamma$,
$\JofR{\eta}$ is not admissible.
\end{quote}
Since $i$ is a cofinal $\Sigma_0$ embedding, $\cM^{\prime}[H]$ also
satisfies this sentence. Fix a name $\dot{x}$ in $\cM^{\prime}$ such that
$x=\dot{x}_{H}$. Since $\crit(j)>\mu$ we may pick $\dot{x}$ so that it
is fixed by $j$. Let $p\in H$ be such that in $\cM^{\prime}$,
$p$ forces the statement:
\begin{quote}
There is an ordinal $\gamma$ such that
$\JofR{\gamma}\models\formulaphi[\dot{x},\R]$ and for all $\eta\leq\gamma$,
$\JofR{\eta}$ is not admissible.
\end{quote}
Again, since  $\crit(j)>\mu$, we have that $j(p)=p$. Thus $p$ forces the
same statement over $\cM$. So in $\cM[H]$ we have that
\begin{quote}
There is an ordinal $\gamma$ such that
$\JofR{\gamma}\models\formulaphi[x,\R]$ and for all $\eta\leq\gamma$,
$\JofR{\eta}$ is not admissible.
\end{quote}
Fix such an ordinal $\gamma$ in $\cM[H]$. Since $\R\intersect\cM[H]=\R^*$,
we have that $J_{\gamma}(\R^*)\models\formulaphi[x,\R^*]$. Since
$J_{\eta}(\R^*)$ is not admissible for all $\eta\leq\gamma$, we have that
$\gamma$ is less than, for instance,
$\bigl((2^{\aleph_0})^+\bigr)^{\cM[H]}$.
This completes the proof of the lemma. For the ordinal $\alpha$ mentioned
in the statement of the lemma we can take, say,
$\alpha=(\mu^{++})^{\cM}$
\end{proof}

The previous Lemma gave us an $\alpha$ such that
 $\Sigma_1(\JofR{\kappa})$ goes down to $J_{\alpha}(\R^*)$.
The next corollary says that $\alpha$ can be chosen so that
 $\Sigma_1(\JofR{\kappa})$ also goes up.

\begin{corollary}
\label{Is_Correct}
Under the hypotheses of the previous Lemma,
there is an ordinal $\alpha<\delta_1$
such that whenever $x\in\R^*$ and $\formulaphi$ is a $\Sigma_1$ formula
then $\JofR{\kappa}\models\formulaphi[x,\R]$ iff
 $J_{\alpha}(\R^*)\models\formulaphi[x,\R^*]$.
\end{corollary}
\begin{proof}
Let $\alpha_0$ be given by the previous lemma. If $\alpha_0$ does not
satisfy our corollary, then there is some $\Sigma_1$ formula $\formulaphi$,
and some $x\in\R^*$ such that
$J_{\alpha_0}(\R^*)\models\formulaphi[x,\R^*]$, and
$\JofR{\kappa}\not\models\formulaphi[x,\R]$. Let
$\gamma\leq\alpha_0$ be least such that there is some $\Sigma_1$ formula
$\formulaphi$, and some $x\in\R^*$ such that
$J_{\gamma}(\R^*)\models\formulaphi[x,\R^*]$, and
$\JofR{\kappa}\not\models\formulaphi[x,\R]$. Then $\gamma$ is a
successor ordinal. Let $\alpha_1$ be such that $\gamma=\alpha_1+1$.
We claim that $\alpha_1$ witnesses that our corollary is true. So let
$\formulaphi$ by any $\Sigma_1$ formula and let $x\in\R^*$. If
$J_{\alpha_1}(\R^*)\models\formulaphi[x,\R^*]$, then by definition of
$\gamma$, $\JofR{\kappa}\models\formulaphi[x,\R]$. Conversely,
suppose that $\JofR{\kappa}\models\formulaphi[x,\R]$. Fix
$y\in\R^*$, and fix a $\Sigma_1$ formula $\psi$ so that
$J_{\gamma}(\R^*)\models\psi[y,\R^*]$ but
$\JofR{\kappa}\not\models\psi[y,\R]$. Then $\JofR{\kappa}$ satisfies
the following $\Sigma_1$ statement about $\angles{x,y}$:
\begin{quote}
There is an ordinal $\beta$ such that $\JbetaR\models\formulaphi[x,\R]$
and $\JbetaR\not\models\psi[y,\R]$.
\end{quote}
Thus $J_{\alpha_0}(\R^*)$ satisfies the same statement about $\angles{x,y}$.
 So there is an
ordinal $\beta<\alpha_0$ such that
$J_{\beta}(\R^*)\models\formulaphi[x,\R^*]$, and
$J_{\beta}(\R^*)\not\models\psi[y,\R^*]$. Since
$J_{\gamma}(\R^*)\models\psi[y,\R^*]$, we have that $\beta\leq\alpha_1$.
Thus
$J_{\alpha_1}(\R^*)\models\formulaphi[x,\R^*]$.
\end{proof}

\begin{corollary}
\label{Every_Definable_Real_Is_In}
Let $\cM$ be a countable, realizable, tame premouse. Suppose that
$\delta_1<\delta_2<\delta_3<\cdots$ is an increasing $\omega$-sequence of
ordinals such that each $\delta_i$ is a Woodin cardinal of $\cM$,
and the $\delta_i$ are cofinal in the ordinals of $\cM$.
Let $x$ be a real and suppose that for some
$\beta<\kappa^{\R}$, $x$ is
ordinal definable over $\JbetaR$.  Then
$x\in\cM$.
\end{corollary}
\begin{proof}
By Lemma \ref{MakeRealGeneric} there
is a partial order $\Q\subseteq J^{\cM}_{\delta_1}$, with
$\Q\in \cM$ and there is an
iteration tree $\cT$ on $\cM$ of countable length $\theta+1$ such that
\begin{itemize}
\item[(a)] $\cM^{\cT}_{\theta}$ is realizable, and
\item[(b)] $D^{\cT}=\emptyset$ so that $i^{\cT}_{0,\theta}$ is defined,
and
\item[(c)] $x$ is $\cM^{\cT}_{\theta}$-generic over
$i^{\cT}_{0,\theta}(\Q)$.
\end{itemize}
Let $\cM^{\prime}=\cM^{\cT}_{\theta}$ and let
$\delta_1^{\prime}=i^{\cT}_{0,\theta}(\delta_1)$. It suffices to show that
$x\in\cM^{\prime}$. Let $\P$ be be the partial order in $\cM^{\prime}$ for
collapsing $\delta_1^{\prime}$ to be countable. Then there is a filter $H$
which is $\cM^{\prime}$-generic over $\P$ with $x\in\cM^{\prime}[H]$.
Let $\R^*=\R\intersect\cM^{\prime}[H]$.
Let us apply the previous corollary to $\cM^{\prime}$ and $H$. This gives
us an ordinal $\alpha\in\cM^{\prime}$ such that $J_{\alpha}(\R^*)$ satisfies
the following $\Sigma_1$ statement about $x$:
\begin{quote}
There is a $\beta$ such that $x$ is ordinal definable over
$\JbetaR$.
\end{quote}
Thus there is a $\beta$ in $\cM^{\prime}$ so that $x$ is ordinal definable
over $J_{\beta}(\R^*)$. This means that $x$ is ordinal definable
in $\cM^{\prime}[H]$. As $\P$ is a homogeneous forcing, $x\in\cM^{\prime}$.
(Actually, since $\cM^{\prime}$ is not a model of $\ZFC$, we are really using
that $x$ is ordinal definable in $J_{\gamma}^{\cM^{\prime}}[H]$, for
some $\gamma$ such that $J_{\gamma}^{\cM^{\prime}}\models\ZFC$.)
\end{proof}


\begin{theorem}
Assume that there are $\omega$ Woodin cardinals. Let $\cM$ be a countable,
realizable premouse, such that $\cM$ is not fairly small, but every
proper initial segment of $\cM$ is fairly small. Then the reals of
$\cM$ are exactly equal to the set of reals $x$ such that $x$ is ordinal
definable over $\JbetaR$, for some $\beta<\kappa^{\R}$.
\end{theorem}
\begin{proof}
This follows immediately from Theorem \ref{Every_Real_Is_Definable}
on page \pageref{Every_Real_Is_Definable}, and
Corollary \ref{Every_Definable_Real_Is_In} above.
\end{proof}

\begin{remark}
Assume that there are $\omega$ Woodin cardinals. It is shown in
\cite{Many_Woodins} that this implies that there is a proper class
premouse $L[\vec{E}]$
which satisfies $\ZFC +$``There are $\omega$ Woodin cardinals.''
Let $\cM$ be the $\initseg$-least initial segment of $L[\vec{E}]$ which is
fairly big. It is easy to see that $\rho_{\omega}(\cM)=\omega$. $\cM$ is
thus the unique fully iterable premouse which is sound, projects to
$\omega$, has $\omega$ Woodin cardinals cofinal in its ordinals, and has
the property that every proper initial segment is fairly small.
We may think of $\cM$ as the canonical, minimal inner model for the
theory: $\ZFC - $ Replacement $+$ ``There exists $\omega$ Woodin cardinals
cofinal in the ordinals.'' The previous theorem says that
$\Ahyp=\R\intersect\cM$.
\end{remark}
