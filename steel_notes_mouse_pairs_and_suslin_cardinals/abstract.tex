\documentclass[oneside,12pt]{amsart}

\usepackage{amsmath,amssymb,latexsym,eucal,amsthm,rotating}
\usepackage[shortlabels]{enumitem}

%%%%%%%%%%%%%%%%%%%%%%%%%%%%%%%%%%%%%%%%%%%%%
% Macros
%%%%%%%%%%%%%%%%%%%%%%%%%%%%%%%%%%%%%%%%%%%%%
\newcommand{\R}{\mathbb{R}}
\newcommand{\setof}[2]{\left\{ \, #1 \, \left| \, #2 \, \right.\right\}}
\newcommand{\intersect}{\cap}
\newcommand{\bSigma}{\boldsymbol{\Sigma}}
\newcommand{\defeq}{=_{\text{def}}}

\pagestyle{plain}

\begin{document}

\title{The Mouse Set Theorem Just Past Projective}
\begin{abstract}
We will give the proof of the mouse set theorem for the pointclass
$\Pi^1_{\omega+1}$ ($\defeq \forall^{\R} \Sigma^1_{\omega}$,
where $\Sigma^1_{\omega}\defeq$  recursive infinite disjunctions of projective formulae.)
These pointclasses lie at the start of the next projective-like hierarchy
beyond the projective. $\Pi^1_{\omega+1}$  is a scaled pointclass
and may be considered to be analogous to $\Pi^1_3$. Also $\Pi^1_{\omega+1} = \Pi_2(J_2(\R))$.

Let $A=\setof{x\in\R}{x\text{ is } \Delta^1_{\omega+1} \text{ in a countable ordinal.}}$.
$A$ is analogous to $Q_3$. We will describe a mouse $M$ and show that
$\R\intersect M = A$. This is analogous to $\R\intersect M_1^{\sharp} = Q_3$.

$M$ is the least mouse such that for all $n\in\omega$, there is a $\delta_n$ such
that $\delta_n$ is a cardinal of $M$ and is Woodin in $M$ ``with respect to
$\Sigma^1_n$ subsets of $\delta_n$." This mouse was isolated in 1993
and at that time we proved that $\R\intersect M \subseteq A$, but we were
not able to prove that $A \subseteq  M$. Recently Woodin proved
that $A \subseteq M$ yielding the mouse set theorem.

This talk will focus on Woodin's proof that if $x\in\R$ is $\Delta^1_{\omega+1}$
in a countable ordinal, then $x\in M$.


\end{abstract}

\maketitle

\end{document}
