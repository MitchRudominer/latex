\documentclass[oneside,12pt]{amsart}

\usepackage{amsmath,amssymb,latexsym,eucal,amsthm,rotating}
\usepackage[shortlabels]{enumitem}

%%%%%%%%%%%%%%%%%%%%%%%%%%%%%%%%%%%%%%%%%%%%%
% Common Set Theory Constructs
%%%%%%%%%%%%%%%%%%%%%%%%%%%%%%%%%%%%%%%%%%%%%

\newcommand{\setof}[2]{\left\{ \, #1 \, \left| \, #2 \, \right.\right\}}
\newcommand{\lsetof}[2]{\left\{\left. \, #1 \, \right| \, #2 \,  \right\}}
\newcommand{\bigsetof}[2]{\bigl\{ \, #1 \, \bigm | \, #2 \,\bigr\}}
\newcommand{\Bigsetof}[2]{\Bigl\{ \, #1 \, \Bigm | \, #2 \,\Bigr\}}
\newcommand{\biggsetof}[2]{\biggl\{ \, #1 \, \biggm | \, #2 \,\biggr\}}
\newcommand{\Biggsetof}[2]{\Biggl\{ \, #1 \, \Biggm | \, #2 \,\Biggr\}}
\newcommand{\dotsetof}[2]{\left\{ \, #1 \, : \, #2 \, \right\}}
\newcommand{\bigdotsetof}[2]{\bigl\{ \, #1 \, : \, #2 \,\bigr\}}
\newcommand{\Bigdotsetof}[2]{\Bigl\{ \, #1 \, \Bigm : \, #2 \,\Bigr\}}
\newcommand{\biggdotsetof}[2]{\biggl\{ \, #1 \, \biggm : \, #2 \,\biggr\}}
\newcommand{\Biggdotsetof}[2]{\Biggl\{ \, #1 \, \Biggm : \, #2 \,\Biggr\}}
\newcommand{\sequence}[2]{\left\langle \, #1 \,\left| \, #2 \, \right. \right\rangle}
\newcommand{\lsequence}[2]{\left\langle\left. \, #1 \, \right| \,#2 \,  \right\rangle}
\newcommand{\bigsequence}[2]{\bigl\langle \,#1 \, \bigm | \, #2 \, \bigr\rangle}
\newcommand{\Bigsequence}[2]{\Bigl\langle \,#1 \, \Bigm | \, #2 \, \Bigr\rangle}
\newcommand{\biggsequence}[2]{\biggl\langle \,#1 \, \biggm | \, #2 \, \biggr\rangle}
\newcommand{\Biggsequence}[2]{\Biggl\langle \,#1 \, \Biggm | \, #2 \, \Biggr\rangle}
\newcommand{\singleton}[1]{\left\{#1\right\}}
\newcommand{\angles}[1]{\left\langle #1 \right\rangle}
\newcommand{\bigangles}[1]{\bigl\langle #1 \bigr\rangle}
\newcommand{\Bigangles}[1]{\Bigl\langle #1 \Bigr\rangle}
\newcommand{\biggangles}[1]{\biggl\langle #1 \biggr\rangle}
\newcommand{\Biggangles}[1]{\Biggl\langle #1 \Biggr\rangle}


\newcommand{\force}[1]{\Vert\!\underset{\!\!\!\!\!#1}{\!\!\!\relbar\!\!\!%
\relbar\!\!\relbar\!\!\relbar\!\!\!\relbar\!\!\relbar\!\!\relbar\!\!\!%
\relbar\!\!\relbar\!\!\relbar}}
\newcommand{\longforce}[1]{\Vert\!\underset{\!\!\!\!\!#1}{\!\!\!\relbar\!\!\!%
\relbar\!\!\relbar\!\!\relbar\!\!\!\relbar\!\!\relbar\!\!\relbar\!\!\!%
\relbar\!\!\relbar\!\!\relbar\!\!\relbar\!\!\relbar\!\!\relbar\!\!\relbar\!\!\relbar}}
\newcommand{\nforce}[1]{\Vert\!\underset{\!\!\!\!\!#1}{\!\!\!\relbar\!\!\!%
\relbar\!\!\relbar\!\!\relbar\!\!\!\relbar\!\!\relbar\!\!\relbar\!\!\!%
\relbar\!\!\not\relbar\!\!\relbar}}
\newcommand{\forcein}[2]{\overset{#2}{\Vert\underset{\!\!\!\!\!#1}%
{\!\!\!\relbar\!\!\!\relbar\!\!\relbar\!\!\relbar\!\!\!\relbar\!\!\relbar\!%
\!\relbar\!\!\!\relbar\!\!\relbar\!\!\relbar\!\!\relbar\!\!\!\relbar\!\!%
\relbar\!\!\relbar}}}

\newcommand{\pre}[2]{{}^{#2}{#1}}

\newcommand{\restr}{\!\!\upharpoonright\!}

%%%%%%%%%%%%%%%%%%%%%%%%%%%%%%%%%%%%%%%%%%%%%
% Set-Theoretic Connectives
%%%%%%%%%%%%%%%%%%%%%%%%%%%%%%%%%%%%%%%%%%%%%

\newcommand{\intersect}{\cap}
\newcommand{\union}{\cup}
\newcommand{\Intersection}[1]{\bigcap\limits_{#1}}
\newcommand{\Union}[1]{\bigcup\limits_{#1}}
\newcommand{\adjoin}{{}^\frown}
\newcommand{\forces}{\Vdash}

%%%%%%%%%%%%%%%%%%%%%%%%%%%%%%%%%%%%%%%%%%%%%
% Miscellaneous
%%%%%%%%%%%%%%%%%%%%%%%%%%%%%%%%%%%%%%%%%%%%%
\newcommand{\defeq}{=_{\text{def}}}
\newcommand{\Turingleq}{\leq_{\text{T}}}
\newcommand{\Turingless}{<_{\text{T}}}
\newcommand{\lexleq}{\leq_{\text{lex}}}
\newcommand{\lexless}{<_{\text{lex}}}
\newcommand{\Turingequiv}{\equiv_{\text{T}}}
\newcommand{\isomorphic}{\cong}

%%%%%%%%%%%%%%%%%%%%%%%%%%%%%%%%%%%%%%%%%%%%%
% Constants
%%%%%%%%%%%%%%%%%%%%%%%%%%%%%%%%%%%%%%%%%%%%%
\newcommand{\R}{\mathbb{R}}
\renewcommand{\P}{\mathbb{P}}
\newcommand{\Q}{\mathbb{Q}}
\newcommand{\Z}{\mathbb{Z}}
\newcommand{\Zpos}{\Z^{+}}
\newcommand{\Znonneg}{\Z^{\geq 0}}
\newcommand{\C}{\mathbb{C}}
\newcommand{\N}{\mathbb{N}}
\newcommand{\B}{\mathbb{B}}
\newcommand{\Bairespace}{\pre{\omega}{\omega}}
\newcommand{\LofR}{L(\R)}
\newcommand{\JofR}[1]{J_{#1}(\R)}
\newcommand{\SofR}[1]{S_{#1}(\R)}
\newcommand{\JalphaR}{\JofR{\alpha}}
\newcommand{\JbetaR}{\JofR{\beta}}
\newcommand{\JlambdaR}{\JofR{\lambda}}
\newcommand{\SalphaR}{\SofR{\alpha}}
\newcommand{\SbetaR}{\SofR{\beta}}
\newcommand{\Pkl}{\mathcal{P}_{\kappa}(\lambda)}
\DeclareMathOperator{\con}{con}
\DeclareMathOperator{\ORD}{OR}
\DeclareMathOperator{\Ord}{OR}
\DeclareMathOperator{\WO}{WO}
\DeclareMathOperator{\OD}{OD}
\DeclareMathOperator{\HOD}{HOD}
\DeclareMathOperator{\HC}{HC}
\DeclareMathOperator{\WF}{WF}
\DeclareMathOperator{\wfp}{wfp}
\DeclareMathOperator{\HF}{HF}
\newcommand{\One}{I}
\newcommand{\ONE}{I}
\newcommand{\Two}{II}
\newcommand{\TWO}{II}
\newcommand{\Mladder}{M^{\text{ld}}}

%%%%%%%%%%%%%%%%%%%%%%%%%%%%%%%%%%%%%%%%%%%%%
% Commutative Algebra Constants
%%%%%%%%%%%%%%%%%%%%%%%%%%%%%%%%%%%%%%%%%%%%%
\DeclareMathOperator{\dottimes}{\dot{\times}}
\DeclareMathOperator{\dotminus}{\dot{-}}
\DeclareMathOperator{\Spec}{Spec}

%%%%%%%%%%%%%%%%%%%%%%%%%%%%%%%%%%%%%%%%%%%%%
% Theories
%%%%%%%%%%%%%%%%%%%%%%%%%%%%%%%%%%%%%%%%%%%%%
\DeclareMathOperator{\ZFC}{ZFC}
\DeclareMathOperator{\ZF}{ZF}
\DeclareMathOperator{\AD}{AD}
\DeclareMathOperator{\ADR}{AD_{\R}}
\DeclareMathOperator{\KP}{KP}
\DeclareMathOperator{\PD}{PD}
\DeclareMathOperator{\CH}{CH}
\DeclareMathOperator{\GCH}{GCH}
\DeclareMathOperator{\HPC}{HPC} % HOD pair capturing
%%%%%%%%%%%%%%%%%%%%%%%%%%%%%%%%%%%%%%%%%%%%%
% Iteration Trees
%%%%%%%%%%%%%%%%%%%%%%%%%%%%%%%%%%%%%%%%%%%%%

\newcommand{\pred}{\text{-pred}}

%%%%%%%%%%%%%%%%%%%%%%%%%%%%%%%%%%%%%%%%%%%%%%%%
% Operator Names
%%%%%%%%%%%%%%%%%%%%%%%%%%%%%%%%%%%%%%%%%%%%%%%%
\DeclareMathOperator{\Det}{Det}
\DeclareMathOperator{\dom}{dom}
\DeclareMathOperator{\ran}{ran}
\DeclareMathOperator{\range}{ran}
\DeclareMathOperator{\image}{image}
\DeclareMathOperator{\crit}{crit}
\DeclareMathOperator{\card}{card}
\DeclareMathOperator{\cf}{cf}
\DeclareMathOperator{\cof}{cof}
\DeclareMathOperator{\rank}{rank}
\DeclareMathOperator{\ot}{o.t.}
\DeclareMathOperator{\ords}{o}
\DeclareMathOperator{\ro}{r.o.}
\DeclareMathOperator{\rud}{rud}
\DeclareMathOperator{\Powerset}{\mathcal{P}}
\DeclareMathOperator{\length}{lh}
\DeclareMathOperator{\lh}{lh}
\DeclareMathOperator{\limit}{lim}
\DeclareMathOperator{\fld}{fld}
\DeclareMathOperator{\projection}{p}
\DeclareMathOperator{\Ult}{Ult}
\DeclareMathOperator{\ULT}{Ult}
\DeclareMathOperator{\Coll}{Coll}
\DeclareMathOperator{\Col}{Col}
\DeclareMathOperator{\Hull}{Hull}
\DeclareMathOperator*{\dirlim}{dir lim}
\DeclareMathOperator{\Scale}{Scale}
\DeclareMathOperator{\supp}{supp}
\DeclareMathOperator{\trancl}{tran.cl.}
\DeclareMathOperator{\trace}{Tr}
\DeclareMathOperator{\diag}{diag}
\DeclareMathOperator{\spn}{span}
\DeclareMathOperator{\sgn}{sgn}
%%%%%%%%%%%%%%%%%%%%%%%%%%%%%%%%%%%%%%%%%%%%%
% Logical Connectives
%%%%%%%%%%%%%%%%%%%%%%%%%%%%%%%%%%%%%%%%%%%%%
\newcommand{\IImplies}{\Longrightarrow}
\newcommand{\SkipImplies}{\quad\Longrightarrow\quad}
\newcommand{\Ifff}{\Longleftrightarrow}
\newcommand{\iimplies}{\longrightarrow}
\newcommand{\ifff}{\longleftrightarrow}
\newcommand{\Implies}{\Rightarrow}
\newcommand{\Iff}{\Leftrightarrow}
\renewcommand{\implies}{\rightarrow}
\renewcommand{\iff}{\leftrightarrow}
\newcommand{\AND}{\wedge}
\newcommand{\OR}{\vee}
\newcommand{\st}{\text{ s.t. }}
\newcommand{\Or}{\text{ or }}

%%%%%%%%%%%%%%%%%%%%%%%%%%%%%%%%%%%%%%%%%%%%%
% Function Arrows
%%%%%%%%%%%%%%%%%%%%%%%%%%%%%%%%%%%%%%%%%%%%%

\newcommand{\injection}{\xrightarrow{\text{1-1}}}
\newcommand{\surjection}{\xrightarrow{\text{onto}}}
\newcommand{\bijection}{\xrightarrow[\text{onto}]{\text{1-1}}}
\newcommand{\cofmap}{\xrightarrow{\text{cof}}}
\newcommand{\map}{\rightarrow}

%%%%%%%%%%%%%%%%%%%%%%%%%%%%%%%%%%%%%%%%%%%%%
% Mouse Comparison Operators
%%%%%%%%%%%%%%%%%%%%%%%%%%%%%%%%%%%%%%%%%%%%%
\newcommand{\initseg}{\trianglelefteq}
\newcommand{\properseg}{\lhd}
\newcommand{\notinitseg}{\ntrianglelefteq}
\newcommand{\notproperseg}{\ntriangleleft}

%%%%%%%%%%%%%%%%%%%%%%%%%%%%%%%%%%%%%%%%%%%%%
%           calligraphic letters
%%%%%%%%%%%%%%%%%%%%%%%%%%%%%%%%%%%%%%%%%%%%%
\newcommand{\cA}{\mathcal{A}}
\newcommand{\cB}{\mathcal{B}}
\newcommand{\cC}{\mathcal{C}}
\newcommand{\cD}{\mathcal{D}}
\newcommand{\cE}{\mathcal{E}}
\newcommand{\cF}{\mathcal{F}}
\newcommand{\cG}{\mathcal{G}}
\newcommand{\cH}{\mathcal{H}}
\newcommand{\cI}{\mathcal{I}}
\newcommand{\cJ}{\mathcal{J}}
\newcommand{\cK}{\mathcal{K}}
\newcommand{\cL}{\mathcal{L}}
\newcommand{\cM}{\mathcal{M}}
\newcommand{\cN}{\mathcal{N}}
\newcommand{\cO}{\mathcal{O}}
\newcommand{\cP}{\mathcal{P}}
\newcommand{\cQ}{\mathcal{Q}}
\newcommand{\cR}{\mathcal{R}}
\newcommand{\cS}{\mathcal{S}}
\newcommand{\cT}{\mathcal{T}}
\newcommand{\cU}{\mathcal{U}}
\newcommand{\cV}{\mathcal{V}}
\newcommand{\cW}{\mathcal{W}}
\newcommand{\cX}{\mathcal{X}}
\newcommand{\cY}{\mathcal{Y}}
\newcommand{\cZ}{\mathcal{Z}}


%%%%%%%%%%%%%%%%%%%%%%%%%%%%%%%%%%%%%%%%%%%%%
%          Primed Letters
%%%%%%%%%%%%%%%%%%%%%%%%%%%%%%%%%%%%%%%%%%%%%
\newcommand{\aprime}{a^{\prime}}
\newcommand{\bprime}{b^{\prime}}
\newcommand{\cprime}{c^{\prime}}
\newcommand{\dprime}{d^{\prime}}
\newcommand{\eprime}{e^{\prime}}
\newcommand{\fprime}{f^{\prime}}
\newcommand{\gprime}{g^{\prime}}
\newcommand{\hprime}{h^{\prime}}
\newcommand{\iprime}{i^{\prime}}
\newcommand{\jprime}{j^{\prime}}
\newcommand{\kprime}{k^{\prime}}
\newcommand{\lprime}{l^{\prime}}
\newcommand{\mprime}{m^{\prime}}
\newcommand{\nprime}{n^{\prime}}
\newcommand{\oprime}{o^{\prime}}
\newcommand{\pprime}{p^{\prime}}
\newcommand{\qprime}{q^{\prime}}
\newcommand{\rprime}{r^{\prime}}
\newcommand{\sprime}{s^{\prime}}
\newcommand{\tprime}{t^{\prime}}
\newcommand{\uprime}{u^{\prime}}
\newcommand{\vprime}{v^{\prime}}
\newcommand{\wprime}{w^{\prime}}
\newcommand{\xprime}{x^{\prime}}
\newcommand{\yprime}{y^{\prime}}
\newcommand{\zprime}{z^{\prime}}
\newcommand{\Aprime}{A^{\prime}}
\newcommand{\Bprime}{B^{\prime}}
\newcommand{\Cprime}{C^{\prime}}
\newcommand{\Dprime}{D^{\prime}}
\newcommand{\Eprime}{E^{\prime}}
\newcommand{\Fprime}{F^{\prime}}
\newcommand{\Gprime}{G^{\prime}}
\newcommand{\Hprime}{H^{\prime}}
\newcommand{\Iprime}{I^{\prime}}
\newcommand{\Jprime}{J^{\prime}}
\newcommand{\Kprime}{K^{\prime}}
\newcommand{\Lprime}{L^{\prime}}
\newcommand{\Mprime}{M^{\prime}}
\newcommand{\Nprime}{N^{\prime}}
\newcommand{\Oprime}{O^{\prime}}
\newcommand{\Pprime}{P^{\prime}}
\newcommand{\Qprime}{Q^{\prime}}
\newcommand{\Rprime}{R^{\prime}}
\newcommand{\Sprime}{S^{\prime}}
\newcommand{\Tprime}{T^{\prime}}
\newcommand{\Uprime}{U^{\prime}}
\newcommand{\Vprime}{V^{\prime}}
\newcommand{\Wprime}{W^{\prime}}
\newcommand{\Xprime}{X^{\prime}}
\newcommand{\Yprime}{Y^{\prime}}
\newcommand{\Zprime}{Z^{\prime}}
\newcommand{\alphaprime}{\alpha^{\prime}}
\newcommand{\betaprime}{\beta^{\prime}}
\newcommand{\gammaprime}{\gamma^{\prime}}
\newcommand{\Gammaprime}{\Gamma^{\prime}}
\newcommand{\deltaprime}{\delta^{\prime}}
\newcommand{\epsilonprime}{\epsilon^{\prime}}
\newcommand{\kappaprime}{\kappa^{\prime}}
\newcommand{\lambdaprime}{\lambda^{\prime}}
\newcommand{\rhoprime}{\rho^{\prime}}
\newcommand{\Sigmaprime}{\Sigma^{\prime}}
\newcommand{\tauprime}{\tau^{\prime}}
\newcommand{\xiprime}{\xi^{\prime}}
\newcommand{\thetaprime}{\theta^{\prime}}
\newcommand{\Omegaprime}{\Omega^{\prime}}
\newcommand{\cMprime}{\cM^{\prime}}
\newcommand{\cNprime}{\cN^{\prime}}
\newcommand{\cPprime}{\cP^{\prime}}
\newcommand{\cQprime}{\cQ^{\prime}}
\newcommand{\cRprime}{\cR^{\prime}}
\newcommand{\cSprime}{\cS^{\prime}}
\newcommand{\cTprime}{\cT^{\prime}}

%%%%%%%%%%%%%%%%%%%%%%%%%%%%%%%%%%%%%%%%%%%%%
%          bar Letters
%%%%%%%%%%%%%%%%%%%%%%%%%%%%%%%%%%%%%%%%%%%%%
\newcommand{\abar}{\bar{a}}
\newcommand{\bbar}{\bar{b}}
\newcommand{\cbar}{\bar{c}}
\newcommand{\ibar}{\bar{i}}
\newcommand{\jbar}{\bar{j}}
\newcommand{\nbar}{\bar{n}}
\newcommand{\xbar}{\bar{x}}
\newcommand{\ybar}{\bar{y}}
\newcommand{\zbar}{\bar{z}}
\newcommand{\pibar}{\bar{\pi}}
\newcommand{\phibar}{\bar{\varphi}}
\newcommand{\psibar}{\bar{\psi}}
\newcommand{\thetabar}{\bar{\theta}}
\newcommand{\nubar}{\bar{\nu}}

%%%%%%%%%%%%%%%%%%%%%%%%%%%%%%%%%%%%%%%%%%%%%
%          star Letters
%%%%%%%%%%%%%%%%%%%%%%%%%%%%%%%%%%%%%%%%%%%%%
\newcommand{\phistar}{\phi^{*}}
\newcommand{\Mstar}{M^{*}}

%%%%%%%%%%%%%%%%%%%%%%%%%%%%%%%%%%%%%%%%%%%%%
%          dotletters Letters
%%%%%%%%%%%%%%%%%%%%%%%%%%%%%%%%%%%%%%%%%%%%%
\newcommand{\Gdot}{\dot{G}}

%%%%%%%%%%%%%%%%%%%%%%%%%%%%%%%%%%%%%%%%%%%%%
%         check Letters
%%%%%%%%%%%%%%%%%%%%%%%%%%%%%%%%%%%%%%%%%%%%%
\newcommand{\deltacheck}{\check{\delta}}
\newcommand{\gammacheck}{\check{\gamma}}


%%%%%%%%%%%%%%%%%%%%%%%%%%%%%%%%%%%%%%%%%%%%%
%          Formulas
%%%%%%%%%%%%%%%%%%%%%%%%%%%%%%%%%%%%%%%%%%%%%

\newcommand{\formulaphi}{\text{\large $\varphi$}}
\newcommand{\Formulaphi}{\text{\Large $\varphi$}}


%%%%%%%%%%%%%%%%%%%%%%%%%%%%%%%%%%%%%%%%%%%%%
%          Fraktur Letters
%%%%%%%%%%%%%%%%%%%%%%%%%%%%%%%%%%%%%%%%%%%%%

\newcommand{\fa}{\mathfrak{a}}
\newcommand{\fb}{\mathfrak{b}}
\newcommand{\fc}{\mathfrak{c}}
\newcommand{\fk}{\mathfrak{k}}
\newcommand{\fp}{\mathfrak{p}}
\newcommand{\fq}{\mathfrak{q}}
\newcommand{\fr}{\mathfrak{r}}
\newcommand{\fA}{\mathfrak{A}}
\newcommand{\fB}{\mathfrak{B}}
\newcommand{\fC}{\mathfrak{C}}
\newcommand{\fD}{\mathfrak{D}}

%%%%%%%%%%%%%%%%%%%%%%%%%%%%%%%%%%%%%%%%%%%%%
%          Bold Letters
%%%%%%%%%%%%%%%%%%%%%%%%%%%%%%%%%%%%%%%%%%%%%
\newcommand{\ba}{\mathbf{a}}
\newcommand{\bb}{\mathbf{b}}
\newcommand{\bc}{\mathbf{c}}
\newcommand{\bd}{\mathbf{d}}
\newcommand{\be}{\mathbf{e}}
\newcommand{\bu}{\mathbf{u}}
\newcommand{\bv}{\mathbf{v}}
\newcommand{\bw}{\mathbf{w}}
\newcommand{\bx}{\mathbf{x}}
\newcommand{\by}{\mathbf{y}}
\newcommand{\bz}{\mathbf{z}}
\newcommand{\bSigma}{\boldsymbol{\Sigma}}
\newcommand{\bPi}{\boldsymbol{\Pi}}
\newcommand{\bDelta}{\boldsymbol{\Delta}}
\newcommand{\bdelta}{\boldsymbol{\delta}}
\newcommand{\bgamma}{\boldsymbol{\gamma}}
\newcommand{\bGamma}{\boldsymbol{\Gamma}}

%%%%%%%%%%%%%%%%%%%%%%%%%%%%%%%%%%%%%%%%%%%%%
%         Bold numbers
%%%%%%%%%%%%%%%%%%%%%%%%%%%%%%%%%%%%%%%%%%%%%
\newcommand{\bzero}{\mathbf{0}}

%%%%%%%%%%%%%%%%%%%%%%%%%%%%%%%%%%%%%%%%%%%%%
% Projective-Like Pointclasses
%%%%%%%%%%%%%%%%%%%%%%%%%%%%%%%%%%%%%%%%%%%%%
\newcommand{\Sa}[2][\alpha]{\Sigma_{(#1,#2)}}
\newcommand{\Pa}[2][\alpha]{\Pi_{(#1,#2)}}
\newcommand{\Da}[2][\alpha]{\Delta_{(#1,#2)}}
\newcommand{\Aa}[2][\alpha]{A_{(#1,#2)}}
\newcommand{\Ca}[2][\alpha]{C_{(#1,#2)}}
\newcommand{\Qa}[2][\alpha]{Q_{(#1,#2)}}
\newcommand{\da}[2][\alpha]{\delta_{(#1,#2)}}
\newcommand{\leqa}[2][\alpha]{\leq_{(#1,#2)}}
\newcommand{\lessa}[2][\alpha]{<_{(#1,#2)}}
\newcommand{\equiva}[2][\alpha]{\equiv_{(#1,#2)}}


\newcommand{\Sl}[1]{\Sa[\lambda]{#1}}
\newcommand{\Pl}[1]{\Pa[\lambda]{#1}}
\newcommand{\Dl}[1]{\Da[\lambda]{#1}}
\newcommand{\Al}[1]{\Aa[\lambda]{#1}}
\newcommand{\Cl}[1]{\Ca[\lambda]{#1}}
\newcommand{\Ql}[1]{\Qa[\lambda]{#1}}

\newcommand{\San}{\Sa{n}}
\newcommand{\Pan}{\Pa{n}}
\newcommand{\Dan}{\Da{n}}
\newcommand{\Can}{\Ca{n}}
\newcommand{\Qan}{\Qa{n}}
\newcommand{\Aan}{\Aa{n}}
\newcommand{\dan}{\da{n}}
\newcommand{\leqan}{\leqa{n}}
\newcommand{\lessan}{\lessa{n}}
\newcommand{\equivan}{\equiva{n}}

\newcommand{\SigmaOneOmega}{\Sigma^1_{\omega}}
\newcommand{\SigmaOneOmegaPlusOne}{\Sigma^1_{\omega+1}}
\newcommand{\PiOneOmega}{\Pi^1_{\omega}}
\newcommand{\PiOneOmegaPlusOne}{\Pi^1_{\omega+1}}
\newcommand{\DeltaOneOmegaPlusOne}{\Delta^1_{\omega+1}}

%%%%%%%%%%%%%%%%%%%%%%%%%%%%%%%%%%%%%%%%%%%%%
% Linear Algebra
%%%%%%%%%%%%%%%%%%%%%%%%%%%%%%%%%%%%%%%%%%%%%
\newcommand{\transpose}[1]{{#1}^{\text{T}}}
\newcommand{\norm}[1]{\lVert{#1}\rVert}
\newcommand\aug{\fboxsep=-\fboxrule\!\!\!\fbox{\strut}\!\!\!}

%%%%%%%%%%%%%%%%%%%%%%%%%%%%%%%%%%%%%%%%%%%%%
% Number Theory
%%%%%%%%%%%%%%%%%%%%%%%%%%%%%%%%%%%%%%%%%%%%%
\newcommand{\av}[1]{\lvert#1\rvert}
\DeclareMathOperator{\divides}{\mid}
\DeclareMathOperator{\ndivides}{\nmid}
\DeclareMathOperator{\lcm}{lcm}
\DeclareMathOperator{\sign}{sign}
\newcommand{\floor}[1]{\left\lfloor{#1}\right\rfloor}
\DeclareMathOperator{\ConCl}{CC}
\DeclareMathOperator{\ord}{ord}


%%%%%%%%%%%%%%%%%%%%%%%%%%%%%%%%%%%%%%%%%%%%%%%%%%%%%%%%%%%%%%%%%%%%%%%%%%%
%%  Theorem-Like Declarations
%%%%%%%%%%%%%%%%%%%%%%%%%%%%%%%%%%%%%%%%%%%%%%%%%%%%%%%%%%%%%%%%%%%%%%%%%%

\newtheorem{theorem}{Theorem}[section]
\newtheorem{lemma}[theorem]{Lemma}
\newtheorem{corollary}[theorem]{Corollary}
\newtheorem{proposition}[theorem]{Proposition}


\theoremstyle{definition}

\newtheorem{definition}[theorem]{Definition}
\newtheorem{conjecture}[theorem]{Conjecture}
\newtheorem{remark}[theorem]{Remark}
\newtheorem{remarks}[theorem]{Remarks}
\newtheorem{notation}[theorem]{Notation}

\theoremstyle{remark}

\newtheorem*{note}{Note}
\newtheorem*{warning}{Warning}
\newtheorem*{question}{Question}
\newtheorem*{example}{Example}
\newtheorem*{fact}{Fact}


\newenvironment*{subproof}[1][Proof]
{\begin{proof}[#1]}{\renewcommand{\qedsymbol}{$\diamondsuit$} \end{proof}}

\newenvironment*{case}[1]
{\textbf{Case #1.  }\itshape }{}

\newenvironment*{claim}[1][Claim]
{\textbf{#1.  }\itshape }{}


\pagestyle{plain}

\begin{document}

\title{Mouse Pairs and Suslin Cardinals in a Type I Hierarchy}
\author{John R. Steel}

\maketitle

\tableofcontents

We assume $\ADR$ and $\HPC$. Following Jackson \cite{Jackson1}:

Let $\lambda_1$ be a limit of Suslin cardinals, and $\cf(\lambda_1)=\omega$.
Put $S_{<\lambda_1} =\Union{\alpha<\lambda_1} S_{\alpha}$, and
\begin{align*}
\bSigma_0 &= \Union{\omega} S_{<\lambda_1} \\
\bPi_0 &= \Intersection{\omega} S_{<\lambda_1} \
\end{align*}

and

\begin{align*}
\bSigma_{n+1} &= \exists^{\R}\bPi_n \textbf{,\quad}
\bPi_{n+1} = \forall^{\R}\bSigma_n \\
\bdelta_n&=\sup\setof{|R|}{R\text{ is a }\bDelta_n\text{ pwo of } \R}
\end{align*}

for $n\geq 1$. [Rudominer: Should this be $n\geq 0$?]
We have $\Scale(\bSigma_{2n+2})$ and $\Scale(\bPi_{2n+1})$,
for all $n$, with norms into $\bdelta_{2n+1}$. We also have

$$ \bdelta_{2n+1} = \lambda_{2n+1}^+,$$
where $\lambda_{2n+1}$ is a Suslin cardinal of cofinality $\omega$.
(So $\bdelta_1=\lambda_1^+$.) The sequence $\lambda_1,\bdelta_1,\lambda_3,\bdelta_3,\cdots$
enumerates the Suslin cardinals in this hierarchy. For $\kappa=\bdelta_{2n+1}$,
$S_{\kappa}=\bSigma_{2n+2}$. For $\kappa=\lambda_{2n+1}$,
$S_{\kappa}=\bSigma_{2n+1}$.

If $\lambda_1=\omega$, then $\bSigma_n = \bSigma^1_n$ (for $n\geq 1$) in  the
ordinary projective hierarchy. In that case, we have the envelopes

$$
\bGamma^1_{k} = \Union{n<\omega} \Game^{k}(\omega n - \bPi^1_1)
$$

for $k\geq0$. There are self-justifying systems sealing those envelopes. For $k$ even, this gives us an optimal scale on $\bSigma^1_{k+1}$, and for $k$ odd, an optimal scale on $\bPi^1_{k+1}$. By results of
Hjorth, Woodin and Sargsyan (See \cite{Sargsyan1}),
$$
\lambda^1_{2k+1} = \sup\setof{|\leq|}{\leq\text{ is a $\bGamma^1_{2k-1}$ pwo of $\R$}}.
$$

The $\bGamma^1_{k}$ are related to the mice $M_n = $ minimal model with $n$ Woodins, in various ways. For example, let

$$
T^n_{k+1}(\varphi,x) \Ifff M_k \models \varphi\left[ x,\omega_1,\cdots,\omega_n\right],
$$
then each $T^n_{k+1}$ is in $\bGamma^1_{k+1}$, and the  $T^n_{k+1}$ are Wadge cofinal in
$\bGamma^1_{k+1}$.

Our goal is to generalize this theory to the type I hierarchy $\bSigma_n,\bPi_n$ we have
fixed above. The main problem is how to define the analogs of $\bGamma^1_{0}$ and
$M_0=L$. This is a problem even when we are considering the second projective-like hierarchy,
$\bSigma_n= \bSigma_{n+1}^{J_2(\R)}$. (I.e. $\lambda_1=\bdelta^1_{\omega}$.)
Hugh Woodin recently solved this problem in its mouse set incarnation, by showing that the reals in Rudominer's ``ladder mouse'' are precisely the reals that are $\Delta_2^{J_2(\R)}$ in some
$\alpha<\omega_1$. (It turns out that $M^{\text{ld}}$ is somewhat more analogous to $M_1$.)
We show here that Woodin's solution to the Rudominer problem adapts (easily) to the general type I hierarchy above (assuming HPC). We then show that it yields good analogues of the $\bGamma^1_n$'s
and $M_n$'s.

As a corollary we obtain

\begin{theorem} ($\AD^+$) Let $\kappa$ be a Suslin cardinal, and $(P,\Sigma)$ a
mouse pair such that $\kappa<\ords\left( M_{\infty}(P,\Sigma)  \right)$.
Suppose $\cf(\lambda) = \omega$, where $\lambda$ is the largest limit of Suslin
cardinals $\leq \kappa$.

Then $\kappa$ is a cutpoint of $M_{\infty}(P,\Sigma)$.
\end{theorem}


\begin{remark}
It is shown in \cite{Steel1} that if $\kappa$ is a cardinal and a cutpoint of
$M_{\infty}(P,\Sigma)$ for some mouse pair $(P,\Sigma)$, then $\kappa$ is Suslin.
\end{remark}

\begin{corollary}
Assume $\AD_{\R} + \HPC$, and let $\kappa$ be a Suslin cardinal. Suppose
$\cf(\lambda)=\omega$, where $\lambda$ is the largest limit of Suslin cardinals
$\leq \kappa$; then $\kappa$ is a cutpoint of the $\HOD$-sequence.
\end{corollary}

\begin{remark}
It is shown in \cite{Steel1} that the theorem and corollary hold when $\kappa$
is a limit of Suslins, or the next Suslin after a limit of Suslins.
We of course conjecture that they hold for any Suslin cardinal $\kappa$.
\end{remark}

\section{Ladder Mice for $J_2(\R)$}

Woodin's work characterizing $\Delta_2^{J_2(\R)}$-in-$\alpha$ as a mouse set is
unpublished, so we devote this section to describing it.

The ladder mouse $M^{\text{ld}}$ was defined by Rudominer
(\cite{My_Thesis} and \cite{Mouse_Sets} ).

\begin{definition}
$M^{\text{ld}}$ is the minimal iterable pure extender mouse such that for each $k<\omega$,
 there is a $\delta<\ords(M^{\text{ld}})$ such that
\begin{enumerate}[(a)]
\item $M_k(M^{\text{ld}} \vert \delta) \properseg M^{\text{ld}} $, and
\item $M_k(M^{\text{ld}} \vert \delta) \models \delta$ is Woodin, and
\item $M^{\text{ld}} \models \delta$ is a cardinal.
\end{enumerate}

Here $M_k(N)$ is the minimal mouse with $k$ Woodins over $N$. Let $\delta_k$
be the least $\delta$ as in (a) - (c); then $\delta_k<\delta_{k+1}$, and
$\ords(M^{\text{ld}}) = \sup_k \delta_k$. These are the rungs of our ladder.

\end{definition}

Rudominer showed in \cite{My_Thesis} that every real belonging to
$M^{\text{ld}}$ is $\Delta_2^{J_2(\R)}$ in some $\alpha<\omega_1$.

For let $N$ be the $\alpha^{\text{th}}$ level of $M^{\text{ld}}$ that projects to
$\omega$. Say that a premouse $P$ is  $\Pi_1$-iterable iff $\forall \cT$ on $P$,
$\forall k < \omega$,  $\exists$ maximal branch $b$ of
$\cT$ s.t. $ M_k(M(\cT))\initseg M_b^{\cT}$ or $M_b^{\cT}\initseg M_k(M(\cT))$.
This property is $\Pi_1^{J_2(\R)}$.

\begin{aside}
That is what we called $\Pi_0$ in the general setting. It corresponds to infinite recursive
conjunctions of analytical predicates. [Rudominer: Then why bother calling
it $\Pi_1$-iterable here? Why not call it either $\Pi_1^{J_2(\R)}$-iterable
or $\Pi_0$-iterable.]
\end{aside}

One can show that $N$ is the unique $\Pi_1$-iterable premouse that thinks it is
the $\alpha^{\text{th}}$ level projecting to $\omega$. So if $x\in{}^{\omega}\omega$
codes the theory of $N$, the $\singleton{x}$ is $\Pi_1^{J_2(\R)}$ in $\alpha$,
hence $x$ is $\Delta_2^{J_2(\R)}$ in $\alpha$.

\begin{theorem}[Woodin, unpublished] Let $x$ be $\Delta_2^{J_2(\R)}$ in $\alpha<\omega_1$;
then $x\in M^{\text{ld}}$.
\end{theorem}

\begin{proof}
Let $x$ be $\Delta_2^{J_2(\R)}$ in $\alpha$. We assume $\alpha=0$ for simplicity.
(Otherwise, iterate the least measurable of $M^{\text{ld}}$ past $\alpha$, generating
$M^{\text{ld}}\to P$. The argument to come shows $x\in P[g]$ for all $g$
on $\Col(\omega,\alpha)$, hence $x\in P$, hence $x\in M^{\text{ld}}$.)
So we have recursive sequences $\angles{A_n}$ and $\angles{B_n}$ such that
for $n\geq 1$
\begin{enumerate}[(i)]
\item $A_n$ and $B_n$ are $\Sigma^1_{2n}$ subsets of $\R\times\omega$,
uniformly in $n$, and
\item for any $k\in\omega$
\begin{align*}
k\in x &\Iff \exists y \bigwedge_n A_n(y,k) \\
k\notin x &\Iff \exists y \bigwedge_n B_n(y,k). \\
\end{align*}
\end{enumerate}
(We assume $x\subseteq\omega$ now.)

Let $\phibar^n=\sequence{\varphi^n_i}{i\in\omega}$ be a $\Sigma^1_{2n}$
scale on $A_n$, and $\psibar^n=\sequence{\psi^n_i}{i\in\omega}$ be a $\Sigma^1_{2n}$
scale on $B_n$. Let $\leq^A_{n,i}$ be the prewellorder
$\leq_{\varphi^n_i}$, and $\leq^B_{n,i}=\leq_{\psi^n_i}$. Let
\begin{align*}
A_k(y) &\Iff  \bigwedge_n A_n(y,k) \\
B_k(y) &\Iff  \bigwedge_n B_n(y,k).
\end{align*}

We get scales $\thetabar^k$ and $\nubar^k$ on $A_k$ and $B_k$ via

\begin{align*}
\theta^k_{\angles{n,i}}(y) &= \varphi^n_i(y,k) \\
\nu^k_{\angles{n,i}}(y) &= \psi^n_i(y,k).
\end{align*}

Let $S^k\subseteq(\omega \times \bdelta^1_{\omega})^{<\omega}$ and
$T^k\subseteq(\omega \times \bdelta^1_{\omega})^{<\omega}$  be the trees of these scales,
so that

\begin{align*}
k\in x &\Iff \left[ S^k \right] \not= \emptyset \\
k\notin x &\Iff \left[ T^k \right] \not= \emptyset.
\end{align*}

Now we look inside $M^{\text{ld}}$. This model is projectively correct, so it
has the prewellorders $\leq^A_{n,i}$ and $\leq^B_{n,i}$ restricted to its own
reals. Also, for any $y\in M^{\text{ld}}$,

\begin{align*}
\left( M^{\text{ld}} \models A_k(y)  \right) &\Implies A_k(y) \text{, and} \\
\left( M^{\text{ld}} \models B_k(y)  \right) &\Implies B_k(y).
\end{align*}

[Rudominer: Replace $\Implies$ with $\Iff$ above?]

Let $S^k_{M}$ and $T^k_{M}$ be the trees of the scales
$\left( \thetabar^k \right)^{M^{\text{ld}}}$ and
$\left( \nubar^k \right)^{M^{\text{ld}}}$ computed in ${M^{\text{ld}}}$
using the pwo's $\leq^A_{n,i}\intersect M^{\text{ld}}$ and
$\leq^B_{n,i}\intersect M^{\text{ld}}$. These are isomorphic to subtress
of the full $S^k$ and $T^k$. The projections in $M^{\text{ld}}$ are
$A_k\intersect M^{\text{ld}}$ and
$B_k\intersect M^{\text{ld}}$ respectively. For each $k$, at least on of
$S^k_{M}$ and $T^k_{M}$ is wellfounded.

\begin{claim}[Claim 1]
For any $k\in\omega$, $k\in x$ iff $M^{\text{ld}} \models \text{`` }T^k_M$ is
wellfounded, and either $S^k_M$ is illfounded, or $|T^k_M|<|S^k_M|$.''
\end{claim}

Clearly Claim 1 implies $x\in M^{\text{ld}}$, as desired.

\begin{subproof}{(of Claim 1)}
It is enough to show $\Rightarrow$, since then we get $\Leftarrow$ by assuming
$k\notin x$ and using the same argument to show $S^k_M$ is wellfounded and
$|S^k_M| < |T^k_m|$.

So fix $k_0\in x$, and fix $y$ s.t. $A_{k_0}(y)$. Since $B_{k_0} = \emptyset$,
$T^{k_0}_M$ is wellfounded. We must see that $|T^{k_0}_M| < |S^{k_0}_M|$ or
$S^{k_0}_M$ is illfounded.

Let for any $k$

\begin{align*}
\delta_k &=  \text{ least cardinal } \delta \text{ of } M^{\text{ld}} \text{ s.t. }
    M_{2k+5}(M^{\text{ld}}|\delta) \initseg M^{\text{ld}} \text{ and } \\
N_k      &=  M_{2k+5}(M^{\text{ld}}|\delta_k)
\end{align*}
[Rudominer: Two problems with the above text: Firstly we have already defined
$\delta_k$ to be the $k$-th rung of the ladder. Secondly we forgot above to
say that $\delta$ needs to be locally Woodin.]

So $\delta_k$ is Woodin in $N_k$, which has $2k+5$ Woodins above it as well. The
$\delta_k$'s are cofinal in $\lambda=\ords(M^{\text{ld}})$. By iterating,
we may assume that for each $k$, there is an $N_k$-generic $g_k$ on
$\Col(\omega,\delta_k)$ such that

$$
y\in N_k[g_k].
$$

We caution that $g_k$ is only $N_k$-generic, not necessarily $M^{\text{ld}}$-generic,
because $\delta_k$ is only Woodin in $N_k$, not the full $M^{\text{ld}}$.
(But the relevant genericity iterations do move the whold $M^{\text{ld}}$ along,
because $\delta_k$ is a cardinal of $M^{\text{ld}}$. This is what motivated
Rudominer's definition in the first place.)

We now build a generic $G$ for the countable stationary tower $\Q^{M^{\text{ld}}}_{<\lambda}$
such that for all $k$, $G\intersect \Q_{<\delta_k}$ is $N_k$-generic. (Note that
$\Q^{M^{\text{ld}}}_{<\delta_k}=\Q^{N_k}_{<\delta_k}$ because $\delta_k$ is
a cardinal of $M^{\text{ld}}$. But $M^{\text{ld}}$ has more dense subsets of $\Q_{<\delta_k}$
than $N_k$ does.)

The following claim is what we need to build $G$. It comes out of the basics of stationary tower forcing.
See \cite{Larson_Book}.

\begin{claim}[Subclaim 2]
Working in $M^{\text{ld}}$: for any $a\in\Q_{<\delta_k}$, there is a $b\in\Q_{<\delta_{k+1}}$
such that $b\leq a$ in $Q_{<\delta_{k+1}}$, and

$$
b\forces \dot{G} \intersect \Q_{<\delta_k} \text{ is } N_k\text{-generic}.
$$

\end{claim}

\begin{subproof}{(of Subclaim 2)}

Let $X\prec M^{\text{ld}} | \delta_k^+$ be countable and $D$ be a maximal antichain of $\Q_{<\delta_k}$.
(We are working in $M^{\text{ld}}$ now.) Recall that $X$ \emph{captures} $D$ iff
$\exists c \in X\intersect D \left (X\intersect \Union{}c \in c  \right)$.

Let

\begin{align*}
b = \big\{ \, &X\prec M^{\text{ld}} | \delta_k^+ \, \big| \, X \text{ is countable and } X\intersect \Union{}a \in a \\
              &\text{and for every maximal antichain $D$ s.t. $D\in X\intersect N_k$, $X$ captures } D \, \big\}.
\end{align*}

The proofs of \cite{Larson_Book} show easily that $b$ is stationary. (The main point is that for each $D\in N_k$,
there are arbitrarily large inaccessible $\gamma<\delta_k$ such that $D\intersect\Q_{<\gamma}$ is maximal
and semiproper. This is where $N_k\models \delta_k \text{ is Woodin}$ gets used. But then given any structure $\fA$
on $\left(M^{\text{ld}} \vert \delta_k^+ \right)$, we get $X\prec\fA$ s.t. $X\in b$ as follows:
let $X_0\prec\fA$ with $X_0\intersect \Union{}a\in a$.
Pick some $D_0\in X_0\intersect N_k$ that is maximal.
We have $\gamma_0>\ords(\Union{}a)$ s.t. $\gamma_0\in X_0$ and $D_0\intersect\Q_{<\gamma_0}$
is semiproper. This implies we can end extend $X_0$ below $\gamma_0$ to $X_1\prec\fA$ s.t. $X_1$ captures
$D_0\intersect\Q_{<\gamma_0}$, hence captures $D_0$. Now pick a $D_1\in X_1\intersect N_k$ that is maximal, and
end-extend $X_1\prec X_2$ s.t. $X_2$ captures $D_1$. End-extending means $X_2\intersect\gamma_0 = X_1\intersect \gamma_0$,
so $X_2$ still captures $D_1$. Etc. The desired $X$ is $\Union{n} X_n$.)

Again, \cite{Larson_Book} shows $b\forces \dot{G} \intersect \Q_{<\delta_k}$ is $N_k$-generic.

(Let $D\in N_k$ be a maximal antichain. If $c\leq b$, then
$$\forall X\in c \, \exists! \, d_X \in X\intersect D \, (X\intersect \Union{}d_X\in d_X).$$
Then $\exists e$ s.t. for stationary many
$X\in c$, $e=d_X$, say for $X\in c^*$. This means $c^*\leq e$, so
$c^*\forces \dot{G} \intersect D = \emptyset$.)

[Rudominer: Should this say $c^*\forces \dot{G} \intersect D \not= \emptyset$?]


\end{subproof}

\end{subproof}

\end{proof}

\bibliographystyle{amsalpha}

\bibliography{math}

\end{document}
