% $Header$

\documentclass{beamer}
%\documentclass[handout]{beamer}

\usepackage{amsmath,amssymb,latexsym,eucal,amsthm,graphicx}
%%%%%%%%%%%%%%%%%%%%%%%%%%%%%%%%%%%%%%%%%%%%%
% Common Set Theory Constructs
%%%%%%%%%%%%%%%%%%%%%%%%%%%%%%%%%%%%%%%%%%%%%

\newcommand{\setof}[2]{\left\{ \, #1 \, \left| \, #2 \, \right.\right\}}
\newcommand{\lsetof}[2]{\left\{\left. \, #1 \, \right| \, #2 \,  \right\}}
\newcommand{\bigsetof}[2]{\bigl\{ \, #1 \, \bigm | \, #2 \,\bigr\}}
\newcommand{\Bigsetof}[2]{\Bigl\{ \, #1 \, \Bigm | \, #2 \,\Bigr\}}
\newcommand{\biggsetof}[2]{\biggl\{ \, #1 \, \biggm | \, #2 \,\biggr\}}
\newcommand{\Biggsetof}[2]{\Biggl\{ \, #1 \, \Biggm | \, #2 \,\Biggr\}}
\newcommand{\dotsetof}[2]{\left\{ \, #1 \, : \, #2 \, \right\}}
\newcommand{\bigdotsetof}[2]{\bigl\{ \, #1 \, : \, #2 \,\bigr\}}
\newcommand{\Bigdotsetof}[2]{\Bigl\{ \, #1 \, \Bigm : \, #2 \,\Bigr\}}
\newcommand{\biggdotsetof}[2]{\biggl\{ \, #1 \, \biggm : \, #2 \,\biggr\}}
\newcommand{\Biggdotsetof}[2]{\Biggl\{ \, #1 \, \Biggm : \, #2 \,\Biggr\}}
\newcommand{\sequence}[2]{\left\langle \, #1 \,\left| \, #2 \, \right. \right\rangle}
\newcommand{\lsequence}[2]{\left\langle\left. \, #1 \, \right| \,#2 \,  \right\rangle}
\newcommand{\bigsequence}[2]{\bigl\langle \,#1 \, \bigm | \, #2 \, \bigr\rangle}
\newcommand{\Bigsequence}[2]{\Bigl\langle \,#1 \, \Bigm | \, #2 \, \Bigr\rangle}
\newcommand{\biggsequence}[2]{\biggl\langle \,#1 \, \biggm | \, #2 \, \biggr\rangle}
\newcommand{\Biggsequence}[2]{\Biggl\langle \,#1 \, \Biggm | \, #2 \, \Biggr\rangle}
\newcommand{\singleton}[1]{\left\{#1\right\}}
\newcommand{\angles}[1]{\left\langle #1 \right\rangle}
\newcommand{\bigangles}[1]{\bigl\langle #1 \bigr\rangle}
\newcommand{\Bigangles}[1]{\Bigl\langle #1 \Bigr\rangle}
\newcommand{\biggangles}[1]{\biggl\langle #1 \biggr\rangle}
\newcommand{\Biggangles}[1]{\Biggl\langle #1 \Biggr\rangle}


\newcommand{\force}[1]{\Vert\!\underset{\!\!\!\!\!#1}{\!\!\!\relbar\!\!\!%
\relbar\!\!\relbar\!\!\relbar\!\!\!\relbar\!\!\relbar\!\!\relbar\!\!\!%
\relbar\!\!\relbar\!\!\relbar}}
\newcommand{\nforce}[1]{\Vert\!\underset{\!\!\!\!\!#1}{\!\!\!\relbar\!\!\!%
\relbar\!\!\relbar\!\!\relbar\!\!\!\relbar\!\!\relbar\!\!\relbar\!\!\!%
\relbar\!\!\not\relbar\!\!\relbar}}
\newcommand{\forcein}[2]{\overset{#2}{\Vert\underset{\!\!\!\!\!#1}%
{\!\!\!\relbar\!\!\!\relbar\!\!\relbar\!\!\relbar\!\!\!\relbar\!\!\relbar\!%
\!\relbar\!\!\!\relbar\!\!\relbar\!\!\relbar\!\!\relbar\!\!\!\relbar\!\!%
\relbar\!\!\relbar}}}

\newcommand{\pre}[2]{{}^{#2}\!{#1}}

\newcommand{\restr}{\!\!\upharpoonright\!}

%%%%%%%%%%%%%%%%%%%%%%%%%%%%%%%%%%%%%%%%%%%%%
% Set-Theoretic Connectives
%%%%%%%%%%%%%%%%%%%%%%%%%%%%%%%%%%%%%%%%%%%%%

\newcommand{\intersect}{\cap}
\newcommand{\union}{\cup}
\newcommand{\Intersection}[1]{\bigcap\limits_{#1}}
\newcommand{\Union}[1]{\bigcup\limits_{#1}}
\newcommand{\adjoin}{{}^\frown}
\newcommand{\forces}{\Vdash}

%%%%%%%%%%%%%%%%%%%%%%%%%%%%%%%%%%%%%%%%%%%%%
% Miscellaneous
%%%%%%%%%%%%%%%%%%%%%%%%%%%%%%%%%%%%%%%%%%%%%
\newcommand{\defeq}{=_{\text{def}}}
\newcommand{\Turingleq}{\leq_{\text{T}}}
\newcommand{\Turingless}{<_{\text{T}}}
\newcommand{\lexleq}{\leq_{\text{lex}}}
\newcommand{\lexless}{<_{\text{lex}}}
\newcommand{\Turingequiv}{\equiv_{\text{T}}}

%%%%%%%%%%%%%%%%%%%%%%%%%%%%%%%%%%%%%%%%%%%%%
% Constants
%%%%%%%%%%%%%%%%%%%%%%%%%%%%%%%%%%%%%%%%%%%%%
\newcommand{\R}{\mathbb{R}}
\renewcommand{\P}{\mathbb{P}}
\newcommand{\Q}{\mathbb{Q}}
\newcommand{\Z}{\mathbb{Z}}
\newcommand{\C}{\mathbb{C}}
\newcommand{\N}{\mathbb{N}}
\newcommand{\B}{\mathbb{B}}
\newcommand{\LofR}{L(\R)}
\newcommand{\JofR}[1]{J_{#1}(\R)}
\newcommand{\SofR}[1]{S_{#1}(\R)}
\newcommand{\JalphaR}{\JofR{\alpha}}
\newcommand{\JbetaR}{\JofR{\beta}}
\newcommand{\JlambdaR}{\JofR{\lambda}}
\newcommand{\SalphaR}{\SofR{\alpha}}
\newcommand{\SbetaR}{\SofR{\beta}}
\newcommand{\Pkl}{\mathcal{P}_{\kappa}(\lambda)}
\DeclareMathOperator{\con}{con}
\DeclareMathOperator{\ORD}{OR}
\DeclareMathOperator{\Ord}{OR}
\DeclareMathOperator{\WO}{WO}
\DeclareMathOperator{\OD}{OD}
\DeclareMathOperator{\HOD}{HOD}
\DeclareMathOperator{\HC}{HC}
\DeclareMathOperator{\WF}{WF}
\DeclareMathOperator{\HF}{HF}
\newcommand{\One}{I}
\newcommand{\ONE}{I}
\newcommand{\Two}{II}
\newcommand{\TWO}{II}

%%%%%%%%%%%%%%%%%%%%%%%%%%%%%%%%%%%%%%%%%%%%%
% Commutative Algebra Constants
%%%%%%%%%%%%%%%%%%%%%%%%%%%%%%%%%%%%%%%%%%%%%
\DeclareMathOperator{\dottimes}{\dot{\times}}

%%%%%%%%%%%%%%%%%%%%%%%%%%%%%%%%%%%%%%%%%%%%%
% Theories
%%%%%%%%%%%%%%%%%%%%%%%%%%%%%%%%%%%%%%%%%%%%%
\DeclareMathOperator{\ZFC}{ZFC}
\DeclareMathOperator{\ZF}{ZF}
\DeclareMathOperator{\AD}{AD}
\DeclareMathOperator{\ADR}{AD_{\R}}
\DeclareMathOperator{\KP}{KP}
\DeclareMathOperator{\PD}{PD}
\DeclareMathOperator{\CH}{CH}
\DeclareMathOperator{\HPC}{HPC} % HOD pair capturing
%%%%%%%%%%%%%%%%%%%%%%%%%%%%%%%%%%%%%%%%%%%%%
% Iteration Trees
%%%%%%%%%%%%%%%%%%%%%%%%%%%%%%%%%%%%%%%%%%%%%

\newcommand{\pred}{\text{-pred}}

%%%%%%%%%%%%%%%%%%%%%%%%%%%%%%%%%%%%%%%%%%%%%%%%
% Operator Names
%%%%%%%%%%%%%%%%%%%%%%%%%%%%%%%%%%%%%%%%%%%%%%%%
\DeclareMathOperator{\Det}{Det}
\DeclareMathOperator{\dom}{dom}
\DeclareMathOperator{\ran}{ran}
\DeclareMathOperator{\range}{ran}
\DeclareMathOperator{\image}{image}
\DeclareMathOperator{\crit}{crit}
\DeclareMathOperator{\card}{card}
\DeclareMathOperator{\cf}{cf}
\DeclareMathOperator{\cof}{cof}
\DeclareMathOperator{\rank}{rank}
\DeclareMathOperator{\ot}{o.t.}
\DeclareMathOperator{\ords}{o}
\DeclareMathOperator{\ro}{r.o.}
\DeclareMathOperator{\rud}{rud}
\DeclareMathOperator{\Powerset}{\mathcal{P}}
\DeclareMathOperator{\length}{lh}
\DeclareMathOperator{\lh}{lh}
\DeclareMathOperator{\limit}{lim}
\DeclareMathOperator{\fld}{fld}
\DeclareMathOperator{\projection}{p}
\DeclareMathOperator{\Ult}{Ult}
\DeclareMathOperator{\ULT}{Ult}
\DeclareMathOperator{\Coll}{Coll}
\DeclareMathOperator{\Col}{Col}
\DeclareMathOperator{\Hull}{Hull}
\DeclareMathOperator*{\dirlim}{dir lim}
\DeclareMathOperator{\Scale}{Scale}
\DeclareMathOperator{\supp}{supp}
\DeclareMathOperator{\trancl}{tran.cl.}
\DeclareMathOperator{\trace}{Tr}
\DeclareMathOperator{\diag}{diag}
\DeclareMathOperator{\spn}{span}
\DeclareMathOperator{\sgn}{sgn}
%%%%%%%%%%%%%%%%%%%%%%%%%%%%%%%%%%%%%%%%%%%%%
% Logical Connectives
%%%%%%%%%%%%%%%%%%%%%%%%%%%%%%%%%%%%%%%%%%%%%
\newcommand{\IImplies}{\Longrightarrow}
\newcommand{\SkipImplies}{\quad\Longrightarrow\quad}
\newcommand{\Ifff}{\Longleftrightarrow}
\newcommand{\iimplies}{\longrightarrow}
\newcommand{\ifff}{\longleftrightarrow}
\newcommand{\Implies}{\Rightarrow}
\newcommand{\Iff}{\Leftrightarrow}
\renewcommand{\implies}{\rightarrow}
\renewcommand{\iff}{\leftrightarrow}
\newcommand{\AND}{\wedge}
\newcommand{\OR}{\vee}
\newcommand{\st}{\text{ s.t. }}
\newcommand{\Or}{\text{ or }}

%%%%%%%%%%%%%%%%%%%%%%%%%%%%%%%%%%%%%%%%%%%%%
% Function Arrows
%%%%%%%%%%%%%%%%%%%%%%%%%%%%%%%%%%%%%%%%%%%%%

\newcommand{\injection}{\xrightarrow{\text{1-1}}}
\newcommand{\surjection}{\xrightarrow{\text{onto}}}
\newcommand{\bijection}{\xrightarrow[\text{onto}]{\text{1-1}}}
\newcommand{\cofmap}{\xrightarrow{\text{cof}}}
\newcommand{\map}{\rightarrow}

%%%%%%%%%%%%%%%%%%%%%%%%%%%%%%%%%%%%%%%%%%%%%
% Mouse Comparison Operators
%%%%%%%%%%%%%%%%%%%%%%%%%%%%%%%%%%%%%%%%%%%%%
\newcommand{\initseg}{\trianglelefteq}
\newcommand{\properseg}{\lhd}
\newcommand{\notinitseg}{\ntrianglelefteq}
\newcommand{\notproperseg}{\ntriangleleft}

%%%%%%%%%%%%%%%%%%%%%%%%%%%%%%%%%%%%%%%%%%%%%
%           calligraphic letters
%%%%%%%%%%%%%%%%%%%%%%%%%%%%%%%%%%%%%%%%%%%%%
\newcommand{\cA}{\mathcal{A}}
\newcommand{\cB}{\mathcal{B}}
\newcommand{\cC}{\mathcal{C}}
\newcommand{\cD}{\mathcal{D}}
\newcommand{\cE}{\mathcal{E}}
\newcommand{\cF}{\mathcal{F}}
\newcommand{\cG}{\mathcal{G}}
\newcommand{\cH}{\mathcal{H}}
\newcommand{\cI}{\mathcal{I}}
\newcommand{\cJ}{\mathcal{J}}
\newcommand{\cK}{\mathcal{K}}
\newcommand{\cL}{\mathcal{L}}
\newcommand{\cM}{\mathcal{M}}
\newcommand{\cN}{\mathcal{N}}
\newcommand{\cO}{\mathcal{O}}
\newcommand{\cP}{\mathcal{P}}
\newcommand{\cQ}{\mathcal{Q}}
\newcommand{\cR}{\mathcal{R}}
\newcommand{\cS}{\mathcal{S}}
\newcommand{\cT}{\mathcal{T}}
\newcommand{\cU}{\mathcal{U}}
\newcommand{\cV}{\mathcal{V}}
\newcommand{\cW}{\mathcal{W}}
\newcommand{\cX}{\mathcal{X}}
\newcommand{\cY}{\mathcal{Y}}
\newcommand{\cZ}{\mathcal{Z}}


%%%%%%%%%%%%%%%%%%%%%%%%%%%%%%%%%%%%%%%%%%%%%
%          Primed Letters
%%%%%%%%%%%%%%%%%%%%%%%%%%%%%%%%%%%%%%%%%%%%%
\newcommand{\aprime}{a^{\prime}}
\newcommand{\bprime}{b^{\prime}}
\newcommand{\cprime}{c^{\prime}}
\newcommand{\dprime}{d^{\prime}}
\newcommand{\eprime}{e^{\prime}}
\newcommand{\fprime}{f^{\prime}}
\newcommand{\gprime}{g^{\prime}}
\newcommand{\hprime}{h^{\prime}}
\newcommand{\iprime}{i^{\prime}}
\newcommand{\jprime}{j^{\prime}}
\newcommand{\kprime}{k^{\prime}}
\newcommand{\lprime}{l^{\prime}}
\newcommand{\mprime}{m^{\prime}}
\newcommand{\nprime}{n^{\prime}}
\newcommand{\oprime}{o^{\prime}}
\newcommand{\pprime}{p^{\prime}}
\newcommand{\qprime}{q^{\prime}}
\newcommand{\rprime}{r^{\prime}}
\newcommand{\sprime}{s^{\prime}}
\newcommand{\tprime}{t^{\prime}}
\newcommand{\uprime}{u^{\prime}}
\newcommand{\vprime}{v^{\prime}}
\newcommand{\wprime}{w^{\prime}}
\newcommand{\xprime}{x^{\prime}}
\newcommand{\yprime}{y^{\prime}}
\newcommand{\zprime}{z^{\prime}}
\newcommand{\Aprime}{A^{\prime}}
\newcommand{\Bprime}{B^{\prime}}
\newcommand{\Cprime}{C^{\prime}}
\newcommand{\Dprime}{D^{\prime}}
\newcommand{\Eprime}{E^{\prime}}
\newcommand{\Fprime}{F^{\prime}}
\newcommand{\Gprime}{G^{\prime}}
\newcommand{\Hprime}{H^{\prime}}
\newcommand{\Iprime}{I^{\prime}}
\newcommand{\Jprime}{J^{\prime}}
\newcommand{\Kprime}{K^{\prime}}
\newcommand{\Lprime}{L^{\prime}}
\newcommand{\Mprime}{M^{\prime}}
\newcommand{\Nprime}{N^{\prime}}
\newcommand{\Oprime}{O^{\prime}}
\newcommand{\Pprime}{P^{\prime}}
\newcommand{\Qprime}{Q^{\prime}}
\newcommand{\Rprime}{R^{\prime}}
\newcommand{\Sprime}{S^{\prime}}
\newcommand{\Tprime}{T^{\prime}}
\newcommand{\Uprime}{U^{\prime}}
\newcommand{\Vprime}{V^{\prime}}
\newcommand{\Wprime}{W^{\prime}}
\newcommand{\Xprime}{X^{\prime}}
\newcommand{\Yprime}{Y^{\prime}}
\newcommand{\Zprime}{Z^{\prime}}
\newcommand{\alphaprime}{\alpha^{\prime}}
\newcommand{\betaprime}{\beta^{\prime}}
\newcommand{\gammaprime}{\gamma^{\prime}}
\newcommand{\Gammaprime}{\Gamma^{\prime}}
\newcommand{\deltaprime}{\delta^{\prime}}
\newcommand{\epsilonprime}{\epsilon^{\prime}}
\newcommand{\kappaprime}{\kappa^{\prime}}
\newcommand{\lambdaprime}{\lambda^{\prime}}
\newcommand{\rhoprime}{\rho^{\prime}}
\newcommand{\Sigmaprime}{\Sigma^{\prime}}
\newcommand{\tauprime}{\tau^{\prime}}
\newcommand{\xiprime}{\xi^{\prime}}
\newcommand{\thetaprime}{\theta^{\prime}}
\newcommand{\Omegaprime}{\Omega^{\prime}}
\newcommand{\cMprime}{\cM^{\prime}}
\newcommand{\cNprime}{\cN^{\prime}}
\newcommand{\cPprime}{\cP^{\prime}}
\newcommand{\cQprime}{\cQ^{\prime}}
\newcommand{\cRprime}{\cR^{\prime}}
\newcommand{\cSprime}{\cS^{\prime}}
\newcommand{\cTprime}{\cT^{\prime}}

%%%%%%%%%%%%%%%%%%%%%%%%%%%%%%%%%%%%%%%%%%%%%
%          bar Letters
%%%%%%%%%%%%%%%%%%%%%%%%%%%%%%%%%%%%%%%%%%%%%
\newcommand{\abar}{\bar{a}}
\newcommand{\bbar}{\bar{b}}
\newcommand{\zbar}{\bar{z}}
\newcommand{\phibar}{\bar{\varphi}}
\newcommand{\psibar}{\bar{\psi}}
\newcommand{\thetabar}{\bar{\theta}}
\newcommand{\nubar}{\bar{\nu}}

%%%%%%%%%%%%%%%%%%%%%%%%%%%%%%%%%%%%%%%%%%%%%
%          star Letters
%%%%%%%%%%%%%%%%%%%%%%%%%%%%%%%%%%%%%%%%%%%%%
\newcommand{\phistar}{\phi^{*}}


%%%%%%%%%%%%%%%%%%%%%%%%%%%%%%%%%%%%%%%%%%%%%
%          Formulas
%%%%%%%%%%%%%%%%%%%%%%%%%%%%%%%%%%%%%%%%%%%%%

\newcommand{\formulaphi}{\text{\large $\varphi$}}
\newcommand{\Formulaphi}{\text{\Large $\varphi$}}


%%%%%%%%%%%%%%%%%%%%%%%%%%%%%%%%%%%%%%%%%%%%%
%          Fraktur Letters
%%%%%%%%%%%%%%%%%%%%%%%%%%%%%%%%%%%%%%%%%%%%%

\newcommand{\fa}{\mathfrak{a}}
\newcommand{\fb}{\mathfrak{b}}
\newcommand{\fc}{\mathfrak{c}}
\newcommand{\fk}{\mathfrak{k}}
\newcommand{\fp}{\mathfrak{p}}
\newcommand{\fq}{\mathfrak{q}}
\newcommand{\fr}{\mathfrak{r}}
\newcommand{\fA}{\mathfrak{A}}
\newcommand{\fB}{\mathfrak{B}}
\newcommand{\fC}{\mathfrak{C}}
\newcommand{\fD}{\mathfrak{D}}

%%%%%%%%%%%%%%%%%%%%%%%%%%%%%%%%%%%%%%%%%%%%%
%          Bold Letters
%%%%%%%%%%%%%%%%%%%%%%%%%%%%%%%%%%%%%%%%%%%%%
\newcommand{\ba}{\mathbf{a}}
\newcommand{\bb}{\mathbf{b}}
\newcommand{\bc}{\mathbf{c}}
\newcommand{\bd}{\mathbf{d}}
\newcommand{\be}{\mathbf{e}}
\newcommand{\bu}{\mathbf{u}}
\newcommand{\bv}{\mathbf{v}}
\newcommand{\bw}{\mathbf{w}}
\newcommand{\bx}{\mathbf{x}}
\newcommand{\by}{\mathbf{y}}
\newcommand{\bz}{\mathbf{z}}
\newcommand{\bSigma}{\boldsymbol{\Sigma}}
\newcommand{\bPi}{\boldsymbol{\Pi}}
\newcommand{\bDelta}{\boldsymbol{\Delta}}
\newcommand{\bdelta}{\boldsymbol{\delta}}
\newcommand{\bgamma}{\boldsymbol{\gamma}}
\newcommand{\bGamma}{\boldsymbol{\Gamma}}

%%%%%%%%%%%%%%%%%%%%%%%%%%%%%%%%%%%%%%%%%%%%%
%         Bold numbers
%%%%%%%%%%%%%%%%%%%%%%%%%%%%%%%%%%%%%%%%%%%%%
\newcommand{\bzero}{\mathbf{0}}

%%%%%%%%%%%%%%%%%%%%%%%%%%%%%%%%%%%%%%%%%%%%%
% Projective-Like Pointclasses
%%%%%%%%%%%%%%%%%%%%%%%%%%%%%%%%%%%%%%%%%%%%%
\newcommand{\Sa}[2][\alpha]{\Sigma_{(#1,#2)}}
\newcommand{\Pa}[2][\alpha]{\Pi_{(#1,#2)}}
\newcommand{\Da}[2][\alpha]{\Delta_{(#1,#2)}}
\newcommand{\Aa}[2][\alpha]{A_{(#1,#2)}}
\newcommand{\Ca}[2][\alpha]{C_{(#1,#2)}}
\newcommand{\Qa}[2][\alpha]{Q_{(#1,#2)}}
\newcommand{\da}[2][\alpha]{\delta_{(#1,#2)}}
\newcommand{\leqa}[2][\alpha]{\leq_{(#1,#2)}}
\newcommand{\lessa}[2][\alpha]{<_{(#1,#2)}}
\newcommand{\equiva}[2][\alpha]{\equiv_{(#1,#2)}}


\newcommand{\Sl}[1]{\Sa[\lambda]{#1}}
\newcommand{\Pl}[1]{\Pa[\lambda]{#1}}
\newcommand{\Dl}[1]{\Da[\lambda]{#1}}
\newcommand{\Al}[1]{\Aa[\lambda]{#1}}
\newcommand{\Cl}[1]{\Ca[\lambda]{#1}}
\newcommand{\Ql}[1]{\Qa[\lambda]{#1}}

\newcommand{\San}{\Sa{n}}
\newcommand{\Pan}{\Pa{n}}
\newcommand{\Dan}{\Da{n}}
\newcommand{\Can}{\Ca{n}}
\newcommand{\Qan}{\Qa{n}}
\newcommand{\Aan}{\Aa{n}}
\newcommand{\dan}{\da{n}}
\newcommand{\leqan}{\leqa{n}}
\newcommand{\lessan}{\lessa{n}}
\newcommand{\equivan}{\equiva{n}}

%%%%%%%%%%%%%%%%%%%%%%%%%%%%%%%%%%%%%%%%%%%%%
% Linear Algebra
%%%%%%%%%%%%%%%%%%%%%%%%%%%%%%%%%%%%%%%%%%%%%
\newcommand{\transpose}[1]{{#1}^{\text{T}}}
\newcommand{\norm}[1]{\lVert{#1}\rVert}
\newcommand\aug{\fboxsep=-\fboxrule\!\!\!\fbox{\strut}\!\!\!}

%%%%%%%%%%%%%%%%%%%%%%%%%%%%%%%%%%%%%%%%%%%%%
% Number Theory
%%%%%%%%%%%%%%%%%%%%%%%%%%%%%%%%%%%%%%%%%%%%%
\DeclareMathOperator{\Spec}{Spec}
\newcommand{\av}[1]{\lvert#1\rvert}
\DeclareMathOperator{\divides}{\mid}
\DeclareMathOperator{\ndivides}{\nmid}


\graphicspath{{images/}}

\newtheorem*{claim}{claim}
\newtheorem*{observation}{Observation}
\newtheorem*{warning}{Warning}
\newtheorem*{question}{Question}
\newtheorem{remark}[theorem]{Remark}

\newenvironment*{subproof}[1][Proof]
{\begin{proof}[#1]}{\renewcommand{\qedsymbol}{$\diamondsuit$} \end{proof}}

\mode<presentation>
{
  \usetheme{Singapore}
  % or ...

  \setbeamercovered{invisible}
  % or whatever (possibly just delete it)
}


\usepackage[english]{babel}
% or whatever

\usepackage[latin1]{inputenc}
% or whatever

\usepackage{times}
\usepackage[T1]{fontenc}
% Or whatever. Note that the encoding and the font should match. If T1
% does not look nice, try deleting the line with the fontenc.

\title{Lesson 17 \\ Rank, Kernel and Image}
\subtitle{Math 325, Linear Algebra \\ Spring 2019 \\ SFSU}
\author{Mitch Rudominer}
\date{}



% If you wish to uncover everything in a step-wise fashion, uncomment
% the following command:

\beamerdefaultoverlayspecification{<+->}

\begin{document}

\begin{frame}
  \titlepage
\end{frame}

%%%%%%%%%%%%%%%%%%%%%%%%%%%%%%%%%%%%%%%%%%%%%%%%%%%%%%%%%%%%%%%%%%%%%%%%%

\begin{frame}{Rank of a matrix}

\begin{itemize}
\item \textbf{Definition.} Let $A$ be an $m\times n$ matrix.
\item The span of the columns of $A$ is called the \emph{column space} of $A$.
\item \textbf{Definition.} The \emph{rank} of $A$, written $\rank(A)$,
is the dimension of the column space of $A$.\
\item Equivalently, the size of a maximal linearly independent subset of
the columns of $A$.
\end{itemize}
\end{frame}
%%%%%%%%%%%%%%%%%%%%%%%%%%%%%%%%%%%%%%%%%%%%%%%%%%%%%%%%%%%%%%%%%%%%%%%%%

\begin{frame}{Example of rank}

\begin{itemize}
\item Find the rank of $A$.
\item $A=
\begin{pmatrix}
1  & 0  & 1 \\
0  & 1  & 3 \\
2  & 0  & 2 \\
\end{pmatrix}
$.
\item The first two columns are linearly independent but the third column
is a linear combination of the first two columns.
\item So $\rank(A) = 2$.
\item And the first two columns are a basis for the column space.
\end{itemize}


\end{frame}

%%%%%%%%%%%%%%%%%%%%%%%%%%%%%%%%%%%%%%%%%%%%%%%%%%%%%%%%%%%%%%%%%%%%%%%%%

\begin{frame}{Example 2 of rank}

\begin{itemize}
\item Find the rank of $A$.
\item $A=
\begin{pmatrix}
-3  & 2  & 1 \\
0   & 1  & 3 \\
0   & 0  & 2 \\
\end{pmatrix}
$.
\item The three columns are linearly independent. We know this because the
matrix is upper-triangular with nonzeros on the diagonal.
\item So $\rank(A) = 3$.
\item And all three columns form a basis for the column space.
\end{itemize}

\end{frame}

%%%%%%%%%%%%%%%%%%%%%%%%%%%%%%%%%%%%%%%%%%%%%%%%%%%%%%%%%%%%%%%%%%%%%%%%%

\begin{frame}{The basic columns of a matrix}

\begin{itemize}
\item \textbf{Definition.} Let $A = \left(\bv_1 \vert \bv_2 \vert \cdots \vert \bv_n\right)$
be an $m\times n$ matrix with columns $\bv_1,\bv_2,\cdots,\bv_n$.
\item The \emph{basic} columns of $A$ are defined as follows:
\item If $A$ is the zero matrix then it has no basic columns.
\item Let $\bv_f$ be the first nonzero column. Then $\bv_f$ is the first basic column.
\item So for example if $\bv_1\not=\bzero$ then $\bv_1 = \bv_f$ is the first basic column.
\item For $i>f$, $\bv_i$ is a basic column iff it is not in the span of the earlier columns.
\item i.e $\bv_i$ is a basic column iff $\bv_i\notin\spn(\bv_1,\bv_2,\cdots,\bv_{i-i})$.
\end{itemize}

\end{frame}

%%%%%%%%%%%%%%%%%%%%%%%%%%%%%%%%%%%%%%%%%%%%%%%%%%%%%%%%%%%%%%%%%%%%%%%%%

\begin{frame}{The basic columns are a maximal linearly independent set}

\begin{itemize}
\item \textbf{Lemma.} Let $A = \left(\bv_1 \vert \bv_2 \vert \cdots \vert \bv_n\right)$
be an $m\times n$ matrix with columns $\bv_1,\bv_2,\cdots,\bv_n$.
\item Let $\bv_{i_1},\bv_{i_2},\cdots,\bv_{i_r}$ be the $r$ basic columns of $A$.
\item Then $\singleton{\bv_{i_1},\bv_{i_2},\cdots,\bv_{i_r}}$ is a maximal linearly independent
subset of the columns of $A$.
\item So $\rank(A) = r$
\item and $\singleton{\bv_{i_1},\bv_{i_2},\cdots,\bv_{i_r}}$ form a basis for the
column space of $A$.
\end{itemize}

\end{frame}

%%%%%%%%%%%%%%%%%%%%%%%%%%%%%%%%%%%%%%%%%%%%%%%%%%%%%%%%%%%%%%%%%%%%%%%%%

\begin{frame}{The basic columns are a maximal linearly independent set}

\begin{itemize}
\item \textbf{proof.} $\singleton{\bv_{i_1},\bv_{i_2},\cdots,\bv_{i_r}}$ is linearly
independent:
\item Suppose $c_1\bv_{i_1} + c_2\bv_{i_2} + \cdots + c_r \bv_{i_r} = 0$.
\item Let $k$ be greatest such that $c_k\not=0$.
\item So $c_1\bv_{i_1} + c_2\bv_{i_2} + \cdots + c_k \bv_{i_k} = 0$.
\item Then we could divide by $c_k$ and solve for $\bv_{i_k}$.
\item $\bv_{i_k} = -(c_1/c_k)\bv_{i_1} -(c_2/c_k)\bv_{i_2} \cdots -(c_{k-1}/c_k)\bv_{i_{k-1}}$.
\item So we have expressed $\bv_{i_k}$ as a linear combination of the earlier $\bv_i$.
\item But by definition of $\bv_{i_k}$ it is not in the span of the earlier columns. $\diamondsuit$
\item $\singleton{\bv_{i_1},\bv_{i_2},\cdots,\bv_{i_r}}$ is maximal:
\item Every other
column was not chosen to be a basic column because it is in the span of the
earlier columns. So throwing it in would give us a linearly dependent set. $\qed$.
\end{itemize}

\end{frame}
%%%%%%%%%%%%%%%%%%%%%%%%%%%%%%%%%%%%%%%%%%%%%%%%%%%%%%%%%%%%%%%%%%%%%%%%%

\begin{frame}{Example}

\begin{itemize}
\item Let
$$A = \left(\bv_1 \vert \bv_2 \vert \bv_3 \vert \bv_4 \vert \bv_5 \right) =
\begin{pmatrix}
0 & 0 & 0 & 0 & 1 \\
0 & 0 & 1 & 1 & 0 \\
0 & 1 & 0 & 1 & 0 \\
\end{pmatrix}.
$$
\item $\bv_1 = \bzero$ is not a basic column.
\item $\bv_2
=
\begin{pmatrix}
0 \\
0 \\
1 \\
\end{pmatrix}
$
is the first basic column.
\item $\bv_3 =
\begin{pmatrix}
0 \\
1 \\
0 \\
\end{pmatrix}
$
is not a multiple of $\bv_2$ so it is the second basic column.
\end{itemize}

\end{frame}

%%%%%%%%%%%%%%%%%%%%%%%%%%%%%%%%%%%%%%%%%%%%%%%%%%%%%%%%%%%%%%%%%%%%%%%%%

\begin{frame}{Example continued}

\begin{itemize}
\item We have
$$A = \left(\bv_1 \vert \bv_2 \vert \bv_3 \vert \bv_4 \vert \bv_5 \right) =
\begin{pmatrix}
0 & 0 & 0 & 0 & 1 \\
0 & 0 & 1 & 1 & 0 \\
0 & 1 & 0 & 1 & 0 \\
\end{pmatrix}.
$$
\item $\bv_4
=
\begin{pmatrix}
0 \\
1 \\
1 \\
\end{pmatrix}
$
is in the span of $\bv_2,\bv_3$ so it is not a basic column.
\item $\bv_5
=
\begin{pmatrix}
1 \\
0 \\
0 \\
\end{pmatrix}
$
is not in the span of $\bv_2,\bv_3$ so it is the third basic column.
\item So the basic columns of $A$ are $\bv_2,\bv_3,\bv_5$.
\item So these form a basis for the column space of $A$
\item and $\rank(A) = 3$.
\end{itemize}

\end{frame}

%%%%%%%%%%%%%%%%%%%%%%%%%%%%%%%%%%%%%%%%%%%%%%%%%%%%%%%%%%%%%%%%%%%%%%%%%
\begin{frame}{Basic columns in row-echelon form}

\begin{itemize}
\item \textbf{Theorem.} Let $A$ be an $m\times n$ matrix in row-echelon form.
\item Suppose $A$ has $r$ pivot columns $\bv_1,\bv_2,\cdots\bv_r$.
\item Then $\singleton{\bv_1,\bv_2,\cdots,\bv_r}$ are the basic columns of $A$.
\item So $\singleton{\bv_1,\bv_2,\cdots,\bv_r}$ is a maximal linearly independent subset of the columns of $A$.
\item So $\rank(A)=r$.
\item And $\singleton{\bv_1,\bv_2,\cdots,\bv_r}$ forms a basis for the column space of $A$.
\end{itemize}

\end{frame}

%%%%%%%%%%%%%%%%%%%%%%%%%%%%%%%%%%%%%%%%%%%%%%%%%%%%%%%%%%%%%%%%%%%%%%%%%

\begin{frame}{Example}

Let
$
A=
\begin{pmatrix}
5 & -2 &  7  & 0 \\
0 & 0  & -1  & 1 \\
0 & 0  &  0  & 3 \\
0 & 0  &  0  & 0 \\
\end{pmatrix}
$
\begin{itemize}
\item Is $A$ in row-echelon form?
\item Yes.
\item Give a basis for the column space of $A$.
\item The pivot columns of $A$ are
$
\begin{pmatrix}
5 \\
0 \\
0 \\
0 \\
\end{pmatrix},
\begin{pmatrix}
 7 \\
-1 \\
 0 \\
 0 \\
\end{pmatrix},
\begin{pmatrix}
0 \\
1 \\
3 \\
0 \\
\end{pmatrix}
$.
\item What is $\rank(A)$?
\item 3.
\end{itemize}

\end{frame}

%%%%%%%%%%%%%%%%%%%%%%%%%%%%%%%%%%%%%%%%%%%%%%%%%%%%%%%%%%%%%%%%%%%%%%%%%

\begin{frame}{The pivot columns are linearly independent}

$
\begin{pmatrix}
\textcolor{red}{\mathbf{3}} & \mathbf{1}                   &  0  & 4 & \mathbf{5}                  & -7 \\
\mathbf{0}                  & \textcolor{red}{\mathbf{-1}} & -2  & 1 & \mathbf{0}                  & 0  \\
\mathbf{0}                  & \mathbf{0}                   &  0  & 0 & \textcolor{red}{\mathbf{2}} & -4 \\
\mathbf{0}                  & \mathbf{0}                   &  0  & 0 & \mathbf{0}                  & 0  \\
\end{pmatrix}
$
\begin{itemize}
\item The pivot columns are linearly independent.
\item One way to see this is to notice that if we pull out only the pivot columns
we can add additional columns to form a matrix in upper-triangular form with all zeros on the diagonal.
\end{itemize}

\end{frame}

%%%%%%%%%%%%%%%%%%%%%%%%%%%%%%%%%%%%%%%%%%%%%%%%%%%%%%%%%%%%%%%%%%%%%%%%%

\begin{frame}{The pivot columns are linearly independent}

$
\begin{pmatrix}
\textcolor{red}{\mathbf{3}} & \mathbf{1}                   & \mathbf{5}                  & 0 \\
\mathbf{0}                  & \textcolor{red}{\mathbf{-1}} & \mathbf{0}                  & 0 \\
\mathbf{0}                  & \mathbf{0}                   & \textcolor{red}{\mathbf{2}} & 0 \\
\mathbf{0}                  & \mathbf{0}                   & \mathbf{0}                  & 1
\end{pmatrix}
$
\begin{itemize}
\item For example here we show the pivot columns from the matrix on the previous slide
embedded into an upper-triangular matrix with nonzeros on the diagonal.
\end{itemize}

\end{frame}

%%%%%%%%%%%%%%%%%%%%%%%%%%%%%%%%%%%%%%%%%%%%%%%%%%%%%%%%%%%%%%%%%%%%%%%%%

\begin{frame}{The span of the pivot columns}

$
\begin{pmatrix}
\textcolor{red}{\mathbf{3}} & \mathbf{1}                   &  0  & 4 & \mathbf{5}                  & -7 \\
\mathbf{0}                  & \textcolor{red}{\mathbf{-1}} & -2  & 1 & \mathbf{0}                  & 0  \\
\mathbf{0}                  & \mathbf{0}                   &  0  & 0 & \textcolor{red}{\mathbf{2}} & -4 \\
\mathbf{0}                  & \mathbf{0}                   &  0  & 0 & \mathbf{0}                  & 0  \\
\end{pmatrix}
$
\begin{itemize}
\item Notice $\bv_1$ is a scalar multiple of the first standard basis vector $\be_1$.
\item So $\spn(\bv_1) = \spn(\be_1)$.
\item Notice that $\be_2\in\spn(\be_1,\bv_2)=\spn(\bv_1,\bv_2)$.
\item $\be_2 = -1(\bv_2 - \be_1)$.
\item So $\spn(\bv_1,\bv_2) = \spn(\be_1,\be_2)$.
\item Notice that $\be_3\in\spn(\be_1,\be_2,\bv_5)=\spn(\bv_1,\bv_2,\bv_5)$.
\item So $\spn(\bv_1,\bv_2,\bv_5) = \spn(\be_1,\be_2,\be_3)$.
\item In general the span of the first $k$ pivot columns is equal to the
span of the first $k$ standard basis vectors $\be_1,\be_2,\cdots,\be_k$.
\end{itemize}

\end{frame}

%%%%%%%%%%%%%%%%%%%%%%%%%%%%%%%%%%%%%%%%%%%%%%%%%%%%%%%%%%%%%%%%%%%%%%%%%

\begin{frame}{The non-pivot columns}

$
\begin{pmatrix}
\textcolor{red}{\mathbf{3}} & \mathbf{1}                   &  0  & 4 & \mathbf{5}                  & -7 \\
\mathbf{0}                  & \textcolor{red}{\mathbf{-1}} & -2  & 1 & \mathbf{0}                  & 0  \\
\mathbf{0}                  & \mathbf{0}                   &  0  & 0 & \textcolor{red}{\mathbf{2}} & -4 \\
\mathbf{0}                  & \mathbf{0}                   &  0  & 0 & \mathbf{0}                  & 0  \\
\end{pmatrix}
$
\begin{itemize}
\item The non-pivot columns are in the span of the earlier pivot columns.
\item In our example, $\bv_3,\bv_4\in\spn(\bv_1,\bv_2) = \spn(\be_1,\be_2)$.
\item And $\bv_6\in\spn(\bv_1,\bv_2,\bv_5) = \spn(\be_1,\be_2,\be_3)$.
\item This is because, by definition of row-echelon form, the non-pivot columns must have zeros
in all entries below the pivot to the immediate left.
\end{itemize}

\end{frame}
%%%%%%%%%%%%%%%%%%%%%%%%%%%%%%%%%%%%%%%%%%%%%%%%%%%%%%%%%%%%%%%%%%%%%%%%%

\begin{frame}{Gaussian Elimination for Finding the Basic Columns}

\begin{itemize}
\item \textbf{Theorem} Suppose $A$ is an $m\times n$ matrix.
\item Supose $G$ is an invertible matrix and $E=GA$.
\item Then the indices of the basic columns of $A$ are the same as the
indices of the basic columns of $E$.
\item Suppose $E$ is in row-echelon form.
\item Then the pivot columns of $E$ are the basic columns of $E$.
\item So, to find the basic columns of $A$, put $A$ into row-echelon form.
\item The indices of the pivot columns of the row-echelon form matrix
are the indices of the basic columns of the original matrix.
\end{itemize}

\end{frame}

%%%%%%%%%%%%%%%%%%%%%%%%%%%%%%%%%%%%%%%%%%%%%%%%%%%%%%%%%%%%%%%%%%%%%%%%%

\begin{frame}{Example}

\begin{itemize}
\item Let $A=
\begin{pmatrix}
0 & 2 & 1 \\
1 & 1 & 0 \\
2 & 0 & -1
\end{pmatrix}
$
\item Find $\rank(A)$ and find a basis for the column space of $A$.
\item We perform Gaussian elimination
\item $
\begin{pmatrix}
1 & 1 & 0 \\
0 & 2 & 1 \\
2 & 0 & -1
\end{pmatrix}
\SkipImplies
\begin{pmatrix}
1 & 1 & 0 \\
0 & 2 & 1 \\
0 & -2 & -1
\end{pmatrix}
\SkipImplies
\begin{pmatrix}
1 & 1 & 0 \\
0 & 2 & 1 \\
0 & 0 & 0
\end{pmatrix}
$
\item So $\rank(A)=2$ and a basis for the column space of $A$ is
\item $
\begin{pmatrix}
0 \\
1 \\
2
\end{pmatrix}
\text{, and }
\begin{pmatrix}
2 \\
1 \\
0
\end{pmatrix}
$
\end{itemize}

\end{frame}


%%%%%%%%%%%%%%%%%%%%%%%%%%%%%%%%%%%%%%%%%%%%%%%%%%%%%%%%%%%%%%%%%%%%%%%%%

\begin{frame}{Linear transformations}

\begin{itemize}
\item The definition of linear transformation makes sense on an arbitrary vector space.
\item \textbf{Definition.} Let $V$ and $W$ be two vector spaces.
\item Let $T:V\map W$.
\item Then $T$ is \emph{linear} iff
\item (a) $T$ preserves vector addition: $T(\bv+\bw) = T(\bv) + T(\bw)$.
\item (b) $T$ preserves scalar multiplication: $T(c\bv) = c T(\bv)$.
\item Linear Algebra is the study of linear transformations between vector spaces.
\end{itemize}

\end{frame}
%%%%%%%%%%%%%%%%%%%%%%%%%%%%%%%%%%%%%%%%%%%%%%%%%%%%%%%%%%%%%%%%%%%%%%%%%%

\begin{frame}{The image of zero}

\begin{itemize}
\item \textbf{Lemma.} Let $T:V\map W$ be a linear transformation.
\item Then $T(\bzero) = \bzero$.
\item \textbf{proof.} Let $\bw = T(\bzero)$. We must show that $\bw=\bzero$.
\item $\bw = T(\bzero)$
\item $= T(0 \bzero)$
\item $= 0 T(\bzero)$
\item $= 0 \bw$
\item $= \bzero$. $\qed$
\end{itemize}

\end{frame}
%%%%%%%%%%%%%%%%%%%%%%%%%%%%%%%%%%%%%%%%%%%%%%%%%%%%%%%%%%%%%%%%%%%%%%%%%%
\begin{frame}{The Image of a linear transformation}

\begin{itemize}
\item \textbf{Lemma.} Let $T:V\map W$ be a linear transformation.
\item Then $\ran(T)$ is a subspace of $W$.
\item The range of $T$ is also called the \emph{image} of $T$.
\item \textbf{proof.} We must show $\ran(T)$ contains $\bzero$ and is closed
under addition and scalar multiplication.
\item $T(\bzero) = \bzero$. So $\bzero\in\ran(T)$.
\item Let $\bw_1,\bw_2\in\ran(T)$. Let $c\in\R$. We will show $\bw_1+\bw_2\in\ran(T)$ and $c\bw_1\in\ran(T)$.
\item Let $\bv_1$ and $\bv_2$ be such that $T(\bv_1)=\bw_1$ and $T(\bv_2)=\bw_2$.
\item $T(\bv_1+\bv_2)=T(\bv_1) + T(\bv_2) = \bw_1+\bw_2$.
\item So $\bw_1+\bw_2\in\ran(T)$.
\item $T(c\bv_1)=cT(\bv_1)=c\bw_1$.
\item So $c\bw_1\in\ran(T)$.
\item So $\ran(T)$ is a subspace of $W$. $\qed$
\end{itemize}

\end{frame}

%%%%%%%%%%%%%%%%%%%%%%%%%%%%%%%%%%%%%%%%%%%%%%%%%%%%%%%%%%%%%%%%%%%%%%%%%%
\begin{frame}{The Column Space is the Image}

\begin{itemize}
\item \textbf{Lemma.} Let $A$ be an $m\times n$ matrix.
\item Then the column space of $A$ is the image of $T_A$.
\item And $\rank(A) = $ the dimension of the image of $A$.
\end{itemize}

\end{frame}

%%%%%%%%%%%%%%%%%%%%%%%%%%%%%%%%%%%%%%%%%%%%%%%%%%%%%%%%%%%%%%%%%%%%%%%%%

\begin{frame}{Example}

\begin{itemize}
\item Let $A=
\begin{pmatrix}
0 & 2 & 1 \\
1 & 1 & 0 \\
2 & 0 & -1
\end{pmatrix}
$
\item Find the dimension of and a basis for the image of $T_A$.
\item We perform Gaussian elimination
\item $
\begin{pmatrix}
1 & 1 & 0 \\
0 & 2 & 1 \\
2 & 0 & -1
\end{pmatrix}
\SkipImplies
\begin{pmatrix}
1 & 1 & 0 \\
0 & 2 & 1 \\
0 & -2 & -1
\end{pmatrix}
\SkipImplies
\begin{pmatrix}
1 & 1 & 0 \\
0 & 2 & 1 \\
0 & 0 & 0
\end{pmatrix}
$
\item So the dimension is 2 and a basis for the image of $T_A$ is
\item $
\begin{pmatrix}
0 \\
1 \\
2
\end{pmatrix}
\text{, and }
\begin{pmatrix}
2 \\
1 \\
0
\end{pmatrix}
$
\end{itemize}

\end{frame}

%%%%%%%%%%%%%%%%%%%%%%%%%%%%%%%%%%%%%%%%%%%%%%%%%%%%%%%%%%%%%%%%%%%%%%%%%%

\begin{frame}{The Kernel of a linear transformation}

\begin{itemize}
\item \textbf{Definition.} Let $T:V\map W$ be a linear transformation.
\item The \textbf{kernel} of $T$, $\ker(T)$ is
\item $\setof{v\in V}{T(\bv) = \bzero}$.
\item $\ker(T) = T^{-1}[\singleton{\bzero}]$.
\item The kernel of $T$ is also called the \textbf{nullspace} of $T$.
\end{itemize}

\end{frame}

%%%%%%%%%%%%%%%%%%%%%%%%%%%%%%%%%%%%%%%%%%%%%%%%%%%%%%%%%%%%%%%%%%%%%%%%%%

\begin{frame}{The Kernel is a subspace}

\begin{itemize}
\item \textbf{Lemma.} Let $T:V\map W$ be a linear transformation.
\item Then $\ker(T)$ is a subspace of $V$.
\item \textbf{proof.} We must show $\ker(T)$ contains $\bzero$ and is closed
under addition and scalar multiplication.
\item $T(\bzero) = \bzero$. So $\bzero\in\ker(T)$.
\item Let $\bv_1,\bv_2\in\ker(T)$. Let $c\in\R$. We will show $\bv_1+\bv_2\in\ker(T)$ and $c\bv_1\in\ker(T)$.
\item $T(\bv_1+\bv_2)=T(\bv_1) + T(\bv_2) = \bzero + \bzero = \bzero$.
\item So $\bv_1+\bv_2\in\ker(T)$.
\item $T(c\bv_1)=cT(\bv_1)=c\bzero=\bzero$.
\item So $c\bv_1\in\ker(T)$.
\item So $\ker(T)$ is a subspace of $V$. $\qed$
\end{itemize}

\end{frame}

%%%%%%%%%%%%%%%%%%%%%%%%%%%%%%%%%%%%%%%%%%%%%%%%%%%%%%%%%%%%%%%%%%%%%%%%%%

\begin{frame}{One-to-one linear transformations}

\begin{itemize}
\item \textbf{Theorem.} Let $T:V\map W$ be a linear transformation.
\item Then $T$ is one-to-one if and only if $\ker(T) = \singleton{\bzero}$.
\item \textbf{proof.} If $T$ is one-to-one then $\ker(T) = \singleton{\bzero}$.
\item This is easy because if there were a second vector $\bv\in\ker(T)$, $\bv\not=\bzero$,
\item then we would have $T(\bv) = T(\bzero)$ so $T$ is not one-to-one.
\item Conversely suppose $\ker(T) = \singleton{\bzero}$ and we will show $T$ is one-to-one.
\item If $T(\bv_1) = T(\bv_2)$ then $T(\bv_1  - \bv_2) = \bzero$ so $\bv_1 - \bv_2\in\ker(T)$.
\item So $\bv_1 - \bv_2 = \bzero$.
\item So $\bv_1 = \bv_2$. So $T$ is one-to-one. $\qed$.
\end{itemize}
\end{frame}
%%%%%%%%%%%%%%%%%%%%%%%%%%%%%%%%%%%%%%%%%%%%%%%%%%%%%%%%%%%%%%%%%%%%%%%%%%

\begin{frame}{Homogeneous system of equations}

\begin{itemize}
\item \textbf{Definition.} A system of linear equations is called
\emph{homogeneous} if the right had side of the system consists of all zeroes.
\item As a matrix equation this means $\bb=0$ so the system looks like
$$A\bx=\bzero.$$
\item Notice: $\bx$ is a solution iff $\bx\in\ker(T_A)$.
\item Therefore the set of all solutions is a subspace.
\end{itemize}

\end{frame}

%%%%%%%%%%%%%%%%%%%%%%%%%%%%%%%%%%%%%%%%%%%%%%%%%%%%%%%%%%%%%%%%%%%%%%%%
\begin{frame}{Homogeneous systems and null spaces}
\begin{itemize}
\item Find the kernel of the matrix
$A=
\begin{pmatrix}
1 & 2 & 2 & 3 \\
2 & 4 & 1 & 3 \\
3 & 6 & 1 & 4
\end{pmatrix}
$
\item Solution: This is the same as solving the homogeneous system of equations
\item $A\bx=\bzero$.
\item Use Gaussian elimination to bring $A$ to row-echelon form.
\item Do back-substitution to get the general solution.
\end{itemize}
\end{frame}

%%%%%%%%%%%%%%%%%%%%%%%%%%%%%%%%%%%%%%%%%%%%%%%%%%%%%%%%%%%%%%%%%%%%%%%%
\begin{frame}{Solving a homogeneous system}
\begin{itemize}
\item Solve
$
\begin{pmatrix}
1 & 2 & 2 & 3 \\
2 & 4 & 1 & 3 \\
3 & 6 & 1 & 4
\end{pmatrix}
\begin{pmatrix}
x \\ y \\ z \\ w
\end{pmatrix}=
\begin{pmatrix}
0 \\ 0 \\ 0
\end{pmatrix}
$
\item
$
\begin{pmatrix}
1 & 2 & 2 & 3 & \aug & 0 \\
2 & 4 & 1 & 3 & \aug & 0  \\
3 & 6 & 1 & 4 & \aug & 0
\end{pmatrix}
\SkipImplies
\begin{pmatrix}
1 & 2 & 2 & 3 & \aug & 0  \\
0 & 0 & -3 & -3  & \aug & 0  \\
0 & 0 & -5 & -5 & \aug & 0
\end{pmatrix}
\SkipImplies
\begin{pmatrix}
1 & 2 & 2 & 3    & \aug & 0  \\
0 & 0 & -3 & -3  & \aug & 0  \\
0 & 0 & 0 & 0    & \aug & 0
\end{pmatrix}
$
\end{itemize}
\end{frame}


%%%%%%%%%%%%%%%%%%%%%%%%%%%%%%%%%%%%%%%%%%%%%%%%%%%%%%%%%%%%%%%%%%%%%%%%
\begin{frame}{Solving a homogeneous system}

\begin{itemize}
\item We have reduced the homogeneous system to
\item
$
\begin{pmatrix}
1 & 2 & 2 & 3 \\
0 & 0 & -3 & -3 \\
0 & 0 & 0 & 0
\end{pmatrix}
\begin{pmatrix}
x \\ y \\ z \\ w
\end{pmatrix}
=
\begin{pmatrix}
0 \\ 0 \\  0
\end{pmatrix}
$
\item The basic variables are $x$ and $z$. The free variables are $y$ and $w$.
\item $-3z -3w = 0 \SkipImplies -3z = 3w $
\item $ \SkipImplies z = - w$.
\item $x + 2y +2z +3w = 0 $
\item $\SkipImplies x +2y + 2(-w) + 3w = 0$
\item $\SkipImplies x + 2y  + w = 0 \SkipImplies x = -2y - w$.
\end{itemize}
\end{frame}

%%%%%%%%%%%%%%%%%%%%%%%%%%%%%%%%%%%%%%%%%%%%%%%%%%%%%%%%%%%%%%%%%%%%%%%%
\begin{frame}{Finding the null space}

\begin{itemize}
\item The general solution to the homogeneous system is
\item
$
\begin{pmatrix}
x \\ y \\ z \\ w
\end{pmatrix}
=
\begin{pmatrix}
-2y - w \\
y \\
-w \\
w
\end{pmatrix}
$
\item
$
\begin{pmatrix}
x \\ y \\ z \\ w
\end{pmatrix}
=
y
\begin{pmatrix}
-2 \\
1 \\
0 \\
0
\end{pmatrix}
+
w
\begin{pmatrix}
-1 \\
0 \\
-1 \\
1
\end{pmatrix}
$
\item The kernel of $A$ is the subspace of $\R^4$ spanned by:
$
\singleton{
\begin{pmatrix}
-2 \\
1 \\
0 \\
0
\end{pmatrix}
,
\begin{pmatrix}
-1 \\
0 \\
-1 \\
1
\end{pmatrix}
}
$
\end{itemize}
\end{frame}

%%%%%%%%%%%%%%%%%%%%%%%%%%%%%%%%%%%%%%%%%%%%%%%%%%%%%%%%%%%%%%%%%%%%%%%%
\begin{frame}{Fundamental Theorem of Linear Algebra (Part 1)}
\begin{itemize}
\item \textbf{Theorem.} Let $A$ be an $m\times n$ matrix of rank $r$.
\item Then $\dim(\ran(A)) = r$, and
\item $\dim(\ker(A)) = n-r$.
\item Example. Let
 $ A =
\begin{pmatrix}
1 & 2 & 2 & 3 \\
2 & 4 & 1 & 3 \\
3 & 6 & 1 & 4
\end{pmatrix}
$
\item Then as we saw earlier, $A$ has two basic columns. (The row-echelon form has 2 pivot columns.)
\item $\rank(A) = 2$. So $\dim(\ran(A)) = 2$.
\item $\dim(\ker(A)) = 4 - 2 =2$.
\end{itemize}
\end{frame}

%%%%%%%%%%%%%%%%%%%%%%%%%%%%%%%%%%%%%%%%%%%%%%%%%%%%%%%%%%%%%%%%%%%%%%%%
\begin{frame}{Finding a basis for the null space}

\begin{itemize}
\item Let
$A=
\begin{pmatrix}
1 & 2 & 2 & 3 \\
2 & 4 & 1 & 3 \\
3 & 6 & 1 & 4
\end{pmatrix}
$
\item We saw before that the kernel of $A$ is the subspace of $\R^4$ spanned by:
$
\singleton{
\begin{pmatrix}
-2 \\
1 \\
0 \\
0
\end{pmatrix}
,
\begin{pmatrix}
-1 \\
0 \\
-1 \\
1
\end{pmatrix}
}
$
\item Those two vectors span $\ker(A)$.
\item Now we know more: Those two vectors form a basis for the kernel.
\item We know this because we know that $\dim(\ker(A)) = 2$.
\end{itemize}
\end{frame}

%%%%%%%%%%%%%%%%%%%%%%%%%%%%%%%%%%%%%%%%%%%%%%%%%%%%%%%%%%%%%%%%%%%%%%%%
\begin{frame}{Proof of the fundamental theorem}
\begin{itemize}
\item \textbf{Proof} Let $A$ be an $m\times n$ matrix of rank $r$.
\item We must show that $\dim(\ker(A)) = n -r$.
\item $A$ has $r$ basic variables and $n-r$  free variables.
\item Using the generalized back-substitution technique we showed above
how to find the general solution to the homogeneous system $A\bx=\bzero$.
\item The general solution is expressed as the span of $n-r$ vectors, one for
each free variable.
\item So the only thing we need to prove is that the $n-r$ vectors are
linearly independent.
\item This follows from the fact if the $i$-th variable is free then the
vector corresponding to the $i$-th variable has a 1 in position $i$ and all
of the other vectors have a 0 in position $i$. $\qed$.
\end{itemize}
\end{frame}

%%%%%%%%%%%%%%%%%%%%%%%%%%%%%%%%%%%%%%%%%%%%%%%%%%%%%%%%%%%%%%%%%%%%%%%%
\begin{frame}{Example of proof.}
\begin{itemize}
\item For example, if
$A=
\begin{pmatrix}
1 & 2 & 2 & 3 \\
2 & 4 & 1 & 3 \\
3 & 6 & 1 & 4
\end{pmatrix}
$
\item The general solution to the homogeneous system $A\bx=\bzero$ is
\item
$
\begin{pmatrix}
x \\ y \\ z \\ w
\end{pmatrix}
=
\begin{pmatrix}
-2y - w \\
y \\
-w \\
w
\end{pmatrix}
=
y
\begin{pmatrix}
-2 \\
1 \\
0 \\
0
\end{pmatrix}
+
w
\begin{pmatrix}
-1 \\
0 \\
-1 \\
1
\end{pmatrix}
$
\item The free variables are $y$ and $w$ in positions 2 and 4.
\item The first vector has a 1 in position 2 and a 0 in position 4.
\item The second vector has a 0 in position 2 and a 1 in position 4.
\item So the two vectors are linearly independent.
\end{itemize}
\end{frame}

%%%%%%%%%%%%%%%%%%%%%%%%%%%%%%%%%%%%%%%%%%%%%%%%%%%%%%%%%%%%%%%%%%%%%%%%
\begin{frame}{Fundamental Theorem of Linear Algebra (More formal version)}
\begin{itemize}
\item \textbf{Theorem.} Let $V$ be an $n$-dimensional vector space.
\item Let $T:V\map W$ be a linear transformation.
\item Let $r=\dim(\ran(T))$.
\item Then $\dim(\ker(T)) = n - r$.
\item i.e. $\dim(\ker(T)) + \dim(\ran(T)) = \dim(V)$.
\item \textbf{proof.} Let $\bv_1,\bv_2,\cdots,\bv_s$ be a basis for $\ker(T)$.
\item Let $\bu_1,\bu_2,\cdots,\bu_t$ be additional vectors so that
\item $\bv_1,\bv_2,\cdots,\bv_s,\bu_1,\bu_2,\cdots,\bu_t$ are a basis for $V$.
\item So $s+t = n$.
\item It suffices to show that $T(u_1), T(u_2),\cdots, T(u_t)$ forms a basis for $\ran(T)$.
\item Because then $t=r$ and $\dim(\ker(T)) = s = n-r$.
\end{itemize}
\end{frame}

%%%%%%%%%%%%%%%%%%%%%%%%%%%%%%%%%%%%%%%%%%%%%%%%%%%%%%%%%%%%%%%%%%%%%%%%
\begin{frame}{proof continued.}
\begin{itemize}
\item Since every element of $V$ is in the span of $\bv_1,\bv_2,\cdots,\bv_s,\bu_1,\bu_2,\cdots,\bu_t$,
\item every element of $\ran(T)$ is in the span of $T(\bv_1),T(\bv_2),\cdots,T(\bv_s), T(\bu_1),T(\bu_2),\cdots,T(\bu_t)$.
\item But $T(\bv_1) = T(\bv_2) = \cdots = T(\bv_s) = 0$.
\item So every element of $\ran(T)$ is in the span of $T(\bu_1),T(\bu_2),\cdots,T(\bu_t)$.
\item Suppose $c_1 T(\bu_1) + c_2 T(\bu_2) + \cdots + c_t T(\bu_t) = \bzero.$
\item Then $T(c_1\bu_1 + c_2 \bu_2 + \cdots + c_t\bu_t) = \bzero$.
\item So $c_1\bu_1 + c_2 \bu_2 + \cdots + c_t\bu_t \in \ker(T)$.
\item So $c_1\bu_1 + c_2 \bu_2 + \cdots + c_t\bu_t$ is in the span of $\bv_1,\bv_2,\cdots,\bv_s$.
\item But $\bv_1,\bv_2,\cdots,\bv_s,\bu_1,\bu_2,\cdots,\bu_t$ are linearly independent.
\item So $c_1\bu_1 + c_2 \bu_2 + \cdots + c_t\bu_t = \bzero$.
\item So $c_1 = c_2 = \cdots = c_t = 0$.
\item So $T(\bu_1), T(\bu_2),\cdots, T(\bu_t)$  are linearly independent. $\qed$
\end{itemize}
\end{frame}



\end{document}


