% $Header$

\documentclass{beamer}
%\documentclass[handout]{beamer}

\usepackage{amsmath,amssymb,latexsym,eucal,amsthm,graphicx,xcolor}
%%%%%%%%%%%%%%%%%%%%%%%%%%%%%%%%%%%%%%%%%%%%%
% Common Set Theory Constructs
%%%%%%%%%%%%%%%%%%%%%%%%%%%%%%%%%%%%%%%%%%%%%

\newcommand{\setof}[2]{\left\{ \, #1 \, \left| \, #2 \, \right.\right\}}
\newcommand{\lsetof}[2]{\left\{\left. \, #1 \, \right| \, #2 \,  \right\}}
\newcommand{\bigsetof}[2]{\bigl\{ \, #1 \, \bigm | \, #2 \,\bigr\}}
\newcommand{\Bigsetof}[2]{\Bigl\{ \, #1 \, \Bigm | \, #2 \,\Bigr\}}
\newcommand{\biggsetof}[2]{\biggl\{ \, #1 \, \biggm | \, #2 \,\biggr\}}
\newcommand{\Biggsetof}[2]{\Biggl\{ \, #1 \, \Biggm | \, #2 \,\Biggr\}}
\newcommand{\dotsetof}[2]{\left\{ \, #1 \, : \, #2 \, \right\}}
\newcommand{\bigdotsetof}[2]{\bigl\{ \, #1 \, : \, #2 \,\bigr\}}
\newcommand{\Bigdotsetof}[2]{\Bigl\{ \, #1 \, \Bigm : \, #2 \,\Bigr\}}
\newcommand{\biggdotsetof}[2]{\biggl\{ \, #1 \, \biggm : \, #2 \,\biggr\}}
\newcommand{\Biggdotsetof}[2]{\Biggl\{ \, #1 \, \Biggm : \, #2 \,\Biggr\}}
\newcommand{\sequence}[2]{\left\langle \, #1 \,\left| \, #2 \, \right. \right\rangle}
\newcommand{\lsequence}[2]{\left\langle\left. \, #1 \, \right| \,#2 \,  \right\rangle}
\newcommand{\bigsequence}[2]{\bigl\langle \,#1 \, \bigm | \, #2 \, \bigr\rangle}
\newcommand{\Bigsequence}[2]{\Bigl\langle \,#1 \, \Bigm | \, #2 \, \Bigr\rangle}
\newcommand{\biggsequence}[2]{\biggl\langle \,#1 \, \biggm | \, #2 \, \biggr\rangle}
\newcommand{\Biggsequence}[2]{\Biggl\langle \,#1 \, \Biggm | \, #2 \, \Biggr\rangle}
\newcommand{\singleton}[1]{\left\{#1\right\}}
\newcommand{\angles}[1]{\left\langle #1 \right\rangle}
\newcommand{\bigangles}[1]{\bigl\langle #1 \bigr\rangle}
\newcommand{\Bigangles}[1]{\Bigl\langle #1 \Bigr\rangle}
\newcommand{\biggangles}[1]{\biggl\langle #1 \biggr\rangle}
\newcommand{\Biggangles}[1]{\Biggl\langle #1 \Biggr\rangle}


\newcommand{\force}[1]{\Vert\!\underset{\!\!\!\!\!#1}{\!\!\!\relbar\!\!\!%
\relbar\!\!\relbar\!\!\relbar\!\!\!\relbar\!\!\relbar\!\!\relbar\!\!\!%
\relbar\!\!\relbar\!\!\relbar}}
\newcommand{\longforce}[1]{\Vert\!\underset{\!\!\!\!\!#1}{\!\!\!\relbar\!\!\!%
\relbar\!\!\relbar\!\!\relbar\!\!\!\relbar\!\!\relbar\!\!\relbar\!\!\!%
\relbar\!\!\relbar\!\!\relbar\!\!\relbar\!\!\relbar\!\!\relbar\!\!\relbar\!\!\relbar}}
\newcommand{\nforce}[1]{\Vert\!\underset{\!\!\!\!\!#1}{\!\!\!\relbar\!\!\!%
\relbar\!\!\relbar\!\!\relbar\!\!\!\relbar\!\!\relbar\!\!\relbar\!\!\!%
\relbar\!\!\not\relbar\!\!\relbar}}
\newcommand{\forcein}[2]{\overset{#2}{\Vert\underset{\!\!\!\!\!#1}%
{\!\!\!\relbar\!\!\!\relbar\!\!\relbar\!\!\relbar\!\!\!\relbar\!\!\relbar\!%
\!\relbar\!\!\!\relbar\!\!\relbar\!\!\relbar\!\!\relbar\!\!\!\relbar\!\!%
\relbar\!\!\relbar}}}

\newcommand{\pre}[2]{{}^{#2}{#1}}

\newcommand{\restr}{\!\!\upharpoonright\!}

%%%%%%%%%%%%%%%%%%%%%%%%%%%%%%%%%%%%%%%%%%%%%
% Set-Theoretic Connectives
%%%%%%%%%%%%%%%%%%%%%%%%%%%%%%%%%%%%%%%%%%%%%

\newcommand{\intersect}{\cap}
\newcommand{\union}{\cup}
\newcommand{\Intersection}[1]{\bigcap\limits_{#1}}
\newcommand{\Union}[1]{\bigcup\limits_{#1}}
\newcommand{\adjoin}{{}^\frown}
\newcommand{\forces}{\Vdash}

%%%%%%%%%%%%%%%%%%%%%%%%%%%%%%%%%%%%%%%%%%%%%
% Miscellaneous
%%%%%%%%%%%%%%%%%%%%%%%%%%%%%%%%%%%%%%%%%%%%%
\newcommand{\defeq}{=_{\text{def}}}
\newcommand{\Turingleq}{\leq_{\text{T}}}
\newcommand{\Turingless}{<_{\text{T}}}
\newcommand{\lexleq}{\leq_{\text{lex}}}
\newcommand{\lexless}{<_{\text{lex}}}
\newcommand{\Turingequiv}{\equiv_{\text{T}}}
\newcommand{\isomorphic}{\cong}

%%%%%%%%%%%%%%%%%%%%%%%%%%%%%%%%%%%%%%%%%%%%%
% Constants
%%%%%%%%%%%%%%%%%%%%%%%%%%%%%%%%%%%%%%%%%%%%%
\newcommand{\R}{\mathbb{R}}
\renewcommand{\P}{\mathbb{P}}
\newcommand{\Q}{\mathbb{Q}}
\newcommand{\Z}{\mathbb{Z}}
\newcommand{\Zpos}{\Z^{+}}
\newcommand{\Znonneg}{\Z^{\geq 0}}
\newcommand{\C}{\mathbb{C}}
\newcommand{\N}{\mathbb{N}}
\newcommand{\B}{\mathbb{B}}
\newcommand{\Bairespace}{\pre{\omega}{\omega}}
\newcommand{\LofR}{L(\R)}
\newcommand{\JofR}[1]{J_{#1}(\R)}
\newcommand{\SofR}[1]{S_{#1}(\R)}
\newcommand{\JalphaR}{\JofR{\alpha}}
\newcommand{\JbetaR}{\JofR{\beta}}
\newcommand{\JlambdaR}{\JofR{\lambda}}
\newcommand{\SalphaR}{\SofR{\alpha}}
\newcommand{\SbetaR}{\SofR{\beta}}
\newcommand{\Pkl}{\mathcal{P}_{\kappa}(\lambda)}
\DeclareMathOperator{\con}{con}
\DeclareMathOperator{\ORD}{OR}
\DeclareMathOperator{\Ord}{OR}
\DeclareMathOperator{\WO}{WO}
\DeclareMathOperator{\OD}{OD}
\DeclareMathOperator{\HOD}{HOD}
\DeclareMathOperator{\HC}{HC}
\DeclareMathOperator{\WF}{WF}
\DeclareMathOperator{\wfp}{wfp}
\DeclareMathOperator{\HF}{HF}
\newcommand{\One}{I}
\newcommand{\ONE}{I}
\newcommand{\Two}{II}
\newcommand{\TWO}{II}
\newcommand{\Mladder}{M^{\text{ld}}}

%%%%%%%%%%%%%%%%%%%%%%%%%%%%%%%%%%%%%%%%%%%%%
% Commutative Algebra Constants
%%%%%%%%%%%%%%%%%%%%%%%%%%%%%%%%%%%%%%%%%%%%%
\DeclareMathOperator{\dottimes}{\dot{\times}}
\DeclareMathOperator{\dotminus}{\dot{-}}
\DeclareMathOperator{\Spec}{Spec}

%%%%%%%%%%%%%%%%%%%%%%%%%%%%%%%%%%%%%%%%%%%%%
% Theories
%%%%%%%%%%%%%%%%%%%%%%%%%%%%%%%%%%%%%%%%%%%%%
\DeclareMathOperator{\ZFC}{ZFC}
\DeclareMathOperator{\ZF}{ZF}
\DeclareMathOperator{\AD}{AD}
\DeclareMathOperator{\ADR}{AD_{\R}}
\DeclareMathOperator{\KP}{KP}
\DeclareMathOperator{\PD}{PD}
\DeclareMathOperator{\CH}{CH}
\DeclareMathOperator{\GCH}{GCH}
\DeclareMathOperator{\HPC}{HPC} % HOD pair capturing
%%%%%%%%%%%%%%%%%%%%%%%%%%%%%%%%%%%%%%%%%%%%%
% Iteration Trees
%%%%%%%%%%%%%%%%%%%%%%%%%%%%%%%%%%%%%%%%%%%%%

\newcommand{\pred}{\text{-pred}}

%%%%%%%%%%%%%%%%%%%%%%%%%%%%%%%%%%%%%%%%%%%%%%%%
% Operator Names
%%%%%%%%%%%%%%%%%%%%%%%%%%%%%%%%%%%%%%%%%%%%%%%%
\DeclareMathOperator{\Det}{Det}
\DeclareMathOperator{\dom}{dom}
\DeclareMathOperator{\ran}{ran}
\DeclareMathOperator{\range}{ran}
\DeclareMathOperator{\image}{image}
\DeclareMathOperator{\crit}{crit}
\DeclareMathOperator{\card}{card}
\DeclareMathOperator{\cf}{cf}
\DeclareMathOperator{\cof}{cof}
\DeclareMathOperator{\rank}{rank}
\DeclareMathOperator{\ot}{o.t.}
\DeclareMathOperator{\ords}{o}
\DeclareMathOperator{\ro}{r.o.}
\DeclareMathOperator{\rud}{rud}
\DeclareMathOperator{\Powerset}{\mathcal{P}}
\DeclareMathOperator{\length}{lh}
\DeclareMathOperator{\lh}{lh}
\DeclareMathOperator{\limit}{lim}
\DeclareMathOperator{\fld}{fld}
\DeclareMathOperator{\projection}{p}
\DeclareMathOperator{\Ult}{Ult}
\DeclareMathOperator{\ULT}{Ult}
\DeclareMathOperator{\Coll}{Coll}
\DeclareMathOperator{\Col}{Col}
\DeclareMathOperator{\Hull}{Hull}
\DeclareMathOperator*{\dirlim}{dir lim}
\DeclareMathOperator{\Scale}{Scale}
\DeclareMathOperator{\supp}{supp}
\DeclareMathOperator{\trancl}{tran.cl.}
\DeclareMathOperator{\trace}{Tr}
\DeclareMathOperator{\diag}{diag}
\DeclareMathOperator{\spn}{span}
\DeclareMathOperator{\sgn}{sgn}
%%%%%%%%%%%%%%%%%%%%%%%%%%%%%%%%%%%%%%%%%%%%%
% Logical Connectives
%%%%%%%%%%%%%%%%%%%%%%%%%%%%%%%%%%%%%%%%%%%%%
\newcommand{\IImplies}{\Longrightarrow}
\newcommand{\SkipImplies}{\quad\Longrightarrow\quad}
\newcommand{\Ifff}{\Longleftrightarrow}
\newcommand{\iimplies}{\longrightarrow}
\newcommand{\ifff}{\longleftrightarrow}
\newcommand{\Implies}{\Rightarrow}
\newcommand{\Iff}{\Leftrightarrow}
\renewcommand{\implies}{\rightarrow}
\renewcommand{\iff}{\leftrightarrow}
\newcommand{\AND}{\wedge}
\newcommand{\OR}{\vee}
\newcommand{\st}{\text{ s.t. }}
\newcommand{\Or}{\text{ or }}

%%%%%%%%%%%%%%%%%%%%%%%%%%%%%%%%%%%%%%%%%%%%%
% Function Arrows
%%%%%%%%%%%%%%%%%%%%%%%%%%%%%%%%%%%%%%%%%%%%%

\newcommand{\injection}{\xrightarrow{\text{1-1}}}
\newcommand{\surjection}{\xrightarrow{\text{onto}}}
\newcommand{\bijection}{\xrightarrow[\text{onto}]{\text{1-1}}}
\newcommand{\cofmap}{\xrightarrow{\text{cof}}}
\newcommand{\map}{\rightarrow}

%%%%%%%%%%%%%%%%%%%%%%%%%%%%%%%%%%%%%%%%%%%%%
% Mouse Comparison Operators
%%%%%%%%%%%%%%%%%%%%%%%%%%%%%%%%%%%%%%%%%%%%%
\newcommand{\initseg}{\trianglelefteq}
\newcommand{\properseg}{\lhd}
\newcommand{\notinitseg}{\ntrianglelefteq}
\newcommand{\notproperseg}{\ntriangleleft}

%%%%%%%%%%%%%%%%%%%%%%%%%%%%%%%%%%%%%%%%%%%%%
%           calligraphic letters
%%%%%%%%%%%%%%%%%%%%%%%%%%%%%%%%%%%%%%%%%%%%%
\newcommand{\cA}{\mathcal{A}}
\newcommand{\cB}{\mathcal{B}}
\newcommand{\cC}{\mathcal{C}}
\newcommand{\cD}{\mathcal{D}}
\newcommand{\cE}{\mathcal{E}}
\newcommand{\cF}{\mathcal{F}}
\newcommand{\cG}{\mathcal{G}}
\newcommand{\cH}{\mathcal{H}}
\newcommand{\cI}{\mathcal{I}}
\newcommand{\cJ}{\mathcal{J}}
\newcommand{\cK}{\mathcal{K}}
\newcommand{\cL}{\mathcal{L}}
\newcommand{\cM}{\mathcal{M}}
\newcommand{\cN}{\mathcal{N}}
\newcommand{\cO}{\mathcal{O}}
\newcommand{\cP}{\mathcal{P}}
\newcommand{\cQ}{\mathcal{Q}}
\newcommand{\cR}{\mathcal{R}}
\newcommand{\cS}{\mathcal{S}}
\newcommand{\cT}{\mathcal{T}}
\newcommand{\cU}{\mathcal{U}}
\newcommand{\cV}{\mathcal{V}}
\newcommand{\cW}{\mathcal{W}}
\newcommand{\cX}{\mathcal{X}}
\newcommand{\cY}{\mathcal{Y}}
\newcommand{\cZ}{\mathcal{Z}}


%%%%%%%%%%%%%%%%%%%%%%%%%%%%%%%%%%%%%%%%%%%%%
%          Primed Letters
%%%%%%%%%%%%%%%%%%%%%%%%%%%%%%%%%%%%%%%%%%%%%
\newcommand{\aprime}{a^{\prime}}
\newcommand{\bprime}{b^{\prime}}
\newcommand{\cprime}{c^{\prime}}
\newcommand{\dprime}{d^{\prime}}
\newcommand{\eprime}{e^{\prime}}
\newcommand{\fprime}{f^{\prime}}
\newcommand{\gprime}{g^{\prime}}
\newcommand{\hprime}{h^{\prime}}
\newcommand{\iprime}{i^{\prime}}
\newcommand{\jprime}{j^{\prime}}
\newcommand{\kprime}{k^{\prime}}
\newcommand{\lprime}{l^{\prime}}
\newcommand{\mprime}{m^{\prime}}
\newcommand{\nprime}{n^{\prime}}
\newcommand{\oprime}{o^{\prime}}
\newcommand{\pprime}{p^{\prime}}
\newcommand{\qprime}{q^{\prime}}
\newcommand{\rprime}{r^{\prime}}
\newcommand{\sprime}{s^{\prime}}
\newcommand{\tprime}{t^{\prime}}
\newcommand{\uprime}{u^{\prime}}
\newcommand{\vprime}{v^{\prime}}
\newcommand{\wprime}{w^{\prime}}
\newcommand{\xprime}{x^{\prime}}
\newcommand{\yprime}{y^{\prime}}
\newcommand{\zprime}{z^{\prime}}
\newcommand{\Aprime}{A^{\prime}}
\newcommand{\Bprime}{B^{\prime}}
\newcommand{\Cprime}{C^{\prime}}
\newcommand{\Dprime}{D^{\prime}}
\newcommand{\Eprime}{E^{\prime}}
\newcommand{\Fprime}{F^{\prime}}
\newcommand{\Gprime}{G^{\prime}}
\newcommand{\Hprime}{H^{\prime}}
\newcommand{\Iprime}{I^{\prime}}
\newcommand{\Jprime}{J^{\prime}}
\newcommand{\Kprime}{K^{\prime}}
\newcommand{\Lprime}{L^{\prime}}
\newcommand{\Mprime}{M^{\prime}}
\newcommand{\Nprime}{N^{\prime}}
\newcommand{\Oprime}{O^{\prime}}
\newcommand{\Pprime}{P^{\prime}}
\newcommand{\Qprime}{Q^{\prime}}
\newcommand{\Rprime}{R^{\prime}}
\newcommand{\Sprime}{S^{\prime}}
\newcommand{\Tprime}{T^{\prime}}
\newcommand{\Uprime}{U^{\prime}}
\newcommand{\Vprime}{V^{\prime}}
\newcommand{\Wprime}{W^{\prime}}
\newcommand{\Xprime}{X^{\prime}}
\newcommand{\Yprime}{Y^{\prime}}
\newcommand{\Zprime}{Z^{\prime}}
\newcommand{\alphaprime}{\alpha^{\prime}}
\newcommand{\betaprime}{\beta^{\prime}}
\newcommand{\gammaprime}{\gamma^{\prime}}
\newcommand{\Gammaprime}{\Gamma^{\prime}}
\newcommand{\deltaprime}{\delta^{\prime}}
\newcommand{\epsilonprime}{\epsilon^{\prime}}
\newcommand{\kappaprime}{\kappa^{\prime}}
\newcommand{\lambdaprime}{\lambda^{\prime}}
\newcommand{\rhoprime}{\rho^{\prime}}
\newcommand{\Sigmaprime}{\Sigma^{\prime}}
\newcommand{\tauprime}{\tau^{\prime}}
\newcommand{\xiprime}{\xi^{\prime}}
\newcommand{\thetaprime}{\theta^{\prime}}
\newcommand{\Omegaprime}{\Omega^{\prime}}
\newcommand{\cMprime}{\cM^{\prime}}
\newcommand{\cNprime}{\cN^{\prime}}
\newcommand{\cPprime}{\cP^{\prime}}
\newcommand{\cQprime}{\cQ^{\prime}}
\newcommand{\cRprime}{\cR^{\prime}}
\newcommand{\cSprime}{\cS^{\prime}}
\newcommand{\cTprime}{\cT^{\prime}}

%%%%%%%%%%%%%%%%%%%%%%%%%%%%%%%%%%%%%%%%%%%%%
%          bar Letters
%%%%%%%%%%%%%%%%%%%%%%%%%%%%%%%%%%%%%%%%%%%%%
\newcommand{\abar}{\bar{a}}
\newcommand{\bbar}{\bar{b}}
\newcommand{\cbar}{\bar{c}}
\newcommand{\ibar}{\bar{i}}
\newcommand{\jbar}{\bar{j}}
\newcommand{\nbar}{\bar{n}}
\newcommand{\xbar}{\bar{x}}
\newcommand{\ybar}{\bar{y}}
\newcommand{\zbar}{\bar{z}}
\newcommand{\pibar}{\bar{\pi}}
\newcommand{\phibar}{\bar{\varphi}}
\newcommand{\psibar}{\bar{\psi}}
\newcommand{\thetabar}{\bar{\theta}}
\newcommand{\nubar}{\bar{\nu}}

%%%%%%%%%%%%%%%%%%%%%%%%%%%%%%%%%%%%%%%%%%%%%
%          star Letters
%%%%%%%%%%%%%%%%%%%%%%%%%%%%%%%%%%%%%%%%%%%%%
\newcommand{\phistar}{\phi^{*}}
\newcommand{\Mstar}{M^{*}}

%%%%%%%%%%%%%%%%%%%%%%%%%%%%%%%%%%%%%%%%%%%%%
%          dotletters Letters
%%%%%%%%%%%%%%%%%%%%%%%%%%%%%%%%%%%%%%%%%%%%%
\newcommand{\Gdot}{\dot{G}}

%%%%%%%%%%%%%%%%%%%%%%%%%%%%%%%%%%%%%%%%%%%%%
%         check Letters
%%%%%%%%%%%%%%%%%%%%%%%%%%%%%%%%%%%%%%%%%%%%%
\newcommand{\deltacheck}{\check{\delta}}
\newcommand{\gammacheck}{\check{\gamma}}


%%%%%%%%%%%%%%%%%%%%%%%%%%%%%%%%%%%%%%%%%%%%%
%          Formulas
%%%%%%%%%%%%%%%%%%%%%%%%%%%%%%%%%%%%%%%%%%%%%

\newcommand{\formulaphi}{\text{\large $\varphi$}}
\newcommand{\Formulaphi}{\text{\Large $\varphi$}}


%%%%%%%%%%%%%%%%%%%%%%%%%%%%%%%%%%%%%%%%%%%%%
%          Fraktur Letters
%%%%%%%%%%%%%%%%%%%%%%%%%%%%%%%%%%%%%%%%%%%%%

\newcommand{\fa}{\mathfrak{a}}
\newcommand{\fb}{\mathfrak{b}}
\newcommand{\fc}{\mathfrak{c}}
\newcommand{\fk}{\mathfrak{k}}
\newcommand{\fp}{\mathfrak{p}}
\newcommand{\fq}{\mathfrak{q}}
\newcommand{\fr}{\mathfrak{r}}
\newcommand{\fA}{\mathfrak{A}}
\newcommand{\fB}{\mathfrak{B}}
\newcommand{\fC}{\mathfrak{C}}
\newcommand{\fD}{\mathfrak{D}}

%%%%%%%%%%%%%%%%%%%%%%%%%%%%%%%%%%%%%%%%%%%%%
%          Bold Letters
%%%%%%%%%%%%%%%%%%%%%%%%%%%%%%%%%%%%%%%%%%%%%
\newcommand{\ba}{\mathbf{a}}
\newcommand{\bb}{\mathbf{b}}
\newcommand{\bc}{\mathbf{c}}
\newcommand{\bd}{\mathbf{d}}
\newcommand{\be}{\mathbf{e}}
\newcommand{\bu}{\mathbf{u}}
\newcommand{\bv}{\mathbf{v}}
\newcommand{\bw}{\mathbf{w}}
\newcommand{\bx}{\mathbf{x}}
\newcommand{\by}{\mathbf{y}}
\newcommand{\bz}{\mathbf{z}}
\newcommand{\bSigma}{\boldsymbol{\Sigma}}
\newcommand{\bPi}{\boldsymbol{\Pi}}
\newcommand{\bDelta}{\boldsymbol{\Delta}}
\newcommand{\bdelta}{\boldsymbol{\delta}}
\newcommand{\bgamma}{\boldsymbol{\gamma}}
\newcommand{\bGamma}{\boldsymbol{\Gamma}}

%%%%%%%%%%%%%%%%%%%%%%%%%%%%%%%%%%%%%%%%%%%%%
%         Bold numbers
%%%%%%%%%%%%%%%%%%%%%%%%%%%%%%%%%%%%%%%%%%%%%
\newcommand{\bzero}{\mathbf{0}}

%%%%%%%%%%%%%%%%%%%%%%%%%%%%%%%%%%%%%%%%%%%%%
% Projective-Like Pointclasses
%%%%%%%%%%%%%%%%%%%%%%%%%%%%%%%%%%%%%%%%%%%%%
\newcommand{\Sa}[2][\alpha]{\Sigma_{(#1,#2)}}
\newcommand{\Pa}[2][\alpha]{\Pi_{(#1,#2)}}
\newcommand{\Da}[2][\alpha]{\Delta_{(#1,#2)}}
\newcommand{\Aa}[2][\alpha]{A_{(#1,#2)}}
\newcommand{\Ca}[2][\alpha]{C_{(#1,#2)}}
\newcommand{\Qa}[2][\alpha]{Q_{(#1,#2)}}
\newcommand{\da}[2][\alpha]{\delta_{(#1,#2)}}
\newcommand{\leqa}[2][\alpha]{\leq_{(#1,#2)}}
\newcommand{\lessa}[2][\alpha]{<_{(#1,#2)}}
\newcommand{\equiva}[2][\alpha]{\equiv_{(#1,#2)}}


\newcommand{\Sl}[1]{\Sa[\lambda]{#1}}
\newcommand{\Pl}[1]{\Pa[\lambda]{#1}}
\newcommand{\Dl}[1]{\Da[\lambda]{#1}}
\newcommand{\Al}[1]{\Aa[\lambda]{#1}}
\newcommand{\Cl}[1]{\Ca[\lambda]{#1}}
\newcommand{\Ql}[1]{\Qa[\lambda]{#1}}

\newcommand{\San}{\Sa{n}}
\newcommand{\Pan}{\Pa{n}}
\newcommand{\Dan}{\Da{n}}
\newcommand{\Can}{\Ca{n}}
\newcommand{\Qan}{\Qa{n}}
\newcommand{\Aan}{\Aa{n}}
\newcommand{\dan}{\da{n}}
\newcommand{\leqan}{\leqa{n}}
\newcommand{\lessan}{\lessa{n}}
\newcommand{\equivan}{\equiva{n}}

\newcommand{\SigmaOneOmega}{\Sigma^1_{\omega}}
\newcommand{\SigmaOneOmegaPlusOne}{\Sigma^1_{\omega+1}}
\newcommand{\PiOneOmega}{\Pi^1_{\omega}}
\newcommand{\PiOneOmegaPlusOne}{\Pi^1_{\omega+1}}
\newcommand{\DeltaOneOmegaPlusOne}{\Delta^1_{\omega+1}}

%%%%%%%%%%%%%%%%%%%%%%%%%%%%%%%%%%%%%%%%%%%%%
% Linear Algebra
%%%%%%%%%%%%%%%%%%%%%%%%%%%%%%%%%%%%%%%%%%%%%
\newcommand{\transpose}[1]{{#1}^{\text{T}}}
\newcommand{\norm}[1]{\lVert{#1}\rVert}
\newcommand\aug{\fboxsep=-\fboxrule\!\!\!\fbox{\strut}\!\!\!}

%%%%%%%%%%%%%%%%%%%%%%%%%%%%%%%%%%%%%%%%%%%%%
% Number Theory
%%%%%%%%%%%%%%%%%%%%%%%%%%%%%%%%%%%%%%%%%%%%%
\newcommand{\av}[1]{\lvert#1\rvert}
\DeclareMathOperator{\divides}{\mid}
\DeclareMathOperator{\ndivides}{\nmid}
\DeclareMathOperator{\lcm}{lcm}
\DeclareMathOperator{\sign}{sign}
\newcommand{\floor}[1]{\left\lfloor{#1}\right\rfloor}
\DeclareMathOperator{\ConCl}{CC}
\DeclareMathOperator{\ord}{ord}



\graphicspath{{images/}}

\newtheorem*{claim}{claim}
\newtheorem*{observation}{Observation}
\newtheorem*{warning}{Warning}
\newtheorem*{question}{Question}
\newtheorem{remark}[theorem]{Remark}

\newenvironment*{subproof}[1][Proof]
{\begin{proof}[#1]}{\renewcommand{\qedsymbol}{$\diamondsuit$} \end{proof}}

\mode<presentation>
{
  \usetheme{Singapore}
  % or ...

  \setbeamercovered{invisible}
  % or whatever (possibly just delete it)
}


\usepackage[english]{babel}
% or whatever

\usepackage[latin1]{inputenc}
% or whatever

\usepackage{times}
\usepackage[T1]{fontenc}
% Or whatever. Note that the encoding and the font should match. If T1
% does not look nice, try deleting the line with the fontenc.

\title{Lesson 15 \\ Gaussian Elimination}
\subtitle{Math 325, Linear Algebra \\ Spring 2019 \\ SFSU}
\author{Mitch Rudominer}
\date{}



% If you wish to uncover everything in a step-wise fashion, uncomment
% the following command:

\beamerdefaultoverlayspecification{<+->}

\begin{document}

\begin{frame}
  \titlepage
\end{frame}

%%%%%%%%%%%%%%%%%%%%%%%%%%%%%%%%%%%%%%%%%%%%%%%%%%%%%%%%%%%%%%%%%%%%%%%%%

\begin{frame}{Theory of Gaussian Elimination}

\begin{itemize}
\item \textbf{Theorem.} Let $A$ be an $m\times n$ matrix.
\item Then there is an invertible $m\times m$ matrix $G$ such that
$GA = B$ is in row-echelon form.
\item Furthermore $G$ can be expressed as a product of elementary row operations
of types I and II.
\item $G = E_k \cdots E_2 E_1$.
\item In other words $E_k \cdots E_2 E_1 A = B$ is in row-echelon form.
\item Furthermore, the indices of the basic columns of $A$ are the same as those for $B$.
\end{itemize}
\end{frame}

%%%%%%%%%%%%%%%%%%%%%%%%%%%%%%%%%%%%%%%%%%%%%%%%%%%%%%%%%%%%%%%%%%%%%%%%%

\begin{frame}{The Gaussian Elimination Algorithm}

\begin{itemize}
\item We are going to prove the theorem by describing an algorithm,
called \emph{Gaussian elimination}.
\item Given an $m\times n$ matrix $A$, the algorithm specifies a
sequence of \emph{row operations} to perform.
\item Each row operation is given by an elementary matrix of type I or II.
\item The algorithm then results in a sequence of matrices:
\item $A_0 = A$
\item $A_1 = E_1 A$
\item $A_2 = E_2 E_1 A$
\item $A_k = E_k \cdots E_2 E_1 A$.
\item $A_k$ will be in row-echelon form.
\end{itemize}
\end{frame}

%%%%%%%%%%%%%%%%%%%%%%%%%%%%%%%%%%%%%%%%%%%%%%%%%%%%%%%%%%%%%%%%%%%%%%%%%

\begin{frame}{Uses of the Gaussian Elimination Algorithm}

\begin{itemize}
\item Let $A$ be an $m\times n$ matrix.
\item Suppose we perform the Guassian elimination algorithm to get $B=GA$
with $B$ in row-echelon form. What does this buy us?
\item Lot's of great things:
\item We now know which are the basic columns of $A$.
\item (Because they are the same as the pivot columns of $B$.)
\item So we know $\rank(A)$
\item and we have a basis for the range of $T_A$.
\end{itemize}
\end{frame}

%%%%%%%%%%%%%%%%%%%%%%%%%%%%%%%%%%%%%%%%%%%%%%%%%%%%%%%%%%%%%%%%%%%%%%%%%

\begin{frame}{Gaussian Elimination For Solving Systems of Equations}

\begin{itemize}
\item We have $B=GA$ with $G$ invertible and $B$ in row-echelon form.
\item We can now solve the system of linear equations $A\bx=\bb$.
\item $\bx$ is a solution iff $\bx$ is a solution to $B\bx=G\bb$.
\item Since $B$ is in row-echelon form, we can easily solve $B\bx=G\bb$ using
back-substitution.
\item For example we now know how to find $\ker(A)$, because that is the solution to
the homogeneous system $A\bx=\bzero$.
\end{itemize}
\end{frame}

%%%%%%%%%%%%%%%%%%%%%%%%%%%%%%%%%%%%%%%%%%%%%%%%%%%%%%%%%%%%%%%%%%%%%%%%%

\begin{frame}{Gaussian Elimination On Non-Singular Matrices}

\begin{itemize}
\item Suppose $A$ is a non-singular square matrix.
\item Suppose we do Guassian elimination and get $B=G A$ with $B$ in
row-echelon form.
\item Then in fact $B$ is upper-triangular with nonzeros on the diagonal.
\item So we can find the unique solution to $A\bx=\bb$.
\item As we will see later, we will also be able to find $A^{-1}$
\end{itemize}
\end{frame}

%%%%%%%%%%%%%%%%%%%%%%%%%%%%%%%%%%%%%%%%%%%%%%%%%%%%%%%%%%%%%%%%%%%%%%%%%

\begin{frame}{Gaussian Elimination With Back Substituion is Efficient}

\begin{itemize}
\item Linear Algegra has many real-world applications
\item in science and engineering and computer science.
\item In these applications the matrices often have millions of entries.
\item Using efficient algorithms is essential.
\item Gaussian elimination is often one of the most efficient algorithms
to use for all of the problems we mentioned above.
\item Gaussian elimination with back substitution is often one of the
best algorithms for solving large systems of linear equations.
\end{itemize}
\end{frame}

%%%%%%%%%%%%%%%%%%%%%%%%%%%%%%%%%%%%%%%%%%%%%%%%%%%%%%%%%%%%%%%%%%%%%%%%%

\begin{frame}{Gaussian Elimination Algorithm by Example}

\begin{itemize}
\item Use Gaussian elimination to bring the following matrix to row-echelon
form. Find the rank and the basic columns.
\item $
\begin{pmatrix}
0 & 0  &   0  &  0   &  1  &  2 \\
0 & 1  &  17  & -11  &  0  &  1  \\
0 & 2  &  34  & -20  &  0  &  0 \\
0 & -1 &  -17   &  6 &  3  &  1 \\
\end{pmatrix}
$
\item First skip any all-zero columns on the left.
\item Start working with the first non-zero column. In our case that is column 2.
\item This will be the first pivot column.
\item We need a non-zero entry in the first position of the first pivot column.
\item Currently we have a 0 in the first position.
\item We have to perform a swap in order to put a nonzero there. (An elementary row operation of type II)
\item Swap rows 1 and 2
\end{itemize}
\end{frame}

%%%%%%%%%%%%%%%%%%%%%%%%%%%%%%%%%%%%%%%%%%%%%%%%%%%%%%%%%%%%%%%%%%%%%%%%%

\begin{frame}{Gaussian Elimination Algorithm by Example}

\begin{itemize}
\item $
\begin{pmatrix}
0 & \textbf{1} &  17    & -11   &  0  &  1  \\
0 &         0  &   0    &   0    &  1  &  2 \\
0 &         2  &  34    &  -20  &  0  &  0 \\
0 &        -1  &  -17   &  6    &  3  &  1 \\
\end{pmatrix}
$
\item Now the bolded 1 is our first pivot.
\item Next we use row operations of type I in order to change all of the
entries below the first pivot to zero.
\item Replace row 3 with row 3 minus 2 times row 1
\item $
\begin{pmatrix}
0 & \textbf{1} &  17   & -11  &  0  &  1  \\
0 &         0  &   0   &  0   &  1  &  2 \\
0 &         0  &   0   &  2   &  0  &  -2 \\
0 &        -1  &  -17   &  6   &  3  &  1 \\
\end{pmatrix}
$
\end{itemize}
\end{frame}

%%%%%%%%%%%%%%%%%%%%%%%%%%%%%%%%%%%%%%%%%%%%%%%%%%%%%%%%%%%%%%%%%%%%%%%%%

\begin{frame}{Gaussian Elimination Algorithm by Example}

\begin{itemize}
\item $
\begin{pmatrix}
0 & \textbf{1} &  17   & -11  &  0  &  1  \\
0 &         0  &  0    &  0   &  1  &  2 \\
0 &         0  &  0    &  2   &  0  &  -2 \\
0 &        -1  &  -17  &  6   &  3  &  1 \\
\end{pmatrix}
$
\item Replace row 4 with row 4 plus row 1
\item $
\begin{pmatrix}
0 & \textbf{1} &  17   &  -11   &  0  &  1  \\
0 &         0  &   0   &   0    &  1  &  2 \\
0 &         0  &   0   &   2   &  0  &  -2 \\
0 &         0  &   0   &  -5   &  3  &  2 \\
\end{pmatrix}
$
\item Now we need to find the next pivot column. Look for the
next column with a nonzero below the first pivot.
\item Column 3 does not have this.
\item Column 4 does. So column 4 is our next pivot column.
\item But column 4 does not have a nonzero in row 2.
\end{itemize}
\end{frame}

%%%%%%%%%%%%%%%%%%%%%%%%%%%%%%%%%%%%%%%%%%%%%%%%%%%%%%%%%%%%%%%%%%%%%%%%

\begin{frame}{Gaussian Elimination Algorithm by Example}

\begin{itemize}
\item $
\begin{pmatrix}
0 & \textbf{1} &  17   &  -11   &  0  &  1   \\
0 &         0  &   0   &   0    &  1  &  2   \\
0 &         0  &   0   &   2   &   0  &  -2  \\
0 &         0  &   0   &  -5   &   3  &  2   \\
\end{pmatrix}
$
\item Swap rows 2 and 3. (An elementary row operation of type II.)
\item $
\begin{pmatrix}
0 & \textbf{1} &  17   &          -11   &  0  &  1  \\
0 &         0  &   0   &   \textbf{2}   &  0  &  -2 \\
0 &         0  &   0   &           0    &  1  &  2 \\
0 &         0  &   0   &          -5    &  3  &  2 \\
\end{pmatrix}
$
\item Now the bolded 2 is our second pivot.
\item Next perform row operations of type I in order to change all of the
entries below the second pivot to zero.
\end{itemize}
\end{frame}

%%%%%%%%%%%%%%%%%%%%%%%%%%%%%%%%%%%%%%%%%%%%%%%%%%%%%%%%%%%%%%%%%%%%%%%%

\begin{frame}{Gaussian Elimination Algorithm by Example}

\begin{itemize}
\item $
\begin{pmatrix}
0 & \textbf{1} &  17   &          -11   &  0  &  1  \\
0 &         0  &   0   &   \textbf{2}   &  0  &  -2 \\
0 &         0  &   0   &           0    &  1  &  2 \\
0 &         0  &   0   &          -5    &  3  &  2 \\
\end{pmatrix}
$
\item Replace row 4 with row 4 + $\frac{5}{2}$ times row 2.
\item $
\begin{pmatrix}
0 & \textbf{1} &  17   &          -11   &  0  &   1  \\
0 &         0  &   0   &   \textbf{2}   &  0  &  -2  \\
0 &         0  &   0   &           0    &  1  &   2  \\
0 &         0  &   0   &           0    &  3  &   -3  \\
\end{pmatrix}
$
\item Now we need to find the next pivot column. Look for the
next column with a nonzero below the second pivot.
\item Column 5 has one. So that is our next pivot column.
\end{itemize}
\end{frame}

%%%%%%%%%%%%%%%%%%%%%%%%%%%%%%%%%%%%%%%%%%%%%%%%%%%%%%%%%%%%%%%%%%%%%%%%

\begin{frame}{Gaussian Elimination Algorithm by Example}

\begin{itemize}
\item $
\begin{pmatrix}
0 & \textbf{1} &  17   &          -11   &          0   &    1  \\
0 &         0  &   0   &   \textbf{2}   &          0   &   -2  \\
0 &         0  &   0   &           0    &  \textbf{1}  &   2  \\
0 &         0  &   0   &           0    &          3   &   -3  \\
\end{pmatrix}
$
\item The bolded 1 in column 5 is our next pivot.
\item Perform row operations of type I in order to change all of the
entries below the third pivot to zero.
\item Replace row 4 with row 4 minus 3 times row 3.
\item $
\begin{pmatrix}
0 & \textbf{1} &  17   &          -11   &          0   &    1  \\
0 &         0  &   0   &   \textbf{2}   &          0   &   -2  \\
0 &         0  &   0   &           0    &  \textbf{1}  &   2  \\
0 &         0  &   0   &           0    &          0   &   -9  \\
\end{pmatrix}
$
\end{itemize}
\end{frame}

%%%%%%%%%%%%%%%%%%%%%%%%%%%%%%%%%%%%%%%%%%%%%%%%%%%%%%%%%%%%%%%%%%%%%%%%

\begin{frame}{Gaussian Elimination Algorithm by Example}

\begin{itemize}
\item $
\begin{pmatrix}
0 & \textbf{1} &  17   &          -11   &          0   &          1  \\
0 &         0  &   0   &   \textbf{2}   &          0   &          -2  \\
0 &         0  &   0   &           0    &  \textbf{1}  &           2  \\
0 &         0  &   0   &           0    &          3   &   \textbf{-9 } \\
\end{pmatrix}
$
\item Now the matrix is in row-echelon form.
\item The sixth column is the fourth pivot column and the bolded -9 is the
fourth pivot.
\item So the rank of \emph{both} the row-echelon matrix and the \emph{original}
matrix is 4.
\item And the basic columns of \emph{both} the row-echelon matrix and the \emph{original}
matrix are columns 2,4,5,6.
\end{itemize}
\end{frame}

%%%%%%%%%%%%%%%%%%%%%%%%%%%%%%%%%%%%%%%%%%%%%%%%%%%%%%%%%%%%%%%%%%%%%%%%%

\begin{frame}{Gaussian Elimination Algorithm by Example}

\begin{itemize}
\item Just to emphasize this point:
\item $
\begin{pmatrix}
0 & 0  &   0  &  0   &  1  &  2 \\
0 & 1  &  17  & -11  &  0  &  1  \\
0 & 2  &  34  & -20  &  0  &  0 \\
0 & -1 &  -17   &  6 &  3  &  1 \\
\end{pmatrix}
$
\item The rank of this matrix is 4.
\item The basic columns of this matrix are 2,4,5,6.
\end{itemize}
\end{frame}

%%%%%%%%%%%%%%%%%%%%%%%%%%%%%%%%%%%%%%%%%%%%%%%%%%%%%%%%%%%%%%%%%%%%%%%%%

\begin{frame}{Why are the basic columns the same?}

\begin{itemize}
\item Why is it true that the basic columns of the row-echelon matrix are
the same as the basic columns of the original matrix?
\item It is because a linear bijection is an \emph{isomorphism.}
\item In mathematics, an isopmorphism is a bijection between two structures
that preserves all of the form of the structures.
\item \textbf{Theorem.} Let $V$ and $W$ be vector spaces and let
$T:V\map W$ be a linear bijection.
\item Let $\bv_1,\bv_2,\cdots,\bv_n$ be vectors in $V$.
\item Then $\bv_1,\bv_2,\cdots,\bv_n$ are linearly independent in $V$
iff $T(\bv_1),T(\bv_2),\cdots T(\bv_n)$ are linearly independent in $W$.
\textbf{proof.} Left to an exercise. (optional)
\end{itemize}
\end{frame}

%%%%%%%%%%%%%%%%%%%%%%%%%%%%%%%%%%%%%%%%%%%%%%%%%%%%%%%%%%%%%%%%%%%%%%%%%

\begin{frame}{Why are the basic columns the same?}

\begin{itemize}
\item Now suppose $A$ is an $n\times m$ matrix and $G$ is an $n\times n$ invertible matrix
and $B=GA$.
\item Write $A=(\bv_1 \vert \bv_2 \vert \cdots \vert \bv_m)$ and $B=(\bw_1 \vert \bw_2 \vert \cdots \bw_m)$.
\item $T_G:\R^n\map \R^n$ is a linear bijection and so it is an isomorphism.
\item Notice that $T_G$ sends each column of $A$ to the corresponding column of $B$:
\item $T_G(\bv_1) = \bw_1, T_G(\bv_2) = \bw_2, \cdots T_G(\bv_m) = \bw_m$.
\item By the previous theorem, any set of the $\bv_i$ are linearly independent iff the coresponding set of the $\bw_i$
are linearly independent.
\item So the computation of which are the basic columns of $A$ yields the same result as the computation of which are the basic columns of $B$. $\qed$

\end{itemize}
\end{frame}

%%%%%%%%%%%%%%%%%%%%%%%%%%%%%%%%%%%%%%%%%%%%%%%%%%%%%%%%%%%%%%%%%%%%%%%%%

\begin{frame}{Gaussian Elimination to Solve Systems of Linear Equations}

Solve the following system of equations:

\pause

\begin{align*}
 \quad v - w         &= 3 \\
-2u + 4v - w         &= 1 \\
-2u + 5v - 4w        &= -2 \\
\end{align*}

\pause

\begin{itemize}
\item Let $A=
\begin{pmatrix}
 0 & 1 & -1 \\
-2 & 4 & -1 \\
-2 & 5 & -4 \\
\end{pmatrix}
$,
let
$
\bb =
\begin{pmatrix}
3 \\ 1 \\ -2
\end{pmatrix}
$
\item Solve $A\bx=\bb$.
\end{itemize}
\end{frame}

%%%%%%%%%%%%%%%%%%%%%%%%%%%%%%%%%%%%%%%%%%%%%%%%%%%%%%%%%%%%%%%%%%%%%%%%%

\begin{frame}{Gaussian Elimination to Solve Systems of Linear Equations}

\begin{itemize}
\item \textbf{Lemma.} If $G$ is invertible then $A\bx=\bb$ iff $GA\bx=G\bb$.
\item \textbf{proof.} First suppose that $A\bx=\bb$.
\item Then we can multiply $G$ on the left on both sides of the equation
to get $GA\bx=G\bb$.
\item Now suppose that $GA\bx=G\bb$.
\item Since $G$ is invertible we can multiply both sides of the equation
on the left by $G^{-1}$.
\item $G^{-1}GA\bx = G^{-1}G\bb$.
\item $A\bx=\bb$. $\qed$
\end{itemize}
\end{frame}

%%%%%%%%%%%%%%%%%%%%%%%%%%%%%%%%%%%%%%%%%%%%%%%%%%%%%%%%%%%%%%%%%%%%%%%%%

\begin{frame}{Gaussian Elimination to Solve Systems of Linear Equations}

\begin{itemize}
\item Now suppose $G = E_k \cdots E_2 E_1$ is a product of elementary matrices of types I and II.
\item And suppose $B = G A$ is in row echelon form.
\item Then we have $A\bx=\bb$ iff $GA\bx = G\bb$
\item iff $E_k \cdots E_2 E_1 A \bx = E_k \cdots E_2 E_1 \bb$.
\item So we need to perform the same elementary row operations to $\bb$ as we do to
$A$.
\item In order to keep track of this we form the \emph{augmented} matrix.
\end{itemize}
\end{frame}

%%%%%%%%%%%%%%%%%%%%%%%%%%%%%%%%%%%%%%%%%%%%%%%%%%%%%%%%%%%%%%%%%%%%%%%%%

\begin{frame}{The Augmented Matrix}

\begin{itemize}
\item \textbf{Definition.} If $A\bx=\bb$ is a system of linear equations then the
\emph{augmented matrix} for the system is the matrix whose columns are the columns
of $A$ with one additional column on the right consisting of $\bb$. We draw
a vertical line between the columns of $A$ and $\bb$.
\item For the system we are currently working with
$A=
\begin{pmatrix}
 0 & 1 & -1 \\
-2 & 4 & -1 \\
-2 & 5 & -4 \\
\end{pmatrix}
$,
and
$
\bb =
\begin{pmatrix}
3 \\ 1 \\ -2
\end{pmatrix}
$.
\item The augmented matrix is
$
\begin{pmatrix}
 0 & 1 & -1  &  \aug & 3  \\
-2 & 4 & -1  &  \aug & 1  \\
-2 & 5 & -4  &  \aug & -2  \\
\end{pmatrix}
$
\end{itemize}
\end{frame}

%%%%%%%%%%%%%%%%%%%%%%%%%%%%%%%%%%%%%%%%%%%%%%%%%%%%%%%%%%%%%%%%%%%%%%%%%

\begin{frame}{Gaussian Elimination with an Augmented Matrix}

\begin{itemize}
\item When performing Gaussian elimination on an augmented matrix, we perform
the row operations on the full augmented matrix,
\item but the goal is to get only the main matrix (to the left of the vertical line)
to be in row-echelon form.
\item We do not include the extra column when considering row-echelon form.
\item This is a simple bookkeeping trick that allows us to multiply $A$ and $\bb$
by the same sequence of elementary matrices.
\end{itemize}
\end{frame}

%%%%%%%%%%%%%%%%%%%%%%%%%%%%%%%%%%%%%%%%%%%%%%%%%%%%%%%%%%%%%%%%%%%%%%%%%

\begin{frame}{Example Gaussian Elimination with an Augmented Matrix}

\begin{itemize}
\item $
\begin{pmatrix}
 0 & 1 & -1  &  \aug & 3  \\
-2 & 4 & -1  &  \aug & 1  \\
-2 & 5 & -4  &  \aug & -2  \\
\end{pmatrix}
$
\item Swap rows 1 and 2
\item $
\begin{pmatrix}
\textbf{-2} & 4 & -1  &  \aug & 1  \\
        0   & 1 & -1  &  \aug & 3  \\
        -2  & 5 & -4  &  \aug & -2  \\
\end{pmatrix}
$
\item Replace row 3 with row 3 minus row 1
\item $
\begin{pmatrix}
\textbf{-2} & 4 & -1  &  \aug & 1  \\
        0   & 1 & -1  &  \aug & 3  \\
        0   & 1 & -3  &  \aug & -3  \\
\end{pmatrix}
$
\end{itemize}
\end{frame}

%%%%%%%%%%%%%%%%%%%%%%%%%%%%%%%%%%%%%%%%%%%%%%%%%%%%%%%%%%%%%%%%%%%%%%%%%

\begin{frame}{Example Gaussian Elimination with an Augmented Matrix}

\begin{itemize}
\item $
\begin{pmatrix}
\textbf{-2} &         4  & -1  &  \aug & 1  \\
        0   & \textbf{1} & -1  &  \aug & 3  \\
        0   &         1  & -3  &  \aug & -3  \\
\end{pmatrix}
$
\item Replace row 3 with row 3 minus row 2.
\item $
\begin{pmatrix}
\textbf{-2} &         4  &          -1  &  \aug & 1  \\
        0   & \textbf{1} &          -1  &  \aug & 3  \\
        0   &         0  & \textbf{-2}  &  \aug & -6  \\
\end{pmatrix}
$
\item That is now in row-echelon form.
\item Actually upper-triangular with nonzeros on the diagonal. So both matrices,
the final one and the original one, are invertible with rank 3.
\end{itemize}
\end{frame}

%%%%%%%%%%%%%%%%%%%%%%%%%%%%%%%%%%%%%%%%%%%%%%%%%%%%%%%%%%%%%%%%%%%%%%%%%

\begin{frame}{Back substitution.}

\begin{itemize}
\item $
\begin{pmatrix}
\textbf{-2} &         4  &          -1  &  \aug & 1  \\
        0   & \textbf{1} &          -1  &  \aug & 3  \\
        0   &         0  & \textbf{-2}  &  \aug & -6  \\
\end{pmatrix}
$
\item To solve the equation we use back-substitution.
\item $-2 w = -6 \quad \implies \quad w = 3$
\item $v - w = 3 \quad \implies \quad v - 3 = 3 \quad \implies \quad v=6$.
\item $-2 u + v4 -w  = 1 \quad \implies \quad -2u +24 - 3 = 1 \quad \implies \quad u = 10$.
\item This is the solution to the \emph{original} system too.
\end{itemize}
\end{frame}

%%%%%%%%%%%%%%%%%%%%%%%%%%%%%%%%%%%%%%%%%%%%%%%%%%%%%%%%%%%%%%%%%%%%%%%%%

\begin{frame}{Gaussian Elimination to Solve Systems of Linear Equations}

We have solved the original system of linear equations.

\pause

\begin{align*}
 \quad v - w         &= 3 \\
-2u + 4v - w         &= 1 \\
-2u + 5v - 4w        &= -2 \\
\end{align*}

\pause

The solution is $u = 10, v=6, w=3$.

\end{frame}

%%%%%%%%%%%%%%%%%%%%%%%%%%%%%%%%%%%%%%%%%%%%%%%%%%%%%%%%%%%%%%%%%%%%%%%%%

\begin{frame}{Lower unitriangular matrices}

\begin{itemize}

\item \textbf{Definition.} A square matrix is \emph{lower-triangular}
iff all entries below the main diagonal are zero.

\item \textbf{Definition.} A square matrix is \emph{lower unitriangular}
iff it is lower-triangular and all if its diagonal entries are 1.

\item The elementary matrices of type 1 that are used in Gaussian elimination
are all lower unitriangular.

\item For example the $3\times 3$ matrix that adds 2 times row 2 to row 3.

\item
$
\begin{pmatrix}
1 & 0 & 0 \\
0 & 1 & 0 \\
0 & 2 & 1 \\
\end{pmatrix}
$

\end{itemize}

\end{frame}

%%%%%%%%%%%%%%%%%%%%%%%%%%%%%%%%%%%%%%%%%%%%%%%%%%%%%%%%%%%%%%%%%%%%%%%%%

\begin{frame}{$LU$ Factorization}

\begin{itemize}

\item Notice that a square matrix $A$ is non-singular iff when we apply
Guassian elimination on it to bring it to row-echelon form, the resulting
form is actually upper-triangular with non-zeros on the diagonal.

\item Consider the special case of a non-singular matrix $A$ that has the
property that during Gaussian elimination, no elementary matrices of
type II need to be used. Only elementary matrices of type I.

\item In that case Gaussian eliminiation gives us:

\item $L_k\cdots L_2 L_1 A = U$

\item where each $L_i$ is lower unitriangular and $U$ us upper-triangular with
nonzeros on the diagonal.

\end{itemize}

\end{frame}

%%%%%%%%%%%%%%%%%%%%%%%%%%%%%%%%%%%%%%%%%%%%%%%%%%%%%%%%%%%%%%%%%%%%%%%%%

\begin{frame}{$LU$ Factorization}

\begin{itemize}

\item $L_k\cdots L_2 L_1 A = U$

\item Multiplying both sides on the left by the inverse of $L_k\cdots L_2 L_1$ gives:

\item $A = L_1^{-1} L_2^{-1} \cdots L_k^{-1} U$.

\item Notice that each of the $L_i^{-1}$ are also lower unitriangular.
\item (Recall how we form the inverse of an elementary matrix of type I.)

\item Let $L = L_1^{-1} L_2^{-1} \cdots L_k^{-1}$.

\item Then $L$  is also lower unitriangular. (Optional exercise.)

\item So we have $A = LU$.

\end{itemize}

\end{frame}

%%%%%%%%%%%%%%%%%%%%%%%%%%%%%%%%%%%%%%%%%%%%%%%%%%%%%%%%%%%%%%%%%%%%%%%%%

\begin{frame}{$LU$ Factorization}

\begin{itemize}

\item Let $A$ be a non-singular $n\times n$ matrix and suppose that
during Guassian elimination there is no need for any swaps.

\item Then $A$ can be \emph{factored} into $A=LU$, where

\item $L$ is lower unitriangular
\item and $U$ is upper triangular with nonzeros on the diagonal.

\end{itemize}

\end{frame}




\end{document}


