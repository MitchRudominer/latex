% $Header$

\documentclass{beamer}

\usepackage{amsmath,amssymb,latexsym,eucal,amsthm,graphicx}
%%%%%%%%%%%%%%%%%%%%%%%%%%%%%%%%%%%%%%%%%%%%%
% Common Set Theory Constructs
%%%%%%%%%%%%%%%%%%%%%%%%%%%%%%%%%%%%%%%%%%%%%

\newcommand{\setof}[2]{\left\{ \, #1 \, \left| \, #2 \, \right.\right\}}
\newcommand{\lsetof}[2]{\left\{\left. \, #1 \, \right| \, #2 \,  \right\}}
\newcommand{\bigsetof}[2]{\bigl\{ \, #1 \, \bigm | \, #2 \,\bigr\}}
\newcommand{\Bigsetof}[2]{\Bigl\{ \, #1 \, \Bigm | \, #2 \,\Bigr\}}
\newcommand{\biggsetof}[2]{\biggl\{ \, #1 \, \biggm | \, #2 \,\biggr\}}
\newcommand{\Biggsetof}[2]{\Biggl\{ \, #1 \, \Biggm | \, #2 \,\Biggr\}}
\newcommand{\dotsetof}[2]{\left\{ \, #1 \, : \, #2 \, \right\}}
\newcommand{\bigdotsetof}[2]{\bigl\{ \, #1 \, : \, #2 \,\bigr\}}
\newcommand{\Bigdotsetof}[2]{\Bigl\{ \, #1 \, \Bigm : \, #2 \,\Bigr\}}
\newcommand{\biggdotsetof}[2]{\biggl\{ \, #1 \, \biggm : \, #2 \,\biggr\}}
\newcommand{\Biggdotsetof}[2]{\Biggl\{ \, #1 \, \Biggm : \, #2 \,\Biggr\}}
\newcommand{\sequence}[2]{\left\langle \, #1 \,\left| \, #2 \, \right. \right\rangle}
\newcommand{\lsequence}[2]{\left\langle\left. \, #1 \, \right| \,#2 \,  \right\rangle}
\newcommand{\bigsequence}[2]{\bigl\langle \,#1 \, \bigm | \, #2 \, \bigr\rangle}
\newcommand{\Bigsequence}[2]{\Bigl\langle \,#1 \, \Bigm | \, #2 \, \Bigr\rangle}
\newcommand{\biggsequence}[2]{\biggl\langle \,#1 \, \biggm | \, #2 \, \biggr\rangle}
\newcommand{\Biggsequence}[2]{\Biggl\langle \,#1 \, \Biggm | \, #2 \, \Biggr\rangle}
\newcommand{\singleton}[1]{\left\{#1\right\}}
\newcommand{\angles}[1]{\left\langle #1 \right\rangle}
\newcommand{\bigangles}[1]{\bigl\langle #1 \bigr\rangle}
\newcommand{\Bigangles}[1]{\Bigl\langle #1 \Bigr\rangle}
\newcommand{\biggangles}[1]{\biggl\langle #1 \biggr\rangle}
\newcommand{\Biggangles}[1]{\Biggl\langle #1 \Biggr\rangle}


\newcommand{\force}[1]{\Vert\!\underset{\!\!\!\!\!#1}{\!\!\!\relbar\!\!\!%
\relbar\!\!\relbar\!\!\relbar\!\!\!\relbar\!\!\relbar\!\!\relbar\!\!\!%
\relbar\!\!\relbar\!\!\relbar}}
\newcommand{\longforce}[1]{\Vert\!\underset{\!\!\!\!\!#1}{\!\!\!\relbar\!\!\!%
\relbar\!\!\relbar\!\!\relbar\!\!\!\relbar\!\!\relbar\!\!\relbar\!\!\!%
\relbar\!\!\relbar\!\!\relbar\!\!\relbar\!\!\relbar\!\!\relbar\!\!\relbar\!\!\relbar}}
\newcommand{\nforce}[1]{\Vert\!\underset{\!\!\!\!\!#1}{\!\!\!\relbar\!\!\!%
\relbar\!\!\relbar\!\!\relbar\!\!\!\relbar\!\!\relbar\!\!\relbar\!\!\!%
\relbar\!\!\not\relbar\!\!\relbar}}
\newcommand{\forcein}[2]{\overset{#2}{\Vert\underset{\!\!\!\!\!#1}%
{\!\!\!\relbar\!\!\!\relbar\!\!\relbar\!\!\relbar\!\!\!\relbar\!\!\relbar\!%
\!\relbar\!\!\!\relbar\!\!\relbar\!\!\relbar\!\!\relbar\!\!\!\relbar\!\!%
\relbar\!\!\relbar}}}

\newcommand{\pre}[2]{{}^{#2}{#1}}

\newcommand{\restr}{\!\!\upharpoonright\!}

%%%%%%%%%%%%%%%%%%%%%%%%%%%%%%%%%%%%%%%%%%%%%
% Set-Theoretic Connectives
%%%%%%%%%%%%%%%%%%%%%%%%%%%%%%%%%%%%%%%%%%%%%

\newcommand{\intersect}{\cap}
\newcommand{\union}{\cup}
\newcommand{\Intersection}[1]{\bigcap\limits_{#1}}
\newcommand{\Union}[1]{\bigcup\limits_{#1}}
\newcommand{\adjoin}{{}^\frown}
\newcommand{\forces}{\Vdash}

%%%%%%%%%%%%%%%%%%%%%%%%%%%%%%%%%%%%%%%%%%%%%
% Miscellaneous
%%%%%%%%%%%%%%%%%%%%%%%%%%%%%%%%%%%%%%%%%%%%%
\newcommand{\defeq}{=_{\text{def}}}
\newcommand{\Turingleq}{\leq_{\text{T}}}
\newcommand{\Turingless}{<_{\text{T}}}
\newcommand{\lexleq}{\leq_{\text{lex}}}
\newcommand{\lexless}{<_{\text{lex}}}
\newcommand{\Turingequiv}{\equiv_{\text{T}}}
\newcommand{\isomorphic}{\cong}

%%%%%%%%%%%%%%%%%%%%%%%%%%%%%%%%%%%%%%%%%%%%%
% Constants
%%%%%%%%%%%%%%%%%%%%%%%%%%%%%%%%%%%%%%%%%%%%%
\newcommand{\R}{\mathbb{R}}
\renewcommand{\P}{\mathbb{P}}
\newcommand{\Q}{\mathbb{Q}}
\newcommand{\Z}{\mathbb{Z}}
\newcommand{\Zpos}{\Z^{+}}
\newcommand{\Znonneg}{\Z^{\geq 0}}
\newcommand{\C}{\mathbb{C}}
\newcommand{\N}{\mathbb{N}}
\newcommand{\B}{\mathbb{B}}
\newcommand{\Bairespace}{\pre{\omega}{\omega}}
\newcommand{\LofR}{L(\R)}
\newcommand{\JofR}[1]{J_{#1}(\R)}
\newcommand{\SofR}[1]{S_{#1}(\R)}
\newcommand{\JalphaR}{\JofR{\alpha}}
\newcommand{\JbetaR}{\JofR{\beta}}
\newcommand{\JlambdaR}{\JofR{\lambda}}
\newcommand{\SalphaR}{\SofR{\alpha}}
\newcommand{\SbetaR}{\SofR{\beta}}
\newcommand{\Pkl}{\mathcal{P}_{\kappa}(\lambda)}
\DeclareMathOperator{\con}{con}
\DeclareMathOperator{\ORD}{OR}
\DeclareMathOperator{\Ord}{OR}
\DeclareMathOperator{\WO}{WO}
\DeclareMathOperator{\OD}{OD}
\DeclareMathOperator{\HOD}{HOD}
\DeclareMathOperator{\HC}{HC}
\DeclareMathOperator{\WF}{WF}
\DeclareMathOperator{\wfp}{wfp}
\DeclareMathOperator{\HF}{HF}
\newcommand{\One}{I}
\newcommand{\ONE}{I}
\newcommand{\Two}{II}
\newcommand{\TWO}{II}
\newcommand{\Mladder}{M^{\text{ld}}}

%%%%%%%%%%%%%%%%%%%%%%%%%%%%%%%%%%%%%%%%%%%%%
% Commutative Algebra Constants
%%%%%%%%%%%%%%%%%%%%%%%%%%%%%%%%%%%%%%%%%%%%%
\DeclareMathOperator{\dottimes}{\dot{\times}}
\DeclareMathOperator{\dotminus}{\dot{-}}
\DeclareMathOperator{\Spec}{Spec}

%%%%%%%%%%%%%%%%%%%%%%%%%%%%%%%%%%%%%%%%%%%%%
% Theories
%%%%%%%%%%%%%%%%%%%%%%%%%%%%%%%%%%%%%%%%%%%%%
\DeclareMathOperator{\ZFC}{ZFC}
\DeclareMathOperator{\ZF}{ZF}
\DeclareMathOperator{\AD}{AD}
\DeclareMathOperator{\ADR}{AD_{\R}}
\DeclareMathOperator{\KP}{KP}
\DeclareMathOperator{\PD}{PD}
\DeclareMathOperator{\CH}{CH}
\DeclareMathOperator{\GCH}{GCH}
\DeclareMathOperator{\HPC}{HPC} % HOD pair capturing
%%%%%%%%%%%%%%%%%%%%%%%%%%%%%%%%%%%%%%%%%%%%%
% Iteration Trees
%%%%%%%%%%%%%%%%%%%%%%%%%%%%%%%%%%%%%%%%%%%%%

\newcommand{\pred}{\text{-pred}}

%%%%%%%%%%%%%%%%%%%%%%%%%%%%%%%%%%%%%%%%%%%%%%%%
% Operator Names
%%%%%%%%%%%%%%%%%%%%%%%%%%%%%%%%%%%%%%%%%%%%%%%%
\DeclareMathOperator{\Det}{Det}
\DeclareMathOperator{\dom}{dom}
\DeclareMathOperator{\ran}{ran}
\DeclareMathOperator{\range}{ran}
\DeclareMathOperator{\image}{image}
\DeclareMathOperator{\crit}{crit}
\DeclareMathOperator{\card}{card}
\DeclareMathOperator{\cf}{cf}
\DeclareMathOperator{\cof}{cof}
\DeclareMathOperator{\rank}{rank}
\DeclareMathOperator{\ot}{o.t.}
\DeclareMathOperator{\ords}{o}
\DeclareMathOperator{\ro}{r.o.}
\DeclareMathOperator{\rud}{rud}
\DeclareMathOperator{\Powerset}{\mathcal{P}}
\DeclareMathOperator{\length}{lh}
\DeclareMathOperator{\lh}{lh}
\DeclareMathOperator{\limit}{lim}
\DeclareMathOperator{\fld}{fld}
\DeclareMathOperator{\projection}{p}
\DeclareMathOperator{\Ult}{Ult}
\DeclareMathOperator{\ULT}{Ult}
\DeclareMathOperator{\Coll}{Coll}
\DeclareMathOperator{\Col}{Col}
\DeclareMathOperator{\Hull}{Hull}
\DeclareMathOperator*{\dirlim}{dir lim}
\DeclareMathOperator{\Scale}{Scale}
\DeclareMathOperator{\supp}{supp}
\DeclareMathOperator{\trancl}{tran.cl.}
\DeclareMathOperator{\trace}{Tr}
\DeclareMathOperator{\diag}{diag}
\DeclareMathOperator{\spn}{span}
\DeclareMathOperator{\sgn}{sgn}
%%%%%%%%%%%%%%%%%%%%%%%%%%%%%%%%%%%%%%%%%%%%%
% Logical Connectives
%%%%%%%%%%%%%%%%%%%%%%%%%%%%%%%%%%%%%%%%%%%%%
\newcommand{\IImplies}{\Longrightarrow}
\newcommand{\SkipImplies}{\quad\Longrightarrow\quad}
\newcommand{\Ifff}{\Longleftrightarrow}
\newcommand{\iimplies}{\longrightarrow}
\newcommand{\ifff}{\longleftrightarrow}
\newcommand{\Implies}{\Rightarrow}
\newcommand{\Iff}{\Leftrightarrow}
\renewcommand{\implies}{\rightarrow}
\renewcommand{\iff}{\leftrightarrow}
\newcommand{\AND}{\wedge}
\newcommand{\OR}{\vee}
\newcommand{\st}{\text{ s.t. }}
\newcommand{\Or}{\text{ or }}

%%%%%%%%%%%%%%%%%%%%%%%%%%%%%%%%%%%%%%%%%%%%%
% Function Arrows
%%%%%%%%%%%%%%%%%%%%%%%%%%%%%%%%%%%%%%%%%%%%%

\newcommand{\injection}{\xrightarrow{\text{1-1}}}
\newcommand{\surjection}{\xrightarrow{\text{onto}}}
\newcommand{\bijection}{\xrightarrow[\text{onto}]{\text{1-1}}}
\newcommand{\cofmap}{\xrightarrow{\text{cof}}}
\newcommand{\map}{\rightarrow}

%%%%%%%%%%%%%%%%%%%%%%%%%%%%%%%%%%%%%%%%%%%%%
% Mouse Comparison Operators
%%%%%%%%%%%%%%%%%%%%%%%%%%%%%%%%%%%%%%%%%%%%%
\newcommand{\initseg}{\trianglelefteq}
\newcommand{\properseg}{\lhd}
\newcommand{\notinitseg}{\ntrianglelefteq}
\newcommand{\notproperseg}{\ntriangleleft}

%%%%%%%%%%%%%%%%%%%%%%%%%%%%%%%%%%%%%%%%%%%%%
%           calligraphic letters
%%%%%%%%%%%%%%%%%%%%%%%%%%%%%%%%%%%%%%%%%%%%%
\newcommand{\cA}{\mathcal{A}}
\newcommand{\cB}{\mathcal{B}}
\newcommand{\cC}{\mathcal{C}}
\newcommand{\cD}{\mathcal{D}}
\newcommand{\cE}{\mathcal{E}}
\newcommand{\cF}{\mathcal{F}}
\newcommand{\cG}{\mathcal{G}}
\newcommand{\cH}{\mathcal{H}}
\newcommand{\cI}{\mathcal{I}}
\newcommand{\cJ}{\mathcal{J}}
\newcommand{\cK}{\mathcal{K}}
\newcommand{\cL}{\mathcal{L}}
\newcommand{\cM}{\mathcal{M}}
\newcommand{\cN}{\mathcal{N}}
\newcommand{\cO}{\mathcal{O}}
\newcommand{\cP}{\mathcal{P}}
\newcommand{\cQ}{\mathcal{Q}}
\newcommand{\cR}{\mathcal{R}}
\newcommand{\cS}{\mathcal{S}}
\newcommand{\cT}{\mathcal{T}}
\newcommand{\cU}{\mathcal{U}}
\newcommand{\cV}{\mathcal{V}}
\newcommand{\cW}{\mathcal{W}}
\newcommand{\cX}{\mathcal{X}}
\newcommand{\cY}{\mathcal{Y}}
\newcommand{\cZ}{\mathcal{Z}}


%%%%%%%%%%%%%%%%%%%%%%%%%%%%%%%%%%%%%%%%%%%%%
%          Primed Letters
%%%%%%%%%%%%%%%%%%%%%%%%%%%%%%%%%%%%%%%%%%%%%
\newcommand{\aprime}{a^{\prime}}
\newcommand{\bprime}{b^{\prime}}
\newcommand{\cprime}{c^{\prime}}
\newcommand{\dprime}{d^{\prime}}
\newcommand{\eprime}{e^{\prime}}
\newcommand{\fprime}{f^{\prime}}
\newcommand{\gprime}{g^{\prime}}
\newcommand{\hprime}{h^{\prime}}
\newcommand{\iprime}{i^{\prime}}
\newcommand{\jprime}{j^{\prime}}
\newcommand{\kprime}{k^{\prime}}
\newcommand{\lprime}{l^{\prime}}
\newcommand{\mprime}{m^{\prime}}
\newcommand{\nprime}{n^{\prime}}
\newcommand{\oprime}{o^{\prime}}
\newcommand{\pprime}{p^{\prime}}
\newcommand{\qprime}{q^{\prime}}
\newcommand{\rprime}{r^{\prime}}
\newcommand{\sprime}{s^{\prime}}
\newcommand{\tprime}{t^{\prime}}
\newcommand{\uprime}{u^{\prime}}
\newcommand{\vprime}{v^{\prime}}
\newcommand{\wprime}{w^{\prime}}
\newcommand{\xprime}{x^{\prime}}
\newcommand{\yprime}{y^{\prime}}
\newcommand{\zprime}{z^{\prime}}
\newcommand{\Aprime}{A^{\prime}}
\newcommand{\Bprime}{B^{\prime}}
\newcommand{\Cprime}{C^{\prime}}
\newcommand{\Dprime}{D^{\prime}}
\newcommand{\Eprime}{E^{\prime}}
\newcommand{\Fprime}{F^{\prime}}
\newcommand{\Gprime}{G^{\prime}}
\newcommand{\Hprime}{H^{\prime}}
\newcommand{\Iprime}{I^{\prime}}
\newcommand{\Jprime}{J^{\prime}}
\newcommand{\Kprime}{K^{\prime}}
\newcommand{\Lprime}{L^{\prime}}
\newcommand{\Mprime}{M^{\prime}}
\newcommand{\Nprime}{N^{\prime}}
\newcommand{\Oprime}{O^{\prime}}
\newcommand{\Pprime}{P^{\prime}}
\newcommand{\Qprime}{Q^{\prime}}
\newcommand{\Rprime}{R^{\prime}}
\newcommand{\Sprime}{S^{\prime}}
\newcommand{\Tprime}{T^{\prime}}
\newcommand{\Uprime}{U^{\prime}}
\newcommand{\Vprime}{V^{\prime}}
\newcommand{\Wprime}{W^{\prime}}
\newcommand{\Xprime}{X^{\prime}}
\newcommand{\Yprime}{Y^{\prime}}
\newcommand{\Zprime}{Z^{\prime}}
\newcommand{\alphaprime}{\alpha^{\prime}}
\newcommand{\betaprime}{\beta^{\prime}}
\newcommand{\gammaprime}{\gamma^{\prime}}
\newcommand{\Gammaprime}{\Gamma^{\prime}}
\newcommand{\deltaprime}{\delta^{\prime}}
\newcommand{\epsilonprime}{\epsilon^{\prime}}
\newcommand{\kappaprime}{\kappa^{\prime}}
\newcommand{\lambdaprime}{\lambda^{\prime}}
\newcommand{\rhoprime}{\rho^{\prime}}
\newcommand{\Sigmaprime}{\Sigma^{\prime}}
\newcommand{\tauprime}{\tau^{\prime}}
\newcommand{\xiprime}{\xi^{\prime}}
\newcommand{\thetaprime}{\theta^{\prime}}
\newcommand{\Omegaprime}{\Omega^{\prime}}
\newcommand{\cMprime}{\cM^{\prime}}
\newcommand{\cNprime}{\cN^{\prime}}
\newcommand{\cPprime}{\cP^{\prime}}
\newcommand{\cQprime}{\cQ^{\prime}}
\newcommand{\cRprime}{\cR^{\prime}}
\newcommand{\cSprime}{\cS^{\prime}}
\newcommand{\cTprime}{\cT^{\prime}}

%%%%%%%%%%%%%%%%%%%%%%%%%%%%%%%%%%%%%%%%%%%%%
%          bar Letters
%%%%%%%%%%%%%%%%%%%%%%%%%%%%%%%%%%%%%%%%%%%%%
\newcommand{\abar}{\bar{a}}
\newcommand{\bbar}{\bar{b}}
\newcommand{\cbar}{\bar{c}}
\newcommand{\ibar}{\bar{i}}
\newcommand{\jbar}{\bar{j}}
\newcommand{\nbar}{\bar{n}}
\newcommand{\xbar}{\bar{x}}
\newcommand{\ybar}{\bar{y}}
\newcommand{\zbar}{\bar{z}}
\newcommand{\pibar}{\bar{\pi}}
\newcommand{\phibar}{\bar{\varphi}}
\newcommand{\psibar}{\bar{\psi}}
\newcommand{\thetabar}{\bar{\theta}}
\newcommand{\nubar}{\bar{\nu}}

%%%%%%%%%%%%%%%%%%%%%%%%%%%%%%%%%%%%%%%%%%%%%
%          star Letters
%%%%%%%%%%%%%%%%%%%%%%%%%%%%%%%%%%%%%%%%%%%%%
\newcommand{\phistar}{\phi^{*}}
\newcommand{\Mstar}{M^{*}}

%%%%%%%%%%%%%%%%%%%%%%%%%%%%%%%%%%%%%%%%%%%%%
%          dotletters Letters
%%%%%%%%%%%%%%%%%%%%%%%%%%%%%%%%%%%%%%%%%%%%%
\newcommand{\Gdot}{\dot{G}}

%%%%%%%%%%%%%%%%%%%%%%%%%%%%%%%%%%%%%%%%%%%%%
%         check Letters
%%%%%%%%%%%%%%%%%%%%%%%%%%%%%%%%%%%%%%%%%%%%%
\newcommand{\deltacheck}{\check{\delta}}
\newcommand{\gammacheck}{\check{\gamma}}


%%%%%%%%%%%%%%%%%%%%%%%%%%%%%%%%%%%%%%%%%%%%%
%          Formulas
%%%%%%%%%%%%%%%%%%%%%%%%%%%%%%%%%%%%%%%%%%%%%

\newcommand{\formulaphi}{\text{\large $\varphi$}}
\newcommand{\Formulaphi}{\text{\Large $\varphi$}}


%%%%%%%%%%%%%%%%%%%%%%%%%%%%%%%%%%%%%%%%%%%%%
%          Fraktur Letters
%%%%%%%%%%%%%%%%%%%%%%%%%%%%%%%%%%%%%%%%%%%%%

\newcommand{\fa}{\mathfrak{a}}
\newcommand{\fb}{\mathfrak{b}}
\newcommand{\fc}{\mathfrak{c}}
\newcommand{\fk}{\mathfrak{k}}
\newcommand{\fp}{\mathfrak{p}}
\newcommand{\fq}{\mathfrak{q}}
\newcommand{\fr}{\mathfrak{r}}
\newcommand{\fA}{\mathfrak{A}}
\newcommand{\fB}{\mathfrak{B}}
\newcommand{\fC}{\mathfrak{C}}
\newcommand{\fD}{\mathfrak{D}}

%%%%%%%%%%%%%%%%%%%%%%%%%%%%%%%%%%%%%%%%%%%%%
%          Bold Letters
%%%%%%%%%%%%%%%%%%%%%%%%%%%%%%%%%%%%%%%%%%%%%
\newcommand{\ba}{\mathbf{a}}
\newcommand{\bb}{\mathbf{b}}
\newcommand{\bc}{\mathbf{c}}
\newcommand{\bd}{\mathbf{d}}
\newcommand{\be}{\mathbf{e}}
\newcommand{\bu}{\mathbf{u}}
\newcommand{\bv}{\mathbf{v}}
\newcommand{\bw}{\mathbf{w}}
\newcommand{\bx}{\mathbf{x}}
\newcommand{\by}{\mathbf{y}}
\newcommand{\bz}{\mathbf{z}}
\newcommand{\bSigma}{\boldsymbol{\Sigma}}
\newcommand{\bPi}{\boldsymbol{\Pi}}
\newcommand{\bDelta}{\boldsymbol{\Delta}}
\newcommand{\bdelta}{\boldsymbol{\delta}}
\newcommand{\bgamma}{\boldsymbol{\gamma}}
\newcommand{\bGamma}{\boldsymbol{\Gamma}}

%%%%%%%%%%%%%%%%%%%%%%%%%%%%%%%%%%%%%%%%%%%%%
%         Bold numbers
%%%%%%%%%%%%%%%%%%%%%%%%%%%%%%%%%%%%%%%%%%%%%
\newcommand{\bzero}{\mathbf{0}}

%%%%%%%%%%%%%%%%%%%%%%%%%%%%%%%%%%%%%%%%%%%%%
% Projective-Like Pointclasses
%%%%%%%%%%%%%%%%%%%%%%%%%%%%%%%%%%%%%%%%%%%%%
\newcommand{\Sa}[2][\alpha]{\Sigma_{(#1,#2)}}
\newcommand{\Pa}[2][\alpha]{\Pi_{(#1,#2)}}
\newcommand{\Da}[2][\alpha]{\Delta_{(#1,#2)}}
\newcommand{\Aa}[2][\alpha]{A_{(#1,#2)}}
\newcommand{\Ca}[2][\alpha]{C_{(#1,#2)}}
\newcommand{\Qa}[2][\alpha]{Q_{(#1,#2)}}
\newcommand{\da}[2][\alpha]{\delta_{(#1,#2)}}
\newcommand{\leqa}[2][\alpha]{\leq_{(#1,#2)}}
\newcommand{\lessa}[2][\alpha]{<_{(#1,#2)}}
\newcommand{\equiva}[2][\alpha]{\equiv_{(#1,#2)}}


\newcommand{\Sl}[1]{\Sa[\lambda]{#1}}
\newcommand{\Pl}[1]{\Pa[\lambda]{#1}}
\newcommand{\Dl}[1]{\Da[\lambda]{#1}}
\newcommand{\Al}[1]{\Aa[\lambda]{#1}}
\newcommand{\Cl}[1]{\Ca[\lambda]{#1}}
\newcommand{\Ql}[1]{\Qa[\lambda]{#1}}

\newcommand{\San}{\Sa{n}}
\newcommand{\Pan}{\Pa{n}}
\newcommand{\Dan}{\Da{n}}
\newcommand{\Can}{\Ca{n}}
\newcommand{\Qan}{\Qa{n}}
\newcommand{\Aan}{\Aa{n}}
\newcommand{\dan}{\da{n}}
\newcommand{\leqan}{\leqa{n}}
\newcommand{\lessan}{\lessa{n}}
\newcommand{\equivan}{\equiva{n}}

\newcommand{\SigmaOneOmega}{\Sigma^1_{\omega}}
\newcommand{\SigmaOneOmegaPlusOne}{\Sigma^1_{\omega+1}}
\newcommand{\PiOneOmega}{\Pi^1_{\omega}}
\newcommand{\PiOneOmegaPlusOne}{\Pi^1_{\omega+1}}
\newcommand{\DeltaOneOmegaPlusOne}{\Delta^1_{\omega+1}}

%%%%%%%%%%%%%%%%%%%%%%%%%%%%%%%%%%%%%%%%%%%%%
% Linear Algebra
%%%%%%%%%%%%%%%%%%%%%%%%%%%%%%%%%%%%%%%%%%%%%
\newcommand{\transpose}[1]{{#1}^{\text{T}}}
\newcommand{\norm}[1]{\lVert{#1}\rVert}
\newcommand\aug{\fboxsep=-\fboxrule\!\!\!\fbox{\strut}\!\!\!}

%%%%%%%%%%%%%%%%%%%%%%%%%%%%%%%%%%%%%%%%%%%%%
% Number Theory
%%%%%%%%%%%%%%%%%%%%%%%%%%%%%%%%%%%%%%%%%%%%%
\newcommand{\av}[1]{\lvert#1\rvert}
\DeclareMathOperator{\divides}{\mid}
\DeclareMathOperator{\ndivides}{\nmid}
\DeclareMathOperator{\lcm}{lcm}
\DeclareMathOperator{\sign}{sign}
\newcommand{\floor}[1]{\left\lfloor{#1}\right\rfloor}
\DeclareMathOperator{\ConCl}{CC}
\DeclareMathOperator{\ord}{ord}



\graphicspath{{images/}}

\newtheorem*{claim}{claim}
\newtheorem*{observation}{Observation}
\newtheorem*{warning}{Warning}
\newtheorem*{question}{Question}
\newtheorem{remark}[theorem]{Remark}

\newenvironment*{subproof}[1][Proof]
{\begin{proof}[#1]}{\renewcommand{\qedsymbol}{$\diamondsuit$} \end{proof}}

\mode<presentation>
{
  \usetheme{Singapore}
  % or ...

  \setbeamercovered{invisible}
  % or whatever (possibly just delete it)
}


\usepackage[english]{babel}
% or whatever

\usepackage[latin1]{inputenc}
% or whatever

\usepackage{times}
\usepackage[T1]{fontenc}
% Or whatever. Note that the encoding and the font should match. If T1
% does not look nice, try deleting the line with the fontenc.

\title{Lesson 3 \\ Vectors in $\R^n$}
\subtitle{Math 325, Linear Algebra \\ Fall 2018 \\ SFSU}
\author{Mitch Rudominer}
\date{}



% If you wish to uncover everything in a step-wise fashion, uncomment
% the following command:

\beamerdefaultoverlayspecification{<+->}

\begin{document}

\begin{frame}
  \titlepage
\end{frame}

\begin{frame}{What is $\R^n$?}

\begin{itemize}
\item In this class, we define $\R^n$ to be the set of $n\times 1$ column vectors of real numbers.
\item $\R^1=\R$ = the set of real numbers.
\item $\R^2 = \setof{(x,y)^{\text{T}}}{x,y\in\R}$ = the set of column vectors of size 2.
\item $\R^3 = \setof{(x,y,z)^{\text{T}}}{x,y,z\in\R}$ = the set of
column vectors of size 3.
\item In other classes, when matrices are not relevant, there is no distinction between row
and column vectors.
\item Then people say that $\R^n$ is the set of $n$-tuples of real numbers
\item without specifying whether it is a row or a column.
\item But in this class we specifically want vectors in $\R^n$ to be
represented as column vectors.
\item So then what will row vectors represent?
\item Something called \emph{co-vectors.}
\end{itemize}

\end{frame}

\begin{frame}{Vectors in $\R^n$?}

\begin{itemize}
\item The following all mean exactly the same thing
\item (a) $\bv$ is a vector in $\R^n$
\item (b) $\bv \in \R^n$
\item (c) $\bv$ is an $n\times 1$ column vector.
\item Also sometimes, if the context is clear, we will just say
\item (d) $\bv$ is a vector.
\end{itemize}

\end{frame}

\beamerdefaultoverlayspecification{}

\begin{frame}{Visualizing $\R^n$}

\begin{columns}
\column[T]{5cm}
\begin{itemize}
\item<1-> We visualize $\R^1$ as a number line.
\item<2-> We visualize $\R^2$ as the $x$- $y$- plane
\item<3-> We visuzlize $\R^3$ as $x$- $y$- $z$- space
\item<4-> Most people can't vizualize more than 3 dimensions so for $n>=4$
we don't directly try to vizualize $\R^n$.
\item<5-> Sometimes we use $\R^2$ or $\R^3$ as
\emph{approximations} or \emph{analogs} of $\R^n$.
\end{itemize}

\column[T]{5cm}
\includegraphics<1->[scale=0.1]{number-line}

\bigskip

\includegraphics<2->[scale=0.1]{plane}

\bigskip

\includegraphics<3->[scale=0.1]{space}

\end{columns}

\end{frame}

\begin{frame}{Vizualizing elements of $\R^n$}

\begin{columns}
\column[T]{5cm}
\begin{itemize}
\item<1-> Let $\bv = (6, 7)^{\text{T}} \in \R^2$
\item<2-> We may chose to visualize $\bv$ as a point
\item<3-> Or, we may chose to visualize it as an arrow with its base at the origin.
\item<4-> These two pictures represent the same thing,
\item<5-> the column vector $(6, 7)^{\text{T}}$.
\end{itemize}
\column[T]{5cm}
\includegraphics<2>[scale=0.15]{point}
\includegraphics<3->[scale=0.15]{vector}

\end{columns}

\end{frame}

\beamerdefaultoverlayspecification{<+->}

\begin{frame}{Notation for Vectors in $\R^n$}

\begin{itemize}
\item In this class we will use boldface letters like $\bv$ and $\bw$ to name
vectors.
\item And corresponding lightface subscripted letters $v_i$, $w_j$ for the
\emph{components} of the vectors.
\item For example if $\bv\in\R^4$ then $\bv=(v_1, v_2, v_3, v_4)^{\text{T}}$.
\item On the other hand if we had four vectors in $\R^n$ we might name them
$\bv_1$, $\bv_2$, $\bv_3$, $\bv_4$.
\item In \emph{hand-writing} it is difficult to indicate boldface so instead
let's use arrows over the tops of the letters. So in hand-writing
$\vec{v}$ and $\vec{w}$ are vectors.
\item And the components of $\vec{v}$ are $v_1$, $v_2$, etc.
\item And $\vec{v}_1$, $\vec{v_2}$ indicates two vectors.
\end{itemize}

\end{frame}

\begin{frame}{Adding Vectors in $\R^n$}

\begin{itemize}
\item If $\bv, \bw \in \R^n$ then $\bv+\bw$ is just a special case of matrix addition.
\item The addition is componentwise.
\item If $$\bv=(v_1, v_2, \cdots , v_n)^{\text{T}}$$ and $$\bw=(w_1, w_2, \cdots, w_n)^{\text{T}}$$
then $$\bv+\bw = (v_1 + w_1, v_2 + w_2, \cdots, v_n + w_n)^{\text{T}}$$
\item Example: Suppose $$v = (6, 7)^{\text{T}}$$ and $$w=(2, -3)^{\text{T}}$$ then
$$v+w=(8, 4)^{\text{T}}$$
\end{itemize}

\end{frame}

\beamerdefaultoverlayspecification{}

\begin{frame}{Geometric Representation of Adding Vectors in $\R^n$}

\begin{columns}
\column[T]{5cm}
\begin{itemize}
\item<1-> Let $\bv = (6, 7)^{\text{T}}$
\item<2-> Let $\bw=(2, -3)^{\text{T}}$
\item<3-> Temporarily move $w$ so that its base point is at the tip of $\bv$.
\item<4-> The tip of the moved $\bw$ is at the location of $\bv+\bw$
\item<5-> $\bv+\bw = (8, 4)^{\text{T}}$
\end{itemize}

\column[T]{5cm}
\includegraphics<1>[scale=0.15]{vector}
\includegraphics<2>[scale=0.15]{two-vectors}
\includegraphics<3-4>[scale=0.15]{vector-moved}
\includegraphics<5->[scale=0.15]{vector-sum}

\end{columns}

\end{frame}


\beamerdefaultoverlayspecification{<+->}

\begin{frame}{Scalar Multiplication of Vectors in $\R^n$}

\begin{itemize}
\item If $\bv \in \R^n$ and $c\in\R$, $c \bv$ is just a special case
of scalar multiplication of matrices.
\item If $$\bv=(v_1, v_2, \cdots , v_n)^{\text{T}}$$
then $$c \bv = (c v_1, c v_2, \cdots, c v_n)^{\text{T}}$$
\item Example: Suppose $$\bv = (6, 7)^{\text{T}}$$ and $c=-3$. Then
$$c \bv=(-18, -21)^{\text{T}}$$
\end{itemize}

\end{frame}

\beamerdefaultoverlayspecification{}

\begin{frame}{Geometric Representation of Scalar Multiplication in $\R^n$}

\begin{columns}
\column[T]{5cm}
\begin{itemize}
\item<1-> Let $\bv = (6, 7)^{\text{T}}$
\item<2-> Let $c=0.5$
\item<3-> $c \bv=(3, 3.5)^{\text{T}}$
\item<4-> Let $\beta=-1.5$
\item<5-> $\beta \bv=(-9, -10.5)^{\text{T}}$
\end{itemize}

\column[T]{5cm}
\includegraphics<1-2>[scale=0.15]{vector}
\includegraphics<3-4>[scale=0.15]{vector-times-half}
\includegraphics<5->[scale=0.15]{two-scalar-multiples}

\end{columns}

\end{frame}

\beamerdefaultoverlayspecification{<+->}

\begin{frame}{Linear Combinations of Vectors in $\R^n$}

\begin{itemize}
\item Let $\bv_1, \bv_2, \cdots  \bv_k$ be $k$ vectors.
\item Let $c_1, c_2, \cdots c_k$ be $k$ scalars.
\item Then $$c_1 \bv_1 + c_2 \bv_2 + \cdots + c_k \bv_k$$ is
called a \emph{linear combination} of $\bv_1, \bv_2, \cdots \bv_k$.
The $c_i$ are called the \emph{coefficients} of the linear combination.
\item Using Sigma notation we would write this as
$$\sum_{i=1}^{k} c_i \bv_i$$
\item Example. Let $\bv = (2, 1, 0)^{\text{T}}$ and $\bw = (0, 1, 3)^{\text{T}}$ be vectors in $\R^3$.
Express $\bu = (6, 8, 15)^{\text{T}}$ as a linear combination of $\bv$ and $\bw$.
\item Answer: $\bu = 3 \bv + 5 \bw$.
\end{itemize}

\end{frame}

\begin{frame}{Standard Basis Vectors in $\R^n$}

\begin{itemize}
\item In $\R_n$, for $i=1\cdots n$,
 let $\be_i$ be the vector that has a 1 for component $i$ and
a 0 for all other components. This is called the $i$-th
\emph{standard basis vector} of $\R^n$.
\item If we need to keep track of which $\R^n$ these vectors are from we can
call them $\be^n_i$.
\item For example in $\R^4$ the 4 standard basis vectors are
 $\be_1 = (1,0,0,0)^{\text{T}}$, $\be_2 = (0,1,0,0)^{\text{T}}$,
$\be_3 = (0,0,1,0)^{\text{T}}$, $\be_4 = (0,0,0,1)^{\text{T}}$.
\item If we need to keep track of the fact that these vectors are from $\R^4$
we can call them $\be_1^4, \be_2^4, \be_3^4, \be_4^4$.
\end{itemize}

\end{frame}

\begin{frame}{Linear Combinations of the Standard Basis Vectors in $\R^n$}

\begin{lemma}
In $\R^n$, every vector can be expressed as a linear combination of the standard
basis vectors.
\end{lemma}

\begin{proof}
\begin{itemize}
\item Let $\bv = (v_1, v_2, \cdots v_n)^{\text{T}}$.
\item Then we just use the $i$-th component of $\bv$ as the $i$-th coefficient
of the linear combination.
\item That is,
$$\bv = \sum _{i=1}^{n} v_i \be_i$$
\end{itemize}
\end{proof}

\end{frame}

\begin{frame}{Linear Combinations of the Standard Basis Vectors in $\R^n$}

\begin{itemize}
\item For example, let $\bv = (2, 0, -1.5, 7)^{\text{T}} \in \R^4$.
\item Express $\bv$ as a linear combination of the standard basis vectors of
$\R^4$.
\item Answer $$\bv = 2 \be_1 + 0 \be_2 + (-1.5) \be_3 + 7 \be_4$$
\item We could omit the term with the 0 coefficient and replace the addition
of a negative with a subtraction and instead write
$$\bv = 2 \be_1 -1.5 \be_3 + 7 \be_4$$
\end{itemize}

\end{frame}

\begin{frame}{The zero vector}

\begin{itemize}
\item The vector of length $n$ with all zero components will be written $\mathbf{0}_n$.
\item $$\mathbf{0}_3 = (0,0,0)^{\text{T}}$$
\item If $n$ is clear from context or irrelevent we will just write $\mathbf{0}$.
\item $\mathbf{0}_n$ corresponds to the origin in $\R^n$.
\item For any vector $\bv\in\R^n$, $\bv+\mathbf{0} = \bv$.
\item $\mathbf{0}_n$ is the same thing as $O_{n\times 1}$.
\end{itemize}

\end{frame}

\begin{frame}{Dot Product of Vectors in $\R^n$}

\begin{itemize}
\item If $\bv, \bw \in \R^n$ we define $\bv \cdot \bw$ to be the sum of the products of the components of $\bv$ with the corresponding components of $\bw$.
\item If $\bv=(v_1, v_2, \cdots , v_n)^{\text{T}}$ and $\bw=(w_1, w_2, \cdots, w_n)^{\text{T}}$
\item then $\bv \cdot \bw = v_1  w_1 +  v_2  w_2 + \cdots +  v_n  w_n$.
\item Notice $\bv \cdot \bw = \bv^{\text{T}} \bw$.
\item So we have two equivalent ways of writing the dot product.
\item Example: Suppose $\bv = (6, 7)^{\text{T}}$ and $\bw=(2, -3)^{\text{T}}$.
\item Then $\bv \cdot \bw$
\item $=\bv^{\text{T}} \bw$
\item
$$
= (6, 7) \begin{pmatrix} 2 \\ -3 \end{pmatrix}
$$
\item $12 + (-21) = -9$
\end{itemize}

\end{frame}

\begin{frame}{Dot product of orthogonal vectors}

\begin{itemize}
\item We will study the dot product more in later lessons.
\item For now, recall from earlier classes $\bv \cdot \bw = 0$ iff
$\bv$ and $\bw$ are orthogonal (i.e. perpendicular).
\item Example. Let $\bv=(2, 0, 0)^{\text{T}}$ and $\bw = (0, 3, 0)^{\text{T}}$.
Then $\bv \cdot \bw = 0$.
\item For any vector $\bv$, $\bv \cdot \mathbf{0} = 0$.
\end{itemize}

\end{frame}

\end{document}


