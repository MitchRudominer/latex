% $Header$

\documentclass{beamer}

\usepackage{amsmath,amssymb,latexsym,eucal,amsthm,graphicx}
%%%%%%%%%%%%%%%%%%%%%%%%%%%%%%%%%%%%%%%%%%%%%
% Common Set Theory Constructs
%%%%%%%%%%%%%%%%%%%%%%%%%%%%%%%%%%%%%%%%%%%%%

\newcommand{\setof}[2]{\left\{ \, #1 \, \left| \, #2 \, \right.\right\}}
\newcommand{\lsetof}[2]{\left\{\left. \, #1 \, \right| \, #2 \,  \right\}}
\newcommand{\bigsetof}[2]{\bigl\{ \, #1 \, \bigm | \, #2 \,\bigr\}}
\newcommand{\Bigsetof}[2]{\Bigl\{ \, #1 \, \Bigm | \, #2 \,\Bigr\}}
\newcommand{\biggsetof}[2]{\biggl\{ \, #1 \, \biggm | \, #2 \,\biggr\}}
\newcommand{\Biggsetof}[2]{\Biggl\{ \, #1 \, \Biggm | \, #2 \,\Biggr\}}
\newcommand{\dotsetof}[2]{\left\{ \, #1 \, : \, #2 \, \right\}}
\newcommand{\bigdotsetof}[2]{\bigl\{ \, #1 \, : \, #2 \,\bigr\}}
\newcommand{\Bigdotsetof}[2]{\Bigl\{ \, #1 \, \Bigm : \, #2 \,\Bigr\}}
\newcommand{\biggdotsetof}[2]{\biggl\{ \, #1 \, \biggm : \, #2 \,\biggr\}}
\newcommand{\Biggdotsetof}[2]{\Biggl\{ \, #1 \, \Biggm : \, #2 \,\Biggr\}}
\newcommand{\sequence}[2]{\left\langle \, #1 \,\left| \, #2 \, \right. \right\rangle}
\newcommand{\lsequence}[2]{\left\langle\left. \, #1 \, \right| \,#2 \,  \right\rangle}
\newcommand{\bigsequence}[2]{\bigl\langle \,#1 \, \bigm | \, #2 \, \bigr\rangle}
\newcommand{\Bigsequence}[2]{\Bigl\langle \,#1 \, \Bigm | \, #2 \, \Bigr\rangle}
\newcommand{\biggsequence}[2]{\biggl\langle \,#1 \, \biggm | \, #2 \, \biggr\rangle}
\newcommand{\Biggsequence}[2]{\Biggl\langle \,#1 \, \Biggm | \, #2 \, \Biggr\rangle}
\newcommand{\singleton}[1]{\left\{#1\right\}}
\newcommand{\angles}[1]{\left\langle #1 \right\rangle}
\newcommand{\bigangles}[1]{\bigl\langle #1 \bigr\rangle}
\newcommand{\Bigangles}[1]{\Bigl\langle #1 \Bigr\rangle}
\newcommand{\biggangles}[1]{\biggl\langle #1 \biggr\rangle}
\newcommand{\Biggangles}[1]{\Biggl\langle #1 \Biggr\rangle}


\newcommand{\force}[1]{\Vert\!\underset{\!\!\!\!\!#1}{\!\!\!\relbar\!\!\!%
\relbar\!\!\relbar\!\!\relbar\!\!\!\relbar\!\!\relbar\!\!\relbar\!\!\!%
\relbar\!\!\relbar\!\!\relbar}}
\newcommand{\longforce}[1]{\Vert\!\underset{\!\!\!\!\!#1}{\!\!\!\relbar\!\!\!%
\relbar\!\!\relbar\!\!\relbar\!\!\!\relbar\!\!\relbar\!\!\relbar\!\!\!%
\relbar\!\!\relbar\!\!\relbar\!\!\relbar\!\!\relbar\!\!\relbar\!\!\relbar\!\!\relbar}}
\newcommand{\nforce}[1]{\Vert\!\underset{\!\!\!\!\!#1}{\!\!\!\relbar\!\!\!%
\relbar\!\!\relbar\!\!\relbar\!\!\!\relbar\!\!\relbar\!\!\relbar\!\!\!%
\relbar\!\!\not\relbar\!\!\relbar}}
\newcommand{\forcein}[2]{\overset{#2}{\Vert\underset{\!\!\!\!\!#1}%
{\!\!\!\relbar\!\!\!\relbar\!\!\relbar\!\!\relbar\!\!\!\relbar\!\!\relbar\!%
\!\relbar\!\!\!\relbar\!\!\relbar\!\!\relbar\!\!\relbar\!\!\!\relbar\!\!%
\relbar\!\!\relbar}}}

\newcommand{\pre}[2]{{}^{#2}{#1}}

\newcommand{\restr}{\!\!\upharpoonright\!}

%%%%%%%%%%%%%%%%%%%%%%%%%%%%%%%%%%%%%%%%%%%%%
% Set-Theoretic Connectives
%%%%%%%%%%%%%%%%%%%%%%%%%%%%%%%%%%%%%%%%%%%%%

\newcommand{\intersect}{\cap}
\newcommand{\union}{\cup}
\newcommand{\Intersection}[1]{\bigcap\limits_{#1}}
\newcommand{\Union}[1]{\bigcup\limits_{#1}}
\newcommand{\adjoin}{{}^\frown}
\newcommand{\forces}{\Vdash}

%%%%%%%%%%%%%%%%%%%%%%%%%%%%%%%%%%%%%%%%%%%%%
% Miscellaneous
%%%%%%%%%%%%%%%%%%%%%%%%%%%%%%%%%%%%%%%%%%%%%
\newcommand{\defeq}{=_{\text{def}}}
\newcommand{\Turingleq}{\leq_{\text{T}}}
\newcommand{\Turingless}{<_{\text{T}}}
\newcommand{\lexleq}{\leq_{\text{lex}}}
\newcommand{\lexless}{<_{\text{lex}}}
\newcommand{\Turingequiv}{\equiv_{\text{T}}}
\newcommand{\isomorphic}{\cong}

%%%%%%%%%%%%%%%%%%%%%%%%%%%%%%%%%%%%%%%%%%%%%
% Constants
%%%%%%%%%%%%%%%%%%%%%%%%%%%%%%%%%%%%%%%%%%%%%
\newcommand{\R}{\mathbb{R}}
\renewcommand{\P}{\mathbb{P}}
\newcommand{\Q}{\mathbb{Q}}
\newcommand{\Z}{\mathbb{Z}}
\newcommand{\Zpos}{\Z^{+}}
\newcommand{\Znonneg}{\Z^{\geq 0}}
\newcommand{\C}{\mathbb{C}}
\newcommand{\N}{\mathbb{N}}
\newcommand{\B}{\mathbb{B}}
\newcommand{\Bairespace}{\pre{\omega}{\omega}}
\newcommand{\LofR}{L(\R)}
\newcommand{\JofR}[1]{J_{#1}(\R)}
\newcommand{\SofR}[1]{S_{#1}(\R)}
\newcommand{\JalphaR}{\JofR{\alpha}}
\newcommand{\JbetaR}{\JofR{\beta}}
\newcommand{\JlambdaR}{\JofR{\lambda}}
\newcommand{\SalphaR}{\SofR{\alpha}}
\newcommand{\SbetaR}{\SofR{\beta}}
\newcommand{\Pkl}{\mathcal{P}_{\kappa}(\lambda)}
\DeclareMathOperator{\con}{con}
\DeclareMathOperator{\ORD}{OR}
\DeclareMathOperator{\Ord}{OR}
\DeclareMathOperator{\WO}{WO}
\DeclareMathOperator{\OD}{OD}
\DeclareMathOperator{\HOD}{HOD}
\DeclareMathOperator{\HC}{HC}
\DeclareMathOperator{\WF}{WF}
\DeclareMathOperator{\wfp}{wfp}
\DeclareMathOperator{\HF}{HF}
\newcommand{\One}{I}
\newcommand{\ONE}{I}
\newcommand{\Two}{II}
\newcommand{\TWO}{II}
\newcommand{\Mladder}{M^{\text{ld}}}

%%%%%%%%%%%%%%%%%%%%%%%%%%%%%%%%%%%%%%%%%%%%%
% Commutative Algebra Constants
%%%%%%%%%%%%%%%%%%%%%%%%%%%%%%%%%%%%%%%%%%%%%
\DeclareMathOperator{\dottimes}{\dot{\times}}
\DeclareMathOperator{\dotminus}{\dot{-}}
\DeclareMathOperator{\Spec}{Spec}

%%%%%%%%%%%%%%%%%%%%%%%%%%%%%%%%%%%%%%%%%%%%%
% Theories
%%%%%%%%%%%%%%%%%%%%%%%%%%%%%%%%%%%%%%%%%%%%%
\DeclareMathOperator{\ZFC}{ZFC}
\DeclareMathOperator{\ZF}{ZF}
\DeclareMathOperator{\AD}{AD}
\DeclareMathOperator{\ADR}{AD_{\R}}
\DeclareMathOperator{\KP}{KP}
\DeclareMathOperator{\PD}{PD}
\DeclareMathOperator{\CH}{CH}
\DeclareMathOperator{\GCH}{GCH}
\DeclareMathOperator{\HPC}{HPC} % HOD pair capturing
%%%%%%%%%%%%%%%%%%%%%%%%%%%%%%%%%%%%%%%%%%%%%
% Iteration Trees
%%%%%%%%%%%%%%%%%%%%%%%%%%%%%%%%%%%%%%%%%%%%%

\newcommand{\pred}{\text{-pred}}

%%%%%%%%%%%%%%%%%%%%%%%%%%%%%%%%%%%%%%%%%%%%%%%%
% Operator Names
%%%%%%%%%%%%%%%%%%%%%%%%%%%%%%%%%%%%%%%%%%%%%%%%
\DeclareMathOperator{\Det}{Det}
\DeclareMathOperator{\dom}{dom}
\DeclareMathOperator{\ran}{ran}
\DeclareMathOperator{\range}{ran}
\DeclareMathOperator{\image}{image}
\DeclareMathOperator{\crit}{crit}
\DeclareMathOperator{\card}{card}
\DeclareMathOperator{\cf}{cf}
\DeclareMathOperator{\cof}{cof}
\DeclareMathOperator{\rank}{rank}
\DeclareMathOperator{\ot}{o.t.}
\DeclareMathOperator{\ords}{o}
\DeclareMathOperator{\ro}{r.o.}
\DeclareMathOperator{\rud}{rud}
\DeclareMathOperator{\Powerset}{\mathcal{P}}
\DeclareMathOperator{\length}{lh}
\DeclareMathOperator{\lh}{lh}
\DeclareMathOperator{\limit}{lim}
\DeclareMathOperator{\fld}{fld}
\DeclareMathOperator{\projection}{p}
\DeclareMathOperator{\Ult}{Ult}
\DeclareMathOperator{\ULT}{Ult}
\DeclareMathOperator{\Coll}{Coll}
\DeclareMathOperator{\Col}{Col}
\DeclareMathOperator{\Hull}{Hull}
\DeclareMathOperator*{\dirlim}{dir lim}
\DeclareMathOperator{\Scale}{Scale}
\DeclareMathOperator{\supp}{supp}
\DeclareMathOperator{\trancl}{tran.cl.}
\DeclareMathOperator{\trace}{Tr}
\DeclareMathOperator{\diag}{diag}
\DeclareMathOperator{\spn}{span}
\DeclareMathOperator{\sgn}{sgn}
%%%%%%%%%%%%%%%%%%%%%%%%%%%%%%%%%%%%%%%%%%%%%
% Logical Connectives
%%%%%%%%%%%%%%%%%%%%%%%%%%%%%%%%%%%%%%%%%%%%%
\newcommand{\IImplies}{\Longrightarrow}
\newcommand{\SkipImplies}{\quad\Longrightarrow\quad}
\newcommand{\Ifff}{\Longleftrightarrow}
\newcommand{\iimplies}{\longrightarrow}
\newcommand{\ifff}{\longleftrightarrow}
\newcommand{\Implies}{\Rightarrow}
\newcommand{\Iff}{\Leftrightarrow}
\renewcommand{\implies}{\rightarrow}
\renewcommand{\iff}{\leftrightarrow}
\newcommand{\AND}{\wedge}
\newcommand{\OR}{\vee}
\newcommand{\st}{\text{ s.t. }}
\newcommand{\Or}{\text{ or }}

%%%%%%%%%%%%%%%%%%%%%%%%%%%%%%%%%%%%%%%%%%%%%
% Function Arrows
%%%%%%%%%%%%%%%%%%%%%%%%%%%%%%%%%%%%%%%%%%%%%

\newcommand{\injection}{\xrightarrow{\text{1-1}}}
\newcommand{\surjection}{\xrightarrow{\text{onto}}}
\newcommand{\bijection}{\xrightarrow[\text{onto}]{\text{1-1}}}
\newcommand{\cofmap}{\xrightarrow{\text{cof}}}
\newcommand{\map}{\rightarrow}

%%%%%%%%%%%%%%%%%%%%%%%%%%%%%%%%%%%%%%%%%%%%%
% Mouse Comparison Operators
%%%%%%%%%%%%%%%%%%%%%%%%%%%%%%%%%%%%%%%%%%%%%
\newcommand{\initseg}{\trianglelefteq}
\newcommand{\properseg}{\lhd}
\newcommand{\notinitseg}{\ntrianglelefteq}
\newcommand{\notproperseg}{\ntriangleleft}

%%%%%%%%%%%%%%%%%%%%%%%%%%%%%%%%%%%%%%%%%%%%%
%           calligraphic letters
%%%%%%%%%%%%%%%%%%%%%%%%%%%%%%%%%%%%%%%%%%%%%
\newcommand{\cA}{\mathcal{A}}
\newcommand{\cB}{\mathcal{B}}
\newcommand{\cC}{\mathcal{C}}
\newcommand{\cD}{\mathcal{D}}
\newcommand{\cE}{\mathcal{E}}
\newcommand{\cF}{\mathcal{F}}
\newcommand{\cG}{\mathcal{G}}
\newcommand{\cH}{\mathcal{H}}
\newcommand{\cI}{\mathcal{I}}
\newcommand{\cJ}{\mathcal{J}}
\newcommand{\cK}{\mathcal{K}}
\newcommand{\cL}{\mathcal{L}}
\newcommand{\cM}{\mathcal{M}}
\newcommand{\cN}{\mathcal{N}}
\newcommand{\cO}{\mathcal{O}}
\newcommand{\cP}{\mathcal{P}}
\newcommand{\cQ}{\mathcal{Q}}
\newcommand{\cR}{\mathcal{R}}
\newcommand{\cS}{\mathcal{S}}
\newcommand{\cT}{\mathcal{T}}
\newcommand{\cU}{\mathcal{U}}
\newcommand{\cV}{\mathcal{V}}
\newcommand{\cW}{\mathcal{W}}
\newcommand{\cX}{\mathcal{X}}
\newcommand{\cY}{\mathcal{Y}}
\newcommand{\cZ}{\mathcal{Z}}


%%%%%%%%%%%%%%%%%%%%%%%%%%%%%%%%%%%%%%%%%%%%%
%          Primed Letters
%%%%%%%%%%%%%%%%%%%%%%%%%%%%%%%%%%%%%%%%%%%%%
\newcommand{\aprime}{a^{\prime}}
\newcommand{\bprime}{b^{\prime}}
\newcommand{\cprime}{c^{\prime}}
\newcommand{\dprime}{d^{\prime}}
\newcommand{\eprime}{e^{\prime}}
\newcommand{\fprime}{f^{\prime}}
\newcommand{\gprime}{g^{\prime}}
\newcommand{\hprime}{h^{\prime}}
\newcommand{\iprime}{i^{\prime}}
\newcommand{\jprime}{j^{\prime}}
\newcommand{\kprime}{k^{\prime}}
\newcommand{\lprime}{l^{\prime}}
\newcommand{\mprime}{m^{\prime}}
\newcommand{\nprime}{n^{\prime}}
\newcommand{\oprime}{o^{\prime}}
\newcommand{\pprime}{p^{\prime}}
\newcommand{\qprime}{q^{\prime}}
\newcommand{\rprime}{r^{\prime}}
\newcommand{\sprime}{s^{\prime}}
\newcommand{\tprime}{t^{\prime}}
\newcommand{\uprime}{u^{\prime}}
\newcommand{\vprime}{v^{\prime}}
\newcommand{\wprime}{w^{\prime}}
\newcommand{\xprime}{x^{\prime}}
\newcommand{\yprime}{y^{\prime}}
\newcommand{\zprime}{z^{\prime}}
\newcommand{\Aprime}{A^{\prime}}
\newcommand{\Bprime}{B^{\prime}}
\newcommand{\Cprime}{C^{\prime}}
\newcommand{\Dprime}{D^{\prime}}
\newcommand{\Eprime}{E^{\prime}}
\newcommand{\Fprime}{F^{\prime}}
\newcommand{\Gprime}{G^{\prime}}
\newcommand{\Hprime}{H^{\prime}}
\newcommand{\Iprime}{I^{\prime}}
\newcommand{\Jprime}{J^{\prime}}
\newcommand{\Kprime}{K^{\prime}}
\newcommand{\Lprime}{L^{\prime}}
\newcommand{\Mprime}{M^{\prime}}
\newcommand{\Nprime}{N^{\prime}}
\newcommand{\Oprime}{O^{\prime}}
\newcommand{\Pprime}{P^{\prime}}
\newcommand{\Qprime}{Q^{\prime}}
\newcommand{\Rprime}{R^{\prime}}
\newcommand{\Sprime}{S^{\prime}}
\newcommand{\Tprime}{T^{\prime}}
\newcommand{\Uprime}{U^{\prime}}
\newcommand{\Vprime}{V^{\prime}}
\newcommand{\Wprime}{W^{\prime}}
\newcommand{\Xprime}{X^{\prime}}
\newcommand{\Yprime}{Y^{\prime}}
\newcommand{\Zprime}{Z^{\prime}}
\newcommand{\alphaprime}{\alpha^{\prime}}
\newcommand{\betaprime}{\beta^{\prime}}
\newcommand{\gammaprime}{\gamma^{\prime}}
\newcommand{\Gammaprime}{\Gamma^{\prime}}
\newcommand{\deltaprime}{\delta^{\prime}}
\newcommand{\epsilonprime}{\epsilon^{\prime}}
\newcommand{\kappaprime}{\kappa^{\prime}}
\newcommand{\lambdaprime}{\lambda^{\prime}}
\newcommand{\rhoprime}{\rho^{\prime}}
\newcommand{\Sigmaprime}{\Sigma^{\prime}}
\newcommand{\tauprime}{\tau^{\prime}}
\newcommand{\xiprime}{\xi^{\prime}}
\newcommand{\thetaprime}{\theta^{\prime}}
\newcommand{\Omegaprime}{\Omega^{\prime}}
\newcommand{\cMprime}{\cM^{\prime}}
\newcommand{\cNprime}{\cN^{\prime}}
\newcommand{\cPprime}{\cP^{\prime}}
\newcommand{\cQprime}{\cQ^{\prime}}
\newcommand{\cRprime}{\cR^{\prime}}
\newcommand{\cSprime}{\cS^{\prime}}
\newcommand{\cTprime}{\cT^{\prime}}

%%%%%%%%%%%%%%%%%%%%%%%%%%%%%%%%%%%%%%%%%%%%%
%          bar Letters
%%%%%%%%%%%%%%%%%%%%%%%%%%%%%%%%%%%%%%%%%%%%%
\newcommand{\abar}{\bar{a}}
\newcommand{\bbar}{\bar{b}}
\newcommand{\cbar}{\bar{c}}
\newcommand{\ibar}{\bar{i}}
\newcommand{\jbar}{\bar{j}}
\newcommand{\nbar}{\bar{n}}
\newcommand{\xbar}{\bar{x}}
\newcommand{\ybar}{\bar{y}}
\newcommand{\zbar}{\bar{z}}
\newcommand{\pibar}{\bar{\pi}}
\newcommand{\phibar}{\bar{\varphi}}
\newcommand{\psibar}{\bar{\psi}}
\newcommand{\thetabar}{\bar{\theta}}
\newcommand{\nubar}{\bar{\nu}}

%%%%%%%%%%%%%%%%%%%%%%%%%%%%%%%%%%%%%%%%%%%%%
%          star Letters
%%%%%%%%%%%%%%%%%%%%%%%%%%%%%%%%%%%%%%%%%%%%%
\newcommand{\phistar}{\phi^{*}}
\newcommand{\Mstar}{M^{*}}

%%%%%%%%%%%%%%%%%%%%%%%%%%%%%%%%%%%%%%%%%%%%%
%          dotletters Letters
%%%%%%%%%%%%%%%%%%%%%%%%%%%%%%%%%%%%%%%%%%%%%
\newcommand{\Gdot}{\dot{G}}

%%%%%%%%%%%%%%%%%%%%%%%%%%%%%%%%%%%%%%%%%%%%%
%         check Letters
%%%%%%%%%%%%%%%%%%%%%%%%%%%%%%%%%%%%%%%%%%%%%
\newcommand{\deltacheck}{\check{\delta}}
\newcommand{\gammacheck}{\check{\gamma}}


%%%%%%%%%%%%%%%%%%%%%%%%%%%%%%%%%%%%%%%%%%%%%
%          Formulas
%%%%%%%%%%%%%%%%%%%%%%%%%%%%%%%%%%%%%%%%%%%%%

\newcommand{\formulaphi}{\text{\large $\varphi$}}
\newcommand{\Formulaphi}{\text{\Large $\varphi$}}


%%%%%%%%%%%%%%%%%%%%%%%%%%%%%%%%%%%%%%%%%%%%%
%          Fraktur Letters
%%%%%%%%%%%%%%%%%%%%%%%%%%%%%%%%%%%%%%%%%%%%%

\newcommand{\fa}{\mathfrak{a}}
\newcommand{\fb}{\mathfrak{b}}
\newcommand{\fc}{\mathfrak{c}}
\newcommand{\fk}{\mathfrak{k}}
\newcommand{\fp}{\mathfrak{p}}
\newcommand{\fq}{\mathfrak{q}}
\newcommand{\fr}{\mathfrak{r}}
\newcommand{\fA}{\mathfrak{A}}
\newcommand{\fB}{\mathfrak{B}}
\newcommand{\fC}{\mathfrak{C}}
\newcommand{\fD}{\mathfrak{D}}

%%%%%%%%%%%%%%%%%%%%%%%%%%%%%%%%%%%%%%%%%%%%%
%          Bold Letters
%%%%%%%%%%%%%%%%%%%%%%%%%%%%%%%%%%%%%%%%%%%%%
\newcommand{\ba}{\mathbf{a}}
\newcommand{\bb}{\mathbf{b}}
\newcommand{\bc}{\mathbf{c}}
\newcommand{\bd}{\mathbf{d}}
\newcommand{\be}{\mathbf{e}}
\newcommand{\bu}{\mathbf{u}}
\newcommand{\bv}{\mathbf{v}}
\newcommand{\bw}{\mathbf{w}}
\newcommand{\bx}{\mathbf{x}}
\newcommand{\by}{\mathbf{y}}
\newcommand{\bz}{\mathbf{z}}
\newcommand{\bSigma}{\boldsymbol{\Sigma}}
\newcommand{\bPi}{\boldsymbol{\Pi}}
\newcommand{\bDelta}{\boldsymbol{\Delta}}
\newcommand{\bdelta}{\boldsymbol{\delta}}
\newcommand{\bgamma}{\boldsymbol{\gamma}}
\newcommand{\bGamma}{\boldsymbol{\Gamma}}

%%%%%%%%%%%%%%%%%%%%%%%%%%%%%%%%%%%%%%%%%%%%%
%         Bold numbers
%%%%%%%%%%%%%%%%%%%%%%%%%%%%%%%%%%%%%%%%%%%%%
\newcommand{\bzero}{\mathbf{0}}

%%%%%%%%%%%%%%%%%%%%%%%%%%%%%%%%%%%%%%%%%%%%%
% Projective-Like Pointclasses
%%%%%%%%%%%%%%%%%%%%%%%%%%%%%%%%%%%%%%%%%%%%%
\newcommand{\Sa}[2][\alpha]{\Sigma_{(#1,#2)}}
\newcommand{\Pa}[2][\alpha]{\Pi_{(#1,#2)}}
\newcommand{\Da}[2][\alpha]{\Delta_{(#1,#2)}}
\newcommand{\Aa}[2][\alpha]{A_{(#1,#2)}}
\newcommand{\Ca}[2][\alpha]{C_{(#1,#2)}}
\newcommand{\Qa}[2][\alpha]{Q_{(#1,#2)}}
\newcommand{\da}[2][\alpha]{\delta_{(#1,#2)}}
\newcommand{\leqa}[2][\alpha]{\leq_{(#1,#2)}}
\newcommand{\lessa}[2][\alpha]{<_{(#1,#2)}}
\newcommand{\equiva}[2][\alpha]{\equiv_{(#1,#2)}}


\newcommand{\Sl}[1]{\Sa[\lambda]{#1}}
\newcommand{\Pl}[1]{\Pa[\lambda]{#1}}
\newcommand{\Dl}[1]{\Da[\lambda]{#1}}
\newcommand{\Al}[1]{\Aa[\lambda]{#1}}
\newcommand{\Cl}[1]{\Ca[\lambda]{#1}}
\newcommand{\Ql}[1]{\Qa[\lambda]{#1}}

\newcommand{\San}{\Sa{n}}
\newcommand{\Pan}{\Pa{n}}
\newcommand{\Dan}{\Da{n}}
\newcommand{\Can}{\Ca{n}}
\newcommand{\Qan}{\Qa{n}}
\newcommand{\Aan}{\Aa{n}}
\newcommand{\dan}{\da{n}}
\newcommand{\leqan}{\leqa{n}}
\newcommand{\lessan}{\lessa{n}}
\newcommand{\equivan}{\equiva{n}}

\newcommand{\SigmaOneOmega}{\Sigma^1_{\omega}}
\newcommand{\SigmaOneOmegaPlusOne}{\Sigma^1_{\omega+1}}
\newcommand{\PiOneOmega}{\Pi^1_{\omega}}
\newcommand{\PiOneOmegaPlusOne}{\Pi^1_{\omega+1}}
\newcommand{\DeltaOneOmegaPlusOne}{\Delta^1_{\omega+1}}

%%%%%%%%%%%%%%%%%%%%%%%%%%%%%%%%%%%%%%%%%%%%%
% Linear Algebra
%%%%%%%%%%%%%%%%%%%%%%%%%%%%%%%%%%%%%%%%%%%%%
\newcommand{\transpose}[1]{{#1}^{\text{T}}}
\newcommand{\norm}[1]{\lVert{#1}\rVert}
\newcommand\aug{\fboxsep=-\fboxrule\!\!\!\fbox{\strut}\!\!\!}

%%%%%%%%%%%%%%%%%%%%%%%%%%%%%%%%%%%%%%%%%%%%%
% Number Theory
%%%%%%%%%%%%%%%%%%%%%%%%%%%%%%%%%%%%%%%%%%%%%
\newcommand{\av}[1]{\lvert#1\rvert}
\DeclareMathOperator{\divides}{\mid}
\DeclareMathOperator{\ndivides}{\nmid}
\DeclareMathOperator{\lcm}{lcm}
\DeclareMathOperator{\sign}{sign}
\newcommand{\floor}[1]{\left\lfloor{#1}\right\rfloor}
\DeclareMathOperator{\ConCl}{CC}
\DeclareMathOperator{\ord}{ord}



\graphicspath{{images/}}

\newtheorem*{claim}{claim}
\newtheorem*{observation}{Observation}
\newtheorem*{warning}{Warning}
\newtheorem*{question}{Question}
\newtheorem{remark}[theorem]{Remark}

\newenvironment*{subproof}[1][Proof]
{\begin{proof}[#1]}{\renewcommand{\qedsymbol}{$\diamondsuit$} \end{proof}}

\mode<presentation>
{
  \usetheme{Singapore}
  % or ...

  \setbeamercovered{invisible}
  % or whatever (possibly just delete it)
}


\usepackage[english]{babel}
% or whatever

\usepackage[latin1]{inputenc}
% or whatever

\usepackage{times}
\usepackage[T1]{fontenc}
% Or whatever. Note that the encoding and the font should match. If T1
% does not look nice, try deleting the line with the fontenc.

\title{Lesson 6 \\ Matrices and Linear Transformations}
\subtitle{Math 325, Linear Algebra \\ Fall 2018 \\ SFSU}
\author{Mitch Rudominer}
\date{}


% If you wish to uncover everything in a step-wise fashion, uncomment
% the following command:

\beamerdefaultoverlayspecification{<+->}

\begin{document}

\begin{frame}
  \titlepage
\end{frame}



\begin{frame}{Matrices and Linear Functions}

\begin{itemize}
\item What does Lesson 2 Matrices have to do with Lesson 5 Linear functions on $\R^n$?
\item It turns out that an $n\times m$ matrix contains precisely the information we need
to describe a linear function from $\R^m$ to $\R^n$.
\item There is a one-to-one correspondence between $n\times m$ matrices and linear
functions from $\R^m$ to $\R^n$
\item Every matrix gives us a linear function.
\item Every linear function is given by a matrix.
\end{itemize}

\end{frame}

%%%%%%%%%%%%%%%%%%%%%%%%%%%%%%%%%%%%%%%%%%%%%%%%%%%%%%%%%%%%%%%%%%%%%

\begin{frame}{Linear transformations}

\begin{itemize}
\item Linear functions from $\R^m$ to $\R^n$ are also often called
linear \emph{transformations}.
\item A linear transformation is exactly the same thing as a linear function.
\item Especially in the context of matrices it is common to use the term
linear transformation instead.
\item From now on we will use the term linear transformation.
\item When using the term \emph{linear transformation}, it is common to use
letters like $T$, $S$ and $R$ to represent the linear transformation rather
than letters like $f$, $g$ and $h$.
\end{itemize}

\end{frame}

%%%%%%%%%%%%%%%%%%%%%%%%%%%%%%%%%%%%%%%%%%%%%%%%%%%%%%%%%%%%%%%%%%%%

\begin{frame}{The linear transformation associated with a matrix.}

\begin{itemize}
\item Definition:
\item Let $A$ be an $n\times m$ matrix.
\item Then associated with $A$ is a function $T_A:\R^m\map\R^n$.
\item $T_A$ is called the \emph{linear transformation associated with $A$}.
\item $T_A$ is defined as follows. Let $\bv\in\R^m$.
\item Then $T_A(\bv) = A \bv$.
\item Check that this multiplication makes sense and the output is in $\R^n$.
\item We will show that in fact it is linear.
\end{itemize}
\end{frame}

%%%%%%%%%%%%%%%%%%%%%%%%%%%%%%%%%%%%%%%%%%%%%%%%%%%%%%%%%%%%%%%%%%%%

\begin{frame}{Example}

\begin{itemize}
\item Example. Let
$$
\begin{matrix}
A =
&
\begin{pmatrix}
1 & -1  & 0 \\
2 &  0  & -1
\end{pmatrix}
&
\bv =
&
\begin{pmatrix}
1 \\ -2 \\ 3
\end{pmatrix}
\end{matrix}
$$
\item Then
$$
T_A(\bv) =
\begin{pmatrix}
3 \\-1
\end{pmatrix}
$$
\end{itemize}

\end{frame}

%%%%%%%%%%%%%%%%%%%%%%%%%%%%%%%%%%%%%%%%%%%%%%%%%%%%%%%%%%%%%%%%%%%%
\begin{frame}{$T_A$ is linear}

\begin{itemize}
\item \textbf{Lemma.} Let $A$ be an $n\times m$ matrix.
\item Then $T_A$ is linear.
\item Proof.
\item We must show that $T_A$ preserves vector addtion.
$$T_A(\bv + \bw) = T_A(\bv) + T_A(\bw)$$
\item But this is true because matrix multiplication distributes over matrix addtion.
$$A(\bv + \bw) = A\bv + A\bw$$
\item We must show that $T_A$ preserves scalar multiplication.
$$T_A(c \bv) = c T_A(\bv)$$
\item But this is true because matrix multiplication is compatible with scalar multiplication.
$$A(c \bv ) = c A\bv$$
\item So $T_A$ is linear. $\qed$
\end{itemize}
\end{frame}

%%%%%%%%%%%%%%%%%%%%%%%%%%%%%%%%%%%%%%%%%%%%%%%%%%%%%%%%%%%%%%%%%%%%
\beamerdefaultoverlayspecification{}

\begin{frame}{The images of the standard basis vectors.}

\begin{lemma}
Let $A$ be an $n\times m$ matrix.

\pause

\bigskip

Then the columns of $A$ are the
images of the standard basis vectors under $T_A$.

\bigskip


\pause

$T_A(\be_j)$ is the $j$-th column of $A$.
\end{lemma}
\end{frame}

%%%%%%%%%%%%%%%%%%%%%%%%%%%%%%%%%%%%%%%%%%%%%%%%%%%%%%%%%%%%%%%%%%%%

\begin{frame}{Example}

Let
$$ A =
\begin{pmatrix}
1 & 2 & 3 & 4 \\
5 & 6 & 7 & 8 \\
9 & 10 & 11 & 12
\end{pmatrix}
$$

\pause

Compute $T_A(\be_3)$.

\pause

$$T_A(\be_3) = A \be_3$$


\pause

$$
=
\begin{pmatrix}
1 & 2  & 3  & 4 \\
5 & 6  & 7  & 8 \\
9 & 10 & 11 & 12
\end{pmatrix}
\begin{pmatrix}
0 \\ 0 \\ 1 \\ 0
\end{pmatrix}
=
\begin{pmatrix}
3 \\ 7 \\ 11
\end{pmatrix}
$$
\end{frame}

%%%%%%%%%%%%%%%%%%%%%%%%%%%%%%%%%%%%%%%%%%%%%%%%%%%%%%%%%%%%%%%%%%%%

\beamerdefaultoverlayspecification{}

\begin{frame}{Values at the standard basis vectors}

A linear transformation is determined by its values at the standard basis vectors.

\uncover<2->{
\begin{lemma}
Let $T:\R^m\map\R^n$ and $S:\R^m\map\R^n$  be two linear transformations. \\
\uncover<3->{Suppose that $T(\be_j) = S(\be_j)$ for $j=1, 2, \cdots m$.} \\
\uncover<4->{Then $T = S$.}
\end{lemma}
}

\end{frame}

%%%%%%%%%%%%%%%%%%%%%%%%%%%%%%%%%%%%%%%%%%%%%%%%%%%%%%%%%%%%%%%%%%%%

\begin{frame}{Proof of Lemma}

\begin{proof}
Let $\bv\in\R^m$ be any vector.
\uncover<2->{We must show that $T(\bv)=S(\bv)$.}
\uncover<3->{
Now $\bv = (v_1,v_2,\cdots v_n)^{\text{T}}$.
So $\bv=\sum_j{v_j \be_j}$.
}

\begin{align*}
\uncover<4->{T(\bv) &= T\left(\sum_j{v_j \be_j}\right) \\}
\uncover<5->{&= \sum_j{v_j T(\be_j)} \text{ \qquad (by linearity)}\\}
\uncover<6->{&= \sum_j{v_j S(\be_j)} \text { \qquad (by assumption)}\\}
\uncover<7->{&= S\left(\sum_j{v_j \be_j}\right) { \qquad  (by linearity)} \\}
\uncover<8->{&= S(\bv)}
\end{align*}
\end{proof}
\end{frame}

%%%%%%%%%%%%%%%%%%%%%%%%%%%%%%%%%%%%%%%%%%%%%%%%%%%%%%%%%%%%%%%%%%%%
\begin{frame}{Every linear transformation is given by a matrix}

\begin{theorem}
Let $T:\R^m\map\R^n$ be a linear transformation.

\pause

\small

Then there is an $n\times m$ matrix $A$ such that $T = T_A$
\end{theorem}

\pause

\begin{proof}
For $j=1\dots m$, let $\ba_j = T(\be_j)$.

\pause

i.e. let $\ba_1 = T(\be_1), \ba_2 = T(\be_2), \cdots \ba_m = T(\be_m)$.

\pause

So each $\ba_j\in\R^n$.

\pause

Let $A = (\ba_1 \vert \ba_2 \vert \cdots \vert \ba_m) $ be the matrix composed of the columns $\ba_1, \ba_2 , \cdots \ba_m$.

\pause

So $A$ is an $n\times m$ matrix.

\pause

Then $T_A(\be_j) = \ba_j = T(\be_j)$.

\pause

So $T_A = T$.

\end{proof}

\end{frame}

%%%%%%%%%%%%%%%%%%%%%%%%%%%%%%%%%%%%%%%%%%%%%%%%%%%%%%%%%%%%%%%%%%%%

\beamerdefaultoverlayspecification{<+->}

\begin{frame}{Example}
\begin{itemize}
\item Find a linear transformation from $T:\R^3\map\R^4$ such that
\item
$$
\begin{matrix}
T\begin{pmatrix} 1 \\ 0 \\ 0  \end{pmatrix} = \begin{pmatrix} 3 \\ 0 \\ 1 \\ 2  \end{pmatrix}
\text{; and }
&
T\begin{pmatrix} 0 \\ 1 \\ 0  \end{pmatrix} = \begin{pmatrix} 2 \\ -1 \\ 0 \\ 4  \end{pmatrix}
\text{; and }
&
T \begin{pmatrix} 0 \\ 0 \\ 1  \end{pmatrix} = \begin{pmatrix} 1 \\ 1 \\ 5 \\ 5  \end{pmatrix}
\end{matrix}
$$
\item Let
$$
A =
\begin{pmatrix}
3 & 2 & 1 \\
0 & -1 & 1 \\
1 & 0 & 5 \\
2 & 4 & 5
\end{pmatrix}
$$
\item Then $T_A$ is the desired linear transformation.
\end{itemize}
\end{frame}

%%%%%%%%%%%%%%%%%%%%%%%%%%%%%%%%%%%%%%%%%%%%%%%%%%%%%%%%%%%%%%%%%%%%

\begin{frame}{Linear Functionals}
\begin{itemize}
\item When the target space of a linear transformation is $\R^1$ then there
are other names names for it.
\item Let $T:\R^m\map\R^1$ be a linear transformation. Then $T$ is also called a
\item \emph{linear functional}.
\item Also $T$ is called a \emph{covector}.
\end{itemize}
\end{frame}

%%%%%%%%%%%%%%%%%%%%%%%%%%%%%%%%%%%%%%%%%%%%%%%%%%%%%%%%%%%%%%%%%%%%

\begin{frame}{Covectors}
\begin{itemize}
\item Let $T:\R^m\map\R^1$ be a linear functional.
\item Because every linear transformation is given by matrix, $T$ is given
by a matrix. What size matrix?
\item $T$ is given by an $m\times 1$ column vector.
\item Example: Let $T:\R^3\map\R$ be the linear functional such that
$$
\begin{matrix}
T\begin{pmatrix} 1 \\ 0 \\ 0  \end{pmatrix} = 4
\text{; and }
&
T\begin{pmatrix} 0 \\ 1 \\ 0  \end{pmatrix} = -5
\text{; and }
&
T \begin{pmatrix} 0 \\ 0 \\ 1  \end{pmatrix} = 6
\end{matrix}
$$
\item Find the column vector that represents $T$.
\item Let $\ba = (4, -5, 6)$.
\item Then $T = T_{\ba}$.
\end{itemize}
\end{frame}

%%%%%%%%%%%%%%%%%%%%%%%%%%%%%%%%%%%%%%%%%%%%%%%%%%%%%%%%%%%%%%%%%%%%

\begin{frame}{Row vectors represent covectors}
\begin{itemize}
\item So column vectors represent vectors in $\R^m$.
\item And row vectors represent covectors from $\R^m$.
\item When you see a row vector like
$$(4, -5, 6)$$
\item think of it as an object that \emph{operates} on column vectors.
\item When given a column vector like
$$
\begin{pmatrix}
1 \\ -1 \\ 0
\end{pmatrix}
$$
\item Then it is the \emph{job} of $(4, -5, 6)$ to \emph{hit}
$$
\begin{pmatrix}
1 \\ -1 \\ 0
\end{pmatrix}
$$
\item and produce the output $9$.
\end{itemize}
\end{frame}

%%%%%%%%%%%%%%%%%%%%%%%%%%%%%%%%%%%%%%%%%%%%%%%%%%%%%%%%%%%%%%%%%%%%

\begin{frame}{The rows of a matrix}
\begin{itemize}
\item Let
$$A =
\begin{pmatrix}
1 &  0 & -2 & 0 \\
0 &  0 &  1 & 1 \\
2 & -1 &  0 & 5
\end{pmatrix}
$$
\item We learned the significance of the columns of $A$.
\item They are the images of the standard basis vectors under $T_A$.
\item What is the significance of the \emph{rows} of $A$?
\item They are the \emph{component covectors} of $A$.
\item $T_A:\R^4\map\R^3$.
\item So $T_A$ has three component functions, let's call them $f_1, f_2, f_3$.
\item Let $T_A(\bv) = \bw$.
\item
$$T_A(\bv) =
\begin{pmatrix}
w_1 \\ w_2 \\ w_3
\end{pmatrix}
=
\begin{pmatrix}
f_1(\bv) \\ f_2(\bv) \\ f_3(\bv)
\end{pmatrix}
$$
\end{itemize}
\end{frame}

%%%%%%%%%%%%%%%%%%%%%%%%%%%%%%%%%%%%%%%%%%%%%%%%%%%%%%%%%%%%%%%%%%%%

\begin{frame}{Component covectors}
\begin{itemize}
\item Let
$$
\begin{matrix}
A =
\begin{pmatrix}
1 &  0 & -2 & 0 \\
0 &  0 &  1 & 1 \\
2 & -1 &  0 & 5
\end{pmatrix}
&
\bv =
\begin{pmatrix}
1 \\ 1 \\ 0 \\ -1
\end{pmatrix}
\end{matrix}
$$
\item Then
$$
T_A(\bv) =
A \bv
=
\begin{pmatrix}
1 \\ -1 \\ -4
\end{pmatrix}
=
\begin{pmatrix}
f_1(\bv) \\ f_2(\bv) \\ f_3(\bv)
\end{pmatrix}
$$
\item So $f_1(\bv) = 1$, $f_2(\bv) = -1$, $f_3(\bv) = -4$.
\end{itemize}
\end{frame}

%%%%%%%%%%%%%%%%%%%%%%%%%%%%%%%%%%%%%%%%%%%%%%%%%%%%%%%%%%%%%%%%%%%%

\begin{frame}{Rows are component covectors}
\begin{itemize}
\item Let
$$A =
\begin{pmatrix}
1 &  0 & -2 & 0 \\
0 &  0 &  1 & 1 \\
2 & -1 &  0 & 5
\end{pmatrix}
$$
\item Let $\ba_1, \ba_2, \ba_3$ be the rows of $A$.
\item Then
$$T_A(\bv) =
A \bv
=
\begin{pmatrix}
\ba_1 \bv \\ \ba_2 \bv \\ \ba_3 \bv
\end{pmatrix}
=
\begin{pmatrix}
f_1(\bv) \\ f_2(\bv) \\ f_3(\bv)
\end{pmatrix}
$$
\item So $f_1 = T_{\ba_1}$,$f_2 = T_{\ba_2}$, $f_3 = T_{\ba_3}$.
\item The rows of $A$ are the component covectors of $T_A$.
\end{itemize}
\end{frame}

%%%%%%%%%%%%%%%%%%%%%%%%%%%%%%%%%%%%%%%%%%%%%%%%%%%%%%%%%%%%%%%%%%%%

\beamerdefaultoverlayspecification{<+->}

\begin{frame}{Example}
\begin{itemize}
\item Find a linear transformation $T:\R^2\map\R^2$ with component covectors
$\ba_1$ and $\ba_2$ such that
\item
$$
\begin{matrix}
T_{\ba_1}\begin{pmatrix} 1 \\ 0 \end{pmatrix} = 5
\text{; and }
&
T_{\ba_1}\begin{pmatrix} 0 \\ 1  \end{pmatrix} = 6
\text{; and }
\end{matrix}
$$
\item
$$
\begin{matrix}
T_{\ba_2}\begin{pmatrix} 1 \\ 0   \end{pmatrix} = 3
\text{; and }
&
T_{\ba_2}\begin{pmatrix} 0 \\ 1   \end{pmatrix} = 4
\end{matrix}
$$
\item Let
$$
A =
\begin{pmatrix}
5 & 6 \\
3 & 4
\end{pmatrix}
$$
\item Then $T_A$ is the desired linear transformation.
\end{itemize}
\end{frame}

\end{document}


