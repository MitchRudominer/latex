% $Header$

\documentclass{beamer}
%\documentclass[handout]{beamer}

\usepackage{amsmath,amssymb,latexsym,eucal,amsthm,graphicx}
%%%%%%%%%%%%%%%%%%%%%%%%%%%%%%%%%%%%%%%%%%%%%
% Common Set Theory Constructs
%%%%%%%%%%%%%%%%%%%%%%%%%%%%%%%%%%%%%%%%%%%%%

\newcommand{\setof}[2]{\left\{ \, #1 \, \left| \, #2 \, \right.\right\}}
\newcommand{\lsetof}[2]{\left\{\left. \, #1 \, \right| \, #2 \,  \right\}}
\newcommand{\bigsetof}[2]{\bigl\{ \, #1 \, \bigm | \, #2 \,\bigr\}}
\newcommand{\Bigsetof}[2]{\Bigl\{ \, #1 \, \Bigm | \, #2 \,\Bigr\}}
\newcommand{\biggsetof}[2]{\biggl\{ \, #1 \, \biggm | \, #2 \,\biggr\}}
\newcommand{\Biggsetof}[2]{\Biggl\{ \, #1 \, \Biggm | \, #2 \,\Biggr\}}
\newcommand{\dotsetof}[2]{\left\{ \, #1 \, : \, #2 \, \right\}}
\newcommand{\bigdotsetof}[2]{\bigl\{ \, #1 \, : \, #2 \,\bigr\}}
\newcommand{\Bigdotsetof}[2]{\Bigl\{ \, #1 \, \Bigm : \, #2 \,\Bigr\}}
\newcommand{\biggdotsetof}[2]{\biggl\{ \, #1 \, \biggm : \, #2 \,\biggr\}}
\newcommand{\Biggdotsetof}[2]{\Biggl\{ \, #1 \, \Biggm : \, #2 \,\Biggr\}}
\newcommand{\sequence}[2]{\left\langle \, #1 \,\left| \, #2 \, \right. \right\rangle}
\newcommand{\lsequence}[2]{\left\langle\left. \, #1 \, \right| \,#2 \,  \right\rangle}
\newcommand{\bigsequence}[2]{\bigl\langle \,#1 \, \bigm | \, #2 \, \bigr\rangle}
\newcommand{\Bigsequence}[2]{\Bigl\langle \,#1 \, \Bigm | \, #2 \, \Bigr\rangle}
\newcommand{\biggsequence}[2]{\biggl\langle \,#1 \, \biggm | \, #2 \, \biggr\rangle}
\newcommand{\Biggsequence}[2]{\Biggl\langle \,#1 \, \Biggm | \, #2 \, \Biggr\rangle}
\newcommand{\singleton}[1]{\left\{#1\right\}}
\newcommand{\angles}[1]{\left\langle #1 \right\rangle}
\newcommand{\bigangles}[1]{\bigl\langle #1 \bigr\rangle}
\newcommand{\Bigangles}[1]{\Bigl\langle #1 \Bigr\rangle}
\newcommand{\biggangles}[1]{\biggl\langle #1 \biggr\rangle}
\newcommand{\Biggangles}[1]{\Biggl\langle #1 \Biggr\rangle}


\newcommand{\force}[1]{\Vert\!\underset{\!\!\!\!\!#1}{\!\!\!\relbar\!\!\!%
\relbar\!\!\relbar\!\!\relbar\!\!\!\relbar\!\!\relbar\!\!\relbar\!\!\!%
\relbar\!\!\relbar\!\!\relbar}}
\newcommand{\longforce}[1]{\Vert\!\underset{\!\!\!\!\!#1}{\!\!\!\relbar\!\!\!%
\relbar\!\!\relbar\!\!\relbar\!\!\!\relbar\!\!\relbar\!\!\relbar\!\!\!%
\relbar\!\!\relbar\!\!\relbar\!\!\relbar\!\!\relbar\!\!\relbar\!\!\relbar\!\!\relbar}}
\newcommand{\nforce}[1]{\Vert\!\underset{\!\!\!\!\!#1}{\!\!\!\relbar\!\!\!%
\relbar\!\!\relbar\!\!\relbar\!\!\!\relbar\!\!\relbar\!\!\relbar\!\!\!%
\relbar\!\!\not\relbar\!\!\relbar}}
\newcommand{\forcein}[2]{\overset{#2}{\Vert\underset{\!\!\!\!\!#1}%
{\!\!\!\relbar\!\!\!\relbar\!\!\relbar\!\!\relbar\!\!\!\relbar\!\!\relbar\!%
\!\relbar\!\!\!\relbar\!\!\relbar\!\!\relbar\!\!\relbar\!\!\!\relbar\!\!%
\relbar\!\!\relbar}}}

\newcommand{\pre}[2]{{}^{#2}{#1}}

\newcommand{\restr}{\!\!\upharpoonright\!}

%%%%%%%%%%%%%%%%%%%%%%%%%%%%%%%%%%%%%%%%%%%%%
% Set-Theoretic Connectives
%%%%%%%%%%%%%%%%%%%%%%%%%%%%%%%%%%%%%%%%%%%%%

\newcommand{\intersect}{\cap}
\newcommand{\union}{\cup}
\newcommand{\Intersection}[1]{\bigcap\limits_{#1}}
\newcommand{\Union}[1]{\bigcup\limits_{#1}}
\newcommand{\adjoin}{{}^\frown}
\newcommand{\forces}{\Vdash}

%%%%%%%%%%%%%%%%%%%%%%%%%%%%%%%%%%%%%%%%%%%%%
% Miscellaneous
%%%%%%%%%%%%%%%%%%%%%%%%%%%%%%%%%%%%%%%%%%%%%
\newcommand{\defeq}{=_{\text{def}}}
\newcommand{\Turingleq}{\leq_{\text{T}}}
\newcommand{\Turingless}{<_{\text{T}}}
\newcommand{\lexleq}{\leq_{\text{lex}}}
\newcommand{\lexless}{<_{\text{lex}}}
\newcommand{\Turingequiv}{\equiv_{\text{T}}}
\newcommand{\isomorphic}{\cong}

%%%%%%%%%%%%%%%%%%%%%%%%%%%%%%%%%%%%%%%%%%%%%
% Constants
%%%%%%%%%%%%%%%%%%%%%%%%%%%%%%%%%%%%%%%%%%%%%
\newcommand{\R}{\mathbb{R}}
\renewcommand{\P}{\mathbb{P}}
\newcommand{\Q}{\mathbb{Q}}
\newcommand{\Z}{\mathbb{Z}}
\newcommand{\Zpos}{\Z^{+}}
\newcommand{\Znonneg}{\Z^{\geq 0}}
\newcommand{\C}{\mathbb{C}}
\newcommand{\N}{\mathbb{N}}
\newcommand{\B}{\mathbb{B}}
\newcommand{\Bairespace}{\pre{\omega}{\omega}}
\newcommand{\LofR}{L(\R)}
\newcommand{\JofR}[1]{J_{#1}(\R)}
\newcommand{\SofR}[1]{S_{#1}(\R)}
\newcommand{\JalphaR}{\JofR{\alpha}}
\newcommand{\JbetaR}{\JofR{\beta}}
\newcommand{\JlambdaR}{\JofR{\lambda}}
\newcommand{\SalphaR}{\SofR{\alpha}}
\newcommand{\SbetaR}{\SofR{\beta}}
\newcommand{\Pkl}{\mathcal{P}_{\kappa}(\lambda)}
\DeclareMathOperator{\con}{con}
\DeclareMathOperator{\ORD}{OR}
\DeclareMathOperator{\Ord}{OR}
\DeclareMathOperator{\WO}{WO}
\DeclareMathOperator{\OD}{OD}
\DeclareMathOperator{\HOD}{HOD}
\DeclareMathOperator{\HC}{HC}
\DeclareMathOperator{\WF}{WF}
\DeclareMathOperator{\wfp}{wfp}
\DeclareMathOperator{\HF}{HF}
\newcommand{\One}{I}
\newcommand{\ONE}{I}
\newcommand{\Two}{II}
\newcommand{\TWO}{II}
\newcommand{\Mladder}{M^{\text{ld}}}

%%%%%%%%%%%%%%%%%%%%%%%%%%%%%%%%%%%%%%%%%%%%%
% Commutative Algebra Constants
%%%%%%%%%%%%%%%%%%%%%%%%%%%%%%%%%%%%%%%%%%%%%
\DeclareMathOperator{\dottimes}{\dot{\times}}
\DeclareMathOperator{\dotminus}{\dot{-}}
\DeclareMathOperator{\Spec}{Spec}

%%%%%%%%%%%%%%%%%%%%%%%%%%%%%%%%%%%%%%%%%%%%%
% Theories
%%%%%%%%%%%%%%%%%%%%%%%%%%%%%%%%%%%%%%%%%%%%%
\DeclareMathOperator{\ZFC}{ZFC}
\DeclareMathOperator{\ZF}{ZF}
\DeclareMathOperator{\AD}{AD}
\DeclareMathOperator{\ADR}{AD_{\R}}
\DeclareMathOperator{\KP}{KP}
\DeclareMathOperator{\PD}{PD}
\DeclareMathOperator{\CH}{CH}
\DeclareMathOperator{\GCH}{GCH}
\DeclareMathOperator{\HPC}{HPC} % HOD pair capturing
%%%%%%%%%%%%%%%%%%%%%%%%%%%%%%%%%%%%%%%%%%%%%
% Iteration Trees
%%%%%%%%%%%%%%%%%%%%%%%%%%%%%%%%%%%%%%%%%%%%%

\newcommand{\pred}{\text{-pred}}

%%%%%%%%%%%%%%%%%%%%%%%%%%%%%%%%%%%%%%%%%%%%%%%%
% Operator Names
%%%%%%%%%%%%%%%%%%%%%%%%%%%%%%%%%%%%%%%%%%%%%%%%
\DeclareMathOperator{\Det}{Det}
\DeclareMathOperator{\dom}{dom}
\DeclareMathOperator{\ran}{ran}
\DeclareMathOperator{\range}{ran}
\DeclareMathOperator{\image}{image}
\DeclareMathOperator{\crit}{crit}
\DeclareMathOperator{\card}{card}
\DeclareMathOperator{\cf}{cf}
\DeclareMathOperator{\cof}{cof}
\DeclareMathOperator{\rank}{rank}
\DeclareMathOperator{\ot}{o.t.}
\DeclareMathOperator{\ords}{o}
\DeclareMathOperator{\ro}{r.o.}
\DeclareMathOperator{\rud}{rud}
\DeclareMathOperator{\Powerset}{\mathcal{P}}
\DeclareMathOperator{\length}{lh}
\DeclareMathOperator{\lh}{lh}
\DeclareMathOperator{\limit}{lim}
\DeclareMathOperator{\fld}{fld}
\DeclareMathOperator{\projection}{p}
\DeclareMathOperator{\Ult}{Ult}
\DeclareMathOperator{\ULT}{Ult}
\DeclareMathOperator{\Coll}{Coll}
\DeclareMathOperator{\Col}{Col}
\DeclareMathOperator{\Hull}{Hull}
\DeclareMathOperator*{\dirlim}{dir lim}
\DeclareMathOperator{\Scale}{Scale}
\DeclareMathOperator{\supp}{supp}
\DeclareMathOperator{\trancl}{tran.cl.}
\DeclareMathOperator{\trace}{Tr}
\DeclareMathOperator{\diag}{diag}
\DeclareMathOperator{\spn}{span}
\DeclareMathOperator{\sgn}{sgn}
%%%%%%%%%%%%%%%%%%%%%%%%%%%%%%%%%%%%%%%%%%%%%
% Logical Connectives
%%%%%%%%%%%%%%%%%%%%%%%%%%%%%%%%%%%%%%%%%%%%%
\newcommand{\IImplies}{\Longrightarrow}
\newcommand{\SkipImplies}{\quad\Longrightarrow\quad}
\newcommand{\Ifff}{\Longleftrightarrow}
\newcommand{\iimplies}{\longrightarrow}
\newcommand{\ifff}{\longleftrightarrow}
\newcommand{\Implies}{\Rightarrow}
\newcommand{\Iff}{\Leftrightarrow}
\renewcommand{\implies}{\rightarrow}
\renewcommand{\iff}{\leftrightarrow}
\newcommand{\AND}{\wedge}
\newcommand{\OR}{\vee}
\newcommand{\st}{\text{ s.t. }}
\newcommand{\Or}{\text{ or }}

%%%%%%%%%%%%%%%%%%%%%%%%%%%%%%%%%%%%%%%%%%%%%
% Function Arrows
%%%%%%%%%%%%%%%%%%%%%%%%%%%%%%%%%%%%%%%%%%%%%

\newcommand{\injection}{\xrightarrow{\text{1-1}}}
\newcommand{\surjection}{\xrightarrow{\text{onto}}}
\newcommand{\bijection}{\xrightarrow[\text{onto}]{\text{1-1}}}
\newcommand{\cofmap}{\xrightarrow{\text{cof}}}
\newcommand{\map}{\rightarrow}

%%%%%%%%%%%%%%%%%%%%%%%%%%%%%%%%%%%%%%%%%%%%%
% Mouse Comparison Operators
%%%%%%%%%%%%%%%%%%%%%%%%%%%%%%%%%%%%%%%%%%%%%
\newcommand{\initseg}{\trianglelefteq}
\newcommand{\properseg}{\lhd}
\newcommand{\notinitseg}{\ntrianglelefteq}
\newcommand{\notproperseg}{\ntriangleleft}

%%%%%%%%%%%%%%%%%%%%%%%%%%%%%%%%%%%%%%%%%%%%%
%           calligraphic letters
%%%%%%%%%%%%%%%%%%%%%%%%%%%%%%%%%%%%%%%%%%%%%
\newcommand{\cA}{\mathcal{A}}
\newcommand{\cB}{\mathcal{B}}
\newcommand{\cC}{\mathcal{C}}
\newcommand{\cD}{\mathcal{D}}
\newcommand{\cE}{\mathcal{E}}
\newcommand{\cF}{\mathcal{F}}
\newcommand{\cG}{\mathcal{G}}
\newcommand{\cH}{\mathcal{H}}
\newcommand{\cI}{\mathcal{I}}
\newcommand{\cJ}{\mathcal{J}}
\newcommand{\cK}{\mathcal{K}}
\newcommand{\cL}{\mathcal{L}}
\newcommand{\cM}{\mathcal{M}}
\newcommand{\cN}{\mathcal{N}}
\newcommand{\cO}{\mathcal{O}}
\newcommand{\cP}{\mathcal{P}}
\newcommand{\cQ}{\mathcal{Q}}
\newcommand{\cR}{\mathcal{R}}
\newcommand{\cS}{\mathcal{S}}
\newcommand{\cT}{\mathcal{T}}
\newcommand{\cU}{\mathcal{U}}
\newcommand{\cV}{\mathcal{V}}
\newcommand{\cW}{\mathcal{W}}
\newcommand{\cX}{\mathcal{X}}
\newcommand{\cY}{\mathcal{Y}}
\newcommand{\cZ}{\mathcal{Z}}


%%%%%%%%%%%%%%%%%%%%%%%%%%%%%%%%%%%%%%%%%%%%%
%          Primed Letters
%%%%%%%%%%%%%%%%%%%%%%%%%%%%%%%%%%%%%%%%%%%%%
\newcommand{\aprime}{a^{\prime}}
\newcommand{\bprime}{b^{\prime}}
\newcommand{\cprime}{c^{\prime}}
\newcommand{\dprime}{d^{\prime}}
\newcommand{\eprime}{e^{\prime}}
\newcommand{\fprime}{f^{\prime}}
\newcommand{\gprime}{g^{\prime}}
\newcommand{\hprime}{h^{\prime}}
\newcommand{\iprime}{i^{\prime}}
\newcommand{\jprime}{j^{\prime}}
\newcommand{\kprime}{k^{\prime}}
\newcommand{\lprime}{l^{\prime}}
\newcommand{\mprime}{m^{\prime}}
\newcommand{\nprime}{n^{\prime}}
\newcommand{\oprime}{o^{\prime}}
\newcommand{\pprime}{p^{\prime}}
\newcommand{\qprime}{q^{\prime}}
\newcommand{\rprime}{r^{\prime}}
\newcommand{\sprime}{s^{\prime}}
\newcommand{\tprime}{t^{\prime}}
\newcommand{\uprime}{u^{\prime}}
\newcommand{\vprime}{v^{\prime}}
\newcommand{\wprime}{w^{\prime}}
\newcommand{\xprime}{x^{\prime}}
\newcommand{\yprime}{y^{\prime}}
\newcommand{\zprime}{z^{\prime}}
\newcommand{\Aprime}{A^{\prime}}
\newcommand{\Bprime}{B^{\prime}}
\newcommand{\Cprime}{C^{\prime}}
\newcommand{\Dprime}{D^{\prime}}
\newcommand{\Eprime}{E^{\prime}}
\newcommand{\Fprime}{F^{\prime}}
\newcommand{\Gprime}{G^{\prime}}
\newcommand{\Hprime}{H^{\prime}}
\newcommand{\Iprime}{I^{\prime}}
\newcommand{\Jprime}{J^{\prime}}
\newcommand{\Kprime}{K^{\prime}}
\newcommand{\Lprime}{L^{\prime}}
\newcommand{\Mprime}{M^{\prime}}
\newcommand{\Nprime}{N^{\prime}}
\newcommand{\Oprime}{O^{\prime}}
\newcommand{\Pprime}{P^{\prime}}
\newcommand{\Qprime}{Q^{\prime}}
\newcommand{\Rprime}{R^{\prime}}
\newcommand{\Sprime}{S^{\prime}}
\newcommand{\Tprime}{T^{\prime}}
\newcommand{\Uprime}{U^{\prime}}
\newcommand{\Vprime}{V^{\prime}}
\newcommand{\Wprime}{W^{\prime}}
\newcommand{\Xprime}{X^{\prime}}
\newcommand{\Yprime}{Y^{\prime}}
\newcommand{\Zprime}{Z^{\prime}}
\newcommand{\alphaprime}{\alpha^{\prime}}
\newcommand{\betaprime}{\beta^{\prime}}
\newcommand{\gammaprime}{\gamma^{\prime}}
\newcommand{\Gammaprime}{\Gamma^{\prime}}
\newcommand{\deltaprime}{\delta^{\prime}}
\newcommand{\epsilonprime}{\epsilon^{\prime}}
\newcommand{\kappaprime}{\kappa^{\prime}}
\newcommand{\lambdaprime}{\lambda^{\prime}}
\newcommand{\rhoprime}{\rho^{\prime}}
\newcommand{\Sigmaprime}{\Sigma^{\prime}}
\newcommand{\tauprime}{\tau^{\prime}}
\newcommand{\xiprime}{\xi^{\prime}}
\newcommand{\thetaprime}{\theta^{\prime}}
\newcommand{\Omegaprime}{\Omega^{\prime}}
\newcommand{\cMprime}{\cM^{\prime}}
\newcommand{\cNprime}{\cN^{\prime}}
\newcommand{\cPprime}{\cP^{\prime}}
\newcommand{\cQprime}{\cQ^{\prime}}
\newcommand{\cRprime}{\cR^{\prime}}
\newcommand{\cSprime}{\cS^{\prime}}
\newcommand{\cTprime}{\cT^{\prime}}

%%%%%%%%%%%%%%%%%%%%%%%%%%%%%%%%%%%%%%%%%%%%%
%          bar Letters
%%%%%%%%%%%%%%%%%%%%%%%%%%%%%%%%%%%%%%%%%%%%%
\newcommand{\abar}{\bar{a}}
\newcommand{\bbar}{\bar{b}}
\newcommand{\cbar}{\bar{c}}
\newcommand{\ibar}{\bar{i}}
\newcommand{\jbar}{\bar{j}}
\newcommand{\nbar}{\bar{n}}
\newcommand{\xbar}{\bar{x}}
\newcommand{\ybar}{\bar{y}}
\newcommand{\zbar}{\bar{z}}
\newcommand{\pibar}{\bar{\pi}}
\newcommand{\phibar}{\bar{\varphi}}
\newcommand{\psibar}{\bar{\psi}}
\newcommand{\thetabar}{\bar{\theta}}
\newcommand{\nubar}{\bar{\nu}}

%%%%%%%%%%%%%%%%%%%%%%%%%%%%%%%%%%%%%%%%%%%%%
%          star Letters
%%%%%%%%%%%%%%%%%%%%%%%%%%%%%%%%%%%%%%%%%%%%%
\newcommand{\phistar}{\phi^{*}}
\newcommand{\Mstar}{M^{*}}

%%%%%%%%%%%%%%%%%%%%%%%%%%%%%%%%%%%%%%%%%%%%%
%          dotletters Letters
%%%%%%%%%%%%%%%%%%%%%%%%%%%%%%%%%%%%%%%%%%%%%
\newcommand{\Gdot}{\dot{G}}

%%%%%%%%%%%%%%%%%%%%%%%%%%%%%%%%%%%%%%%%%%%%%
%         check Letters
%%%%%%%%%%%%%%%%%%%%%%%%%%%%%%%%%%%%%%%%%%%%%
\newcommand{\deltacheck}{\check{\delta}}
\newcommand{\gammacheck}{\check{\gamma}}


%%%%%%%%%%%%%%%%%%%%%%%%%%%%%%%%%%%%%%%%%%%%%
%          Formulas
%%%%%%%%%%%%%%%%%%%%%%%%%%%%%%%%%%%%%%%%%%%%%

\newcommand{\formulaphi}{\text{\large $\varphi$}}
\newcommand{\Formulaphi}{\text{\Large $\varphi$}}


%%%%%%%%%%%%%%%%%%%%%%%%%%%%%%%%%%%%%%%%%%%%%
%          Fraktur Letters
%%%%%%%%%%%%%%%%%%%%%%%%%%%%%%%%%%%%%%%%%%%%%

\newcommand{\fa}{\mathfrak{a}}
\newcommand{\fb}{\mathfrak{b}}
\newcommand{\fc}{\mathfrak{c}}
\newcommand{\fk}{\mathfrak{k}}
\newcommand{\fp}{\mathfrak{p}}
\newcommand{\fq}{\mathfrak{q}}
\newcommand{\fr}{\mathfrak{r}}
\newcommand{\fA}{\mathfrak{A}}
\newcommand{\fB}{\mathfrak{B}}
\newcommand{\fC}{\mathfrak{C}}
\newcommand{\fD}{\mathfrak{D}}

%%%%%%%%%%%%%%%%%%%%%%%%%%%%%%%%%%%%%%%%%%%%%
%          Bold Letters
%%%%%%%%%%%%%%%%%%%%%%%%%%%%%%%%%%%%%%%%%%%%%
\newcommand{\ba}{\mathbf{a}}
\newcommand{\bb}{\mathbf{b}}
\newcommand{\bc}{\mathbf{c}}
\newcommand{\bd}{\mathbf{d}}
\newcommand{\be}{\mathbf{e}}
\newcommand{\bu}{\mathbf{u}}
\newcommand{\bv}{\mathbf{v}}
\newcommand{\bw}{\mathbf{w}}
\newcommand{\bx}{\mathbf{x}}
\newcommand{\by}{\mathbf{y}}
\newcommand{\bz}{\mathbf{z}}
\newcommand{\bSigma}{\boldsymbol{\Sigma}}
\newcommand{\bPi}{\boldsymbol{\Pi}}
\newcommand{\bDelta}{\boldsymbol{\Delta}}
\newcommand{\bdelta}{\boldsymbol{\delta}}
\newcommand{\bgamma}{\boldsymbol{\gamma}}
\newcommand{\bGamma}{\boldsymbol{\Gamma}}

%%%%%%%%%%%%%%%%%%%%%%%%%%%%%%%%%%%%%%%%%%%%%
%         Bold numbers
%%%%%%%%%%%%%%%%%%%%%%%%%%%%%%%%%%%%%%%%%%%%%
\newcommand{\bzero}{\mathbf{0}}

%%%%%%%%%%%%%%%%%%%%%%%%%%%%%%%%%%%%%%%%%%%%%
% Projective-Like Pointclasses
%%%%%%%%%%%%%%%%%%%%%%%%%%%%%%%%%%%%%%%%%%%%%
\newcommand{\Sa}[2][\alpha]{\Sigma_{(#1,#2)}}
\newcommand{\Pa}[2][\alpha]{\Pi_{(#1,#2)}}
\newcommand{\Da}[2][\alpha]{\Delta_{(#1,#2)}}
\newcommand{\Aa}[2][\alpha]{A_{(#1,#2)}}
\newcommand{\Ca}[2][\alpha]{C_{(#1,#2)}}
\newcommand{\Qa}[2][\alpha]{Q_{(#1,#2)}}
\newcommand{\da}[2][\alpha]{\delta_{(#1,#2)}}
\newcommand{\leqa}[2][\alpha]{\leq_{(#1,#2)}}
\newcommand{\lessa}[2][\alpha]{<_{(#1,#2)}}
\newcommand{\equiva}[2][\alpha]{\equiv_{(#1,#2)}}


\newcommand{\Sl}[1]{\Sa[\lambda]{#1}}
\newcommand{\Pl}[1]{\Pa[\lambda]{#1}}
\newcommand{\Dl}[1]{\Da[\lambda]{#1}}
\newcommand{\Al}[1]{\Aa[\lambda]{#1}}
\newcommand{\Cl}[1]{\Ca[\lambda]{#1}}
\newcommand{\Ql}[1]{\Qa[\lambda]{#1}}

\newcommand{\San}{\Sa{n}}
\newcommand{\Pan}{\Pa{n}}
\newcommand{\Dan}{\Da{n}}
\newcommand{\Can}{\Ca{n}}
\newcommand{\Qan}{\Qa{n}}
\newcommand{\Aan}{\Aa{n}}
\newcommand{\dan}{\da{n}}
\newcommand{\leqan}{\leqa{n}}
\newcommand{\lessan}{\lessa{n}}
\newcommand{\equivan}{\equiva{n}}

\newcommand{\SigmaOneOmega}{\Sigma^1_{\omega}}
\newcommand{\SigmaOneOmegaPlusOne}{\Sigma^1_{\omega+1}}
\newcommand{\PiOneOmega}{\Pi^1_{\omega}}
\newcommand{\PiOneOmegaPlusOne}{\Pi^1_{\omega+1}}
\newcommand{\DeltaOneOmegaPlusOne}{\Delta^1_{\omega+1}}

%%%%%%%%%%%%%%%%%%%%%%%%%%%%%%%%%%%%%%%%%%%%%
% Linear Algebra
%%%%%%%%%%%%%%%%%%%%%%%%%%%%%%%%%%%%%%%%%%%%%
\newcommand{\transpose}[1]{{#1}^{\text{T}}}
\newcommand{\norm}[1]{\lVert{#1}\rVert}
\newcommand\aug{\fboxsep=-\fboxrule\!\!\!\fbox{\strut}\!\!\!}

%%%%%%%%%%%%%%%%%%%%%%%%%%%%%%%%%%%%%%%%%%%%%
% Number Theory
%%%%%%%%%%%%%%%%%%%%%%%%%%%%%%%%%%%%%%%%%%%%%
\newcommand{\av}[1]{\lvert#1\rvert}
\DeclareMathOperator{\divides}{\mid}
\DeclareMathOperator{\ndivides}{\nmid}
\DeclareMathOperator{\lcm}{lcm}
\DeclareMathOperator{\sign}{sign}
\newcommand{\floor}[1]{\left\lfloor{#1}\right\rfloor}
\DeclareMathOperator{\ConCl}{CC}
\DeclareMathOperator{\ord}{ord}



\graphicspath{{images/}}

\newtheorem*{claim}{claim}
\newtheorem*{observation}{Observation}
\newtheorem*{warning}{Warning}
\newtheorem*{question}{Question}
\newtheorem{remark}[theorem]{Remark}

\newenvironment*{subproof}[1][Proof]
{\begin{proof}[#1]}{\renewcommand{\qedsymbol}{$\diamondsuit$} \end{proof}}

\mode<presentation>
{
  \usetheme{Singapore}
  % or ...

  \setbeamercovered{invisible}
  % or whatever (possibly just delete it)
}


\usepackage[english]{babel}
% or whatever

\usepackage[latin1]{inputenc}
% or whatever

\usepackage{times}
\usepackage[T1]{fontenc}
% Or whatever. Note that the encoding and the font should match. If T1
% does not look nice, try deleting the line with the fontenc.

\title{Lesson 12 \\ Bases and Dimension}
\subtitle{Math 325, Linear Algebra \\ Fall 2018 \\ SFSU}
\author{Mitch Rudominer}
\date{}



% If you wish to uncover everything in a step-wise fashion, uncomment
% the following command:

\beamerdefaultoverlayspecification{<+->}

\begin{document}

\begin{frame}
  \titlepage
\end{frame}

%%%%%%%%%%%%%%%%%%%%%%%%%%%%%%%%%%%%%%%%%%%%%%%%%%%%%%%%%%%%%%%%%%%%%%%%%

\begin{frame}{Basis}

\begin{itemize}
\item \textbf{Definition.} Let $V$ be a vector space and let $\bv_1,\bv_2,\cdots,\bv_n$ be $n$ vectors in $V$.
\item We say that $\bv_1,\bv_2,\cdots,\bv_n$ is a \emph{basis} for $V$ if
\item (i) $\bv_1,\bv_2,\cdots,\bv_n$ are linearly independent
\item (ii) $\bv_1,\bv_2,\cdots,\bv_n$ span $V$.
\end{itemize}
\end{frame}
%%%%%%%%%%%%%%%%%%%%%%%%%%%%%%%%%%%%%%%%%%%%%%%%%%%%%%%%%%%%%%%%%%%%%%%%%

\begin{frame}{Example of a basis}

\begin{itemize}
\item The standard basis vectors of $\R^n$ are a basis for $\R^n$.
\item Let $V = \R^3$.
\item Let $\be_1 =
\begin{pmatrix}
1 \\ 0 \\ 0
\end{pmatrix}
$,
$\be_2 =
\begin{pmatrix}
0 \\ 1 \\ 0
\end{pmatrix}
$,
$\be_3=
\begin{pmatrix}
0 \\ 0 \\ 1
\end{pmatrix}
$.

\item Then $\be_1,\be_2,\be_3$ are a basis for $\R^3$.
\item To see this we need to recall that they are
\item (i) linearly independent, and
\item (ii) span $\R^3$.
\item So the standard basis vectors form a basis.
\end{itemize}

\end{frame}
%%%%%%%%%%%%%%%%%%%%%%%%%%%%%%%%%%%%%%%%%%%%%%%%%%%%%%%%%%%%%%%%%%%%%%%%%

\begin{frame}{Dimension}

\begin{itemize}
\item \textbf{Definition.} Let $V$ be a vector space
\item Suppose $V$ has a basis consisting of $n$ vectors.
\item Then we say that the \emph{dimension} of $V$ is $n$.
\item Or that $V$ is $n$-dimensional.
\item \textbf{Example.} $\R^n$ is $n$-dimensional.
\item Because $\R^n$ has a basis consisting of the $n$ standard basis vectors.
\end{itemize}

\end{frame}
%%%%%%%%%%%%%%%%%%%%%%%%%%%%%%%%%%%%%%%%%%%%%%%%%%%%%%%%%%%%%%%%%%%%%%%%%
\begin{frame}{Is dimension well-defined?}

\begin{itemize}
\item Let $V$ be an $n$-dimensional vector space.
\item So $V$ has a basis consisting of $n$ vectors.
\item Question: Is it possible that $V$ also has a different basis
consisting of $m$ vectors, for some $m\not=n$?
\item If this is possible, then $V$ is also $m$-dimensional.
\item And so the dimension of $V$ is not well-defined.
\item We will show that this cannot happen.
\item We will show that every basis for $V$ has the same size.
\end{itemize}

\end{frame}
%%%%%%%%%%%%%%%%%%%%%%%%%%%%%%%%%%%%%%%%%%%%%%%%%%%%%%%%%%%%%%%%%%%%%%%%%
\begin{frame}{Dimension is well-defined}

\begin{itemize}
\item \textbf{Lemma.} Let $V$ be a vector space and let $\bv_1,\bv_2,\cdots,\bv_n$ be a basis for $V$.
\item Let $\bw_1,\bw_2,\cdots,\bw_n$  be any $n$ linearly independent vectors of $V$.
\item Then $\bw_1,\bw_2,\cdots,\bw_n$ are also a basis.
\item \textbf{Proof.} Let's just deal with the two dimensional case.  Let $V$ be a two dimensional
vector space with basis $\bv_1,\bv_2$.
\item Let $\bw_1,\bw_2$ be linearly
independent. We will show they also form a basis.
\item We need to show that the span of $\bw_1,\bw_2$ is all of $V$.
\item Notice that it suffices to show that $\bv_1,\bv_2\in\spn(\bw_1,\bw_2)$.
\end{itemize}

\end{frame}

%%%%%%%%%%%%%%%%%%%%%%%%%%%%%%%%%%%%%%%%%%%%%%%%%%%%%%%%%%%%%%%%%%%%%%%%%
\begin{frame}{Proof continued.}

\begin{itemize}
\item $\bw_1 = c_1\bv_1 + c_2\bv_2$, for some scalars $c_1, c_2$.
\item Since $\bw_1\not=\bzero$, either $c_1$ or $c_2$ is nonzero. Say $c_1 \not= 0$.
\item So then $\bv_1 = (1/c_1)\bw_1 - (c_2/c_1)\bv_2$.
\item So $\bv_1\in\spn(\bw_1,\bv_2)$.
\item But then $\bv_1,\bv_2\in\spn(\bw_1,\bv_2)$.
\item So $\spn(\bw_1,\bv_2)=V$.
\end{itemize}

\end{frame}

%%%%%%%%%%%%%%%%%%%%%%%%%%%%%%%%%%%%%%%%%%%%%%%%%%%%%%%%%%%%%%%%%%%%%%%%%
\begin{frame}{Proof continued.}

\begin{itemize}
\item $\bw_2 = c_1\bw_1 + c_2\bv_2$, for some scalars $c_1, c_2$.
\item Since $\bw_1, \bw_2$, are linearly independent, $c_2 \not= 0$.
\item So then $\bv_2 = (1/c_2)\bw_2 - (c_1/c_2)\bw_1$.
\item So $\bv_2$ is in the span of $\bw_1$ and $\bw_2$.
\item So $V=\spn(\bw_1,\bv_2)\subseteq\spn(\bw_1,\bw_2)$.
\item So $\spn(\bw_1,\bw_2) = V$.
\item So $\bw_1,\bw_2$ span all of $V$ and so they form a basis. $\qed$
\end{itemize}

\end{frame}
%%%%%%%%%%%%%%%%%%%%%%%%%%%%%%%%%%%%%%%%%%%%%%%%%%%%%%%%%%%%%%%%%%%%%%%%%
\begin{frame}{More vectors than the dimension}

\begin{itemize}
\item \textbf{Lemma.} Let $V$ be a vector space and let $\bv_1,\bv_2,\cdots,\bv_n$ be a basis for $V$.
\item Let $\bw_1,\bw_2,\cdots\bw_m$ be any other vectors with $m > n$.
\item Then $\bw_1,\bw_2,\cdots\bw_m$ are linearly dependent.
\item \textbf{proof.} Suppose $\bw_1,\bw_2,\cdots\bw_m$ are linearly independent. We will derive a contradiction.
\item $\bw_1,\bw_2,\cdots,\bw_n,\cdots,\bw_m$ are linearly independent.
\item In particular $\bw_1,\bw_2,\cdots,\bw_n$ are linearly independent.
\item So by the previous lemma $\bw_1,\bw_2,\cdots,\bw_n$ are a basis
\item so then $\bw_{n+1}$ can be written as a linear combination of $\bw_1,\bw_2,\cdots,\bw_n$
\item but then $\bw_1,\bw_2,\cdots\bw_m$ are not linearly independent. $\qed$
\end{itemize}

\end{frame}

%%%%%%%%%%%%%%%%%%%%%%%%%%%%%%%%%%%%%%%%%%%%%%%%%%%%%%%%%%%%%%%%%%%%%%%%
\begin{frame}{Example}

\begin{itemize}
\item Let $A=
\begin{pmatrix}
1 & 2 & 3 & 4 \\
5 & 6 & 7 & 8 \\
9 & 10 & 11 & 12\\
\end{pmatrix}
$
\item Is $T_A$ one-to-one?
\item The columns of $A$ consist of four vectors in $\R^3$.
\item $\R^3$ is 3-dimensional.
\item By the previous lemma, a set of four vectors in a three dimensional space is necessarily linearly dependent.
\item So the columns of $A$ are linearly dependent.
\item So there is a nonzero $\bx$ such that $A\bx=\bzero$.
\item So $\ker(T_A)$ is not-trivial.
\item So $T_A$ is not one-to-one.
\end{itemize}

\end{frame}

%%%%%%%%%%%%%%%%%%%%%%%%%%%%%%%%%%%%%%%%%%%%%%%%%%%%%%%%%%%%%%%%%%%%%%%%
\begin{frame}{Underdetermined systems}

\begin{itemize}
\item Let $A$ be a rectangular matrix with more columns than rows.
\item Then the columns of $A$ are linearly dependent. Therefore:
\item There are infinitely many $\bx$ such that $A\bx=\bzero$.
\item $\ker(T_A)$ is not trivial.
\item $T_A$ is not one-to-one.
\item For any $\bb$ the system $A\bx=\bb$ does not have a unique solution:
\item It either has zero solutions or infinitely many solutions.
\item If there are more unknowns than equations there is never a unique solution.
\end{itemize}

\end{frame}
%%%%%%%%%%%%%%%%%%%%%%%%%%%%%%%%%%%%%%%%%%%%%%%%%%%%%%%%%%%%%%%%%%%%%%%%%
\begin{frame}{Dimension is well-defined}

\begin{itemize}
\item \textbf{Theorem.} Let $V$ be a vector space and suppose $V$ has two different bases,
\item one with $n$ elements and one with $m$ elements.
\item Then $n=m$.
\item i.e. all bases for $V$ have the same number of elements. So the dimension of $V$ is well-defined.
\item \textbf{Proof.} This follows from the previous lemma
\item because if one basis had fewer
elements than the other one, then the one with more elements would be linearly dependent. $\qed$
\end{itemize}

\end{frame}

%%%%%%%%%%%%%%%%%%%%%%%%%%%%%%%%%%%%%%%%%%%%%%%%%%%%%%%%%%%%%%%%%%%%%%%%%
\begin{frame}{Other bases for $\R^n$}

\begin{itemize}
\item The standard basis vectors form a basis for $\R^n$.
\item But $\R^n$ has other bases besides the standard basis.
\item But every basis for $\R^n$ has $n$ elements.
\end{itemize}

\end{frame}

%%%%%%%%%%%%%%%%%%%%%%%%%%%%%%%%%%%%%%%%%%%%%%%%%%%%%%%%%%%%%%%%%%%%%%%%%
\begin{frame}{Example: The columns of an upper-triangular matrix.}

\begin{itemize}
\item The columns of an upper-triangular matrix with nonzeros on the
diagonal always form a basis.
\item Let
$
\bv_1 =
\begin{pmatrix}
-4 \\ 0 \\ 0
\end{pmatrix}
$,
$
\bv_2 =
\begin{pmatrix}
3 \\ -8 \\ 0
\end{pmatrix}
$,
$
\bv_3 =
\begin{pmatrix}
-6 \\ 11 \\ -37
\end{pmatrix}
$
\item Then $\bv_1,\bv_2,\bv_3$ form a basis for $\R^3$. Why?
\item They are linearly independent, because the matrix is upper-triangular with nonzero diagonal entries.
\item But then by an earlier lemma, in a three dimensional space, any set of three linearly independent
vectors form a basis.
\end{itemize}

\end{frame}

%%%%%%%%%%%%%%%%%%%%%%%%%%%%%%%%%%%%%%%%%%%%%%%%%%%%%%%%%%%%%%%%%%%%%%%%%
\begin{frame}{Coordinates}

\begin{itemize}
\item \textbf{Theorem.} Let $V$ be an $n$-dimensional vector space. Let $\bv_1,\bv_2,\cdots,\bv_n$ be a basis.
\item Let $\bv$ be any vector in $V$.
\item Then there are unique numbers $c_1,c_2,\cdots,c_n$ such that $\bv = c_1 \bv_1 + c_2\bv_2 + \cdots + c_n\bv_n$.
\item The $c_i$ are called the \emph{coordinates} of $\bv$ with respect to the basis $\bv_1,\bv_2,\cdots,\bv_n$.
\item \textbf{proof}. We know that there is at least one set of coordinates $c_1,c_2,\cdots,c_n$ such that $\bv = c_1 \bv_1 + c_2\bv_2 + \cdots + c_n\bv_n$.
\item This follows from the fact that $\bv_1,\bv_2,\cdots,\bv_n$ span $V$.
\item So suppose that there were a second set of coordinates $d_1,d_2,\cdots,d_n$ such that $\bv = d_1 \bv_1 + d_2\bv_2 + \cdots + d_n\bv_n$.
\end{itemize}
\end{frame}

%%%%%%%%%%%%%%%%%%%%%%%%%%%%%%%%%%%%%%%%%%%%%%%%%%%%%%%%%%%%%%%%%%%%%%%%%
\begin{frame}{Proof continued}

\begin{itemize}
\item So suppose that there were a second set of coordinates $d_1,d_2,\cdots,d_n$ such that $\bv = d_1 \bv_1 + d_2\bv_2 + \cdots + d_n\bv_n$.
\item Then $c_1 \bv_1 + c_2\bv_2 + \cdots + c_n\bv_n = d_1 \bv_1 + d_2\bv_2 + \cdots + d_n\bv_n$ so
\item $(c_1 -d_1)\bv_1 + (c_2 - d_2)\bv_2 + \cdots + (c_n-d_n)\bv_n = \bzero$.
\item Since $\bv_1,\bv_2,\cdots,\bv_n$ are linearly independent we must have that $(c_1 -d_1)=0, (c_2 -d_2)=0,\cdots,(c_n -d_n)=0$.
\item So $c_1=d_1,c_2=d_2,\cdots,c_n=d_n$.
\item So in fact there were never two different sets of coordinates. The two sets of coordinates were the same. $\qed$
\end{itemize}
\end{frame}

%%%%%%%%%%%%%%%%%%%%%%%%%%%%%%%%%%%%%%%%%%%%%%%%%%%%%%%%%%%%%%%%%%%%%%%%%
\begin{frame}{Coordinates in a non-standard basis}

\begin{itemize}
\item Let $\bv_1 =
\begin{pmatrix}
1 \\ 0
\end{pmatrix}
,\bv_2=
\begin{pmatrix}
1 \\ 1
\end{pmatrix}
$.
\item Then $\bv_1,\bv_2$ form a basis for $\R^2$. (We know this because they
form the columns of an upper-triangular matrix with nonzeros on the diagonal.)
\item Let $\bv=
\begin{pmatrix}
2 \\ 5
\end{pmatrix}
$
\item Find the coordinates of $\bv$ in the basis $\bv_1,\bv_2$.
\item This is asking you to find $c_1,c_2$ such that $\bv = c_1\bv_1 + c_2\bv_2$.
\item This is asking you to solve $A\bc = \bv$ where $A=\left(\bv_1 \vert \bv_2\right)$.
\item Solution: $\bv = -3\bv_1 + 5\bv_2$.
\end{itemize}
\end{frame}

%%%%%%%%%%%%%%%%%%%%%%%%%%%%%%%%%%%%%%%%%%%%%%%%%%%%%%%%%%%%%%%%%%%%%%%%%
\begin{frame}{Geometry of coordinates}

\begin{itemize}
\item You can think of any basis for $\R^n$ as giving non-standard coordinate axes.
\item Consider the previous example
\item Let $\bv_1 =
\begin{pmatrix}
1 \\ 0
\end{pmatrix}
,\bv_2=
\begin{pmatrix}
1 \\ 1
\end{pmatrix}
$.
\item We found  $\bv = -3\bv_1 + 5\bv_2$.
\item We can visualize this as: $\bv$ is the point in the plane that is
obtained by starting at the origin and first moving 5 units along the
``$\bv_2$-axis'' and then $-3$ units along the $\bv_1$ axis.
\item Note that each unit along the ``$\bv_2$-axis'' has length the
length of $\bv_2$.
\end{itemize}
\end{frame}
%%%%%%%%%%%%%%%%%%%%%%%%%%%%%%%%%%%%%%%%%%%%%%%%%%%%%%%%%%%%%%%%%%%%%%%%%
\begin{frame}{Basis from a spanning set}

\begin{itemize}
\item \textbf{Theorem.} Let $V$ be a vector space.
\item Let $\bv_1,\bv_2,\cdots,\bv_n$ be a spanning set for $V$.
\item Then there is a subset of the $\bv_i$ that form a basis for $V$.
\item So $\dim(V)\leq n$.
\item \textbf{proof} Let $S$ be a maximal linearly independent subset of the $\bv_i$.
\item That is, let $S$ be a subset of the $\bv_i$ such that the vectors in $S$ are linearly
independent, but if we added any other $\bv_i$ to $S$ it would become linearly dependent.
\item Then every other $\bv_i$ is a linear combination of elements of $S$.
\item Since the $\bv_i$ are a spanning set, every vector in $V$ is a linear combination of elements of $S$.
\item So $S$ spans $V$ and $S$ is linearly independent.
\item So $S$ forms a basis. $\qed$
\end{itemize}
\end{frame}

%%%%%%%%%%%%%%%%%%%%%%%%%%%%%%%%%%%%%%%%%%%%%%%%%%%%%%%%%%%%%%%%%%%%%%%%%
\begin{frame}{Example basis from a spanning set}

\begin{itemize}
\item Let $\bv_1 =
\begin{pmatrix}
2 \\ 0 \\ 0
\end{pmatrix}
,\bv_2=
\begin{pmatrix}
-1 \\ 1 \\ 0
\end{pmatrix}
,\bv_3=
\begin{pmatrix}
1 \\ 1 \\ 0
\end{pmatrix}
,\bv_4=
\begin{pmatrix}
1 \\ 1 \\ 1
\end{pmatrix}
$.
\item $\singleton{\bv_1,\bv_2,\bv_3,\bv_4}$ spans $\R^3$.
\item But the vectors are not linearly independent, so it is not a basis.
\item Let $S_1 = \singleton{\bv_1}$. $S_1$ is a linearly independent set but it is not maximal.
\item Let $S_2 = \singleton{\bv_1,\bv_2}$. $S_2$ is a linearly independent set but it is not maximal.
\item If we were to add $\bv_3$ to $S_2$ the resulting set would not be linearly independent.
\item Let $S=\singleton{\bv_1,\bv_2,\bv_4}$.
\item $S$ is a maximal linearly independent subset of $\singleton{\bv_1,\bv_2,\bv_3,\bv_4}$.
\item $\bv_1,\bv_2,\bv_4$ form a basis.
\end{itemize}
\end{frame}
%%%%%%%%%%%%%%%%%%%%%%%%%%%%%%%%%%%%%%%%%%%%%%%%%%%%%%%%%%%%%%%%%%%%%%%%%
\begin{frame}{Dimension of subspaces}

\begin{itemize}
\item \textbf{Theorem.} Let $V$ be an $n$-dimensional vector space.
\item Let $W$ be a subspace of $V$.
\item Then $\dim(W) \leq \dim(V)$.
\item \textbf{proof}
\item By a previous lemma, any collection of more than $n$ vectors in $V$ is linearly dependent.
\item So then any collection of more than $n$ vectors of $W$ is linearly dependent and so does not form a basis.
\item So we know that the dimension of $W$ cannot be greater than the dimension of $V$.
\item So why aren't we done with the proof yet?
\item How do we know $W$ has a basis at all? That's what we still need to show.
\end{itemize}
\end{frame}

%%%%%%%%%%%%%%%%%%%%%%%%%%%%%%%%%%%%%%%%%%%%%%%%%%%%%%%%%%%%%%%%%%%%%%%%%
\begin{frame}{proof, continued}

\begin{itemize}
\item If $W$ is the zero space then it has dimension zero so we are done.
\item So now assume that $W$ is not the zero space. So let $\bw_1\in W$, $\bw_1\not=\bzero$.
\item If $\bw_1$ spans $W$ then $\singleton{\bw_1}$ is a basis for $W$ and so then $W$ is one dimensional and we are done.
\item So suppose $\bw_1$ does not span $W$. Let $\bw_2\in W$, $\bw_2\notin\spn(\bw_1)$.
\item So then $\singleton{\bw_1,\bw_2}$ are linearly independent.
\item If $\singleton{\bw_1,\bw_2}$ is a spanning set for $W$ then  $\singleton{\bw_1,\bw_2}$ is a basis and
so $W$ is two dimensional and we are done.
\item So suppose $\bw_3\in W$, $\bw_3\notin\spn(\bw_1,\bw_2)$.
\end{itemize}

\end{frame}

%%%%%%%%%%%%%%%%%%%%%%%%%%%%%%%%%%%%%%%%%%%%%%%%%%%%%%%%%%%%%%%%%%%%%%%%%
\begin{frame}{proof, continued}

\begin{itemize}
\item By continuing in this way we can form a larger and larger set of linearly independent vectors.
\item But we know this process must stop because we cannot ever have more then $n$ linearly independent vectors in $V$.
\item So eventually we have a basis for $W$. $\qed$.
\end{itemize}

\end{frame}

%%%%%%%%%%%%%%%%%%%%%%%%%%%%%%%%%%%%%%%%%%%%%%%%%%%%%%%%%%%%%%%%%%%%%%%%%
\begin{frame}{Example: Subspaces of $\R^3$.}


\begin{itemize}
\item Let $W$ be a subspace of $\R^3$.
\item What are the possibilities for what $W$ looks like?
\item $W$ could be the zero space $\singleton{\bzero}$. Then $W$ is zero-dimensional.
\item $W$ could be spanned by a single non-zero vector $\bv$. Then $W$ is one-dimensional.
\item $W$ is the line through $\bv$.
\item Suppose $W$ is not zero or one dimensional. $W$ could be spanned by two linearly independent vectors $\bv_1,\bv_2$. Then $W$ is two dimensional.
\item $W$ is the plane through $\bv_2,\bv_2$.
\item Otherwise $W$ is three dimensional and so $W$ is all of $\R^3$.
\end{itemize}

\end{frame}

%%%%%%%%%%%%%%%%%%%%%%%%%%%%%%%%%%%%%%%%%%%%%%%%%%%%%%%%%%%%%%%%%%%%%%%%%
\begin{frame}{Subspace of full dimension.}

\begin{itemize}
\item \textbf{Theorem.} Let $V$ be an $n$-dimensional vector space.
\item Let $W$ be a subspace of dimension $n$.
\item Then $W=V$.
\item \textbf{proof} Let $\bv_1,\bv_2,\cdots,\bv_n$ be a basis for $W$.
\item Then the $\bv_i$ are linearly independent and there are $n$ of them.
\item So they are also a basis for $V$. So all of $V$ is inside of $W$. $\qed$.
\end{itemize}

\end{frame}

%%%%%%%%%%%%%%%%%%%%%%%%%%%%%%%%%%%%%%%%%%%%%%%%%%%%%%%%%%%%%%%%%%%%%%%%%
\begin{frame}{Infinite dimensional vector spaces}

\begin{itemize}
\item Some vector spaces do not have a finite basis.
\item These are called infinite dimensional.
\item Most of the function spaces we considered are infinite dimensional.
\item For example consider the space of all polynomial functions.
\item Notice that the following set is linearly independent:
\item $x, x^2, x^3, x^4, \cdots$.
\item Since there are infinitely many linearly independent vectors, this vector space cannot have any finite dimension.
\item It is infinite dimensional.
\item In this class we will be concerned almost exclusively with finite dimensional vector spaces.
\end{itemize}

\end{frame}

\end{document}


