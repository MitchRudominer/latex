% $Header$

\documentclass{beamer}
%\documentclass[handout]{beamer}

\usepackage{amsmath,amssymb,latexsym,eucal,amsthm,graphicx,xcolor}
%%%%%%%%%%%%%%%%%%%%%%%%%%%%%%%%%%%%%%%%%%%%%
% Common Set Theory Constructs
%%%%%%%%%%%%%%%%%%%%%%%%%%%%%%%%%%%%%%%%%%%%%

\newcommand{\setof}[2]{\left\{ \, #1 \, \left| \, #2 \, \right.\right\}}
\newcommand{\lsetof}[2]{\left\{\left. \, #1 \, \right| \, #2 \,  \right\}}
\newcommand{\bigsetof}[2]{\bigl\{ \, #1 \, \bigm | \, #2 \,\bigr\}}
\newcommand{\Bigsetof}[2]{\Bigl\{ \, #1 \, \Bigm | \, #2 \,\Bigr\}}
\newcommand{\biggsetof}[2]{\biggl\{ \, #1 \, \biggm | \, #2 \,\biggr\}}
\newcommand{\Biggsetof}[2]{\Biggl\{ \, #1 \, \Biggm | \, #2 \,\Biggr\}}
\newcommand{\dotsetof}[2]{\left\{ \, #1 \, : \, #2 \, \right\}}
\newcommand{\bigdotsetof}[2]{\bigl\{ \, #1 \, : \, #2 \,\bigr\}}
\newcommand{\Bigdotsetof}[2]{\Bigl\{ \, #1 \, \Bigm : \, #2 \,\Bigr\}}
\newcommand{\biggdotsetof}[2]{\biggl\{ \, #1 \, \biggm : \, #2 \,\biggr\}}
\newcommand{\Biggdotsetof}[2]{\Biggl\{ \, #1 \, \Biggm : \, #2 \,\Biggr\}}
\newcommand{\sequence}[2]{\left\langle \, #1 \,\left| \, #2 \, \right. \right\rangle}
\newcommand{\lsequence}[2]{\left\langle\left. \, #1 \, \right| \,#2 \,  \right\rangle}
\newcommand{\bigsequence}[2]{\bigl\langle \,#1 \, \bigm | \, #2 \, \bigr\rangle}
\newcommand{\Bigsequence}[2]{\Bigl\langle \,#1 \, \Bigm | \, #2 \, \Bigr\rangle}
\newcommand{\biggsequence}[2]{\biggl\langle \,#1 \, \biggm | \, #2 \, \biggr\rangle}
\newcommand{\Biggsequence}[2]{\Biggl\langle \,#1 \, \Biggm | \, #2 \, \Biggr\rangle}
\newcommand{\singleton}[1]{\left\{#1\right\}}
\newcommand{\angles}[1]{\left\langle #1 \right\rangle}
\newcommand{\bigangles}[1]{\bigl\langle #1 \bigr\rangle}
\newcommand{\Bigangles}[1]{\Bigl\langle #1 \Bigr\rangle}
\newcommand{\biggangles}[1]{\biggl\langle #1 \biggr\rangle}
\newcommand{\Biggangles}[1]{\Biggl\langle #1 \Biggr\rangle}


\newcommand{\force}[1]{\Vert\!\underset{\!\!\!\!\!#1}{\!\!\!\relbar\!\!\!%
\relbar\!\!\relbar\!\!\relbar\!\!\!\relbar\!\!\relbar\!\!\relbar\!\!\!%
\relbar\!\!\relbar\!\!\relbar}}
\newcommand{\longforce}[1]{\Vert\!\underset{\!\!\!\!\!#1}{\!\!\!\relbar\!\!\!%
\relbar\!\!\relbar\!\!\relbar\!\!\!\relbar\!\!\relbar\!\!\relbar\!\!\!%
\relbar\!\!\relbar\!\!\relbar\!\!\relbar\!\!\relbar\!\!\relbar\!\!\relbar\!\!\relbar}}
\newcommand{\nforce}[1]{\Vert\!\underset{\!\!\!\!\!#1}{\!\!\!\relbar\!\!\!%
\relbar\!\!\relbar\!\!\relbar\!\!\!\relbar\!\!\relbar\!\!\relbar\!\!\!%
\relbar\!\!\not\relbar\!\!\relbar}}
\newcommand{\forcein}[2]{\overset{#2}{\Vert\underset{\!\!\!\!\!#1}%
{\!\!\!\relbar\!\!\!\relbar\!\!\relbar\!\!\relbar\!\!\!\relbar\!\!\relbar\!%
\!\relbar\!\!\!\relbar\!\!\relbar\!\!\relbar\!\!\relbar\!\!\!\relbar\!\!%
\relbar\!\!\relbar}}}

\newcommand{\pre}[2]{{}^{#2}{#1}}

\newcommand{\restr}{\!\!\upharpoonright\!}

%%%%%%%%%%%%%%%%%%%%%%%%%%%%%%%%%%%%%%%%%%%%%
% Set-Theoretic Connectives
%%%%%%%%%%%%%%%%%%%%%%%%%%%%%%%%%%%%%%%%%%%%%

\newcommand{\intersect}{\cap}
\newcommand{\union}{\cup}
\newcommand{\Intersection}[1]{\bigcap\limits_{#1}}
\newcommand{\Union}[1]{\bigcup\limits_{#1}}
\newcommand{\adjoin}{{}^\frown}
\newcommand{\forces}{\Vdash}

%%%%%%%%%%%%%%%%%%%%%%%%%%%%%%%%%%%%%%%%%%%%%
% Miscellaneous
%%%%%%%%%%%%%%%%%%%%%%%%%%%%%%%%%%%%%%%%%%%%%
\newcommand{\defeq}{=_{\text{def}}}
\newcommand{\Turingleq}{\leq_{\text{T}}}
\newcommand{\Turingless}{<_{\text{T}}}
\newcommand{\lexleq}{\leq_{\text{lex}}}
\newcommand{\lexless}{<_{\text{lex}}}
\newcommand{\Turingequiv}{\equiv_{\text{T}}}
\newcommand{\isomorphic}{\cong}

%%%%%%%%%%%%%%%%%%%%%%%%%%%%%%%%%%%%%%%%%%%%%
% Constants
%%%%%%%%%%%%%%%%%%%%%%%%%%%%%%%%%%%%%%%%%%%%%
\newcommand{\R}{\mathbb{R}}
\renewcommand{\P}{\mathbb{P}}
\newcommand{\Q}{\mathbb{Q}}
\newcommand{\Z}{\mathbb{Z}}
\newcommand{\Zpos}{\Z^{+}}
\newcommand{\Znonneg}{\Z^{\geq 0}}
\newcommand{\C}{\mathbb{C}}
\newcommand{\N}{\mathbb{N}}
\newcommand{\B}{\mathbb{B}}
\newcommand{\Bairespace}{\pre{\omega}{\omega}}
\newcommand{\LofR}{L(\R)}
\newcommand{\JofR}[1]{J_{#1}(\R)}
\newcommand{\SofR}[1]{S_{#1}(\R)}
\newcommand{\JalphaR}{\JofR{\alpha}}
\newcommand{\JbetaR}{\JofR{\beta}}
\newcommand{\JlambdaR}{\JofR{\lambda}}
\newcommand{\SalphaR}{\SofR{\alpha}}
\newcommand{\SbetaR}{\SofR{\beta}}
\newcommand{\Pkl}{\mathcal{P}_{\kappa}(\lambda)}
\DeclareMathOperator{\con}{con}
\DeclareMathOperator{\ORD}{OR}
\DeclareMathOperator{\Ord}{OR}
\DeclareMathOperator{\WO}{WO}
\DeclareMathOperator{\OD}{OD}
\DeclareMathOperator{\HOD}{HOD}
\DeclareMathOperator{\HC}{HC}
\DeclareMathOperator{\WF}{WF}
\DeclareMathOperator{\wfp}{wfp}
\DeclareMathOperator{\HF}{HF}
\newcommand{\One}{I}
\newcommand{\ONE}{I}
\newcommand{\Two}{II}
\newcommand{\TWO}{II}
\newcommand{\Mladder}{M^{\text{ld}}}

%%%%%%%%%%%%%%%%%%%%%%%%%%%%%%%%%%%%%%%%%%%%%
% Commutative Algebra Constants
%%%%%%%%%%%%%%%%%%%%%%%%%%%%%%%%%%%%%%%%%%%%%
\DeclareMathOperator{\dottimes}{\dot{\times}}
\DeclareMathOperator{\dotminus}{\dot{-}}
\DeclareMathOperator{\Spec}{Spec}

%%%%%%%%%%%%%%%%%%%%%%%%%%%%%%%%%%%%%%%%%%%%%
% Theories
%%%%%%%%%%%%%%%%%%%%%%%%%%%%%%%%%%%%%%%%%%%%%
\DeclareMathOperator{\ZFC}{ZFC}
\DeclareMathOperator{\ZF}{ZF}
\DeclareMathOperator{\AD}{AD}
\DeclareMathOperator{\ADR}{AD_{\R}}
\DeclareMathOperator{\KP}{KP}
\DeclareMathOperator{\PD}{PD}
\DeclareMathOperator{\CH}{CH}
\DeclareMathOperator{\GCH}{GCH}
\DeclareMathOperator{\HPC}{HPC} % HOD pair capturing
%%%%%%%%%%%%%%%%%%%%%%%%%%%%%%%%%%%%%%%%%%%%%
% Iteration Trees
%%%%%%%%%%%%%%%%%%%%%%%%%%%%%%%%%%%%%%%%%%%%%

\newcommand{\pred}{\text{-pred}}

%%%%%%%%%%%%%%%%%%%%%%%%%%%%%%%%%%%%%%%%%%%%%%%%
% Operator Names
%%%%%%%%%%%%%%%%%%%%%%%%%%%%%%%%%%%%%%%%%%%%%%%%
\DeclareMathOperator{\Det}{Det}
\DeclareMathOperator{\dom}{dom}
\DeclareMathOperator{\ran}{ran}
\DeclareMathOperator{\range}{ran}
\DeclareMathOperator{\image}{image}
\DeclareMathOperator{\crit}{crit}
\DeclareMathOperator{\card}{card}
\DeclareMathOperator{\cf}{cf}
\DeclareMathOperator{\cof}{cof}
\DeclareMathOperator{\rank}{rank}
\DeclareMathOperator{\ot}{o.t.}
\DeclareMathOperator{\ords}{o}
\DeclareMathOperator{\ro}{r.o.}
\DeclareMathOperator{\rud}{rud}
\DeclareMathOperator{\Powerset}{\mathcal{P}}
\DeclareMathOperator{\length}{lh}
\DeclareMathOperator{\lh}{lh}
\DeclareMathOperator{\limit}{lim}
\DeclareMathOperator{\fld}{fld}
\DeclareMathOperator{\projection}{p}
\DeclareMathOperator{\Ult}{Ult}
\DeclareMathOperator{\ULT}{Ult}
\DeclareMathOperator{\Coll}{Coll}
\DeclareMathOperator{\Col}{Col}
\DeclareMathOperator{\Hull}{Hull}
\DeclareMathOperator*{\dirlim}{dir lim}
\DeclareMathOperator{\Scale}{Scale}
\DeclareMathOperator{\supp}{supp}
\DeclareMathOperator{\trancl}{tran.cl.}
\DeclareMathOperator{\trace}{Tr}
\DeclareMathOperator{\diag}{diag}
\DeclareMathOperator{\spn}{span}
\DeclareMathOperator{\sgn}{sgn}
%%%%%%%%%%%%%%%%%%%%%%%%%%%%%%%%%%%%%%%%%%%%%
% Logical Connectives
%%%%%%%%%%%%%%%%%%%%%%%%%%%%%%%%%%%%%%%%%%%%%
\newcommand{\IImplies}{\Longrightarrow}
\newcommand{\SkipImplies}{\quad\Longrightarrow\quad}
\newcommand{\Ifff}{\Longleftrightarrow}
\newcommand{\iimplies}{\longrightarrow}
\newcommand{\ifff}{\longleftrightarrow}
\newcommand{\Implies}{\Rightarrow}
\newcommand{\Iff}{\Leftrightarrow}
\renewcommand{\implies}{\rightarrow}
\renewcommand{\iff}{\leftrightarrow}
\newcommand{\AND}{\wedge}
\newcommand{\OR}{\vee}
\newcommand{\st}{\text{ s.t. }}
\newcommand{\Or}{\text{ or }}

%%%%%%%%%%%%%%%%%%%%%%%%%%%%%%%%%%%%%%%%%%%%%
% Function Arrows
%%%%%%%%%%%%%%%%%%%%%%%%%%%%%%%%%%%%%%%%%%%%%

\newcommand{\injection}{\xrightarrow{\text{1-1}}}
\newcommand{\surjection}{\xrightarrow{\text{onto}}}
\newcommand{\bijection}{\xrightarrow[\text{onto}]{\text{1-1}}}
\newcommand{\cofmap}{\xrightarrow{\text{cof}}}
\newcommand{\map}{\rightarrow}

%%%%%%%%%%%%%%%%%%%%%%%%%%%%%%%%%%%%%%%%%%%%%
% Mouse Comparison Operators
%%%%%%%%%%%%%%%%%%%%%%%%%%%%%%%%%%%%%%%%%%%%%
\newcommand{\initseg}{\trianglelefteq}
\newcommand{\properseg}{\lhd}
\newcommand{\notinitseg}{\ntrianglelefteq}
\newcommand{\notproperseg}{\ntriangleleft}

%%%%%%%%%%%%%%%%%%%%%%%%%%%%%%%%%%%%%%%%%%%%%
%           calligraphic letters
%%%%%%%%%%%%%%%%%%%%%%%%%%%%%%%%%%%%%%%%%%%%%
\newcommand{\cA}{\mathcal{A}}
\newcommand{\cB}{\mathcal{B}}
\newcommand{\cC}{\mathcal{C}}
\newcommand{\cD}{\mathcal{D}}
\newcommand{\cE}{\mathcal{E}}
\newcommand{\cF}{\mathcal{F}}
\newcommand{\cG}{\mathcal{G}}
\newcommand{\cH}{\mathcal{H}}
\newcommand{\cI}{\mathcal{I}}
\newcommand{\cJ}{\mathcal{J}}
\newcommand{\cK}{\mathcal{K}}
\newcommand{\cL}{\mathcal{L}}
\newcommand{\cM}{\mathcal{M}}
\newcommand{\cN}{\mathcal{N}}
\newcommand{\cO}{\mathcal{O}}
\newcommand{\cP}{\mathcal{P}}
\newcommand{\cQ}{\mathcal{Q}}
\newcommand{\cR}{\mathcal{R}}
\newcommand{\cS}{\mathcal{S}}
\newcommand{\cT}{\mathcal{T}}
\newcommand{\cU}{\mathcal{U}}
\newcommand{\cV}{\mathcal{V}}
\newcommand{\cW}{\mathcal{W}}
\newcommand{\cX}{\mathcal{X}}
\newcommand{\cY}{\mathcal{Y}}
\newcommand{\cZ}{\mathcal{Z}}


%%%%%%%%%%%%%%%%%%%%%%%%%%%%%%%%%%%%%%%%%%%%%
%          Primed Letters
%%%%%%%%%%%%%%%%%%%%%%%%%%%%%%%%%%%%%%%%%%%%%
\newcommand{\aprime}{a^{\prime}}
\newcommand{\bprime}{b^{\prime}}
\newcommand{\cprime}{c^{\prime}}
\newcommand{\dprime}{d^{\prime}}
\newcommand{\eprime}{e^{\prime}}
\newcommand{\fprime}{f^{\prime}}
\newcommand{\gprime}{g^{\prime}}
\newcommand{\hprime}{h^{\prime}}
\newcommand{\iprime}{i^{\prime}}
\newcommand{\jprime}{j^{\prime}}
\newcommand{\kprime}{k^{\prime}}
\newcommand{\lprime}{l^{\prime}}
\newcommand{\mprime}{m^{\prime}}
\newcommand{\nprime}{n^{\prime}}
\newcommand{\oprime}{o^{\prime}}
\newcommand{\pprime}{p^{\prime}}
\newcommand{\qprime}{q^{\prime}}
\newcommand{\rprime}{r^{\prime}}
\newcommand{\sprime}{s^{\prime}}
\newcommand{\tprime}{t^{\prime}}
\newcommand{\uprime}{u^{\prime}}
\newcommand{\vprime}{v^{\prime}}
\newcommand{\wprime}{w^{\prime}}
\newcommand{\xprime}{x^{\prime}}
\newcommand{\yprime}{y^{\prime}}
\newcommand{\zprime}{z^{\prime}}
\newcommand{\Aprime}{A^{\prime}}
\newcommand{\Bprime}{B^{\prime}}
\newcommand{\Cprime}{C^{\prime}}
\newcommand{\Dprime}{D^{\prime}}
\newcommand{\Eprime}{E^{\prime}}
\newcommand{\Fprime}{F^{\prime}}
\newcommand{\Gprime}{G^{\prime}}
\newcommand{\Hprime}{H^{\prime}}
\newcommand{\Iprime}{I^{\prime}}
\newcommand{\Jprime}{J^{\prime}}
\newcommand{\Kprime}{K^{\prime}}
\newcommand{\Lprime}{L^{\prime}}
\newcommand{\Mprime}{M^{\prime}}
\newcommand{\Nprime}{N^{\prime}}
\newcommand{\Oprime}{O^{\prime}}
\newcommand{\Pprime}{P^{\prime}}
\newcommand{\Qprime}{Q^{\prime}}
\newcommand{\Rprime}{R^{\prime}}
\newcommand{\Sprime}{S^{\prime}}
\newcommand{\Tprime}{T^{\prime}}
\newcommand{\Uprime}{U^{\prime}}
\newcommand{\Vprime}{V^{\prime}}
\newcommand{\Wprime}{W^{\prime}}
\newcommand{\Xprime}{X^{\prime}}
\newcommand{\Yprime}{Y^{\prime}}
\newcommand{\Zprime}{Z^{\prime}}
\newcommand{\alphaprime}{\alpha^{\prime}}
\newcommand{\betaprime}{\beta^{\prime}}
\newcommand{\gammaprime}{\gamma^{\prime}}
\newcommand{\Gammaprime}{\Gamma^{\prime}}
\newcommand{\deltaprime}{\delta^{\prime}}
\newcommand{\epsilonprime}{\epsilon^{\prime}}
\newcommand{\kappaprime}{\kappa^{\prime}}
\newcommand{\lambdaprime}{\lambda^{\prime}}
\newcommand{\rhoprime}{\rho^{\prime}}
\newcommand{\Sigmaprime}{\Sigma^{\prime}}
\newcommand{\tauprime}{\tau^{\prime}}
\newcommand{\xiprime}{\xi^{\prime}}
\newcommand{\thetaprime}{\theta^{\prime}}
\newcommand{\Omegaprime}{\Omega^{\prime}}
\newcommand{\cMprime}{\cM^{\prime}}
\newcommand{\cNprime}{\cN^{\prime}}
\newcommand{\cPprime}{\cP^{\prime}}
\newcommand{\cQprime}{\cQ^{\prime}}
\newcommand{\cRprime}{\cR^{\prime}}
\newcommand{\cSprime}{\cS^{\prime}}
\newcommand{\cTprime}{\cT^{\prime}}

%%%%%%%%%%%%%%%%%%%%%%%%%%%%%%%%%%%%%%%%%%%%%
%          bar Letters
%%%%%%%%%%%%%%%%%%%%%%%%%%%%%%%%%%%%%%%%%%%%%
\newcommand{\abar}{\bar{a}}
\newcommand{\bbar}{\bar{b}}
\newcommand{\cbar}{\bar{c}}
\newcommand{\ibar}{\bar{i}}
\newcommand{\jbar}{\bar{j}}
\newcommand{\nbar}{\bar{n}}
\newcommand{\xbar}{\bar{x}}
\newcommand{\ybar}{\bar{y}}
\newcommand{\zbar}{\bar{z}}
\newcommand{\pibar}{\bar{\pi}}
\newcommand{\phibar}{\bar{\varphi}}
\newcommand{\psibar}{\bar{\psi}}
\newcommand{\thetabar}{\bar{\theta}}
\newcommand{\nubar}{\bar{\nu}}

%%%%%%%%%%%%%%%%%%%%%%%%%%%%%%%%%%%%%%%%%%%%%
%          star Letters
%%%%%%%%%%%%%%%%%%%%%%%%%%%%%%%%%%%%%%%%%%%%%
\newcommand{\phistar}{\phi^{*}}
\newcommand{\Mstar}{M^{*}}

%%%%%%%%%%%%%%%%%%%%%%%%%%%%%%%%%%%%%%%%%%%%%
%          dotletters Letters
%%%%%%%%%%%%%%%%%%%%%%%%%%%%%%%%%%%%%%%%%%%%%
\newcommand{\Gdot}{\dot{G}}

%%%%%%%%%%%%%%%%%%%%%%%%%%%%%%%%%%%%%%%%%%%%%
%         check Letters
%%%%%%%%%%%%%%%%%%%%%%%%%%%%%%%%%%%%%%%%%%%%%
\newcommand{\deltacheck}{\check{\delta}}
\newcommand{\gammacheck}{\check{\gamma}}


%%%%%%%%%%%%%%%%%%%%%%%%%%%%%%%%%%%%%%%%%%%%%
%          Formulas
%%%%%%%%%%%%%%%%%%%%%%%%%%%%%%%%%%%%%%%%%%%%%

\newcommand{\formulaphi}{\text{\large $\varphi$}}
\newcommand{\Formulaphi}{\text{\Large $\varphi$}}


%%%%%%%%%%%%%%%%%%%%%%%%%%%%%%%%%%%%%%%%%%%%%
%          Fraktur Letters
%%%%%%%%%%%%%%%%%%%%%%%%%%%%%%%%%%%%%%%%%%%%%

\newcommand{\fa}{\mathfrak{a}}
\newcommand{\fb}{\mathfrak{b}}
\newcommand{\fc}{\mathfrak{c}}
\newcommand{\fk}{\mathfrak{k}}
\newcommand{\fp}{\mathfrak{p}}
\newcommand{\fq}{\mathfrak{q}}
\newcommand{\fr}{\mathfrak{r}}
\newcommand{\fA}{\mathfrak{A}}
\newcommand{\fB}{\mathfrak{B}}
\newcommand{\fC}{\mathfrak{C}}
\newcommand{\fD}{\mathfrak{D}}

%%%%%%%%%%%%%%%%%%%%%%%%%%%%%%%%%%%%%%%%%%%%%
%          Bold Letters
%%%%%%%%%%%%%%%%%%%%%%%%%%%%%%%%%%%%%%%%%%%%%
\newcommand{\ba}{\mathbf{a}}
\newcommand{\bb}{\mathbf{b}}
\newcommand{\bc}{\mathbf{c}}
\newcommand{\bd}{\mathbf{d}}
\newcommand{\be}{\mathbf{e}}
\newcommand{\bu}{\mathbf{u}}
\newcommand{\bv}{\mathbf{v}}
\newcommand{\bw}{\mathbf{w}}
\newcommand{\bx}{\mathbf{x}}
\newcommand{\by}{\mathbf{y}}
\newcommand{\bz}{\mathbf{z}}
\newcommand{\bSigma}{\boldsymbol{\Sigma}}
\newcommand{\bPi}{\boldsymbol{\Pi}}
\newcommand{\bDelta}{\boldsymbol{\Delta}}
\newcommand{\bdelta}{\boldsymbol{\delta}}
\newcommand{\bgamma}{\boldsymbol{\gamma}}
\newcommand{\bGamma}{\boldsymbol{\Gamma}}

%%%%%%%%%%%%%%%%%%%%%%%%%%%%%%%%%%%%%%%%%%%%%
%         Bold numbers
%%%%%%%%%%%%%%%%%%%%%%%%%%%%%%%%%%%%%%%%%%%%%
\newcommand{\bzero}{\mathbf{0}}

%%%%%%%%%%%%%%%%%%%%%%%%%%%%%%%%%%%%%%%%%%%%%
% Projective-Like Pointclasses
%%%%%%%%%%%%%%%%%%%%%%%%%%%%%%%%%%%%%%%%%%%%%
\newcommand{\Sa}[2][\alpha]{\Sigma_{(#1,#2)}}
\newcommand{\Pa}[2][\alpha]{\Pi_{(#1,#2)}}
\newcommand{\Da}[2][\alpha]{\Delta_{(#1,#2)}}
\newcommand{\Aa}[2][\alpha]{A_{(#1,#2)}}
\newcommand{\Ca}[2][\alpha]{C_{(#1,#2)}}
\newcommand{\Qa}[2][\alpha]{Q_{(#1,#2)}}
\newcommand{\da}[2][\alpha]{\delta_{(#1,#2)}}
\newcommand{\leqa}[2][\alpha]{\leq_{(#1,#2)}}
\newcommand{\lessa}[2][\alpha]{<_{(#1,#2)}}
\newcommand{\equiva}[2][\alpha]{\equiv_{(#1,#2)}}


\newcommand{\Sl}[1]{\Sa[\lambda]{#1}}
\newcommand{\Pl}[1]{\Pa[\lambda]{#1}}
\newcommand{\Dl}[1]{\Da[\lambda]{#1}}
\newcommand{\Al}[1]{\Aa[\lambda]{#1}}
\newcommand{\Cl}[1]{\Ca[\lambda]{#1}}
\newcommand{\Ql}[1]{\Qa[\lambda]{#1}}

\newcommand{\San}{\Sa{n}}
\newcommand{\Pan}{\Pa{n}}
\newcommand{\Dan}{\Da{n}}
\newcommand{\Can}{\Ca{n}}
\newcommand{\Qan}{\Qa{n}}
\newcommand{\Aan}{\Aa{n}}
\newcommand{\dan}{\da{n}}
\newcommand{\leqan}{\leqa{n}}
\newcommand{\lessan}{\lessa{n}}
\newcommand{\equivan}{\equiva{n}}

\newcommand{\SigmaOneOmega}{\Sigma^1_{\omega}}
\newcommand{\SigmaOneOmegaPlusOne}{\Sigma^1_{\omega+1}}
\newcommand{\PiOneOmega}{\Pi^1_{\omega}}
\newcommand{\PiOneOmegaPlusOne}{\Pi^1_{\omega+1}}
\newcommand{\DeltaOneOmegaPlusOne}{\Delta^1_{\omega+1}}

%%%%%%%%%%%%%%%%%%%%%%%%%%%%%%%%%%%%%%%%%%%%%
% Linear Algebra
%%%%%%%%%%%%%%%%%%%%%%%%%%%%%%%%%%%%%%%%%%%%%
\newcommand{\transpose}[1]{{#1}^{\text{T}}}
\newcommand{\norm}[1]{\lVert{#1}\rVert}
\newcommand\aug{\fboxsep=-\fboxrule\!\!\!\fbox{\strut}\!\!\!}

%%%%%%%%%%%%%%%%%%%%%%%%%%%%%%%%%%%%%%%%%%%%%
% Number Theory
%%%%%%%%%%%%%%%%%%%%%%%%%%%%%%%%%%%%%%%%%%%%%
\newcommand{\av}[1]{\lvert#1\rvert}
\DeclareMathOperator{\divides}{\mid}
\DeclareMathOperator{\ndivides}{\nmid}
\DeclareMathOperator{\lcm}{lcm}
\DeclareMathOperator{\sign}{sign}
\newcommand{\floor}[1]{\left\lfloor{#1}\right\rfloor}
\DeclareMathOperator{\ConCl}{CC}
\DeclareMathOperator{\ord}{ord}



\graphicspath{{images/}}

\newtheorem*{claim}{claim}
\newtheorem*{observation}{Observation}
\newtheorem*{warning}{Warning}
\newtheorem*{question}{Question}
\newtheorem{remark}[theorem]{Remark}

\newenvironment*{subproof}[1][Proof]
{\begin{proof}[#1]}{\renewcommand{\qedsymbol}{$\diamondsuit$} \end{proof}}

\mode<presentation>
{
  \usetheme{Singapore}
  % or ...

  \setbeamercovered{invisible}
  % or whatever (possibly just delete it)
}


\usepackage[english]{babel}
% or whatever

\usepackage[latin1]{inputenc}
% or whatever

\usepackage{times}
\usepackage[T1]{fontenc}
% Or whatever. Note that the encoding and the font should match. If T1
% does not look nice, try deleting the line with the fontenc.

\title{Lesson 19 \\ Inner Products and Orthogonality}
\subtitle{Math 325, Linear Algebra \\ Fall 2018 \\ SFSU}
\author{Mitch Rudominer}
\date{}



% If you wish to uncover everything in a step-wise fashion, uncomment
% the following command:

\beamerdefaultoverlayspecification{<+->}

\begin{document}

\begin{frame}
  \titlepage
\end{frame}

%%%%%%%%%%%%%%%%%%%%%%%%%%%%%%%%%%%%%%%%%%%%%%%%%%%%%%%%%%%%%%%%%%%%%%%%%

\begin{frame}{$\R^n$ is more than a vector space}

\begin{itemize}
\item If $\bv$ is a vector in $\R^n$ we know how to take its length.
\item But if $\bv$ is a vector in an abstract vector space, there is no meaning to its length.
\item If $\bv$ and $\bw$ are vectors in $\R^n$ we know how to measure the angle between them.
\item We know whether or not the two vectors are orthogonal.
\item But if $\bv$ and $\bw$ are vectors in an abstract vector space, the notions of angle and orthogonal don't make sense.
\item What is going on? Why do these geometric concepts make sense in $\R^n$ but not in an abstract vector space?
\item $\R^n$ is more than just a vector space. $\R^n$ has additional structure.
\item It is possible to give an abstract vector space this additional structure.
\item It is called an inner product.
\end{itemize}
\end{frame}

%%%%%%%%%%%%%%%%%%%%%%%%%%%%%%%%%%%%%%%%%%%%%%%%%%%%%%%%%%%%%%%%%%%%%%%%%

\begin{frame}{Inner Product on a Real Vector Space}

\begin{itemize}
\item \textbf{Definition} Let $V$ be a real vector space.
\item An \emph{inner product} on $V$ is a function from $V\times V$ to $\R$
\item written as $\angles{\bv,\bw}$, such that, the function is
\item (i) Bilinear: $\angles{c\bv+d\bu,\bw} = c\angles{\bv,\bw} + d\angles{\bu,\bw}$, and
\item $\angles{\bv,c\bu + d\bw} = c\angles{\bv,\bu} + d\angles{\bv,\bw}$, and
\item (ii) Symmetric: $\angles{\bv,\bw}=\angles{\bw,\bv}$, and
\item (iii) Positive Definite: $\angles{\bv,\bv} > 0$ if $\bv\not=0$.
\item A vector space $V$ along with an inner product $\angles{\cdot, \cdot}$, is
called an \emph{inner product space.}
\end{itemize}
\end{frame}

%%%%%%%%%%%%%%%%%%%%%%%%%%%%%%%%%%%%%%%%%%%%%%%%%%%%%%%%%%%%%%%%%%%%%%%%%

\begin{frame}{The Dot Product on  $\R^n$}

\begin{itemize}
\item The main example of an inner product is the dot product on $\R^n$.
\item Recall that for $\bv,\bw\in\R^n$, $\bv\cdot\bw = \transpose{\bv}\bw$.
\item If $\bv = \transpose{(v_1,v_2,\cdots,v_n)}$ and $\bw = \transpose{(w_1,w_2,\cdots,w_n)}$
\item then $\bv\cdot\bw = v_1 w_1 + v_2 w_2 + \cdots + v_n w_n$.
\end{itemize}
\end{frame}

%%%%%%%%%%%%%%%%%%%%%%%%%%%%%%%%%%%%%%%%%%%%%%%%%%%%%%%%%%%%%%%%%%%%%%%%%

\begin{frame}{The Dot Product is an inner product.}

\begin{itemize}
\item \textbf{Claim} The function $\angles{\bv,\bw} = \bv\cdot\bw$ is an inner product.
\item Consequently $\R^n$ along with the dot product is an inner-product space.
\item \textbf{proof.} Firstly, since $\bv\cdot\bw\in\R$,  the dot product
is a function from $\R^n\times\R^n$ to $\R$.
\item Next we need to check the three properties:
\item (i) Bilinear: $\angles{c\bv+d\bu,\bw} = c\angles{\bv,\bw} + d\angles{\bu,\bw}$.
\item $(c\bv+d\bu) \cdot \bw = c(\bv\cdot\bw) + d(\bu\cdot\bw)$
\item (ii) Symmetric: $\angles{\bv,\bw}=\angles{\bw,\bv}$.
\item $\bv\cdot\bw = \bw \cdot \bv$
\item (iii) Positive Definite: $\angles{\bv,\bv} > 0$ if $\bv\not=0$.
\item If $\bv = \transpose{(v_1,v_2,\cdots,v_n)}$ then
\item $\bv\cdot\bv = v_1^2 + v_2^2 + \cdot v_n^2 \geq 0$.
\item If $\bv\not=0$  then one of the $v_i\not=0$ and so $\bv\cdot\bv > 0$.
\end{itemize}
\end{frame}

%%%%%%%%%%%%%%%%%%%%%%%%%%%%%%%%%%%%%%%%%%%%%%%%%%%%%%%%%%%%%%%%%%%%%%%%%

\begin{frame}{Length and Angle}

\begin{itemize}
\item Let $\left(V,\angles{\cdot,\cdot}\right)$ be and inner product space.
\item The inner product on $V$ imbues $V$ with a kind of \emph{geometry.}
\item It allows us to define \emph{length} and \emph{angle}.
\item We start with length.
\end{itemize}
\end{frame}
%%%%%%%%%%%%%%%%%%%%%%%%%%%%%%%%%%%%%%%%%%%%%%%%%%%%%%%%%%%%%%%%%%%%%%%%%

\begin{frame}{Norms on vector spaces}

\begin{itemize}
\item In Euclidean space $\R^n$, we know how to measure the length of a vector.
\item When we generalize the notion of length to an abstract vector space we get:
\item \textbf{Definition.} Let $V$ be a vector space.
\item A \emph{norm} on $V$ is a function from $V$ to $\R$, written $\norm{\bv}$, such that
\item (i) $\norm{\bv} > 0 $ for all $\bv\in V, \bv\not=0$.
\item (ii) $\norm{c\bv} = c\norm{\bv}$ for all $\bv\in V$ and all scalars $c$.
\item (iii) triangle inequality: $\norm{\bv+\bw} \leq \norm{\bv} + \norm{\bw}$.
\item You should think of a norm on a vector space as a way of assigning a length to the vectors.
\end{itemize}
\end{frame}


%%%%%%%%%%%%%%%%%%%%%%%%%%%%%%%%%%%%%%%%%%%%%%%%%%%%%%%%%%%%%%%%%%%%%%%%%
\begin{frame}{The norm associated with an inner product}

\begin{itemize}
\item Every inner product space comes with an associated norm.
\item \textbf{Definition} Let $\left(V, \angles{\cdot,\cdot}\right)$ be an inner-product space.
\item Then there is an \emph{associated norm} defined as follows:
\item $\norm{\bv} = \sqrt{\angles{\bv,\bv}}$.
\item \textbf{Claim.} This gives a norm.
\item \textbf{proof.}(i) $\norm{\bv} > 0 $ for all $\bv\in V, \bv\not=0$.
\item $\angles{\bv,\bv} > 0$ by positive definiteness, so $\sqrt{\angles{\bv,\bv}} > 0$.
\item (ii) $\norm{c\bv} = c\norm{\bv}$ for all $\bv\in V$ and all scalars $c$.
\item $\sqrt{\angles{c\bv,c\bv}} = \sqrt{c^2 \angles{\bv,\bv}} = c\sqrt{\angles{\bv,\bv}}$.
\item (iii) triangle inequality: $\norm{\bv+\bw} \leq \norm{\bv} + \norm{\bw}$.
\item In order to prove the triangle inequality we need another famous inequality.
\end{itemize}
\end{frame}

%%%%%%%%%%%%%%%%%%%%%%%%%%%%%%%%%%%%%%%%%%%%%%%%%%%%%%%%%%%%%%%%%%%%%%%%%
\begin{frame}{The Cauchy-Schwarz Inequality}

\begin{itemize}
\item \textbf{Theorem}  Let $\left(V, \angles{\cdot,\cdot}\right)$ be an inner-product space.
\item Then for all $\bv,\bw\in V$
\item $|\angles{\bv,\bw}| \leq \norm{\bv} \norm{\bw}$.
\item See the proof in the textbook. We will omit it here.
\item But for an example, let's see why the inequality is true for the dot product on $\R^n$.
\item You may remember from other classes that in $\R^2$ and $\R^3$, we have the following geometric property of the dot product:
\item $\bv\cdot\bw = \norm{\bv}\norm{\bw}\cos(\theta)$ where
\item $\theta$ is the acute angle between $\bv$ and $\bw$.
\item Since $|\cos(\theta)|\leq 1$, it follows that  $|\bv\cdot\bw| \leq \norm{\bv}\norm{\bw}$
\end{itemize}
\end{frame}

%%%%%%%%%%%%%%%%%%%%%%%%%%%%%%%%%%%%%%%%%%%%%%%%%%%%%%%%%%%%%%%%%%%%%%%%%
\begin{frame}{The Triangle Inequality}

\begin{itemize}
\item The Cauchy-Scharz inequality allows us to prove the triangle inequality
\item (iii) triangle inequality: $\norm{\bv+\bw} \leq \norm{\bv} + \norm{\bw}$.
\item $\norm{\bv+\bw}^2 = \angles{\bv+\bw,\bv+\bw}$
\item $=\angles{\bv,\bv}+2\angles{\bv,\bw} +\angles{\bw,\bw}$, by bi-linearity
\item $\leq \norm{\bv}^2+2\norm{\bv}\norm{\bw} +\norm{\bw}^2$, by Cauchy-Shwarz
\item $= \left(\norm{\bv} + \norm{\bw}\right)^2$.
\item So $\norm{\bv+\bw} \leq \norm{\bv} + \norm{\bw}$ $\qed$
\end{itemize}
\end{frame}

%%%%%%%%%%%%%%%%%%%%%%%%%%%%%%%%%%%%%%%%%%%%%%%%%%%%%%%%%%%%%%%%%%%%%%%%%
\begin{frame}{Definition of angle}

\begin{itemize}
\item \textbf{Definition.} Let $\left(V, \angles{\cdot,\cdot}\right)$ be an inner-product space.
\item Let $\bv$ and $\bw$ be two nonzero vectors.
\item Then the \emph{angle between} $\bv$ and $\bw$ is defined to be
\item the arccosine of $\angles{\bv,\bw}/\norm{\bv}\norm{\bw}$.
\item By Cauchy-Shwarz $\lvert\angles{\bv,\bw}/\norm{\bv}\norm{\bw}\rvert\leq 1$ so the arcosine is defined.
\end{itemize}
\end{frame}

%%%%%%%%%%%%%%%%%%%%%%%%%%%%%%%%%%%%%%%%%%%%%%%%%%%%%%%%%%%%%%%%%%%%%%%%%
\begin{frame}{Example}

\begin{itemize}
\item In $\R^2$, let $\bv=(1,2)^{T}$ and $\bw=(2,3)^{T}$
\item Use the standard dot product as the inner product on $\R^2$.
\item Find $\norm{\bv}, \norm{\bw}$ and the angle between $\bv$ and $\bw$.
\item Verify the Cauchy-Schwarz inequality for these two vectors
\item $\angles{\bv,\bw} = \bv\cdot\bw = 2 + 6 = 8$.
\item $\norm{\bv} = \sqrt{\angles{\bv,\bv}} =\sqrt{\bv\cdot\bv} = \sqrt{1^2+2^2} = \sqrt{5}$.
\item $\norm{\bw} = \sqrt{\angles{\bw,\bw}} =\sqrt{\bw\cdot\bw} = \sqrt{2^2+3^2} = \sqrt{13}$.
\end{itemize}
\end{frame}

%%%%%%%%%%%%%%%%%%%%%%%%%%%%%%%%%%%%%%%%%%%%%%%%%%%%%%%%%%%%%%%%%%%%%%%%%
\begin{frame}{Example continued}

\begin{itemize}
\item $\angles{\bv,\bw} =  8$.
\item $\norm{\bv} = \sqrt{5}$.
\item $\norm{\bw} =  \sqrt{13}$.
\item Cauchy-Schwarz: $|\angles{\bv,\bw}| \leq \norm{\bv} \norm{\bw}$.
\item $|8| \leq \sqrt{5}\sqrt{13} = \sqrt{65}$. True since $8=\sqrt{64}$.
\item The angle between the two vectors is $\arccos\left(\angles{\bv,\bw}/\norm{\bv}\norm{\bw}\right)$
\item $=\arccos(8/\sqrt{65}) = \arccos(\sqrt{64/65})$
\item This is about $0.124$ radians or about $7$ degrees.
\end{itemize}
\end{frame}

%%%%%%%%%%%%%%%%%%%%%%%%%%%%%%%%%%%%%%%%%%%%%%%%%%%%%%%%%%%%%%%%%%%%%%%%%
\begin{frame}{Orthogonality}

\begin{itemize}
\item \textbf{Definition.} Let $\left(V, \angles{\cdot,\cdot}\right)$ be an inner-product space.
\item Let $\bv$ and $\bw$ be two vectors in $V$. Then we say that $\bv$ and $\bw$ are \emph{orthogonal}
\item if $\angles{\bv,\bw}=0$.
\item Example: In Euclidean space $\R^n$, equipped with the dot product,
\item two vectors are orthogonal iff they are perpendicular, i.e. meet at right angles.
\item Orthogonality is the abstract analog of that in arbitrary vector spaces.
\end{itemize}
\end{frame}

%%%%%%%%%%%%%%%%%%%%%%%%%%%%%%%%%%%%%%%%%%%%%%%%%%%%%%%%%%%%%%%%%%%%%%%%%
\begin{frame}{Orthonormal basis}

\begin{itemize}
\item \textbf{Definition.} Let $V$ be an inner-product space.
\item Let $\bv_1,\bv_2,\cdots,\bv_n$ be a set of vectors in $V$.
\item Then the set is called \emph{orthogonal} iff every two distinct vectors in it are orthogonal.
\item i.e. $\angles{\bv_i,\bv_j} = 0$ for all $i\not=j$.
\item The set is called \emph{orthonormal} if it is orthogonal and in addition all of the elements have norm one.
\item i.e. $\angles{\bv_i,\bv_i} = 1$ for all $i$.
\item Of particular importance is the case where $\bv_1,\bv_2,\cdots,\bv_n$  form a basis for $V$.
\item Then we say they form an orthogonal (or orthonormal) basis.
\end{itemize}
\end{frame}

%%%%%%%%%%%%%%%%%%%%%%%%%%%%%%%%%%%%%%%%%%%%%%%%%%%%%%%%%%%%%%%%%%%%%%%%%
\begin{frame}{Converting orthogonal to orthonormal}

\begin{itemize}
\item If $\bv_1,\bv_2,\cdots,\bv_n$ is an orthogonal basis then we can convert it into an orthonormal basis,
\item by dividing each basis element by its norm.
\item Notice that none of the basis elements can have norm zero.
\item Let $\bu_1 = \bv_1/\norm{\bv_1}, \bu_2=\bv_2/\norm{\bv_2},\cdots,\bu_n=\bv_n/\norm{\bv_n}$.
\item Then $\bu_1,\bu_2,\cdots,\bu_n$ is an orthonormal basis.
\item \textbf{proof.} First notice that the norms of the $\bu_i$ are 1.
\item $\norm{\bu_i} = \norm{\frac{\bv_i}{\norm{\bv_i}}} = \frac{1}{\norm{\bv_i}}\norm{\bv_i} = 1$.
\item Then notice that the $\bu_i$ are still orthogonal.
\item $\angles{\bu_i,\bu_j} = \angles{\frac{\bv_i}{\norm{\bv_i}},\frac{\bv_j}{\norm{\bv_j}}}$
\item $= \frac{1}{\norm{\bv_i}} \frac{1}{\norm{\bv_j}}\angles{\bv_i,\bv_j} = 0$.
\end{itemize}
\end{frame}

%%%%%%%%%%%%%%%%%%%%%%%%%%%%%%%%%%%%%%%%%%%%%%%%%%%%%%%%%%%%%%%%%%%%%%%%%
\begin{frame}{Example}

\begin{itemize}
\item The standard basis in $\R^n$ is an orthonormal basis.
\item For example in $\R^3$ with
\item $\be_1=
\begin{pmatrix}
1\\0\\0
\end{pmatrix},
\be_2=
\begin{pmatrix}
0\\1\\0
\end{pmatrix},
\be_3=
\begin{pmatrix}
0\\0\\1
\end{pmatrix}
$
\item We have $\angles{\be_1,\be_2} = \angles{\be_1,\be_3} = \angles{\be_2,\be_3} = 0$.
\item $\angles{\be_1,\be_1}=\angles{\be_2,\be_2}=\angles{\be_3,\be_3}=1$.
\item So $\singleton{\be_1,\be_2,\be_3}$ is an orthonormal basis for $\R^3$.
\end{itemize}
\end{frame}

%%%%%%%%%%%%%%%%%%%%%%%%%%%%%%%%%%%%%%%%%%%%%%%%%%%%%%%%%%%%%%%%%%%%%%%%%
\begin{frame}{Example}

\begin{itemize}
\item In $\R^2$, let $\bv=
\begin{pmatrix}
1 \\ 1
\end{pmatrix}
\text{, and }
\bw=
\begin{pmatrix}
2 \\ 3
\end{pmatrix}
$.
\item Then $\angles{\bv,\bw} = 2 + 3 = 5$.
\item So $\bv$ and $\bw$ are not orthogonal.
\item But they are linearly independent and since there are two of them in $\R^2$ they form a basis.
\item But it is not an orthogonal basis.
\end{itemize}
\end{frame}

%%%%%%%%%%%%%%%%%%%%%%%%%%%%%%%%%%%%%%%%%%%%%%%%%%%%%%%%%%%%%%%%%%%%%%%%%
\begin{frame}{Example}

\begin{itemize}
\item In $\R^2$, let $\bv=
\begin{pmatrix}
1 \\ 1
\end{pmatrix}
\text{, and }
\bw=
\begin{pmatrix}
-1 \\ 1
\end{pmatrix}
$.
\item Then $\angles{\bv,\bw}=0$. So $\bv$ and $\bw$ are orthogonal.
\item $\angles{\bv,\bv}=\angles{\bw,\bw}=2$.
\item $\norm{\bv}=\norm{\bw}=\sqrt{2}$.
\item So $\singleton{\bv,\bw}$ is an orthogonal basis for $\R^2$ but not an orthonormal basis.
\item Let $\bu_1 = \frac{1}{\sqrt{2}}\bv_1, \bu_2=\frac{1}{\sqrt{2}}\bw$.
\item Then $\singleton{\bu_1,\bu_2}$ is an orthonormal basis for $\R^2$.
\end{itemize}
\end{frame}

%%%%%%%%%%%%%%%%%%%%%%%%%%%%%%%%%%%%%%%%%%%%%%%%%%%%%%%%%%%%%%%%%%%%%%%%%
\begin{frame}{Coordinates in an orthonormal basis}
\begin{itemize}
\item Working in an orthonormal basis has advantages.
\item For example it is easy to compute the coordinates of any vector relative to an orthonormal basis.
\item \textbf{Theorem} Let $\bv_1,\bv_2,\cdots,\bv_n$ be an orthogonal basis for an inner-product space.
\item Let $\bv$ be any vector in the inner product space. Then
\item $$\bv=\frac{\angles{\bv,\bv_1}}{\norm{\bv_1}^2}\bv_1 + \frac{\angles{\bv,\bv_2}}{\norm{\bv_2}^2}\bv_2+\cdots+\frac{\angles{\bv,\bv_n}}{\norm{\bv_n}^2}\bv_n.$$
\item In particular if $\bv_1,\bv_2,\cdots,\bv_n$ is an orthonormal basis then
\item $$\bv=\angles{\bv,\bv_1}\bv_1+\angles{\bv,\bv_2}\bv_2+\cdots+\angles{\bv,\bv_n}\bv_n.$$
\end{itemize}
\end{frame}

%%%%%%%%%%%%%%%%%%%%%%%%%%%%%%%%%%%%%%%%%%%%%%%%%%%%%%%%%%%%%%%%%%%%%%%%%
\begin{frame}{Computing coordinates in different bases}
\begin{itemize}
\item Let's compute the coordinates of the vector $\bv=\begin{pmatrix} 1\\2 \end{pmatrix}$ in three different bases.
\item Basis 1: $\singleton{
\frac{1}{\sqrt{2}}
\begin{pmatrix} 1\\1 \end{pmatrix},
\frac{1}{\sqrt{2}}
\begin{pmatrix} -1\\1 \end{pmatrix},
}$
\item Basis 2: $\singleton{
\begin{pmatrix} 1\\1 \end{pmatrix},
\begin{pmatrix} -1\\1 \end{pmatrix},
}$
\item Basis 3: $\singleton{
\begin{pmatrix} 1\\1 \end{pmatrix},
\begin{pmatrix} 2\\3 \end{pmatrix},
}$
\end{itemize}
\end{frame}

%%%%%%%%%%%%%%%%%%%%%%%%%%%%%%%%%%%%%%%%%%%%%%%%%%%%%%%%%%%%%%%%%%%%%%%%%
\begin{frame}{Computing coordinates in an orthonormal basis}
\begin{itemize}
\item Let's compute the coordinates of the vector $\bv=\begin{pmatrix} 1\\2 \end{pmatrix}$ in basis 1.
\item Basis 1: $\singleton{
\bv_1=\frac{1}{\sqrt{2}}
\begin{pmatrix} 1\\1 \end{pmatrix},
\bv_2=\frac{1}{\sqrt{2}}
\begin{pmatrix} -1\\1 \end{pmatrix},
}$
\item This is an orthonormal basis so the theorem tells us that $\bv=\angles{\bv,\bv_1}\bv_1+\angles{\bv,\bv_2}\bv_2$.
\item $\angles{\bv,\bv_1} = \frac{1}{\sqrt{2}}(1 + 2) = \frac{3}{\sqrt{2}}$
\item $\angles{\bv,\bv_2} = \frac{1}{\sqrt{2}}(-1 + 2) = \frac{1}{\sqrt{2}}$
\item So $\bv=\frac{3}{\sqrt{2}}\bv_1 + \frac{1}{\sqrt{2}}\bv_2$.
\item Check
\begin{align*}
\frac{3}{\sqrt{2}}\frac{1}{\sqrt{2}}
\begin{pmatrix}1\\1\\\end{pmatrix} +  \frac{1}{\sqrt{2}}\frac{1}{\sqrt{2}}\begin{pmatrix}-1\\1\end{pmatrix}
&=\frac{3}{2}\begin{pmatrix}1\\1\end{pmatrix}+\frac{1}{2}\begin{pmatrix}-1\\1\end{pmatrix} \\
&=\begin{pmatrix}1\\2\end{pmatrix}
\end{align*}
\end{itemize}
\end{frame}

%%%%%%%%%%%%%%%%%%%%%%%%%%%%%%%%%%%%%%%%%%%%%%%%%%%%%%%%%%%%%%%%%%%%%%%%%
\begin{frame}{Computing coordinates in an orthogonal basis}
\begin{itemize}
\item Let's compute the coordinates of the vector $\bv=\begin{pmatrix} 1\\2 \end{pmatrix}$ in basis 2.
\item Basis 2: $\singleton{
\bv_1=
\begin{pmatrix} 1\\1 \end{pmatrix},
\bv_2=
\begin{pmatrix} -1\\1 \end{pmatrix},
}$
\item This is an orthogonal basis so the theorem tells us that $\bv=\frac{\angles{\bv,\bv_1}}{\norm{\bv_1}^2}\bv_1 + \frac{\angles{\bv,\bv_2}}{\norm{\bv_2}^2}\bv_2$.
\item $\angles{\bv,\bv_1} = 1 + 2 = 3$
\item $\angles{\bv,\bv_2} = -1 + 2 = 1$.
\item $\norm{\bv_1}^2=2$, $\norm{\bv_2}^2=2$.
\item So $\bv=\frac{3}{2}\bv_1 + \frac{1}{2}\bv_2$.
\item Check
$$
\frac{3}{2}
\begin{pmatrix}1\\1\\\end{pmatrix} +  \frac{1}{2}\begin{pmatrix}-1\\1\end{pmatrix}
=\begin{pmatrix}1\\2\end{pmatrix}
$$
\end{itemize}
\end{frame}

%%%%%%%%%%%%%%%%%%%%%%%%%%%%%%%%%%%%%%%%%%%%%%%%%%%%%%%%%%%%%%%%%%%%%%%%%
\begin{frame}{Computing coordinates in a non-orthogonal basis}
\begin{itemize}
\item Let's compute the coordinates of the vector $\bv=\begin{pmatrix} 1\\2 \end{pmatrix}$ in basis 3.
\item Basis 3: $\singleton{
\bv_1=
\begin{pmatrix} 1\\1 \end{pmatrix},
\bv_2=
\begin{pmatrix} 2\\3 \end{pmatrix},
}$
\item This basis is not orthogonal so the theorem doesn't apply. We have to fall back to earlier ideas.
\item We are looking for two scalars $x$ and $y$ such that $x\bv_1+y\bv_2=\bv$
\item This is a system of linear equations. Solve
\item $
\begin{pmatrix}
1 & 2 \\
1 & 3
\end{pmatrix}
\begin{pmatrix}
x \\
y
\end{pmatrix}
=
\begin{pmatrix}
1 \\
2
\end{pmatrix}
$
\end{itemize}
\end{frame}
%%%%%%%%%%%%%%%%%%%%%%%%%%%%%%%%%%%%%%%%%%%%%%%%%%%%%%%%%%%%%%%%%%%%%%%%%
\begin{frame}{Computing coordinates in a non-orthogonal basis}
\begin{itemize}
\item Form the augmented matrix and do Gaussian elimination and back-substitution
\item $\begin{pmatrix}
1 & 2 & \aug & 1\\
1 & 3 & \aug & 2
\end{pmatrix}
\SkipImplies
\begin{pmatrix}
1 & 2 & \aug & 1\\
0 & 1 & \aug & 1
\end{pmatrix}
$
\item $y=1$
\item $x + 2y = 1 \SkipImplies x +2 = 1 \SkipImplies x = -1$.
\item So $\bv = -\bv_1 + \bv_2$.
\item Check
$$
-\begin{pmatrix}1\\1\\\end{pmatrix} + \begin{pmatrix}2\\3\end{pmatrix}
=\begin{pmatrix}1\\2\end{pmatrix}
$$
\end{itemize}
\end{frame}
%%%%%%%%%%%%%%%%%%%%%%%%%%%%%%%%%%%%%%%%%%%%%%%%%%%%%%%%%%%%%%%%%%%%%%%%%
\begin{frame}{Proof of the theorem}
\begin{itemize}
\item \textbf{Proof.} Suppose $\bv_1,\bv_2,\cdots,\bv_n$ is an orthogonal basis.
\item Let $\bv$ be any vector. Then $\bv$ has some coordinates in terms of the basis.
\item So write $\bv = c_1\bv_1 + c_2\bv_2 + \cdots + c_n\bv_n$.
\item Now let's compute the inner-product of $\bv$ with each of the basis elements.
\item $\angles{\bv,\bv_i} = c_1\angles{\bv_1,\bv_i} + c_2\angles{\bv_2,\bv_i} + \cdots + c_i\angles{\bv_i,\bv_i} + \cdots + c_n\angles{\bv_n,\bv_i}$.
\item Since the set of basis vectors is an orthogonal set, $\angles{\bv_j,\bv_i} = 0$ for all $j\not=i$.
\item So $\angles{\bv,\bv_i} = c_i\angles{\bv_i,\bv_i}$.
\item So $c_i = \frac{\angles{\bv,\bv_i}}{\angles{\bv_i,\bv_i}} = \frac{\angles{\bv,\bv_i}}{\norm{\bv_i}^2}$. $\qed$
\end{itemize}
\end{frame}

%%%%%%%%%%%%%%%%%%%%%%%%%%%%%%%%%%%%%%%%%%%%%%%%%%%%%%%%%%%%%%%%%%%%%%%%%
\begin{frame}{An orthonormal set is automatically linearly independent}
\begin{itemize}
\item \textbf{Prop.} Suppose $\bv_1,\bv_2,\cdots,\bv_n$ is an orthogonal set.
\item Then it is also a linearly independent set. (So it is a basis for the space it spans.)
\item \textbf{proof.} Suppose $c_1\bv_1+c_2\bv_2+\cdots+c_n\bv_n=\bzero$.
\item We will show that $c_1=c_2=\cdots=c_n=0$.
\item Take the inner-product of both sides of the equation above with the vector $\bv_i$.
\item For all $j\not=i$, $\angles{\bv_i,\bv_j} = 0$. So we get
\item $c_i\angles{\bv_i,\bv_i} = \angles{\bv_i,\bzero}$.
\item So $c_i\norm{\bv_i}^2 = 0$.
\item So $c_i=0$.
\item But this is true for all $i$, so $c_i=0$ for all $i$. $\qed$.
\end{itemize}

\end{frame}
%%%%%%%%%%%%%%%%%%%%%%%%%%%%%%%%%%%%%%%%%%%%%%%%%%%%%%%%%%%%%%%%%%%%%%%%%
\begin{frame}{A vector orthogonal to a subspace}
\begin{itemize}
\item \textbf{Definition} Suppose $W$ is a subspace of the inner-product space $V$.
\item Let $\bv$ be a vector in $V$.
\item Then we say that $\bv$ is orthogonal to $W$, written $\bv\perp W$, if
\item $\bv$ is orthogonal to every element of $W$.
\item Example: In $\R^3$ let $W$ be the $x$,$y$-plane and let $\bv=\be_3$.
\item Then $\bv\perp W$.
\end{itemize}
\end{frame}
%%%%%%%%%%%%%%%%%%%%%%%%%%%%%%%%%%%%%%%%%%%%%%%%%%%%%%%%%%%%%%%%%%%%%%%%%
\begin{frame}{A vector orthogonal to a basis}
\begin{itemize}
\item \textbf{Lemma} Suppose $W$ is a subspace of the inner-product space $V$.
\item Let $\bw_1,\bw_2,\cdots,\bw_n$ be a (not necessarily orthogonal) spanning set for $W$.
\item Let $\bv$ be a vector in $V$.
\item Suppose that $\bv\perp\bw_1,\bv\perp\bw_2\cdots,\bv\perp\bw_n$.
\item Then $\bv\perp W$.
\item \textbf{proof} Let $\bw\in W$. We must show that $\bv\perp\bw$.
\item We can write $\bw=c_1\bw_1+c_2\bw_2+\cdots+c_n\bw_n$.
\item So $\angles{\bv,\bw} = c_1\angles{\bv,\bw_1} + c_2\angles{\bv,\bw_2}+\cdots+\angles{\bv,\bw_n}$
\item $=0 + 0 + \cdots + 0$.
\item So $\bv\perp\bw$. $\qed$.
\end{itemize}
\end{frame}
%%%%%%%%%%%%%%%%%%%%%%%%%%%%%%%%%%%%%%%%%%%%%%%%%%%%%%%%%%%%%%%%%%%%%%%%%
\begin{frame}{Example}
\begin{itemize}
\item In $\R^3$ let $W$ be the plane spanned by the vectors
\item $\bw_1=
\begin{pmatrix}
1 \\ 1 \\ 2
\end{pmatrix},
\bw_2=
\begin{pmatrix}
0 \\ -1 \\ 2
\end{pmatrix}
$.
\item Let $\bv=
\begin{pmatrix}
-4 \\ 2 \\ 1
\end{pmatrix}
$
\item Then $\bv\perp W$.
\item To see this just note that $\bv\perp\bw_1$ and $\bv\perp\bw_2$.
\end{itemize}
\end{frame}

%%%%%%%%%%%%%%%%%%%%%%%%%%%%%%%%%%%%%%%%%%%%%%%%%%%%%%%%%%%%%%%%%%%%%%%%%
\begin{frame}{Orthogonal subspaces}
\begin{itemize}
\item \textbf{Definition} Suppose $W$ and $U$ are subspaces of the inner-product space $V$.
\item Then we say that $W$ and $U$ are orthogonal, written $W\perp U$ iff
\item for all $\bw\in W$ and all $\bu\in U$, $\bw\perp\bu$.
\item Example. In $\R^3$ the $x$- and $y$-axes are orthogonal.
\item Example. In $\R^3$ the $x$- $y$-plane and the $z$-axis are orthogonal.
\end{itemize}
\end{frame}

%%%%%%%%%%%%%%%%%%%%%%%%%%%%%%%%%%%%%%%%%%%%%%%%%%%%%%%%%%%%%%%%%%%%%%%%%
\begin{frame}{Orthogonal Compliment}
\begin{itemize}
\item \textbf{Definition} Let $W$ be a subspace of the inner-product space $V$.
\item The \emph{orthogonal compliment} of $W$, written $W^{\perp}$, also pronounced ``$W$-perp'', is
\item $\setof{\bv\in V}{\bv\perp W}$.
\item Example: Let $W$ be the $x$-, $y$-plane in $\R^3$. Then $W^{\perp}$ is the $z$-axis.
\item Example: Let $W$ be the $z$-axis in $\R^3$. Then $W^{\perp}$ is the $x$-, $y$-plane.
\end{itemize}
\end{frame}

%%%%%%%%%%%%%%%%%%%%%%%%%%%%%%%%%%%%%%%%%%%%%%%%%%%%%%%%%%%%%%%%%%%%%%%%%
\begin{frame}{Orthogonal Compliment is a Subspace}
\begin{itemize}
\item \textbf{Proposition} Let $W$ be a subspace of the inner-product space $V$.
\item Then $W^{\perp}$ is a subspace.
\item \textbf{proof.} Let's check the three conditions of being a subspace:
\item (1) $\bzero\in W^{\perp}$.
\item This is true because $\angles{\bzero, \bv} = 0$ for all $\bv$.
\item (2) $\bv,\bu\in W^{\perp} \SkipImplies \bv+\bu \in W^{\perp}$.
\item This is true because $\angles{\bv+\bu,\bw}=\angles{\bv,\bw}+\angles{\bu,\bw} = 0 + 0 = 0$.
\item (3) $\bv\in W \SkipImplies c\bv\in W$.
\item This is true because $\angles{c\bv,\bw} = c\angles{\bv,\bw} = c0 = 0$. $\qed$
\end{itemize}
\end{frame}

%%%%%%%%%%%%%%%%%%%%%%%%%%%%%%%%%%%%%%%%%%%%%%%%%%%%%%%%%%%%%%%%%%%%%%%%%
\begin{frame}{$W\intersect W^{\perp}$}
\begin{itemize}
\item \textbf{Proposition} Let $W$ be a subspace of the inner-product space $V$.
\item Then $W\intersect W^{\perp} = \singleton{\bzero}$.
\item \textbf{proof.} Suppose $\bv\in W\intersect W^{\perp}$.
\item But then $\bv$ is orthogonal to itself.
\item So $\angles{\bv,\bv} = 0$.
\item So $\norm{\bv} = 0$.
\item So $\bv=\bzero$. $\qed$.
\end{itemize}
\end{frame}

%%%%%%%%%%%%%%%%%%%%%%%%%%%%%%%%%%%%%%%%%%%%%%%%%%%%%%%%%%%%%%%%%%%%%%%%%
\begin{frame}{Orthonormal Bases Exist}
\begin{itemize}
\item \textbf{Theorem} Let $V$ be a finite dimensional inner product space.
\item Then $V$ has an orthonormal bases.
\item \textbf{Proof sketch} Let $\bv_1,\bv_2,\cdots,\bv_n$ be any bases for $V$.
\item There is an algorithm called the \emph{Gram-Schmidt algorithm} that modifies
this bases to produce an orthonormal bases $\bu_1,\bu_2,\cdots,\bu_n$.
\item We don't have time to learn this algorithm. Read Section 4.2 if you are interested.
\item Here we'll give a rough sketch.
\end{itemize}
\end{frame}
%%%%%%%%%%%%%%%%%%%%%%%%%%%%%%%%%%%%%%%%%%%%%%%%%%%%%%%%%%%%%%%%%%%%%%%%%
\begin{frame}{The Gram-Schmidt Algorithm}
\begin{itemize}
\item Let $\bu_1 = \frac{1}{\norm{\bv_1}}\bv_1$.
\item Then $\norm{\bu_1} = 1$ and $\bu_1,\bv_2,\cdots,\bv_n$ is still a basis.
\item Let $\bw_2 = \bv_2 - \angles{\bv_2,\bu_1}\bu_1$.
\item Then $\angles{\bw_2,\bu_1} = \angles{\bv_2,\bu_1} - \angles{\bv_2,\bu_1}\angles{\bu_1,\bu_1} = 0$.
\item Let $\bu_2 = \frac{1}{\norm{\bw_2}}\bw_2$.
\item Then $\norm{\bu_2}=1$ and $\angles{\bu_1,\bu_2}=0$ and $\bu_1, \bu_2,\bv_3,\cdots,\bv_n$ is still a basis.
\item Continuing in this we we replace each $\bv_i$ with a $\bu_i$ and so obtain an orthonormal basis. $\qed$.
\end{itemize}
\end{frame}

%%%%%%%%%%%%%%%%%%%%%%%%%%%%%%%%%%%%%%%%%%%%%%%%%%%%%%%%%%%%%%%%%%%%%%%%%
\begin{frame}{Orthogonal projection}
\begin{itemize}
\item \textbf{Theorem} Let $V$ be a finite dimensional inner product space.
\item Let $W$ be a subspace of $V$. Let $\bv\in V$.
\item Then there are unique vectors $\bv_1,\bv_2$ such that
\item $\bv=\bv_1+\bv_2$ and
\item $\bv_1 \in W$ and $\bv_2\in W^{\perp}$.
\item Intuitively: $\bv_1$ is the ``part of $\bv$ parallel to $W$'', and
\item $\bv_2$ is the ``part of $\bv$ orthogonal to $W$.''
\item \textbf{Definition.} $\bv_1$ is called the \emph{orthogonal} projection of $\bv$ onto $W$.
\end{itemize}

\end{frame}
%%%%%%%%%%%%%%%%%%%%%%%%%%%%%%%%%%%%%%%%%%%%%%%%%%%%%%%%%%%%%%%%%%%%%%%%%
\begin{frame}{Example Orthogonal projection}
\begin{itemize}
\item \textbf{Example} In $\R^3$ let $W$ be the $x$- $y$-plane. Let $\bv=
\begin{pmatrix}1\\2\\3 \end{pmatrix}$.
\item Then $\bv=
\begin{pmatrix}
1\\2\\0
\end{pmatrix}
+
\begin{pmatrix}
0\\0\\3
\end{pmatrix}
$.
\item $\begin{pmatrix}
1\\2\\0
\end{pmatrix}$
is the orthogonal project of $\bv$ onto the $x$- $y$-plane.
\end{itemize}

\end{frame}

%%%%%%%%%%%%%%%%%%%%%%%%%%%%%%%%%%%%%%%%%%%%%%%%%%%%%%%%%%%%%%%%%%%%%%%%%
\begin{frame}{Proof of Orthogonal projection theorem}
\begin{itemize}
\item \textbf{proof} Let $V$ be a finite dimensional inner product space.
\item Let $W$ be a subspace of $V$. Let $\bv\in V$.
\item Let $\bw_1,\bw_2,\cdots,\bw_n$ be an orthonormal basis for $W$.
\item Let $\bv_1=\angles{\bv,\bw_1}\bw_1+\angles{\bv,\bw_2}\bw_2+\cdots+\angles{\bv,\bw_n}\bw_n.$
\item Let $\bv_2 = \bv - \bv_1$.
\item Then $\bv=\bv_1+\bv_2$ and $\bv_1\in W$. We need to see that $\bv_2\in W^{\perp}$.
\item First, let $\bw_i$ be one of the basis elements and compute $\angles{\bv_1, \bw_i}$.
\item $\angles{\bv_1, \bw_i} = \angles{\bv,\bw_1}\angles{\bw_1,\bw_i} + \angles{\bv,\bw_2}\angles{\bw_2,\bw_i} +
\cdots + \angles{\bv,\bw_i}\angles{\bw_i,\bw_i} + \cdots + \angles{\bv,\bw_n}\angles{\bw_n,\bw_i} $
\item $=\angles{\bv,\bw_i}\angles{\bw_i,\bw_i} = \angles{\bv,\bw_i}$.
\item Now, to see that $\bv_2\in W^{\perp}$, take the inner product of $\bv_2$ with any of the $\bw_i$.
\item $\angles{\bv_2, \bw_i} = \angles{\bv, \bw_i} - \angles{\bv_1, \bw_i}$
\item $=\angles{\bv, \bw_i} - \angles{\bv,\bw_i} = 0$.
\end{itemize}

\end{frame}

%%%%%%%%%%%%%%%%%%%%%%%%%%%%%%%%%%%%%%%%%%%%%%%%%%%%%%%%%%%%%%%%%%%%%%%%%
\begin{frame}{Proof continued}
\begin{itemize}
\item So $\bv_2\in W^{\perp}$.
\item So we have written $\bv=\bv_1+\bv_2$ with $\bv_1\in W$ and $\bv_2\in W^{\perp}$.
\item To complete the proof we have to show that the decomposition is unique.
\item Suppose that we could also write $\bv=\bv^{\prime}_1+\bv^{\prime}_2$ with $\bv^{\prime}_1\in W$ and $\bv^{\prime}_2\in W^{\perp}$.
\item $\bv_1+\bv_2=\bv^{\prime}_1+\bv^{\prime}_2 \SkipImplies \bv_1 - \bv^{\prime}_1 = \bv^{\prime}_2 - \bv_2$.
\item The left-hand-side is in $W$ and the right-hand-side is in $W^{\perp}$.
\item But $W\intersect W^{\perp} = \singleton{\bzero}$.
\item So $\bv_1=\bv_1^{\prime}$ and $\bv_2=\bv_2^{\prime}$. $\qed$.
\end{itemize}

\end{frame}

%%%%%%%%%%%%%%%%%%%%%%%%%%%%%%%%%%%%%%%%%%%%%%%%%%%%%%%%%%%%%%%%%%%%%%%%%
\begin{frame}{$V=W+W^{\perp}$}
\begin{itemize}
\item \textbf{Theorem} Let $V$ be a finite dimensional inner product space.
\item Let $W$ be a subspace of $V$.
\item Then $V= W + W^{\perp}$. This means
\item $\spn\left(W\union W^{\perp}\right) = V$, and
\item $W\intersect W^{\perp} = \singleton{\bzero}$.
\item In particular $\dim(W) + \dim(W^{\perp}) = \dim(V)$.
\item \textbf{Proof.} This is equivalent to the previous theorem.
\item Example. In $\R^3$ the $z$-axis and the $x$- $y$-plane are orthogonal compliments.
\item The dimension of the $z$-axis is 1 and the dimension of the $x$- $y$-plane is 2
\item and 1 + 2 = 3.
\end{itemize}
\end{frame}

%%%%%%%%%%%%%%%%%%%%%%%%%%%%%%%%%%%%%%%%%%%%%%%%%%%%%%%%%%%%%%%%%%%%%%%%%
\begin{frame}{Computing Orthogonal Compliments}
\begin{itemize}
\item \textbf{Proposition.} Let $V$ be a finite dimensional inner product space.
\item Let $W$ be a subspace of $V$. Let $\bv\in V$.
\item Let $\bw_1,\bw_2,\cdots,\bw_n$ be an \emph{orthogonal} basis for $W$.
\item Then the orthogonal projection of $\bv$ onto $W$ is given by
\item $\frac{\angles{\bv,\bw_1}}{\norm{\bw_1}^2}\bv_1 + \frac{\angles{\bv,\bw_2}}{\norm{\bw_2}^2}\bv_2+\cdots+\frac{\angles{\bv,\bw_n}}{\norm{\bw_n}^2}\bv_n.$
\end{itemize}
\end{frame}

%%%%%%%%%%%%%%%%%%%%%%%%%%%%%%%%%%%%%%%%%%%%%%%%%%%%%%%%%%%%%%%%%%%%%%%%%
\begin{frame}{Example}
\begin{itemize}
\item Let $W$ be the subspace of $\R^4$ spanned by the following three vectors:
\item $
\bw_1=
\begin{pmatrix}
0  \\
1  \\
1\\
0
\end{pmatrix}
\text{, }
\bw_2=
\begin{pmatrix}
1 \\
0\\
0 \\
-1
\end{pmatrix}
\text{, }
\bw_3=
\begin{pmatrix}
1 \\
1 \\
-1 \\
1
\end{pmatrix}
.$
\item Let $\bv=
\begin{pmatrix}
1 \\
1 \\
2 \\
2
\end{pmatrix}
$.
\item Find the orthogonal projection of $\bv$ onto $W$.
\item Notice that $\bw_1,\bw_2,\bw_3$ is an orthogonal basis for $W$.
\end{itemize}
\end{frame}

%%%%%%%%%%%%%%%%%%%%%%%%%%%%%%%%%%%%%%%%%%%%%%%%%%%%%%%%%%%%%%%%%%%%%%%%%
\begin{frame}{Example Continued}
\begin{itemize}
\item $
\bw_1=
\begin{pmatrix}
0  \\
1  \\
1\\
0
\end{pmatrix}
\text{, }
\bw_2=
\begin{pmatrix}
1 \\
0\\
0 \\
-1
\end{pmatrix}
\text{, }
\bw_3=
\begin{pmatrix}
1 \\
1 \\
-1 \\
1
\end{pmatrix}
\text{, }
\bv=
\begin{pmatrix}
1 \\
1 \\
2 \\
2
\end{pmatrix}
$
\item By the proposition, the orthogonal projection is given by
\item $\frac{\angles{\bv,\bw_1}}{\norm{\bw_1}^2}\bw_1 + \frac{\angles{\bv,\bw_2}}{\norm{\bw_2}^2}\bw_2 + \frac{\angles{\bv,\bw_3}}{\norm{\bw_3}^2}\bw_3$.
\item $\angles{\bv,\bw_1}=3,\angles{\bv,\bw_2}=-1,\angles{\bv,\bw_3}=2$.
\item $\norm{\bw_1}^2=2,\norm{\bw_2}^2=2,\norm{\bw_3}^2=4$.
\item So the orthogonal project of $\bv$ onto $W$ is:
\item $\frac{3}{2}\bw_1 - \frac{1}{2}\bw_2 + \frac{1}{2}\bw_3$.
\item  $ =
\frac{3}{2}
\begin{pmatrix}
0  \\
1  \\
1\\
0
\end{pmatrix}
- \frac{1}{2}
\begin{pmatrix}
1 \\
0\\
0 \\
-1
\end{pmatrix}
+ \frac{1}{2}\
\begin{pmatrix}
1 \\
1 \\
-1 \\
1
\end{pmatrix}
=
\begin{pmatrix}
0 \\
2 \\
1 \\
1
\end{pmatrix}
$
\end{itemize}
\end{frame}

%%%%%%%%%%%%%%%%%%%%%%%%%%%%%%%%%%%%%%%%%%%%%%%%%%%%%%%%%%%%%%%%%%%%%%%%%
\begin{frame}{Example Continued}
\begin{itemize}
\item $
\bw_1=
\begin{pmatrix}
0  \\
1  \\
1\\
0
\end{pmatrix}
\text{, }
\bw_2=
\begin{pmatrix}
1 \\
0\\
0 \\
-1
\end{pmatrix}
\text{, }
\bw_3=
\begin{pmatrix}
1 \\
1 \\
-1 \\
1
\end{pmatrix}
\text{, }
\bv=
\begin{pmatrix}
1 \\
1 \\
2 \\
2
\end{pmatrix}
$
\item The orthogonal project of $\bv$ onto $W$ is $\bv_1=
\begin{pmatrix}
0 \\
2 \\
1 \\
1
\end{pmatrix}
$
\item Check: Let $\bv_2 = \bv - \bv_1$. $\bv_2$ should be in $W^{\perp}$.
\item $\bv_2=
\begin{pmatrix}
1 \\
1 \\
2 \\
2
\end{pmatrix}
-
\begin{pmatrix}
0 \\
2 \\
1 \\
1
\end{pmatrix}
=
\begin{pmatrix}
1 \\
-1 \\
1 \\
1
\end{pmatrix}
$
\item $\angles{\bv_2,\bw_1}=\angles{\bv_2,\bw_2}=\angles{\bv_2,\bw_3} = 0$.
\item So $\bv_2\in W^{\perp}$.
\end{itemize}
\end{frame}

%%%%%%%%%%%%%%%%%%%%%%%%%%%%%%%%%%%%%%%%%%%%%%%%%%%%%%%%%%%%%%%%%%%%%%%%%

\begin{frame}{The Row Space of a Matrix}
\begin{itemize}
\item \textbf{Definition} Let $A$ be an $m\times n$ matrix.
\item Then the \emph{row space} of $A$ is the column space of $\transpose{A}$.
\item Equivalently the image of $\transpose{A}$.
\item Example: $A=
\begin{pmatrix}
1 & 2 & 3 \\
4 & 5 & 6
\end{pmatrix}
$
\item The row space of $A$ is the 2-dimensional subspace of $\R^3$ spanned by
\item $
\singleton{
\begin{pmatrix}
1 \\ 2 \\ 3
\end{pmatrix},
\begin{pmatrix}
4 \\ 5 \\ 6
\end{pmatrix}
}
$
\end{itemize}
\end{frame}


%%%%%%%%%%%%%%%%%%%%%%%%%%%%%%%%%%%%%%%%%%%%%%%%%%%%%%%%%%%%%%%%%%%%%%%%%
\begin{frame}{The Orthogonal Compliment of the Kernel}
\begin{itemize}
\item \textbf{Theorem} Let $A$ be any matrix.
\item Then the row space of $A$ and the kernel of $A$ are orthogonal compliments.
\item \textbf{proof} Let $W$ be the row space of $A$.
\item $\bx\in\ker(A)$ iff $\transpose{\bw} \bx = 0$ for every row $\transpose{\bw}$ of $A$,
\item iff $\angles{\bw,\bx} = 0$ every row $\transpose{\bw}$ of $A$,
\item iff $\bx\in W^{\perp}$. $\qed$.
\end{itemize}
\end{frame}

%%%%%%%%%%%%%%%%%%%%%%%%%%%%%%%%%%%%%%%%%%%%%%%%%%%%%%%%%%%%%%%%%%%%%%%%%
\begin{frame}{The Row Rank of a Matrix}
\begin{itemize}
\item \textbf{Definition} Let $A$ be an $m\times n$ matrix.
\item The \emph{row rank} of $A$ is the dimension of the row space of $A$.
\item The normal rank of $A$ is sometimes called the \emph{column rank}.
\item \textbf{Theorem.} The row-rank of $A$ is equal to the column rank of $A$.
\item Let $r$ be the column rank of $A$.
\item Previously we learned that $\dim(\ker(A)) = n  - r$.
\item So $\dim(\ker(A)) + r = n$.
\item But the column space of $A$ is the orthogonal compliment of the kernel.
\item So its also true that $\dim(\ker(A)) + \dim(\text{Column Space of $A$}) = n$.
\item So $\dim(\text{Column Space of $A$}) = r$. $\qed$
\end{itemize}
\end{frame}

%%%%%%%%%%%%%%%%%%%%%%%%%%%%%%%%%%%%%%%%%%%%%%%%%%%%%%%%%%%%%%%%%%%%%%%%%
\begin{frame}{Example}
\begin{itemize}
\item Let $V$ be the subspace of $\R^3$ spanned by the following three vectors:
\item $\bv_1=
\begin{pmatrix}
1  \\
0  \\
3   \\
\end{pmatrix}
\text{, }
\bv_2=
\begin{pmatrix}
-1 \\
3\\
3 \\
\end{pmatrix}
\text{, }
\bv_3=
\begin{pmatrix}
5 \\
-9 \\
-3 \\
\end{pmatrix}
$
\item Find a basis for $V^{\perp}$. What is the dimension of $V^{\perp}$?
\item Solution: Let $A$ be the matrix with rows $\transpose{\bv_1}, \transpose{\bv_2}, \transpose{\bv_3}$.
\item $A=
\begin{pmatrix}
1 & 0 & 3   \\
-1 & 3 & 3  \\
5 & -9 & -3
\end{pmatrix}
$
\item Then we are looking for the kernel of $A$. So solve the homogeneous system of equations.
\end{itemize}
\end{frame}

%%%%%%%%%%%%%%%%%%%%%%%%%%%%%%%%%%%%%%%%%%%%%%%%%%%%%%%%%%%%%%%%%%%%%%%%%
\begin{frame}{Example Continued}
\begin{itemize}
\item So solve the homogeneous system of equations.
\item $
\begin{pmatrix}
1 & 0 & 3   \\
-1 & 3 & 3  \\
5 & -9 & -3
\end{pmatrix}
\SkipImplies
\begin{pmatrix}
1 & 0 & 3    \\
0 & 3 & 6    \\
0 & -9 & -18
\end{pmatrix}
\SkipImplies
\begin{pmatrix}
1 & 0 & 3    \\
0 & 3 & 6    \\
0 & 0 &
\end{pmatrix}
$
\item Let's call the variables $x,y,z$. So $x$ and $y$ are basic and $z$ is free.
\item $3y+6z=0\SkipImplies y = -2z$.
\item $x +3z =0 \SkipImplies x=-3z$.
\item So the general solution is $
\begin{pmatrix}
x\\y\\z
\end{pmatrix}
=
\begin{pmatrix}
-3z\\-2z\\z
\end{pmatrix}
=
z
\begin{pmatrix}
-3\\-2\\1
\end{pmatrix}
$
\item So $
\begin{pmatrix}
-3\\-2\\1
\end{pmatrix}
$
is a basis for $V^{\perp}$ and $V^{\perp}$ is 1-dimensional.
\end{itemize}
\end{frame}

%%%%%%%%%%%%%%%%%%%%%%%%%%%%%%%%%%%%%%%%%%%%%%%%%%%%%%%%%%%%%%%%%%%%%%%%%
\begin{frame}{Example}
\begin{itemize}
\item Find a basis for the orthogonal compliment of the
kernel of the following matrix. What is the dimension of the orthogonal
compliment of the kernel?
\item $\begin{pmatrix}
1 & 2 & 3 & 4 \\
1 & 0 & 0 & 1 \\
2 & 2 & 3 & 5 \\
\end{pmatrix}
$
\item Solution. The orthogonal compliment of the kernel is the row space.
\item So we are asked to find a basis and dimension for the row space.
\item i.e. a basis and dimension for the column space of the transpose.
\item $\begin{pmatrix}
1 & 1 & 2 \\
1 & 0 & 1 \\
3 & 0 & 3 \\
4 & 1 & 5 \\
\end{pmatrix}
$
\end{itemize}
\end{frame}

%%%%%%%%%%%%%%%%%%%%%%%%%%%%%%%%%%%%%%%%%%%%%%%%%%%%%%%%%%%%%%%%%%%%%%%%%
\begin{frame}{Example continued}
\begin{itemize}
\item Find a basis and dimension for the column space of
\item $\begin{pmatrix}
1 & 1 & 2 \\
1 & 0 & 1 \\
3 & 0 & 3 \\
4 & 1 & 5 \\
\end{pmatrix}
$
\item We don't even need to do Gaussian elimination. We can see the solution by inspection.
\item The first two columns are linearly independent. And the final column is the sum of the first two.
\item So the rank is 2 and a basis is given by the first two columns.
\end{itemize}
\end{frame}

%%%%%%%%%%%%%%%%%%%%%%%%%%%%%%%%%%%%%%%%%%%%%%%%%%%%%%%%%%%%%%%%%%%%%%%%%



\end{document}


