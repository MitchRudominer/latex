\documentclass[oneside,12pt]{amsart}

\usepackage{amsmath,amssymb,latexsym,eucal,amsthm,graphicx}

%%%%%%%%%%%%%%%%%%%%%%%%%%%%%%%%%%%%%%%%%%%%%
% Common Set Theory Constructs
%%%%%%%%%%%%%%%%%%%%%%%%%%%%%%%%%%%%%%%%%%%%%

\newcommand{\setof}[2]{\left\{ \, #1 \, \left| \, #2 \, \right.\right\}}
\newcommand{\lsetof}[2]{\left\{\left. \, #1 \, \right| \, #2 \,  \right\}}
\newcommand{\bigsetof}[2]{\bigl\{ \, #1 \, \bigm | \, #2 \,\bigr\}}
\newcommand{\Bigsetof}[2]{\Bigl\{ \, #1 \, \Bigm | \, #2 \,\Bigr\}}
\newcommand{\biggsetof}[2]{\biggl\{ \, #1 \, \biggm | \, #2 \,\biggr\}}
\newcommand{\Biggsetof}[2]{\Biggl\{ \, #1 \, \Biggm | \, #2 \,\Biggr\}}
\newcommand{\dotsetof}[2]{\left\{ \, #1 \, : \, #2 \, \right\}}
\newcommand{\bigdotsetof}[2]{\bigl\{ \, #1 \, : \, #2 \,\bigr\}}
\newcommand{\Bigdotsetof}[2]{\Bigl\{ \, #1 \, \Bigm : \, #2 \,\Bigr\}}
\newcommand{\biggdotsetof}[2]{\biggl\{ \, #1 \, \biggm : \, #2 \,\biggr\}}
\newcommand{\Biggdotsetof}[2]{\Biggl\{ \, #1 \, \Biggm : \, #2 \,\Biggr\}}
\newcommand{\sequence}[2]{\left\langle \, #1 \,\left| \, #2 \, \right. \right\rangle}
\newcommand{\lsequence}[2]{\left\langle\left. \, #1 \, \right| \,#2 \,  \right\rangle}
\newcommand{\bigsequence}[2]{\bigl\langle \,#1 \, \bigm | \, #2 \, \bigr\rangle}
\newcommand{\Bigsequence}[2]{\Bigl\langle \,#1 \, \Bigm | \, #2 \, \Bigr\rangle}
\newcommand{\biggsequence}[2]{\biggl\langle \,#1 \, \biggm | \, #2 \, \biggr\rangle}
\newcommand{\Biggsequence}[2]{\Biggl\langle \,#1 \, \Biggm | \, #2 \, \Biggr\rangle}
\newcommand{\singleton}[1]{\left\{#1\right\}}
\newcommand{\angles}[1]{\left\langle #1 \right\rangle}
\newcommand{\bigangles}[1]{\bigl\langle #1 \bigr\rangle}
\newcommand{\Bigangles}[1]{\Bigl\langle #1 \Bigr\rangle}
\newcommand{\biggangles}[1]{\biggl\langle #1 \biggr\rangle}
\newcommand{\Biggangles}[1]{\Biggl\langle #1 \Biggr\rangle}


\newcommand{\force}[1]{\Vert\!\underset{\!\!\!\!\!#1}{\!\!\!\relbar\!\!\!%
\relbar\!\!\relbar\!\!\relbar\!\!\!\relbar\!\!\relbar\!\!\relbar\!\!\!%
\relbar\!\!\relbar\!\!\relbar}}
\newcommand{\longforce}[1]{\Vert\!\underset{\!\!\!\!\!#1}{\!\!\!\relbar\!\!\!%
\relbar\!\!\relbar\!\!\relbar\!\!\!\relbar\!\!\relbar\!\!\relbar\!\!\!%
\relbar\!\!\relbar\!\!\relbar\!\!\relbar\!\!\relbar\!\!\relbar\!\!\relbar\!\!\relbar}}
\newcommand{\nforce}[1]{\Vert\!\underset{\!\!\!\!\!#1}{\!\!\!\relbar\!\!\!%
\relbar\!\!\relbar\!\!\relbar\!\!\!\relbar\!\!\relbar\!\!\relbar\!\!\!%
\relbar\!\!\not\relbar\!\!\relbar}}
\newcommand{\forcein}[2]{\overset{#2}{\Vert\underset{\!\!\!\!\!#1}%
{\!\!\!\relbar\!\!\!\relbar\!\!\relbar\!\!\relbar\!\!\!\relbar\!\!\relbar\!%
\!\relbar\!\!\!\relbar\!\!\relbar\!\!\relbar\!\!\relbar\!\!\!\relbar\!\!%
\relbar\!\!\relbar}}}

\newcommand{\pre}[2]{{}^{#2}{#1}}

\newcommand{\restr}{\!\!\upharpoonright\!}

%%%%%%%%%%%%%%%%%%%%%%%%%%%%%%%%%%%%%%%%%%%%%
% Set-Theoretic Connectives
%%%%%%%%%%%%%%%%%%%%%%%%%%%%%%%%%%%%%%%%%%%%%

\newcommand{\intersect}{\cap}
\newcommand{\union}{\cup}
\newcommand{\Intersection}[1]{\bigcap\limits_{#1}}
\newcommand{\Union}[1]{\bigcup\limits_{#1}}
\newcommand{\adjoin}{{}^\frown}
\newcommand{\forces}{\Vdash}

%%%%%%%%%%%%%%%%%%%%%%%%%%%%%%%%%%%%%%%%%%%%%
% Miscellaneous
%%%%%%%%%%%%%%%%%%%%%%%%%%%%%%%%%%%%%%%%%%%%%
\newcommand{\defeq}{=_{\text{def}}}
\newcommand{\Turingleq}{\leq_{\text{T}}}
\newcommand{\Turingless}{<_{\text{T}}}
\newcommand{\lexleq}{\leq_{\text{lex}}}
\newcommand{\lexless}{<_{\text{lex}}}
\newcommand{\Turingequiv}{\equiv_{\text{T}}}
\newcommand{\isomorphic}{\cong}

%%%%%%%%%%%%%%%%%%%%%%%%%%%%%%%%%%%%%%%%%%%%%
% Constants
%%%%%%%%%%%%%%%%%%%%%%%%%%%%%%%%%%%%%%%%%%%%%
\newcommand{\R}{\mathbb{R}}
\renewcommand{\P}{\mathbb{P}}
\newcommand{\Q}{\mathbb{Q}}
\newcommand{\Z}{\mathbb{Z}}
\newcommand{\Zpos}{\Z^{+}}
\newcommand{\Znonneg}{\Z^{\geq 0}}
\newcommand{\C}{\mathbb{C}}
\newcommand{\N}{\mathbb{N}}
\newcommand{\B}{\mathbb{B}}
\newcommand{\Bairespace}{\pre{\omega}{\omega}}
\newcommand{\LofR}{L(\R)}
\newcommand{\JofR}[1]{J_{#1}(\R)}
\newcommand{\SofR}[1]{S_{#1}(\R)}
\newcommand{\JalphaR}{\JofR{\alpha}}
\newcommand{\JbetaR}{\JofR{\beta}}
\newcommand{\JlambdaR}{\JofR{\lambda}}
\newcommand{\SalphaR}{\SofR{\alpha}}
\newcommand{\SbetaR}{\SofR{\beta}}
\newcommand{\Pkl}{\mathcal{P}_{\kappa}(\lambda)}
\DeclareMathOperator{\con}{con}
\DeclareMathOperator{\ORD}{OR}
\DeclareMathOperator{\Ord}{OR}
\DeclareMathOperator{\WO}{WO}
\DeclareMathOperator{\OD}{OD}
\DeclareMathOperator{\HOD}{HOD}
\DeclareMathOperator{\HC}{HC}
\DeclareMathOperator{\WF}{WF}
\DeclareMathOperator{\wfp}{wfp}
\DeclareMathOperator{\HF}{HF}
\newcommand{\One}{I}
\newcommand{\ONE}{I}
\newcommand{\Two}{II}
\newcommand{\TWO}{II}
\newcommand{\Mladder}{M^{\text{ld}}}

%%%%%%%%%%%%%%%%%%%%%%%%%%%%%%%%%%%%%%%%%%%%%
% Commutative Algebra Constants
%%%%%%%%%%%%%%%%%%%%%%%%%%%%%%%%%%%%%%%%%%%%%
\DeclareMathOperator{\dottimes}{\dot{\times}}
\DeclareMathOperator{\dotminus}{\dot{-}}
\DeclareMathOperator{\Spec}{Spec}

%%%%%%%%%%%%%%%%%%%%%%%%%%%%%%%%%%%%%%%%%%%%%
% Theories
%%%%%%%%%%%%%%%%%%%%%%%%%%%%%%%%%%%%%%%%%%%%%
\DeclareMathOperator{\ZFC}{ZFC}
\DeclareMathOperator{\ZF}{ZF}
\DeclareMathOperator{\AD}{AD}
\DeclareMathOperator{\ADR}{AD_{\R}}
\DeclareMathOperator{\KP}{KP}
\DeclareMathOperator{\PD}{PD}
\DeclareMathOperator{\CH}{CH}
\DeclareMathOperator{\GCH}{GCH}
\DeclareMathOperator{\HPC}{HPC} % HOD pair capturing
%%%%%%%%%%%%%%%%%%%%%%%%%%%%%%%%%%%%%%%%%%%%%
% Iteration Trees
%%%%%%%%%%%%%%%%%%%%%%%%%%%%%%%%%%%%%%%%%%%%%

\newcommand{\pred}{\text{-pred}}

%%%%%%%%%%%%%%%%%%%%%%%%%%%%%%%%%%%%%%%%%%%%%%%%
% Operator Names
%%%%%%%%%%%%%%%%%%%%%%%%%%%%%%%%%%%%%%%%%%%%%%%%
\DeclareMathOperator{\Det}{Det}
\DeclareMathOperator{\dom}{dom}
\DeclareMathOperator{\ran}{ran}
\DeclareMathOperator{\range}{ran}
\DeclareMathOperator{\image}{image}
\DeclareMathOperator{\crit}{crit}
\DeclareMathOperator{\card}{card}
\DeclareMathOperator{\cf}{cf}
\DeclareMathOperator{\cof}{cof}
\DeclareMathOperator{\rank}{rank}
\DeclareMathOperator{\ot}{o.t.}
\DeclareMathOperator{\ords}{o}
\DeclareMathOperator{\ro}{r.o.}
\DeclareMathOperator{\rud}{rud}
\DeclareMathOperator{\Powerset}{\mathcal{P}}
\DeclareMathOperator{\length}{lh}
\DeclareMathOperator{\lh}{lh}
\DeclareMathOperator{\limit}{lim}
\DeclareMathOperator{\fld}{fld}
\DeclareMathOperator{\projection}{p}
\DeclareMathOperator{\Ult}{Ult}
\DeclareMathOperator{\ULT}{Ult}
\DeclareMathOperator{\Coll}{Coll}
\DeclareMathOperator{\Col}{Col}
\DeclareMathOperator{\Hull}{Hull}
\DeclareMathOperator*{\dirlim}{dir lim}
\DeclareMathOperator{\Scale}{Scale}
\DeclareMathOperator{\supp}{supp}
\DeclareMathOperator{\trancl}{tran.cl.}
\DeclareMathOperator{\trace}{Tr}
\DeclareMathOperator{\diag}{diag}
\DeclareMathOperator{\spn}{span}
\DeclareMathOperator{\sgn}{sgn}
%%%%%%%%%%%%%%%%%%%%%%%%%%%%%%%%%%%%%%%%%%%%%
% Logical Connectives
%%%%%%%%%%%%%%%%%%%%%%%%%%%%%%%%%%%%%%%%%%%%%
\newcommand{\IImplies}{\Longrightarrow}
\newcommand{\SkipImplies}{\quad\Longrightarrow\quad}
\newcommand{\Ifff}{\Longleftrightarrow}
\newcommand{\iimplies}{\longrightarrow}
\newcommand{\ifff}{\longleftrightarrow}
\newcommand{\Implies}{\Rightarrow}
\newcommand{\Iff}{\Leftrightarrow}
\renewcommand{\implies}{\rightarrow}
\renewcommand{\iff}{\leftrightarrow}
\newcommand{\AND}{\wedge}
\newcommand{\OR}{\vee}
\newcommand{\st}{\text{ s.t. }}
\newcommand{\Or}{\text{ or }}

%%%%%%%%%%%%%%%%%%%%%%%%%%%%%%%%%%%%%%%%%%%%%
% Function Arrows
%%%%%%%%%%%%%%%%%%%%%%%%%%%%%%%%%%%%%%%%%%%%%

\newcommand{\injection}{\xrightarrow{\text{1-1}}}
\newcommand{\surjection}{\xrightarrow{\text{onto}}}
\newcommand{\bijection}{\xrightarrow[\text{onto}]{\text{1-1}}}
\newcommand{\cofmap}{\xrightarrow{\text{cof}}}
\newcommand{\map}{\rightarrow}

%%%%%%%%%%%%%%%%%%%%%%%%%%%%%%%%%%%%%%%%%%%%%
% Mouse Comparison Operators
%%%%%%%%%%%%%%%%%%%%%%%%%%%%%%%%%%%%%%%%%%%%%
\newcommand{\initseg}{\trianglelefteq}
\newcommand{\properseg}{\lhd}
\newcommand{\notinitseg}{\ntrianglelefteq}
\newcommand{\notproperseg}{\ntriangleleft}

%%%%%%%%%%%%%%%%%%%%%%%%%%%%%%%%%%%%%%%%%%%%%
%           calligraphic letters
%%%%%%%%%%%%%%%%%%%%%%%%%%%%%%%%%%%%%%%%%%%%%
\newcommand{\cA}{\mathcal{A}}
\newcommand{\cB}{\mathcal{B}}
\newcommand{\cC}{\mathcal{C}}
\newcommand{\cD}{\mathcal{D}}
\newcommand{\cE}{\mathcal{E}}
\newcommand{\cF}{\mathcal{F}}
\newcommand{\cG}{\mathcal{G}}
\newcommand{\cH}{\mathcal{H}}
\newcommand{\cI}{\mathcal{I}}
\newcommand{\cJ}{\mathcal{J}}
\newcommand{\cK}{\mathcal{K}}
\newcommand{\cL}{\mathcal{L}}
\newcommand{\cM}{\mathcal{M}}
\newcommand{\cN}{\mathcal{N}}
\newcommand{\cO}{\mathcal{O}}
\newcommand{\cP}{\mathcal{P}}
\newcommand{\cQ}{\mathcal{Q}}
\newcommand{\cR}{\mathcal{R}}
\newcommand{\cS}{\mathcal{S}}
\newcommand{\cT}{\mathcal{T}}
\newcommand{\cU}{\mathcal{U}}
\newcommand{\cV}{\mathcal{V}}
\newcommand{\cW}{\mathcal{W}}
\newcommand{\cX}{\mathcal{X}}
\newcommand{\cY}{\mathcal{Y}}
\newcommand{\cZ}{\mathcal{Z}}


%%%%%%%%%%%%%%%%%%%%%%%%%%%%%%%%%%%%%%%%%%%%%
%          Primed Letters
%%%%%%%%%%%%%%%%%%%%%%%%%%%%%%%%%%%%%%%%%%%%%
\newcommand{\aprime}{a^{\prime}}
\newcommand{\bprime}{b^{\prime}}
\newcommand{\cprime}{c^{\prime}}
\newcommand{\dprime}{d^{\prime}}
\newcommand{\eprime}{e^{\prime}}
\newcommand{\fprime}{f^{\prime}}
\newcommand{\gprime}{g^{\prime}}
\newcommand{\hprime}{h^{\prime}}
\newcommand{\iprime}{i^{\prime}}
\newcommand{\jprime}{j^{\prime}}
\newcommand{\kprime}{k^{\prime}}
\newcommand{\lprime}{l^{\prime}}
\newcommand{\mprime}{m^{\prime}}
\newcommand{\nprime}{n^{\prime}}
\newcommand{\oprime}{o^{\prime}}
\newcommand{\pprime}{p^{\prime}}
\newcommand{\qprime}{q^{\prime}}
\newcommand{\rprime}{r^{\prime}}
\newcommand{\sprime}{s^{\prime}}
\newcommand{\tprime}{t^{\prime}}
\newcommand{\uprime}{u^{\prime}}
\newcommand{\vprime}{v^{\prime}}
\newcommand{\wprime}{w^{\prime}}
\newcommand{\xprime}{x^{\prime}}
\newcommand{\yprime}{y^{\prime}}
\newcommand{\zprime}{z^{\prime}}
\newcommand{\Aprime}{A^{\prime}}
\newcommand{\Bprime}{B^{\prime}}
\newcommand{\Cprime}{C^{\prime}}
\newcommand{\Dprime}{D^{\prime}}
\newcommand{\Eprime}{E^{\prime}}
\newcommand{\Fprime}{F^{\prime}}
\newcommand{\Gprime}{G^{\prime}}
\newcommand{\Hprime}{H^{\prime}}
\newcommand{\Iprime}{I^{\prime}}
\newcommand{\Jprime}{J^{\prime}}
\newcommand{\Kprime}{K^{\prime}}
\newcommand{\Lprime}{L^{\prime}}
\newcommand{\Mprime}{M^{\prime}}
\newcommand{\Nprime}{N^{\prime}}
\newcommand{\Oprime}{O^{\prime}}
\newcommand{\Pprime}{P^{\prime}}
\newcommand{\Qprime}{Q^{\prime}}
\newcommand{\Rprime}{R^{\prime}}
\newcommand{\Sprime}{S^{\prime}}
\newcommand{\Tprime}{T^{\prime}}
\newcommand{\Uprime}{U^{\prime}}
\newcommand{\Vprime}{V^{\prime}}
\newcommand{\Wprime}{W^{\prime}}
\newcommand{\Xprime}{X^{\prime}}
\newcommand{\Yprime}{Y^{\prime}}
\newcommand{\Zprime}{Z^{\prime}}
\newcommand{\alphaprime}{\alpha^{\prime}}
\newcommand{\betaprime}{\beta^{\prime}}
\newcommand{\gammaprime}{\gamma^{\prime}}
\newcommand{\Gammaprime}{\Gamma^{\prime}}
\newcommand{\deltaprime}{\delta^{\prime}}
\newcommand{\epsilonprime}{\epsilon^{\prime}}
\newcommand{\kappaprime}{\kappa^{\prime}}
\newcommand{\lambdaprime}{\lambda^{\prime}}
\newcommand{\rhoprime}{\rho^{\prime}}
\newcommand{\Sigmaprime}{\Sigma^{\prime}}
\newcommand{\tauprime}{\tau^{\prime}}
\newcommand{\xiprime}{\xi^{\prime}}
\newcommand{\thetaprime}{\theta^{\prime}}
\newcommand{\Omegaprime}{\Omega^{\prime}}
\newcommand{\cMprime}{\cM^{\prime}}
\newcommand{\cNprime}{\cN^{\prime}}
\newcommand{\cPprime}{\cP^{\prime}}
\newcommand{\cQprime}{\cQ^{\prime}}
\newcommand{\cRprime}{\cR^{\prime}}
\newcommand{\cSprime}{\cS^{\prime}}
\newcommand{\cTprime}{\cT^{\prime}}

%%%%%%%%%%%%%%%%%%%%%%%%%%%%%%%%%%%%%%%%%%%%%
%          bar Letters
%%%%%%%%%%%%%%%%%%%%%%%%%%%%%%%%%%%%%%%%%%%%%
\newcommand{\abar}{\bar{a}}
\newcommand{\bbar}{\bar{b}}
\newcommand{\cbar}{\bar{c}}
\newcommand{\ibar}{\bar{i}}
\newcommand{\jbar}{\bar{j}}
\newcommand{\nbar}{\bar{n}}
\newcommand{\xbar}{\bar{x}}
\newcommand{\ybar}{\bar{y}}
\newcommand{\zbar}{\bar{z}}
\newcommand{\pibar}{\bar{\pi}}
\newcommand{\phibar}{\bar{\varphi}}
\newcommand{\psibar}{\bar{\psi}}
\newcommand{\thetabar}{\bar{\theta}}
\newcommand{\nubar}{\bar{\nu}}

%%%%%%%%%%%%%%%%%%%%%%%%%%%%%%%%%%%%%%%%%%%%%
%          star Letters
%%%%%%%%%%%%%%%%%%%%%%%%%%%%%%%%%%%%%%%%%%%%%
\newcommand{\phistar}{\phi^{*}}
\newcommand{\Mstar}{M^{*}}

%%%%%%%%%%%%%%%%%%%%%%%%%%%%%%%%%%%%%%%%%%%%%
%          dotletters Letters
%%%%%%%%%%%%%%%%%%%%%%%%%%%%%%%%%%%%%%%%%%%%%
\newcommand{\Gdot}{\dot{G}}

%%%%%%%%%%%%%%%%%%%%%%%%%%%%%%%%%%%%%%%%%%%%%
%         check Letters
%%%%%%%%%%%%%%%%%%%%%%%%%%%%%%%%%%%%%%%%%%%%%
\newcommand{\deltacheck}{\check{\delta}}
\newcommand{\gammacheck}{\check{\gamma}}


%%%%%%%%%%%%%%%%%%%%%%%%%%%%%%%%%%%%%%%%%%%%%
%          Formulas
%%%%%%%%%%%%%%%%%%%%%%%%%%%%%%%%%%%%%%%%%%%%%

\newcommand{\formulaphi}{\text{\large $\varphi$}}
\newcommand{\Formulaphi}{\text{\Large $\varphi$}}


%%%%%%%%%%%%%%%%%%%%%%%%%%%%%%%%%%%%%%%%%%%%%
%          Fraktur Letters
%%%%%%%%%%%%%%%%%%%%%%%%%%%%%%%%%%%%%%%%%%%%%

\newcommand{\fa}{\mathfrak{a}}
\newcommand{\fb}{\mathfrak{b}}
\newcommand{\fc}{\mathfrak{c}}
\newcommand{\fk}{\mathfrak{k}}
\newcommand{\fp}{\mathfrak{p}}
\newcommand{\fq}{\mathfrak{q}}
\newcommand{\fr}{\mathfrak{r}}
\newcommand{\fA}{\mathfrak{A}}
\newcommand{\fB}{\mathfrak{B}}
\newcommand{\fC}{\mathfrak{C}}
\newcommand{\fD}{\mathfrak{D}}

%%%%%%%%%%%%%%%%%%%%%%%%%%%%%%%%%%%%%%%%%%%%%
%          Bold Letters
%%%%%%%%%%%%%%%%%%%%%%%%%%%%%%%%%%%%%%%%%%%%%
\newcommand{\ba}{\mathbf{a}}
\newcommand{\bb}{\mathbf{b}}
\newcommand{\bc}{\mathbf{c}}
\newcommand{\bd}{\mathbf{d}}
\newcommand{\be}{\mathbf{e}}
\newcommand{\bu}{\mathbf{u}}
\newcommand{\bv}{\mathbf{v}}
\newcommand{\bw}{\mathbf{w}}
\newcommand{\bx}{\mathbf{x}}
\newcommand{\by}{\mathbf{y}}
\newcommand{\bz}{\mathbf{z}}
\newcommand{\bSigma}{\boldsymbol{\Sigma}}
\newcommand{\bPi}{\boldsymbol{\Pi}}
\newcommand{\bDelta}{\boldsymbol{\Delta}}
\newcommand{\bdelta}{\boldsymbol{\delta}}
\newcommand{\bgamma}{\boldsymbol{\gamma}}
\newcommand{\bGamma}{\boldsymbol{\Gamma}}

%%%%%%%%%%%%%%%%%%%%%%%%%%%%%%%%%%%%%%%%%%%%%
%         Bold numbers
%%%%%%%%%%%%%%%%%%%%%%%%%%%%%%%%%%%%%%%%%%%%%
\newcommand{\bzero}{\mathbf{0}}

%%%%%%%%%%%%%%%%%%%%%%%%%%%%%%%%%%%%%%%%%%%%%
% Projective-Like Pointclasses
%%%%%%%%%%%%%%%%%%%%%%%%%%%%%%%%%%%%%%%%%%%%%
\newcommand{\Sa}[2][\alpha]{\Sigma_{(#1,#2)}}
\newcommand{\Pa}[2][\alpha]{\Pi_{(#1,#2)}}
\newcommand{\Da}[2][\alpha]{\Delta_{(#1,#2)}}
\newcommand{\Aa}[2][\alpha]{A_{(#1,#2)}}
\newcommand{\Ca}[2][\alpha]{C_{(#1,#2)}}
\newcommand{\Qa}[2][\alpha]{Q_{(#1,#2)}}
\newcommand{\da}[2][\alpha]{\delta_{(#1,#2)}}
\newcommand{\leqa}[2][\alpha]{\leq_{(#1,#2)}}
\newcommand{\lessa}[2][\alpha]{<_{(#1,#2)}}
\newcommand{\equiva}[2][\alpha]{\equiv_{(#1,#2)}}


\newcommand{\Sl}[1]{\Sa[\lambda]{#1}}
\newcommand{\Pl}[1]{\Pa[\lambda]{#1}}
\newcommand{\Dl}[1]{\Da[\lambda]{#1}}
\newcommand{\Al}[1]{\Aa[\lambda]{#1}}
\newcommand{\Cl}[1]{\Ca[\lambda]{#1}}
\newcommand{\Ql}[1]{\Qa[\lambda]{#1}}

\newcommand{\San}{\Sa{n}}
\newcommand{\Pan}{\Pa{n}}
\newcommand{\Dan}{\Da{n}}
\newcommand{\Can}{\Ca{n}}
\newcommand{\Qan}{\Qa{n}}
\newcommand{\Aan}{\Aa{n}}
\newcommand{\dan}{\da{n}}
\newcommand{\leqan}{\leqa{n}}
\newcommand{\lessan}{\lessa{n}}
\newcommand{\equivan}{\equiva{n}}

\newcommand{\SigmaOneOmega}{\Sigma^1_{\omega}}
\newcommand{\SigmaOneOmegaPlusOne}{\Sigma^1_{\omega+1}}
\newcommand{\PiOneOmega}{\Pi^1_{\omega}}
\newcommand{\PiOneOmegaPlusOne}{\Pi^1_{\omega+1}}
\newcommand{\DeltaOneOmegaPlusOne}{\Delta^1_{\omega+1}}

%%%%%%%%%%%%%%%%%%%%%%%%%%%%%%%%%%%%%%%%%%%%%
% Linear Algebra
%%%%%%%%%%%%%%%%%%%%%%%%%%%%%%%%%%%%%%%%%%%%%
\newcommand{\transpose}[1]{{#1}^{\text{T}}}
\newcommand{\norm}[1]{\lVert{#1}\rVert}
\newcommand\aug{\fboxsep=-\fboxrule\!\!\!\fbox{\strut}\!\!\!}

%%%%%%%%%%%%%%%%%%%%%%%%%%%%%%%%%%%%%%%%%%%%%
% Number Theory
%%%%%%%%%%%%%%%%%%%%%%%%%%%%%%%%%%%%%%%%%%%%%
\newcommand{\av}[1]{\lvert#1\rvert}
\DeclareMathOperator{\divides}{\mid}
\DeclareMathOperator{\ndivides}{\nmid}
\DeclareMathOperator{\lcm}{lcm}
\DeclareMathOperator{\sign}{sign}
\newcommand{\floor}[1]{\left\lfloor{#1}\right\rfloor}
\DeclareMathOperator{\ConCl}{CC}
\DeclareMathOperator{\ord}{ord}


%%%%%%%%%%%%%%%%%%%%%%%%%%%%%%%%%%%%%%%%%%%%%%%%%%%%%%%%%%%%%%%%%%%%%%%%%%%
%%  Theorem-Like Declarations
%%%%%%%%%%%%%%%%%%%%%%%%%%%%%%%%%%%%%%%%%%%%%%%%%%%%%%%%%%%%%%%%%%%%%%%%%%

\newtheorem{theorem}{Theorem}[section]
\newtheorem{lemma}[theorem]{Lemma}
\newtheorem{corollary}[theorem]{Corollary}
\newtheorem{proposition}[theorem]{Proposition}


\theoremstyle{definition}

\newtheorem{definition}[theorem]{Definition}
\newtheorem{conjecture}[theorem]{Conjecture}
\newtheorem{remark}[theorem]{Remark}
\newtheorem{remarks}[theorem]{Remarks}
\newtheorem{notation}[theorem]{Notation}

\theoremstyle{remark}

\newtheorem*{note}{Note}
\newtheorem*{warning}{Warning}
\newtheorem*{question}{Question}
\newtheorem*{example}{Example}
\newtheorem*{fact}{Fact}


\newenvironment*{subproof}[1][Proof]
{\begin{proof}[#1]}{\renewcommand{\qedsymbol}{$\diamondsuit$} \end{proof}}

\newenvironment*{case}[1]
{\textbf{Case #1.  }\itshape }{}

\newenvironment*{claim}[1][Claim]
{\textbf{#1.  }\itshape }{}


\graphicspath{{images/}}

\pagestyle{plain}

\begin{document}

\title{Solutions to Homework for Lesson 16 \\ Applications of Gaussian Elimination}
\author{Math 325, Linear Algebra \\ Mitch Rudominer \\ Fall 2018 \\ SFSU }
\date{}

\maketitle


\textbf{Problem 1}

\bigskip

%%%%%%%%%%%%%%%%%%%%%%%%%%%%%%%%%%%%% (1 a)

\textbf{(a)} There is a unique solution:

\bigskip

$
\begin{pmatrix}
x \\  y
\end{pmatrix}
=
\begin{pmatrix}
\frac{-1}{2} \\[6pt]
\frac{-3}{4}
\end{pmatrix}
$


\bigskip

\textbf{(b)} There are infinitely many solutions:

\bigskip

$
\begin{pmatrix}
x \\  y \\z
\end{pmatrix}
=
\begin{pmatrix}
1-2z \\
-1+z\\
z
\end{pmatrix}
=
\begin{pmatrix}
1\\
-1 \\
0
\end{pmatrix}
+
z
\begin{pmatrix}
-2 \\
1 \\
1
\end{pmatrix}
$

\bigskip

\textbf{(c)} There are no solutions.

\bigskip

\textbf{(d)} There is a unique solution:

\bigskip

$
\begin{pmatrix}
x \\  y \\ z
\end{pmatrix}
=
\begin{pmatrix}
1 \\
-2 \\
1
\end{pmatrix}
$

\bigskip

\textbf{(e)}  There are infinitely many solutions:

\bigskip

$
\begin{pmatrix}
x \\  y \\z \\ w
\end{pmatrix}
=
\begin{pmatrix}
5-2z \\
1 \\
z \\
0
\end{pmatrix}
=
\begin{pmatrix}
5\\
1 \\
0 \\
0
\end{pmatrix}
+
z
\begin{pmatrix}
-2 \\
0 \\
1 \\
0
\end{pmatrix}
$


\bigskip

\textbf{(f)} There is a unique solution:

\bigskip

$
\begin{pmatrix}
x \\  y \\ z
\end{pmatrix}
=
\begin{pmatrix}
1 \\
0 \\
1
\end{pmatrix}
$

\bigskip

\textbf{(g)} There is a unique solution:

\bigskip

$
\begin{pmatrix}
x \\  y \\ z \\ w
\end{pmatrix}
=
\begin{pmatrix}
2 \\
1 \\
3 \\
1
\end{pmatrix}
$


%%%%%%%%%%%%%%%%%%%%%%%%%%%%%%%%%%%%%%%%%%%%%%%%%%%%%%%%%%%%%%

\bigskip

\textbf{Problem 3.}

\textbf{(b)} The three vectors are linearly independent so they do form a basis.
The way you can tell that they are linearly independent is to form a matrix
out of them and use Gaussian elimination to determine that the rank is 3.

\bigskip

\textbf{(c)} The three vectors are not linearly independent so they do not form
a basis. The way you can tell that they are not linearly independent is to
form a matrix out of them and use Gaussian elimination to determine that the rank is 2.

%%%%%%%%%%%%%%%%%%%%%%%%%%%%%%%%%%%%%%%%%%%%%%%%%%%%%

\bigskip

\textbf{Problem 4.} To solve 2.4.3, form a matrix out of the 4 vectors and
perform Gaussian elimination on it. This reveals that the rank of the matrix
is 2.

\bigskip

\textbf{(a)} The vectors do not span $\R^3$ since the dimension of their span is 2.

\textbf{(b)} The vectors are not linearly independent since  there are four of them in $\R^3$.

\textbf{(c)} They do not form a basis for $\R^3$ because they are neither linearly independent
nor span. It is not possible to choose a subset of them that form a basis since they do
not span.

\textbf{(d)} The dimension of the span is 2.

%%%%%%%%%%%%%%%%%%%%%%%%%%%%%%%%%%%%%%%%%%%%%%%%%%%%%

\bigskip

\textbf{Problem 5.} To solve 2.4.4, form a matrix out of the 4 vectors and
perform Gaussian elimination on it. This reveals that the rank of the matrix
is 3 and that the first 3 columns are basic columns.

\bigskip

\textbf{(a)} The vectors do  span $\R^3$ since the dimension of their span is 3.

\textbf{(b)} The vectors are not linearly independent since  there are four of them in $\R^3$.

\textbf{(c)} They do not form a basis for $\R^3$ because they are not linearly independent.
It is possible to chose a subset of them that form a basis, namely the first three vectors
form a basis.

\textbf{(d)} The dimension of the span is 3.

\bigskip

%%%%%%%%%%%%%%%%%%%%%%%%%%%%%%%%%%%%%%%%%%%%%%%%%%%%%

\bigskip

\textbf{Problem 6.} These exercises are asking you for a basis for the plane
given by a single linear equation. This is equivalent to asking for a basis
for the solution space to the homogeneous system with a single equation.
Since the system has a single equation, the corresponding matrix is already in row-echelon
form with a single basic column. When doing generalized back substitution we may
solve for any one of the variables in terms of the other ones.

\textbf{(a)} $z-2y=0 \SkipImplies z = 2y$. A solution to the homogeneous system
is
$
\begin{pmatrix}
x \\ y \\ z
\end{pmatrix}
=
\begin{pmatrix}
x \\ y \\ 2y
\end{pmatrix}
=
x
\begin{pmatrix}
1 \\ 0 \\ 0
\end{pmatrix}
+
y
\begin{pmatrix}
0 \\ 1 \\ 2
\end{pmatrix}
$

So a basis is
$
\singleton{
\begin{pmatrix}
1 \\ 0 \\ 0
\end{pmatrix}
,
\begin{pmatrix}
0 \\ 1 \\ 2
\end{pmatrix}
}
$

\bigskip

\textbf{(b)} $4x+3y-z=0 \SkipImplies z = 4x+3y$. A solution to the homogeneous system
is
$
\begin{pmatrix}
x \\ y \\ z
\end{pmatrix}
=
\begin{pmatrix}
x \\ y \\ 4x+3y
\end{pmatrix}
=
x
\begin{pmatrix}
1 \\ 0 \\ 4
\end{pmatrix}
+
y
\begin{pmatrix}
0 \\ 1 \\ 3
\end{pmatrix}
$

So a basis is
$
\singleton{
\begin{pmatrix}
1 \\ 0 \\ 4
\end{pmatrix}
,
\begin{pmatrix}
0 \\ 1 \\ 3
\end{pmatrix}
}
$

\bigskip

\textbf{(c)} $x+2y+z-w=0 \SkipImplies w = x+2y+z$. A solution to the homogeneous system
is
$
\begin{pmatrix}
x \\ y \\ z \\ w
\end{pmatrix}
=
\begin{pmatrix}
x \\ y \\z \\ x+2y+z
\end{pmatrix}
=
x
\begin{pmatrix}
1 \\ 0 \\ 0 \\ 1
\end{pmatrix}
+
y
\begin{pmatrix}
0 \\ 1 \\ 0 \\ 2
\end{pmatrix}
+
z
\begin{pmatrix}
0 \\ 0\\ 1 \\ 1
\end{pmatrix}
$

So a basis is
$
\singleton{
\begin{pmatrix}
1 \\ 0 \\ 0 \\ 1
\end{pmatrix}
,
\begin{pmatrix}
0 \\ 1 \\ 0 \\ 2
\end{pmatrix}
,
\begin{pmatrix}
0 \\ 0\\ 1 \\ 1
\end{pmatrix}
}
$

%%%%%%%%%%%%%%%%%%%%%%%%%%%%%%%%%%%%%%%%%%%%%%%%%%%%%
\bigskip

\textbf{Problem 7.}

\textbf{(a)}
$2x -y + 5z = 0 \SkipImplies y = 2x + 5z$

A basis for the kernel is:

$
\singleton{
\begin{pmatrix}
1 \\ 2 \\ 0
\end{pmatrix},
\begin{pmatrix}
0 \\ 5 \\ 1
\end{pmatrix}
}
$

The kernel is two dimensional so it is a plane.

\bigskip

\textbf{(b)}
\bigskip
$
\begin{pmatrix}
1 & 2 & -1 \\
3 & -2 & 0
\end{pmatrix}
\SkipImplies
\begin{pmatrix}
1 & 2 & -1 \\
0 & -8 & 3
\end{pmatrix}
$

\bigskip

\begin{itemize}
\item $-8y+3z = 0 \SkipImplies y = \frac{3}{8}z$
\item $x+2y -z = 0 \SkipImplies x + \frac{3}{4}z - z = 0 \SkipImplies x = \frac{1}{4}z$
\end{itemize}

A basis for the kernel is:

$
\singleton{
\begin{pmatrix}
\frac{1}{4} \\[6pt]
 \frac{3}{8}  \\
 1
\end{pmatrix}
}
$

\bigskip

The kernel is one dimensional and so it is a line.

\bigskip

\textbf{(c)}
\bigskip
$
\begin{pmatrix}
2 & 6 & -4 \\
-1 & -3 & 2
\end{pmatrix}
\SkipImplies
\begin{pmatrix}
2 & 6 & -4 \\
0 & 0 & 0
\end{pmatrix}
$

\bigskip

 $2x + 6y -4z = 0 \SkipImplies x = -3y +2z$

A basis for the kernel is:

$
\singleton{
\begin{pmatrix}
-3 \\ 1 \\ 0
\end{pmatrix},
\begin{pmatrix}
2 \\0 \\ 1
\end{pmatrix}
}
$

\bigskip

The kernel is two dimensional and so it is a plane



\bigskip

%%%%%%%%%%%%%%%%%%%%%%%%%%%%%%%%%%%%%%%%%%%%%%%%%%%%%

\textbf{Problem 8.} This is false. For example let
$
A=
\begin{pmatrix}
0 & 1 \\
0 & 0
\end{pmatrix}
$.
Let $\bv =
\begin{pmatrix}
1 \\ 0
\end{pmatrix}
$
Then $\bv\in\ran(A)\intersect\ker(A)$.

\bigskip

\textbf{Problem 10.} Exercise 2.5.21---but only for the image and kernel.

The way to solve these problems is to start by solving the homogeneous
system $A\bx=\bzero$.


\bigskip

\textbf{(a)}

\bigskip

$
\begin{pmatrix}
1 & -3 \\
2 & -6
\end{pmatrix}
\SkipImplies
\begin{pmatrix}
1 & -3 \\
0 & 0
\end{pmatrix}
$

\bigskip

$x - 3y = 0 \SkipImplies x = 3y$

\bigskip

So the solution to the homogeneous system is

$
\begin{pmatrix}
x \\
y
\end{pmatrix}
=
\begin{pmatrix}
3y \\ y
\end{pmatrix}
=
y
\begin{pmatrix}
3 \\ 1
\end{pmatrix}
$

So a basis for the kernel is
$\singleton{
\begin{pmatrix}
3 \\ 1
\end{pmatrix}
}$.

A basis for the image is the set of basic columns from the original matrix:

$\singleton{
\begin{pmatrix}
1 \\ 2
\end{pmatrix}
}$.

\bigskip

\textbf{(b)}

\bigskip

$
\begin{pmatrix}
0 & 0 & -8  \\
1 & 2 & -1  \\
2 & 4 & 6 &
\end{pmatrix}
\SkipImplies
\begin{pmatrix}
1 & 2 & -1  \\
0 & 0 & -8  \\
2 & 4 &  6 &
\end{pmatrix}
\SkipImplies
\begin{pmatrix}
1 & 2 & -1 \\
0 & 0 & -8 \\
0 & 0 &  8
\end{pmatrix}
\SkipImplies
\begin{pmatrix}
1 & 2 & -1 \\
0 & 0 & -8 \\
0 & 0 &  0
\end{pmatrix}
$

\begin{itemize}
\item $-8z = 0 \SkipImplies z = 0$
\item $x + 2y - z = 0 \SkipImplies x + 2y = 0 \SkipImplies x = -2y$
\end{itemize}


So the solution to the homogeneous system is

$
\begin{pmatrix}
x \\
y \\
z \\
\end{pmatrix}
=
\begin{pmatrix}
-2y \\
y \\
0
\end{pmatrix}
=
y
\begin{pmatrix}
-2 \\ 1 \\ 0
\end{pmatrix}
$

So a basis for the kernel is
$\singleton{
\begin{pmatrix}
-2 \\ 1 \\ 0
\end{pmatrix}
}$.

A basis for the image is the set of basic columns from the original matrix:

$\singleton{
\begin{pmatrix}
0 \\ 1 \\ 2
\end{pmatrix},
\begin{pmatrix}
-8 \\ -1 \\ 6
\end{pmatrix}
}$.

\bigskip

\textbf{(c)}

\bigskip

$
\begin{pmatrix}
1 & 1 & 2 & 1  \\
1 & 0 & -1 & 3  \\
2 & 3 & 7 & 0
\end{pmatrix}
\SkipImplies
\begin{pmatrix}
1 & 1 & 2 & 1 \\
0 & -1 & -3 & 2  \\
0 & 1 & 3 & -2
\end{pmatrix}
\SkipImplies
\begin{pmatrix}
1 & 1 & 2 & 1  \\
0 & -1 & -3 & 2 \\
0 & 0 & 0 & 0
\end{pmatrix}
$

\begin{itemize}
\item $-y -3z + 2w = 0 \SkipImplies y = -3z + 2w$
\item $x + y +2z + w = 0 \SkipImplies x -3z + 2w + 2z + w = 0 \SkipImplies x-z+3w = 0 \SkipImplies x = z-3w$.
\end{itemize}


So the solution to the homogeneous system is

$
\begin{pmatrix}
x \\
y \\
z \\
w
\end{pmatrix}
=
\begin{pmatrix}
z-3w \\
-3z+2w \\
z \\
w \\
\end{pmatrix}
=
z
\begin{pmatrix}
1 \\ -3 \\ 1 \\0
\end{pmatrix}
+
w
\begin{pmatrix}
-3 \\ 2 \\ 0 \\1
\end{pmatrix}
$

So a basis for the kernel is
$\singleton{
\begin{pmatrix}
1 \\ -3 \\ 1 \\0
\end{pmatrix},
\begin{pmatrix}
-3 \\ 2 \\ 0 \\1
\end{pmatrix}
}$

A basis for the image is the set of basic columns from the original matrix:

$\singleton{
\begin{pmatrix}
1 \\ 1 \\2
\end{pmatrix},
\begin{pmatrix}
1 \\ 0 \\3
\end{pmatrix}
}$.

\bigskip

\textbf{(d)}

\bigskip

$
\begin{pmatrix}
1 & -3 & 2 & 2 & 1   \\
0 &  3 & -6 & 0 & -2  \\
2 & -3 & -2 & 4 & 0  \\
3 & -3 & -6 & 6 & 3  \\
1 &  0 & -4 & 2 & 3
\end{pmatrix}
\SkipImplies
\begin{pmatrix}
1 & -3 & 2 & 2 & 1  \\
0 & 3 & -6 & 0 & -2 \\
0 & 3 & -6 & 0 & -2 \\
0 & 6 & -12 & 0 & 0 \\
0 & 3 & -6 & 0 & 2
\end{pmatrix}
\SkipImplies
\begin{pmatrix}
1 & -3 & 2 & 2 &  1   \\
0 & 3 & -6 & 0 & -2  \\
0 & 0 & 0 & 0 &   0    \\
0 & 0 & 0 & 0 &   4    \\
0 & 0 & 0 & 0 &   4
\end{pmatrix}
\SkipImplies
\begin{pmatrix}
1 & -3 & 2 & 2 & 1   \\
0 & 3 & -6 & 0 & -2  \\
0 & 0 & 0 & 0 & 4    \\
0 & 0 & 0 & 0 & 0    \\
0 & 0 & 0 & 0 & 4
\end{pmatrix}
\SkipImplies
\begin{pmatrix}
1 & -3 & 2 & 2 & 1   \\
0 & 3 & -6 & 0 & -2  \\
0 & 0 & 0 & 0 & 4    \\
0 & 0 & 0 & 0 & 0    \\
0 & 0 & 0 & 0 & 0
\end{pmatrix}
$

Let's call the five variables $x,y,z,w,v$.

\begin{itemize}
\item $4v=0 \SkipImplies v=0$
\item $3y -6z -2v = 0 \SkipImplies 3y -6z = 0 \SkipImplies y=2z$.
\item $x - 3y +2z +2w +v = 0 \SkipImplies x -6z +2z +2w = 0 \SkipImplies x = 4z - 2w$
\end{itemize}


So the solution to the homogeneous system is

$
\begin{pmatrix}
x \\
y \\
z \\
w \\
v
\end{pmatrix}
=
\begin{pmatrix}
4z-2w\\
2z\\
z \\
w \\
0
\end{pmatrix}
=
z
\begin{pmatrix}
4 \\ 2 \\ 1 \\ 0 \\ 0
\end{pmatrix}
+
w
\begin{pmatrix}
-2 \\ 0 \\0 \\1 \\ 0
\end{pmatrix}
$

So a basis for the kernel is
$\singleton{
\begin{pmatrix}
4 \\ 2 \\ 1 \\ 0 \\ 0
\end{pmatrix},
\begin{pmatrix}
-2 \\ 0 \\0 \\1 \\ 0
\end{pmatrix}
}$

A basis for the image is the set of basic columns from the original matrix:

$\singleton{
\begin{pmatrix}
1  \\
0  \\
2  \\
3  \\
1
\end{pmatrix},
\begin{pmatrix}
-3 \\
 3 \\
-3 \\
-3 \\
 0
\end{pmatrix},
\begin{pmatrix}
 1 \\
-2 \\
 0  \\
 3  \\
 3
\end{pmatrix}
}$.


\bigskip

\textbf{Problem 11.} Exercise 2.5.22 on page 118.

First we use Gaussian elimination in order to find the basic columns.

\bigskip

$
\begin{pmatrix}
-1 & 2 & 0 & -3 & 5 \\
2 & -4 & 1 & 1 & -4 \\
-3 & 6 & 2 & 0 & 8
\end{pmatrix}
\SkipImplies
\begin{pmatrix}
-1 & 2 & 0 & -3 & 5 \\
0 & 0 & 1 & -5 & 6 \\
0 & 0 & 2 & 9 & -7
\end{pmatrix}
\SkipImplies
\begin{pmatrix}
-1 & 2 & 0 & -3 & 5 \\
0 & 0 & 1 & -5 & 6 \\
0 & 0 & 0 & 19 & -19
\end{pmatrix}
$

\bigskip

So the basic columns are 1, 3 and 4. This means that these columns form a basis
for the column space. That is true in both the original matrix and the row-echelon form
matrix.

Now we are asked to express the other columns as a linear combination of the
basic columns. We can use the fact that the answer is the same for the original
matrix and the row-echelon form matrix. Also we can use the fact that each
non-basic column must be a linear combination of the basic columns to its left.

\begin{itemize}
  \item Let's call the columns $\bc_1,\bc_2,\bc_3,\bc_4,\bc_5$.
  \item $\bc_2 = -2 \bc_1$
  \item  $\bc_5 = -2 \bc_1 + \bc_3 -\bc_4$
\end{itemize}

If you don't see how I got the solution for $\bc_5$
another way to think about solving that is: Solve the system of equations
that has the matrix with columns $\bc_1,\bc_3, \bc_4$ on the left and $\bc_5$ on the right.
We want to find $x,y,z$ such that $x\bc_1+y\bc_3+z\bc_4 = \bc_5$.
We can use the matrix that is already in row-echelon form and just do back substition.

\begin{itemize}
\item $19z = -19 \SkipImplies z = -1$
\item $y - 5z = 6 \SkipImplies y + 5 = 6 \SkipImplies y = 1$
\item $-x -3z = 5 \SkipImplies -x + 3 = 5 \SkipImplies x = -2$.
\end{itemize}

\bigskip

\textbf{Problem 12.} For each of the matrices in Exercise 2.5.23 on page 118,
give the dimension of the kernel and the image.

\bigskip

\textbf{(i)} The two columns are clearly linearly dependent so the rank of
the matrix is 1. So the dimension of the image is 1 and the dimension of
the kernel is 1.

\bigskip

\textbf{(ii)} It is clear that the first and third columns are scalar multiples
of the second column. So the rank of the matrix is 1. So the dimension of
the image is 1 and the dimension of the kernel is 2.

\bigskip


\textbf{(iii)} It is clear that the two columns are linearly independent, so
the rank of the matrix is 2. So the dimension of the image is 2 and the dimension
of the kernel is 0.

\bigskip

\textbf{(iv)} I can't tell the rank of the matrix by inspection so I will do
Gaussian elimination.

\bigskip

$
\begin{pmatrix}
2 & -5 & -1 \\
1 & -6 & -4 \\
3 & -4 & 2
\end{pmatrix}
\SkipImplies
\begin{pmatrix}
2 & -5 & -1 \\
0 & \frac{-7}{2} & \frac{-7}{2} \\
0 & \frac{7}{2} & \frac{7}{2}
\end{pmatrix}
\SkipImplies
\begin{pmatrix}
2 & -5 & -1 \\
0 & \frac{-7}{2} & \frac{-7}{2} \\
0 & 0 & 0
\end{pmatrix}
$

\bigskip

So the rank of the matrix is 2. So the dimension of the image is 2 and the
dimension of the kernel is 1.

\bigskip

\textbf{(v)} I can't tell the rank of the matrix by inspection so I will do
Gaussian elimination.

\bigskip

$
\begin{pmatrix}
2 & 5 & 7 \\
6 & 13 & 19 \\
3 & 8 & 11 \\
1 & 2 & 3
\end{pmatrix}
\SkipImplies
\begin{pmatrix}
2 & 5 & 7 \\
0 & -2 & -2 \\
0 & \frac{1}{2} & \frac{1}{2} \\
0 & \frac{-1}{2} & \frac{-1}{2}
\end{pmatrix}
\SkipImplies
\begin{pmatrix}
2 & 5 & 7 \\
0 & -2 & -2 \\
0 & 0 & 0 \\
0 & 0 & 0
\end{pmatrix}
$

\bigskip

So the rank of the matrix is 2. So the dimension of the image is 2 and the
dimension of the kernel is 1.

\bigskip

\textbf{(vi)} I can't tell the rank of the matrix by inspection so I will do
Gaussian elimination.

\bigskip

$
\begin{pmatrix}
1 & 2 & 3 & 4 \\
3 & 2 & 4 & 1 \\
1 & -2 & 2 & 7 \\
3 & 6 & 5 & -2
\end{pmatrix}
\SkipImplies
\begin{pmatrix}
1 & 2 & 3 & 4 \\
0 & -4 & -5 & -11 \\
0 & -4 & -1 & 3 \\
0 & 0 & -4 & -14
\end{pmatrix}
\SkipImplies
\begin{pmatrix}
1 & 2 & 3 & 4 \\
0 & -4 & -5 & -11 \\
0 & 0 & 4 & 14 \\
0 & 0 & -4 & -14
\end{pmatrix}
\SkipImplies
\begin{pmatrix}
1 & 2 & 3 & 4 \\
0 & -4 & -5 & -11 \\
0 & 0 & 4 & 14 \\
0 & 0 & 0 & 0
\end{pmatrix}
$

\bigskip

So the rank of the matrix is 3. So the dimension of the image is 3 and the
dimension of the kernel is 1.

\bigskip

\textbf{(vii)} I can't tell the rank of the matrix by inspection so I will do
Gaussian elimination.

\bigskip

$
\begin{pmatrix}
2 & 4 & 0 & -6 & 0 \\
1 & 2 & 3 & 15 & 0 \\
3 & 6 & -1 & 15 & 5 \\
-3 & -6 & 2 & 21 & -6
\end{pmatrix}
\SkipImplies
\begin{pmatrix}
2 & 4 & 0 & -6 & 0 \\
0 & 0 & 3 & 18 & 0 \\
0 & 0 & -1 & 24 & 5 \\
0 & 0 & 2 & 12 & -6
\end{pmatrix}
\SkipImplies
\begin{pmatrix}
2 & 4 & 0 & -6 & 0 \\
0 & 0 & 3 & 18 & 0 \\
0 & 0 & 0 & 30 & 5 \\
0 & 0 & 0 & 0 & -6
\end{pmatrix}
$

\bigskip

So the rank of the matrix is 4. So the dimension of the image is 4 and the
dimension of the kernel is 1.


\bigskip

\textbf{Problem 13.} Exercise 2.5.24 on page 118.

\bigskip

\textbf{(a)} The two vectors from $\R^3$ are linearly independent, as is
easily seen from inspection. Therefore they are a basis for the subspace
they span, which has dimension 2.

\bigskip

\textbf{(b)} It is clear from inspection that the second and third vectors
are scalar multiles of the first vector. Therefore the first vector is a basis
for the subspace, which has dimension 1.

\bigskip

\textbf{(c)} I can't tell by inspection whether or not the four vectors from
$\R^4$ are linearly independent so I will form a matrix with those vectors
as columns and perform Gaussian elimination.

$
\begin{pmatrix}
1 & 1 & 2 & 1 \\
0 & 0 & 2 & 2 \\
1 & 0 & 1 & 3 \\
0 & 1 & 0 & -3
\end{pmatrix}
\SkipImplies
\begin{pmatrix}
1 & 1 & 2 & 1 \\
0 & 0 & 2 & 2 \\
0 & -1 & -1 & 2 \\
0 & 1 & 0 & -3
\end{pmatrix}
\SkipImplies
\begin{pmatrix}
1 & 1 & 2 & 1 \\
0 & -1 & -1 & 2 \\
0 & 0 & 2 & 2 \\
0 & 1 & 0 & -3
\end{pmatrix}
\SkipImplies
\begin{pmatrix}
1 & 1 & 2 & 1 \\
0 & -1 & -1 & 2 \\
0 & 0 & 2 & 2 \\
0 & 0 & -1 & -1
\end{pmatrix}
\SkipImplies
\begin{pmatrix}
1 & 1 & 2 & 1 \\
0 & -1 & -1 & 2 \\
0 & 0 & 2 & 2 \\
0 & 0 & 0 & 0
\end{pmatrix}
$

So we have learned that the rank of the matrix is 3 and the first three columns
are basic columns. Therefore the dimension of the subspace spanned by the
four given vectors is 3 and the first three of the given vectors form a basis for it.

\bigskip

\textbf{(d)} We are given five vector in $\R^4$ so we know that they are not
linearly independent. But I cannot tell by inspection what the dimension of
their span is. So we use Gaussian elimination.

$
\begin{pmatrix}
1 & 0 & -3 & 1 & 2 \\
0 & 1 & -4 & -3 & 1 \\
-3 & 2 & 1 & -8 & -6 \\
2 & -3 & 6 & 7 & 9
\end{pmatrix}
\SkipImplies
\begin{pmatrix}
1 & 0 & -3 & 1 & 2 \\
0 & 1 & -4 & -3 & 1 \\
0 & 2 & -8 & -5 & 0 \\
0 & -3 & 12 & 5 & 5
\end{pmatrix}
\SkipImplies
\begin{pmatrix}
1 & 0 & -3 & 1 & 2 \\
0 & 1 & -4 & -3 & 1 \\
0 & 0 & 0 & 1 & -2 \\
0 & 0 & 0 & -4 & 8
\end{pmatrix}
\SkipImplies
\begin{pmatrix}
1 & 0 & -3 & 1 & 2 \\
0 & 1 & -4 & -3 & 1 \\
0 & 0 & 0 & 1 & -2 \\
0 & 0 & 0 & 0 & 0
\end{pmatrix}
$

So the matrix has rank 3 and the basic columns are 1, 2 and 5.
Therefore the first, second and fifth of the original set of vectors forms
a basis for the subspace spanned by all five vectors, and that subspace
has dimension 3.

\textbf{(e)} We are given six vector in $\R^5$ so we know that they are not
linearly independent. But I cannot tell by inspection what the dimension of
their span is. So we use Gaussian elimination.

$
\begin{pmatrix}
1 & 2 & 3 & 0 & 1 & 1 \\
1 & -1 & 0 & -3 & 3 & 0 \\
-1 & 2 & 1 & 4 & -1 & 3 \\
1 & 2 & 3 & 0 & 2 & 2 \\
1 & 1 & 2 & -1 & 1 & 0
\end{pmatrix}
\SkipImplies
\begin{pmatrix}
1 & 2 & 3 & 0 & 1 & 1 \\
0 & -3 & -3 & -3 & 2 & -1 \\
0 & 4 & 4 & 4 & 0 & 4 \\
0 & 0 & 0 & 0 & 1 & 1 \\
0 & -1 & -1 & -1 & 0 & -1
\end{pmatrix}
\SkipImplies
\begin{pmatrix}
1 & 2 & 3 & 0 & 1 & 1 \\
0 & -3 & -3 & -3 & 2 & -1 \\
0 & 0 & 0 & 0 & \frac{8}{3} & \frac{8}{3} \\
0 & 0 & 0 & 0 & 1 & 1 \\
0 & 0 & 0 & 0 & \frac{-2}{3} & \frac{-2}{3}
\end{pmatrix}
\SkipImplies
\begin{pmatrix}
1 & 2 & 3 & 0 & 1 & 1 \\
0 & -3 & -3 & -3 & 2 & -1 \\
0 & 0 & 0 & 0 & 1 & 1 \\
0 & 0 & 0 & 0 & \frac{8}{3} & \frac{8}{3} \\
0 & 0 & 0 & 0 & \frac{-2}{3} & \frac{-2}{3}
\end{pmatrix}
\SkipImplies
\begin{pmatrix}
1 & 2 & 3 & 0 & 1 & 1 \\
0 & -3 & -3 & -3 & 2 & -1 \\
0 & 0 & 0 & 0 & 1 & 1 \\
0 & 0 & 0 & 0 & 0 & 0 \\
0 & 0 & 0 & 0 & 0 & 0
\end{pmatrix}
$

So the matrix has rank 3 and the basic columns are 1, 2 and 5.
Therefore the first, second and fifth of the original set of vectors forms
a basis for the subspace spanned by all six vectors, and that subspace
has dimension 3.



\end{document}
