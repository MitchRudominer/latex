% $Header$

\documentclass{beamer}
%\documentclass[handout]{beamer}

\usepackage{amsmath,amssymb,latexsym,eucal,amsthm,graphicx,xcolor}
%%%%%%%%%%%%%%%%%%%%%%%%%%%%%%%%%%%%%%%%%%%%%
% Common Set Theory Constructs
%%%%%%%%%%%%%%%%%%%%%%%%%%%%%%%%%%%%%%%%%%%%%

\newcommand{\setof}[2]{\left\{ \, #1 \, \left| \, #2 \, \right.\right\}}
\newcommand{\lsetof}[2]{\left\{\left. \, #1 \, \right| \, #2 \,  \right\}}
\newcommand{\bigsetof}[2]{\bigl\{ \, #1 \, \bigm | \, #2 \,\bigr\}}
\newcommand{\Bigsetof}[2]{\Bigl\{ \, #1 \, \Bigm | \, #2 \,\Bigr\}}
\newcommand{\biggsetof}[2]{\biggl\{ \, #1 \, \biggm | \, #2 \,\biggr\}}
\newcommand{\Biggsetof}[2]{\Biggl\{ \, #1 \, \Biggm | \, #2 \,\Biggr\}}
\newcommand{\dotsetof}[2]{\left\{ \, #1 \, : \, #2 \, \right\}}
\newcommand{\bigdotsetof}[2]{\bigl\{ \, #1 \, : \, #2 \,\bigr\}}
\newcommand{\Bigdotsetof}[2]{\Bigl\{ \, #1 \, \Bigm : \, #2 \,\Bigr\}}
\newcommand{\biggdotsetof}[2]{\biggl\{ \, #1 \, \biggm : \, #2 \,\biggr\}}
\newcommand{\Biggdotsetof}[2]{\Biggl\{ \, #1 \, \Biggm : \, #2 \,\Biggr\}}
\newcommand{\sequence}[2]{\left\langle \, #1 \,\left| \, #2 \, \right. \right\rangle}
\newcommand{\lsequence}[2]{\left\langle\left. \, #1 \, \right| \,#2 \,  \right\rangle}
\newcommand{\bigsequence}[2]{\bigl\langle \,#1 \, \bigm | \, #2 \, \bigr\rangle}
\newcommand{\Bigsequence}[2]{\Bigl\langle \,#1 \, \Bigm | \, #2 \, \Bigr\rangle}
\newcommand{\biggsequence}[2]{\biggl\langle \,#1 \, \biggm | \, #2 \, \biggr\rangle}
\newcommand{\Biggsequence}[2]{\Biggl\langle \,#1 \, \Biggm | \, #2 \, \Biggr\rangle}
\newcommand{\singleton}[1]{\left\{#1\right\}}
\newcommand{\angles}[1]{\left\langle #1 \right\rangle}
\newcommand{\bigangles}[1]{\bigl\langle #1 \bigr\rangle}
\newcommand{\Bigangles}[1]{\Bigl\langle #1 \Bigr\rangle}
\newcommand{\biggangles}[1]{\biggl\langle #1 \biggr\rangle}
\newcommand{\Biggangles}[1]{\Biggl\langle #1 \Biggr\rangle}


\newcommand{\force}[1]{\Vert\!\underset{\!\!\!\!\!#1}{\!\!\!\relbar\!\!\!%
\relbar\!\!\relbar\!\!\relbar\!\!\!\relbar\!\!\relbar\!\!\relbar\!\!\!%
\relbar\!\!\relbar\!\!\relbar}}
\newcommand{\nforce}[1]{\Vert\!\underset{\!\!\!\!\!#1}{\!\!\!\relbar\!\!\!%
\relbar\!\!\relbar\!\!\relbar\!\!\!\relbar\!\!\relbar\!\!\relbar\!\!\!%
\relbar\!\!\not\relbar\!\!\relbar}}
\newcommand{\forcein}[2]{\overset{#2}{\Vert\underset{\!\!\!\!\!#1}%
{\!\!\!\relbar\!\!\!\relbar\!\!\relbar\!\!\relbar\!\!\!\relbar\!\!\relbar\!%
\!\relbar\!\!\!\relbar\!\!\relbar\!\!\relbar\!\!\relbar\!\!\!\relbar\!\!%
\relbar\!\!\relbar}}}

\newcommand{\pre}[2]{{}^{#2}\!{#1}}

\newcommand{\restr}{\!\!\upharpoonright\!}

%%%%%%%%%%%%%%%%%%%%%%%%%%%%%%%%%%%%%%%%%%%%%
% Set-Theoretic Connectives
%%%%%%%%%%%%%%%%%%%%%%%%%%%%%%%%%%%%%%%%%%%%%

\newcommand{\intersect}{\cap}
\newcommand{\union}{\cup}
\newcommand{\Intersection}[1]{\bigcap\limits_{#1}}
\newcommand{\Union}[1]{\bigcup\limits_{#1}}
\newcommand{\adjoin}{{}^\frown}
\newcommand{\forces}{\Vdash}

%%%%%%%%%%%%%%%%%%%%%%%%%%%%%%%%%%%%%%%%%%%%%
% Miscellaneous
%%%%%%%%%%%%%%%%%%%%%%%%%%%%%%%%%%%%%%%%%%%%%
\newcommand{\defeq}{=_{\text{def}}}
\newcommand{\Turingleq}{\leq_{\text{T}}}
\newcommand{\Turingless}{<_{\text{T}}}
\newcommand{\lexleq}{\leq_{\text{lex}}}
\newcommand{\lexless}{<_{\text{lex}}}
\newcommand{\Turingequiv}{\equiv_{\text{T}}}

%%%%%%%%%%%%%%%%%%%%%%%%%%%%%%%%%%%%%%%%%%%%%
% Constants
%%%%%%%%%%%%%%%%%%%%%%%%%%%%%%%%%%%%%%%%%%%%%
\newcommand{\R}{\mathbb{R}}
\renewcommand{\P}{\mathbb{P}}
\newcommand{\Q}{\mathbb{Q}}
\newcommand{\Z}{\mathbb{Z}}
\newcommand{\C}{\mathbb{C}}
\newcommand{\N}{\mathbb{N}}
\newcommand{\B}{\mathbb{B}}
\newcommand{\LofR}{L(\R)}
\newcommand{\JofR}[1]{J_{#1}(\R)}
\newcommand{\SofR}[1]{S_{#1}(\R)}
\newcommand{\JalphaR}{\JofR{\alpha}}
\newcommand{\JbetaR}{\JofR{\beta}}
\newcommand{\JlambdaR}{\JofR{\lambda}}
\newcommand{\SalphaR}{\SofR{\alpha}}
\newcommand{\SbetaR}{\SofR{\beta}}
\newcommand{\Pkl}{\mathcal{P}_{\kappa}(\lambda)}
\DeclareMathOperator{\con}{con}
\DeclareMathOperator{\ORD}{OR}
\DeclareMathOperator{\Ord}{OR}
\DeclareMathOperator{\WO}{WO}
\DeclareMathOperator{\OD}{OD}
\DeclareMathOperator{\HOD}{HOD}
\DeclareMathOperator{\HC}{HC}
\DeclareMathOperator{\WF}{WF}
\DeclareMathOperator{\HF}{HF}
\newcommand{\One}{I}
\newcommand{\ONE}{I}
\newcommand{\Two}{II}
\newcommand{\TWO}{II}

%%%%%%%%%%%%%%%%%%%%%%%%%%%%%%%%%%%%%%%%%%%%%
% Commutative Algebra Constants
%%%%%%%%%%%%%%%%%%%%%%%%%%%%%%%%%%%%%%%%%%%%%
\DeclareMathOperator{\dottimes}{\dot{\times}}

%%%%%%%%%%%%%%%%%%%%%%%%%%%%%%%%%%%%%%%%%%%%%
% Theories
%%%%%%%%%%%%%%%%%%%%%%%%%%%%%%%%%%%%%%%%%%%%%
\DeclareMathOperator{\ZFC}{ZFC}
\DeclareMathOperator{\ZF}{ZF}
\DeclareMathOperator{\AD}{AD}
\DeclareMathOperator{\ADR}{AD_{\R}}
\DeclareMathOperator{\KP}{KP}
\DeclareMathOperator{\PD}{PD}
\DeclareMathOperator{\CH}{CH}
\DeclareMathOperator{\HPC}{HPC} % HOD pair capturing
%%%%%%%%%%%%%%%%%%%%%%%%%%%%%%%%%%%%%%%%%%%%%
% Iteration Trees
%%%%%%%%%%%%%%%%%%%%%%%%%%%%%%%%%%%%%%%%%%%%%

\newcommand{\pred}{\text{-pred}}

%%%%%%%%%%%%%%%%%%%%%%%%%%%%%%%%%%%%%%%%%%%%%%%%
% Operator Names
%%%%%%%%%%%%%%%%%%%%%%%%%%%%%%%%%%%%%%%%%%%%%%%%
\DeclareMathOperator{\Det}{Det}
\DeclareMathOperator{\dom}{dom}
\DeclareMathOperator{\ran}{ran}
\DeclareMathOperator{\range}{ran}
\DeclareMathOperator{\image}{image}
\DeclareMathOperator{\crit}{crit}
\DeclareMathOperator{\card}{card}
\DeclareMathOperator{\cf}{cf}
\DeclareMathOperator{\cof}{cof}
\DeclareMathOperator{\rank}{rank}
\DeclareMathOperator{\ot}{o.t.}
\DeclareMathOperator{\ords}{o}
\DeclareMathOperator{\ro}{r.o.}
\DeclareMathOperator{\rud}{rud}
\DeclareMathOperator{\Powerset}{\mathcal{P}}
\DeclareMathOperator{\length}{lh}
\DeclareMathOperator{\lh}{lh}
\DeclareMathOperator{\limit}{lim}
\DeclareMathOperator{\fld}{fld}
\DeclareMathOperator{\projection}{p}
\DeclareMathOperator{\Ult}{Ult}
\DeclareMathOperator{\ULT}{Ult}
\DeclareMathOperator{\Coll}{Coll}
\DeclareMathOperator{\Col}{Col}
\DeclareMathOperator{\Hull}{Hull}
\DeclareMathOperator*{\dirlim}{dir lim}
\DeclareMathOperator{\Scale}{Scale}
\DeclareMathOperator{\supp}{supp}
\DeclareMathOperator{\trancl}{tran.cl.}
\DeclareMathOperator{\trace}{Tr}
\DeclareMathOperator{\diag}{diag}
\DeclareMathOperator{\spn}{span}
\DeclareMathOperator{\sgn}{sgn}
%%%%%%%%%%%%%%%%%%%%%%%%%%%%%%%%%%%%%%%%%%%%%
% Logical Connectives
%%%%%%%%%%%%%%%%%%%%%%%%%%%%%%%%%%%%%%%%%%%%%
\newcommand{\IImplies}{\Longrightarrow}
\newcommand{\SkipImplies}{\quad\Longrightarrow\quad}
\newcommand{\Ifff}{\Longleftrightarrow}
\newcommand{\iimplies}{\longrightarrow}
\newcommand{\ifff}{\longleftrightarrow}
\newcommand{\Implies}{\Rightarrow}
\newcommand{\Iff}{\Leftrightarrow}
\renewcommand{\implies}{\rightarrow}
\renewcommand{\iff}{\leftrightarrow}
\newcommand{\AND}{\wedge}
\newcommand{\OR}{\vee}
\newcommand{\st}{\text{ s.t. }}
\newcommand{\Or}{\text{ or }}

%%%%%%%%%%%%%%%%%%%%%%%%%%%%%%%%%%%%%%%%%%%%%
% Function Arrows
%%%%%%%%%%%%%%%%%%%%%%%%%%%%%%%%%%%%%%%%%%%%%

\newcommand{\injection}{\xrightarrow{\text{1-1}}}
\newcommand{\surjection}{\xrightarrow{\text{onto}}}
\newcommand{\bijection}{\xrightarrow[\text{onto}]{\text{1-1}}}
\newcommand{\cofmap}{\xrightarrow{\text{cof}}}
\newcommand{\map}{\rightarrow}

%%%%%%%%%%%%%%%%%%%%%%%%%%%%%%%%%%%%%%%%%%%%%
% Mouse Comparison Operators
%%%%%%%%%%%%%%%%%%%%%%%%%%%%%%%%%%%%%%%%%%%%%
\newcommand{\initseg}{\trianglelefteq}
\newcommand{\properseg}{\lhd}
\newcommand{\notinitseg}{\ntrianglelefteq}
\newcommand{\notproperseg}{\ntriangleleft}

%%%%%%%%%%%%%%%%%%%%%%%%%%%%%%%%%%%%%%%%%%%%%
%           calligraphic letters
%%%%%%%%%%%%%%%%%%%%%%%%%%%%%%%%%%%%%%%%%%%%%
\newcommand{\cA}{\mathcal{A}}
\newcommand{\cB}{\mathcal{B}}
\newcommand{\cC}{\mathcal{C}}
\newcommand{\cD}{\mathcal{D}}
\newcommand{\cE}{\mathcal{E}}
\newcommand{\cF}{\mathcal{F}}
\newcommand{\cG}{\mathcal{G}}
\newcommand{\cH}{\mathcal{H}}
\newcommand{\cI}{\mathcal{I}}
\newcommand{\cJ}{\mathcal{J}}
\newcommand{\cK}{\mathcal{K}}
\newcommand{\cL}{\mathcal{L}}
\newcommand{\cM}{\mathcal{M}}
\newcommand{\cN}{\mathcal{N}}
\newcommand{\cO}{\mathcal{O}}
\newcommand{\cP}{\mathcal{P}}
\newcommand{\cQ}{\mathcal{Q}}
\newcommand{\cR}{\mathcal{R}}
\newcommand{\cS}{\mathcal{S}}
\newcommand{\cT}{\mathcal{T}}
\newcommand{\cU}{\mathcal{U}}
\newcommand{\cV}{\mathcal{V}}
\newcommand{\cW}{\mathcal{W}}
\newcommand{\cX}{\mathcal{X}}
\newcommand{\cY}{\mathcal{Y}}
\newcommand{\cZ}{\mathcal{Z}}


%%%%%%%%%%%%%%%%%%%%%%%%%%%%%%%%%%%%%%%%%%%%%
%          Primed Letters
%%%%%%%%%%%%%%%%%%%%%%%%%%%%%%%%%%%%%%%%%%%%%
\newcommand{\aprime}{a^{\prime}}
\newcommand{\bprime}{b^{\prime}}
\newcommand{\cprime}{c^{\prime}}
\newcommand{\dprime}{d^{\prime}}
\newcommand{\eprime}{e^{\prime}}
\newcommand{\fprime}{f^{\prime}}
\newcommand{\gprime}{g^{\prime}}
\newcommand{\hprime}{h^{\prime}}
\newcommand{\iprime}{i^{\prime}}
\newcommand{\jprime}{j^{\prime}}
\newcommand{\kprime}{k^{\prime}}
\newcommand{\lprime}{l^{\prime}}
\newcommand{\mprime}{m^{\prime}}
\newcommand{\nprime}{n^{\prime}}
\newcommand{\oprime}{o^{\prime}}
\newcommand{\pprime}{p^{\prime}}
\newcommand{\qprime}{q^{\prime}}
\newcommand{\rprime}{r^{\prime}}
\newcommand{\sprime}{s^{\prime}}
\newcommand{\tprime}{t^{\prime}}
\newcommand{\uprime}{u^{\prime}}
\newcommand{\vprime}{v^{\prime}}
\newcommand{\wprime}{w^{\prime}}
\newcommand{\xprime}{x^{\prime}}
\newcommand{\yprime}{y^{\prime}}
\newcommand{\zprime}{z^{\prime}}
\newcommand{\Aprime}{A^{\prime}}
\newcommand{\Bprime}{B^{\prime}}
\newcommand{\Cprime}{C^{\prime}}
\newcommand{\Dprime}{D^{\prime}}
\newcommand{\Eprime}{E^{\prime}}
\newcommand{\Fprime}{F^{\prime}}
\newcommand{\Gprime}{G^{\prime}}
\newcommand{\Hprime}{H^{\prime}}
\newcommand{\Iprime}{I^{\prime}}
\newcommand{\Jprime}{J^{\prime}}
\newcommand{\Kprime}{K^{\prime}}
\newcommand{\Lprime}{L^{\prime}}
\newcommand{\Mprime}{M^{\prime}}
\newcommand{\Nprime}{N^{\prime}}
\newcommand{\Oprime}{O^{\prime}}
\newcommand{\Pprime}{P^{\prime}}
\newcommand{\Qprime}{Q^{\prime}}
\newcommand{\Rprime}{R^{\prime}}
\newcommand{\Sprime}{S^{\prime}}
\newcommand{\Tprime}{T^{\prime}}
\newcommand{\Uprime}{U^{\prime}}
\newcommand{\Vprime}{V^{\prime}}
\newcommand{\Wprime}{W^{\prime}}
\newcommand{\Xprime}{X^{\prime}}
\newcommand{\Yprime}{Y^{\prime}}
\newcommand{\Zprime}{Z^{\prime}}
\newcommand{\alphaprime}{\alpha^{\prime}}
\newcommand{\betaprime}{\beta^{\prime}}
\newcommand{\gammaprime}{\gamma^{\prime}}
\newcommand{\Gammaprime}{\Gamma^{\prime}}
\newcommand{\deltaprime}{\delta^{\prime}}
\newcommand{\epsilonprime}{\epsilon^{\prime}}
\newcommand{\kappaprime}{\kappa^{\prime}}
\newcommand{\lambdaprime}{\lambda^{\prime}}
\newcommand{\rhoprime}{\rho^{\prime}}
\newcommand{\Sigmaprime}{\Sigma^{\prime}}
\newcommand{\tauprime}{\tau^{\prime}}
\newcommand{\xiprime}{\xi^{\prime}}
\newcommand{\thetaprime}{\theta^{\prime}}
\newcommand{\Omegaprime}{\Omega^{\prime}}
\newcommand{\cMprime}{\cM^{\prime}}
\newcommand{\cNprime}{\cN^{\prime}}
\newcommand{\cPprime}{\cP^{\prime}}
\newcommand{\cQprime}{\cQ^{\prime}}
\newcommand{\cRprime}{\cR^{\prime}}
\newcommand{\cSprime}{\cS^{\prime}}
\newcommand{\cTprime}{\cT^{\prime}}

%%%%%%%%%%%%%%%%%%%%%%%%%%%%%%%%%%%%%%%%%%%%%
%          bar Letters
%%%%%%%%%%%%%%%%%%%%%%%%%%%%%%%%%%%%%%%%%%%%%
\newcommand{\abar}{\bar{a}}
\newcommand{\bbar}{\bar{b}}
\newcommand{\zbar}{\bar{z}}
\newcommand{\phibar}{\bar{\varphi}}
\newcommand{\psibar}{\bar{\psi}}
\newcommand{\thetabar}{\bar{\theta}}
\newcommand{\nubar}{\bar{\nu}}

%%%%%%%%%%%%%%%%%%%%%%%%%%%%%%%%%%%%%%%%%%%%%
%          star Letters
%%%%%%%%%%%%%%%%%%%%%%%%%%%%%%%%%%%%%%%%%%%%%
\newcommand{\phistar}{\phi^{*}}


%%%%%%%%%%%%%%%%%%%%%%%%%%%%%%%%%%%%%%%%%%%%%
%          Formulas
%%%%%%%%%%%%%%%%%%%%%%%%%%%%%%%%%%%%%%%%%%%%%

\newcommand{\formulaphi}{\text{\large $\varphi$}}
\newcommand{\Formulaphi}{\text{\Large $\varphi$}}


%%%%%%%%%%%%%%%%%%%%%%%%%%%%%%%%%%%%%%%%%%%%%
%          Fraktur Letters
%%%%%%%%%%%%%%%%%%%%%%%%%%%%%%%%%%%%%%%%%%%%%

\newcommand{\fa}{\mathfrak{a}}
\newcommand{\fb}{\mathfrak{b}}
\newcommand{\fc}{\mathfrak{c}}
\newcommand{\fk}{\mathfrak{k}}
\newcommand{\fp}{\mathfrak{p}}
\newcommand{\fq}{\mathfrak{q}}
\newcommand{\fr}{\mathfrak{r}}
\newcommand{\fA}{\mathfrak{A}}
\newcommand{\fB}{\mathfrak{B}}
\newcommand{\fC}{\mathfrak{C}}
\newcommand{\fD}{\mathfrak{D}}

%%%%%%%%%%%%%%%%%%%%%%%%%%%%%%%%%%%%%%%%%%%%%
%          Bold Letters
%%%%%%%%%%%%%%%%%%%%%%%%%%%%%%%%%%%%%%%%%%%%%
\newcommand{\ba}{\mathbf{a}}
\newcommand{\bb}{\mathbf{b}}
\newcommand{\bc}{\mathbf{c}}
\newcommand{\bd}{\mathbf{d}}
\newcommand{\be}{\mathbf{e}}
\newcommand{\bu}{\mathbf{u}}
\newcommand{\bv}{\mathbf{v}}
\newcommand{\bw}{\mathbf{w}}
\newcommand{\bx}{\mathbf{x}}
\newcommand{\by}{\mathbf{y}}
\newcommand{\bz}{\mathbf{z}}
\newcommand{\bSigma}{\boldsymbol{\Sigma}}
\newcommand{\bPi}{\boldsymbol{\Pi}}
\newcommand{\bDelta}{\boldsymbol{\Delta}}
\newcommand{\bdelta}{\boldsymbol{\delta}}
\newcommand{\bgamma}{\boldsymbol{\gamma}}
\newcommand{\bGamma}{\boldsymbol{\Gamma}}

%%%%%%%%%%%%%%%%%%%%%%%%%%%%%%%%%%%%%%%%%%%%%
%         Bold numbers
%%%%%%%%%%%%%%%%%%%%%%%%%%%%%%%%%%%%%%%%%%%%%
\newcommand{\bzero}{\mathbf{0}}

%%%%%%%%%%%%%%%%%%%%%%%%%%%%%%%%%%%%%%%%%%%%%
% Projective-Like Pointclasses
%%%%%%%%%%%%%%%%%%%%%%%%%%%%%%%%%%%%%%%%%%%%%
\newcommand{\Sa}[2][\alpha]{\Sigma_{(#1,#2)}}
\newcommand{\Pa}[2][\alpha]{\Pi_{(#1,#2)}}
\newcommand{\Da}[2][\alpha]{\Delta_{(#1,#2)}}
\newcommand{\Aa}[2][\alpha]{A_{(#1,#2)}}
\newcommand{\Ca}[2][\alpha]{C_{(#1,#2)}}
\newcommand{\Qa}[2][\alpha]{Q_{(#1,#2)}}
\newcommand{\da}[2][\alpha]{\delta_{(#1,#2)}}
\newcommand{\leqa}[2][\alpha]{\leq_{(#1,#2)}}
\newcommand{\lessa}[2][\alpha]{<_{(#1,#2)}}
\newcommand{\equiva}[2][\alpha]{\equiv_{(#1,#2)}}


\newcommand{\Sl}[1]{\Sa[\lambda]{#1}}
\newcommand{\Pl}[1]{\Pa[\lambda]{#1}}
\newcommand{\Dl}[1]{\Da[\lambda]{#1}}
\newcommand{\Al}[1]{\Aa[\lambda]{#1}}
\newcommand{\Cl}[1]{\Ca[\lambda]{#1}}
\newcommand{\Ql}[1]{\Qa[\lambda]{#1}}

\newcommand{\San}{\Sa{n}}
\newcommand{\Pan}{\Pa{n}}
\newcommand{\Dan}{\Da{n}}
\newcommand{\Can}{\Ca{n}}
\newcommand{\Qan}{\Qa{n}}
\newcommand{\Aan}{\Aa{n}}
\newcommand{\dan}{\da{n}}
\newcommand{\leqan}{\leqa{n}}
\newcommand{\lessan}{\lessa{n}}
\newcommand{\equivan}{\equiva{n}}

%%%%%%%%%%%%%%%%%%%%%%%%%%%%%%%%%%%%%%%%%%%%%
% Linear Algebra
%%%%%%%%%%%%%%%%%%%%%%%%%%%%%%%%%%%%%%%%%%%%%
\newcommand{\transpose}[1]{{#1}^{\text{T}}}
\newcommand{\norm}[1]{\lVert{#1}\rVert}
\newcommand\aug{\fboxsep=-\fboxrule\!\!\!\fbox{\strut}\!\!\!}

%%%%%%%%%%%%%%%%%%%%%%%%%%%%%%%%%%%%%%%%%%%%%
% Number Theory
%%%%%%%%%%%%%%%%%%%%%%%%%%%%%%%%%%%%%%%%%%%%%
\DeclareMathOperator{\Spec}{Spec}
\newcommand{\av}[1]{\lvert#1\rvert}
\DeclareMathOperator{\divides}{\mid}
\DeclareMathOperator{\ndivides}{\nmid}


\graphicspath{{images/}}

\newtheorem*{claim}{claim}
\newtheorem*{observation}{Observation}
\newtheorem*{warning}{Warning}
\newtheorem*{question}{Question}
\newtheorem{remark}[theorem]{Remark}

\newenvironment*{subproof}[1][Proof]
{\begin{proof}[#1]}{\renewcommand{\qedsymbol}{$\diamondsuit$} \end{proof}}

\mode<presentation>
{
  \usetheme{Singapore}
  % or ...

  \setbeamercovered{invisible}
  % or whatever (possibly just delete it)
}


\usepackage[english]{babel}
% or whatever

\usepackage[latin1]{inputenc}
% or whatever

\usepackage{times}
\usepackage[T1]{fontenc}
% Or whatever. Note that the encoding and the font should match. If T1
% does not look nice, try deleting the line with the fontenc.

\title{Lesson 16 \\ Applications of Gaussian Elimination}
\subtitle{Math 325, Linear Algebra \\ Fall 2018 \\ SFSU}
\author{Mitch Rudominer}
\date{}



% If you wish to uncover everything in a step-wise fashion, uncomment
% the following command:

\beamerdefaultoverlayspecification{<+->}

\begin{document}

\begin{frame}
  \titlepage
\end{frame}

%%%%%%%%%%%%%%%%%%%%%%%%%%%%%%%%%%%%%%%%%%%%%%%%%%%%%%%%%%%%%%%%%%%%%%%%%

\begin{frame}{General Systems of Linear Equations}

\begin{itemize}
\item Suppose $A\bx=\bb$ is a general system of linear equations.
\item Recall that the possibilities for the solution set are:
\item (i) There are no solutions. The system is incompatible (or inconsistent.)
\item This happens when $\bb\notin\ran(T_A)$.
\item (ii) There is exactly one solution.
\item This happens when $\bb\in\ran(T_A)$  and $T_A$ is one-to-one ( $\ker(A)$ is trivial).
\item (iii) There are infinitely many solutions.
\item This happens when $\bb\in\ran(T_A)$  and $T_A$ is not one-to-one ( $\ker(A)$ is not trivial).
\item If $A$ is in row-echelon form then it is easy to see which of these three cases occur.
\end{itemize}
\end{frame}

%%%%%%%%%%%%%%%%%%%%%%%%%%%%%%%%%%%%%%%%%%%%%%%%%%%%%%%%%%%%%%%%%%%%%%%%%

\begin{frame}{Incompatible systems}

\begin{itemize}
\item Suppose $A\bx=\bb$ is a general system of linear equations, and $A$ is in row-echelon form.
\item Then the system is incompatible iff
\item there is a row of $A$ that is all zeros but the corresponding component of $\bb$ is not zero.
\item Example:
$$
A =
\begin{pmatrix}
1 & 2 & 3 & 4 \\
0 & 0 & 5 & 6 \\
0 & 0 & 0 & 7 \\
0 & 0 & 0 & 0 \\
\end{pmatrix}
\quad
\bb=
\begin{pmatrix}
2 \\ 3 \\ 4 \\ 5
\end{pmatrix}
$$

\end{itemize}
\end{frame}

%%%%%%%%%%%%%%%%%%%%%%%%%%%%%%%%%%%%%%%%%%%%%%%%%%%%%%%%%%%%%%%%%%%%%%%%%

\begin{frame}{Unique solutions}

\begin{itemize}
\item Suppose $A\bx=\bb$ is a general system of linear equations, and $A$ is in row-echelon form.
\item Then the system has a unique solution iff
\item $A$ is  upper triangular with nonzeros on the diagonal,
\item with all zero rows at the bottom allowed as long as the corresponding components of $\bb$ are also zero.
\item Example:
$$
A =
\begin{pmatrix}
1 & 2 & 3 & 4 \\
0 & 5 & 6 & 7 \\
0 & 0 & 8 & 9 \\
0 & 0 & 0 & 1 \\
0 & 0 & 0 & 0 \\
\end{pmatrix}
\quad
\bb=
\begin{pmatrix}
2 \\ 3 \\ 4 \\ 5 \\ 0
\end{pmatrix}
$$
\item We can find the unique solution using back-substitution.
\item We ignore the all-zero rows at the bottom.

\end{itemize}
\end{frame}

%%%%%%%%%%%%%%%%%%%%%%%%%%%%%%%%%%%%%%%%%%%%%%%%%%%%%%%%%%%%%%%%%%%%%%%%
\begin{frame}{Underdetermined systems}

\begin{itemize}
\item Suppose $A\bx=\bb$ is a general system of linear equations, and $A$ is in row-echelon form.
\item Then the system has infinitely many solutions iff
\item neither of the previous two cases occur
\item Example:
$$
A =
\begin{pmatrix}
1 & 2 & 3 & 4 \\
0 & 0 & 5 & 6 \\
0 & 0 & 0 & 8 \\
0 & 0 & 0 & 0 \\
\end{pmatrix}
\quad
\bb=
\begin{pmatrix}
2 \\ 3 \\ 4 \\ 0
\end{pmatrix}
$$
\item We can find the infinitely many solutions through generalized back-substitution.

\end{itemize}
\end{frame}

%%%%%%%%%%%%%%%%%%%%%%%%%%%%%%%%%%%%%%%%%%%%%%%%%%%%%%%%%%%%%%%%%%%%%%%%
\begin{frame}{Basic and Free Variables}

\begin{itemize}
\item Suppose $A\bx=\bb$ is a general system of linear equations, and $A$ is in row-echelon form.
\item The variables corresponding to the pivot columns are called the \emph{basic} variables.
\item The variables corresponding to the non-pivot columns are called the \emph{free} variables.
\item Example:
$$
\begin{pmatrix}
1 & 2 & 3 & 4 \\
0 & 0 & 5 & 6 \\
0 & 0 & 0 & 8 \\
0 & 0 & 0 & 0 \\
\end{pmatrix}
\begin{pmatrix}
x \\ y \\ z \\ w
\end{pmatrix}
=
\begin{pmatrix}
2 \\ 3 \\ 4 \\ 0
\end{pmatrix}
$$
\item The basic variables are $x,z,w$. The free variables are: $y$.
\item The number of basic variables is equal to the rank of the matrix.
\end{itemize}
\end{frame}

%%%%%%%%%%%%%%%%%%%%%%%%%%%%%%%%%%%%%%%%%%%%%%%%%%%%%%%%%%%%%%%%%%%%%%%%
\begin{frame}{Example of an underdetermined system}

\begin{itemize}
\item \textbf{Example} Solve the following general system of linear equations:
\item
$
\begin{pmatrix}
1 & 2 & 2 & 3 \\
2 & 4 & 1 & 3 \\
3 & 6 & 1 & 4
\end{pmatrix}
\begin{pmatrix}
x \\
y \\
z \\
w \\
\end{pmatrix}
=
\begin{pmatrix}
4 \\
5 \\
7
\end{pmatrix}
$
\item Step 1: Form the augmented matrix.
\item
$
\begin{pmatrix}
1 & 2 & 2 & 3  & \aug &  4 \\
2 & 4 & 1 & 3  & \aug &  5 \\
3 & 6 & 1 & 4  & \aug &  7
\end{pmatrix}
$
\item Step 2: Perform Gaussian elimination
\end{itemize}
\end{frame}

%%%%%%%%%%%%%%%%%%%%%%%%%%%%%%%%%%%%%%%%%%%%%%%%%%%%%%%%%%%%%%%%%%%%%%%%
\begin{frame}{Step 2: Perform Gaussian Elimination}

\begin{itemize}
\item Step 1: Form the augmented matrix.
\item Step 2: Perform Gaussian elimination
\item
$
\begin{pmatrix}
1 & 2 & 2 & 3  & \aug &  4 \\
2 & 4 & 1 & 3  & \aug &  5 \\
3 & 6 & 1 & 4  & \aug &  7
\end{pmatrix}
$
\item
$
\SkipImplies
\begin{pmatrix}
1 & 2 & 2 &   3  & \aug &  4 \\
0 & 0 & -3 & -3  & \aug & -3 \\
0 & 0 & -5 & -5  & \aug & -5
\end{pmatrix}
$
\item
$
\SkipImplies
\begin{pmatrix}
1 & 2 &  2 &  3 &   \aug &  4 \\
0 & 0 & -3 & -3 &   \aug & -3 \\
0 & 0 &  0 &  0 &   \aug & 0
\end{pmatrix}
$
\item Step 3. Perform generalized back-substitution.
\end{itemize}
\end{frame}



%%%%%%%%%%%%%%%%%%%%%%%%%%%%%%%%%%%%%%%%%%%%%%%%%%%%%%%%%%%%%%%%%%%%%%%%
\begin{frame}{Generalized back-substitution}

\begin{itemize}
\item Generalized back-substitution is similar to simple back-substitution, except that we solve for the basic variable \emph{in terms of} the free variables.
\item
$
\begin{pmatrix}
1 & 2 & 2 & 3 \\
0 & 0 & -3 & -3 \\
0 & 0 & 0 & 0
\end{pmatrix}
\begin{pmatrix}
x \\ y \\ z \\ w
\end{pmatrix}
=
\begin{pmatrix}
4 \\ -3 \\  0
\end{pmatrix}
$
\item The basic variables are $x$ and $z$. The free variables are $y$ and $w$.
\item $-3z -3w = -3 \SkipImplies -3z = -3 + 3w $
\item $ \SkipImplies z = 1 - w$.
\item $x + 2y +2z +3w = 4 $
\item $\SkipImplies x +2y + 2(1-w) + 3w = 4$
\item $\SkipImplies x + 2y +2 +w = 4 \SkipImplies x = 2 -2y - w$.
\end{itemize}
\end{frame}


%%%%%%%%%%%%%%%%%%%%%%%%%%%%%%%%%%%%%%%%%%%%%%%%%%%%%%%%%%%%%%%%%%%%%%%%
\begin{frame}{General solution}

\begin{itemize}
\item
Original problem: $
\begin{pmatrix}
1 & 2 & 2 & 3 \\
2 & 4 & 1 & 3 \\
3 & 6 & 1 & 4
\end{pmatrix}
\begin{pmatrix}
x \\
y \\
z \\
w \\
\end{pmatrix}
=
\begin{pmatrix}
4 \\
5 \\
7
\end{pmatrix}
$
\item
General solution: $
\begin{pmatrix}
x \\ y \\ z \\ w
\end{pmatrix}
=
\begin{pmatrix}
2 -2y - w \\
y \\
1-w \\
w
\end{pmatrix}
$
\item
$
\begin{pmatrix}
x \\ y \\ z \\ w
\end{pmatrix}
=
\begin{pmatrix}
2 \\
0 \\
1 \\
0
\end{pmatrix}
+
y
\begin{pmatrix}
-2 \\
1 \\
0 \\
0
\end{pmatrix}
+
w
\begin{pmatrix}
-1 \\
0 \\
-1 \\
1
\end{pmatrix}
$
\item The free variables are \emph{parameters} to the solution set.
\item As the free variables range over all real numbers, we get all solutions.
\end{itemize}
\end{frame}

%%%%%%%%%%%%%%%%%%%%%%%%%%%%%%%%%%%%%%%%%%%%%%%%%%%%%%%%%%%%%%%%%%%%%%%%
\begin{frame}{Particular Solution}

\begin{itemize}
\item
Original problem: $
\begin{pmatrix}
1 & 2 & 2 & 3 \\
2 & 4 & 1 & 3 \\
3 & 6 & 1 & 4
\end{pmatrix}
\begin{pmatrix}
x \\
y \\
z \\
w \\
\end{pmatrix}
=
\begin{pmatrix}
4 \\
5 \\
7
\end{pmatrix}
$
\item
General solution: $
\begin{pmatrix}
x \\ y \\ z \\ w
\end{pmatrix}
=
\begin{pmatrix}
2 \\
0 \\
1 \\
0
\end{pmatrix}
+
y
\begin{pmatrix}
-2 \\
1 \\
0 \\
0
\end{pmatrix}
+
w
\begin{pmatrix}
-1 \\
0 \\
-1 \\
1
\end{pmatrix}
$
\item
$
\begin{pmatrix}
x \\ y \\ z \\ w
\end{pmatrix}
=
\begin{pmatrix}
2 \\
0 \\
1 \\
0
\end{pmatrix}
$
is a \emph{particular solution}.

\end{itemize}
\end{frame}

%%%%%%%%%%%%%%%%%%%%%%%%%%%%%%%%%%%%%%%%%%%%%%%%%%%%%%%%%%%%%%%%%%%%%%%%
\begin{frame}{Particular solution plus a subspace}

\begin{itemize}
\item
Original problem: $
\begin{pmatrix}
1 & 2 & 2 & 3 \\
2 & 4 & 1 & 3 \\
3 & 6 & 1 & 4
\end{pmatrix}
\begin{pmatrix}
x \\
y \\
z \\
w \\
\end{pmatrix}
=
\begin{pmatrix}
4 \\
5 \\
7
\end{pmatrix}
$
\item
General solution: $
\begin{pmatrix}
x \\ y \\ z \\ w
\end{pmatrix}
=
\begin{pmatrix}
2 \\
0 \\
1 \\
0
\end{pmatrix}
+
y
\begin{pmatrix}
-2 \\
1 \\
0 \\
0
\end{pmatrix}
+
w
\begin{pmatrix}
-1 \\
0 \\
-1 \\
1
\end{pmatrix}
$
\item Every other solution is obtained by adding to the particular solution an element of the subspace of $\R^4$ spanned by the vectors
\item
$
\singleton{
\begin{pmatrix}
-2 \\
1 \\
0 \\
0
\end{pmatrix}
,
\begin{pmatrix}
-1 \\
0 \\
-1 \\
1
\end{pmatrix}
}
$

\end{itemize}
\end{frame}

%%%%%%%%%%%%%%%%%%%%%%%%%%%%%%%%%%%%%%%%%%%%%%%%%%%%%%%%%%%%%%%%%%%%%%%%
\begin{frame}{Homogeneous systems and null spaces}
\begin{itemize}
\item Find the kernel of the matrix
$A=
\begin{pmatrix}
1 & 2 & 2 & 3 \\
2 & 4 & 1 & 3 \\
3 & 6 & 1 & 4
\end{pmatrix}
$
\item Solution: This is the same as solving the homogeneous system of equations
\item $A\bx=\bzero$.
\item Technique: Form an augmented matrix.
\item Use Gaussian elimination to bring $A$ to row-echelon form.
\item Do back-substitution to get the general solution.
\item (When solving a homogeneous system we don't really need to bother with the
 augmented matrix since the right-hand-side
of the equation consists of all zeros)
\end{itemize}
\end{frame}

%%%%%%%%%%%%%%%%%%%%%%%%%%%%%%%%%%%%%%%%%%%%%%%%%%%%%%%%%%%%%%%%%%%%%%%%
\begin{frame}{Solving a homogeneous system}
\begin{itemize}
\item Solve
$
\begin{pmatrix}
1 & 2 & 2 & 3 \\
2 & 4 & 1 & 3 \\
3 & 6 & 1 & 4
\end{pmatrix}
\begin{pmatrix}
x \\ y \\ z \\ w
\end{pmatrix}=
\begin{pmatrix}
0 \\ 0 \\ 0
\end{pmatrix}
$
\item
$
\begin{pmatrix}
1 & 2 & 2 & 3 & \aug & 0 \\
2 & 4 & 1 & 3 & \aug & 0  \\
3 & 6 & 1 & 4 & \aug & 0
\end{pmatrix}
\SkipImplies
\begin{pmatrix}
1 & 2 & 2 & 3 & \aug & 0  \\
0 & 0 & -3 & -3  & \aug & 0  \\
0 & 0 & -5 & -5 & \aug & 0
\end{pmatrix}
\SkipImplies
\begin{pmatrix}
1 & 2 & 2 & 3    & \aug & 0  \\
0 & 0 & -3 & -3  & \aug & 0  \\
0 & 0 & 0 & 0    & \aug & 0
\end{pmatrix}
$
\item This is the same row-echelon form matrix we were just considering.
\end{itemize}
\end{frame}


%%%%%%%%%%%%%%%%%%%%%%%%%%%%%%%%%%%%%%%%%%%%%%%%%%%%%%%%%%%%%%%%%%%%%%%%
\begin{frame}{Solving a homogeneous system}

\begin{itemize}
\item We have reduced the homogeneous system to
\item
$
\begin{pmatrix}
1 & 2 & 2 & 3 \\
0 & 0 & -3 & -3 \\
0 & 0 & 0 & 0
\end{pmatrix}
\begin{pmatrix}
x \\ y \\ z \\ w
\end{pmatrix}
=
\begin{pmatrix}
0 \\ 0 \\  0
\end{pmatrix}
$
\item The basic variables are $x$ and $z$. The free variables are $y$ and $w$.
\item $-3z -3w = 0 \SkipImplies -3z = 3w $
\item $ \SkipImplies z = - w$.
\item $x + 2y +2z +3w = 0 $
\item $\SkipImplies x +2y + 2(-w) + 3w = 0$
\item $\SkipImplies x + 2y  + w = 0 \SkipImplies x = -2y - w$.
\end{itemize}
\end{frame}

%%%%%%%%%%%%%%%%%%%%%%%%%%%%%%%%%%%%%%%%%%%%%%%%%%%%%%%%%%%%%%%%%%%%%%%%
\begin{frame}{Finding the null space}

\begin{itemize}
\item The general solution to the homogeneous system is
\item
$
\begin{pmatrix}
x \\ y \\ z \\ w
\end{pmatrix}
=
\begin{pmatrix}
-2y - w \\
y \\
-w \\
w
\end{pmatrix}
$
\item
$
\begin{pmatrix}
x \\ y \\ z \\ w
\end{pmatrix}
=
y
\begin{pmatrix}
-2 \\
1 \\
0 \\
0
\end{pmatrix}
+
w
\begin{pmatrix}
-1 \\
0 \\
-1 \\
1
\end{pmatrix}
$
\item The kernel of $A$ is the subspace of $\R^4$ spanned by:
$
\singleton{
\begin{pmatrix}
-2 \\
1 \\
0 \\
0
\end{pmatrix}
,
\begin{pmatrix}
-1 \\
0 \\
-1 \\
1
\end{pmatrix}
}
$
\end{itemize}
\end{frame}

%%%%%%%%%%%%%%%%%%%%%%%%%%%%%%%%%%%%%%%%%%%%%%%%%%%%%%%%%%%%%%%%%%%%%%%%
\begin{frame}{Solution to a general system}

\begin{itemize}
\item In an earlier lesson we learned the following:
\item \textbf{Theorem} Let $A\bx=\bb$ be a general system of linear equations.
\item Let $\bx_0$ be a particular solution. $A\bx_0 = \bb$.
\item Then the general solution set is
\item $\setof{\bx_0 + \bz}{\bz\in\ker(A)}$.
\item
\textbf{Example.} $
\begin{pmatrix}
1 & 2 & 2 & 3 \\
2 & 4 & 1 & 3 \\
3 & 6 & 1 & 4
\end{pmatrix}
\begin{pmatrix}
x \\
y \\
z \\
w \\
\end{pmatrix}
=
\begin{pmatrix}
4 \\
5 \\
7
\end{pmatrix}
$
\item
General solution: $
\begin{pmatrix}
x \\ y \\ z \\ w
\end{pmatrix}
=
\begin{pmatrix}
2 \\
0 \\
1 \\
0
\end{pmatrix}
+
y
\begin{pmatrix}
-2 \\
1 \\
0 \\
0
\end{pmatrix}
+
w
\begin{pmatrix}
-1 \\
0 \\
-1 \\
1
\end{pmatrix}
$
\item This is of the form $\setof{\bx_0 + \bz}{\bz\in\ker(A)}$.
\end{itemize}
\end{frame}
%%%%%%%%%%%%%%%%%%%%%%%%%%%%%%%%%%%%%%%%%%%%%%%%%%%%%%%%%%%%%%%%%%%%%%%%
\begin{frame}{Fundamental Theorem of Linear Algebra (Part 1)}
\begin{itemize}
\item \textbf{Theorem.} Let $A$ be an $m\times n$ matrix of rank $r$.
\item Then $\dim(\ran(A)) = r$, and
\item $\dim(\ker(A)) = n-r$.
\item Example. Let
 $ A =
\begin{pmatrix}
1 & 2 & 2 & 3 \\
2 & 4 & 1 & 3 \\
3 & 6 & 1 & 4
\end{pmatrix}
$
\item Then as we saw earlier, $A$ has two basic columns. (The row-echelon form has 2 pivot columns.)
\item $\rank(A) = 2$. So $\dim(\ran(A)) = 2$.
\item $\dim(\ker(A)) = 4 - 2 =2$.
\end{itemize}
\end{frame}

%%%%%%%%%%%%%%%%%%%%%%%%%%%%%%%%%%%%%%%%%%%%%%%%%%%%%%%%%%%%%%%%%%%%%%%%
\begin{frame}{Finding a basis for the null space}

\begin{itemize}
\item Let
$A=
\begin{pmatrix}
1 & 2 & 2 & 3 \\
2 & 4 & 1 & 3 \\
3 & 6 & 1 & 4
\end{pmatrix}
$
\item We saw before that the kernel of $A$ is the subspace of $\R^4$ spanned by:
$
\singleton{
\begin{pmatrix}
-2 \\
1 \\
0 \\
0
\end{pmatrix}
,
\begin{pmatrix}
-1 \\
0 \\
-1 \\
1
\end{pmatrix}
}
$
\item Those two vectors span $\ker(A)$.
\item Now we know more: Those two vectors form a basis for the kernel.
\item We know this because we know that $\dim(\ker(A)) = 2$.
\end{itemize}
\end{frame}

%%%%%%%%%%%%%%%%%%%%%%%%%%%%%%%%%%%%%%%%%%%%%%%%%%%%%%%%%%%%%%%%%%%%%%%%
\begin{frame}{Proof of the fundamental theorem}
\begin{itemize}
\item \textbf{Proof} Let $A$ be an $m\times n$ matrix of rank $r$.
\item We must show that $\dim(\ker(A)) = n -r$.
\item $A$ has $r$ basic variables and $n-r$  free variables.
\item Using the generalized back-substitution technique we showed above
how to find the general solution to the homogeneous system $A\bx=\bzero$.
\item The general solution is expressed as the span of $n-r$ vectors, one for
each free variable.
\item So the only thing we need to prove is that the $n-r$ vectors are
linearly independent.
\item This follows from the fact if the $i$-th variable is free then the
vector corresponding to the $i$-th variable has a 1 in position $i$ and all
of the other vectors have a 0 in position $i$. $\qed$.
\end{itemize}
\end{frame}

%%%%%%%%%%%%%%%%%%%%%%%%%%%%%%%%%%%%%%%%%%%%%%%%%%%%%%%%%%%%%%%%%%%%%%%%
\begin{frame}{Example of proof.}
\begin{itemize}
\item For example, if
$A=
\begin{pmatrix}
1 & 2 & 2 & 3 \\
2 & 4 & 1 & 3 \\
3 & 6 & 1 & 4
\end{pmatrix}
$
\item The general solution to the homogeneous system $A\bx=\bzero$ is
\item
$
\begin{pmatrix}
x \\ y \\ z \\ w
\end{pmatrix}
=
\begin{pmatrix}
-2y - w \\
y \\
-w \\
w
\end{pmatrix}
=
y
\begin{pmatrix}
-2 \\
1 \\
0 \\
0
\end{pmatrix}
+
w
\begin{pmatrix}
-1 \\
0 \\
-1 \\
1
\end{pmatrix}
$
\item The free variables are $y$ and $w$ in positions 2 and 4.
\item The first vector has a 1 in position 2 and a 0 in position 4.
\item The second vector has a 0 in position 2 and a 1 in position 4.
\item So the two vectors are linearly independent.
\end{itemize}
\end{frame}

%%%%%%%%%%%%%%%%%%%%%%%%%%%%%%%%%%%%%%%%%%%%%%%%%%%%%%%%%%%%%%%%%%%%%%%%
\begin{frame}{Fundamental Theorem of Linear Algebra (More formal version)}
\begin{itemize}
\item \textbf{Theorem.} Let $V$ be an $n$-dimensional vector space.
\item Let $T:V\map W$ be a linear transformation.
\item Let $r=\dim(\ran(T))$.
\item Then $\dim(\ker(T)) = n - r$.
\item i.e. $\dim(\ker(T)) + \dim(\ran(T)) = \dim(V)$.
\item \textbf{proof.} Let $\bv_1,\bv_2,\cdots,\bv_s$ be a basis for $\ker(T)$.
\item Let $\bu_1,\bu_2,\cdots,\bu_t$ be additional vectors so that
\item $\bv_1,\bv_2,\cdots,\bv_s,\bu_1,\bu_2,\cdots,\bu_t$ are a basis for $V$.
\item So $s+t = n$.
\item It suffices to show that $T(u_1), T(u_2),\cdots, T(u_t)$ forms a basis for $\ran(T)$.
\item Because then $t=r$ and $\dim(\ker(T)) = s = n-r$.
\end{itemize}
\end{frame}

%%%%%%%%%%%%%%%%%%%%%%%%%%%%%%%%%%%%%%%%%%%%%%%%%%%%%%%%%%%%%%%%%%%%%%%%
\begin{frame}{proof continued.}
\begin{itemize}
\item Since every element of $V$ is in the span of $\bv_1,\bv_2,\cdots,\bv_s,\bu_1,\bu_2,\cdots,\bu_t$,
\item every element of $\ran(T)$ is in the span of $T(\bv_1),T(\bv_2),\cdots,T(\bv_s), T(\bu_1),T(\bu_2),\cdots,T(\bu_t)$.
\item But $T(\bv_1) = T(\bv_2) = \cdots = T(\bv_s) = 0$.
\item So every element of $\ran(T)$ is in the span of $T(\bu_1),T(\bu_2),\cdots,T(\bu_t)$.
\item Suppose $c_1 T(\bu_1) + c_2 T(\bu_2) + \cdots + c_t T(\bu_t) = \bzero.$
\item Then $T(c_1\bu_1 + c_2 \bu_2 + \cdots + c_t\bu_t) = \bzero$.
\item So $c_1\bu_1 + c_2 \bu_2 + \cdots + c_t\bu_t \in \ker(T)$.
\item So $c_1\bu_1 + c_2 \bu_2 + \cdots + c_t\bu_t$ is in the span of $\bv_1,\bv_2,\cdots,\bv_s$.
\item But $\bv_1,\bv_2,\cdots,\bv_s,\bu_1,\bu_2,\cdots,\bu_t$ are linearly independent.
\item So $c_1\bu_1 + c_2 \bu_2 + \cdots + c_t\bu_t = \bzero$.
\item So $c_1 = c_2 = \cdots = c_t = 0$.
\item So $T(\bu_1), T(\bu_2),\cdots, T(\bu_t)$  are linearly independent. $\qed$
\end{itemize}
\end{frame}


%%%%%%%%%%%%%%%%%%%%%%%%%%%%%%%%%%%%%%%%%%%%%%%%%%%%%%%%%%%%%%%%%%%%%%%%
\begin{frame}{Checking the span}
\begin{itemize}
\item Determine whether or not the vector
$\bb= \transpose{(1,1,1)}$
is in the span of the vectors
$
\begin{pmatrix}
1 \\
2 \\
3
\end{pmatrix},
\begin{pmatrix}
2\\
4\\
6
\end{pmatrix},
\begin{pmatrix}
2\\
1\\
1
\end{pmatrix},
\begin{pmatrix}
3 \\
3 \\
4
\end{pmatrix}
$
\item Solution: This is the same as asking whether or not the system $A\bx=\bb$
is compatible
\item where $A$ is the $3\times 4$ matrix whose columns are the four vectors given
above.
\end{itemize}
\end{frame}

%%%%%%%%%%%%%%%%%%%%%%%%%%%%%%%%%%%%%%%%%%%%%%%%%%%%%%%%%%%%%%%%%%%%%%%%
\begin{frame}{Checking the span, continued}
\begin{itemize}
\item Step 1: Form the augmented matrix.
\item
$
\begin{pmatrix}
1 & 2 & 2 & 3  & \aug &  1 \\
2 & 4 & 1 & 3  & \aug &  1 \\
3 & 6 & 1 & 4  & \aug &  1
\end{pmatrix}
$
\item Step 2: Perform Gaussian elimination
\item
$
\SkipImplies
\begin{pmatrix}
1 & 2 & 2 &   3  & \aug &  1 \\
0 & 0 & -3 & -3  & \aug & -1 \\
0 & 0 & -5 & -5  & \aug & -2
\end{pmatrix}
$
\item
$
\SkipImplies
\begin{pmatrix}
1 & 2 &  2 &  3 &   \aug &  1 \\
0 & 0 & -3 & -3 &   \aug & -1 \\
0 & 0 &  0 &  0 &   \aug & -\frac{1}{3}
\end{pmatrix}
$
\item This system is incompatible.
\item so $\bb$ is not in the span of the four original columns.
\end{itemize}
\end{frame}

%%%%%%%%%%%%%%%%%%%%%%%%%%%%%%%%%%%%%%%%%%%%%%%%%%%%%%%%%%%%%%%%%%%%%%%%
\begin{frame}{More examples in the homework}
\begin{itemize}
\item Do the homework for lesson 16.
\item There are more examples of applications of Gaussian elimination.
\end{itemize}
\end{frame}






\end{document}


