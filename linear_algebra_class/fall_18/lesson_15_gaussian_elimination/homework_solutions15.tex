\documentclass[oneside,12pt]{amsart}

\usepackage{amsmath,amssymb,latexsym,eucal,amsthm,graphicx}

%%%%%%%%%%%%%%%%%%%%%%%%%%%%%%%%%%%%%%%%%%%%%
% Common Set Theory Constructs
%%%%%%%%%%%%%%%%%%%%%%%%%%%%%%%%%%%%%%%%%%%%%

\newcommand{\setof}[2]{\left\{ \, #1 \, \left| \, #2 \, \right.\right\}}
\newcommand{\lsetof}[2]{\left\{\left. \, #1 \, \right| \, #2 \,  \right\}}
\newcommand{\bigsetof}[2]{\bigl\{ \, #1 \, \bigm | \, #2 \,\bigr\}}
\newcommand{\Bigsetof}[2]{\Bigl\{ \, #1 \, \Bigm | \, #2 \,\Bigr\}}
\newcommand{\biggsetof}[2]{\biggl\{ \, #1 \, \biggm | \, #2 \,\biggr\}}
\newcommand{\Biggsetof}[2]{\Biggl\{ \, #1 \, \Biggm | \, #2 \,\Biggr\}}
\newcommand{\dotsetof}[2]{\left\{ \, #1 \, : \, #2 \, \right\}}
\newcommand{\bigdotsetof}[2]{\bigl\{ \, #1 \, : \, #2 \,\bigr\}}
\newcommand{\Bigdotsetof}[2]{\Bigl\{ \, #1 \, \Bigm : \, #2 \,\Bigr\}}
\newcommand{\biggdotsetof}[2]{\biggl\{ \, #1 \, \biggm : \, #2 \,\biggr\}}
\newcommand{\Biggdotsetof}[2]{\Biggl\{ \, #1 \, \Biggm : \, #2 \,\Biggr\}}
\newcommand{\sequence}[2]{\left\langle \, #1 \,\left| \, #2 \, \right. \right\rangle}
\newcommand{\lsequence}[2]{\left\langle\left. \, #1 \, \right| \,#2 \,  \right\rangle}
\newcommand{\bigsequence}[2]{\bigl\langle \,#1 \, \bigm | \, #2 \, \bigr\rangle}
\newcommand{\Bigsequence}[2]{\Bigl\langle \,#1 \, \Bigm | \, #2 \, \Bigr\rangle}
\newcommand{\biggsequence}[2]{\biggl\langle \,#1 \, \biggm | \, #2 \, \biggr\rangle}
\newcommand{\Biggsequence}[2]{\Biggl\langle \,#1 \, \Biggm | \, #2 \, \Biggr\rangle}
\newcommand{\singleton}[1]{\left\{#1\right\}}
\newcommand{\angles}[1]{\left\langle #1 \right\rangle}
\newcommand{\bigangles}[1]{\bigl\langle #1 \bigr\rangle}
\newcommand{\Bigangles}[1]{\Bigl\langle #1 \Bigr\rangle}
\newcommand{\biggangles}[1]{\biggl\langle #1 \biggr\rangle}
\newcommand{\Biggangles}[1]{\Biggl\langle #1 \Biggr\rangle}


\newcommand{\force}[1]{\Vert\!\underset{\!\!\!\!\!#1}{\!\!\!\relbar\!\!\!%
\relbar\!\!\relbar\!\!\relbar\!\!\!\relbar\!\!\relbar\!\!\relbar\!\!\!%
\relbar\!\!\relbar\!\!\relbar}}
\newcommand{\longforce}[1]{\Vert\!\underset{\!\!\!\!\!#1}{\!\!\!\relbar\!\!\!%
\relbar\!\!\relbar\!\!\relbar\!\!\!\relbar\!\!\relbar\!\!\relbar\!\!\!%
\relbar\!\!\relbar\!\!\relbar\!\!\relbar\!\!\relbar\!\!\relbar\!\!\relbar\!\!\relbar}}
\newcommand{\nforce}[1]{\Vert\!\underset{\!\!\!\!\!#1}{\!\!\!\relbar\!\!\!%
\relbar\!\!\relbar\!\!\relbar\!\!\!\relbar\!\!\relbar\!\!\relbar\!\!\!%
\relbar\!\!\not\relbar\!\!\relbar}}
\newcommand{\forcein}[2]{\overset{#2}{\Vert\underset{\!\!\!\!\!#1}%
{\!\!\!\relbar\!\!\!\relbar\!\!\relbar\!\!\relbar\!\!\!\relbar\!\!\relbar\!%
\!\relbar\!\!\!\relbar\!\!\relbar\!\!\relbar\!\!\relbar\!\!\!\relbar\!\!%
\relbar\!\!\relbar}}}

\newcommand{\pre}[2]{{}^{#2}{#1}}

\newcommand{\restr}{\!\!\upharpoonright\!}

%%%%%%%%%%%%%%%%%%%%%%%%%%%%%%%%%%%%%%%%%%%%%
% Set-Theoretic Connectives
%%%%%%%%%%%%%%%%%%%%%%%%%%%%%%%%%%%%%%%%%%%%%

\newcommand{\intersect}{\cap}
\newcommand{\union}{\cup}
\newcommand{\Intersection}[1]{\bigcap\limits_{#1}}
\newcommand{\Union}[1]{\bigcup\limits_{#1}}
\newcommand{\adjoin}{{}^\frown}
\newcommand{\forces}{\Vdash}

%%%%%%%%%%%%%%%%%%%%%%%%%%%%%%%%%%%%%%%%%%%%%
% Miscellaneous
%%%%%%%%%%%%%%%%%%%%%%%%%%%%%%%%%%%%%%%%%%%%%
\newcommand{\defeq}{=_{\text{def}}}
\newcommand{\Turingleq}{\leq_{\text{T}}}
\newcommand{\Turingless}{<_{\text{T}}}
\newcommand{\lexleq}{\leq_{\text{lex}}}
\newcommand{\lexless}{<_{\text{lex}}}
\newcommand{\Turingequiv}{\equiv_{\text{T}}}
\newcommand{\isomorphic}{\cong}

%%%%%%%%%%%%%%%%%%%%%%%%%%%%%%%%%%%%%%%%%%%%%
% Constants
%%%%%%%%%%%%%%%%%%%%%%%%%%%%%%%%%%%%%%%%%%%%%
\newcommand{\R}{\mathbb{R}}
\renewcommand{\P}{\mathbb{P}}
\newcommand{\Q}{\mathbb{Q}}
\newcommand{\Z}{\mathbb{Z}}
\newcommand{\Zpos}{\Z^{+}}
\newcommand{\Znonneg}{\Z^{\geq 0}}
\newcommand{\C}{\mathbb{C}}
\newcommand{\N}{\mathbb{N}}
\newcommand{\B}{\mathbb{B}}
\newcommand{\Bairespace}{\pre{\omega}{\omega}}
\newcommand{\LofR}{L(\R)}
\newcommand{\JofR}[1]{J_{#1}(\R)}
\newcommand{\SofR}[1]{S_{#1}(\R)}
\newcommand{\JalphaR}{\JofR{\alpha}}
\newcommand{\JbetaR}{\JofR{\beta}}
\newcommand{\JlambdaR}{\JofR{\lambda}}
\newcommand{\SalphaR}{\SofR{\alpha}}
\newcommand{\SbetaR}{\SofR{\beta}}
\newcommand{\Pkl}{\mathcal{P}_{\kappa}(\lambda)}
\DeclareMathOperator{\con}{con}
\DeclareMathOperator{\ORD}{OR}
\DeclareMathOperator{\Ord}{OR}
\DeclareMathOperator{\WO}{WO}
\DeclareMathOperator{\OD}{OD}
\DeclareMathOperator{\HOD}{HOD}
\DeclareMathOperator{\HC}{HC}
\DeclareMathOperator{\WF}{WF}
\DeclareMathOperator{\wfp}{wfp}
\DeclareMathOperator{\HF}{HF}
\newcommand{\One}{I}
\newcommand{\ONE}{I}
\newcommand{\Two}{II}
\newcommand{\TWO}{II}
\newcommand{\Mladder}{M^{\text{ld}}}

%%%%%%%%%%%%%%%%%%%%%%%%%%%%%%%%%%%%%%%%%%%%%
% Commutative Algebra Constants
%%%%%%%%%%%%%%%%%%%%%%%%%%%%%%%%%%%%%%%%%%%%%
\DeclareMathOperator{\dottimes}{\dot{\times}}
\DeclareMathOperator{\dotminus}{\dot{-}}
\DeclareMathOperator{\Spec}{Spec}

%%%%%%%%%%%%%%%%%%%%%%%%%%%%%%%%%%%%%%%%%%%%%
% Theories
%%%%%%%%%%%%%%%%%%%%%%%%%%%%%%%%%%%%%%%%%%%%%
\DeclareMathOperator{\ZFC}{ZFC}
\DeclareMathOperator{\ZF}{ZF}
\DeclareMathOperator{\AD}{AD}
\DeclareMathOperator{\ADR}{AD_{\R}}
\DeclareMathOperator{\KP}{KP}
\DeclareMathOperator{\PD}{PD}
\DeclareMathOperator{\CH}{CH}
\DeclareMathOperator{\GCH}{GCH}
\DeclareMathOperator{\HPC}{HPC} % HOD pair capturing
%%%%%%%%%%%%%%%%%%%%%%%%%%%%%%%%%%%%%%%%%%%%%
% Iteration Trees
%%%%%%%%%%%%%%%%%%%%%%%%%%%%%%%%%%%%%%%%%%%%%

\newcommand{\pred}{\text{-pred}}

%%%%%%%%%%%%%%%%%%%%%%%%%%%%%%%%%%%%%%%%%%%%%%%%
% Operator Names
%%%%%%%%%%%%%%%%%%%%%%%%%%%%%%%%%%%%%%%%%%%%%%%%
\DeclareMathOperator{\Det}{Det}
\DeclareMathOperator{\dom}{dom}
\DeclareMathOperator{\ran}{ran}
\DeclareMathOperator{\range}{ran}
\DeclareMathOperator{\image}{image}
\DeclareMathOperator{\crit}{crit}
\DeclareMathOperator{\card}{card}
\DeclareMathOperator{\cf}{cf}
\DeclareMathOperator{\cof}{cof}
\DeclareMathOperator{\rank}{rank}
\DeclareMathOperator{\ot}{o.t.}
\DeclareMathOperator{\ords}{o}
\DeclareMathOperator{\ro}{r.o.}
\DeclareMathOperator{\rud}{rud}
\DeclareMathOperator{\Powerset}{\mathcal{P}}
\DeclareMathOperator{\length}{lh}
\DeclareMathOperator{\lh}{lh}
\DeclareMathOperator{\limit}{lim}
\DeclareMathOperator{\fld}{fld}
\DeclareMathOperator{\projection}{p}
\DeclareMathOperator{\Ult}{Ult}
\DeclareMathOperator{\ULT}{Ult}
\DeclareMathOperator{\Coll}{Coll}
\DeclareMathOperator{\Col}{Col}
\DeclareMathOperator{\Hull}{Hull}
\DeclareMathOperator*{\dirlim}{dir lim}
\DeclareMathOperator{\Scale}{Scale}
\DeclareMathOperator{\supp}{supp}
\DeclareMathOperator{\trancl}{tran.cl.}
\DeclareMathOperator{\trace}{Tr}
\DeclareMathOperator{\diag}{diag}
\DeclareMathOperator{\spn}{span}
\DeclareMathOperator{\sgn}{sgn}
%%%%%%%%%%%%%%%%%%%%%%%%%%%%%%%%%%%%%%%%%%%%%
% Logical Connectives
%%%%%%%%%%%%%%%%%%%%%%%%%%%%%%%%%%%%%%%%%%%%%
\newcommand{\IImplies}{\Longrightarrow}
\newcommand{\SkipImplies}{\quad\Longrightarrow\quad}
\newcommand{\Ifff}{\Longleftrightarrow}
\newcommand{\iimplies}{\longrightarrow}
\newcommand{\ifff}{\longleftrightarrow}
\newcommand{\Implies}{\Rightarrow}
\newcommand{\Iff}{\Leftrightarrow}
\renewcommand{\implies}{\rightarrow}
\renewcommand{\iff}{\leftrightarrow}
\newcommand{\AND}{\wedge}
\newcommand{\OR}{\vee}
\newcommand{\st}{\text{ s.t. }}
\newcommand{\Or}{\text{ or }}

%%%%%%%%%%%%%%%%%%%%%%%%%%%%%%%%%%%%%%%%%%%%%
% Function Arrows
%%%%%%%%%%%%%%%%%%%%%%%%%%%%%%%%%%%%%%%%%%%%%

\newcommand{\injection}{\xrightarrow{\text{1-1}}}
\newcommand{\surjection}{\xrightarrow{\text{onto}}}
\newcommand{\bijection}{\xrightarrow[\text{onto}]{\text{1-1}}}
\newcommand{\cofmap}{\xrightarrow{\text{cof}}}
\newcommand{\map}{\rightarrow}

%%%%%%%%%%%%%%%%%%%%%%%%%%%%%%%%%%%%%%%%%%%%%
% Mouse Comparison Operators
%%%%%%%%%%%%%%%%%%%%%%%%%%%%%%%%%%%%%%%%%%%%%
\newcommand{\initseg}{\trianglelefteq}
\newcommand{\properseg}{\lhd}
\newcommand{\notinitseg}{\ntrianglelefteq}
\newcommand{\notproperseg}{\ntriangleleft}

%%%%%%%%%%%%%%%%%%%%%%%%%%%%%%%%%%%%%%%%%%%%%
%           calligraphic letters
%%%%%%%%%%%%%%%%%%%%%%%%%%%%%%%%%%%%%%%%%%%%%
\newcommand{\cA}{\mathcal{A}}
\newcommand{\cB}{\mathcal{B}}
\newcommand{\cC}{\mathcal{C}}
\newcommand{\cD}{\mathcal{D}}
\newcommand{\cE}{\mathcal{E}}
\newcommand{\cF}{\mathcal{F}}
\newcommand{\cG}{\mathcal{G}}
\newcommand{\cH}{\mathcal{H}}
\newcommand{\cI}{\mathcal{I}}
\newcommand{\cJ}{\mathcal{J}}
\newcommand{\cK}{\mathcal{K}}
\newcommand{\cL}{\mathcal{L}}
\newcommand{\cM}{\mathcal{M}}
\newcommand{\cN}{\mathcal{N}}
\newcommand{\cO}{\mathcal{O}}
\newcommand{\cP}{\mathcal{P}}
\newcommand{\cQ}{\mathcal{Q}}
\newcommand{\cR}{\mathcal{R}}
\newcommand{\cS}{\mathcal{S}}
\newcommand{\cT}{\mathcal{T}}
\newcommand{\cU}{\mathcal{U}}
\newcommand{\cV}{\mathcal{V}}
\newcommand{\cW}{\mathcal{W}}
\newcommand{\cX}{\mathcal{X}}
\newcommand{\cY}{\mathcal{Y}}
\newcommand{\cZ}{\mathcal{Z}}


%%%%%%%%%%%%%%%%%%%%%%%%%%%%%%%%%%%%%%%%%%%%%
%          Primed Letters
%%%%%%%%%%%%%%%%%%%%%%%%%%%%%%%%%%%%%%%%%%%%%
\newcommand{\aprime}{a^{\prime}}
\newcommand{\bprime}{b^{\prime}}
\newcommand{\cprime}{c^{\prime}}
\newcommand{\dprime}{d^{\prime}}
\newcommand{\eprime}{e^{\prime}}
\newcommand{\fprime}{f^{\prime}}
\newcommand{\gprime}{g^{\prime}}
\newcommand{\hprime}{h^{\prime}}
\newcommand{\iprime}{i^{\prime}}
\newcommand{\jprime}{j^{\prime}}
\newcommand{\kprime}{k^{\prime}}
\newcommand{\lprime}{l^{\prime}}
\newcommand{\mprime}{m^{\prime}}
\newcommand{\nprime}{n^{\prime}}
\newcommand{\oprime}{o^{\prime}}
\newcommand{\pprime}{p^{\prime}}
\newcommand{\qprime}{q^{\prime}}
\newcommand{\rprime}{r^{\prime}}
\newcommand{\sprime}{s^{\prime}}
\newcommand{\tprime}{t^{\prime}}
\newcommand{\uprime}{u^{\prime}}
\newcommand{\vprime}{v^{\prime}}
\newcommand{\wprime}{w^{\prime}}
\newcommand{\xprime}{x^{\prime}}
\newcommand{\yprime}{y^{\prime}}
\newcommand{\zprime}{z^{\prime}}
\newcommand{\Aprime}{A^{\prime}}
\newcommand{\Bprime}{B^{\prime}}
\newcommand{\Cprime}{C^{\prime}}
\newcommand{\Dprime}{D^{\prime}}
\newcommand{\Eprime}{E^{\prime}}
\newcommand{\Fprime}{F^{\prime}}
\newcommand{\Gprime}{G^{\prime}}
\newcommand{\Hprime}{H^{\prime}}
\newcommand{\Iprime}{I^{\prime}}
\newcommand{\Jprime}{J^{\prime}}
\newcommand{\Kprime}{K^{\prime}}
\newcommand{\Lprime}{L^{\prime}}
\newcommand{\Mprime}{M^{\prime}}
\newcommand{\Nprime}{N^{\prime}}
\newcommand{\Oprime}{O^{\prime}}
\newcommand{\Pprime}{P^{\prime}}
\newcommand{\Qprime}{Q^{\prime}}
\newcommand{\Rprime}{R^{\prime}}
\newcommand{\Sprime}{S^{\prime}}
\newcommand{\Tprime}{T^{\prime}}
\newcommand{\Uprime}{U^{\prime}}
\newcommand{\Vprime}{V^{\prime}}
\newcommand{\Wprime}{W^{\prime}}
\newcommand{\Xprime}{X^{\prime}}
\newcommand{\Yprime}{Y^{\prime}}
\newcommand{\Zprime}{Z^{\prime}}
\newcommand{\alphaprime}{\alpha^{\prime}}
\newcommand{\betaprime}{\beta^{\prime}}
\newcommand{\gammaprime}{\gamma^{\prime}}
\newcommand{\Gammaprime}{\Gamma^{\prime}}
\newcommand{\deltaprime}{\delta^{\prime}}
\newcommand{\epsilonprime}{\epsilon^{\prime}}
\newcommand{\kappaprime}{\kappa^{\prime}}
\newcommand{\lambdaprime}{\lambda^{\prime}}
\newcommand{\rhoprime}{\rho^{\prime}}
\newcommand{\Sigmaprime}{\Sigma^{\prime}}
\newcommand{\tauprime}{\tau^{\prime}}
\newcommand{\xiprime}{\xi^{\prime}}
\newcommand{\thetaprime}{\theta^{\prime}}
\newcommand{\Omegaprime}{\Omega^{\prime}}
\newcommand{\cMprime}{\cM^{\prime}}
\newcommand{\cNprime}{\cN^{\prime}}
\newcommand{\cPprime}{\cP^{\prime}}
\newcommand{\cQprime}{\cQ^{\prime}}
\newcommand{\cRprime}{\cR^{\prime}}
\newcommand{\cSprime}{\cS^{\prime}}
\newcommand{\cTprime}{\cT^{\prime}}

%%%%%%%%%%%%%%%%%%%%%%%%%%%%%%%%%%%%%%%%%%%%%
%          bar Letters
%%%%%%%%%%%%%%%%%%%%%%%%%%%%%%%%%%%%%%%%%%%%%
\newcommand{\abar}{\bar{a}}
\newcommand{\bbar}{\bar{b}}
\newcommand{\cbar}{\bar{c}}
\newcommand{\ibar}{\bar{i}}
\newcommand{\jbar}{\bar{j}}
\newcommand{\nbar}{\bar{n}}
\newcommand{\xbar}{\bar{x}}
\newcommand{\ybar}{\bar{y}}
\newcommand{\zbar}{\bar{z}}
\newcommand{\pibar}{\bar{\pi}}
\newcommand{\phibar}{\bar{\varphi}}
\newcommand{\psibar}{\bar{\psi}}
\newcommand{\thetabar}{\bar{\theta}}
\newcommand{\nubar}{\bar{\nu}}

%%%%%%%%%%%%%%%%%%%%%%%%%%%%%%%%%%%%%%%%%%%%%
%          star Letters
%%%%%%%%%%%%%%%%%%%%%%%%%%%%%%%%%%%%%%%%%%%%%
\newcommand{\phistar}{\phi^{*}}
\newcommand{\Mstar}{M^{*}}

%%%%%%%%%%%%%%%%%%%%%%%%%%%%%%%%%%%%%%%%%%%%%
%          dotletters Letters
%%%%%%%%%%%%%%%%%%%%%%%%%%%%%%%%%%%%%%%%%%%%%
\newcommand{\Gdot}{\dot{G}}

%%%%%%%%%%%%%%%%%%%%%%%%%%%%%%%%%%%%%%%%%%%%%
%         check Letters
%%%%%%%%%%%%%%%%%%%%%%%%%%%%%%%%%%%%%%%%%%%%%
\newcommand{\deltacheck}{\check{\delta}}
\newcommand{\gammacheck}{\check{\gamma}}


%%%%%%%%%%%%%%%%%%%%%%%%%%%%%%%%%%%%%%%%%%%%%
%          Formulas
%%%%%%%%%%%%%%%%%%%%%%%%%%%%%%%%%%%%%%%%%%%%%

\newcommand{\formulaphi}{\text{\large $\varphi$}}
\newcommand{\Formulaphi}{\text{\Large $\varphi$}}


%%%%%%%%%%%%%%%%%%%%%%%%%%%%%%%%%%%%%%%%%%%%%
%          Fraktur Letters
%%%%%%%%%%%%%%%%%%%%%%%%%%%%%%%%%%%%%%%%%%%%%

\newcommand{\fa}{\mathfrak{a}}
\newcommand{\fb}{\mathfrak{b}}
\newcommand{\fc}{\mathfrak{c}}
\newcommand{\fk}{\mathfrak{k}}
\newcommand{\fp}{\mathfrak{p}}
\newcommand{\fq}{\mathfrak{q}}
\newcommand{\fr}{\mathfrak{r}}
\newcommand{\fA}{\mathfrak{A}}
\newcommand{\fB}{\mathfrak{B}}
\newcommand{\fC}{\mathfrak{C}}
\newcommand{\fD}{\mathfrak{D}}

%%%%%%%%%%%%%%%%%%%%%%%%%%%%%%%%%%%%%%%%%%%%%
%          Bold Letters
%%%%%%%%%%%%%%%%%%%%%%%%%%%%%%%%%%%%%%%%%%%%%
\newcommand{\ba}{\mathbf{a}}
\newcommand{\bb}{\mathbf{b}}
\newcommand{\bc}{\mathbf{c}}
\newcommand{\bd}{\mathbf{d}}
\newcommand{\be}{\mathbf{e}}
\newcommand{\bu}{\mathbf{u}}
\newcommand{\bv}{\mathbf{v}}
\newcommand{\bw}{\mathbf{w}}
\newcommand{\bx}{\mathbf{x}}
\newcommand{\by}{\mathbf{y}}
\newcommand{\bz}{\mathbf{z}}
\newcommand{\bSigma}{\boldsymbol{\Sigma}}
\newcommand{\bPi}{\boldsymbol{\Pi}}
\newcommand{\bDelta}{\boldsymbol{\Delta}}
\newcommand{\bdelta}{\boldsymbol{\delta}}
\newcommand{\bgamma}{\boldsymbol{\gamma}}
\newcommand{\bGamma}{\boldsymbol{\Gamma}}

%%%%%%%%%%%%%%%%%%%%%%%%%%%%%%%%%%%%%%%%%%%%%
%         Bold numbers
%%%%%%%%%%%%%%%%%%%%%%%%%%%%%%%%%%%%%%%%%%%%%
\newcommand{\bzero}{\mathbf{0}}

%%%%%%%%%%%%%%%%%%%%%%%%%%%%%%%%%%%%%%%%%%%%%
% Projective-Like Pointclasses
%%%%%%%%%%%%%%%%%%%%%%%%%%%%%%%%%%%%%%%%%%%%%
\newcommand{\Sa}[2][\alpha]{\Sigma_{(#1,#2)}}
\newcommand{\Pa}[2][\alpha]{\Pi_{(#1,#2)}}
\newcommand{\Da}[2][\alpha]{\Delta_{(#1,#2)}}
\newcommand{\Aa}[2][\alpha]{A_{(#1,#2)}}
\newcommand{\Ca}[2][\alpha]{C_{(#1,#2)}}
\newcommand{\Qa}[2][\alpha]{Q_{(#1,#2)}}
\newcommand{\da}[2][\alpha]{\delta_{(#1,#2)}}
\newcommand{\leqa}[2][\alpha]{\leq_{(#1,#2)}}
\newcommand{\lessa}[2][\alpha]{<_{(#1,#2)}}
\newcommand{\equiva}[2][\alpha]{\equiv_{(#1,#2)}}


\newcommand{\Sl}[1]{\Sa[\lambda]{#1}}
\newcommand{\Pl}[1]{\Pa[\lambda]{#1}}
\newcommand{\Dl}[1]{\Da[\lambda]{#1}}
\newcommand{\Al}[1]{\Aa[\lambda]{#1}}
\newcommand{\Cl}[1]{\Ca[\lambda]{#1}}
\newcommand{\Ql}[1]{\Qa[\lambda]{#1}}

\newcommand{\San}{\Sa{n}}
\newcommand{\Pan}{\Pa{n}}
\newcommand{\Dan}{\Da{n}}
\newcommand{\Can}{\Ca{n}}
\newcommand{\Qan}{\Qa{n}}
\newcommand{\Aan}{\Aa{n}}
\newcommand{\dan}{\da{n}}
\newcommand{\leqan}{\leqa{n}}
\newcommand{\lessan}{\lessa{n}}
\newcommand{\equivan}{\equiva{n}}

\newcommand{\SigmaOneOmega}{\Sigma^1_{\omega}}
\newcommand{\SigmaOneOmegaPlusOne}{\Sigma^1_{\omega+1}}
\newcommand{\PiOneOmega}{\Pi^1_{\omega}}
\newcommand{\PiOneOmegaPlusOne}{\Pi^1_{\omega+1}}
\newcommand{\DeltaOneOmegaPlusOne}{\Delta^1_{\omega+1}}

%%%%%%%%%%%%%%%%%%%%%%%%%%%%%%%%%%%%%%%%%%%%%
% Linear Algebra
%%%%%%%%%%%%%%%%%%%%%%%%%%%%%%%%%%%%%%%%%%%%%
\newcommand{\transpose}[1]{{#1}^{\text{T}}}
\newcommand{\norm}[1]{\lVert{#1}\rVert}
\newcommand\aug{\fboxsep=-\fboxrule\!\!\!\fbox{\strut}\!\!\!}

%%%%%%%%%%%%%%%%%%%%%%%%%%%%%%%%%%%%%%%%%%%%%
% Number Theory
%%%%%%%%%%%%%%%%%%%%%%%%%%%%%%%%%%%%%%%%%%%%%
\newcommand{\av}[1]{\lvert#1\rvert}
\DeclareMathOperator{\divides}{\mid}
\DeclareMathOperator{\ndivides}{\nmid}
\DeclareMathOperator{\lcm}{lcm}
\DeclareMathOperator{\sign}{sign}
\newcommand{\floor}[1]{\left\lfloor{#1}\right\rfloor}
\DeclareMathOperator{\ConCl}{CC}
\DeclareMathOperator{\ord}{ord}


%%%%%%%%%%%%%%%%%%%%%%%%%%%%%%%%%%%%%%%%%%%%%%%%%%%%%%%%%%%%%%%%%%%%%%%%%%%
%%  Theorem-Like Declarations
%%%%%%%%%%%%%%%%%%%%%%%%%%%%%%%%%%%%%%%%%%%%%%%%%%%%%%%%%%%%%%%%%%%%%%%%%%

\newtheorem{theorem}{Theorem}[section]
\newtheorem{lemma}[theorem]{Lemma}
\newtheorem{corollary}[theorem]{Corollary}
\newtheorem{proposition}[theorem]{Proposition}


\theoremstyle{definition}

\newtheorem{definition}[theorem]{Definition}
\newtheorem{conjecture}[theorem]{Conjecture}
\newtheorem{remark}[theorem]{Remark}
\newtheorem{remarks}[theorem]{Remarks}
\newtheorem{notation}[theorem]{Notation}

\theoremstyle{remark}

\newtheorem*{note}{Note}
\newtheorem*{warning}{Warning}
\newtheorem*{question}{Question}
\newtheorem*{example}{Example}
\newtheorem*{fact}{Fact}


\newenvironment*{subproof}[1][Proof]
{\begin{proof}[#1]}{\renewcommand{\qedsymbol}{$\diamondsuit$} \end{proof}}

\newenvironment*{case}[1]
{\textbf{Case #1.  }\itshape }{}

\newenvironment*{claim}[1][Claim]
{\textbf{#1.  }\itshape }{}


\graphicspath{{images/}}

\pagestyle{plain}

\begin{document}

\title{Solutions to Homework for Lesson 15 \\ Gaussian Elimination}
\author{Math 325, Linear Algebra \\ Mitch Rudominer \\ Fall 2018 \\ SFSU }
\date{}

\maketitle


\textbf{Problem 1}

\bigskip

%%%%%%%%%%%%%%%%%%%%%%%%%%%%%%%%%%%%% (1 a)

\textbf{(a)} Form the augmented matrix
$
\begin{pmatrix}
 1 & -1 &  \aug & 7  \\
 1 &  2 &  \aug & 3  \\
\end{pmatrix}
$

Perform Gaussian elimination.


\bigskip

$
\begin{pmatrix}
 1 & -1 &  \aug & 7  \\
 0 &  3 &  \aug & -4  \\
\end{pmatrix}
$


\bigskip

Perform back-substitution.

\begin{itemize}
\item $-3y = 4  \SkipImplies y = -4/3$
\item $x-y=7 \SkipImplies x+4/3=7 \SkipImplies x = 17/3 = 5 \frac{2}{3}$.
\end{itemize}

\bigskip

Check

\bigskip

$
\begin{pmatrix}
 1 & -1 \\
 1 & 2 \\
\end{pmatrix}
\begin{pmatrix}
 \frac{17}{3} \\[6pt]
 -\frac{4}{3}
\end{pmatrix}
=
\begin{pmatrix}
 \frac{21}{3} \\[6pt]
 \frac{9}{3}
\end{pmatrix}
=
\begin{pmatrix}
 7 \\
3
\end{pmatrix}
$

\bigskip

%%%%%%%%%%%%%%%%%%%%%%%%%%%%%%%%%%%%% (1 b)

\textbf{(b)} Form the augmented matrix
$
\begin{pmatrix}
 6 &   1 &  \aug & 5  \\
 3 &  -2 &  \aug & 5  \\
\end{pmatrix}
$

Perform Gaussian elimination.


\bigskip

$
\begin{pmatrix}
 6 &   1           &  \aug & 5             \\[6pt]
 0 &  -\frac{5}{2} &  \aug & \frac{5}{2}  \\
\end{pmatrix}
$

\bigskip

Perform back-substitution.

\begin{itemize}
\item $-\frac{5}{2}v = \frac{5}{2}  \SkipImplies v = -1$
\item $6u+v=5 \SkipImplies 6u-1=5 \SkipImplies u = 1$.
\end{itemize}

\bigskip

Check

\bigskip

$
\begin{pmatrix}
 6 & 1 \\
 3 & -2 \\
\end{pmatrix}
\begin{pmatrix}
 1 \\
 -1 \\
\end{pmatrix}
=
\begin{pmatrix}
 5 \\
 5 \\
\end{pmatrix}
$

\bigskip

%%%%%%%%%%%%%%%%%%%%%%%%%%%%%%%%%%%%% (1 c)

\textbf{(c)} Form the augmented matrix
$
\begin{pmatrix}
  2 &  1 &  2 & \aug &  3  \\
 -1 &  3 &  3 & \aug & -2  \\
  4 & -3 &  0 & \aug &  7  \\
\end{pmatrix}
$

Perform Gaussian elimination.


\bigskip

$
\begin{pmatrix}
  2 &  1           &  2  & \aug &  3            \\[6pt]
  0 &  \frac{7}{2} &  4  & \aug & -\frac{1}{2}  \\[6pt]
  0 & -5           &  -4 & \aug &  1  \\
\end{pmatrix}
\SkipImplies
\begin{pmatrix}
  2 &  1           &  2            & \aug &  3            \\[6pt]
  0 &  \frac{7}{2} &  4            & \aug & -\frac{1}{2}  \\[6pt]
  0 &  0           &  \frac{12}{7} & \aug &  \frac{2}{7}  \\
\end{pmatrix}
$

\bigskip

Perform back-substitution.

\begin{itemize}
\item $\frac{12}{7}w = \frac{2}{7}  \SkipImplies w = \frac{1}{6}$
\item $\frac{7}{2}v + 4w = -\frac{1}{2}  \SkipImplies \frac{7}{2}v + \frac{2}{3} = -\frac{1}{2} \SkipImplies 21v + 4= -3 \SkipImplies v = -\frac{1}{3}$.
\item $2u +v +2w = 3 \SkipImplies 2u -\frac{1}{3} + \frac{1}{3}= 3 \SkipImplies u = \frac{3}{2}$
\end{itemize}

\bigskip

Check

\bigskip

$
\begin{pmatrix}
 2  &  1 &  2  \\
 -1 &  3 &  3  \\
  4 & -3 &  0  \\
\end{pmatrix}
\begin{pmatrix}
 \frac{3}{2} \\[6pt]
 -\frac{1}{3}\\[6pt]
 \frac{1}{6} \\[6pt]
\end{pmatrix}
=
\begin{pmatrix}
\frac{18}{6} \\[6pt]
-\frac{12}{6}\\[6pt]
\frac{42}{6} \\[6pt]
\end{pmatrix}
=
\begin{pmatrix}
3 \\[6pt]
-2\\[6pt]
7 \\[6pt]
\end{pmatrix}
$

\bigskip

%%%%%%%%%%%%%%%%%%%%%%%%%%%%%%%%%%%%% (1 d)

\textbf{(d)} Form the augmented matrix
$
\begin{pmatrix}
  5 & 3 &  -1 & \aug &  9  \\
  3 & 2 &  -1 & \aug &  5  \\
  1 & 1 &   2 & \aug & -1  \\
\end{pmatrix}
$

Perform Gaussian elimination.

\bigskip

$
\begin{pmatrix}
 5 & 3           &  -1            & \aug &    9            \\[6pt]
 0 & \frac{1}{5} &  - \frac{2}{5} & \aug &  - \frac{2}{5}  \\[6pt]
 0 & \frac{2}{5} &   \frac{11}{5} & \aug &   -\frac{14}{5} \\
\end{pmatrix}
\SkipImplies
\begin{pmatrix}
 5 & 3           &  -1            & \aug &    9            \\[6pt]
 0 & \frac{1}{5} &  - \frac{2}{5} & \aug &   -\frac{2}{5}  \\[6pt]
 0 & 0           &   3            & \aug &   -2            \\
\end{pmatrix}
$

\bigskip

Perform back-substitution.

\begin{itemize}
\item $3x_3=-2  \SkipImplies x_3 = -\frac{2}{3}$
\item $\frac{1}{5}x_2 -  \frac{2}{5} x_3 =  -\frac{2}{5} \SkipImplies \frac{3}{15}x_2 +  \frac{4}{15} =  -\frac{6}{15} \SkipImplies x_2 = -\frac{10}{3}$
\item $5x_1 +3x_2 -x_3 = 9 \SkipImplies 5x_1 -10 + \frac{2}{3} = 9 \SkipImplies x_1 = \frac{11}{3}$
\end{itemize}

\bigskip

Check

\bigskip

$
\begin{pmatrix}
5 & 3 &  -1  \\
3 & 2 &  -1  \\
1 & 1 &   2  \\
\end{pmatrix}
\begin{pmatrix}
\frac{11}{3} \\[6pt]
-\frac{10}{3}\\[6pt]
-\frac{2}{3}\\[6pt]
\end{pmatrix}
=
\frac{1}{3}
\begin{pmatrix}
27\\[6pt]
15\\[6pt]
-3\\[6pt]
\end{pmatrix}
=
\begin{pmatrix}
9 \\[6pt]
5\\[6pt]
-1 \\[6pt]
\end{pmatrix}
$

\bigskip

%%%%%%%%%%%%%%%%%%%%%%%%%%%%%%%%%%%%% (1 e)

\textbf{(e)} Form the augmented matrix
$
\begin{pmatrix}
  1  &  1 &  -1 & \aug &  0  \\
  2  & -1 &   3 & \aug &  3  \\
 -1  & -1 &   3 & \aug &  5  \\
\end{pmatrix}
$

Perform Gaussian elimination.

\bigskip

$
\begin{pmatrix}
   1  &  1 &  -1 & \aug &  0  \\
   0  & -3 &   5 & \aug &  3  \\
   0  & 0  &   2 & \aug &  5  \\
\end{pmatrix}
$

\bigskip

Perform back-substitution.

\begin{itemize}
\item $2r=5 \SkipImplies r = \frac{5}{2}$
\item $-3q + 5r = 3 \SkipImplies -3q + \frac{25}{2} = 3 \SkipImplies q = \frac{19}{6}$
\item $p + q -r = 0 \SkipImplies p + \frac{19}{6} - \frac{5}{2} = 0 \SkipImplies p = -\frac{2}{3}$.
\end{itemize}

\bigskip

Check

\bigskip

$
\begin{pmatrix}
 1  &  1 &  -1  \\
 2  & -1 &   3  \\
-1  & -1 &   3  \\
\end{pmatrix}
\begin{pmatrix}
-\frac{2}{3} \\[6pt]
\frac{19}{6}\\[6pt]
\frac{5}{2}\\[6pt]
\end{pmatrix}
=
\frac{1}{6}
\begin{pmatrix}
0\\[6pt]
18\\[6pt]
30\\[6pt]
\end{pmatrix}
=
\begin{pmatrix}
0 \\[6pt]
3\\[6pt]
5 \\[6pt]
\end{pmatrix}
$

\bigskip

%%%%%%%%%%%%%%%%%%%%%%%%%%%%%%%%%%%%% (1 f)

\textbf{(f)} Form the augmented matrix

$
\begin{pmatrix}
 -1  &  1 &  1 &  0 & \aug &  1  \\
  2  & -1 &  0 &  1 & \aug &  0  \\
  1  &  0 &  2 &  3 & \aug &  1  \\
  0  &  1 & -1 & -2 & \aug &  0  \\
\end{pmatrix}
$

Perform Gaussian elimination.

\bigskip

$
\begin{pmatrix}
 -1  &  1 &  1 &  0 & \aug &  1  \\
  0  &  1 &  2 &  1 & \aug &  2  \\
  0  &  1 &  3 &  3 & \aug &  2  \\
  0  &  1 & -1 & -2 & \aug &  0  \\
\end{pmatrix}
\SkipImplies
\begin{pmatrix}
 -1  &  1 &  1 &  0 & \aug &  1  \\
  0  &  1 &  2 &  1 & \aug &  2  \\
  0  &  0 &  1 &  2 & \aug &  0  \\
  0  &  0 & -3 & -3 & \aug &  -2  \\
\end{pmatrix}
\SkipImplies
\begin{pmatrix}
 -1  &  1 &  1 &  0 & \aug &  1  \\
  0  &  1 &  2 &  1 & \aug &  2  \\
  0  &  0 &  1 &  2 & \aug &  0  \\
  0  &  0 &  0 &  3 & \aug &  -2  \\
\end{pmatrix}
$

\bigskip

Perform back-substitution.

\begin{itemize}
\item $3d=-2 \SkipImplies d = -\frac{2}{3}$.
\item $c+2d=0 \SkipImplies c - \frac{4}{3} = 0 \SkipImplies c = \frac{4}{3}$.
\item $b + 2c +d =2 \SkipImplies b + \frac{8}{3} - {2}{3} = 2 \SkipImplies b = 0$.
\item $-a + b  + c = 1 \SkipImplies -a + \frac{4}{3} = 1 \SkipImplies a = \frac{1}{3}$.
\end{itemize}

\bigskip

Check

\bigskip

$
\begin{pmatrix}
 -1  &  1 &  1 &  0 \\
  2  & -1 &  0 &  1 \\
  1  &  0 &  2 &  3 \\
  0  &  1 & -1 & -2 \\
\end{pmatrix}
\begin{pmatrix}
\frac{1}{3}  \\[6pt]
0            \\[6pt]
\frac{4}{3}  \\[6pt]
-\frac{2}{3} \\[6pt]
\end{pmatrix}
=
\frac{1}{3}
\begin{pmatrix}
3 \\
0 \\
3 \\
0 \\
\end{pmatrix}
=
\begin{pmatrix}
1 \\
0 \\
1 \\
0 \\
\end{pmatrix}
$

%%%%%%%%%%%%%%%%%%%%%%%%%%%%%%%%%%%%% (g)

\textbf{(g)} Form the augmented matrix

\bigskip

$
\begin{pmatrix}
  2  & -3 &  1 &  1 & \aug & -1  \\
  1  & -1 & -2 & -1 & \aug &  0  \\
  3  & -2 &  1 &  2 & \aug &  5  \\
  1  &  3 &  2 &  1 & \aug &  3  \\
\end{pmatrix}
$
\bigskip

Perform Gaussian elimination.

\bigskip

$
\begin{pmatrix}
  2  & -3           &  1           &  1           & \aug & -1            \\[6pt]
  0  & \frac{1}{2}  & -\frac{5}{2} & -\frac{3}{2} & \aug &  \frac{1}{2}  \\[6pt]
  0  & \frac{5}{2}  & -\frac{1}{2} &  \frac{1}{2} & \aug &  \frac{13}{2} \\[6pt]
  0  & \frac{9}{2}  &  \frac{3}{2} &  \frac{1}{2} & \aug &  \frac{7}{2}  \\[6pt]
\end{pmatrix}
\SkipImplies
\begin{pmatrix}
  2  & -3           &  1           &  1           & \aug & -1            \\[6pt]
  0  & \frac{1}{2}  & -\frac{5}{2} & -\frac{3}{2} & \aug &  \frac{1}{2}  \\[6pt]
  0  & 0            & 12           &  8           & \aug &  4             \\[6pt]
  0  & 0            & 24           &  14          & \aug & -1  \\[6pt]
\end{pmatrix}
\SkipImplies
\begin{pmatrix}
  2  & -3           &  1           &  1           & \aug & -1            \\[6pt]
  0  & \frac{1}{2}  & -\frac{5}{2} & -\frac{3}{2} & \aug &  \frac{1}{2}  \\[6pt]
  0  & 0            & 12           &  8           & \aug &  4             \\[6pt]
  0  & 0            & 0            &  -2          & \aug & -9             \\[6pt]
\end{pmatrix}
$

\bigskip

Perform back-substitution.

\begin{itemize}
\item $-2w = -9 \SkipImplies w = \frac{9}{2}$.
\item $12z + 8w = 4 \SkipImplies 12z + 36 = 4 \SkipImplies z = -\frac{8}{3}$.
\item $\frac{1}{2}y -\frac{5}{2}z -\frac{3}{2}w = \frac{1}{2} \SkipImplies y +\frac{40}{3} -\frac{27}{2} = 1 \SkipImplies y = \frac{7}{6}$
\item $2x - 3y + z + w = -1 \SkipImplies 2x -\frac{21}{6} -\frac{8}{3} + \frac{9}{2} = -1 \SkipImplies x =  \frac{1}{3}$.
\end{itemize}

\bigskip

Check

\bigskip

$
\begin{pmatrix}
 2  & -3 &  1 &  1 \\
 1  & -1 & -2 & -1 \\
 3  & -2 &  1 &  2 \\
 1  &  3 &  2 &  1 \\
\end{pmatrix}
\begin{pmatrix}
\frac{1}{3}  \\[6pt]
\frac{7}{6}  \\[6pt]
-\frac{8}{3} \\[6pt]
\frac{9}{2} \\[6pt]
\end{pmatrix}
=
\frac{1}{6}
\begin{pmatrix}
2  & -3 &  1 &  1 \\
1  & -1 & -2 & -1 \\
3  & -2 &  1 &  2 \\
1  &  3 &  2 &  1 \\
\end{pmatrix}
\begin{pmatrix}
2  \\
7  \\
-16 \\
27 \\
\end{pmatrix}
=
\begin{pmatrix}
-1 \\
0 \\
5 \\
3 \\
\end{pmatrix}
$

\bigskip

\textbf{Problem 2.}

\bigskip

%%%%%%%%%%%%%%%%%%%%%%%%%%%%%%%%%%%% (2 a)

\textbf{(a)}

Augmented matrix:

\bigskip

$
\begin{pmatrix}
   1  &  7  & \aug & 4  \\
  -2  & -9  & \aug & 2  \\
\end{pmatrix}
$

\bigskip

Solution: $x_1 = -10, x_2 = 2$

\bigskip

%%%%%%%%%%%%%%%%%%%%%%%%%%%%%%%%%%%% (2 b)

\textbf{(b)}

Augmented matrix:

\bigskip

$
\begin{pmatrix}
   3  &  -5  & \aug & -1  \\
   2  &  1  & \aug  & 8  \\
\end{pmatrix}
$

\bigskip

Solution: $z = 3, w = 2$

%%%%%%%%%%%%%%%%%%%%%%%%%%%%%%%%%%%% (2 c)

\bigskip

\textbf{(c)}

Augmented matrix:

\bigskip

$
\begin{pmatrix}
   1  & -2   & 1  & \aug  & 0  \\
   0  &  2   & -8 & \aug  & 8  \\
   -4 &  5   & 9  & \aug  & -9  \\
\end{pmatrix}
$

\bigskip

Solution: $x=29, y=16, z=3$

%%%%%%%%%%%%%%%%%%%%%%%%%%%%%%%%%%%% (2 d)

\bigskip

\textbf{(d)}

Augmented matrix:

\bigskip

$
\begin{pmatrix}
   1   &  4   & -2  & \aug  & 1  \\
   -2  &  0   & -3  & \aug  & -7  \\
   3   &  -2  &  2  & \aug  & -1  \\
\end{pmatrix}
$

\bigskip

Solution: $p=-1, q=2, r=3$

%%%%%%%%%%%%%%%%%%%%%%%%%%%%%%%%%%%% (2 e)

\bigskip

\textbf{(e)}

Augmented matrix:

\bigskip

$
\begin{pmatrix}
   1  &  0   & -2  &  0 & \aug  & -1  \\
   0  &  1   &  0  & -1 & \aug  & 2   \\
   0  & -3   &  2  &  0 & \aug  & 0   \\
  -4   & 0   &  0  &  7 & \aug  & -5  \\
\end{pmatrix}
$

\bigskip

Solution: $x_1=-4, x_2=-1, x_3 = -\frac{3}{2}, x_4=-3$.

%%%%%%%%%%%%%%%%%%%%%%%%%%%%%%%%%%%% (2 f)

\bigskip

\textbf{(f)}

Augmented matrix:

\bigskip

$
\begin{pmatrix}
  -1  &  3   & -1  &  1 & \aug  & -2  \\
   1  & -1   &  3  & -1 & \aug  &  0  \\
   0  &  1   & -1  &  4 & \aug  &  7  \\
   4  & -1   &  1  &  0 & \aug  &  5  \\
\end{pmatrix}
$

\bigskip

Solution: $x=1, y=-1, z=0, w=2$.

\bigskip

\textbf{Problem 3.}

\bigskip

%%%%%%%%%%%%%%%%%%%%%%%%%%%%%%%%%%%% (3a)

\textbf{(a)}

\begin{align*}
3x + 2y   &= 2 \\
-4x -3y &= -1 \\
\end{align*}

Solution $x=4, y=-5$.

\bigskip

%%%%%%%%%%%%%%%%%%%%%%%%%%%%%%%%%%%% (3b)

\textbf{(b)}

\begin{align*}
  x + 2y       &= -3 \\
 -x + 2y +  z  &= -6 \\
-2x     - 3z  &=  1 \\
\end{align*}

Solution $x=1, y=-2, z=-1$.

\bigskip

%%%%%%%%%%%%%%%%%%%%%%%%%%%%%%%%%%%% (3c)

\textbf{(c)}

\begin{align*}
  2x - y          &= 0 \\
 -x + 2y -  z     &= 1 \\
      -y  +2z -w  &= 1 \\
           -z+2w  &= 0 \\
\end{align*}

Solution $x=1, y=2, z=2, w=1$.

\bigskip

\textbf{Problem 4.}

\textbf{(a)} Nonsingular. You can see by inspection that the two columns are linearly independent.

\bigskip

\textbf{(b)} Singular. You can see by inspection that the second column is -2 times the first column.

\bigskip

\textbf{(c)} I can't tell by inspection so I do Gaussian elimination.

\bigskip

$
\begin{pmatrix}
0 & 1 & 2 \\
-1 & 1 & 3 \\
2 & -2 & 0
\end{pmatrix}
\SkipImplies
\begin{pmatrix}
-1 & 1 & 3 \\
0 & 1 & 2 \\
2 & -2 & 0
\end{pmatrix}
\SkipImplies
\begin{pmatrix}
-1 & 1 & 3 \\
0 & 1 & 2 \\
0 & 0 & 6
\end{pmatrix}
$

\bigskip

So the rank is 3 and the matrix is nonsingular.

\bigskip

\textbf{(d)}

$
\begin{pmatrix}
1 & 1 & 3 \\
2 & 2 & 2 \\
3 & -1 & 1
\end{pmatrix}
\SkipImplies
\begin{pmatrix}
1 & 1 & 3 \\
0 & 0 & -4 \\
0 & -4 & -8
\end{pmatrix}
\SkipImplies
\begin{pmatrix}
1 & 1 & 3 \\
0 & -4 & -8 \\
0 & 0 & -4
\end{pmatrix}
$

\bigskip

So the rank is 3 and the matrix is nonsingular.

\bigskip

\textbf{(e)}

$
\begin{pmatrix}
1 & 2 & 3 \\
4 & 5 & 6 \\
7 & 8 & 9
\end{pmatrix}
\SkipImplies
\begin{pmatrix}
1 & 2 & 3 \\
0 & -3 & -6 \\
0 & -6 & -12
\end{pmatrix}
\SkipImplies
\begin{pmatrix}
1 & 2 & 3 \\
0 & -3 & -6 \\
0 & 0 & 0
\end{pmatrix}
$

\bigskip

So the rank is 2 and the matrix is singular.

\bigskip

\textbf{(f)}

$
\begin{pmatrix}
-1 & 1 & 0 & -3 \\
2 & -2 & 4 & 0 \\
1 & -2 & 2 & -1 \\
0 & 1 & 0 & 1
\end{pmatrix}
\SkipImplies
\begin{pmatrix}
-1 & 1 & 0 & -3 \\
0 & 0 & 4 & -6 \\
0 & -1 & 2 & -4 \\
0 & 1 & 0 & 1
\end{pmatrix}
\SkipImplies
\begin{pmatrix}
-1 & 1 & 0 & -3 \\
0 & -1 & 2 & -4 \\
0 & 0 & 4 & -6 \\
0 & 1 & 0 & 1
\end{pmatrix}
\SkipImplies
\begin{pmatrix}
-1 & 1 & 0 & -3 \\
0 & -1 & 2 & -4 \\
0 & 0 & 4 & -6 \\
0 & 0 & 2 & -3
\end{pmatrix}
\SkipImplies
\begin{pmatrix}
-1 & 1 & 0 & -3 \\
0 & -1 & 2 & -4 \\
0 & 0 & 4 & -6 \\
0 & 0 & 0 & 0
\end{pmatrix}
$

\bigskip

So the rank is 3 and the matrix is singular.


\bigskip

\textbf{(g)}

$
\begin{pmatrix}
0 & -1 & 0 & 1 \\
1 & 0 & -1 & 0 \\
0 & 2 & 0 & -2 \\
2 & 0 & 2 & 0
\end{pmatrix}
\SkipImplies
\begin{pmatrix}
1 & 0 & -1 & 0 \\
0 & -1 & 0 & 1 \\
0 & 2 & 0 & -2 \\
2 & 0 & 2 & 0
\end{pmatrix}
\SkipImplies
\begin{pmatrix}
1 & 0 & -1 & 0 \\
0 & -1 & 0 & 1 \\
0 & 2 & 0 & -2 \\
0 & 0 & 4 & 0
\end{pmatrix}
\SkipImplies
\begin{pmatrix}
1 & 0 & -1 & 0 \\
0 & -1 & 0 & 1 \\
0 & 0 & 0 & 0 \\
0 & 0 & 4 & 0
\end{pmatrix}
$

\bigskip

We can stop here. The row of all zeros tells us the matrix is singular.

\bigskip

\textbf{(h)}

$
\begin{pmatrix}
1 & -2 & 0 & 2 \\
4 & 1 & -1 & -1 \\
-8 & -1 & 2 & 1 \\
-4 & -1 & 1 & 2
\end{pmatrix}
\SkipImplies
\begin{pmatrix}
1 & -2 & 0 & 2 \\
0 & 9 & -1 & -9 \\
0 & -17 & 2 & 17 \\
0 & -9 & 1 & 10
\end{pmatrix}
\SkipImplies
\begin{pmatrix}
1 & -2 & 0 & 2 \\
0 & 9 & -1 & -9 \\
0 & 0 & \frac{1}{9} & 0 \\
0 & 0 & 0 & 1
\end{pmatrix}
$

\bigskip

So the rank is 4 and the matrix is nonsingular.

\bigskip

\textbf{Problem 5.} We are looking for numbers $a,b,c$ such that

\begin{align*}
       2b  + c &= -1 \\
 -2a + 4b  + c &=  3 \\
  2a   -b  + c &= -3 \\
\end{align*}

$
\begin{pmatrix}
0 & 2 & 1  & \aug  &-1 \\
-2 & 4 & 1 & \aug  & 3 \\
2 & -1 & 1 & \aug  &-3
\end{pmatrix}
\SkipImplies
\begin{pmatrix}
-2 & 4 & 1 & \aug  & 3 \\
0 & 2 & 1  & \aug  &-1 \\
2 & -1 & 1 & \aug  &-3
\end{pmatrix}
\SkipImplies
\begin{pmatrix}
-2 & 4 & 1 & \aug  & 3 \\
0 & 2 & 1 &  \aug  & -1 \\
0 & 3 & 2 &  \aug  & 0
\end{pmatrix}
\SkipImplies
\begin{pmatrix}
-2 & 4 & 1           &   \aug &  3 \\
 0 & 2 & 1           &   \aug & -1 \\
 0 & 0 & \frac{1}{2} &   \aug & \frac{3}{2}
\end{pmatrix}
$

\bigskip

Solution $a=-4, b=-2, c=3$.

\bigskip

So the equation of the plane is $z=-4x -2y +3$.

\bigskip

%%%%%%%%%%%%%%%%%%%%%%%%%%%%%%%%%%%% (6 a)

\textbf{Problem 6.}

\bigskip

\textbf{(a)}

\bigskip

$
\begin{pmatrix}
1 & -2 & 2 &  \aug &  15 \\
1 & -2 & 1 &  \aug &  10 \\
2 & -1 & -2 & \aug & -10
\end{pmatrix}
\SkipImplies
\begin{pmatrix}
1 & -2 & 2 &  \aug & 15 \\
0 & 0 & -1 &  \aug & -5 \\
0 & 3 & -6 &  \aug & -40
\end{pmatrix}
\SkipImplies
\begin{pmatrix}
1 & -2 & 2 &  \aug & 15 \\
0 & 3 & -6 &  \aug & -40 \\
0 & 0 & -1 &  \aug & -5 \\
\end{pmatrix}
$

\bigskip

Solution $x_1=-\frac{5}{3}, x_2=-\frac{10}{3}, x_3=5$.

\bigskip

%%%%%%%%%%%%%%%%%%%%%%%%%%%%%%%%%%%% (6 b)

\textbf{(b)}

\bigskip

$
\begin{pmatrix}
2 & -1 & 0  & \aug & 1 \\
-4 & 2 & -3 & \aug & -8 \\
1 & -3 & 1  & \aug & 5 \\
\end{pmatrix}
\SkipImplies
\begin{pmatrix}
2 & -1 & 0 &            \aug &   1 \\
0 & 0 & -3 &            \aug &  -6 \\
0 & \frac{-5}{2} & 1 &  \aug &  \frac{9}{2} \\
\end{pmatrix}
\SkipImplies
\begin{pmatrix}
2 & -1 & 0 &            \aug &   1 \\
0 & \frac{-5}{2} & 1 &  \aug &  \frac{9}{2} \\
0 & 0 & -3 &            \aug &  -6 \\
\end{pmatrix}
$

\bigskip

Solution $x_1=0, x_2=-1, x_3=2$.

%%%%%%%%%%%%%%%%%%%%%%%%%%%%%%%%%%%% (6 c)

\textbf{(c)}

\bigskip

$
\begin{pmatrix}
0 & 1 & -1 &  \aug &  4 \\
-2 & -5 & 0 & \aug &  2 \\
1 & 0 & 1 &   \aug & -8 \\
\end{pmatrix}
\SkipImplies
\begin{pmatrix}
-2 & -5 & 0 & \aug &  2 \\
 0 & 1 & -1 & \aug &  4 \\
1 & 0 & 1 &   \aug & -8 \\
\end{pmatrix}
\SkipImplies
\begin{pmatrix}
-2 & -5 & 0 &           \aug &  2 \\
0 & 1 & -1 &            \aug &  4 \\
0 & \frac{-5}{2} & 1 &  \aug &  -7
\end{pmatrix}
\SkipImplies
\begin{pmatrix}
-2 & -5 & 0 &            \aug &   2 \\
0 & 1 & -1 &             \aug &  4 \\
0 & 0 & \frac{-3}{2} &   \aug &  3
\end{pmatrix}
$

\bigskip

Solution $x_1=-6, x_2=2, x_3=-2$.

%%%%%%%%%%%%%%%%%%%%%%%%%%%%%%%%%%%% (6 d)

\textbf{(d)}

\bigskip

$
\begin{pmatrix}
1 & -1 & 1 & -1 & \aug &  0 \\
-2 & 2 & -1 & 1 & \aug &  2 \\
-4 & 4 & 3 & 0 &  \aug &  5 \\
1 & -3 & 0 & 1 &  \aug &  4
\end{pmatrix}
\SkipImplies
\begin{pmatrix}
1 & -1 & 1 & -1 &  \aug & 0 \\
0 & 0 & 1 & -1 &   \aug & 2 \\
0 & 0 & 7 & -4 &   \aug & 5 \\
0 & -2 & -1 & 2 &  \aug & 4
\end{pmatrix}
\SkipImplies
\begin{pmatrix}
1 & -1 & 1 & -1 &  \aug & 0 \\
0 & -2 & -1 & 2 &  \aug & 4 \\
0 & 0 & 7 & -4 &   \aug & 5 \\
0 & 0 & 1 & -1 &   \aug & 2 \\
\end{pmatrix}
\SkipImplies
\begin{pmatrix}
1 & -1 & 1 & -1 &           \aug & 0 \\
0 & -2 & -1 & 2 &           \aug & 4 \\
0 & 0 & 7 & -4 &            \aug & 5 \\
0 & 0 & 0 & \frac{-3}{7} &  \aug & \frac{9}{7}
\end{pmatrix}
$

\bigskip

Solution $x=-\frac{13}{2}, y=-\frac{9}{2}, z = -1, w = -3$.

%%%%%%%%%%%%%%%%%%%%%%%%%%%%%%%%%%%% (6 e)

\textbf{(e)}

\bigskip

$
\begin{pmatrix}
0 & -3 & 2 & 0 & \aug & 0 \\
0 & 0 & 1 & -1 & \aug & 2 \\
1 & 0 & -2 & 0 & \aug & -1 \\
-4 & 0 & 0 & 7 & \aug & -5
\end{pmatrix}
\SkipImplies
\begin{pmatrix}
1 & 0 & -2 & 0 & \aug & -1 \\
0 & 0 & 1 & -1 & \aug & 2 \\
0 & -3 & 2 & 0 & \aug & 0 \\
-4 & 0 & 0 & 7 & \aug & -5
\end{pmatrix}
\SkipImplies
\begin{pmatrix}
1 & 0 & -2 & 0 & \aug & -1 \\
0 & 0 & 1 & -1 & \aug & 2 \\
0 & -3 & 2 & 0 & \aug & 0 \\
0 & 0 & -8 & 7 & \aug & -9
\end{pmatrix}
\SkipImplies
\begin{pmatrix}
1 & 0 & -2 & 0 & \aug & -1 \\
0 & -3 & 2 & 0 & \aug & 0 \\
0 & 0 & 1 & -1 & \aug & 2 \\
0 & 0 & -8 & 7 & \aug & -9
\end{pmatrix}
\SkipImplies
\begin{pmatrix}
1 & 0 & -2 & 0 & \aug & -1 \\
0 & -3 & 2 & 0 & \aug & 0 \\
0 & 0 & 1 & -1 & \aug & 2 \\
0 & 0 & 0 & -1 & \aug & 7
\end{pmatrix}
$

\bigskip

Solution $x_1=-11, x_2=-\frac{10}{3}, x_3=-5,x_4=-7$.

\bigskip

\textbf{Problem 7.}

%%%%%%%%%%%%%%%%%%%%%%%%%%%%%%%%%%%% (7)

\bigskip

\textbf{(a)} We can see by inspection that the two columns are linearly independent, so the rank is 2.

\bigskip

\textbf{(b)} We can see by inspection that column 1 is 2 times column 2 and column 3 is 3 times column 2.
So the rank is 1.

\bigskip

\textbf{(c)} We can see by inspection that column 2 is -1 times column 1 and that column 3 is not a multiple of column 1. So the rank is 2.

\bigskip

\textbf{(d)} I can't tell immediately by inspecting this matrix what the rank is, so I will do Gaussian elimination.

$
\begin{pmatrix}
2 & -1 & 0 \\
2 & -1 & 1 \\
1 & 1 & -1
\end{pmatrix}
\SkipImplies
\begin{pmatrix}
2 & -1 & 0 \\
0 & 0 & 1 \\
0 & \frac{3}{2} & -1
\end{pmatrix}
\SkipImplies
\begin{pmatrix}
2 & -1 & 0 \\
0 & \frac{3}{2} & -1 \\
0 & 0 & 1 \\
\end{pmatrix}
$

\bigskip

So the rank is 3.

\bigskip

\textbf{(e)} Since this matrix has a single column and it is nonzero, its rank is one.

\bigskip

\textbf{(f)} This matrix has a single row and it is nonzero. Its rank is one.

\bigskip

\textbf{(g)} It is clear by inspection that the two columns are linearly independent
so this matrix has rank 2.

\bigskip

\textbf{(h)} I cannot easily determine the rank of this matrix by inspection so I will do Gaussian elimination.

\bigskip

$
\begin{pmatrix}
1 & -1 & 2 & 1 \\
2 & 1 & -1 & 0 \\
1 & 2 & -3 & -1 \\
4 & -1 & 3 & 2 \\
0 & 3 & -5 & -2
\end{pmatrix}
\SkipImplies
\begin{pmatrix}
1 & -1 & 2 & 1 \\
0 & 3 & -5 & -2 \\
0 & 3 & -5 & -2 \\
0 & 3 & -5 & -2 \\
0 & 3 & -5 & -2
\end{pmatrix}
\SkipImplies
\begin{pmatrix}
1 & -1 & 2 & 1 \\
0 & 3 & -5 & -2 \\
0 & 0 & 0 & 0 \\
0 & 0 & 0 & 0 \\
0 & 0 & 0 & 0
\end{pmatrix}
$

\bigskip

So the rank is 2.

\bigskip

\textbf{(i)} I will do Gaussian elimination.

\bigskip

$
\begin{pmatrix}
0 & 0 & 0 & 3 & 1 \\
1 & 2 & -3 & 1 & -2 \\
2 & 4 & -2 & 1 & -2
\end{pmatrix}
\SkipImplies
\begin{pmatrix}
1 & 2 & -3 & 1 & -2 \\
0 & 0 & 0 & 3 & 1 \\
2 & 4 & -2 & 1 & -2
\end{pmatrix}
\SkipImplies
\begin{pmatrix}
1 & 2 & -3 & 1 & -2 \\
0 & 0 & 0 & 3 & 1 \\
0 & 0 & 4 & -1 & 2
\end{pmatrix}
\SkipImplies
\begin{pmatrix}
1 & 2 & -3 & 1 & -2 \\
0 & 0 & 4 & -1 & 2 \\
0 & 0 & 0 & 3 & 1
\end{pmatrix}
$


\bigskip

So the rank is 3.



\end{document}
