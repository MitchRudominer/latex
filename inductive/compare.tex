% This is the file compare.tex

\skipbig

\section{Comparison Lemma for
$\protect{\Game^{\R}}$-Closed Iterable Mice}

\label{section:comparison}

In this section we will give half of the proof that $\Ahyp$ is
a mouse set. We will show that if $x\in\R\intersect\cM$, where
$\cM$ is a fairly small, iterable premouse, then $x\in\Ahyp$. This
is true because, essentially, $\cN\in\Ahyp$, where $\cN\initseg\cM$ is
the $\initseg$-least initial segment of $\cM$ containing $x$.
The reason that $\cN$ is simply
definable is that $\cN$ is the unique premouse
of its ordinal height which is sound, projects to $\omega$, and
is iterable. This would be enough to see that (the real coding)
$\cN$ was in $\Ahyp$, if we could see that the property of being
iterable was sufficiently simply definable. As in section 4 of
\cite{Many_Woodins} and in \cite{Proj_WO_In_Mod}, we will define
a weakening of the notion of full iterability which is simply
definable, and yet is strong enough to compare fairly small premice.
Our weakened notion of iterability is best described using
the notion of an \emph{iteration game.}

In section 1 of \cite{Many_Woodins}, Steel defines
$W\cG_{n}(\cM,\theta)$, the $n$-maximal, weak iteration game on $\cM$, of
length $\theta$.
For our purposes here, we will need
a slight modification of the game $W\cG_{0}(\cM,\omega)$.
The set $\setof{\cM}{\text{ II wins $W\cG_{0}(\cM,\omega)$ }}$
is $\Game^{\R}$-$\Pi^1_1$. We need a notion of iterability which is
$\Game^{\R}$-closed. Below we define an iteration game called the
\emph{closed iteration game on $\cM$.}

Let $\cM$ be a countable, premouse. The \emph{closed iteration game on
$\cM$}  is identical to $W\cG_{0}(\cM,\omega)$,
except that if neither player loses during  the first
$\omega$ rounds of play, then we automatically
declare that player $II$ has won.
That is, we do not demand that the direct limit model be defined
and wellfounded
in order for player $II$ to win. Thus the game is essentially a
game on $\R$ with closed payoff for player $II$. Here are the
rules of the closed iteration game in more detail: The game is played
in $\omega$ rounds. Before beginning round $n<\omega$ we have a premouse
$\cM_n$, and an integer $k_n$ such that $\cM_n$ is $k_n$-sound.
We get started by setting $\cM_0=\cM$ and $k_0=0$. Round $n$ is played
as follows. Player $I$ begins by playing a countable, putative
$k_n$-maximal iteration tree $\cT$ on $\cM_n$. Player $II$ can then either
accept $\cT$ or play a maximal well-founded branch $b$ of $\cT$, with
the proviso that he cannot accept $\cT$ if it has a last, ill-founded
model. If it is impossible for $II$ to play then he loses at
round $n$. Suppose that $II$ does not lose at round $n$.
If $II$ acceptss $\cT$, then we let $\cM_{n+1}$ be the last model
of $\cT$. If $II$  picks a maximal wellfounded branch $b$ of  $\cT$,
then we set $\cM_{n+1}=\cM^{\cT}_{b}$. In either case we let
$k_{n+1}$ be the degree of $\cM_{n+1}$ in $\cT$. If $II$ does not
lose at any round $n<\omega$ then $II$ wins.

\begin{definition}
Let $\cM$ be a countable premouse. Then $\cM$ is \emph{$\Game^{\R}$-closed
iterable} iff player $II$ wins the closed iteration game on
$\cM$.
\end{definition}

\begin{remark}
The set of (reals coding) premice $\cM$ such that player $II$ wins the
closed iteration game on $\cM$ is a a $\Game^{\R}$-closed set.
In other words the set is coinductive.
In other words the set is $\Pi_1(\JofR{\kappa})$.
\end{remark}

Let $\cM$ and $\cN$ be countable premice. We say that $\cM$ and
$\cN$ \emph{can be compared} iff there is a countable iteration
tree $\cT$ on $\cM$ of length $\theta+1$
 and a countable iteration tree $\cU$ on $\cN$ of length $\mu+1$
such that either
\begin{itemize}
\item[(1)] $\cM^{\cT}_{\theta}$ is an initial segment of $\cM^{\cU}_{\mu}$,
and $D^{\cT}\intersect[0,\theta]_{\cT}=\emptyset$, or
\item[(2)] $\cN^{\cU}_{\mu}$ is an initial segment of $\cN^{\cT}_{\theta}$,
and $D^{\cU}\intersect[0,\mu]_{\cU}=\emptyset$.
\end{itemize}

The main result in this section says that if $\cM$ is fully iterable
and fairly small, and $\cN$ is $\Game^{\R}$-closed iterable, then $\cM$
and $\cN$ can be compared. To see that our comparison process terminates,
we will need a certain amount of \emph{generic absoluteness.}
We will get the generic absoluteness we need by assuming that there exists
infinitely many Woodin cardinals in $V$, and  then using the machinery of
Woodin's
\emph{stationary tower forcing.}
  See Chapter 9 of \cite{Martin_Book} for a thorough
treatment of this machinery. Below we simply quote one
fact from the theory
of stationary tower forcing.

\begin{proposition}[Woodin]
\label{firstprop}
Let $\angles{\delta_n \mid n\in\omega}$ be a strictly increasing sequence
of Woodin cardinals. Let $\lambda = \sup_{n} \delta_n$. Let
$\gamma > \lambda$ be any ordinal. Then there is a generic extension
of the universe, $V[G]$, such that in $V[G]$ there is an elementary
embedding $j:V\map \ULT$ such that
\begin{itemize}
\item[(i)] $\gamma$ is in the wellfounded part of $\ULT$,
\item[(ii)] $\crit(j)=\omega_1^V$, and $j(\omega_1^V)=\lambda$, and
\item[(iii)] letting $\R^*$ be the reals of $\ULT$, we have that $\R^*$
is also the set of reals in a symmetric collapse of $V$ up to $\lambda$.
\end{itemize}
\end{proposition}

See section 9.5 of \cite{Martin_Book} for a proof of this proposition.
Now we turn to the main result in this section.

\begin{remark}
Below we will use the notion of \emph{realizability}.
A mouse $\cM$ is \emph{realizable} if, roughly speaking, $\cM$ can be
embedded into a mouse $\bar{\cM}$, where $\bar{\cM}$ has the property
that the extenders on the $\bar{\cM}$ sequence all come from full
extenders in $V$. See \cite{Many_Woodins} for the precise definition
of realizable. In \cite{Many_Woodins} it is shown that if $\cM$ is
realizable, then $\cM$ is ``sufficiently iterable.'' In particular
we will use the fact that if $\cM$ is realizable then $\cM$ is
$\Game^{\R}$-closed iterable.
\end{remark}

\begin{lemma}[Comparison Lemma]
\label{ComparisonLemma}
Assume that there exists $\omega$ Woodin cardinals.
Let $\cM$ be countable, fairly small, realizable premouse. Let $\cN$ be a
countable, $\Game^{\R}$-closed iterable premouse.
Then $\cM$ and $\cN$ can be compared.
\end{lemma}
\begin{proof}
The proof of Theorem 1.10 from \cite{Many_Woodins} shows this.
We will give a sketch of the details.
In the language of that proof, when comparing $\cM$ with $\cN$,
when more than one cofinal realizable branch appears on the $\cM$-side,
or more than one cofinal $\Game^{\R}$-closed iterable branch appears
on the $\cN$-side, we dovetail in a new comparison which guarantees that a
certain ordinal $\delta$ is Woodin in the common ``lined up part'' of
the family of models we are comparing. Thus every time we dovetail
in a new comparison, we get another Woodin cardinal. Suppose we
had to dovetail in a new comparison infinitely many times. Since
$\cM$ is realizable, on the $\cM$ side we have a wellfounded direct
limit model $\cM^{\prime}$. Because we had to dovetail in a new
comparison infinitely many times, in $\cM^{\prime}$ there are
infinitely many ordinals $\delta$ which are Woodin. Thus $\cM^{\prime}$
is not fairly small. But this contradicts our assumption that
$\cM$ is fairly small. Thus we did not have to dovetail in a new
comparison infinitely many times.

In the proof of Theorem 1.10 from \cite{Many_Woodins}, we would
then be in Case 2 of the proof of the Claim. That is, we still need
to see that our comparison does not last for $\omega_1$ stages.
Suppose that our comparison does last for $\omega_1$ stages.
Since we may have had to dovetail in a new comparison some finite number of
times, the construction gives us a finite number of trees on $\cM$ with
length $\omega_1$. Let $\cT$ be one of these trees. Similarly, the
construction gives us a finite number of trees on $\cN$ with length
$\omega_1$. Let $\cU$ be one of these trees.
We can derive the usual
 contradiction if
we can show that there is a cofinal branch $b$ of $\cT$ and a cofinal
branch $c$ of $\cU$. Since $\cM$ is realizable, the proof of
Theorem 1.10 shows that there is a cofinal branch $b$ of $\cT$.
To see that there is a cofinal branch $c$ of $\cU$, we will need
to use our hypothesis that there exists $\omega$ Woodin cardinals
in $V$. Let $\lambda$ be the supremum of the $\omega$ Woodin cardinals.

Let $\R^{\prime}$ be the set of reals in some symmetric collapse of $V$
up to $\lambda$. Then there is some ordinal $\alpha$ such that
$J_{\alpha}(\R^{\prime})$ is admissible. Let $\alpha_0$ be the least
such ordinal. Since the collapse forcing is homogeneous, $\alpha_0$
does not depend on our particular choice of $\R^{\prime}$.

Let $\gamma>\alpha_0$ be any ordinal. Recall that we
have set $\kappa=\kappa^{\R}$.
Let $j:V\map \Ult$ be as in Proposition \ref{firstprop} above.
 Let $\R^*$ be the reals of $\Ult$.
By  Proposition \ref{firstprop}, $\R^*$ is also the set of reals in a
symmetric
collapse of $V$ up to $\lambda$. Thus $\alpha_0$ is the least
$\alpha$ such that $J_{\alpha}(\R^*)$ is admissible.
 Now $\alpha_0$ is in the wellfounded part of $\Ult$.
So $\Ult\models$``$\alpha_0$ is the least
$\alpha$ such that $\JalphaR$ is admissible.'' Thus $j(\kappa)=\alpha_0$.

Now $j(\cU)$ is an iteration tree on $\cN$ which properly extends
$\cU$. By case hypothesis, there is some ordinal $\beta<\omega_1$
such that for all $\eta$ with $\beta<\eta<\omega_1^V$,
$\cU\restr\eta$ has a unique, cofinal, $\Game^{\R}$-closed iterable
branch.
 (Otherwise we would have had to dovetail in a new comparison
$\omega_1$ times.)
Since $\Game^{\R}$-closed $= \Pi_1(\JofR{\kappa})$ and
$j:\JofR{\kappa}\map J_{\alpha_0}(\R^*)$ is fully elementary, we have
that for all $\eta$ with $\beta<\eta<\omega_1^{J_{\alpha_0}(\R^*)}$,
$j(\cU)\restr\eta$ has a unique, cofinal branch that is
$\Game^{\R}$-closed iterable in $J_{\alpha_0}(\R^*)$.
 Since $j(\cU)\restr\omega_1^{V}=\cU$,
 there is
a cofinal branch $c$ of $\cU$ which is definable in $L(\R^*)$ from
$\cU$ and $\alpha_0$. As the collapse forcing is homogeneous, $c\in V$.
\end{proof}

\begin{corollary}
\label{Every_Real_Is_Definable}
Assume that there exists $\omega$ Woodin cardinals. Let $\cM$ be
a countable, realizable, fairly small premouse. Then every
real in $\cM$ is ordinal definable over $\JbetaR$, for some
$\beta<\kappa^{\R}$.
\end{corollary}
\begin{proof}
Fix a real $x$ in $\cM$. Let $\gamma$ be the rank of $x$ in the order
of construction of $\cM$. We will show that $x$ is definable from
$\gamma$. Notice that $x$ is the unique real $\xprime$ such that:
\begin{quote}
there exists a countable, fairly small, $\Game^{\R}$-closed iterable
premouse $\cN$ such that $\xprime\in\cN$ and $\xprime$ is
the $\gamma$th real in
the order of construction of $\cN$.
\end{quote}
(proof:
$\cM$ witnesses that the statement above is true of the real $x$. Suppose
that $\cN$ witnessed that the statement above was true of some other
real $\xprime$. By our Comparison Lemma, $\cM$ and $\cN$ can be compared.
This implies that $\xprime=x$.) Since $\Pi_1(\JofR{\kappa})$ is
closed under real quantification, we have that
$\singleton{x}$ is $\Pi_1(\JofR{\kappa},\gamma)$. Fix a
$\Sigma_1$ formula
$\formulaphi$ such that $x$ is the unique real $\xprime$ so that
$\JofR{\kappa}\models\neg\formulaphi[\xprime,\gamma,\R]$.
Let $f:\R\map\kappa^{\R}$ be defined by: $f(x)=0$, and for $y\not= x$,
$f(y) = $ the least $\alpha<\kappa^{\R}$ such that
$\JalphaR\models\formulaphi[y,\gamma,\R]$. Notice that $f$ is
$\Sigma_1(\JofR{\kappa},\singleton{\gamma,x})$. Thus the range of $f$ is
not
cofinal in $\kappa^{\R}$. Let $\beta=\sup(\range(f))$. Then $x$ is the
unique real $\xprime$ such that
$\JbetaR\models\neg\formulaphi[\xprime,\gamma,\R]$. Thus $x$ is ordinal
definable over $\JbetaR$.
\end{proof}
