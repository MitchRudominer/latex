% This is the file martin.tex

\skipbig

\section{$\protect{\Sigma^*_n}$ Correctness:  A Proof of Martin's Theorem}

\label{section:Martin_Theorem}

In the previous section we showed that $\Ahyp=\R\intersect\cM$,
where $\cM$ is the canonical, minimal inner model for the theory
$\ZFC -$ Replacement $+$ ``There are $\omega$ Woodin cardinals
cofinal in the ordinals.'' In this section we give one application
of this characterization of $\Ahyp$. We show that every real
$x$ which is definable over $\JofR{\kappa}$ is in $\Ahyp$.
We do this by showing that $x\in\cM$. Thus we are giving a purely
inner-model-theoretic proof of Martin's theorem that
every $\Sigma^*_n$ real is in the largest countable inductive set.
The reason that the real $x$ is in $\cM$ is that for each $n\in\omega$,
 $\cM$ can compute
$\Sigma_n(\JofR{\kappa})$ truth. This correctness result
for $\cM$ is
the main content of this section.



Let $\cM$ be a countable, realizable premouse such that $\cM$ is
fairly big, but every proper initial segment of $\cM$ is fairly
small. Corollary \ref{Is_Correct} tells us that
 there is an ordinal $\alpha$ in $\cM$ so that
$(\JalphaR)^{\cM}$ agrees with $\JofR{\kappa}$ on $\Sigma_1$
facts about reals in $\cM$. Consider the following question.
Is it possible to identify $\alpha$ while working in $\cM$? The
next lemma tells us that the answer is yes.

The statement of the following lemma mentions the term cutpoint. Let $\cM$
be a premouse and $\delta\in\ORD^{\cM}$. We say that $\delta$ is a
\emph{cutpoint} of $\cM$ iff for no extender $E$ on the $\cM$ sequence do
we have that $\crit(E)<\delta\leq\length(E)$.

\begin{lemma}
\label{lemmaA}
Assume that there exists $\omega$ Woodin cardinals in $V$.
Let $\cM$ be a countable, realizable premouse such that $\cM$ is
fairly big, but every proper initial segment of $\cM$ is fairly
small.  Let $\delta$ be a  cardinal of $\cM$ and a cutpoint of $\cM$.
Suppose that
 $G$ is $\cM$-generic over $\Col(\omega,\delta)$, with $G\in V$.
Let $\R^*=\R\intersect\cM[G]$.
Suppose that $\alpha_0\in\cM$ is an ordinal
such that whenever $x\in\R^*$ and $\formulaphi$ is a $\Sigma_1$ formula
then $\JofR{\kappa}\models\formulaphi[x,\R]$ iff
 $J_{\alpha_0}(\R^*)\models\formulaphi[x,\R^*]$.  Then $\alpha_0$ is
the least $\alpha\in\cM$ such that there is a wellorder of $\R^*$ which
is definable over $J_{\alpha}(\R^*)$.
\end{lemma}
\begin{proof}
First let us see that there is a wellorder of $\R^*$ which is definable
over $J_{\alpha_0}(\R^*)$.
Notice that for any $\gamma<(\delta^+)^{\cM}$,
 $\cJ_{\gamma}^{\cM}$  is
in $J_{\alpha_0}(\R^*)$ as it is coded by an element of $\R^*$.
Similarly, $G$ is in $J_{\alpha_0}(\R^*)$.
Also notice that every element of $\R^*$ has a $G$-name
in $\cM$ which is essentially a subset of $\delta$.
So it suffices to see that there is a wellorder of
$\cP(\delta)\intersect\cM$
which is definable over $J_{\alpha_0}(\R^*)$.
Let $<_{\cM}$ be the order
of construction in $\cM$. We claim that
 $<_{\cM}\;\restr \cP(\delta)$ is
definable over $J_{\alpha_0}(\R^*)$ with $J_{\delta}^{\cM}$ as
a parameter.


Let $S(X,Y)$ be the binary relation on
$\cP(\delta)\intersect J_{\alpha_0}(\R^*)$ defined by the following:
$S(X,Y)$ holds iff
\begin{quote}
there exists a fairly small
premouse $\cN$ in $J_{\alpha_0}(\R^*)$ which is countable
in $J_{\alpha_0}(\R^*)$ such that $J^{\cN}_{\delta}=J^{\cM}_{\delta}$,
and $\delta$ is a cutpoint of $\cN$, and $\cN$ is $\Game^{\R}$-closed
iterable, and $X,Y\in\cN$ and $X$ is less than $Y$ in the order of
construction of $\cN$.
\end{quote}
Clearly $S$ is definable over
$J_{\alpha_0}(\R^*)$ with $J_{\delta}^{\cM}$ as
a parameter.  We claim that $S(X,Y)$ iff $X<_{\cM}Y$. First suppose that
$X<_{\cM}Y$. Let $\cN$ be the $\initseg$-least initial segment of $\cM$
such that $X,Y\in\cN$. Then clearly $\cN$ witnesses that $S(X,Y)$.

Converesly, suppose that $S(X,Y)$, and let $\cN$ be a premouse that
witnesses this. Let $\gamma=(\delta^+)^{\cM}$. By our comparison theorem,
$\cN$ and $J^{\cM}_{\gamma}$ can be compared. Furthermore, since
$\cN$ and $\cM$ agree through $\delta$ and $\delta$ is a cutpoint of
$\cN$ and $\cM$, the iteration maps resulting from the comparison have
critical points $\geq \delta$. If $X,Y\in\cM$, it follows that
$X<_{\cM}Y$ and we are done. It takes a short argument to see that
it must be true that $X,Y\in\cM$: Let $\bar{\cN}$ and $\bar{\cM}$ be
the mice which result from the comparison of $\cN$ with $J^{\cM}_{\gamma}$.
Suppose for example that $X\notin\cM$. Then it must be the case that
$\bar{\cM}\properseg\bar{\cN}$ and that $X\in\bar{\cN}-\bar{\cM}$. But this
implies that $\card(\gamma)^{\bar{\cN}}\leq\delta$. Thus there is a
set $Z\subseteq\delta$ in $\bar{\cN}$ such that $Z$ codes a wellorder of
order type $\gamma$. But then $Z\in\cN$ and so $Z\in J_{\alpha_0}(\R^*)$.
But this is impossible as $\gamma=(\omega_1)^{J_{\alpha_0}(\R^*)}$.
This contradiction shows that $X,Y\in\cM$, and so $X<_{\cM}Y$.
So we have shown that there is a wellorder of $\R^*$ which is definable
over $J_{\alpha_0}(\R^*)$.

To see that $\alpha_0$ is least with this property,
we must show that there is no
wellorder of $\R^*$ which is \emph{in} $J_{\alpha_0}(\R^*)$. By
$\Sigma_1$-correctness, it suffices to see that there is no
wellorder of $\R$ in $\JofR{\kappa}$. But this follows from our
assumption that there are $\omega$ Woodin cardinals in $V$. In fact
our large cardinal hypothesis implies that every game in
$\JofR{\kappa}$ (and more) is determined. [proof: Using Woodin's
stationary tower forcing as in the end of the
proof of our comparison lemma, Lemma \ref{ComparisonLemma}, we get
a fully elementary generic embedding
$j:\JofR{\kappa}\map J_{\lambda}(\R^{\prime})$, where $\lambda$ is
some ordinal, and where $\R^{\prime}$ is the set of reals in a symmetric
collapse of $V$ up to the sup of the $\omega$ Woodin cardinals. Also
using stationary tower forcing, Woodin has shown that
$L(\R^{\prime})\models\AD$. (See chapter 9 of \cite{Martin_Book}.)
In particular $J_{\lambda}(\R^{\prime})\models\AD$, and so
$\JofR{\kappa}\models\AD$.]
\end{proof}

In an earlier draft of this paper the statement of the above lemma did
not include the hypothesis that $\delta$ is a cutpoint of $\cM$.
We would like to thank the referee for catching a serious blunder
in our proof and providing the necessary correction.

The following is the main technical result of this section. The
lemma says that if $\cM$ has $\omega$ Woodin cardinals cofinal in its
ordinals, then $\cM$ can
compute $\Sigma_n(\JofR{\kappa})$ truth, for every $n$.
Our proof is by induction on $n$. In order to carry out this inductive
proof we need to make the inductive hypothesis \emph{uniform}
in $\cM$. Unfortunately, this slightly complicates the
statement of the lemma.

\begin{lemma}
\label{MoreCorrectness}
 Assume that there exists $\omega$ Woodin cardinals in $V$.
Let $n\geq 1$, and let $\formulaphi$ be a $\Sigma_n$ formula.
Then there is another formula $\psi=\psi_{\varphi}$ such that
whenever $\cM$ is a countable, realizable premouse, and $\cM$ is
fairly big but every proper initial segment of $\cM$ is fairly
small, and $\gamma$ is a cardinal of $\cM$, and
 $G$ is $\cM$-generic over $\Col(\omega,\gamma)$, with $G\in V$,
and $\delta_0, \delta_1, \dots \delta_n$ are Woodin cardinals of
$\cM$ with $\gamma<\delta_0<\delta_1<\cdots<\delta_n$, and
$x\in\R\intersect\cM[G]$,  then
$\JofR{\kappa}\models\formulaphi[x,\R]$ iff
$J^{\cM}_{\delta_n}[G]\models\psi[x,\delta_0,\dots,\delta_{n-1}]$.
\end{lemma}
\begin{proof}
By induction on $n$. First let $n=1$ and let $\formulaphi$ be a
$\Sigma_1$ formula. Our hypotheses on $\cM$  imply that $\cM$ is tame,
and thus  $\delta_0$ is a cutpoint of $\cM$. This will allow us to apply
the previous lemma with $\delta=\delta_0$.
We can take $\psi=\psi_{\varphi}$ to be a
formula such that, given any $\cM$, $\gamma$,  $G$ and
$\delta_0<\delta_1$ as above, and
letting $x\in\R\intersect\cM[G]$,  we have
that $J_{\delta_1}^{\cM}[G]\models\psi[x,\delta_0]$ iff
it is forced over $J_{\delta_1}^{\cM}[G]$ that in a generic extension
via $\Col(\omega,\delta_0)$ the following is true:
\begin{quote}
$J_{\alpha_0}(\R)\models\formulaphi[x]$, where $\alpha_0$ is the least
$\alpha$ such that there is a wellorder of $\R$ definable
over $J_{\alpha}(\R)$.
\end{quote}
Clearly there is such a formula $\psi$.
Furthermore
such a $\psi$ works by  Corollary
\ref{Is_Correct} and Lemma \ref{lemmaA}.


Next let $n>1$ and let $\formulaphi$ be a $\Sigma_n$ formula. Let $\theta$
be a $\Sigma_{n-1}$ formula  so that
$\formulaphi(v_1,v_2) \leftrightarrow (\exists v_0)
\neg\theta(v_0,v_1,v_2)$. Let $\psi^{\prime}=\psi_{\theta}$.
($\psi_{\theta}$ exists by induction.)
Then we can
take $\psi=\psi_{\varphi}$ to be a formula such that, given any
$\cM$, $\gamma$, $G$ and $\delta_0, \dots \delta_n$
 as above, and letting $x\in\R\intersect\cM[G]$,
we have that
$J_{\delta_n}^{\cM}[G]\models\psi[x,\delta_0,\dots\delta_{n-1}]$
iff it is forced over $J_{\delta_n}^{\cM}[G]$ that in a generic
extension via $\Col(\omega,\delta_0)$, there is a real $y$ such
that $\neg\psi^{\prime}[y,x,\delta_1,\dots,\delta_{n-1}]$. Clearly
 there is such a formula $\psi$. Let us see that such
a $\psi$ works.

Fix $\cM$, $\gamma$, $G$ and $\delta_0,\dots,\delta_n$ as above.
Let $x\in\R\intersect\cM[G]$. First suppose
that $J_{\delta_n}^{\cM}[G]\models
\psi[x,\delta_0,\dots\delta_{n-1}]$. Let $H$ be $\cM[G]$-generic
over $\Col(\omega,\delta_0)$, and let
$\R^{*}=\R\intersect\cM[G][H]$. By definition of $\psi$, there is
a $y\in\R^{*}$ such that
$J_{\delta_n}^{\cM}[G][H]\models
\neg\psi^{\prime}[x,y,\delta_1,\dots,\delta_{n-1}]$. Since
$\psi^{\prime}=\psi_{\theta}$, we have by induction that
$\JofR{\kappa}\models\neg\theta[y,x,\R]$.
Thus $\JofR{\kappa}\models\formulaphi[x,\R]$.

Conversely, suppose that $\JofR{\kappa}\models\formulaphi[x,\R]$.
Then there is a real $y$ such that
$\JofR{\kappa}\models\neg\theta[y,x,\R]$. Fix such a $y$.
By Lemma \ref{MakeRealGeneric}, there is
is a partial order $\Q\subseteq J^{\cM}_{\delta_0}$, with
$\Q\in \cM$, and there is an
iteration tree $\cT$ on $\cM$ of countable length $\mu+1$ such that
\begin{itemize}
\item[(a)] $\cM^{\cT}_{\mu}$ is realizable, and
\item[(b)] $D^{\cT}=\emptyset$ so that $i^{\cT}_{0,\mu}$ is defined,
and
\item[(c)] $\crit(E^{\cT}_{\xi}) > \gamma$ for all $\xi<\mu$
(so $G$ is $\cM^{\cT}_{\mu}$-generic over $\Col(\omega,\gamma)$), and
\item[(d)] $y$ is $\cM^{\cT}_{\mu}[G]$-generic over
$i^{\cT}_{0,\mu}(\Q)$.
\end{itemize}
Let $\cM^{\prime}=\cM^{\cT}_{\mu}$, let $i=i^{\cT}_{0,\mu}$, and
let $\delta_k^{\prime}=i(\delta_k)$ for $k\geq 0$.
Let $H$ be $\cM^{\prime}[G]$-generic
over $\Col(\omega,\delta_1^{\prime})$ with $y\in\cM^{\prime}[G][H]$.
By induction, since $\psi^{\prime}=\psi_{\theta}$,
$J_{\delta_n^{\prime}}^{\cM^{\prime}}[G][H]\models
\neg\psi^{\prime}[y,x,\delta_1^{\prime},\dots,\delta^{\prime}_{n-1}]$.
Since the collapse forcing is homogeneous, it is forced over
$J_{\delta_n^{\prime}}^{\cM^{\prime}}[G]$
that in a generic
extension via $\Col(\omega,\delta_0^{\prime})$, there is a real $y$ such
that $\neg\psi^{\prime}[y,x,\delta_2^{\prime},\dots,\delta_{n-1}^{\prime}]$.
Thus $J_{\delta_n^{\prime}}^{\cM^{\prime}}[G]\models
\psi[x,\delta_0^{\prime},\dots,\delta_{n-1}^{\prime}]$.
Since $\crit(i)>\gamma$, it follows that
$J_{\delta_n}^{\cM}[G]\models
\psi[x,\delta_0,\dots,\delta_{n-1}]$.
\end{proof}

\begin{theorem}
Assume there exists $\omega$ Woodin cardinals in $V$. Let $\cM$ be
a countable, realizable premouse such that $\cM$ is fairly big,
but every proper initial segment of $\cM$ is fairly small.
Let $x$ be a real which is definable over $\JofR{\kappa}$.
Then $x\in\cM$.
\end{theorem}
\begin{proof}
Let $n\geq1$ be such that, as a subset of $\omega$,
 $x$ is $\Sigma_n$ definable over
$\JofR{\kappa}$. Then by the previous lemma, $x$ is definable
over $J_{\delta_n}^{\cM}$ from $\delta_0,\dots,\delta_{n-1}$,
where $\delta_0,\cdots,\delta_n$ are the first $n+1$ Woodin cardinals
of $\cM$.
\end{proof}

\begin{corollary}
Every $\Sigma^*_n$ real is in the largest countable inductive set.
\end{corollary}
