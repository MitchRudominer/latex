% This is the file intro.tex

\skipbig

\section{Introduction}

\label{section:intro}

 This paper is best understood as being part of
a certain programme in set theory---a
programme whose goal is to explore the connections between descriptive
set theory and inner model theory. We begin by describing some
of the other contributions to this programme. The following several
results are all of a common theme:  From descriptive set theory one
takes a countable, definable set of reals, $A$. It is then shown that
$A=\R\intersect\cM$, where $\cM$ is some canonical model from inner
model theory. In technical terms, $\cM$ is a mouse. Consequently we
say that $A$ is a \emph{mouse set.}

We begin with some results due to D. A. Martin, J. R. Steel, and
H. Woodin. First a definition.
If $\xi$ is a countable ordinal and $x\in\R$  and $n\geq2$, then  we say
that
$x\in\Delta^1_n(\xi)$ if for every real $w\in\WO$ such that $|w|=\xi$,
$x\in\Delta^1_n(w)$. Set
$$A_n = \setof{x\in\R}{(\exists\xi<\omega_1)\; x\in\Delta^1_n(\xi)}.$$
In short, $A_n$ is the set of reals which are $\Delta^1_n$
in a countable ordinal.
Under the hypothesis of Projective Determinacy ($\PD$),
the sets $A_n$ are of
much interest to descriptive set theorists.
If $n\geq2$ is even then $A_n = C_n$, the largest
countable $\Sigma^1_n$ set. If $n\geq3$ is odd then
$A_n=Q_n$, the largest countable $\Pi^1_n$ set which is closed downward
under
$\Delta^1_n$ degrees. The sets $A_n$ have been studied extensively by
descriptive set theorists. See for example \cite{Kechris1} and
\cite{Q_Theory}.

The work of Martin, Steel, and Woodin shows that there is a connection
between the sets $A_n$, and certain inner models of the form
$L[\vec{E}]$, where $\vec{E}$ is a sequence of extenders.
In the paper ``Projectively Wellordered Inner Models''
\cite{Proj_WO_In_Mod}, Martin,
Steel, and Woodin
prove the following theorem:

\begin{theorem}[Martin, Steel, Woodin]
\label{FirstThm}
Let $n\geq1$ and suppose that there are $n$ Woodin
cardinals with a measurable cardinal above them. Let
$\cM_n$ be the canonical $L[\vec{E}]$ model with $n$ Woodin cardinals. Then
$$\R\intersect\cM_n=A_{n+2}.$$
\end{theorem}

The above theorem is also true with $n=0$. The ``canonical $L[\vec{E}]$
model with $0$ Woodin cardinals'' is just the model $L$. If there is
a measurable cardinal, then $A_2=C_2=\R\intersect L$. Actually, this result
is true under the weaker hypothesis that $\R\intersect L$ is countable.

Now let $A^*$ be the collection of reals which are ordinal definable in
$\LofR$.
To emphasize the parallel with the sets $A_n$ above, note that assuming
that every game in $L(\R)$ is determined ($\AD^{\LofR}$), we have:
$$A^*=\setof{x\in\R}{(\exists\xi<\omega_1)\;
x\in(\Delta^2_1)^{\LofR}(\xi)},$$
and $A^*$ is the largest countable $(\Sigma^2_1)^{\LofR}$ set of reals.

In the paper ``Inner Models with Many Woodin Cardinals''
\cite{Many_Woodins}, Steel
and Woodin prove
the following theorem:

\begin{theorem}[Steel, Woodin]
\label{SecondThm}
 Suppose that there are $\omega$ Woodin
cardinals with a measurable cardinal above them. Let
$\cM_{\omega}$ be the canonical $L[\vec{E}]$ model with $\omega$ Woodin
cardinals. Then
$$\R\intersect\cM_{\omega}=A^*.$$
\end{theorem}

Theorems \ref{FirstThm} and \ref{SecondThm} are obviously similar to
each other. Let us establish some terminology which will allow us to
describe this similarity.
Given a pointclass $\Delta$ let us set:
$$A_{\Delta}=\setof{x\in\R}{(\exists\xi<\omega_1)\; x\in\Delta(\xi)}.$$
Theorems  \ref{FirstThm} and \ref{SecondThm} both establish
results of the form:
$A_{\Delta}$ is a mouse set. Theorem \ref{FirstThm} does this for
$\Delta=\Delta^1_n$. Theorem \ref{SecondThm} does this for
$\Delta=(\Delta^2_1)^{\LofR}$. Furthermore, the two theorems do
not merely show that there \emph{exists some} mouse $\cM$ such
that $A_{\Delta}=\R\intersect\cM$. The theorems actually describe
$\cM$ in terms of the large cardinal axioms that it satisfies.

There is quite a bit of
 room between theorems \ref{FirstThm} and \ref{SecondThm}.
One  way to describe this room is in terms of the levels of the model
$\LofR$. For convenience, we will use the Jensen $J$-hierarchy for
$\LofR$. We define $J_1(\R)$ to be $\R\union\HF$. For
$\lambda$ a limit ordinal we set
$\JofR{\lambda}=\Union{\alpha<\lambda}\JalphaR$. For $\alpha\geq 1$
we set $\JofR{\alpha+1}$ to be the rudimentary closure of
$\JalphaR\union\singleton{\JalphaR}$. Then $\LofR=\Union{\alpha}\JalphaR$.
See \cite{Scales_In_LofR} or \cite{Mouse_Sets} for a more detailed
discussion. Now let $\delta=(\delta^2_1)^{\LofR}$
be the least ordinal such that
$\JofR{\delta}$ is a $\Sigma_1$ elementary submodel of $\LofR$.
Then we can characterize the room between theorems \ref{FirstThm}
and \ref{SecondThm} by pointing out that the first theorem is
concerned  with the pointclasses $\Delta_n(\JofR{1})$, and the
second theorem is concerned with the pointclass $\Delta_1(\JofR{\delta})$.
Assuming $\AD^{\LofR}$, $\delta$ is a large cardinal in $\LofR$ (for
example it is an inaccessible limit of inaccessibles.)
From this point of view there is quite a bit of
 room between
theorems \ref{FirstThm} and \ref{SecondThm}.

Given the above discussion,
it is natural to conjecture that $A_{\Delta}$ is a mouse set for
$\Delta=\Delta_n(\JalphaR)$, for \emph{all} $\alpha$ and \emph{all}
$n$. In \cite{Mouse_Sets} we do make such a conjecture (in a slightly
more precise formulation.) In that paper we are not able to fully
prove the conjecture, but we are able to prove the conjecture for
\emph{some} $\alpha$ and \emph{some} $n$.
Our main theorem in \cite{Mouse_Sets} is
similar to theorems \ref{FirstThm} and \ref{SecondThm} above in that
we not only show that there \emph{exists some} mouse $\cM$ such that
$A_{\Delta}=\R\intersect\cM$, we actually describe $\cM$ in terms of
the large cardinal axioms which it satisfies.  Independently,
and using entirely different techniques, Woodin
has shown that $A_{\Delta}$ is
a mouse set for $\Delta=\Delta_1(\JofR{\lambda})$, for all limit
ordinals $\lambda$. See \cite{Woodins_Mouse_Sets}.
Woodin's proof shows only that there
\emph{exists some} mouse $\cM$ such that
$A_{\Delta}=\R\intersect\cM$. The mouse $\cM$ is not described in terms
of the large cardinal axioms which it satisfies.

Now we turn to the topic of the current paper. In this paper we will show
that $A_{\Delta}$ is a mouse set, for $\Delta =$ the pointclass of
\emph{hyperprojective} sets. (See below for the definition of
hyperprojective.)
 Assuming determinacy, $A_{\Delta}$ is
the largest countable inductive set. Thus there are obvious parallels
between our result, and the results of theorems \ref{FirstThm} and
\ref{SecondThm} above. We may also compare our result with those of
theorems \ref{FirstThm} and \ref{SecondThm} by looking at  levels
of $\LofR$. Another way of describing the main result of this paper
is to say that we will show that $A_{\Delta}$ is a mouse set, for
$\Delta=\Delta_1(\JofR{\kappa})$, where $\kappa=\kappa^{\R}$ is the
least ordinal such that $\JofR{\kappa}$ is admissible. Since
$1<\kappa^{\R}<(\delta^2_1)^{\LofR}$, the result of this paper is
strictly \emph{between} those of theorem \ref{FirstThm} and
and theorem \ref{SecondThm} above.

We will feel free to use terms and concepts from inner model theory.
In particular, we will expect the reader to have some familiarity
with the papers \cite{FSIT} and \cite{Many_Woodins}.

Some of the research for this paper was done while I was a
graduate student at UCLA, although the results do not appear in my
 PhD thesis. I would like to thank my thesis advisor,
Professor John Steel.

\skipmed

\noindent
\textbf{The Pointclass of Inductive Sets}

\skipsmall

The pointclass of inductive sets has several different equivalent
descriptions. (Throughout this paper we shall only be using the
term ``inductive'' in the \emph{lightface} sense. This is what is
called ``absolutely inductive'' in \cite{DST}.)  In section 7C of
\cite{DST} the inductive sets are defined as the \emph{positive
analytical inductive sets on $\R$.} It is shown there that the class
of all inductive sets is the smallest Spector pointclass which is closed
under both $\forall^{\R}$ and $\exists^{\R}$. The compliment of an
inductive set is called \emph{co-inductive.} If $X$ is both inductive
and co-inductive then we say that $X$ is \emph{hyperprojective}. As we
shall not have any need for the concept of
``positive analytical inductive,'' we shall not go into any more detail
about it here. Instead we give two other equivalent  characterizations
of the inductive sets.

\begin{definition}
$\kappa^{\R}$ is the least ordinal $\kappa$ so that
$\JofR{\kappa}$ is admissible.
\end{definition}

For convenience we  make the following convention.

\begin{notation}
For the rest of the paper  $\kappa$ will refer exclusively
to $\kappa^{\R}$.
\end{notation}

 A set $X\subset\R$ is inductive iff
$X$ is $\Sigma_1$ definable over the model $\JofR{\kappa}$, with
the point $\R$ as a parameter. As we shall always allow the point
$\R$ as a parameter in definitions over the models $\JalphaR$, we suppress
mention of it in our notation. Thus we write that $X$ is inductive
iff $X$ is $\Sigma_1(\JofR{\kappa})$. (See \cite{Scales_In_LofR} or
\cite{Mouse_Sets} for a more detailed discussion of definability over
the models $\JalphaR$.) Similarly, $X$ is co-inductive iff $X$ is
$\Pi_1(\JofR{\kappa})$ and $X$ is hyperprojective iff $X$ is
$\Delta_1(\JofR{\kappa})$.

\begin{definition}
$\Ahyp = $ the set of all $x\in\R$ such that
$(\exists\alpha<\kappa^{\R})$ so that $x$ is  definable over
$\JalphaR$ from ordinal parameters.
\end{definition}

It is not difficult to see that $x\in \Ahyp$ iff
 $\singleton{x}$ is $\Sigma_1(\JofR{\kappa},\singleton{\xi})$, for some
ordinal $\xi$, iff
 $\singleton{x}$
is $\Delta_1(\JofR{\kappa},\singleton{\xi})$.
 Assuming determinacy  $\Ahyp$ is countable,
and so we may take $\xi$ to be a countable ordinal. Thus
$\Ahyp$ is what we were above calling $A_{\Delta}$, for
$\Delta = $ the pointclass of hyperprojective sets. Also assuming
determinacy, it is not difficult to see that $\Ahyp$ is the
largest countable inductive set.

In this paper we will show that $\Ahyp$ is a mouse set, that is that
$\Ahyp=\R\intersect\cM$ for some mouse $\cM$. There are
two halves to this proof. In section \ref{section:comparison} we use
a comparison argument to show that $\R\intersect\cM\subseteq\Ahyp$,
and in section \ref{section:correctness} we prove a correctness
theorem which implies that $\Ahyp\subseteq\R\intersect\cM$.

Let $\Sigma^*_0 =$ the class of all sets $X\subseteq\R$ such that
$X=X_1\union X_2$ where $X_1$ is inductive and $X_2$ is coinductive.
 Let $\Pi^*_n$ be the class of complements of $\Sigma^*_n$ sets
and let $\Sigma^*_{n+1}$ be the class of projections of $\Pi^*_n$ sets.
Equivalently,  $\Sigma^*_n=\Sigma_{n+1}(\JofR{\kappa})$, for $n\geq 1$.
(To see this use the fact that $\JofR{\kappa}$ projects to $\R$.
See \cite{Scales_In_LofR}.) In \cite{Largest_This_That}, Martin
shows that $\Ahyp$ is equal to the set of reals which are
$\Sigma^*_n$ for some $n$. In section \ref{section:Martin_Theorem}
we reprove one direction of Martin's theorem using
purely inner-model-theoretic techniques. Using our characterization
of $\Ahyp$ as $\R\intersect\cM$ for some premouse $\cM$, we show that
every $\Sigma^*_n$ real $x$ is in $\Ahyp$ by showing that $x\in\cM$.

We shall need one other characterization of the inductive sets: A set
$A\subseteq\R$ is inductive iff $A$ is $\Game^{\R}$-open.  That is
iff there is a function $x\mapsto G_x$ such that for all reals $x$,
$x\in A \leftrightarrow $ player $I$ wins $G_x$, where $G_x$ is
an open game on $\R$ which is continuously associated to  $x$.
More precisely, $A$ is inductive
iff there is some arithmetic set
$P\subseteq \omega^{<\omega}\times(\omega^{<\omega})^{<\omega}$ such
that $x\in A$ iff
$$(\exists y_1\in\R) (\forall y_2\in\R) (\exists y_3\in\R)
\cdots (\exists n\in\omega) \;
P(x\restr n, y_1\restr n, y_2\restr n,\dots, y_n\restr n).$$
Similarly, $A$ is co-inductive iff $A$ is $\Game^{\R}$-closed.

\skipmed

\noindent
\textbf{Fairly Small Premice}

\skipsmall

In this paper we will show that $\Ahyp$ is a mouse set.
We shall not only show that there exists some mouse $\cM$ such that
$\Ahyp=\R\intersect\cM$, we shall describe $\cM$ in terms
of the large cardinal axioms which it satisfies. By examining
theorems \ref{FirstThm} and \ref{SecondThm} above, we can see that
$\cM$ must satisfy a large cardinal axiom which is stronger than
$\ZFC + $ ``there exists $n$ Woodin cardinals'', for each $n$, but weaker
than $\ZFC + $ ``there exists $\omega$ Woodin cardinals''. To obtain
the large cardinal axiom which $\cM$ will satisfy, we start with the
hypothesis $\ZFC + $ ``there exists $\omega$ Woodin cardinals'', and
we remove the assumption that there are any ordinals above the
supremum of the $\omega$ Woodin cardinals.

The mice used in section 4 of \cite{Many_Woodins} are called $\omega$-small.
The mice used in \cite{Proj_WO_In_Mod} are called $n$-small.
We need a notion which fits between these two notions. Without any
thought as to its use beyond this paper, we will adopt the term
\emph{fairly small.}

\begin{definition}
Let $\cM$ be a premouse. Then we will say that $\cM$ is
\emph{fairly big} iff there is an initial segment
$\cN\initseg\cM$ such that there is an increasing
$\omega$-sequence of ordinals
$\delta_1< \delta_2< \delta_3,\dots$ such that each $\delta_i$
is a Woodin cardinal of $\cN$. If $\cM$ is not fairly big then
we will say the $\cM$ is \emph{fairly small}.
\end{definition}
