%%This is a very basic article template.
%%There is just one section and two subsections.
\documentclass{article}

\usepackage{amsmath,amssymb,latexsym,eucal,amsthm}
%%%%%%%%%%%%%%%%%%%%%%%%%%%%%%%%%%%%%%%%%%%%%
% Common Set Theory Constructs
%%%%%%%%%%%%%%%%%%%%%%%%%%%%%%%%%%%%%%%%%%%%%

\newcommand{\setof}[2]{\left\{ \, #1 \, \left| \, #2 \, \right.\right\}}
\newcommand{\lsetof}[2]{\left\{\left. \, #1 \, \right| \, #2 \,  \right\}}
\newcommand{\bigsetof}[2]{\bigl\{ \, #1 \, \bigm | \, #2 \,\bigr\}}
\newcommand{\Bigsetof}[2]{\Bigl\{ \, #1 \, \Bigm | \, #2 \,\Bigr\}}
\newcommand{\biggsetof}[2]{\biggl\{ \, #1 \, \biggm | \, #2 \,\biggr\}}
\newcommand{\Biggsetof}[2]{\Biggl\{ \, #1 \, \Biggm | \, #2 \,\Biggr\}}
\newcommand{\dotsetof}[2]{\left\{ \, #1 \, : \, #2 \, \right\}}
\newcommand{\bigdotsetof}[2]{\bigl\{ \, #1 \, : \, #2 \,\bigr\}}
\newcommand{\Bigdotsetof}[2]{\Bigl\{ \, #1 \, \Bigm : \, #2 \,\Bigr\}}
\newcommand{\biggdotsetof}[2]{\biggl\{ \, #1 \, \biggm : \, #2 \,\biggr\}}
\newcommand{\Biggdotsetof}[2]{\Biggl\{ \, #1 \, \Biggm : \, #2 \,\Biggr\}}
\newcommand{\sequence}[2]{\left\langle \, #1 \,\left| \, #2 \, \right. \right\rangle}
\newcommand{\lsequence}[2]{\left\langle\left. \, #1 \, \right| \,#2 \,  \right\rangle}
\newcommand{\bigsequence}[2]{\bigl\langle \,#1 \, \bigm | \, #2 \, \bigr\rangle}
\newcommand{\Bigsequence}[2]{\Bigl\langle \,#1 \, \Bigm | \, #2 \, \Bigr\rangle}
\newcommand{\biggsequence}[2]{\biggl\langle \,#1 \, \biggm | \, #2 \, \biggr\rangle}
\newcommand{\Biggsequence}[2]{\Biggl\langle \,#1 \, \Biggm | \, #2 \, \Biggr\rangle}
\newcommand{\singleton}[1]{\left\{#1\right\}}
\newcommand{\angles}[1]{\left\langle #1 \right\rangle}
\newcommand{\bigangles}[1]{\bigl\langle #1 \bigr\rangle}
\newcommand{\Bigangles}[1]{\Bigl\langle #1 \Bigr\rangle}
\newcommand{\biggangles}[1]{\biggl\langle #1 \biggr\rangle}
\newcommand{\Biggangles}[1]{\Biggl\langle #1 \Biggr\rangle}

\newcommand{\force}[1]{\Vert\!\underset{\!\!\!\!\!#1}{\!\!\!\relbar\!\!\!%
\relbar\!\!\relbar\!\!\relbar\!\!\!\relbar\!\!\relbar\!\!\relbar\!\!\!%
\relbar\!\!\relbar\!\!\relbar}}
\newcommand{\nforce}[1]{\Vert\!\underset{\!\!\!\!\!#1}{\!\!\!\relbar\!\!\!%
\relbar\!\!\relbar\!\!\relbar\!\!\!\relbar\!\!\relbar\!\!\relbar\!\!\!%
\relbar\!\!\not\relbar\!\!\relbar}}
\newcommand{\forcein}[2]{\overset{#2}{\Vert\underset{\!\!\!\!\!#1}%
{\!\!\!\relbar\!\!\!\relbar\!\!\relbar\!\!\relbar\!\!\!\relbar\!\!\relbar\!%
\!\relbar\!\!\!\relbar\!\!\relbar\!\!\relbar\!\!\relbar\!\!\!\relbar\!\!%
\relbar\!\!\relbar}}}

\newcommand{\pre}[2]{{}^{#2}\!{#1}}

\newcommand{\restr}{\!\!\upharpoonright\!}

%%%%%%%%%%%%%%%%%%%%%%%%%%%%%%%%%%%%%%%%%%%%%
% Set-Theoretic Connectives
%%%%%%%%%%%%%%%%%%%%%%%%%%%%%%%%%%%%%%%%%%%%%

\newcommand{\intersect}{\cap}
\newcommand{\union}{\cup}
\newcommand{\Intersection}[1]{\bigcap\limits_{#1}}
\newcommand{\Union}[1]{\bigcup\limits_{#1}}

%%%%%%%%%%%%%%%%%%%%%%%%%%%%%%%%%%%%%%%%%%%%%
% Miscellaneous
%%%%%%%%%%%%%%%%%%%%%%%%%%%%%%%%%%%%%%%%%%%%%
\newcommand{\defeq}{=_{\text{def}}}
\newcommand{\Turingleq}{\leq_{\text{T}}}
\newcommand{\Turingless}{<_{\text{T}}}
\newcommand{\lexleq}{\leq_{\text{lex}}}
\newcommand{\lexless}{<_{\text{lex}}}
\newcommand{\Turingequiv}{\equiv_{\text{T}}}

%%%%%%%%%%%%%%%%%%%%%%%%%%%%%%%%%%%%%%%%%%%%%
% Constants
%%%%%%%%%%%%%%%%%%%%%%%%%%%%%%%%%%%%%%%%%%%%%
\newcommand{\R}{\mathbb{R}}
\newcommand{\Rplus}{\mathbb{R}^{+}}
\renewcommand{\P}{\mathbb{P}}
\newcommand{\Q}{\mathbb{Q}}
\newcommand{\Z}{\mathbb{Z}}
\newcommand{\C}{\mathbb{C}}
\newcommand{\N}{\mathbb{N}}
\newcommand{\LofR}{L(\R)}
\newcommand{\JofR}[1]{J_{#1}(\R)}
\newcommand{\SofR}[1]{S_{#1}(\R)}
\newcommand{\JalphaR}{\JofR{\alpha}}
\newcommand{\JbetaR}{\JofR{\beta}}
\newcommand{\JlambdaR}{\JofR{\lambda}}
\newcommand{\SalphaR}{\SofR{\alpha}}
\newcommand{\SbetaR}{\SofR{\beta}}
\DeclareMathOperator{\ORD}{OR}
\DeclareMathOperator{\Ord}{OR}
\DeclareMathOperator{\WO}{WO}
\DeclareMathOperator{\OD}{OD}
\DeclareMathOperator{\HOD}{HOD}
\DeclareMathOperator{\HC}{HC}
\DeclareMathOperator{\WF}{WF}
\DeclareMathOperator{\HF}{HF}
\DeclareMathOperator{\AR}{AR}
\newcommand{\One}{I}
\newcommand{\ONE}{I}
\newcommand{\Two}{II}
\newcommand{\TWO}{II}

%%%%%%%%%%%%%%%%%%%%%%%%%%%%%%%%%%%%%%%%%%%%%
% Theories
%%%%%%%%%%%%%%%%%%%%%%%%%%%%%%%%%%%%%%%%%%%%%
\DeclareMathOperator{\ZFC}{ZFC}
\DeclareMathOperator{\ZF}{ZF}
\DeclareMathOperator{\AD}{AD}
\DeclareMathOperator{\KP}{KP}
\DeclareMathOperator{\PD}{PD}
\newcommand{\CH}{\textbf{CH} }
\newcommand{\AC}{\textbf{AC} }
%%%%%%%%%%%%%%%%%%%%%%%%%%%%%%%%%%%%%%%%%%%%%
% Iteration Trees
%%%%%%%%%%%%%%%%%%%%%%%%%%%%%%%%%%%%%%%%%%%%%

\newcommand{\pred}{\text{-pred}}

%%%%%%%%%%%%%%%%%%%%%%%%%%%%%%%%%%%%%%%%%%%%%%%%
% Operator Names
%%%%%%%%%%%%%%%%%%%%%%%%%%%%%%%%%%%%%%%%%%%%%%%%
\DeclareMathOperator{\Det}{Det}
\DeclareMathOperator{\dom}{dom}
\DeclareMathOperator{\ran}{ran}
\DeclareMathOperator{\range}{ran}
\DeclareMathOperator{\crit}{crit}
\DeclareMathOperator{\card}{card}
\DeclareMathOperator{\cof}{cof}
\DeclareMathOperator{\rank}{rank}
\DeclareMathOperator{\ot}{o.t.}
\DeclareMathOperator{\ro}{r.o.}
\DeclareMathOperator{\rud}{rud}
\DeclareMathOperator{\Powerset}{\mathcal{P}}
\DeclareMathOperator{\length}{lh}
\DeclareMathOperator{\lh}{lh}
\DeclareMathOperator{\fld}{fld}
\DeclareMathOperator{\projection}{p}
\DeclareMathOperator{\Ult}{Ult}
\DeclareMathOperator{\ULT}{Ult}
\DeclareMathOperator{\Coll}{Coll}
\DeclareMathOperator{\Col}{Coll}
\DeclareMathOperator{\Hull}{Hull}
\DeclareMathOperator*{\dirlim}{dir lim}
\DeclareMathOperator{\supp}{supp}
\DeclareMathOperator{\trancl}{tran.cl.}
%%%%%%%%%%%%%%%%%%%%%%%%%%%%%%%%%%%%%%%%%%%%%
% Logical Connectives
%%%%%%%%%%%%%%%%%%%%%%%%%%%%%%%%%%%%%%%%%%%%%
\newcommand{\IImplies}{\Longrightarrow}
\newcommand{\Ifff}{\Longleftrightarrow}
\newcommand{\iimplies}{\longrightarrow}
\newcommand{\ifff}{\longleftrightarrow}
\newcommand{\Implies}{\Rightarrow}
\newcommand{\Iff}{\Leftrightarrow}
\renewcommand{\implies}{\rightarrow}
\renewcommand{\iff}{\leftrightarrow}
\newcommand{\AND}{\wedge}
\newcommand{\OR}{\vee}

%%%%%%%%%%%%%%%%%%%%%%%%%%%%%%%%%%%%%%%%%%%%%
% Function Arrows
%%%%%%%%%%%%%%%%%%%%%%%%%%%%%%%%%%%%%%%%%%%%%

\newcommand{\injection}{\xrightarrow{\text{1-1}}}
\newcommand{\surjection}{\xrightarrow{\text{onto}}}
\newcommand{\bijection}{\xrightarrow[\text{onto}]{\text{1-1}}}
\newcommand{\cofmap}{\xrightarrow{\text{cof}}}
\newcommand{\map}{\rightarrow}

%%%%%%%%%%%%%%%%%%%%%%%%%%%%%%%%%%%%%%%%%%%%%
% Mouse Comparison Operators
%%%%%%%%%%%%%%%%%%%%%%%%%%%%%%%%%%%%%%%%%%%%%
\newcommand{\initseg}{\trianglelefteq}
\newcommand{\properseg}{\lhd}
\newcommand{\notinitseg}{\ntrianglelefteq}
\newcommand{\notproperseg}{\ntriangleleft}

%%%%%%%%%%%%%%%%%%%%%%%%%%%%%%%%%%%%%%%%%%%%%
%           calligraphic letters
%%%%%%%%%%%%%%%%%%%%%%%%%%%%%%%%%%%%%%%%%%%%%
\newcommand{\cA}{\mathcal{A}}
\newcommand{\cB}{\mathcal{B}}
\newcommand{\cC}{\mathcal{C}}
\newcommand{\cD}{\mathcal{D}}
\newcommand{\cE}{\mathcal{E}}
\newcommand{\cF}{\mathcal{F}}
\newcommand{\cG}{\mathcal{G}}
\newcommand{\cH}{\mathcal{H}}
\newcommand{\cI}{\mathcal{I}}
\newcommand{\cJ}{\mathcal{J}}
\newcommand{\cK}{\mathcal{K}}
\newcommand{\cL}{\mathcal{L}}
\newcommand{\cM}{\mathcal{M}}
\newcommand{\cN}{\mathcal{N}}
\newcommand{\cO}{\mathcal{O}}
\newcommand{\cP}{\mathcal{P}}
\newcommand{\cQ}{\mathcal{Q}}
\newcommand{\cR}{\mathcal{R}}
\newcommand{\cS}{\mathcal{S}}
\newcommand{\cT}{\mathcal{T}}
\newcommand{\cU}{\mathcal{U}}
\newcommand{\cV}{\mathcal{V}}
\newcommand{\cW}{\mathcal{W}}
\newcommand{\cX}{\mathcal{X}}
\newcommand{\cY}{\mathcal{Y}}
\newcommand{\cZ}{\mathcal{Z}}


%%%%%%%%%%%%%%%%%%%%%%%%%%%%%%%%%%%%%%%%%%%%%
%          Primed Letters
%%%%%%%%%%%%%%%%%%%%%%%%%%%%%%%%%%%%%%%%%%%%%
\newcommand{\aprime}{a^{\prime}}
\newcommand{\bprime}{b^{\prime}}
\newcommand{\cprime}{c^{\prime}}
\newcommand{\dprime}{d^{\prime}}
\newcommand{\eprime}{e^{\prime}}
\newcommand{\fprime}{f^{\prime}}
\newcommand{\gprime}{g^{\prime}}
\newcommand{\hprime}{h^{\prime}}
\newcommand{\iprime}{i^{\prime}}
\newcommand{\jprime}{j^{\prime}}
\newcommand{\kprime}{k^{\prime}}
\newcommand{\lprime}{l^{\prime}}
\newcommand{\mprime}{m^{\prime}}
\newcommand{\nprime}{n^{\prime}}
\newcommand{\oprime}{o^{\prime}}
\newcommand{\pprime}{p^{\prime}}
\newcommand{\qprime}{q^{\prime}}
\newcommand{\rprime}{r^{\prime}}
\newcommand{\sprime}{s^{\prime}}
\newcommand{\tprime}{t^{\prime}}
\newcommand{\uprime}{u^{\prime}}
\newcommand{\vprime}{v^{\prime}}
\newcommand{\wprime}{w^{\prime}}
\newcommand{\xprime}{x^{\prime}}
\newcommand{\yprime}{y^{\prime}}
\newcommand{\zprime}{z^{\prime}}
\newcommand{\Aprime}{A^{\prime}}
\newcommand{\Bprime}{B^{\prime}}
\newcommand{\Cprime}{C^{\prime}}
\newcommand{\Dprime}{D^{\prime}}
\newcommand{\Eprime}{E^{\prime}}
\newcommand{\Fprime}{F^{\prime}}
\newcommand{\Gprime}{G^{\prime}}
\newcommand{\Hprime}{H^{\prime}}
\newcommand{\Iprime}{I^{\prime}}
\newcommand{\Jprime}{J^{\prime}}
\newcommand{\Kprime}{K^{\prime}}
\newcommand{\Lprime}{L^{\prime}}
\newcommand{\Mprime}{M^{\prime}}
\newcommand{\Nprime}{N^{\prime}}
\newcommand{\Oprime}{O^{\prime}}
\newcommand{\Pprime}{P^{\prime}}
\newcommand{\Qprime}{Q^{\prime}}
\newcommand{\Rprime}{R^{\prime}}
\newcommand{\Sprime}{S^{\prime}}
\newcommand{\Tprime}{T^{\prime}}
\newcommand{\Uprime}{U^{\prime}}
\newcommand{\Vprime}{V^{\prime}}
\newcommand{\Wprime}{W^{\prime}}
\newcommand{\Xprime}{X^{\prime}}
\newcommand{\Yprime}{Y^{\prime}}
\newcommand{\Zprime}{Z^{\prime}}
\newcommand{\alphaprime}{\alpha^{\prime}}
\newcommand{\betaprime}{\beta^{\prime}}
\newcommand{\gammaprime}{\gamma^{\prime}}
\newcommand{\Gammaprime}{\Gamma^{\prime}}
\newcommand{\deltaprime}{\delta^{\prime}}
\newcommand{\epsilonprime}{\epsilon^{\prime}}
\newcommand{\kappaprime}{\kappa^{\prime}}
\newcommand{\lambdaprime}{\lambda^{\prime}}
\newcommand{\rhoprime}{\rho^{\prime}}
\newcommand{\Sigmaprime}{\Sigma^{\prime}}
\newcommand{\tauprime}{\tau^{\prime}}
\newcommand{\xiprime}{\xi^{\prime}}
\newcommand{\thetaprime}{\theta^{\prime}}
\newcommand{\Omegaprime}{\Omega^{\prime}}
\newcommand{\cMprime}{\cM^{\prime}}
\newcommand{\cNprime}{\cN^{\prime}}
\newcommand{\cPprime}{\cP^{\prime}}
\newcommand{\cQprime}{\cQ^{\prime}}
\newcommand{\cRprime}{\cR^{\prime}}
\newcommand{\cSprime}{\cS^{\prime}}
\newcommand{\cTprime}{\cT^{\prime}}

%%%%%%%%%%%%%%%%%%%%%%%%%%%%%%%%%%%%%%%%%%%%%
%          bar Letters
%%%%%%%%%%%%%%%%%%%%%%%%%%%%%%%%%%%%%%%%%%%%%
\newcommand{\abar}{\bar{a}}
\newcommand{\bbar}{\bar{b}}
\newcommand{\fbar}{\bar{f}}
\newcommand{\gbar}{\bar{g}}

%%%%%%%%%%%%%%%%%%%%%%%%%%%%%%%%%%%%%%%%%%%%%
%          Formulas
%%%%%%%%%%%%%%%%%%%%%%%%%%%%%%%%%%%%%%%%%%%%%

\newcommand{\formulaphi}{\text{\large $\varphi$}}
\newcommand{\Formulaphi}{\text{\Large $\varphi$}}


%%%%%%%%%%%%%%%%%%%%%%%%%%%%%%%%%%%%%%%%%%%%%
%          Fraktur Letters
%%%%%%%%%%%%%%%%%%%%%%%%%%%%%%%%%%%%%%%%%%%%%


\newcommand{\fA}{\mathfrak{A}}
\newcommand{\fB}{\mathfrak{B}}
\newcommand{\fC}{\mathfrak{C}}
\newcommand{\fD}{\mathfrak{D}}

%%%%%%%%%%%%%%%%%%%%%%%%%%%%%%%%%%%%%%%%%%%%%
%          Bold Letters
%%%%%%%%%%%%%%%%%%%%%%%%%%%%%%%%%%%%%%%%%%%%%

\newcommand{\bd}{\mathbf{d}}

%%%%%%%%%%%%%%%%%%%%%%%%%%%%%%%%%%%%%%%%%%%%%
% Projective-Like Pointclasses
%%%%%%%%%%%%%%%%%%%%%%%%%%%%%%%%%%%%%%%%%%%%%


\newcommand{\Sa}[2][\alpha]{\Sigma_{(#1,#2)}}
\newcommand{\Pa}[2][\alpha]{\Pi_{(#1,#2)}}
\newcommand{\Da}[2][\alpha]{\Delta_{(#1,#2)}}
\newcommand{\Aa}[2][\alpha]{A_{(#1,#2)}}
\newcommand{\Ca}[2][\alpha]{C_{(#1,#2)}}
\newcommand{\Qa}[2][\alpha]{Q_{(#1,#2)}}
\newcommand{\da}[2][\alpha]{\delta_{(#1,#2)}}
\newcommand{\leqa}[2][\alpha]{\leq_{(#1,#2)}}
\newcommand{\lessa}[2][\alpha]{<_{(#1,#2)}}
\newcommand{\equiva}[2][\alpha]{\equiv_{(#1,#2)}}


\newcommand{\Sl}[1]{\Sa[\lambda]{#1}}
\newcommand{\Pl}[1]{\Pa[\lambda]{#1}}
\newcommand{\Dl}[1]{\Da[\lambda]{#1}}
\newcommand{\Al}[1]{\Aa[\lambda]{#1}}
\newcommand{\Cl}[1]{\Ca[\lambda]{#1}}
\newcommand{\Ql}[1]{\Qa[\lambda]{#1}}

\newcommand{\San}{\Sa{n}}
\newcommand{\Pan}{\Pa{n}}
\newcommand{\Dan}{\Da{n}}
\newcommand{\Can}{\Ca{n}}
\newcommand{\Qan}{\Qa{n}}
\newcommand{\Aan}{\Aa{n}}
\newcommand{\dan}{\da{n}}
\newcommand{\leqan}{\leqa{n}}
\newcommand{\lessan}{\lessa{n}}
\newcommand{\equivan}{\equiva{n}}

%%%%%%%%%%%%%%%%%%%%%%%%%%%%%%%%%%%%%%%%%%%%%%%%%%%%%%%%%%%%%%%%%%%%%%%%%%%
%%  Theorem-Like Declarations
%%%%%%%%%%%%%%%%%%%%%%%%%%%%%%%%%%%%%%%%%%%%%%%%%%%%%%%%%%%%%%%%%%%%%%%%%%

\newtheorem{theorem}{Theorem}[section]
\newtheorem{lemma}[theorem]{Lemma}
\newtheorem{corollary}[theorem]{Corollary}
\newtheorem{proposition}[theorem]{Proposition}


\theoremstyle{definition}

\newtheorem{definition}[theorem]{Definition}
\newtheorem{conjecture}[theorem]{Conjecture}
\newtheorem{remark}[theorem]{Remark}
\newtheorem{example}[theorem]{Example}
\newtheorem{remarks}[theorem]{Remarks}
\newtheorem{notation}[theorem]{Notation}

\theoremstyle{remark}

%\newtheorem*{note}{Note}
\newtheorem*{warning}{Warning}
\newtheorem*{question}{Question}
\newtheorem*{fact}{Fact}
\newtheorem*{problem}{Problem}

\newenvironment*{subproof}[1][Proof]
{\begin{proof}[#1]}{\renewcommand{\qedsymbol}{$\diamondsuit$} \end{proof}}
 
 \newenvironment*{case}[1]
{\textbf{Case #1.  }\itshape }

\newenvironment*{claim}[1][Claim]
{\textbf{#1.  }\itshape }



\newcommand{\skipsmall}{\vspace{1em}}
\newcommand{\skipmed}{\vspace{2em}}
\newcommand{\skipbig}{\vspace{3em}}
\newcommand{\skipsmallminus}{\vspace{-1em}}

\begin{document}

\title{Some Notes on Combining a Query-Independent Ranking with a
Term-Frequency-Based Ranking on Search Results}
\author{Mitch Rudominer}

\maketitle

\section{Introduction}
These are some notes on the question of how to combine Pathways Activity Rank
with a term-frequency-based scoring in the search server.

\section{Rank Combination Functions}

Let $X = \singleton{x_1,\dots,x_n}$ be a set of objects to be ranked. For example,
$X$ might consist of the set of results of a search query, which need to be
ranked by a search engine before being returned to the user. Let $\Rplus$ denote
the set of positive real numbers.

\begin{definition}
A \textbf{rank function} is a function $f:X\map\Rplus$. A rank function $f$ induces an ordering $<_f$ on $X$ given by 
$x_i<_fx_j$ iff $f(x_i)<f(x_j)$. It also induces two \textbf{measurement functions} $\sigma_f$ and $\mu_f$ defined 
by $\sigma_f(x_i,x_j)=|f(x_i)-f(x_j)|$ and $\mu_f(x_i,x_j)=f(x_i)/(x_j)$ if
$x_j<_f x_i$  or $\mu_f(x_i,x_j)=f(x_j)/f(x_i)$ otherwise.
\end{definition}

The intuition behind a measurement function $m$ is that if $x>y$ with respect to some ordering then $m(x,y)$ gives us some kind of measure of how much greater $x$ is
 than $y$. If the ordering is on search results, then $m(x,y)$ gives us a
 measure of how much better the document $x$ is (in general or for a given
 query) than the document $y$. If we were considering a single ranking function
 $f$ in isolation, we would care only about $<_f$ and not about any measurement functions.
 We care about the measurement funcionts in the context of \emph{combining} more than one
 rank funtion.

\begin{definition}
Let $f:X\map\Rplus$ be a rank function. A \textbf{re-scaling} of $f$ is another
rank function $\fbar:X\map\Rplus$ such that there is some increasing function
$\phi:\Rplus\map\Rplus$ such that $\fbar(x)=\phi(f(x))$.
\end{definition}

So a rescaling of $f$ induces the same ordering as $f$ but does not induce the
same measurement functions.  In other
words, $f(x_i)<f(x_j)$ iff $\fbar(x_i)<\fbar(x_j)$ and $f(x_i)=f(x_j)$ iff
$\fbar(x_i)=\fbar(x_j)$, 
but $|f(x_i)-f(x_j)| \neq |\fbar(x_i)-\fbar(x_j)|$.

In these notes we are interested in how to combine two rank functions to get a third
rank function. 
We will take the point of view that one of our two rank functions is the \emph{dominant} one. 
If one were to think of a rank function as providing one party's votes on the relative merits of the elements of the set 
$X$, then the dominant rank function is the one whose votes we wish to give more weight to. We will often use $f$ and $g$ 
for the names of our two rank functions. We use the convention that $f$ is the more dominant one.

\begin{definition}
A \textbf{rank combination function} is a function $H:\Rplus\times\Rplus\map\Rplus$ such that $H$ is non-decreasing in each argument. 
Given two rank functions $f$ and $g$ and a rank combination function $H$, we
get a new rank function $h$ given by $h(x_i) = H(f(x_i),g(x_i))$. 
We call $h$ the $H$-combination of $f$ and $g$.
\end{definition}

\begin{example}
We give some examples of rank combination functions.
\begin{enumerate}
\item Let $0\leq\alpha,\beta\leq 1$. Then $H(x,y) = \alpha x + \beta y$ is a
rank
combination function. Given two rank functions $f$ and $g$ this yields a third
rank function $h$ given by $h(x_i) = \alpha f(x_i) + \beta g(x_i)$.
\item Let $0\leq\alpha,\beta\leq 1$. $H(x,y) = x^{\alpha}y^{\beta}$ is rank combination function.
Given two rank functions $f$ and $g$, this yields a third rank function $h$ given by $h(x_i) = f(x_i)^{\alpha}g(x_i)^{\beta}$.
\end{enumerate}
\end{example}

\begin{remark}
The two examples given above are related by the equation
\begin{equation}
\log(x^{\alpha}y^{\beta}) = \alpha \log(x) + \beta \log(y)
\end{equation}
Since $\log(x)$ is an increasing function, the ordering induced by 
$f(x)^{\alpha}g(x)^{\beta}$ is the same as the ordering induced by
$\alpha \log(f(x)) + \beta \log(g(x))$. In other words, as regards the induced
ordering, \textbf{considering multiplying two rank functions is equivalent to
considering adding their logs.}
\end{remark}

The next lemma indicates that combining two rank functions to give a third one in the manner described above, always yields a 
ranking that satisfies, what has been called in other discussions, \emph{the obvious axiom}.
\begin{lemma}
Suppose $h$ is the $H$-combination of $f$ and $g$, for some $f$,$g$, and $H$. Then
$x_i \leq_f x_j \text{ and } x_i \leq_g x_j \Implies x_i \leq_h x_j$.
\end{lemma}
\begin{proof}
This follows trivially from the fact that $H$ is non-decreasing in each argument.
\end{proof}

The only property of the resultant combined rank function the we care about is
the ordering it induces. Given the previous lemma, the only thing we need to
focus on is how the combination of the two rank functions behaves when the
individual rank functions \textbf{disagree}. In other words, if  $h$ is the $H$-combination of $f$ and $g$
and if $x_i <_f x_j$ but $x_j <_g x_j$, the question we need to focus on is,
what determines whether or not $x_i <_h x_j$?
In the case in which $f$ is the dominant rank function, we might phrase the question:
When will the combined ranking put two elements in a different order than
the dominant ranking did? 
The following two lemmas give the answer for the two rank combination functions
in the example above.  

\begin{lemma}
Let $H(x,y) = \alpha x + \beta y$ for some $0<\alpha,\beta<1$ and let $h$ be the $H$-combination of $f$ and $g$. Fix $i$ and $j$ and 
suppose $x_i <_f x_j$ but $x_i \geq_g x_j$. Let $\sigma_f = f(x_j) - f(x_i)$ and
$\sigma_g = g(x_i) - g(x_j)$. Then:
\begin{enumerate}
\item $\sigma_g > \frac{\alpha}{\beta} \sigma_f \Ifff x_i >_h x_j$
\item $\sigma_g < \frac{\alpha}{\beta} \sigma_f \Ifff x_i <_h x_j$
\item $\sigma_g = \frac{\alpha}{\beta} \sigma_f \Ifff x_i \equiv_h x_j$.
\end{enumerate}
\end{lemma}
\begin{proof}
\begin{equation}
\begin{split}
h(x_j)-h(x_i) &= (\alpha f(x_j) + \beta g(x_j)) - (\alpha f(x_i) + \beta g(x_i)
) \\ &= \alpha(f(x_j) - f(x_i)) + \beta(g(x_j) - g(x_i) \\
              &= \alpha\sigma_f - \beta\sigma_g
\end{split}
\end{equation}
Thus
\begin{enumerate}
\item $\beta\sigma_g > \alpha\sigma_f \Ifff  x_i >_h x_j $
\item $\beta\sigma_g < \alpha\sigma_f \Ifff  x_i <_h x_j $
\item $\beta\sigma_g = \alpha\sigma_f \Ifff  x_i =_h x_j $
\end{enumerate}
\end{proof}

\begin{lemma}
Let $H(x,y) = x^{\alpha}y^{\beta}$ for some $0<\alpha,\beta<1$ and let $h$ be
the $H$-combination of $f$ and $g$. Fix $i$ and $j$ and suppose $x_i <_f x_j$ but 
$x_i \geq_g x_j$. Let $\mu_f = f(x_j) /f(x_i)$ and $\mu_g = g(x_i) /g(x_j)$. Then:
\begin{enumerate}
\item $\mu_g^{\beta} > \mu_f^{\alpha} \Implies x_i >_h x_j$
\item $\mu_g^{\beta} < \mu_f^{\alpha} \Implies x_i <_h x_j$
\item $\mu_g ^{\beta}= \mu_f^{\alpha} \Implies x_i \equiv_h x_j$.
\end{enumerate}
\end{lemma}

\begin{proof}
\begin{equation}
\begin{split}
h(x_j)/h(x_i) &= \frac{f(x_j)^{\alpha}g(x_j)^{\beta}}{f(x_i)^{\alpha}g(x_i)^{\beta}} \\
              &= \frac{f(x_j)^{\alpha}/f(x_i)^{\alpha}}{g(x_i)^{\beta}/g(x_j)^{\beta}} \\
              &= \mu_f^{\alpha} / \mu_g^{\beta}
\end{split}
\end{equation}
\end{proof}


\section{Application To Search Result Ranking}
In this section let $X = \singleton{x_1,\dots,x_n}$ be a collection of documents
indexed by a search server. Let $Q$ be the set of all queries issued to the
search server. 

For each $q\in Q$, let $X_q\subseteq X$
be the subset of $X$ consisting of those documents that
match the query $q$. Let us suppose that for each $q\in Q$, there is some
``correct'' order
$<_q$ of $X_q$. The idea here is that if $x_i<_q x_j$, then document $x_j$ is a
``better'' document, in some absolute sense, given the query $q$. This is of
course not an accurate model of reality. But it simplifies our model.

Ideally we would like for our search server to return results to a given query
$q$ in exactly the order $<_q$. But this is not possible because we do not know
how to compute this order.

Instead we use other rankings of $X_q$ that we can compute, hoping to come close
to the elusive $<_q$. In particular, we might have multiple rankings that each
approximate $<_q$ in different ways. The topic of this section is: Can we combine
two such approximations to get a better approximation?

Let $g:X\map\Rplus$ be a \textbf{query-independent}
ranking, such as Pathways activity rank, and for each $q\in Q$, let $f_q:
X_q\map\Rplus$ be a \textbf{query-dependent} ranking, such as LOG(TF) or BM25.
We think of $<_{f_q}$ and $<_g$ as each being approximations to the true $<_q$.
Since $<_{f_q}$ depends on $q$ and $<_g$ does not, we think of $<_{f_q}$ as
being the dominant ranking, the one that comes closer to the true $<_q$. But we
recognize that $<_{f_q}$ does not perfectly match $<_q$ and we hope that by
combining it with $<_g$ we can get a better approximation of $<_q$. We will
refer to a ranking that more closely approximates the true ranking $<_q$ as a
\textbf{better} ranking.


Our goal is to find a rank combination function
$H:\Rplus\times\Rplus\map\Rplus$ such that, letting $h_q$ be the
$H$-combination
of $f_q$ and $g$, $<_{h_q}$ is a better ranking than $<_g$ or $<_{f_q}$.
Notice that $H$ must not depend on $q$. That is, we need
to find a way of combining $f_q$ and $g$ that does not depend on $q$.

We are only going to consider linear combinations of re-scalings of $f_q$ and
$g$. Thas is we are going to consider rank functions of the form $h_q(x)\defeq\alpha
\fbar_q(x)+\beta \gbar(x)$, where $\gbar$ is a re-scaling of $g$ and $\fbar_q$ is
a re-scaling of $f_q$. As mentioned in the previous section, this encompasses
combinations of the form products of powers of $f$ and $g$ also.
 
 Let us slightly change our notation and write $\AR:X\map\Rplus$ for the
 Pathways Activity
rank function, as calculated by the Rank Engine.
 
\begin{problem}
Do the following:
\begin{enumerate}
  \item Pick a family of term-frequency based ranking functions
  $f_q:X_q\map\Rplus$, for $q\in Q$, that approximates
  $<_q$ well, and that is appropriate for use in a linear combination in step 3
  below. This will probably be a re-scaling of LOG(TF) or BM25 or something
  closely related.
  \item Pick a re-scaling of $\AR$, $g:X\map\Rplus$, that is appropriate for use
  in a linear combination in step 3 below.
  \item Find numbers $\alpha$ and $\beta$ so that for all $q$, letting
  $h_q(x)=\alpha
  f_q(x)+\beta g(x)$, $<_{h_q}$ is a better ranking than $<_{f_q}$.
\end{enumerate}
\end{problem}



\end{document}
