\documentclass[twoside,twocolumn,12pt]{amsart}

%%%%%%%%%%%%%%%%%%%%%%%%%%%%%%%%%%%%%%%%%%%%%%%%%%%%%%%%%%
%% This is the main Latex file for the Paper
%%   "The Circumference and Area of a Circle"
%% by Mitch Rudominer. Last Change date:
%% Format: Latex 2e
%%
%% There are  other files which are needed to compile the
%% paper. They are:
%%
%%%%%%%%%%%%%%%%%%%%%%%%%%%%%%%%%%%%%%%%%%%%%%%%%%%%%%%%%%

\usepackage{amsmath,amssymb,latexsym,eucal,amsthm}

%%%%%%%%%%%%%%%%%%%%%%%%%%%%%%%%%%%%%%%%%%%%%%%%%%%%%%%%%%%%%%%%%%%%%%%%%
%%% The following line is included to make WinEDT's Log-File-Search   %%%
%%% work correctly. It should be removed before this file is          %%%
%%% distributed.  Also, all occurrences of "INPUT" below should be    %%%
%%% replaced by "input".                                              %%%
%%%%%%%%%%%%%%%%%%%%%%%%%%%%%%%%%%%%%%%%%%%%%%%%%%%%%%%%%%%%%%%%%%%%%%%%%

\def\INPUT#1{\typeout{:<+ #1.tex}\input{#1}\typeout{:<-}}

\typeout{:?1011}
%% ^^ This tells WinEDT which types of errors to process.}%%

\INPUT{mathdefs}
%\INPUT{specdefs}

\theoremstyle{plain}

\newtheorem{Theorem}{Theorem}
\newtheorem{Lemma}[Theorem]{Lemma}
\newtheorem{Corollary}[Theorem]{Corollary}

\theoremstyle{definition}

\newtheorem{Definition}[Theorem]{Definition}

\newcommand{\skipsmall}{\vspace{1em}}
\newcommand{\skipmed}{\vspace{2em}}
\newcommand{\skipbig}{\vspace{3em}}
\newcommand{\skipsmallminus}{\vspace{-1em}}

%\renewcommand{\baselinestretch}{1.5}
\setlength{\voffset}{-0.25in}
\setlength{\hoffset}{-0.4in}
\setlength{\headheight}{0in}
\setlength{\headsep}{0in}
\setlength{\footskip}{0.5125in}
\setlength{\textwidth}{9.8in}
\setlength{\oddsidemargin}{0in}
\setlength{\evensidemargin}{0in}
\setlength{\textheight}{6.75in}
\setlength{\topmargin}{0.125in}
\setlength{\columnsep}{1in}
\pagestyle{empty}
\thispagestyle{empty}

%%%%%%%%%%%%%%%%%%%%%%%%%%%%%%%%%%%%%%%%%%%%%%%%%%%%%%


\begin{document}

\thispagestyle{empty}

\title{The Area and Circumference of a Circle}

\author{Mitch Rudominer}

%\address{Department of Mathematics\\
%         Florida International University\\
%         Miami, FL 33199}
%
%\email{rudomine@fiu.edu}

\maketitle

\thispagestyle{empty}

In grade school we were all given the formulas for the area and
circumference of a circle:
$$A=\pi r^2 \text{ and } C = 2\pi r$$
where $\pi\approx 3.14159$.
Most likely these formulas were given with no justification, or even
an intuitive explanation as to why they are true.  In this article
I will discuss the two formulas above, I will give elementary proofs
of them, and I will discuss their history.

To begin with, how should one interpret the formulas $A=\pi r^2$ and
$C=2\pi r$? It is tempting to take the point of view that first
there was this number $\pi$ ``out there,'' and subsequently
 it was discovered
that $\pi$ is useful for
calculating the area and circumference of a circle. This of course would be
misleading.
 The formulas $A=\pi r^2$ and $C=2\pi r$ should be interpreted
as both a \emph{theorem} in geometry, and the \emph{definition} of $\pi$.
The
theorem says that \emph{there is some constant} $k$, such that
for all circles, the area and circumference of the circle are given
by:  $A=k r^2$ and $C=2 k r$. The definition says,
let us agree to use the symbol $\pi$ to refer to this constant.
To be more precise, as I see it, there are at least seven important ideas
associated with the area and circumference formulas given above.
I describe these seven ideas below.
(In order to understand what follows, it would be helpful to
pretend
for a moment that you have never heard of the number $\pi$. Then I will
\emph{define} $\pi$ for you below.)

\vspace{0.1in}

\noindent
\textbf{Idea 1.}
The circumference of a circle is directly proportional to
the radius of the circle. That is, there is some constant $k$ such
that for all circles,  $C=k r$. This implies for instance that if you
double the radius of a circle, then you double its circumference.

\vspace{0.1in}

\noindent
\textbf{Idea 2.} The area of a circle is directly proportional to the square
of the radius of the circle. That is, there is some constant $h$ such
that for all circles,
$A=h r^2$.  This implies for instance that if you double the radius
of a circle, then you quadruple its area.

\vspace{0.1in}

\noindent
\textbf{Idea 3.} The two constants of proportionality mentioned above
are related to each other by the equation  $k=2 h$.

\vspace{0.1in}

\noindent
\textbf{Notation.}
Given Idea 3 above,
let us agree to use the
symbol $\pi$ to refer to the constant $h$. Then Idea 1 says that
$C=2\pi r$ and Idea 2 says that $A=\pi r^2$.

\vspace{0.1in}

\noindent
\textbf{Idea 4.} The number $\pi$ is approximately equal to 3. To be
more precise, $3\frac{1}{8}<\pi<3\frac{1}{7}$.

\vspace{0.1in}

\noindent
\textbf{Idea 5.} $\pi\approx 3.14159265358979323846$. (I think of this as a
separate idea beyond Idea 4 because Idea 4
can actually be discovered by physical experiment, whereas Idea 5 must be
discovered by mathematical analysis.)

\vspace{0.1in}

\noindent
\textbf{Idea 6.} $\pi$ is an irrational number. That is $\pi$ cannot
be expressed exactly
with any finite number of decimal places, or any fraction
of whole numbers.

\vspace{0.1in}

\noindent
\textbf{Idea 7.} Not only is $\pi$ an irrational number, $\pi$ is actually
a \emph{transcendental} number. (I will explain this later.)

\newpage

\noindent
\textbf{\large Some History}

\skipsmall

Ideas 1 through 7 above are listed approximately in the order in
which they occurred historically. I will give a short description of some
of this history.\footnote{All of the historical information in this
article is taken directly from
the two books listed in the references. I will not give any further
references for the various historical claims I make in the article.}

 Ideas 1 and 4 were likely known by prehistoric man, more than
5000 years ago. The reason is that these ideas are fairly easy to discover
experimentally. For example, suppose you are at the beach and you are
sitting on the wet sand. Suppose also that you have some rope with you.
Using a short piece of rope you draw a circle in the wet sand. You do this
by holding one end of the rope down with one hand, or a stick. Then you
pull the rope taught and revolve it around the center point, while drawing
in the wet sand with your finger. If your  piece of rope had length
$r$, then you have drawn a circle of radius $r$. Now you take a second
longer piece of
rope and you lay the rope down in the circular groove in the sand. You
cut the second piece of rope so that it exactly fits once around the
circle. Then you compare your short first piece of rope with your longer
second piece of rope. By laying off the short piece against the longer
piece, you discover that the long piece is a little more than six times
as long as the short piece.  That is, you discover that the circumference
of your circle is a little more than $6r$.
If you repeat this experiment several times,
you will discover that it always comes out the same way, no matter how big
or how small you draw the circle. You have thus discovered that the
circumference of a circle is directly proportional to the radius of the
circle, with a constant of proportionality a little more than 6. If you
measure very carefully you can probably determine that the constant of
proportionality is between $6\frac{1}{4}$ and $6\frac{2}{7}$.

It is a little bit more difficult to discover in this manner the
relationship between the radius of a circle and its \emph{area}. For
this reason, perhaps Idea 2 did not occur as early in history as Idea 1.
Nevertheless, it is not very difficult to imagine some simple physical
experiments which would allow you to discover the fact that if you
double the radius of a circle, you quadruple its area. By carefully
estimating  the area of a circle (by counting the number of small squares
which can be drawn in the circle) it can be discovered that the area of
a circle is proportional to the square of the radius, with the constant
of proportionality being a little more than three. With very careful
measurement it may be discovered that the constant of proportionality
is between $3\frac{1}{8}$ and $3\frac{1}{7}$.

At any rate, it is known that Ideas 1, 2, 3, and 4 were known by certain
ancient civilizations by about 2000 B.C.  In one ancient Egyptian
mathematical document called the Rhind Papyrus, the author's process
for finding the area of a circle with radius $r$ was to use the
formula $A=4(8/9)^2 r^2$. This amounts to using the value
$$\pi=4\left(\frac{8}{9}\right)^2\approx 3.1605.$$ A Babylonian cuneiform
tablet discovered at Susa by a French archeological expedition in 1936
asserts that the area of a circle is equal to $2/25$ times the square
of the circumference. This amounts to using a value of
$$\pi=\frac{25}{8}=3\frac{1}{8}=3.125.$$

How did these ancient peoples arrive at their values for $\pi$? Nobody
knows the answer for certain. From the fact that their values for
$\pi$ are only accurate to the first decimal place, we might guess
that some sort of physical estimation was involved, rather than any
mathematical analysis. This brings us to the history of Idea 5.
Namely, the idea of estimating $\pi$ accurately to many decimal places.
Obviously this kind of accurate estimation can not be done by measuring
the relative lengths of two pieces of rope. There are many different
ways in which mathematical analysis can be used to estimate the value
of $\pi$. I will discuss this a bit later. At this time I would like
to focus on history. Historians of science have devoted considerable
attention to the attempts throughout history to calculate $\pi$ with
further and further accuracy. (Of course, most of the people who worked
on this problem did not use the symbol $\pi$. The first use of the
symbol $\pi$ to stand for the ratio of a circle's circumference to
its diameter was in 1706 by an obscure English writer, William Jones, in
his \emph{Synopsis Palmariorum Matheseos}, or \emph{A New Introduction
to the Mathematics.} The reason for the use of $\pi$ may be that it is
the first letter of the Greek word \emph{perimetros} meaning perimeter.
In 1748 Leonhard Euler used the symbol $\pi$ in his
famous \emph{Introductio in Analysin Infinitorum.} From this point on
the use of the symbol $\pi$ became universal.)

Without any attempt at completeness, let me just
list a few historical records concerning computations of the digits
of $\pi$. In the third century B.C. the Greek mathematician Archimedes,
in his treatise \emph{The Measurement of a Circle}, proved that
$$3\frac{10}{71}<\pi<3\frac{1}{7}.$$
In his work, Archimedes \emph{proved}
the validity of Ideas 1, 2, 3 and 4 above. Note that this is
different than the ancient Egyptians and Babylonians who probably
``knew'' of Ideas 1 through 4, but did not prove them. I will talk
more about proofs later.

In the first millennium A.D., it was
the Chinese who led the way in the approximation of $\pi$. In the third
century, Liu Hui derived the bounds
$$3.141024<\pi<3.142704.$$
And in the fifth century Tsu Chung-Chi obtained
$$3.1415926<\pi<3.1415927.$$

Jumping ahead to the sixteenth century, in 1593 Francois Viete set a
new record by calculating that
$$3.1415926535<\pi<3.1415926537.$$

By the end of the 16th century, $\pi$ was known to 30 decimal places,
by the end of the 18th century it was known to 140 decimal places,
and by the end of the 19th century it was known to 526 decimal places.
Today, with the use of computers, $\pi$ can be calculated to many
more decimal places. One programmer has found $\pi$  to
$500,000$ decimal places!

But is the programmer who found $\pi$ to 500,000 places any closer to
capturing all of $\pi$ than the ancient Babylonians who knew that
$\pi$ was a little more than 3? In one sense, the answer is no. This
is because $\pi$ is an irrational number. Its decimal expansion
is infinite, and non-repeating. This brings us to the history of Idea 6.
Throughout history people were able to find more and more digits
of $\pi$. Maybe if someone were \emph{very} ambitious and \emph{very}
patient, they could find all of the digits of $\pi$, or at least find
a long pattern of digits which repeated over and over again. Any hope
along these lines was dashed in 1767, when the Swiss mathematician
Johann Heinrich Lambert proved the irrationality of $\pi$. In his
treatise \emph{Preliminary Knowledge for Those Who Seek the Quadrature
and Rectification of the Circle}, Lambert proved the following theorem:
\emph{If $x$ is a rational number other than zero, then $\tan x$ cannot
be rational.} Since $\tan(\pi/4)=1$ is rational, $\pi/4$ must be irrational,
and so $\pi$ must be irrational. Some credit for this result must also
be given to Adrien-Marie Legendre. In his \emph{Elements of Geometry} (1794)
Legendre gave a more rigorous proof of a lemma which Lambert had used
in his proof.

Finally, we come to the history of Idea 7, the idea that $\pi$ is a
transcendental number.  Look back at the title of Lambert's treatise
mentioned in the last paragraph. Lambert refers to the \emph{quadrature}
of the circle? What does this mean? This phrase refers to a very old
problem in mathematics,
 the problem of ``squaring the circle.'' In brief the idea is as
follows. Suppose you are given a circle drawn on a piece of paper. Can
you draw a square whose area is equal to the area of the given circle?
But wait, there are some rules! You are allowed to use a writing
utensil,
a straight edge to draw straight lines, and a compass to draw circles.
And that's it. You are not allowed to use a ruler, or a protractor,
or any other tools. Furthermore you are not allowed to use the given
tools in any way other than the prescribed way. In particular you are
not allowed to make any marks on the straight edge in order to record
a distance, and you are not allowed to use the legs of the compass in
order to transfer a length from one place to another.  People worked
on the problem of squaring the circle for over two thousand years, from
prior to 500 B.C., all the way up until 1882. In 1882 it was shown
that the problem of squaring the circle \emph{cannot be solved.}

But what does this have to do with $\pi$? It turns out that the squaring
the circle problem can be solved if and only if it is possible, using
only straight edge and compass, to draw a line segment of length $\pi$.
But it is not  difficult to show that we can draw a line segment of
length $L$, only if $L$ is an algebraic number.  An \emph{algebraic}
number is a number $x$ such that there exists integers $a_0,\dots,a_n$,
with $a_n\not=0$, such that
$$a_n x^n+a_{n-1}x^{n-1}+\dots+a_1x+a_0=0.$$
That is, an algebraic number is one that is the root of a non-trivial
polynomial
with integer coefficients. If a number is not algebraic, then we say
that it is transcendental. So to put these ideas together we have
the following: \emph{If $\pi$ is a transcendental number, then the
circle cannot be squared.}

In 1873, Charles Hermite proved that the number $e$ is transcendental.
In 1882 F. Lindemann extended this result to show that if $r$ and $s$
are distinct \emph{complex} algebraic numbers, then the expression
$e^r+e^s$
cannot be equal to zero. But Euler had already proven his famous equation:
$e^{i\pi}+1=0.$
Letting $r=i\pi$ and $s=0$, we can write $e^r+e^s=0$. Since
$s=0$ is an algebraic number, the only possibility is that $r=i\pi$ is
a transcendental number. But $i$ is an algebraic number. We are forced
to conclude that $\pi$ is a transcendental number.

\skipbig

\noindent
\textbf{\large Proofs}

\skipsmall

So much for history. The rest of this article is devoted to giving
some simple, elementary proofs of Ideas 1 through 5 above.


\vspace{0.1in}


\noindent
\textbf{Ideas 1 and 2.} I claim that Ideas 1 and 2 are actually quite
obvious, once you think about them for a moment.
(I do \emph{not} claim this about Idea 3 though.)
I will give an informal proof of Ideas 1 and 2. My informal proof can
be turned into a formal proof in a number of different ways.
Let us fix some circle $\Omega$. Let $C$ be the circumference of $\Omega$,
$A$ the area of $\Omega$, and $r$ the radius of $\Omega$.
Let $k=C/r$ and let $h=A/r^2$. Now let $\Omegaprime$ be any other
circle, let $\Aprime$ be its area, $\Cprime$  its circumference,
and $\rprime$ its radius. We want to see that $\Cprime/\rprime=k$
and $\Aprime/\rprime=h$. But I claim that
this is obvious, because $\Omegaprime$ can
be obtained by simply \emph{expanding} or \emph{contracting} $\Omega$.

Let me be a bit more rigorous. Let us introduce a coordinate system
to aid in the discussion. The coordinate system is not essential to
my proof, but it will simplify the discussion.
In order to exhibit how elementary
and intuitive my proof is,  I will make as little use of
analytic geometry as possible.
Since both area and arc length are
intuitively preserved
by ``rigid motions,'' we may as well assume that both of our circles
are centered at the origin. Also, to be concrete, let us assume that
$\Omegaprime$ is a bigger circle than $\Omega$, i.e. that $\rprime>r$.
Let $\alpha=\rprime/r$.
By the term ``circle''
we mean the set of all points a given distance from a center point.
Thus $\Omega$ is the set of all points located a distance $r$ from the
origin, and $\Omegaprime$ is the set of all points located a distance
$\rprime$ from the origin.
Consider the transformation $T$ of the coordinate
plane which sends a point $(x,y)$ to the point $(\alpha x, \alpha y)$.
$T$ is an ``expansion'' by a factor of $\alpha$. Suppose that $(x,y)$ is
a point on the circle $\Omega$. Then the distance from $(x,y)$ to
the origin is $r$. $T$ sends the point $(x,y)$ to the point
$(\alpha x, \alpha y)$. Using some simple facts about similar triangles,
  we see that the distance from $(\alpha x, \alpha y)$ to the
origin is $\alpha r =\rprime$. Thus $(\alpha x, \alpha y)$ lies on
the circle $\Omegaprime$. Thus the transformation $T$ maps
$\Omega$ onto $\Omegaprime$. Let us see what $T$ does to area and
arc length.  Again using some simple facts about similar triangles,
it is easy to see that if $(x_1,y_1)$ and $(x_2,y_2)$ are any two
points in the plane and the distance between them is $d$, then the distance
between $(\alpha x_1, \alpha y_1)$ and $(\alpha x_2, \alpha y_2)$
is $\alpha d$. That is, $T$ increases all distances by a factor of $\alpha$.
It follows that $T$ increases the length of any polygonal path by a
factor of $\alpha$. Since our intuitive notion of  arc length corresponds
to the limit of the lengths of approximating polygonal paths, it follows
that $T$ increasing all arc lengths by a factor of $\alpha$. As $T$ sends
$\Omega$ onto $\Omegaprime$ it follows that $\Cprime=\alpha C$.
Thus
$$\frac{\Cprime}{\rprime}=\frac{\alpha C}{\alpha r}=\frac{C}{r}=k.$$
Also, since $T$ increases all distances by a factor of $\alpha$, $T$
increases the area of any square by a factor of $\alpha^2$. Since our
intuitive notion of area corresponds to the limit of the areas of
approximating collections of squares, $T$ increases all areas by a factor
of $\alpha^2$. As $T$ sends $\Omega$ onto $\Omegaprime$ it follows that
$\Aprime=\alpha^2 A$. Thus
$$\frac{\Aprime}{(\rprime)^2}=
\frac{\alpha^2 A}{(\alpha r)^2}=\frac{A}{r^2}=h.$$
In summary, Ideas 1 and 2 follow immediately from the fact that expansions
and contractions act linearly on the distance between two points. The
fact that this is so is an integral part of our geometric intuition. It
is the reason why two similar triangles have the same ratios  of side
lengths.



\vspace{0.2in}

\noindent
\textbf{Idea 3.} Unlike Ideas 1 and 2, I do not claim that Idea 3 is
obvious.
 The proof of Idea 3 requires a more
detailed analysis. Below I will give some arguments which will reprove
Ideas 1 and 2, and also yield a proof of Idea 3.

  My goal is
to derive Ideas 1, 2, and 3 using  ``geometrically intuitive''
reasoning. At several points I will even appeal to a diagram to make my
argument.
It is well known that such appeals to diagrams are dangerous, as diagrams
can often be misleading. To give a completely rigorous geometric proof,
I would have to give a list of geometric \emph{axioms}, and derive all
of my results from these axioms. I will not do this here, because I would
like to keep this article short and simple. The interested reader
is invited to try to translate my proof into a completely rigorous proof
based on a set of axioms.

\begin{Definition}
Let $A(r)= $ the area of the circle of radius $r$.
\end{Definition}

To understand my proof, pretend again
that you have never heard of the number
$\pi$. Now I will \emph{define} $\pi$ for you.

\begin{Definition}
Let $\pi=A(1)$.
\end{Definition}

In the proof below I will mention the trigonometric functions $\sin$,
$\cos$ and $\tan$. Since I intend to use only ``geometrically intuitive''
reasoning, I would like to point out that these trigonometric functions
can be defined on acute angles, with no other assumptions besides
the fact that similar triangles have the same ratio of
side-lengths.

For convenience,
I will use radians to measure angles. In so doing, I must be careful
to avoid ``begging the question.'' When we first encounter radians in
a mathematics class
there are two important facts we learn: (i) There are $2\pi$ radians of
angle measure in a full circle, and (ii) An angle of one radian inscribed
in a circle of radius $r$ cuts off a circular arc of length $r$.
 From these
two properties of radian measure
we could conclude immediately that
the
circumference of a circle is given by $C=2\pi r$.
This of course would be a ``circular argument.'' That is to say, it would
be cheating. The reason is the following:
Fact (ii) is usually taken as the \emph{definition} of radian measure.
Then, the
formula $C=2\pi r$ is used to see
that
fact (i) is true. So it would be cheating
to use facts (i) and (ii) in order to prove that
$C=2\pi r$.
To avoid this problem I will \emph{not assume}
fact (ii) above. Instead I will \emph{define} radian measure so that fact
(i) above holds. That is, let us define  one \emph{radian}
to be the same angle measure as $[360\div(2\pi)]^{\circ}.$ Later,  when we
have proven the formulas $A=\pi r^2$ and  $C=2\pi r$, then fact (ii) above
will follow.


\begin{Lemma}
\label{Lemma0}
For any real number $r>0$ and any real number $\theta$ with
$0<\theta<\pi/2$, we have the following:
$$\pi r^2 \frac{\sin\theta}{\theta}\leq A(r)\leq
\pi r^2 \frac{\tan\theta}{\theta}.$$
\end{Lemma}
\begin{proof}
Consider the  following diagram, in which $o$ is the center
of the circle, and $\angle oac$ is a right angle.

\mbox{\input{fig1.pic}}
Notice that the area of the triangle $oab$ is less than the area of
the circular segment $oab$ which is less than the area of the
triangle $oac$.
Since $\overline{bd}=r\sin\theta$,
the area of triangle $oab$ is $\frac{1}{2}r^2\sin\theta$.
The area of the circular
segment $oab$ is $\frac{\theta}{2\pi}A(r)$. The area of triangle $oac$ is
$\frac{1}{2}\tan\theta$.
Thus we have:
$$\frac{1}{2}r^2\sin\theta\leq\frac{\theta}{2\pi}A(r)\leq
\frac{1}{2}r^2\tan\theta.$$
Multiplying through by $\frac{2\pi}{\theta}$ we have
$$\pi r^2 \frac{\sin\theta}{\theta}\leq A(r)\leq
\pi r^2 \frac{\tan\theta}{\theta}$$
which is what we were trying to prove.
\end{proof}
%
From the above lemma we can immediately derive  a famous limit theorem.
%
\begin{Corollary}
\label{Cor1}
If $\theta$ is measured in radians, then
$$\lim_{\theta\to 0^{+}}\frac{\sin\theta}{\theta}=1.$$
\end{Corollary}
\begin{proof}
Recall that we have defined $\pi=A(1)$. Setting $r=1$ in the previous
lemma we get:
$$\pi \frac{\sin\theta}{\theta}\leq\pi\leq
\pi \frac{\tan\theta}{\theta}.$$
Canceling the term  $\pi$ we get
$$\frac{\sin\theta}{\theta}\leq 1 \leq
\frac{\sin\theta}{\theta}\frac{1}{\cos\theta}.$$
Multiplying through by $\frac{\theta}{\sin\theta}$ yields
$$1\leq\frac{\theta}{\sin\theta}\leq\frac{1}{\cos\theta}.$$
Inverting the fractions and reversing the inequalities yields:
\begin{equation}
\label{x:1}
\cos\theta\leq\frac{\sin\theta}{\theta}\leq 1.
\end{equation}
Just by thinking about right triangles, it is easy to see that
as $\theta$ gets closer and closer to 0, $\cos\theta$ gets closer and
closer to the value $1$. In symbols we write this as
\begin{equation}
\label{x:2}
\lim_{\theta\to 0^{+}}\cos\theta=1.
\end{equation}
From formulas \ref{x:1} and \ref{x:2} it follows that
$$\lim_{\theta\to 0^{+}}\frac{\sin\theta}{\theta}=1$$
which is what we were trying to prove.
\end{proof}
%
From the previous two results we can derive the formula for the area
of a circle.
%
\begin{Theorem}
\label{AreaTheorem}
For all $r>0$, $A(r)=\pi r^2$.
\end{Theorem}
\begin{proof}
By taking the limit as $\theta$ goes to 0 in Lemma \ref{Lemma0} and
then applying Corollary \ref{Cor1} we get:
$$\pi r^2\leq A(r) \leq \pi r^2.$$
\end{proof}

Now we turn to the circumference of a circle.
\begin{Definition}
Let $C(r)= $ the circumference of the circle of radius $r$.
\end{Definition}

%\begin{Lemma}
%For any $r>0$ and any $\theta$ with $0<\theta<\frac{\pi}{2}$,
%we have the following:
%$$2\pi r\frac{\sin\theta}{\theta}\leq C(r)
%\leq 2\pi r\frac{\sin\theta}{\theta}\frac{1}{\cos\theta}. $$
%\end{Lemma}
%\begin{proof}
%Consider the  following diagram, in which $o$ is the center
%of the circle, $\angle oac$ is a right angle, and the point $e$ is chosen
%so that $\angle bce = \angle cbe$.
%
%\vspace{0.05in}
%
%\mbox{\input{fig2.pic}}
%Notice that
%$$r\sin\theta=\overline{bd}<\overline{ba}<\overset{\frown}{ba}<
%(\overline{be}+\overline{ea})=\overline{ca}.$$
%Most of these steps are self explanatory.
%The second inequality is justified by the fact that the shortest distance
%between two points is a straight line. What about the third inequality?
%Why is it true that the circular arc from $b$ to $a$ has shorter
%length than the polygonal path from $b$ to $a$ through the point $e$?
%This is justified intuitively by the fact that while the two paths start
%and end at the same points, the polygonal path goes
%``further out of the way'' than the circular arc.\footnote{I do not see any
%way of deriving this third inequality from any more intuitive facts. If we
%were being completely formal and basing this proof on geometric axioms,
%I believe that we might need to take the third inequality above as an
%axiom.} The last equality is justified by the fact that
%$\overline{be}=\overline{ce}$ (by definition of the point $e$.)
%I have only included the segment $\overline{be}$ in the diagram
%in order to justify the fact the circular arc from $b$ to $a$ has shorter
%length than $\overline{ca}$.
%Now $\overset{\frown}{ba}=\frac{\theta}{2\pi}C(r)$, and
%$\overline{ca}=r\tan\theta$. Thus
%$$r\sin\theta\leq\frac{\theta}{2\pi}C(r)\leq r\tan\theta.$$
%Multiplying through by $2\pi/\theta$ we get
%$$
%2\pi r\frac{\sin\theta}{\theta}\leq C(r)
%\leq 2\pi r\frac{\sin\theta}{\theta}\frac{1}{\cos\theta}.
%$$
%which is what we were trying to prove.
%\end{proof}
 %
The next theorem will complete the proof of Ideas 1, 2, and 3.
%
\begin{Theorem}
 For all $r>0$, $C(r)=2\pi r$.
\end{Theorem}
\begin{proof}
The circumference of a circle is equal to the limit as $n$ goes to infinity
of the perimeter of an inscribed regular $n$-gon. Let us calculate this
perimeter. The following figure illustrates the situation in the case $n=6$.

\mbox{\input{fig4.pic}}

Let $\theta_n=\pi/n$. In the figure, we have
$\sin\theta_n= (\frac{1}{2}\overline{AB})/r$.
So $\frac{1}{2}\overline{AB}=r\sin\theta_n$.
So $\overline{AB}=2r\sin\theta_n$.
So the perimeter of an inscribed regular $n$-gon is
$n2r\sin\theta_n=2\pi r(\sin\theta_n)/\theta_n$. So
$$C(r)=\lim_{n\to\infty}2\pi r\frac{\sin\theta_n}{\theta_n}
=\lim_{\theta\to 0}2\pi r\frac{\sin\theta}{\theta}=2\pi r.$$
\end{proof}

\vspace{0.2in}

\noindent
\textbf{Ideas 4 and 5.} Finally,
we will discuss one elementary way to calculate
approximations to the number $\pi$. In Lemma \ref{Lemma0} above, let
$r=1$ and let $\theta=\pi/n$. The result is the following inequality:
\begin{equation}
\label{PiBounds}
 n\sin\left(\frac{\pi}{n}\right)\leq\pi\leq
n\tan\left(\frac{\pi}{n}\right).
\end{equation}
If we let $n=6$ we get
$$6\times\frac{1}{2}\leq\pi\leq 6\times\frac{1}{\sqrt{3}}.$$
If we then use the approximation $1.73\leq\sqrt{3}$, we arrive at the rough
approximation
$$3\leq\pi\leq 3.5.$$
This approximation is rough, but at least we have \emph{proved} it,
and using very elementary techniques.

The inequalities in \ref{PiBounds} can in principle be used to estimate
$\pi$ with as much accuracy as we desire. If we make the integer $n$ larger
and larger, we will get more and more accurate estimates. Of course this
still leaves us with the problem of estimating $\sin(\pi/n)$ and
$\tan(\pi/n)$. Notice that the solution ``use a calculator to estimate
$\sin(\pi/n)$ and $\tan(\pi/n)$'' is of no help at all. We could of course
use our calculator to immediately
find an estimation of $\pi$ to 8 decimal places. But
then we wouldn't understand how the calculator is arriving at this estimate.
The whole point is to prove
 our estimate of $\pi$ using elementary techniques.
The \emph{half-angle} formulas from trigonometry will come to our
rescue:
$$
\sin\frac{\theta}{2}=\sqrt{\frac{1-\cos\theta}{2}}
\quad\text{ and }\quad
\cos\frac{\theta}{2}=\sqrt{\frac{1+\cos\theta}{2}}
$$
Since we know $\sin(\pi/6)$ and $\cos(\pi/6)$, using the half-angle
formulas repeatedly will allow us to calculate $\sin(\pi/n)$,
$\cos(\pi/n)$, and $\tan(\pi/n)$, for $n=6,12,24,48,96,\cdots$.
With $n=12$ we get the estimate
$$12\times\frac{\sqrt{2-\sqrt{3}}}{2}\leq\pi\leq 12\times(2-\sqrt{3})$$
and by approximating the square roots we get $3.1\leq\pi\leq 3.22$.
With $n=24$ we get the estimate
$$24\times\frac{\sqrt{2-\sqrt{2+\sqrt{3}}}}{2}\leq\pi
\leq 24\times\left(2\sqrt{2+\sqrt{3}}-2-\sqrt{3}\right)$$
and by approximating the square roots we get
$3.13\leq\pi\leq 3.16$. As you can see, our method of approximating
$\pi$ is not very \emph{efficient}. There are far more efficient
algorithms for estimating $\pi$, but we will not go into them here.

\skipsmall

\noindent
\textbf{\large A Different Point of View}

\skipsmall

In the previous section I took the point of view that it is desirable to
prove
the formulas $A=\pi r^2$ and $C=2\pi r$ using geometrically intuitive
reasoning, and using as little abstract analysis as possible. It is
also possible to take the opposite point of view, namely that it is
desirable to prove the two formulas using \emph{no} geometric reasoning,
and using only abstract analysis. I conclude this article with a quick
sketch of how one might do this.

To begin with, one can \emph{define} the circle of radius $r$ to be the
graph of the equation $x^2+y^2=r^2$. Then, letting $f(x)=\sqrt{r^2-x^2}$,
one can \emph{define} the area and circumference
of this circle with the formulas
$$A(r)=4\int_0^r f(x)dx\quad\text{ and }\quad
C(r)=4\int_0^r \sqrt{1+ \big(\fprime(x)\big)^2}dx.$$
Then one can \emph{define} the sine and cosine functions by the
familiar power series:
$$
\sin x =x-\frac{x^3}{3!}+\frac{x^5}{5!}-\frac{x^7}{7!}+\cdots
\quad\text{ and}\quad
\cos x =1-\frac{x^2}{2!}+\frac{x^4}{4!}-\frac{x^6}{6!}+\cdots
$$
With these definitions alone, and without appealing to geometry at all,
one can prove all of the familiar properties of the trigonometric functions.
In particular one can prove that for all $x$, $\sin^2 x + \cos^2 x$=1,
that $\sin$ and $\cos$ are \emph{periodic} functions, and that
the derivative of the function $\sin x$ is $\cos x$, and the derivative
of the function $\cos x$ is $-\sin x$. Then one can \emph{define} the number
$\pi$ as half the period of the $\sin$ function. Then one can
prove that $\sin(\frac{\pi}{2})=1$, that $\sin(-\frac{\pi}{2})=-1$ and
that $\sin x$ is a one-to-one function on the interval $[-\pi/2,\pi/2]$.
Now let $\sin^{-1} x$ be the inverse of the function $\sin x$ over the
interval $[-\pi/2, \pi/2]$. So $\sin^{-1}(1)=\pi/2$, and $\sin^{-1}(0)=0$.
Letting
$F(x)=\frac{r^2}{2}\sin^{-1}\left(\frac{x}{r}\right)+
\frac{x}{2}\sqrt{r^2-x^2}$
and letting $G(x)=r\sin^{-1}(x/r)$
one can prove that $\Fprime(x)=\sqrt{r^2-x^2}=f(x)$
and $\Gprime(x)=r\big(r^2-x^2\big)^{-1/2}=\sqrt{1+ \big(\fprime(x)\big)^2}$.
Finally, one can conclude that
\begin{multline*}
$$A(r)=4\int_0^{r}f(x)dx=4\big[F(x)\big]_0^r=
4\big[\frac{r^2}{2}\sin^{-1}(1)\big]=\\
4\big[\frac{r^2}{2}\frac{\pi}{2}\big]
=\pi r^2
\end{multline*}
and that
\begin{multline*}
C(r)=4\int_0^r \sqrt{1+ \big(\fprime(x)\big)^2}dx=4\big[G(x)\big]_0^r=
4r\big[\sin^{-1}(1)\big]=\\
4r\big[\frac{\pi}{2}\big]=2\pi r.
\end{multline*}

\begin{thebibliography}{1}
\bibitem{one}Petr Beckmann. \emph{A History of Pi}, fourth edition.
The Golem Press, Boulder, CO, 1977.
\bibitem{two}David M.~Burton. \emph{Burton's History of Mathematics: An
Introduction}, third edition. Wm.~C. Brown Publishers, Dubuque, IA, 1995.
\end{thebibliography}

\end{document}
