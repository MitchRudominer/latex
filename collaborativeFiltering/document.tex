%%This is a very basic article template.
%%There is just one section and two subsections.
\documentclass{article}

\usepackage{amsmath,amssymb,latexsym,eucal,amsthm}
%%%%%%%%%%%%%%%%%%%%%%%%%%%%%%%%%%%%%%%%%%%%%
% Common Set Theory Constructs
%%%%%%%%%%%%%%%%%%%%%%%%%%%%%%%%%%%%%%%%%%%%%

\newcommand{\setof}[2]{\left\{ \, #1 \, \left| \, #2 \, \right.\right\}}
\newcommand{\lsetof}[2]{\left\{\left. \, #1 \, \right| \, #2 \,  \right\}}
\newcommand{\bigsetof}[2]{\bigl\{ \, #1 \, \bigm | \, #2 \,\bigr\}}
\newcommand{\Bigsetof}[2]{\Bigl\{ \, #1 \, \Bigm | \, #2 \,\Bigr\}}
\newcommand{\biggsetof}[2]{\biggl\{ \, #1 \, \biggm | \, #2 \,\biggr\}}
\newcommand{\Biggsetof}[2]{\Biggl\{ \, #1 \, \Biggm | \, #2 \,\Biggr\}}
\newcommand{\dotsetof}[2]{\left\{ \, #1 \, : \, #2 \, \right\}}
\newcommand{\bigdotsetof}[2]{\bigl\{ \, #1 \, : \, #2 \,\bigr\}}
\newcommand{\Bigdotsetof}[2]{\Bigl\{ \, #1 \, \Bigm : \, #2 \,\Bigr\}}
\newcommand{\biggdotsetof}[2]{\biggl\{ \, #1 \, \biggm : \, #2 \,\biggr\}}
\newcommand{\Biggdotsetof}[2]{\Biggl\{ \, #1 \, \Biggm : \, #2 \,\Biggr\}}
\newcommand{\sequence}[2]{\left\langle \, #1 \,\left| \, #2 \, \right. \right\rangle}
\newcommand{\lsequence}[2]{\left\langle\left. \, #1 \, \right| \,#2 \,  \right\rangle}
\newcommand{\bigsequence}[2]{\bigl\langle \,#1 \, \bigm | \, #2 \, \bigr\rangle}
\newcommand{\Bigsequence}[2]{\Bigl\langle \,#1 \, \Bigm | \, #2 \, \Bigr\rangle}
\newcommand{\biggsequence}[2]{\biggl\langle \,#1 \, \biggm | \, #2 \, \biggr\rangle}
\newcommand{\Biggsequence}[2]{\Biggl\langle \,#1 \, \Biggm | \, #2 \, \Biggr\rangle}
\newcommand{\singleton}[1]{\left\{#1\right\}}
\newcommand{\angles}[1]{\left\langle #1 \right\rangle}
\newcommand{\bigangles}[1]{\bigl\langle #1 \bigr\rangle}
\newcommand{\Bigangles}[1]{\Bigl\langle #1 \Bigr\rangle}
\newcommand{\biggangles}[1]{\biggl\langle #1 \biggr\rangle}
\newcommand{\Biggangles}[1]{\Biggl\langle #1 \Biggr\rangle}

\newcommand{\force}[1]{\Vert\!\underset{\!\!\!\!\!#1}{\!\!\!\relbar\!\!\!%
\relbar\!\!\relbar\!\!\relbar\!\!\!\relbar\!\!\relbar\!\!\relbar\!\!\!%
\relbar\!\!\relbar\!\!\relbar}}
\newcommand{\nforce}[1]{\Vert\!\underset{\!\!\!\!\!#1}{\!\!\!\relbar\!\!\!%
\relbar\!\!\relbar\!\!\relbar\!\!\!\relbar\!\!\relbar\!\!\relbar\!\!\!%
\relbar\!\!\not\relbar\!\!\relbar}}
\newcommand{\forcein}[2]{\overset{#2}{\Vert\underset{\!\!\!\!\!#1}%
{\!\!\!\relbar\!\!\!\relbar\!\!\relbar\!\!\relbar\!\!\!\relbar\!\!\relbar\!%
\!\relbar\!\!\!\relbar\!\!\relbar\!\!\relbar\!\!\relbar\!\!\!\relbar\!\!%
\relbar\!\!\relbar}}}

\newcommand{\pre}[2]{{}^{#2}\!{#1}}

\newcommand{\restr}{\!\!\upharpoonright\!}

%%%%%%%%%%%%%%%%%%%%%%%%%%%%%%%%%%%%%%%%%%%%%
% Set-Theoretic Connectives
%%%%%%%%%%%%%%%%%%%%%%%%%%%%%%%%%%%%%%%%%%%%%

\newcommand{\intersect}{\cap}
\newcommand{\union}{\cup}
\newcommand{\Intersection}[1]{\bigcap\limits_{#1}}
\newcommand{\Union}[1]{\bigcup\limits_{#1}}

%%%%%%%%%%%%%%%%%%%%%%%%%%%%%%%%%%%%%%%%%%%%%
% Miscellaneous
%%%%%%%%%%%%%%%%%%%%%%%%%%%%%%%%%%%%%%%%%%%%%
\newcommand{\defeq}{=_{\text{def}}}
\newcommand{\Turingleq}{\leq_{\text{T}}}
\newcommand{\Turingless}{<_{\text{T}}}
\newcommand{\lexleq}{\leq_{\text{lex}}}
\newcommand{\lexless}{<_{\text{lex}}}
\newcommand{\Turingequiv}{\equiv_{\text{T}}}

%%%%%%%%%%%%%%%%%%%%%%%%%%%%%%%%%%%%%%%%%%%%%
% Constants
%%%%%%%%%%%%%%%%%%%%%%%%%%%%%%%%%%%%%%%%%%%%%
\newcommand{\R}{\mathbb{R}}
\newcommand{\Rplus}{\mathbb{R}^{+}}
\renewcommand{\P}{\mathbb{P}}
\newcommand{\Q}{\mathbb{Q}}
\newcommand{\Z}{\mathbb{Z}}
\newcommand{\C}{\mathbb{C}}
\newcommand{\N}{\mathbb{N}}
\newcommand{\LofR}{L(\R)}
\newcommand{\JofR}[1]{J_{#1}(\R)}
\newcommand{\SofR}[1]{S_{#1}(\R)}
\newcommand{\JalphaR}{\JofR{\alpha}}
\newcommand{\JbetaR}{\JofR{\beta}}
\newcommand{\JlambdaR}{\JofR{\lambda}}
\newcommand{\SalphaR}{\SofR{\alpha}}
\newcommand{\SbetaR}{\SofR{\beta}}
\DeclareMathOperator{\ORD}{OR}
\DeclareMathOperator{\Ord}{OR}
\DeclareMathOperator{\WO}{WO}
\DeclareMathOperator{\OD}{OD}
\DeclareMathOperator{\HOD}{HOD}
\DeclareMathOperator{\HC}{HC}
\DeclareMathOperator{\WF}{WF}
\DeclareMathOperator{\HF}{HF}
\DeclareMathOperator{\AR}{AR}
\newcommand{\One}{I}
\newcommand{\ONE}{I}
\newcommand{\Two}{II}
\newcommand{\TWO}{II}

%%%%%%%%%%%%%%%%%%%%%%%%%%%%%%%%%%%%%%%%%%%%%
% Theories
%%%%%%%%%%%%%%%%%%%%%%%%%%%%%%%%%%%%%%%%%%%%%
\DeclareMathOperator{\ZFC}{ZFC}
\DeclareMathOperator{\ZF}{ZF}
\DeclareMathOperator{\AD}{AD}
\DeclareMathOperator{\KP}{KP}
\DeclareMathOperator{\PD}{PD}
\newcommand{\CH}{\textbf{CH} }
\newcommand{\AC}{\textbf{AC} }
%%%%%%%%%%%%%%%%%%%%%%%%%%%%%%%%%%%%%%%%%%%%%
% Iteration Trees
%%%%%%%%%%%%%%%%%%%%%%%%%%%%%%%%%%%%%%%%%%%%%

\newcommand{\pred}{\text{-pred}}

%%%%%%%%%%%%%%%%%%%%%%%%%%%%%%%%%%%%%%%%%%%%%%%%
% Operator Names
%%%%%%%%%%%%%%%%%%%%%%%%%%%%%%%%%%%%%%%%%%%%%%%%
\DeclareMathOperator{\Det}{Det}
\DeclareMathOperator{\dom}{dom}
\DeclareMathOperator{\ran}{ran}
\DeclareMathOperator{\range}{ran}
\DeclareMathOperator{\crit}{crit}
\DeclareMathOperator{\card}{card}
\DeclareMathOperator{\cof}{cof}
\DeclareMathOperator{\rank}{rank}
\DeclareMathOperator{\ot}{o.t.}
\DeclareMathOperator{\ro}{r.o.}
\DeclareMathOperator{\rud}{rud}
\DeclareMathOperator{\Powerset}{\mathcal{P}}
\DeclareMathOperator{\length}{lh}
\DeclareMathOperator{\lh}{lh}
\DeclareMathOperator{\fld}{fld}
\DeclareMathOperator{\projection}{p}
\DeclareMathOperator{\Ult}{Ult}
\DeclareMathOperator{\ULT}{Ult}
\DeclareMathOperator{\Coll}{Coll}
\DeclareMathOperator{\Col}{Coll}
\DeclareMathOperator{\Hull}{Hull}
\DeclareMathOperator*{\dirlim}{dir lim}
\DeclareMathOperator{\supp}{supp}
\DeclareMathOperator{\trancl}{tran.cl.}
%%%%%%%%%%%%%%%%%%%%%%%%%%%%%%%%%%%%%%%%%%%%%
% Logical Connectives
%%%%%%%%%%%%%%%%%%%%%%%%%%%%%%%%%%%%%%%%%%%%%
\newcommand{\IImplies}{\Longrightarrow}
\newcommand{\Ifff}{\Longleftrightarrow}
\newcommand{\iimplies}{\longrightarrow}
\newcommand{\ifff}{\longleftrightarrow}
\newcommand{\Implies}{\Rightarrow}
\newcommand{\Iff}{\Leftrightarrow}
\renewcommand{\implies}{\rightarrow}
\renewcommand{\iff}{\leftrightarrow}
\newcommand{\AND}{\wedge}
\newcommand{\OR}{\vee}

%%%%%%%%%%%%%%%%%%%%%%%%%%%%%%%%%%%%%%%%%%%%%
% Function Arrows
%%%%%%%%%%%%%%%%%%%%%%%%%%%%%%%%%%%%%%%%%%%%%

\newcommand{\injection}{\xrightarrow{\text{1-1}}}
\newcommand{\surjection}{\xrightarrow{\text{onto}}}
\newcommand{\bijection}{\xrightarrow[\text{onto}]{\text{1-1}}}
\newcommand{\cofmap}{\xrightarrow{\text{cof}}}
\newcommand{\map}{\rightarrow}

%%%%%%%%%%%%%%%%%%%%%%%%%%%%%%%%%%%%%%%%%%%%%
% Mouse Comparison Operators
%%%%%%%%%%%%%%%%%%%%%%%%%%%%%%%%%%%%%%%%%%%%%
\newcommand{\initseg}{\trianglelefteq}
\newcommand{\properseg}{\lhd}
\newcommand{\notinitseg}{\ntrianglelefteq}
\newcommand{\notproperseg}{\ntriangleleft}

%%%%%%%%%%%%%%%%%%%%%%%%%%%%%%%%%%%%%%%%%%%%%
%           calligraphic letters
%%%%%%%%%%%%%%%%%%%%%%%%%%%%%%%%%%%%%%%%%%%%%
\newcommand{\cA}{\mathcal{A}}
\newcommand{\cB}{\mathcal{B}}
\newcommand{\cC}{\mathcal{C}}
\newcommand{\cD}{\mathcal{D}}
\newcommand{\cE}{\mathcal{E}}
\newcommand{\cF}{\mathcal{F}}
\newcommand{\cG}{\mathcal{G}}
\newcommand{\cH}{\mathcal{H}}
\newcommand{\cI}{\mathcal{I}}
\newcommand{\cJ}{\mathcal{J}}
\newcommand{\cK}{\mathcal{K}}
\newcommand{\cL}{\mathcal{L}}
\newcommand{\cM}{\mathcal{M}}
\newcommand{\cN}{\mathcal{N}}
\newcommand{\cO}{\mathcal{O}}
\newcommand{\cP}{\mathcal{P}}
\newcommand{\cQ}{\mathcal{Q}}
\newcommand{\cR}{\mathcal{R}}
\newcommand{\cS}{\mathcal{S}}
\newcommand{\cT}{\mathcal{T}}
\newcommand{\cU}{\mathcal{U}}
\newcommand{\cV}{\mathcal{V}}
\newcommand{\cW}{\mathcal{W}}
\newcommand{\cX}{\mathcal{X}}
\newcommand{\cY}{\mathcal{Y}}
\newcommand{\cZ}{\mathcal{Z}}


%%%%%%%%%%%%%%%%%%%%%%%%%%%%%%%%%%%%%%%%%%%%%
%          Primed Letters
%%%%%%%%%%%%%%%%%%%%%%%%%%%%%%%%%%%%%%%%%%%%%
\newcommand{\aprime}{a^{\prime}}
\newcommand{\bprime}{b^{\prime}}
\newcommand{\cprime}{c^{\prime}}
\newcommand{\dprime}{d^{\prime}}
\newcommand{\eprime}{e^{\prime}}
\newcommand{\fprime}{f^{\prime}}
\newcommand{\gprime}{g^{\prime}}
\newcommand{\hprime}{h^{\prime}}
\newcommand{\iprime}{i^{\prime}}
\newcommand{\jprime}{j^{\prime}}
\newcommand{\kprime}{k^{\prime}}
\newcommand{\lprime}{l^{\prime}}
\newcommand{\mprime}{m^{\prime}}
\newcommand{\nprime}{n^{\prime}}
\newcommand{\oprime}{o^{\prime}}
\newcommand{\pprime}{p^{\prime}}
\newcommand{\qprime}{q^{\prime}}
\newcommand{\rprime}{r^{\prime}}
\newcommand{\sprime}{s^{\prime}}
\newcommand{\tprime}{t^{\prime}}
\newcommand{\uprime}{u^{\prime}}
\newcommand{\vprime}{v^{\prime}}
\newcommand{\wprime}{w^{\prime}}
\newcommand{\xprime}{x^{\prime}}
\newcommand{\yprime}{y^{\prime}}
\newcommand{\zprime}{z^{\prime}}
\newcommand{\Aprime}{A^{\prime}}
\newcommand{\Bprime}{B^{\prime}}
\newcommand{\Cprime}{C^{\prime}}
\newcommand{\Dprime}{D^{\prime}}
\newcommand{\Eprime}{E^{\prime}}
\newcommand{\Fprime}{F^{\prime}}
\newcommand{\Gprime}{G^{\prime}}
\newcommand{\Hprime}{H^{\prime}}
\newcommand{\Iprime}{I^{\prime}}
\newcommand{\Jprime}{J^{\prime}}
\newcommand{\Kprime}{K^{\prime}}
\newcommand{\Lprime}{L^{\prime}}
\newcommand{\Mprime}{M^{\prime}}
\newcommand{\Nprime}{N^{\prime}}
\newcommand{\Oprime}{O^{\prime}}
\newcommand{\Pprime}{P^{\prime}}
\newcommand{\Qprime}{Q^{\prime}}
\newcommand{\Rprime}{R^{\prime}}
\newcommand{\Sprime}{S^{\prime}}
\newcommand{\Tprime}{T^{\prime}}
\newcommand{\Uprime}{U^{\prime}}
\newcommand{\Vprime}{V^{\prime}}
\newcommand{\Wprime}{W^{\prime}}
\newcommand{\Xprime}{X^{\prime}}
\newcommand{\Yprime}{Y^{\prime}}
\newcommand{\Zprime}{Z^{\prime}}
\newcommand{\alphaprime}{\alpha^{\prime}}
\newcommand{\betaprime}{\beta^{\prime}}
\newcommand{\gammaprime}{\gamma^{\prime}}
\newcommand{\Gammaprime}{\Gamma^{\prime}}
\newcommand{\deltaprime}{\delta^{\prime}}
\newcommand{\epsilonprime}{\epsilon^{\prime}}
\newcommand{\kappaprime}{\kappa^{\prime}}
\newcommand{\lambdaprime}{\lambda^{\prime}}
\newcommand{\rhoprime}{\rho^{\prime}}
\newcommand{\Sigmaprime}{\Sigma^{\prime}}
\newcommand{\tauprime}{\tau^{\prime}}
\newcommand{\xiprime}{\xi^{\prime}}
\newcommand{\thetaprime}{\theta^{\prime}}
\newcommand{\Omegaprime}{\Omega^{\prime}}
\newcommand{\cMprime}{\cM^{\prime}}
\newcommand{\cNprime}{\cN^{\prime}}
\newcommand{\cPprime}{\cP^{\prime}}
\newcommand{\cQprime}{\cQ^{\prime}}
\newcommand{\cRprime}{\cR^{\prime}}
\newcommand{\cSprime}{\cS^{\prime}}
\newcommand{\cTprime}{\cT^{\prime}}

%%%%%%%%%%%%%%%%%%%%%%%%%%%%%%%%%%%%%%%%%%%%%
%          bar Letters
%%%%%%%%%%%%%%%%%%%%%%%%%%%%%%%%%%%%%%%%%%%%%
\newcommand{\abar}{\bar{a}}
\newcommand{\bbar}{\bar{b}}
\newcommand{\fbar}{\bar{f}}
\newcommand{\gbar}{\bar{g}}

%%%%%%%%%%%%%%%%%%%%%%%%%%%%%%%%%%%%%%%%%%%%%
%          Formulas
%%%%%%%%%%%%%%%%%%%%%%%%%%%%%%%%%%%%%%%%%%%%%

\newcommand{\formulaphi}{\text{\large $\varphi$}}
\newcommand{\Formulaphi}{\text{\Large $\varphi$}}


%%%%%%%%%%%%%%%%%%%%%%%%%%%%%%%%%%%%%%%%%%%%%
%          Fraktur Letters
%%%%%%%%%%%%%%%%%%%%%%%%%%%%%%%%%%%%%%%%%%%%%


\newcommand{\fA}{\mathfrak{A}}
\newcommand{\fB}{\mathfrak{B}}
\newcommand{\fC}{\mathfrak{C}}
\newcommand{\fD}{\mathfrak{D}}

%%%%%%%%%%%%%%%%%%%%%%%%%%%%%%%%%%%%%%%%%%%%%
%          Bold Letters
%%%%%%%%%%%%%%%%%%%%%%%%%%%%%%%%%%%%%%%%%%%%%

\newcommand{\bd}{\mathbf{d}}

%%%%%%%%%%%%%%%%%%%%%%%%%%%%%%%%%%%%%%%%%%%%%
% Projective-Like Pointclasses
%%%%%%%%%%%%%%%%%%%%%%%%%%%%%%%%%%%%%%%%%%%%%


\newcommand{\Sa}[2][\alpha]{\Sigma_{(#1,#2)}}
\newcommand{\Pa}[2][\alpha]{\Pi_{(#1,#2)}}
\newcommand{\Da}[2][\alpha]{\Delta_{(#1,#2)}}
\newcommand{\Aa}[2][\alpha]{A_{(#1,#2)}}
\newcommand{\Ca}[2][\alpha]{C_{(#1,#2)}}
\newcommand{\Qa}[2][\alpha]{Q_{(#1,#2)}}
\newcommand{\da}[2][\alpha]{\delta_{(#1,#2)}}
\newcommand{\leqa}[2][\alpha]{\leq_{(#1,#2)}}
\newcommand{\lessa}[2][\alpha]{<_{(#1,#2)}}
\newcommand{\equiva}[2][\alpha]{\equiv_{(#1,#2)}}


\newcommand{\Sl}[1]{\Sa[\lambda]{#1}}
\newcommand{\Pl}[1]{\Pa[\lambda]{#1}}
\newcommand{\Dl}[1]{\Da[\lambda]{#1}}
\newcommand{\Al}[1]{\Aa[\lambda]{#1}}
\newcommand{\Cl}[1]{\Ca[\lambda]{#1}}
\newcommand{\Ql}[1]{\Qa[\lambda]{#1}}

\newcommand{\San}{\Sa{n}}
\newcommand{\Pan}{\Pa{n}}
\newcommand{\Dan}{\Da{n}}
\newcommand{\Can}{\Ca{n}}
\newcommand{\Qan}{\Qa{n}}
\newcommand{\Aan}{\Aa{n}}
\newcommand{\dan}{\da{n}}
\newcommand{\leqan}{\leqa{n}}
\newcommand{\lessan}{\lessa{n}}
\newcommand{\equivan}{\equiva{n}}

%%%%%%%%%%%%%%%%%%%%%%%%%%%%%%%%%%%%%%%%%%%%%%%%%%%%%%%%%%%%%%%%%%%%%%%%%%%
%%  Theorem-Like Declarations
%%%%%%%%%%%%%%%%%%%%%%%%%%%%%%%%%%%%%%%%%%%%%%%%%%%%%%%%%%%%%%%%%%%%%%%%%%

\newtheorem{theorem}{Theorem}[section]
\newtheorem{lemma}[theorem]{Lemma}
\newtheorem{corollary}[theorem]{Corollary}
\newtheorem{proposition}[theorem]{Proposition}


\theoremstyle{definition}

\newtheorem{definition}[theorem]{Definition}
\newtheorem{conjecture}[theorem]{Conjecture}
\newtheorem{remark}[theorem]{Remark}
\newtheorem{example}[theorem]{Example}
\newtheorem{remarks}[theorem]{Remarks}
\newtheorem{notation}[theorem]{Notation}

\theoremstyle{remark}

%\newtheorem*{note}{Note}
\newtheorem*{warning}{Warning}
\newtheorem*{question}{Question}
\newtheorem*{fact}{Fact}
\newtheorem*{problem}{Problem}

\newenvironment*{subproof}[1][Proof]
{\begin{proof}[#1]}{\renewcommand{\qedsymbol}{$\diamondsuit$} \end{proof}}
 
 \newenvironment*{case}[1]
{\textbf{Case #1.  }\itshape }

\newenvironment*{claim}[1][Claim]
{\textbf{#1.  }\itshape }



\newcommand{\skipsmall}{\vspace{1em}}
\newcommand{\skipmed}{\vspace{2em}}
\newcommand{\skipbig}{\vspace{3em}}
\newcommand{\skipsmallminus}{\vspace{-1em}}

\begin{document}

\title{Some Notes on Collaborative Filtering}
\author{Mitch Rudominer}

\maketitle

\section{Introduction}
These are some notes on similarity score and affinity score calculations for
collaborative filtering in the enterprise.

\skipsmall
Given two items in an enterprise, say two documents $D_1$ and $D_2$,
we wish to define a measure of how similar $D_1$ and $D_2$ are to each other. We
wish to define a \textbf{similarity score} $\Sim(D_1,D_2)$ that gives us a
numerical measure of their similarity. We can use such a similarity score to
give context-based item recommendations. If the user navigates to $D_0$, we can
recommend the five, say, other items $D_1,\dots D_5$ that are most similar to
$D_0$, perhaps filtering out those items that we know that the user is already 
aware of. 

There are many ways we might define the similarity between two items. In
Collaborative Filtering we use the \textbf{actions of users} as the data on
which we base our similarity score. Let $\singleton{U_1,\dots U_n}$ be the set
of all users. Each user performs actions on the items in an enterprise; actions
such as viewing, editing, tagging, etc. We may model a single item $D$ as a
numerical vector $\xvec=\angles{x_1,\dots x_n}$ where the number $x_i$
represents the actions of user $U_i$ on $D$. There are many free parameters in
this model such as which actions to use and how to represent the actions
numerically. One simple example is that $x_i$ might represent the number of
times that $U_i$ has viewed $D$.

Given this notion of similarity our recommendation application might be
described compactly via the slogan ``Users who have worked with this item have
also worked with these other items\ldots''.

These notes are concerned with the decision of how to calculate a
collaborative-filtering similarity score. There are at least two distinct parts
to this question. The first part is how do we represent an item as a numeric
vector? Which actions do we use and how do we model these actions numerically.
The second part of the question is, given two numeric vectors, how do we measure
how similar they are to each other. We begin with this second question.

\section{Similarity Functions}


\begin{definition}
A \textbf{similarity function} is a function which takes pairs of
vectors of reals of the same length and gives reals
\begin{equation}
f\vdots\bigcup\limits_{n=1}^{\infty}\R^{n}\times\R^{n}\map\R
\end{equation} 
such that
\begin{enumerate}
  \item $-1\leq f(\xvec,\yvec)\leq f(\xvec,\xvec) = 1$
  \item $f(\vec{x},\vec{y})=f(\vec{y},\vec{x})$
  \item $f$ is invariant under simultaneously permuting the two input vectors.
  That is if $\vec{x} = \angles{x_1,\dots x_i,\dots x_j,\dots x_n} $
  and $\vec{y}=\angles{y_1,\dots y_i,\dots y_j,\dots y_n}$
  and $\vec{x}^{\prime}=\angles{x_1,\dots x_j,\dots x_i,\dots x_n}$
  and $\vec{y}^{\prime}=\angles{y_1,\dots y_j,\dots y_i,\dots y_n}$ then
  $f(\vec{x}^{\prime},\vec{y}^{\prime})=f(\vec{x},\vec{y})$.
  \item $f$ depends on every coordinate as long as the other coordinates are not
  all zero: Fix $n$ and fix $\xvec,\yvec,\zvec$ in $\R^{n}$ with
  $\xvec\not=\vec{0}$.
  Suppose that $\yvec$ and $\zvec$ agree on all coordinates except one. That is
  suppose there is an $i_0<n$ such that $y[i_0] \not = z[i_0]$, $y[i]=z[i]$ for
  $i\not=i_0$. Suppose further that there is some $i\not=i_0$ such that $y[i]\not=0$.
  Then $f(\xvec,\yvec)\not=f(\xvec,\zvec)$.
\end{enumerate}
\end{definition}

Intuitively $f$ yields a measure of how
similar the two input vectors $\vec{x}$ and $\vec{y}$ are, with a maximum
similarity of 1 which is always attained when the two input vectors are the
same, and a minumum similarity of $-1$. (The definition does not force
the minimum value of -1 to be attained. We will look at examples of similarity
functions in which the minum value is 0.) Property 2 says that similarity is
symmetric. Property 3 says that the indexing of the components of the input
vectors is arbitrary and doesn't affect the similarity. Property 4 just rules
out some degenerate examples and is included so that some of our theoretical
results may be stated more cleanly.

\skipsmall

The notation in the definition implies that $f$ may not be defined on all pairs 
vectors but only on some.  $f$ is
\textbf{total} if it takes all pairs of vectors. We will also be interested in
the case where $f$ takes only \textbf{non-negative} vectors $\xvec$, $\yvec$. Also $f$
is \textbf{binary} if it takes only vectors consisting of zeroes and ones.

\skipsmall

$f$ is \textbf{non-negative} if
$f(\xvec,\yvec)\geq 0$ and \textbf{positive} if $f(\xvec,\yvec)> 0$. 

\skipsmall

The \textbf{dot product} of two vectors of length $n$ is defined by
\begin{equation}
\xvec\centerdot\yvec=\sum\limits_{i=1}^{n}x_i y_i
\end{equation}

\skipsmall

The \textbf{Euclidean norm} of a vector $\xvec$ of length $n$ is defined by
\begin{equation}
\norm{\xvec} = \sqrt{\sum\limits_{i=1}^n (x_i)^2}
\end{equation}

\section{Examples of Similarity Functions}
In this section we give some examples of  similarity functions. We will study the
properties of each of these examples later in the paper.

\begin{example}
The \textbf{Euclidean} distance between two vectors $\xvec$ and $\yvec$ is given by
$d(\xvec,\yvec)=\norm{\xvec-\yvec}$. The \textbf{inverse Eculidean
distance} similarity function is the total, positive similarity function defined
by 
\begin{equation}
\simeuc(\xvec,\yvec)=\frac{1}{1+d(\xvec,\yvec)}.
\end{equation} 
Notice that $0<\simeuc((\xvec,\yvec)\leq 1$ and
$\simeuc((\xvec,\yvec)=1$ iff $\xvec=\yvec$.
\end{example}



\begin{example}
The \textbf{Generalized Tanamoto} similarity function is the
non-negative similarity function defined on non-negative vectors by 
\begin{equation}
\simtan(\xvec,\yvec)=\frac{\sum\limits_{i=1}^{n}\min(x_i,y_i)}{\sum\limits_{i=1}^{n}\max(x_i,y_i)}
\end{equation}
unless $\xvec=\yvec=\vec{0}$ in which case we define $\simtan(\vec{0},\vec{0})=1$.
Notice that $\simtan(\xvec,\yvec)=0$ iff
$\xvec\centerdot\yvec=0$ and not $\xvec=\yvec=\vec{0}$,
and $\simtan(\xvec,\yvec)=1$ iff $\xvec=\yvec$. We refer to this function as
\textbf{generalized} Tanamoto because the original Tanamoto function was only
applied to binary vectors. We will examine Tanamoto on binary vectors in a later
section.
\end{example}

\begin{example}
The \textbf{Cosine} similarity function is the total similarity function defined
by
\begin{equation}
\simcos(\xvec,\yvec)=\frac{\xvec\centerdot\yvec}{\norm{\xvec}\norm{\yvec}}.
\end{equation}
This is the cosine of the angle between the vectors $\xvec$
and $\yvec$ in $\R^n$.
Notice $-1\leq\simcos(\xvec,\yvec)\leq 1$, $\simcos(\xvec,\yvec)=0$ iff
$\xvec\centerdot\yvec=0$, $\simcos(\xvec,\yvec)=1$ iff $\xvec=\alpha\yvec$
for some $\alpha>0$ and $\simcos(\xvec,\yvec)=-1$ iff $\xvec=\alpha\yvec$
for some $\alpha<0$
\end{example}

\begin{example}
The \textbf{Correlation Coefficient} similarity function is the 
similarity function defined by
\begin{equation}
\begin{split}
\simcor(\xvec,\yvec)&=\frac{\sum\limits_{i=1}^{n}x_i y_i - \left(\sum\limits_{i=1}^{n}x_i\right)\left(\sum\limits_{i=1}^{n}y_i\right)\frac{1}{n}}{\sqrt{\left[\left(\sum\limits_{i=1}^{n}x_i^2\right)-\left(\sum\limits_{i=1}^{n}x_i\right)^2\frac{1}{n}\right]\left[\left(\sum\limits_{i=1}^{n}y_i^2\right)-\left(\sum\limits_{i=1}^{n}y_i\right)^2\frac{1}{n}\right]}}\\
&=\frac{\frac{1}{n}\sum\limits_{i=1}^{n}x_i y_i - \left(\frac{1}{n}\sum\limits_{i=1}^{n}x_i\right)\left(\frac{1}{n}\sum\limits_{i=1}^{n}y_i\right)}{\sqrt{\left[\left(\frac{1}{n}\sum\limits_{i=1}^{n}x_i^2\right)-\left(\frac{1}{n}\sum\limits_{i=1}^{n}x_i\right)^2\right]\left[\left(\frac{1}{n}\sum\limits_{i=1}^{n}y_i^2\right)-\left(\frac{1}{n}\sum\limits_{i=1}^{n}y_i\right)^2\right]}}\\
&=\frac{\cov(\xvec,\yvec)}{\sigma(\xvec)\sigma(\yvec)}.
\end{split}
\end{equation}
Here $\cov$ means covariance and $\sigma$ means standard deviation. The
correlation coeficient is a concept from probability theory and statistics. We
will explore its properties later in this paper. Notice that the denominator is
zero and so the fraction is not defined if either $\xvec$ or $\yvec$ is
constant.
\end{example}

\section{Binary Input Vectors}
In this section we study the properties of our various similarity functions when
restricted to binary input vectors. In this context it is useful to think of a
similarity function as acting on \emph{sets} instead of vectors.

Let $U=\singleton{1,\dots n}$. If $\xvec$ is a binary vector of length $n$, let
$S_{\xvec}=\setof{i\in n}{x_i=1}$. Thus $S_{\xvec}\subseteq U$ is the set whose characteristic
function is $\xvec$. In this section we will use the notational convention that
$X=S_{\xvec}$ and $Y=S_{\yvec}$. We will think of a similarity function $f$ as
being a function of $X$ and $Y$. 

It will be convenient to introduce some more
notation. Let $a=\card(\setof{i}{x_i=1 \aND y_i=0})=\card(X-Y)$. Let
$b=\card(\setof{i}{y_i=1\aND x_i=0})=\card(Y-X)$. Let
$c=\card(\setof{i}{x_i=y_i=1})=\card(X\intersect Y)$.
Finally let
$d=\card(\setof{i}{x_i=y_i=0})=\card(U - (X\union Y))$. 

Each of our binary similarity functions may be expressed as functions of $a$,
$b$, $c$, and $d$. We will say that a
similarity function is \textbf{neutral} just in case the dependency of $f$ on
$c$ and $d$ is symmetric. One way to think of this intuitively is as follows.
Imagine that $\xvec$ and $\yvec$ represent users' binary ratings of two items. A
$1$ in position $i$ represents a thumbs-up vote by user $i$ and a 0 in position
$i$ represents a thumbs-down vote. Then a similarity function is neutral if it
treats the set of users that gave thumbs-up votes to both items the same way that it 
treats the set of users that gave thumbs-down votes to both items.

We will study the properties of each of our similarity functions on binary input
vectors.

\begin{example}
The \textbf{Tanamoto} similarity function is the non-negative binary
similarity function defined on binary vectors by 
\begin{equation}
\simtan(\xvec,\yvec)=\frac{\sum\limits_{i=1}^{n}\min(x_i,y_i)}{\sum\limits_{i=1}^{n}\max(x_i,y_i)}
\end{equation}
unless $\xvec=\yvec=\vec{0}$ in which case we define $\simtan(\vec{0},\vec{0})=1$.
Notice that the numerator may also be written as $\xvec\centerdot\yvec$. In
terms of sets
\begin{equation}
\simtan(X,Y)=\frac{\card(X\intersect Y)}{\card(X\union Y)}.
\end{equation}
unless $X=Y=\emptyset$ in which case we define $\simtan(\emptyset,\emptyset)=1$.
Notice that $0\leq \simtan(X,Y)\leq 1$ with $\simtan(X,Y)=0$ iff
$X\intersect Y=\emptyset$ and $\simtan(X,Y)=1$ iff $X=Y$.
In terms of $a$, $b$, $c$ and $d$
\begin{equation}
\simtan(X,Y)=\frac{c}{a+b+c}.
\end{equation}
Since Tanamoto depends on $c$ but not on $d$ it is not neutral.
\end{example}

\begin{proposition}
For $a,b,c\geq 0$ and $a+b+c>0$ let 
\begin{equation}
f(a,b,c)=\frac{c}{a+b+c}.
\end{equation}
Then
\begin{enumerate}
  \item $f=0$ iff $c=0$
  \item $f=1$ iff $a=b=0$. 
  \item If $a+b>0$ then $f$ is strictly increasing as a function of $c$ and
  $\lim\limits_{c\to\infty}f=1$
  \item If $c>0$ then $f$ is strictly decreasing as a function of $a$ and
  $\lim\limits_{a\to\infty}f=0$
  \item If $c>0$ then $f$ is strictly decreasing as a function of $b$ and
  $\lim\limits_{b\to\infty}f=0$
\end{enumerate}
\end{proposition}
\begin{proof}
To see that $f$ is strictly increasing in $c$ it suffices to see that the
partial derivative is alwyas positive. We have
\begin{equation} 
\frac{\partial f}{\partial c} = \frac{a+b}{(a+b+c)^2} > 0
\end{equation}
since $a+b>0$. All of the other claims are easy.
\end{proof}

\begin{example}We study the \textbf{Cosine} similarity function on binary input
vectors. Notice that $\xvec\centerdot\yvec=\card(X\intersect Y)$. Also notice
that $\norm{\xvec}=\sqrt{\card(X)}$. Thus
\begin{equation}
\simcos(X,Y) = \frac{\card(X\intersect Y)}{\sqrt{\card(X)\card(Y)}}
\end{equation}
Just as with the Tanamoto function we have $0\leq \simcos(X,Y)\leq 1$ with $\simcos(X,Y)=0$ iff
$X\intersect Y=\emptyset$ and $\simcos(X,Y)=1$ iff $X=Y$. 
In terms of $a$, $b$, $c$ and $d$
\begin{equation}
\simcos(X,Y) = \frac{c}{\sqrt{(a+c)(b+c)}}
\end{equation}
Like Tanamoto, Cosine depends on $c$ but not $d$ so it is not neutral.
\end{example}

\begin{proposition}
For $a,b,c\geq 0$ and $(a+c)(b+c)>0$ let 
\begin{equation}
f(a,b,c)=\frac{c}{\sqrt{(a+c)(b+c)}}.
\end{equation}
Then
\begin{enumerate}
  \item $f=0$ iff $c=0$
  \item $f=1$ iff $a=b=0$.
  \item If $a+b>0$ then $f$ is strictly increasing as a function of $c$ and
  $\lim\limits_{c\to\infty}f=1$
  \item If $c>0$ then $f$ is strictly decreasing as a function of $a$ and
  $\lim\limits_{a\to\infty}f=0$
  \item If $c>0$ then $f$ is strictly decreasing as a function of $b$ and
  $\lim\limits_{b\to\infty}f=0$
\end{enumerate}
\end{proposition}
\begin{proof}
To see that $f$ is strictly increasing in $c$ it suffices to see that the
partial derivative is alwyas positive. We have
\begin{equation}
\begin{split}
\frac{\partial f}{\partial c} &= 
\frac{\sqrt{(a+c)(b+c)}-\frac{1}{2}c\left((a+c)(b+c)\right)^{-1/2}(a+b+2c)}{(a+c)(b+c)}\\ 
&= \frac{(a+c)(b+c)-(\frac{a+b}{2}c+c^2)}{\left((a+c)(b+c)\right)^{3/2}}.
\end{split}
\end{equation}
We have $(a+c)(b+c)=ab+(a+b)c+c^2>\frac{a+b}{2}c+c^2$ since $a+b>0$. All of the
other claims are easy.
\end{proof}

\textbf{NOTE}
By comparing propositions 4.2 and 4.4 we see that on binary input vectors $\simtan$
and $\simcos$ behave similarly.
\textbf{We see no reason to prefer one of these functions over the other
when dealing with binary input.} (That is if we ignore the cost of computation.
Taking this into consideration we prefer Tanamoto as cosine seems more expensive
because it involves $3n$ multiplications and a square root.)


\begin{example} We study the \textbf{Euclidean} similarity function on binary
input vectors. Notice that $d(\xvec,\yvec) = \sqrt{\card(X\vartriangle Y)}$ where
$X \vartriangle Y$ is the \textbf{symmetric difference} of $X$ and $Y$.
$X\vartriangle Y = X - Y \union Y - X$. In terms of $a$, $b$, $c$ and $d$
\begin{equation}
\simeuc(X,Y) = \frac{1}{1+\sqrt{a+b}}
\end{equation}
Notice that in contrast to cosine and Tanamoto, Euclidean does not depend on $c$
or $d$
at all, so it is neutral.
\end{example}
\begin{proposition}
For $a,b\geq 0$  let 
\begin{equation}
f(a,b)=\frac{1}{1+\sqrt{a+b}}.
\end{equation}
Then
\begin{enumerate}
  \item $0<f\leq 1$
  \item $f=1$ iff $a=b=0$.
  \item $f$ is strictly decreasing as a function of $a$ and
  $\lim\limits_{a\to\infty}f=0$
  \item $f$ is strictly decreasing as a function of $b$ and
  $\lim\limits_{b\to\infty}f=0$
\end{enumerate}
\end{proposition}

\begin{example} We study the \textbf{Correlation Coefficient} similarity
function on binary input vectors. For $S\subseteq U$ and $n=\card(U)$ let
$P(S)=\card(S)/n$ be the probability of $S$ and let $\bar{S}=U-S$ be the
compliment of $S$. From the second expression for
the correlation coefficient (given in the example in an earlier section) we see
directly that
\begin{equation}
\begin{split}
\simcor(X,Y) &= \frac{P(X\intersect Y) - P(X)P(Y)}{\sqrt{\left[P(X)-P(X)^2\right]\left[P(Y)-P(Y)^2\right]}}\\
&=\frac{P(X\intersect Y) - P(X)P(Y)}{\sqrt{\left[P(X)P(\bar{X})\right]\left[P(Y)P(\bar{Y})\right]}}
\end{split}
\end{equation}
In terms of $a$, $b$, $c$ and $d$, and using the fact that $n=a+b+c+d$
\begin{equation}
\begin{split}
\simcor(X,Y) &= \frac{\frac{c}{n} -\frac{a+c}{n}\frac{b+c}{n}}{\sqrt{\frac{a+c}{n}\frac{b+d}{n}\frac{b+c}{n}\frac{a+d}{n}}}\\ 
&=\frac{nc - (a+c)(b+c)}{\sqrt{(a+c)(b+d)(b+c)(a+d)}}\\
&=\frac{cd-ab}{\sqrt{(a+c)(b+d)(b+c)(a+d)}}
\end{split}
\end{equation}
Since $\simcor(X,Y)$ depends on $c$ and $d$ symmetrically it is neutral.

$\simcor(X,Y)=0$ iff  $P(X\intersect Y)=P(x)P(Y)$, that is, $X$ and $Y$ are
independent events, and $\simcor(X,Y)>0$ iff
$P(X|Y)>P(X)$ iff $P(Y|X)>P(Y)$. In summary, the sign of $\simcor(X,Y)$ indicates
whether knowing that $i$ is in one of $X$ or $Y$ increases, decrease or has no
effect on the likelihood that $i$ is in the other.
\end{example}

\begin{proposition}
For $a$, $b$, $c$, and $d$ non negative numbers and $(a+c)(b+d)(b+c)(a+d)>0$, let 
\begin{equation}
\rho(a,b,c,d)=\frac{cd-ab}{\sqrt{(a+c)(b+d)(b+c)(a+d)}}
\end{equation}
Then
\begin{enumerate}
  \item $\rho=1$ iff $a=b=0$.
  \item $\rho=-1$ iff $c=d=0$.
  \item If $a=c=0$ or $a=d=0$ or $b=c=0$ or $b=d=0$ then $\rho=0$.
   \item If $a+b>0$, $\rho$ is a strictly increasing function in $c$ and in $d$.
   \item If $c+d>0$, $\rho$ is a strictly decreasing function in $a$ and in $b$.
   \item Suppose $a+b>0$. If we fix $a$ and $b$ and let
   both $c$ and $d$ go
   to infinity we get $\lim\limits_{c,d\to\infty}\rho=1$. But if we fix $a$, $b$
   and $c$ and let $d$ go to infinity we get $\lim\limits_{d\to\infty}\rho<1$.
   Similarly if we fix $a$, $b$ and $d$ and let $c$ go to infinity.
   \item Suppose $c+d>0$. If we fix $c$ and $d$ and let
   both $a$ and $b$ go
   to infinity we get $\lim\limits_{a,b\to\infty}\rho=-1$. But if we fix $c$, $d$
   and $a$ and let $b$ go to infinity we get $\lim\limits_{b\to\infty}\rho>-1$.
   Similarly if we fix $c$, $d$ and $b$ and let $a$ go to infinity.
   
\end{enumerate}
\end{proposition}
\begin{proof}
1. If $a=b=0$ then $\rho=cd/\sqrt{c^2d^2}=1$. If $a>0$ then the numerator of the
expression for $\rho$ is $\leq cd$ and the denominator is greater than $cd$ so
the fraction is less than 1. Similarly if $b>0$.

2.  The proof of 2 is similar to the proof of 1.

3.  These are all obvious.

4. Suppose $a>0$. (The argument is similar if instead we assume that $b>0$.) We
will show that $\rho$ is increasing in $c$. A similar argument shows that $\rho$
is increasing in $d$. It suffices to show that the derivative of $\rho$ with
respect to $c$ is everywhere positive. Let $S=(a+c)(b+d)(b+c)(a+d)$. Then
\begin{equation}
\begin{split}
\frac{\partial \rho}{\partial c} &= \frac{d\sqrt{S} -
\frac{1}{2}(cd-ab)S^{-1/2}(b+d)(a+d)(a+b+2c)}{S}\\ 
&= \frac{dS - (cd-ab)(b+d)(a+d)\left(\frac{a+b}{2}+c\right)}{S^{3/2}}
\end{split}
\end{equation}
To see that $\frac{\partial \rho}{\partial c}>0$ then it suffices to see that
\begin{equation}
d(c+a)(c+b)>(cd-ab)\left(\frac{a+b}{2}+c\right)
\end{equation}
Assume that $a\geq b$. (The argument is similar if we assume instead that $b\geq
a$). Then $d(c+b)\geq cd\geq cd-ab$ and $c + a \geq c + \frac{a+b}{2}$.
If $b>0$ then the first inequality is strict and if $b=0$ then the last
inequality is strict. So in all cases we have that $\frac{\partial
\rho}{\partial c}>0$. 

5. This is similar to 4.

6. Suppose $a>0$, the argument is similar if $b>0$. Fix a small number
$\delta>0$. Let $c$ and $d$ be large enough that $\delta c>a$ and $\delta c>b$
and $\delta d>a$ and $\delta d>b$. Then
\begin{equation}
\begin{split}
\rho &> \frac{cd - \delta^2 cd}{\sqrt{(c+\delta c)(d+\delta d)(c+\delta
c)(d+\delta d)}}\\ 
&= \frac{1-\delta^2}{(1+\delta)^2}.
\end{split}
\end{equation}
Since this last fraction converges to 1 as $\delta$ goes to zero, we see that $\rho$
can be made arbitrarily close
to $1$ by fixing $a$ and $b$ and letting $c$ and $d$ be sufficiently large. On
the other hand, suppse we also fix $c$ and we let $d$ go to infinity. (The
argument is similar with the roles of $c$ and $d$ reversed.) If $c=0$ and $b=0$
then $\rho=0$. If $c=0$ and $b>0$ then $\rho<0$ and $\rho\to 0$ as $d\to\infty$.
So suppose $c>0$. For sufficiently large $d$, $\rho>0$. But
\begin{equation}
\rho \leq \frac{cd}{\sqrt{(c+a)cd^2}} = \sqrt{\frac{c}{(c+a)}} <1
\end{equation}
Thus $\rho$ is bounded away from 1.

7. This is similar to 6.
\end{proof}

\textbf{Note.} Items 1 and 2 in the previous proposition say that $\rho=1$ iff $X=Y$
and $\rho=-1$ iff $Y=\bar{X}$.

\skipsmall

\textbf{Note.} We can understand 6 and 7 in the previous proposition in terms of
conditional probability. Note that
\begin{alignat}{2}
P(X | Y) &= \frac{c}{c+b} &\qquad P(Y | X) &= \frac{c}{c+a} \\
P(\bar{X} | \bar{Y}) &= \frac{d}{d+a} &\qquad P(\bar{Y} | \bar{X}) &=
\frac{d}{d+b}.
\end{alignat}

We can understand 6 and 7 as saying that $\rho$ goes to 1 just in case all of
these go to 1, and $\rho$ goes to -1 just in case all of these go to zero.

\skipsmall

\textbf{Note.} The correlation coefficient similarity function differs from the
cosine and Tanamoto functions in that it depends on $d = \card(U - (X\union Y)$.
But it is interesting to note that if $\card(X\union Y)$ is small compared to
$\card(U)$ then the correlation coefficient is approximately equal to the
cosine:
\begin{equation}
\rho=\frac{c-ab/d}{\sqrt{(a+c)(b/d+1)(b+c)(a/d+1)}}.
\end{equation}
As $d$ goes to infinity this value goes to
\begin{equation}
\frac{c}{\sqrt{(a+c)(b+c)}}
\end{equation}
which is the expression for the cosine similarity function.

\begin{example}
The \textbf{Neutral Cosine} similarity function is a variant of the cosine
similarity function that is neutral. Given a set $X\subseteq U$ we define it's
\textbf{neutral characteristic function} to be the vector $\vec{x}$ defined by
$x[i]=1$ if $i\in X$, $x[i]=-1$ if $i\notin X$. Thus the neutral characteristic
function differs from the usual characteristic function by using the value -1 in
place of the value 0. Now we define the neutral cosine similarity function by
$\simcosneut(X,Y) = \simcos(\vec{x},\vec{y})$ where $\vec{x}$ and $\vec{y}$ are the
neutral characteristic functions of $X$ and $Y$ respectiviely. In terms of $a$,
$b$, $c$ and $d$ we find that the neutral cosine function is given by
\begin{equation}
\simcosneut = \frac{c+d-a-b}{a+b+c+d}.
\end{equation}
So the neutral cosine similarity function is in fact neutral. As with each of
the previous functions we find that $\simcosneut=1$ iff $a=b=0$ iff $X=Y$. As with the
correlation coefficient, $\simcosneut=-1$ iff $c=d=0$ iff $Y = \bar{X}$. 
Whereas $\simcor(X,Y)=0$ iff $X$ and $Y$ are stochastically independent events, we
find that in the case of the neutral cosine, $\simcosneut(X,Y)=0$ iff $\card(X\intersect
Y)+\card(U-(X\union Y)) = \card(X \vartriangle Y)$. Intuitively this means that
the same number of users agreed about the two items as disagreed. Let us examine
an example in which $\simcosneut$ and $\simcor$ disagree. Let
$U=\singleton{1,\dots 14}$. Let $X=\singleton{1,2,3, 4}$. Let
$Y=\singleton{4,5,6,7}$ This yields $a=b=3, c=1, d=7$. Since $cd-ab<0$ we find
that $\simcor(X,Y)<0$. But $c+d-a-b>0$ so $\simcosneut(X,Y)>0$. Neutral Cosine
says the sets have positive similarity because the number of users that agreed
on $X$ and
$Y$ (1 positive + 7 negative = 8) is greater than the number that disagreed (6).
Correlation coefficient
says the sets have negative similarity because $P(X|Y)=1/4 < 2/7 = P(X)$
\end{example}

\section{Properties of Similarity Functions}
In this section we return to the broader context of similarity functions on
arbitrary vectors, not just on binary vectors. As a technique for studying our
various similarity functions we will identify some properties that a similarity
function may or may not have.

\begin{definition} 
A similarity function $f$ is 
\textbf{invariant under scalar multiplication in the weak sense} if 
$f(\lambda\xvec,\lambda\yvec)=f(\xvec,\yvec)$ 
for all $\xvec,\yvec\in\R^{n}$, and all $\lambda\not=0$, such that
both sides of the equation are defined.
\end{definition}

\begin{definition} 
A similarity function $f$ is 
\textbf{invariant under scalar multiplication in the strong sense} if 
\begin{itemize}
  \item $f(\lambda\xvec,\yvec)=f(\xvec,\yvec)$ 
  for all $\xvec,\yvec\in\R^{n}$, and all $\lambda>0$, such that both sides of
  the equation are defined.
  \item $f(\lambda\xvec,\yvec)= -f(\xvec,\yvec)$ 
  for all $\xvec,\yvec\in\R^{n}$, and all $\lambda<0$, such that both sides of
  the equation are defined.
\end{itemize}
\end{definition}

Notice that by property (2) of the definition of a similarity function, in the
above defintion it is equivalent to multiply $\yvec$ by
$\lambda$ instead of $\xvec$. Notice also that invariance
under scalar multiplication in the strong sense implies it in the weak sense.

\begin{definition}A similarity function $f$ is \textbf{monotonic} if
$f(\xvec,\yvec)\geq f(\xvec,\zvec)$ whenever $\zvec$ is such that
$z[i]=y[i]$ for all $i>0$, and $|z[0]-x[0]|>|y[0]-x[0]|$.
\end{definition}

To understand the meaning of monotonic, suppose $\xvec$ and $\yvec$ represent
the ratings of two items by $n$ users. If user $0$ changes his rating of the
second item but all other users' ratings remain unchanged, we get a new rating 
vector $\zvec$ that differs from $\yvec$ only in position $0$. Suppose user $0$
changed his rating of the second item so that his
new rating differs from his rating of the first item by more than his original
rating of the second item did. $f$ is monotonic if the $f$-similarity of the two
items does not increase as a result of this change. By property (3) of the
definition of a similarity function, the definition would be equivalent if we used
a different coordinate instead of the $0^{th}$.

\skipsmall

Our first observation is that two of the above properties are inconsistent with
each other.

\begin{proposition}
If $f$ is invariant under scalar multiplication in the strong sense then $f$ is
not monotonic.
\end{proposition}
\begin{proof}
By property 4 of the definition of a similarity function we can choose two
numbers $x$ and $y$ such that
$f([0,1],[1,x])<f([0,1],[1,y])$. By strong invariance we have
$f([0,x],[1,x])=f([0,1],[1,x])<f([0,1],[1,y])=f([0,x],[1,y])$. Thus $f$ violates monotonicity.
\end{proof}

\section{Scatter Plots}
Notice that by property 3. we may think of the input to a similarity function
as a multi-set (that is a set that allows multiple repetitions of its elements)
of ordered pairs: $\singleton{(x_1,y_1),(x_2,y_2),\dots (x_n,y_n)}$. We may
visualize this as a scatter-plot of points in the $X-Y$ plane. By property 2.
$f$ is invariant under reflecting the scatter plot over the diagonal line $y=x$.

\end{document}
