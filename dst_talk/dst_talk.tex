% $Header$

\documentclass{beamer}

\usepackage{amsmath,amssymb,latexsym,eucal,amsthm}
%%%%%%%%%%%%%%%%%%%%%%%%%%%%%%%%%%%%%%%%%%%%%
% Common Set Theory Constructs
%%%%%%%%%%%%%%%%%%%%%%%%%%%%%%%%%%%%%%%%%%%%%

\newcommand{\setof}[2]{\left\{ \, #1 \, \left| \, #2 \, \right.\right\}}
\newcommand{\lsetof}[2]{\left\{\left. \, #1 \, \right| \, #2 \,  \right\}}
\newcommand{\bigsetof}[2]{\bigl\{ \, #1 \, \bigm | \, #2 \,\bigr\}}
\newcommand{\Bigsetof}[2]{\Bigl\{ \, #1 \, \Bigm | \, #2 \,\Bigr\}}
\newcommand{\biggsetof}[2]{\biggl\{ \, #1 \, \biggm | \, #2 \,\biggr\}}
\newcommand{\Biggsetof}[2]{\Biggl\{ \, #1 \, \Biggm | \, #2 \,\Biggr\}}
\newcommand{\dotsetof}[2]{\left\{ \, #1 \, : \, #2 \, \right\}}
\newcommand{\bigdotsetof}[2]{\bigl\{ \, #1 \, : \, #2 \,\bigr\}}
\newcommand{\Bigdotsetof}[2]{\Bigl\{ \, #1 \, \Bigm : \, #2 \,\Bigr\}}
\newcommand{\biggdotsetof}[2]{\biggl\{ \, #1 \, \biggm : \, #2 \,\biggr\}}
\newcommand{\Biggdotsetof}[2]{\Biggl\{ \, #1 \, \Biggm : \, #2 \,\Biggr\}}
\newcommand{\sequence}[2]{\left\langle \, #1 \,\left| \, #2 \, \right. \right\rangle}
\newcommand{\lsequence}[2]{\left\langle\left. \, #1 \, \right| \,#2 \,  \right\rangle}
\newcommand{\bigsequence}[2]{\bigl\langle \,#1 \, \bigm | \, #2 \, \bigr\rangle}
\newcommand{\Bigsequence}[2]{\Bigl\langle \,#1 \, \Bigm | \, #2 \, \Bigr\rangle}
\newcommand{\biggsequence}[2]{\biggl\langle \,#1 \, \biggm | \, #2 \, \biggr\rangle}
\newcommand{\Biggsequence}[2]{\Biggl\langle \,#1 \, \Biggm | \, #2 \, \Biggr\rangle}
\newcommand{\singleton}[1]{\left\{#1\right\}}
\newcommand{\angles}[1]{\left\langle #1 \right\rangle}
\newcommand{\bigangles}[1]{\bigl\langle #1 \bigr\rangle}
\newcommand{\Bigangles}[1]{\Bigl\langle #1 \Bigr\rangle}
\newcommand{\biggangles}[1]{\biggl\langle #1 \biggr\rangle}
\newcommand{\Biggangles}[1]{\Biggl\langle #1 \Biggr\rangle}


\newcommand{\force}[1]{\Vert\!\underset{\!\!\!\!\!#1}{\!\!\!\relbar\!\!\!%
\relbar\!\!\relbar\!\!\relbar\!\!\!\relbar\!\!\relbar\!\!\relbar\!\!\!%
\relbar\!\!\relbar\!\!\relbar}}
\newcommand{\longforce}[1]{\Vert\!\underset{\!\!\!\!\!#1}{\!\!\!\relbar\!\!\!%
\relbar\!\!\relbar\!\!\relbar\!\!\!\relbar\!\!\relbar\!\!\relbar\!\!\!%
\relbar\!\!\relbar\!\!\relbar\!\!\relbar\!\!\relbar\!\!\relbar\!\!\relbar\!\!\relbar}}
\newcommand{\nforce}[1]{\Vert\!\underset{\!\!\!\!\!#1}{\!\!\!\relbar\!\!\!%
\relbar\!\!\relbar\!\!\relbar\!\!\!\relbar\!\!\relbar\!\!\relbar\!\!\!%
\relbar\!\!\not\relbar\!\!\relbar}}
\newcommand{\forcein}[2]{\overset{#2}{\Vert\underset{\!\!\!\!\!#1}%
{\!\!\!\relbar\!\!\!\relbar\!\!\relbar\!\!\relbar\!\!\!\relbar\!\!\relbar\!%
\!\relbar\!\!\!\relbar\!\!\relbar\!\!\relbar\!\!\relbar\!\!\!\relbar\!\!%
\relbar\!\!\relbar}}}

\newcommand{\pre}[2]{{}^{#2}{#1}}

\newcommand{\restr}{\!\!\upharpoonright\!}

%%%%%%%%%%%%%%%%%%%%%%%%%%%%%%%%%%%%%%%%%%%%%
% Set-Theoretic Connectives
%%%%%%%%%%%%%%%%%%%%%%%%%%%%%%%%%%%%%%%%%%%%%

\newcommand{\intersect}{\cap}
\newcommand{\union}{\cup}
\newcommand{\Intersection}[1]{\bigcap\limits_{#1}}
\newcommand{\Union}[1]{\bigcup\limits_{#1}}
\newcommand{\adjoin}{{}^\frown}
\newcommand{\forces}{\Vdash}

%%%%%%%%%%%%%%%%%%%%%%%%%%%%%%%%%%%%%%%%%%%%%
% Miscellaneous
%%%%%%%%%%%%%%%%%%%%%%%%%%%%%%%%%%%%%%%%%%%%%
\newcommand{\defeq}{=_{\text{def}}}
\newcommand{\Turingleq}{\leq_{\text{T}}}
\newcommand{\Turingless}{<_{\text{T}}}
\newcommand{\lexleq}{\leq_{\text{lex}}}
\newcommand{\lexless}{<_{\text{lex}}}
\newcommand{\Turingequiv}{\equiv_{\text{T}}}
\newcommand{\isomorphic}{\cong}

%%%%%%%%%%%%%%%%%%%%%%%%%%%%%%%%%%%%%%%%%%%%%
% Constants
%%%%%%%%%%%%%%%%%%%%%%%%%%%%%%%%%%%%%%%%%%%%%
\newcommand{\R}{\mathbb{R}}
\renewcommand{\P}{\mathbb{P}}
\newcommand{\Q}{\mathbb{Q}}
\newcommand{\Z}{\mathbb{Z}}
\newcommand{\Zpos}{\Z^{+}}
\newcommand{\Znonneg}{\Z^{\geq 0}}
\newcommand{\C}{\mathbb{C}}
\newcommand{\N}{\mathbb{N}}
\newcommand{\B}{\mathbb{B}}
\newcommand{\Bairespace}{\pre{\omega}{\omega}}
\newcommand{\LofR}{L(\R)}
\newcommand{\JofR}[1]{J_{#1}(\R)}
\newcommand{\SofR}[1]{S_{#1}(\R)}
\newcommand{\JalphaR}{\JofR{\alpha}}
\newcommand{\JbetaR}{\JofR{\beta}}
\newcommand{\JlambdaR}{\JofR{\lambda}}
\newcommand{\SalphaR}{\SofR{\alpha}}
\newcommand{\SbetaR}{\SofR{\beta}}
\newcommand{\Pkl}{\mathcal{P}_{\kappa}(\lambda)}
\DeclareMathOperator{\con}{con}
\DeclareMathOperator{\ORD}{OR}
\DeclareMathOperator{\Ord}{OR}
\DeclareMathOperator{\WO}{WO}
\DeclareMathOperator{\OD}{OD}
\DeclareMathOperator{\HOD}{HOD}
\DeclareMathOperator{\HC}{HC}
\DeclareMathOperator{\WF}{WF}
\DeclareMathOperator{\wfp}{wfp}
\DeclareMathOperator{\HF}{HF}
\newcommand{\One}{I}
\newcommand{\ONE}{I}
\newcommand{\Two}{II}
\newcommand{\TWO}{II}
\newcommand{\Mladder}{M^{\text{ld}}}

%%%%%%%%%%%%%%%%%%%%%%%%%%%%%%%%%%%%%%%%%%%%%
% Commutative Algebra Constants
%%%%%%%%%%%%%%%%%%%%%%%%%%%%%%%%%%%%%%%%%%%%%
\DeclareMathOperator{\dottimes}{\dot{\times}}
\DeclareMathOperator{\dotminus}{\dot{-}}
\DeclareMathOperator{\Spec}{Spec}

%%%%%%%%%%%%%%%%%%%%%%%%%%%%%%%%%%%%%%%%%%%%%
% Theories
%%%%%%%%%%%%%%%%%%%%%%%%%%%%%%%%%%%%%%%%%%%%%
\DeclareMathOperator{\ZFC}{ZFC}
\DeclareMathOperator{\ZF}{ZF}
\DeclareMathOperator{\AD}{AD}
\DeclareMathOperator{\ADR}{AD_{\R}}
\DeclareMathOperator{\KP}{KP}
\DeclareMathOperator{\PD}{PD}
\DeclareMathOperator{\CH}{CH}
\DeclareMathOperator{\GCH}{GCH}
\DeclareMathOperator{\HPC}{HPC} % HOD pair capturing
%%%%%%%%%%%%%%%%%%%%%%%%%%%%%%%%%%%%%%%%%%%%%
% Iteration Trees
%%%%%%%%%%%%%%%%%%%%%%%%%%%%%%%%%%%%%%%%%%%%%

\newcommand{\pred}{\text{-pred}}

%%%%%%%%%%%%%%%%%%%%%%%%%%%%%%%%%%%%%%%%%%%%%%%%
% Operator Names
%%%%%%%%%%%%%%%%%%%%%%%%%%%%%%%%%%%%%%%%%%%%%%%%
\DeclareMathOperator{\Det}{Det}
\DeclareMathOperator{\dom}{dom}
\DeclareMathOperator{\ran}{ran}
\DeclareMathOperator{\range}{ran}
\DeclareMathOperator{\image}{image}
\DeclareMathOperator{\crit}{crit}
\DeclareMathOperator{\card}{card}
\DeclareMathOperator{\cf}{cf}
\DeclareMathOperator{\cof}{cof}
\DeclareMathOperator{\rank}{rank}
\DeclareMathOperator{\ot}{o.t.}
\DeclareMathOperator{\ords}{o}
\DeclareMathOperator{\ro}{r.o.}
\DeclareMathOperator{\rud}{rud}
\DeclareMathOperator{\Powerset}{\mathcal{P}}
\DeclareMathOperator{\length}{lh}
\DeclareMathOperator{\lh}{lh}
\DeclareMathOperator{\limit}{lim}
\DeclareMathOperator{\fld}{fld}
\DeclareMathOperator{\projection}{p}
\DeclareMathOperator{\Ult}{Ult}
\DeclareMathOperator{\ULT}{Ult}
\DeclareMathOperator{\Coll}{Coll}
\DeclareMathOperator{\Col}{Col}
\DeclareMathOperator{\Hull}{Hull}
\DeclareMathOperator*{\dirlim}{dir lim}
\DeclareMathOperator{\Scale}{Scale}
\DeclareMathOperator{\supp}{supp}
\DeclareMathOperator{\trancl}{tran.cl.}
\DeclareMathOperator{\trace}{Tr}
\DeclareMathOperator{\diag}{diag}
\DeclareMathOperator{\spn}{span}
\DeclareMathOperator{\sgn}{sgn}
%%%%%%%%%%%%%%%%%%%%%%%%%%%%%%%%%%%%%%%%%%%%%
% Logical Connectives
%%%%%%%%%%%%%%%%%%%%%%%%%%%%%%%%%%%%%%%%%%%%%
\newcommand{\IImplies}{\Longrightarrow}
\newcommand{\SkipImplies}{\quad\Longrightarrow\quad}
\newcommand{\Ifff}{\Longleftrightarrow}
\newcommand{\iimplies}{\longrightarrow}
\newcommand{\ifff}{\longleftrightarrow}
\newcommand{\Implies}{\Rightarrow}
\newcommand{\Iff}{\Leftrightarrow}
\renewcommand{\implies}{\rightarrow}
\renewcommand{\iff}{\leftrightarrow}
\newcommand{\AND}{\wedge}
\newcommand{\OR}{\vee}
\newcommand{\st}{\text{ s.t. }}
\newcommand{\Or}{\text{ or }}

%%%%%%%%%%%%%%%%%%%%%%%%%%%%%%%%%%%%%%%%%%%%%
% Function Arrows
%%%%%%%%%%%%%%%%%%%%%%%%%%%%%%%%%%%%%%%%%%%%%

\newcommand{\injection}{\xrightarrow{\text{1-1}}}
\newcommand{\surjection}{\xrightarrow{\text{onto}}}
\newcommand{\bijection}{\xrightarrow[\text{onto}]{\text{1-1}}}
\newcommand{\cofmap}{\xrightarrow{\text{cof}}}
\newcommand{\map}{\rightarrow}

%%%%%%%%%%%%%%%%%%%%%%%%%%%%%%%%%%%%%%%%%%%%%
% Mouse Comparison Operators
%%%%%%%%%%%%%%%%%%%%%%%%%%%%%%%%%%%%%%%%%%%%%
\newcommand{\initseg}{\trianglelefteq}
\newcommand{\properseg}{\lhd}
\newcommand{\notinitseg}{\ntrianglelefteq}
\newcommand{\notproperseg}{\ntriangleleft}

%%%%%%%%%%%%%%%%%%%%%%%%%%%%%%%%%%%%%%%%%%%%%
%           calligraphic letters
%%%%%%%%%%%%%%%%%%%%%%%%%%%%%%%%%%%%%%%%%%%%%
\newcommand{\cA}{\mathcal{A}}
\newcommand{\cB}{\mathcal{B}}
\newcommand{\cC}{\mathcal{C}}
\newcommand{\cD}{\mathcal{D}}
\newcommand{\cE}{\mathcal{E}}
\newcommand{\cF}{\mathcal{F}}
\newcommand{\cG}{\mathcal{G}}
\newcommand{\cH}{\mathcal{H}}
\newcommand{\cI}{\mathcal{I}}
\newcommand{\cJ}{\mathcal{J}}
\newcommand{\cK}{\mathcal{K}}
\newcommand{\cL}{\mathcal{L}}
\newcommand{\cM}{\mathcal{M}}
\newcommand{\cN}{\mathcal{N}}
\newcommand{\cO}{\mathcal{O}}
\newcommand{\cP}{\mathcal{P}}
\newcommand{\cQ}{\mathcal{Q}}
\newcommand{\cR}{\mathcal{R}}
\newcommand{\cS}{\mathcal{S}}
\newcommand{\cT}{\mathcal{T}}
\newcommand{\cU}{\mathcal{U}}
\newcommand{\cV}{\mathcal{V}}
\newcommand{\cW}{\mathcal{W}}
\newcommand{\cX}{\mathcal{X}}
\newcommand{\cY}{\mathcal{Y}}
\newcommand{\cZ}{\mathcal{Z}}


%%%%%%%%%%%%%%%%%%%%%%%%%%%%%%%%%%%%%%%%%%%%%
%          Primed Letters
%%%%%%%%%%%%%%%%%%%%%%%%%%%%%%%%%%%%%%%%%%%%%
\newcommand{\aprime}{a^{\prime}}
\newcommand{\bprime}{b^{\prime}}
\newcommand{\cprime}{c^{\prime}}
\newcommand{\dprime}{d^{\prime}}
\newcommand{\eprime}{e^{\prime}}
\newcommand{\fprime}{f^{\prime}}
\newcommand{\gprime}{g^{\prime}}
\newcommand{\hprime}{h^{\prime}}
\newcommand{\iprime}{i^{\prime}}
\newcommand{\jprime}{j^{\prime}}
\newcommand{\kprime}{k^{\prime}}
\newcommand{\lprime}{l^{\prime}}
\newcommand{\mprime}{m^{\prime}}
\newcommand{\nprime}{n^{\prime}}
\newcommand{\oprime}{o^{\prime}}
\newcommand{\pprime}{p^{\prime}}
\newcommand{\qprime}{q^{\prime}}
\newcommand{\rprime}{r^{\prime}}
\newcommand{\sprime}{s^{\prime}}
\newcommand{\tprime}{t^{\prime}}
\newcommand{\uprime}{u^{\prime}}
\newcommand{\vprime}{v^{\prime}}
\newcommand{\wprime}{w^{\prime}}
\newcommand{\xprime}{x^{\prime}}
\newcommand{\yprime}{y^{\prime}}
\newcommand{\zprime}{z^{\prime}}
\newcommand{\Aprime}{A^{\prime}}
\newcommand{\Bprime}{B^{\prime}}
\newcommand{\Cprime}{C^{\prime}}
\newcommand{\Dprime}{D^{\prime}}
\newcommand{\Eprime}{E^{\prime}}
\newcommand{\Fprime}{F^{\prime}}
\newcommand{\Gprime}{G^{\prime}}
\newcommand{\Hprime}{H^{\prime}}
\newcommand{\Iprime}{I^{\prime}}
\newcommand{\Jprime}{J^{\prime}}
\newcommand{\Kprime}{K^{\prime}}
\newcommand{\Lprime}{L^{\prime}}
\newcommand{\Mprime}{M^{\prime}}
\newcommand{\Nprime}{N^{\prime}}
\newcommand{\Oprime}{O^{\prime}}
\newcommand{\Pprime}{P^{\prime}}
\newcommand{\Qprime}{Q^{\prime}}
\newcommand{\Rprime}{R^{\prime}}
\newcommand{\Sprime}{S^{\prime}}
\newcommand{\Tprime}{T^{\prime}}
\newcommand{\Uprime}{U^{\prime}}
\newcommand{\Vprime}{V^{\prime}}
\newcommand{\Wprime}{W^{\prime}}
\newcommand{\Xprime}{X^{\prime}}
\newcommand{\Yprime}{Y^{\prime}}
\newcommand{\Zprime}{Z^{\prime}}
\newcommand{\alphaprime}{\alpha^{\prime}}
\newcommand{\betaprime}{\beta^{\prime}}
\newcommand{\gammaprime}{\gamma^{\prime}}
\newcommand{\Gammaprime}{\Gamma^{\prime}}
\newcommand{\deltaprime}{\delta^{\prime}}
\newcommand{\epsilonprime}{\epsilon^{\prime}}
\newcommand{\kappaprime}{\kappa^{\prime}}
\newcommand{\lambdaprime}{\lambda^{\prime}}
\newcommand{\rhoprime}{\rho^{\prime}}
\newcommand{\Sigmaprime}{\Sigma^{\prime}}
\newcommand{\tauprime}{\tau^{\prime}}
\newcommand{\xiprime}{\xi^{\prime}}
\newcommand{\thetaprime}{\theta^{\prime}}
\newcommand{\Omegaprime}{\Omega^{\prime}}
\newcommand{\cMprime}{\cM^{\prime}}
\newcommand{\cNprime}{\cN^{\prime}}
\newcommand{\cPprime}{\cP^{\prime}}
\newcommand{\cQprime}{\cQ^{\prime}}
\newcommand{\cRprime}{\cR^{\prime}}
\newcommand{\cSprime}{\cS^{\prime}}
\newcommand{\cTprime}{\cT^{\prime}}

%%%%%%%%%%%%%%%%%%%%%%%%%%%%%%%%%%%%%%%%%%%%%
%          bar Letters
%%%%%%%%%%%%%%%%%%%%%%%%%%%%%%%%%%%%%%%%%%%%%
\newcommand{\abar}{\bar{a}}
\newcommand{\bbar}{\bar{b}}
\newcommand{\cbar}{\bar{c}}
\newcommand{\ibar}{\bar{i}}
\newcommand{\jbar}{\bar{j}}
\newcommand{\nbar}{\bar{n}}
\newcommand{\xbar}{\bar{x}}
\newcommand{\ybar}{\bar{y}}
\newcommand{\zbar}{\bar{z}}
\newcommand{\pibar}{\bar{\pi}}
\newcommand{\phibar}{\bar{\varphi}}
\newcommand{\psibar}{\bar{\psi}}
\newcommand{\thetabar}{\bar{\theta}}
\newcommand{\nubar}{\bar{\nu}}

%%%%%%%%%%%%%%%%%%%%%%%%%%%%%%%%%%%%%%%%%%%%%
%          star Letters
%%%%%%%%%%%%%%%%%%%%%%%%%%%%%%%%%%%%%%%%%%%%%
\newcommand{\phistar}{\phi^{*}}
\newcommand{\Mstar}{M^{*}}

%%%%%%%%%%%%%%%%%%%%%%%%%%%%%%%%%%%%%%%%%%%%%
%          dotletters Letters
%%%%%%%%%%%%%%%%%%%%%%%%%%%%%%%%%%%%%%%%%%%%%
\newcommand{\Gdot}{\dot{G}}

%%%%%%%%%%%%%%%%%%%%%%%%%%%%%%%%%%%%%%%%%%%%%
%         check Letters
%%%%%%%%%%%%%%%%%%%%%%%%%%%%%%%%%%%%%%%%%%%%%
\newcommand{\deltacheck}{\check{\delta}}
\newcommand{\gammacheck}{\check{\gamma}}


%%%%%%%%%%%%%%%%%%%%%%%%%%%%%%%%%%%%%%%%%%%%%
%          Formulas
%%%%%%%%%%%%%%%%%%%%%%%%%%%%%%%%%%%%%%%%%%%%%

\newcommand{\formulaphi}{\text{\large $\varphi$}}
\newcommand{\Formulaphi}{\text{\Large $\varphi$}}


%%%%%%%%%%%%%%%%%%%%%%%%%%%%%%%%%%%%%%%%%%%%%
%          Fraktur Letters
%%%%%%%%%%%%%%%%%%%%%%%%%%%%%%%%%%%%%%%%%%%%%

\newcommand{\fa}{\mathfrak{a}}
\newcommand{\fb}{\mathfrak{b}}
\newcommand{\fc}{\mathfrak{c}}
\newcommand{\fk}{\mathfrak{k}}
\newcommand{\fp}{\mathfrak{p}}
\newcommand{\fq}{\mathfrak{q}}
\newcommand{\fr}{\mathfrak{r}}
\newcommand{\fA}{\mathfrak{A}}
\newcommand{\fB}{\mathfrak{B}}
\newcommand{\fC}{\mathfrak{C}}
\newcommand{\fD}{\mathfrak{D}}

%%%%%%%%%%%%%%%%%%%%%%%%%%%%%%%%%%%%%%%%%%%%%
%          Bold Letters
%%%%%%%%%%%%%%%%%%%%%%%%%%%%%%%%%%%%%%%%%%%%%
\newcommand{\ba}{\mathbf{a}}
\newcommand{\bb}{\mathbf{b}}
\newcommand{\bc}{\mathbf{c}}
\newcommand{\bd}{\mathbf{d}}
\newcommand{\be}{\mathbf{e}}
\newcommand{\bu}{\mathbf{u}}
\newcommand{\bv}{\mathbf{v}}
\newcommand{\bw}{\mathbf{w}}
\newcommand{\bx}{\mathbf{x}}
\newcommand{\by}{\mathbf{y}}
\newcommand{\bz}{\mathbf{z}}
\newcommand{\bSigma}{\boldsymbol{\Sigma}}
\newcommand{\bPi}{\boldsymbol{\Pi}}
\newcommand{\bDelta}{\boldsymbol{\Delta}}
\newcommand{\bdelta}{\boldsymbol{\delta}}
\newcommand{\bgamma}{\boldsymbol{\gamma}}
\newcommand{\bGamma}{\boldsymbol{\Gamma}}

%%%%%%%%%%%%%%%%%%%%%%%%%%%%%%%%%%%%%%%%%%%%%
%         Bold numbers
%%%%%%%%%%%%%%%%%%%%%%%%%%%%%%%%%%%%%%%%%%%%%
\newcommand{\bzero}{\mathbf{0}}

%%%%%%%%%%%%%%%%%%%%%%%%%%%%%%%%%%%%%%%%%%%%%
% Projective-Like Pointclasses
%%%%%%%%%%%%%%%%%%%%%%%%%%%%%%%%%%%%%%%%%%%%%
\newcommand{\Sa}[2][\alpha]{\Sigma_{(#1,#2)}}
\newcommand{\Pa}[2][\alpha]{\Pi_{(#1,#2)}}
\newcommand{\Da}[2][\alpha]{\Delta_{(#1,#2)}}
\newcommand{\Aa}[2][\alpha]{A_{(#1,#2)}}
\newcommand{\Ca}[2][\alpha]{C_{(#1,#2)}}
\newcommand{\Qa}[2][\alpha]{Q_{(#1,#2)}}
\newcommand{\da}[2][\alpha]{\delta_{(#1,#2)}}
\newcommand{\leqa}[2][\alpha]{\leq_{(#1,#2)}}
\newcommand{\lessa}[2][\alpha]{<_{(#1,#2)}}
\newcommand{\equiva}[2][\alpha]{\equiv_{(#1,#2)}}


\newcommand{\Sl}[1]{\Sa[\lambda]{#1}}
\newcommand{\Pl}[1]{\Pa[\lambda]{#1}}
\newcommand{\Dl}[1]{\Da[\lambda]{#1}}
\newcommand{\Al}[1]{\Aa[\lambda]{#1}}
\newcommand{\Cl}[1]{\Ca[\lambda]{#1}}
\newcommand{\Ql}[1]{\Qa[\lambda]{#1}}

\newcommand{\San}{\Sa{n}}
\newcommand{\Pan}{\Pa{n}}
\newcommand{\Dan}{\Da{n}}
\newcommand{\Can}{\Ca{n}}
\newcommand{\Qan}{\Qa{n}}
\newcommand{\Aan}{\Aa{n}}
\newcommand{\dan}{\da{n}}
\newcommand{\leqan}{\leqa{n}}
\newcommand{\lessan}{\lessa{n}}
\newcommand{\equivan}{\equiva{n}}

\newcommand{\SigmaOneOmega}{\Sigma^1_{\omega}}
\newcommand{\SigmaOneOmegaPlusOne}{\Sigma^1_{\omega+1}}
\newcommand{\PiOneOmega}{\Pi^1_{\omega}}
\newcommand{\PiOneOmegaPlusOne}{\Pi^1_{\omega+1}}
\newcommand{\DeltaOneOmegaPlusOne}{\Delta^1_{\omega+1}}

%%%%%%%%%%%%%%%%%%%%%%%%%%%%%%%%%%%%%%%%%%%%%
% Linear Algebra
%%%%%%%%%%%%%%%%%%%%%%%%%%%%%%%%%%%%%%%%%%%%%
\newcommand{\transpose}[1]{{#1}^{\text{T}}}
\newcommand{\norm}[1]{\lVert{#1}\rVert}
\newcommand\aug{\fboxsep=-\fboxrule\!\!\!\fbox{\strut}\!\!\!}

%%%%%%%%%%%%%%%%%%%%%%%%%%%%%%%%%%%%%%%%%%%%%
% Number Theory
%%%%%%%%%%%%%%%%%%%%%%%%%%%%%%%%%%%%%%%%%%%%%
\newcommand{\av}[1]{\lvert#1\rvert}
\DeclareMathOperator{\divides}{\mid}
\DeclareMathOperator{\ndivides}{\nmid}
\DeclareMathOperator{\lcm}{lcm}
\DeclareMathOperator{\sign}{sign}
\newcommand{\floor}[1]{\left\lfloor{#1}\right\rfloor}
\DeclareMathOperator{\ConCl}{CC}
\DeclareMathOperator{\ord}{ord}



\newtheorem*{claim}{claim}
\newtheorem*{observation}{Observation}
\newtheorem*{warning}{Warning}
\newtheorem*{question}{Question}
\newtheorem{remark}[theorem]{Remark}

\newenvironment*{subproof}[1][Proof]
{\begin{proof}[#1]}{\renewcommand{\qedsymbol}{$\diamondsuit$} \end{proof}}

\mode<presentation>
{
  \usetheme{Singapore}
  % or ...

  \setbeamercovered{transparent}
  % or whatever (possibly just delete it)
}


\usepackage[english]{babel}
% or whatever

\usepackage[latin1]{inputenc}
% or whatever

\usepackage{times}
\usepackage[T1]{fontenc}
% Or whatever. Note that the encoding and the font should match. If T1
% does not look nice, try deleting the line with the fontenc.


\title{Descriptive Set Theory}


\author{Mitch Rudominer}


\institute{Google}


\date
{February 9, 2018 \\ Dept. of Mathematics, Cal Poly }


% Delete this, if you do not want the table of contents to pop up at
% the beginning of each subsection:
\AtBeginSubsection[]
{
  \begin{frame}<beamer>{Topics}
    \tableofcontents[currentsection,currentsubsection]
  \end{frame}
}


% If you wish to uncover everything in a step-wise fashion, uncomment
% the following command:

\beamerdefaultoverlayspecification{<+->}

\begin{document}

\begin{frame}
  \titlepage
\end{frame}

\begin{frame}{Topics}
  \tableofcontents
  % You might wish to add the option [pausesections]
\end{frame}


% Since this a solution template for a generic talk, very little can
% be said about how it should be structured. However, the talk length
% of between 15min and 45min and the theme suggest that you stick to
% the following rules:

% - Exactly two or three sections (other than the summary).
% - At *most* three subsections per section.
% - Talk about 30s to 2min per frame. So there should be between about
%   15 and 30 frames, all told.

\section{Overview}


\begin{frame}{Descriptive Set Theory}

The study of \emph{definable} sets of reals

\pause

  \begin{itemize}
  \item What does "definable" mean?
  \item There is a "formula" defining membership.
  \item Examples:
  \item $\setof{x\in\R}{\text{decimal expansion of $x$ has inf many 5's}}$
  \item $\setof{y\in\R}{\exists x\in\R \text{ s.t. } y=\fprime(x)}$ where $f$ is some continuous function.
  \item As opposed to a set shown to exist via the Axiom of Choice
  \item "Definable" not precise. Replace with precise concept later.
  \end{itemize}


\end{frame}

\begin{frame}{Themes}

  \begin{itemize}
  \item Arbitrary sets of reals may be exotic.
  \item Simply definable sets are regular.
  \item Definability comes in a \emph{hierarchy of complexity}.
  \item The line between exotic and regular occurs somewhere in this
  hierarchy. Where?
  \item The answer turns out to be \emph{independent} of the axioms of set theory
  and deeply connected to \emph{large cardinals}.
  \end{itemize}
\end{frame}


\section{Constructing Exotic Sets using AC}

\subsection{A non-measurable set}

\begin{frame}{A non-measurable set}

\begin{theorem}
There is a set of reals that is not Lebesgue Measurable.
\end{theorem}

\begin{proof}[Proof sketch (Vitali 1905)]

\begin{itemize}
  \item Say $x\equiv y$ iff $x-y$ is rational.
  \item $\equiv$ is an equivalance relation.
  \item Let $A$ be a set that chooses one element of each equivalance class.
\end{itemize}
\end{proof}

\end{frame}


\subsection{The Axiom of Choice}

\begin{frame}{The Axiom of Choice}

\begin{itemize}
  \item Let $X$ be a family of non-empty sets. Then there exists a choice
  function $f$ for $X$.
  \item Previous proof used AC. How?
\end{itemize}

\begin{question}
Is AC \textbf{necessary} for the previous theorem?
\end{question}

\begin{question}
Can we \textbf{explicitly define} a set $A\subseteq \R$ that is not measurable?
\end{question}

\begin{question}
 Alternatively can we somehow show that every "definable" set
$A\subseteq \R$ \textbf{is} measurable?
\end{question}


\end{frame}



\subsection{The Continuum Hypothesis}

\begin{frame}{The Continuum Hypothesis}

\begin{fact}
\begin{itemize}
  \item $\R$ is uncountable
  \item So $\aleph_0 < \fc$
  \item So $\aleph_0 < \aleph_1 \leq \fc$
\end{itemize}
\end{fact}

\begin{question}
Is there a set $A \subseteq \R$ s.t. $|\Z| < |A| < |\R|$?
\end{question}

\begin{definition}
\begin{itemize}
  \item CH is the assertion that there is no such $A$
  \item Equivalently that $\fc = \aleph_1$.
\end{itemize}
\end{definition}

\end{frame}

\subsection{The Perfect Set Property}

\begin{frame}{Perfect Sets}

\begin{definition}
$A\subseteq\R$ is \emph{perfect} iff it is closed, non-empty
and has no isolated points.
\end{definition}

\begin{lemma}
If $A$ is perfect then $|A| = \fc$.
\end{lemma}

\end{frame}

\begin{frame}{Proof of Lemma}

\begin{proof}{Proof sketch.}
\begin{itemize}
  \item  Let $A$ be a perfect set.

  \item  Let $\cC$ be the set of all infinite sequences of bits.

  \item  We will define an injection $f : \cC \injection A$.


  \item Hint: Define a tree of open intervals labelled by finite sequence of
  bits.

\end{itemize}
\end{proof}
\end{frame}

\begin{frame}{The Perfect Set Propety}

\begin{definition}
\begin{itemize}
  \item $A\subseteq\R$ has the \emph{perfect set property} iff it is either
  countable or contains a perfect set.
  \item So $A$ does \emph{not} violate CH.
  \item So if it were true that all sets had the perfect set property then CH
        would be true.
\end{itemize}
\end{definition}

\end{frame}

\begin{frame}{A set without the perfect set property}

\begin{lemma}
There is a set that does not have the perfect set property.
\end{lemma}

\begin{proof}[Proof sketch]
\begin{itemize}
  \item Enumerate all perfect sets in a sequence of order type $\fc$:
  \item $\sequence{P_{\alpha}}{\alpha < \fc}$
  \item By induction on $\alpha$ build disjoint $A$, $B$ such that both meet $P_{\alpha}$.
  \item Then $A$ and $B$ are disjoint, uncountable, and they each meet every
        perfect set so neither of them contain any perfect set.
\end{itemize}
\end{proof}
\end{frame}

\begin{frame}{Perfect Set Property and AC}
That proof used AC heavily.

\begin{question}
Is AC \textbf{necessary} for the previous lemma?
\end{question}

\begin{question}
Can we \textbf{explicitly define} a set $A\subseteq \R$ that does not have the perfect set property?
\end{question}

\begin{question}
 Alternatively can we somehow show that every "definable" set
$A\subseteq \R$ \textbf{does} have the perfect set property?
\end{question}

\end{frame}

\section{The Regularity of Definable Sets}

\subsection{The simplest definable sets}

\begin{frame}{Open Sets}

We take \emph{open} and \emph{closed} sets as our most simply "definable" sets.

\pause

\begin{itemize}
  \item An open set may be written as a union of countably many
  \emph{rational open intervals}
  $$A = (a_0, b_0) \union (a_1, b_1) \union \cdots$$
  \item Let $x$ be a sequence of integers that encodes the sequences
  $\sequence{a_n}{n\in\omega}$ and $\sequence{b_n}{n\in\omega}$.
  \item We think of $x$ as a \emph{parameter} in the definition
  of $A$.
  \item $A$ is very simply definable relative to the parameter $x$.
\end{itemize}

\end{frame}


\begin{frame}{Closed Sets have the Perfect Set Property}

\begin{theorem}[Cantor-Bendixson, 1883]
Closed sets have the perfect set propety.
\end{theorem}

\begin{proof}[Proof sketch.]
\begin{itemize}
\item Write $F = F^* \union C$, $F^*\intersect C = \emptyset$, where
\item $F^*$ is the set of accumulation points of $F$.
\item Show $C$ is countable
\item Show $F^*$ is perfect if it is not empty.
\end{itemize}
\end{proof}

\begin{corollary}
CH holds for closed sets.
\end{corollary}

\end{frame}

\subsection{The projective hierarchy}

\begin{frame}{Extending the Perfect Set Theorem}

\begin{itemize}
\item How can we extend the theorem that states that closed sets have the
perfect set property?
\item If we consider open and closed sets to be our simplest type of
sets we can try to extend the theorem to more \emph{complex} sets.
\item So how do we make more complex sets out of open and closed sets?
\item We can consider countable unions of closed sets and countable
intersections of open sets.
\item We can continue to make more complex sets by applying the operations
of complimentation, countable union, and countable intersection.
\end{itemize}

\end{frame}

\begin{frame}{The Borel sets}
\begin{definition}
The \emph{Borel sigma algebra} is the smallest collection of subsets
of $\R$ that contains the open sets and is closed under compliment
and countable union.
\end{definition}

\begin{remark}
It's also closed under countable intersections.
\end{remark}

\begin{fact}
All Borel sets are Lebesgue measurable.
\end{fact}

\begin{theorem}[Aleksandrov, Hausdorf, 1916]
All Borel sets have the perfect set property. \\
(So CH holds for the Borel sets.)
\end{theorem}

\end{frame}

\begin{frame}{The Projective Hierarchy}
We want to push the notion of "definable" set beyond the Borel sets.

\pause

For $A\subseteq \R^{k}$ we say that $A$ is

\begin{itemize}
\item  $\bSigma^1_1$ iff it is the \emph{projection} of a Borel subset of $\R^{k+1}$
\item  $\bPi^1_1$ iff it is the compliment of a $\bSigma^1_1$ set.
\item $\bSigma^1_{n+1}$ iff it is the projection of a $\bPi^1_n$ subset of $\R^{k+1}$
\item $\bPi^1_{n}$ iff it is the compliment of a
      $\bSigma^1_{n}$ set
\item $\bDelta^1_{n}$ iff it is both
      $\bSigma^1_{n}$ and $\bPi^1_{n}$
\end{itemize}

\end{frame}

\begin{frame}{Properties of the projective hierarchy}

\begin{fact}
\begin{itemize}
\item  The projective hierarchy forms a \textbf{proper hierarchy}
\item Borel = $\bDelta^1_1$.
\end{itemize}
\end{fact}

\end{frame}

\begin{frame}{Analytic Sets}

\begin{theorem}[Luzin, 1917]
All $\bSigma^1_1$ (and $\bPi^1_1$) sets are Lebesgue measurable.
\end{theorem}

\begin{theorem}[Suslin, 1917]
All $\bSigma^1_1$  sets have the perfect set property.
\end{theorem}

\begin{question}
Are all $\bSigma^1_2$ sets Lebesgue measurable?
\end{question}

\begin{question}
Do all $\bPi^1_1$ sets have the perfect set property?
\end{question}

\end{frame}

\section{The Line between Regular and Exotic}

\subsection{Independence, Consistency Strength and Large Cardinals}

\begin{frame}{Second Incompleteness Theorem}
\begin{theorem}[G\"{o}del's Second Incompleteness]
Any sufficiently strong consistent theory cannot prove its own consistency.
\end{theorem}

\begin{definition}
A \emph{theory} is a set of sentences.
\end{definition}

\begin{itemize}
\item $\con(T)$ means $T$ is a consistent theory.
\item G\"{o}del's theorem says that, assuming $T$ is consistent,  working in
$T$, we cannot prove $\con(T)$.
\item In symbols $T\not\vdash\con(T)$
\end{itemize}

\end{frame}

\begin{frame}{CH is independent of ZFC}

\pause

We can only prove \emph{relative} consistency.

\pause

\begin{theorem}[G\"{o}del, 1938]
$\con(\ZFC) \implies \con(\ZFC + \CH)$
\end{theorem}

\begin{theorem}[Cohen, 1963]
$\con(\ZFC) \implies \con(\ZFC + \neg\CH)$
\end{theorem}

\end{frame}

\begin{frame}{Consistency Strength}
Let $S$ and $T$ be theores.
\begin{itemize}
\item If $\con(S) \implies \con(T)$ we say that the \emph{consistency strength}
of $S$ is greater than or equal to the consistency strength of $T$. We think
of this as a partial order on theories.
\item Now suppose that $S\vdash\con(T)$
\item Notice that then we must have that $\con(T) \not\implies \con(S)$.
\item Because otherwise we would get $S\vdash\con(S)$ which violates second
incompleteness.
\item Therefore we say that $S$ has
\emph{greater consistency strengh} than $T$.
\end{itemize}
\end{frame}

\begin{frame}{Large Cardinals}

\begin{itemize}
\item Large cardinals are the standard candles by which set theorists
measure consistency strength.
\item Infinite cardinals, the existence of which has consistency strength.
\item They come in a hierarcny of consistency strength
\item $\ZFC + \text{ } \exists \text{ innaccessible} \vdash \con(\ZFC)$.
\item $\ZFC + \text{ } \exists \text{ measurable} \vdash \con(\ZFC + \text{ } \exists \text{ innacc.})$.
\item $\ZFC + \text{ } \exists \text{ Woodin} \vdash \con(\ZFC + \text{ } \exists \text{ measurable})$.
\end{itemize}

\end{frame}

\subsection{Lebesgue Measurability and the Perfect Set Property}

\begin{frame}{Why the early descriptive set theorists were frustrated}
\begin{theorem}[G\"{o}del, 1938]

Let $\sigma$ be the sentence that asserts:
\begin{quotation}
\emph{"There is a $\bDelta^1_2$ subset
of $\R$ that is not Lebesgue measurable."}
\end{quotation}

Let $\tau$ be the sentence that asserts:
\begin{quotation}
\emph{"There is a $\bPi^1_1$ subset of $\R$ that does not
have the perfect set property."}
\end{quotation}

\pause

Then
$$\con(\ZFC) \implies \con(\ZFC + \sigma + \tau)$$
\end{theorem}
\end{frame}

\begin{frame}{Large Cardinal Strength From Regularity Properites}
\begin{theorem}[Solovay, 1969]
Let $\sigma$ be the sentence that asserts:
\begin{quotation}
\emph{"All $\bPi^1_1$ sets have the perfect set property". }
\end{quotation}

\pause

Then
$$  \con(\ZFC + \sigma) \implies \con(\ZFC + \text{ } \exists \text{ inacc.})$$
\end{theorem}

\pause

\begin{theorem}[Shelah, 1984]
Let $\sigma$ be the sentence that asserts:
\begin{quotation}
\emph{"All $\bSigma^1_3$ sets are Lebesge measurable". }
\end{quotation}

\pause

Then
$$  \con(\ZFC + \sigma) \implies \con(\ZFC + \text{ } \exists \text{ inacc.})$$
\end{theorem}

\end{frame}


\begin{frame}{Solovay's Model}
\begin{theorem}[Solovay, 1970]

$$\con(\ZFC + \text{ } \exists \text{ inacc.}) \implies \con(\ZFC + \sigma)$$

where

$\sigma$ is the sentence that asserts:

\begin{quotation}
"\textbf{All projective} sets are Lebesgue
measurable and have the perfect set property".
\end{quotation}

\end{theorem}
\end{frame}

\begin{frame}{Solovay's Model, 2}
\begin{theorem}[Solovay, 1970]

$$\con(\ZFC + \text{ } \exists \text{ inacc.}) \implies \con(\ZF + \sigma)$$

where

$\sigma$ is the sentence that asserts:

\begin{quotation}
"\textbf{All} sets are Lebesgue
measurable and have the perfect set property".
\end{quotation}
\end{theorem}
\end{frame}

\begin{frame}{Summarizing the last few slides}

The following are equiconstent:

\begin{itemize}
  \item $\ZFC + \text{ } \exists \text{ inacc.}$
  \item ZFC plus all $\bPi^1_1$ sets have the perfet set property and all $\bSigma^1_2$ sets are Lebesgue measurable.
  \item ZFC plus all projective sets have the perfect set property and are Lebesgue measurable.
  \item ZF plus all sets have the perfect set property and are Lebesgue measurable.
\end{itemize}

\end{frame}


\subsection{Determinacy}

\subsection{The Strength of Projective Determinacy}


\end{document}


