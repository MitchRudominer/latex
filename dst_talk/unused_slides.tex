
\begin{frame}{proof}

\begin{frame}{A non-measurable set}
\begin{theorem}
There is a set of reals that is not Lebesgue Measurable.
\end{theorem}

\vfill

\pause
You need to know very little  about what Lebesgue measure is to
understand the proof. Only that it is
\begin{itemize}
  \item countably additive
  \item translation invariant
\end{itemize}

\end{frame}

\begin{proof}[Proof sketch (Vitali 1905)]
We are going to construct a set
$A \subseteq [0, 1]$ so that $[0,1] = A_0 \union A_1 \union \cdots$ where each
$A_n$ is a rigid copy of $A$. Such an $A$ cannot be measurable. Why?

\begin{itemize}
  \item Say $x\equiv y$ iff $x-y$ is rational.
  \item $\equiv$ is an equivalance relation.
  \item Let $A$ be a set that chooses one element of each equivalance class.
  \item Let $\sequence{q_n}{n = 1,2,\cdots}$ enumerate all rational numbers in
$[0,1]$.
   \item Let $A_n = A + q_n$ (mod $\Z$).
\end{itemize}
\end{proof}
\end{frame}

\begin{frame}{Well-orderings}

\begin{definition}
A \textbf{strict linear ordering} of a set $A$ is a binary relation $<$ that is

\begin{itemize}
  \item transitive
  \item asymmetric
  \item connected
\end{itemize}

\end{definition}

\begin{definition}
A \textbf{well-ordering} of a set $A$ is a srict linear ordering $<$ s.t.

\begin{itemize}
  \item every subset of $A$ has a $<$-least element.
\end{itemize}

\end{definition}

\begin{definition}
A \textbf{well-order} is a pair $(A, <)$ where $<$ is a well-ordering of $A$.
\end{definition}

\end{frame}

\begin{frame}{The Ordinals}

\begin{itemize}
  \item  Every two well-orders $(A, <_A)$ and $(B, <_B)$ are comparable.
  \item  A finite well-order is isomorphic to an integer $n\geq 0$.
  \item The \emph{order-type} of a well-order is the class of all well-orders
  isomorphic to it.
  \item The \emph{ordinals} are canoncial representatives of the
    order-types of the well-orders.
  \item Transfinite counting numbers.
  \item $\omega = \singleton{0, 1, 2, \cdots }$ is the set of non-negative
    integers. $\omega$ is the first infinite ordinal.
  \item  $\omega_1 = $ the set of all countable ordinals, is the first
    uncountable ordinal
\end{itemize}

\end{frame}

\begin{frame}{AC and well-orders}

The following are equivalent:
\begin{itemize}
  \item  The Axiom of Choice
  \item  Every set can be well-ordered.
  \item For every set $A$ there is a bijection between some ordinal and $A$.
\end{itemize}

\end{frame}

\begin{frame}{A well-ordering of the Reals}

\begin{itemize}
  \item  In particular AC implies: \emph{There is a well-ordering of $\R$.}
  \item  This suffices for the Vitali proof.
\end{itemize}

\begin{question}
Can we \textbf{explicitly define} a well-ordering of $\R$?
\end{question}

\begin{fact}
If $<$ is a well-ordering of $\R$ then $<$ itself is
already a non-measurable set (as a subset of $\R^2$).
\end{fact}

\end{frame}

\begin{frame}{Cardinals}

\begin{definition}
\begin{itemize}
  \item $|A| = |B|$ iff there is a bijection $f:A\bijection B$.
  \item Say $A$ and $B$ are \emph{equinumerous} or have the same \emph{cardinality}
  \item We use certain ordinals as the canonical represetatives of each cardinality.
  \item A \emph{cardinal} is an ordinal that is least with its cardinality.
  \item  $\aleph_0 = |\omega|$ is the cardinality of the countable
  \item  $\aleph_1 = |\omega_1|$ is the first uncountable cardinal
  \item $\fc  = 2^{\aleph_0} = |\R|$
\end{itemize}
\end{definition}

\end{frame}

\begin{frame}{CH and Descriptive Set Theory}

\begin{question}
Can we explicitly define a set $A$ that contradicts CH?
\end{question}

\begin{question}
Alternatively can we somehow show that every "definable" subset of
$\R$ satisfies CH?
\end{question}

\end{frame}

\begin{frame}{Proof of Lemma (continued: 2)}

\begin{claim}
If $A$ is perfect and $I$ is an open interval and $I \intersect A \neq
\emptyset$ and $\epsilon > 0$ then there are two open intervals $I_1$ and
$I_2$ of length less $\epsilon$ such that the closures of $I_1$ and
$I_2$ are disjoint and contained in $I$ and such that $I_1$ and $I_2$
both intersect $A$.
\end{claim}

\begin{subproof}[proof of claim]
Let $x_1 \neq x_2$ be in $I \cap A$, which is posible because $A$
contains no isolated points. Let $I_1$ and $I_2$ be sufficiently small,
disjoint intervals containing $x_1$ and $x_2$ respectively.
\end{subproof}

\end{frame}

\begin{frame}{Proof of Lemma (continued: 3)}

For each $s$, a finite sequence of bits, define by induction on $\length(s)$
an open interval $I_s$ such that

\pause

\begin{itemize}
  \item  $I_s \intersect A \neq \emptyset$

  \item   $\length(I_s) < \frac{1}{\length(s) + 1}$

  \item  $\overline{I_{s \adjoin 0}} \subseteq I_s$ and $\overline{I_{s \adjoin 1}} \subseteq I_s$

  \item  $\overline{I_{s \adjoin 0}} \intersect \overline{I_{s \adjoin 1}} = \emptyset$

\end{itemize}

\pause

This is possible by repeatedly applying the claim.

\end{frame}

\begin{frame}{Proof of Lemma (continued: 4)}

Now for $x\in\cC$ define $f(x)$ to be the unique point in the
intersection of $I_s$ such that $s$ is an initial segment of $x$.

Then
$$f : \cC \injection A$$


$\qed$

\end{frame}

\begin{frame}{Computably Open Sets}

\begin{definition}
\begin{itemize}
  \item A rational open interval is an interval $(a, b)$ s.t.
        $a<b$ are rational numbers.
  \item A \emph{computably open} set is a set of the form
  $$(a_0, b_0) \union (a_1, b_1) \union \cdots$$
  where $\sequence{a_n}{n\in\omega}$ and
  $\sequence{b_n}{n\in\omega}$ are computably enumerable
  sequences of rationals.
  \item There are semi-decision procedures for membership in these kinds
  of sets.
  \item $A$ is \emph{computably closed} iff its compliment is computably open.
\end{itemize}
\end{definition}

\end{frame}

\begin{frame}{Closed Sets have the Perfect Set Property}

\begin{theorem}
Closed sets have the perfect set propety.
\end{theorem}

\begin{definition}
Suppose $A\subseteq \R$ and $x\in A$. We will say that $x$ is a
\emph{condensation point} of $A$ if every open neighborhood of $x$ contains
uncountably many elements of $A$.
\end{definition}

\begin{definition}
Let $F$ be any set of reals. Then $F^*$ is the set of condensation points
of $F$.
\end{definition}

\begin{observation}
$F^*$ is closed.
\end{observation}

\end{frame}

\begin{frame}{Proof of Theorem}
\begin{block}{Proof.}
Let $F$ be any closed set. We want to show that either $F$ is countable or
else $F$ contains a perfect subset.  Write $F = F^* \union C$ where
$C=F-F^*$.

\begin{claim}[1]
$C$ is countable.
\end{claim}

\begin{claim}[2]
$F^*$ is perfect if it is not empty
\end{claim}

\pause

These two claims suffice.

\end{block}

\end{frame}

\begin{frame}{Proof of Theorem (continued: 2)}
\begin{subproof}[proof of claim 1]
$C$ is the union of countably many sets $I_n \intersect F$ where the
$I_n$ enumerate the rational open intervals whose intersection
with $F$ are countable.
\end{subproof}

\begin{subproof}[proof of claim 2]
Suppose $F^* \neq \emptyset.$ Let $x\in F^*$ and let $I$ be an open
neighborhood of $x$. Then $I$ contains uncountably many elements of $F$ and
since $C$ is countable, $I$ contains at least one element of $F^*$, so $x$
is not an isolated point of $F^*$.
\end{subproof}

\qed

\end{frame}

\begin{frame}{The Borel Hierarchy}
The Borel sets can be ramified into a hierarchy.

\pause

For $A\subseteq \R^{n}$ we say that $A$ is

\begin{itemize}
\item  $\bSigma^0_1$ iff it is open
\item  $\bPi^0_1$ iff it is closed
\item and then for $\alpha > 1$ a countable ordinal,
\item $\bSigma^0_{\alpha}$ iff it is the countable union of
       $\bPi^0_{\beta}$ sets for $\beta < \alpha$
\item $\bPi^0_{\alpha}$ iff it is the compliment of a
      $\bSigma^0_{\alpha}$ set
\item $\bDelta^0_{\alpha}$ iff it is both
      $\bSigma^0_{\alpha}$ and $\bPi^0_{\alpha}$
\end{itemize}

\begin{fact}
These \textbf{pointclasses} form a \text{proper hierarchy} whose union
is the Borel sigma algebra.
\end{fact}

\begin{remark}
Notice the \textbf{boldface} font for $\bSigma$, $\bPi$ and $\bDelta$.
\end{remark}

\end{frame}
