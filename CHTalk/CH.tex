%%Prosper slide presentation template
%%Build using latex+dvipdf
%%Note that the prosper package must be separately installed
%%See http://prosper.sourceforge.net/
\documentclass[pdf,final]{prosper}

\usepackage{amsmath,amssymb,latexsym,eucal,amsthm}
%%%%%%%%%%%%%%%%%%%%%%%%%%%%%%%%%%%%%%%%%%%%%
% Common Set Theory Constructs
%%%%%%%%%%%%%%%%%%%%%%%%%%%%%%%%%%%%%%%%%%%%%

\newcommand{\setof}[2]{\left\{ \, #1 \, \left| \, #2 \, \right.\right\}}
\newcommand{\lsetof}[2]{\left\{\left. \, #1 \, \right| \, #2 \,  \right\}}
\newcommand{\bigsetof}[2]{\bigl\{ \, #1 \, \bigm | \, #2 \,\bigr\}}
\newcommand{\Bigsetof}[2]{\Bigl\{ \, #1 \, \Bigm | \, #2 \,\Bigr\}}
\newcommand{\biggsetof}[2]{\biggl\{ \, #1 \, \biggm | \, #2 \,\biggr\}}
\newcommand{\Biggsetof}[2]{\Biggl\{ \, #1 \, \Biggm | \, #2 \,\Biggr\}}
\newcommand{\dotsetof}[2]{\left\{ \, #1 \, : \, #2 \, \right\}}
\newcommand{\bigdotsetof}[2]{\bigl\{ \, #1 \, : \, #2 \,\bigr\}}
\newcommand{\Bigdotsetof}[2]{\Bigl\{ \, #1 \, \Bigm : \, #2 \,\Bigr\}}
\newcommand{\biggdotsetof}[2]{\biggl\{ \, #1 \, \biggm : \, #2 \,\biggr\}}
\newcommand{\Biggdotsetof}[2]{\Biggl\{ \, #1 \, \Biggm : \, #2 \,\Biggr\}}
\newcommand{\sequence}[2]{\left\langle \, #1 \,\left| \, #2 \, \right. \right\rangle}
\newcommand{\lsequence}[2]{\left\langle\left. \, #1 \, \right| \,#2 \,  \right\rangle}
\newcommand{\bigsequence}[2]{\bigl\langle \,#1 \, \bigm | \, #2 \, \bigr\rangle}
\newcommand{\Bigsequence}[2]{\Bigl\langle \,#1 \, \Bigm | \, #2 \, \Bigr\rangle}
\newcommand{\biggsequence}[2]{\biggl\langle \,#1 \, \biggm | \, #2 \, \biggr\rangle}
\newcommand{\Biggsequence}[2]{\Biggl\langle \,#1 \, \Biggm | \, #2 \, \Biggr\rangle}
\newcommand{\singleton}[1]{\left\{#1\right\}}
\newcommand{\angles}[1]{\left\langle #1 \right\rangle}
\newcommand{\bigangles}[1]{\bigl\langle #1 \bigr\rangle}
\newcommand{\Bigangles}[1]{\Bigl\langle #1 \Bigr\rangle}
\newcommand{\biggangles}[1]{\biggl\langle #1 \biggr\rangle}
\newcommand{\Biggangles}[1]{\Biggl\langle #1 \Biggr\rangle}

\newcommand{\force}[1]{\Vert\!\underset{\!\!\!\!\!#1}{\!\!\!\relbar\!\!\!%
\relbar\!\!\relbar\!\!\relbar\!\!\!\relbar\!\!\relbar\!\!\relbar\!\!\!%
\relbar\!\!\relbar\!\!\relbar}}
\newcommand{\nforce}[1]{\Vert\!\underset{\!\!\!\!\!#1}{\!\!\!\relbar\!\!\!%
\relbar\!\!\relbar\!\!\relbar\!\!\!\relbar\!\!\relbar\!\!\relbar\!\!\!%
\relbar\!\!\not\relbar\!\!\relbar}}
\newcommand{\forcein}[2]{\overset{#2}{\Vert\underset{\!\!\!\!\!#1}%
{\!\!\!\relbar\!\!\!\relbar\!\!\relbar\!\!\relbar\!\!\!\relbar\!\!\relbar\!%
\!\relbar\!\!\!\relbar\!\!\relbar\!\!\relbar\!\!\relbar\!\!\!\relbar\!\!%
\relbar\!\!\relbar}}}

\newcommand{\pre}[2]{{}^{#2}\!{#1}}

\newcommand{\restr}{\!\!\upharpoonright\!}

%%%%%%%%%%%%%%%%%%%%%%%%%%%%%%%%%%%%%%%%%%%%%
% Set-Theoretic Connectives
%%%%%%%%%%%%%%%%%%%%%%%%%%%%%%%%%%%%%%%%%%%%%

\newcommand{\intersect}{\cap}
\newcommand{\union}{\cup}
\newcommand{\Intersection}[1]{\bigcap\limits_{#1}}
\newcommand{\Union}[1]{\bigcup\limits_{#1}}

%%%%%%%%%%%%%%%%%%%%%%%%%%%%%%%%%%%%%%%%%%%%%
% Miscellaneous
%%%%%%%%%%%%%%%%%%%%%%%%%%%%%%%%%%%%%%%%%%%%%
\newcommand{\defeq}{=_{\text{def}}}
\newcommand{\Turingleq}{\leq_{\text{T}}}
\newcommand{\Turingless}{<_{\text{T}}}
\newcommand{\lexleq}{\leq_{\text{lex}}}
\newcommand{\lexless}{<_{\text{lex}}}
\newcommand{\Turingequiv}{\equiv_{\text{T}}}

%%%%%%%%%%%%%%%%%%%%%%%%%%%%%%%%%%%%%%%%%%%%%
% Constants
%%%%%%%%%%%%%%%%%%%%%%%%%%%%%%%%%%%%%%%%%%%%%
\newcommand{\R}{\mathbb{R}}
\newcommand{\Rplus}{\mathbb{R}^{+}}
\renewcommand{\P}{\mathbb{P}}
\newcommand{\Q}{\mathbb{Q}}
\newcommand{\Z}{\mathbb{Z}}
\newcommand{\C}{\mathbb{C}}
\newcommand{\N}{\mathbb{N}}
\newcommand{\LofR}{L(\R)}
\newcommand{\JofR}[1]{J_{#1}(\R)}
\newcommand{\SofR}[1]{S_{#1}(\R)}
\newcommand{\JalphaR}{\JofR{\alpha}}
\newcommand{\JbetaR}{\JofR{\beta}}
\newcommand{\JlambdaR}{\JofR{\lambda}}
\newcommand{\SalphaR}{\SofR{\alpha}}
\newcommand{\SbetaR}{\SofR{\beta}}
\DeclareMathOperator{\ORD}{OR}
\DeclareMathOperator{\Ord}{OR}
\DeclareMathOperator{\WO}{WO}
\DeclareMathOperator{\OD}{OD}
\DeclareMathOperator{\HOD}{HOD}
\DeclareMathOperator{\HC}{HC}
\DeclareMathOperator{\WF}{WF}
\DeclareMathOperator{\HF}{HF}
\DeclareMathOperator{\AR}{AR}
\newcommand{\One}{I}
\newcommand{\ONE}{I}
\newcommand{\Two}{II}
\newcommand{\TWO}{II}

%%%%%%%%%%%%%%%%%%%%%%%%%%%%%%%%%%%%%%%%%%%%%
% Theories
%%%%%%%%%%%%%%%%%%%%%%%%%%%%%%%%%%%%%%%%%%%%%
\DeclareMathOperator{\ZFC}{ZFC}
\DeclareMathOperator{\ZF}{ZF}
\DeclareMathOperator{\AD}{AD}
\DeclareMathOperator{\KP}{KP}
\DeclareMathOperator{\PD}{PD}
\newcommand{\CH}{\textbf{CH} }
\newcommand{\AC}{\textbf{AC} }
%%%%%%%%%%%%%%%%%%%%%%%%%%%%%%%%%%%%%%%%%%%%%
% Iteration Trees
%%%%%%%%%%%%%%%%%%%%%%%%%%%%%%%%%%%%%%%%%%%%%

\newcommand{\pred}{\text{-pred}}

%%%%%%%%%%%%%%%%%%%%%%%%%%%%%%%%%%%%%%%%%%%%%%%%
% Operator Names
%%%%%%%%%%%%%%%%%%%%%%%%%%%%%%%%%%%%%%%%%%%%%%%%
\DeclareMathOperator{\Det}{Det}
\DeclareMathOperator{\dom}{dom}
\DeclareMathOperator{\ran}{ran}
\DeclareMathOperator{\range}{ran}
\DeclareMathOperator{\crit}{crit}
\DeclareMathOperator{\card}{card}
\DeclareMathOperator{\cof}{cof}
\DeclareMathOperator{\rank}{rank}
\DeclareMathOperator{\ot}{o.t.}
\DeclareMathOperator{\ro}{r.o.}
\DeclareMathOperator{\rud}{rud}
\DeclareMathOperator{\Powerset}{\mathcal{P}}
\DeclareMathOperator{\length}{lh}
\DeclareMathOperator{\lh}{lh}
\DeclareMathOperator{\fld}{fld}
\DeclareMathOperator{\projection}{p}
\DeclareMathOperator{\Ult}{Ult}
\DeclareMathOperator{\ULT}{Ult}
\DeclareMathOperator{\Coll}{Coll}
\DeclareMathOperator{\Col}{Coll}
\DeclareMathOperator{\Hull}{Hull}
\DeclareMathOperator*{\dirlim}{dir lim}
\DeclareMathOperator{\supp}{supp}
\DeclareMathOperator{\trancl}{tran.cl.}
%%%%%%%%%%%%%%%%%%%%%%%%%%%%%%%%%%%%%%%%%%%%%
% Logical Connectives
%%%%%%%%%%%%%%%%%%%%%%%%%%%%%%%%%%%%%%%%%%%%%
\newcommand{\IImplies}{\Longrightarrow}
\newcommand{\Ifff}{\Longleftrightarrow}
\newcommand{\iimplies}{\longrightarrow}
\newcommand{\ifff}{\longleftrightarrow}
\newcommand{\Implies}{\Rightarrow}
\newcommand{\Iff}{\Leftrightarrow}
\renewcommand{\implies}{\rightarrow}
\renewcommand{\iff}{\leftrightarrow}
\newcommand{\AND}{\wedge}
\newcommand{\OR}{\vee}

%%%%%%%%%%%%%%%%%%%%%%%%%%%%%%%%%%%%%%%%%%%%%
% Function Arrows
%%%%%%%%%%%%%%%%%%%%%%%%%%%%%%%%%%%%%%%%%%%%%

\newcommand{\injection}{\xrightarrow{\text{1-1}}}
\newcommand{\surjection}{\xrightarrow{\text{onto}}}
\newcommand{\bijection}{\xrightarrow[\text{onto}]{\text{1-1}}}
\newcommand{\cofmap}{\xrightarrow{\text{cof}}}
\newcommand{\map}{\rightarrow}

%%%%%%%%%%%%%%%%%%%%%%%%%%%%%%%%%%%%%%%%%%%%%
% Mouse Comparison Operators
%%%%%%%%%%%%%%%%%%%%%%%%%%%%%%%%%%%%%%%%%%%%%
\newcommand{\initseg}{\trianglelefteq}
\newcommand{\properseg}{\lhd}
\newcommand{\notinitseg}{\ntrianglelefteq}
\newcommand{\notproperseg}{\ntriangleleft}

%%%%%%%%%%%%%%%%%%%%%%%%%%%%%%%%%%%%%%%%%%%%%
%           calligraphic letters
%%%%%%%%%%%%%%%%%%%%%%%%%%%%%%%%%%%%%%%%%%%%%
\newcommand{\cA}{\mathcal{A}}
\newcommand{\cB}{\mathcal{B}}
\newcommand{\cC}{\mathcal{C}}
\newcommand{\cD}{\mathcal{D}}
\newcommand{\cE}{\mathcal{E}}
\newcommand{\cF}{\mathcal{F}}
\newcommand{\cG}{\mathcal{G}}
\newcommand{\cH}{\mathcal{H}}
\newcommand{\cI}{\mathcal{I}}
\newcommand{\cJ}{\mathcal{J}}
\newcommand{\cK}{\mathcal{K}}
\newcommand{\cL}{\mathcal{L}}
\newcommand{\cM}{\mathcal{M}}
\newcommand{\cN}{\mathcal{N}}
\newcommand{\cO}{\mathcal{O}}
\newcommand{\cP}{\mathcal{P}}
\newcommand{\cQ}{\mathcal{Q}}
\newcommand{\cR}{\mathcal{R}}
\newcommand{\cS}{\mathcal{S}}
\newcommand{\cT}{\mathcal{T}}
\newcommand{\cU}{\mathcal{U}}
\newcommand{\cV}{\mathcal{V}}
\newcommand{\cW}{\mathcal{W}}
\newcommand{\cX}{\mathcal{X}}
\newcommand{\cY}{\mathcal{Y}}
\newcommand{\cZ}{\mathcal{Z}}


%%%%%%%%%%%%%%%%%%%%%%%%%%%%%%%%%%%%%%%%%%%%%
%          Primed Letters
%%%%%%%%%%%%%%%%%%%%%%%%%%%%%%%%%%%%%%%%%%%%%
\newcommand{\aprime}{a^{\prime}}
\newcommand{\bprime}{b^{\prime}}
\newcommand{\cprime}{c^{\prime}}
\newcommand{\dprime}{d^{\prime}}
\newcommand{\eprime}{e^{\prime}}
\newcommand{\fprime}{f^{\prime}}
\newcommand{\gprime}{g^{\prime}}
\newcommand{\hprime}{h^{\prime}}
\newcommand{\iprime}{i^{\prime}}
\newcommand{\jprime}{j^{\prime}}
\newcommand{\kprime}{k^{\prime}}
\newcommand{\lprime}{l^{\prime}}
\newcommand{\mprime}{m^{\prime}}
\newcommand{\nprime}{n^{\prime}}
\newcommand{\oprime}{o^{\prime}}
\newcommand{\pprime}{p^{\prime}}
\newcommand{\qprime}{q^{\prime}}
\newcommand{\rprime}{r^{\prime}}
\newcommand{\sprime}{s^{\prime}}
\newcommand{\tprime}{t^{\prime}}
\newcommand{\uprime}{u^{\prime}}
\newcommand{\vprime}{v^{\prime}}
\newcommand{\wprime}{w^{\prime}}
\newcommand{\xprime}{x^{\prime}}
\newcommand{\yprime}{y^{\prime}}
\newcommand{\zprime}{z^{\prime}}
\newcommand{\Aprime}{A^{\prime}}
\newcommand{\Bprime}{B^{\prime}}
\newcommand{\Cprime}{C^{\prime}}
\newcommand{\Dprime}{D^{\prime}}
\newcommand{\Eprime}{E^{\prime}}
\newcommand{\Fprime}{F^{\prime}}
\newcommand{\Gprime}{G^{\prime}}
\newcommand{\Hprime}{H^{\prime}}
\newcommand{\Iprime}{I^{\prime}}
\newcommand{\Jprime}{J^{\prime}}
\newcommand{\Kprime}{K^{\prime}}
\newcommand{\Lprime}{L^{\prime}}
\newcommand{\Mprime}{M^{\prime}}
\newcommand{\Nprime}{N^{\prime}}
\newcommand{\Oprime}{O^{\prime}}
\newcommand{\Pprime}{P^{\prime}}
\newcommand{\Qprime}{Q^{\prime}}
\newcommand{\Rprime}{R^{\prime}}
\newcommand{\Sprime}{S^{\prime}}
\newcommand{\Tprime}{T^{\prime}}
\newcommand{\Uprime}{U^{\prime}}
\newcommand{\Vprime}{V^{\prime}}
\newcommand{\Wprime}{W^{\prime}}
\newcommand{\Xprime}{X^{\prime}}
\newcommand{\Yprime}{Y^{\prime}}
\newcommand{\Zprime}{Z^{\prime}}
\newcommand{\alphaprime}{\alpha^{\prime}}
\newcommand{\betaprime}{\beta^{\prime}}
\newcommand{\gammaprime}{\gamma^{\prime}}
\newcommand{\Gammaprime}{\Gamma^{\prime}}
\newcommand{\deltaprime}{\delta^{\prime}}
\newcommand{\epsilonprime}{\epsilon^{\prime}}
\newcommand{\kappaprime}{\kappa^{\prime}}
\newcommand{\lambdaprime}{\lambda^{\prime}}
\newcommand{\rhoprime}{\rho^{\prime}}
\newcommand{\Sigmaprime}{\Sigma^{\prime}}
\newcommand{\tauprime}{\tau^{\prime}}
\newcommand{\xiprime}{\xi^{\prime}}
\newcommand{\thetaprime}{\theta^{\prime}}
\newcommand{\Omegaprime}{\Omega^{\prime}}
\newcommand{\cMprime}{\cM^{\prime}}
\newcommand{\cNprime}{\cN^{\prime}}
\newcommand{\cPprime}{\cP^{\prime}}
\newcommand{\cQprime}{\cQ^{\prime}}
\newcommand{\cRprime}{\cR^{\prime}}
\newcommand{\cSprime}{\cS^{\prime}}
\newcommand{\cTprime}{\cT^{\prime}}

%%%%%%%%%%%%%%%%%%%%%%%%%%%%%%%%%%%%%%%%%%%%%
%          bar Letters
%%%%%%%%%%%%%%%%%%%%%%%%%%%%%%%%%%%%%%%%%%%%%
\newcommand{\abar}{\bar{a}}
\newcommand{\bbar}{\bar{b}}
\newcommand{\fbar}{\bar{f}}
\newcommand{\gbar}{\bar{g}}

%%%%%%%%%%%%%%%%%%%%%%%%%%%%%%%%%%%%%%%%%%%%%
%          Formulas
%%%%%%%%%%%%%%%%%%%%%%%%%%%%%%%%%%%%%%%%%%%%%

\newcommand{\formulaphi}{\text{\large $\varphi$}}
\newcommand{\Formulaphi}{\text{\Large $\varphi$}}


%%%%%%%%%%%%%%%%%%%%%%%%%%%%%%%%%%%%%%%%%%%%%
%          Fraktur Letters
%%%%%%%%%%%%%%%%%%%%%%%%%%%%%%%%%%%%%%%%%%%%%


\newcommand{\fA}{\mathfrak{A}}
\newcommand{\fB}{\mathfrak{B}}
\newcommand{\fC}{\mathfrak{C}}
\newcommand{\fD}{\mathfrak{D}}

%%%%%%%%%%%%%%%%%%%%%%%%%%%%%%%%%%%%%%%%%%%%%
%          Bold Letters
%%%%%%%%%%%%%%%%%%%%%%%%%%%%%%%%%%%%%%%%%%%%%

\newcommand{\bd}{\mathbf{d}}

%%%%%%%%%%%%%%%%%%%%%%%%%%%%%%%%%%%%%%%%%%%%%
% Projective-Like Pointclasses
%%%%%%%%%%%%%%%%%%%%%%%%%%%%%%%%%%%%%%%%%%%%%


\newcommand{\Sa}[2][\alpha]{\Sigma_{(#1,#2)}}
\newcommand{\Pa}[2][\alpha]{\Pi_{(#1,#2)}}
\newcommand{\Da}[2][\alpha]{\Delta_{(#1,#2)}}
\newcommand{\Aa}[2][\alpha]{A_{(#1,#2)}}
\newcommand{\Ca}[2][\alpha]{C_{(#1,#2)}}
\newcommand{\Qa}[2][\alpha]{Q_{(#1,#2)}}
\newcommand{\da}[2][\alpha]{\delta_{(#1,#2)}}
\newcommand{\leqa}[2][\alpha]{\leq_{(#1,#2)}}
\newcommand{\lessa}[2][\alpha]{<_{(#1,#2)}}
\newcommand{\equiva}[2][\alpha]{\equiv_{(#1,#2)}}


\newcommand{\Sl}[1]{\Sa[\lambda]{#1}}
\newcommand{\Pl}[1]{\Pa[\lambda]{#1}}
\newcommand{\Dl}[1]{\Da[\lambda]{#1}}
\newcommand{\Al}[1]{\Aa[\lambda]{#1}}
\newcommand{\Cl}[1]{\Ca[\lambda]{#1}}
\newcommand{\Ql}[1]{\Qa[\lambda]{#1}}

\newcommand{\San}{\Sa{n}}
\newcommand{\Pan}{\Pa{n}}
\newcommand{\Dan}{\Da{n}}
\newcommand{\Can}{\Ca{n}}
\newcommand{\Qan}{\Qa{n}}
\newcommand{\Aan}{\Aa{n}}
\newcommand{\dan}{\da{n}}
\newcommand{\leqan}{\leqa{n}}
\newcommand{\lessan}{\lessa{n}}
\newcommand{\equivan}{\equiva{n}}

%%%%%%%%%%%%%%%%%%%%%%%%%%%%%%%%%%%%%%%%%%%%%%%%%%%%%%%%%%%%%%%%%%%%%%%%%%%
%%  Theorem-Like Declarations
%%%%%%%%%%%%%%%%%%%%%%%%%%%%%%%%%%%%%%%%%%%%%%%%%%%%%%%%%%%%%%%%%%%%%%%%%%

\newtheorem{theorem}{Theorem}[section]
\newtheorem{lemma}[theorem]{Lemma}
\newtheorem{corollary}[theorem]{Corollary}
\newtheorem{proposition}[theorem]{Proposition}


\theoremstyle{definition}

\newtheorem{definition}[theorem]{Definition}
\newtheorem{conjecture}[theorem]{Conjecture}
\newtheorem{remark}[theorem]{Remark}
\newtheorem{example}[theorem]{Example}
\newtheorem{remarks}[theorem]{Remarks}
\newtheorem{notation}[theorem]{Notation}

\theoremstyle{remark}

%\newtheorem*{note}{Note}
\newtheorem*{warning}{Warning}
\newtheorem*{question}{Question}
\newtheorem*{fact}{Fact}
\newtheorem*{problem}{Problem}

\newenvironment*{subproof}[1][Proof]
{\begin{proof}[#1]}{\renewcommand{\qedsymbol}{$\diamondsuit$} \end{proof}}
 
 \newenvironment*{case}[1]
{\textbf{Case #1.  }\itshape }

\newenvironment*{claim}[1][Claim]
{\textbf{#1.  }\itshape }



\newcommand{\skipsmall}{\vspace{1em}}
\newcommand{\skipmed}{\vspace{2em}}
\newcommand{\skipbig}{\vspace{3em}}
\newcommand{\skipsmallminus}{\vspace{-1em}}

%\usepackage[T1]{fontenc}
%\usepackage[latin1]{inputenc}
%\usepackage{moreverb}
%\usepackage{graphicx}

\title{The Continuum Hypothesis}
\subtitle{Search/Pathways Friday Colloquium}

\author{Mitch Rudominer}
\institution{BEA}
\email{Mitch.Rudominer@bea.com}

% You can use this macro to put a caption at the bottom of each slide.
%\slideCaption{}
% \Logo This allows you to place a logo on each slide at a specified position.

% This defines the type of transition that should occur between slides.
% \DefaultTransition

\begin{document}

\maketitle

\begin{slide}{Abstract}
If $A$ and $B$ are sets then $|A|=|B|$ means that $A$ and $B$ have the same
size, i.e. the same number of elements. We say that $A$ and $B$ have the same \emph{cardinality}.
$|A|< |B|$ means that $A$ has fewer elements than $B$.

\skipmed
We will give a precise definition of cardinality in this talk.

\skipmed

We will see that there are infinite sets
$A$ and $B$ such that $|A| \neq |B|$. For example, we will see that $|\N| <
|\R|$.

\end{slide}
\begin{slide}{Abstract (2)}
The Continuum Hypothesis, abbreviated \CH, is the statement
that there is no cardinality between $|\N|$ and $|\R|$. That is, \CH says: There
is no set $A$ such that $|\N| < |A| < |\R|$.

\skipmed

Is \CH true or false? It turns out this is a difficult question to
answer because \CH is \emph{independent} of the axioms of mathematics.
That is, it can neither be proved nor disproved.

\skipmed

In this talk we will explain fairly precisely what is meant by the indepence of
\CH and we will give a rough sketch of the proof.

\end{slide}
\begin{slide}{Abstract (3)}

\skipmed

In the talk we we also cover the concept of cardinal and ordinal numbers.

\end{slide}

\begin{slide}{Cross Products}

\begin{definition}
$\angles{a,b}\defeq$ the \textbf{ordered pair} whose first element is $a$ and
whose second element is $b$.
\end{definition}

\skipmed

\begin{definition}
If $A$ and $B$ are sets then 
$ A \times B \defeq\setof{\angles{a,b}}{a\in A, b \in B}$
\end{definition}

\skipmed

\begin{example}
Let $A = \singleton{1,2,3}$ and $B = \singleton{x,y}$. Then
$A \times B = \singleton{ \angles{1,x},\angles{1,y},\angles{2,x},\angles{2,y}, \angles{3,x},\angles{3,y}}$
\end{example}

\end{slide}

\begin{slide}{Functions}

\begin{definition}
If $A$ and $B$ are sets then $f:A \map B$ means that $f$ is a \textbf{function} from A to
B. Intuitively, $f$ is a mapping that, given an element $a\in A$ assigns an
element $f(a) \in B$. Formally, $f\subset A\times B$ such that for all $a \in
A$ there is exactly one element $b \in B$ s.t. $\angles{a,b}\in f$. We write $b =
f(a)$.
\end{definition}

\skipmed

\begin{definition}
If $f:A\map B$ then $A$ is called the \textbf{domain} of $f$. We write $A=\dom (f)$. The
\textbf{range} of $f$ is a subset of $B$ given by $\ran(f)=\setof{f(a)}{a\in A}$.
\end{definition}

\skipmed

\end{slide}

\begin{slide}{Functions (2)}

\begin{definition}
If $f:A\map B$ then $f$ is called \textbf{one-to-one} or an \textbf{injection},
and we write
$f:A\injection B$, if $f(a_1)\neq f(a_2)$ whenever $a_1\neq a_2$. $f$ is called
\textbf{onto} $B$ or a \textbf{surjection}, and we write $f:A\surjection B$, if $B =
\ran(f)$. If $f$ is
both one-to-one and onto then it is called a \textbf{bijection} or a
\textbf{one-to-one-correspondence} and we write $f:A\bijection B$.
\end{definition}


\end{slide}

\begin{slide}{Equipotency}
Notice that if $A$ and $B$ are finite sets and $f:A \bijection B$ then $A$ and
$B$ must have the same number of elements. We use this idea to \textbf{define}
what it means for two infinite sets to have the same number of elements.

\skipmed

\begin{definition}
Let $A$ and $B$ by two (possibly infinite) sets. We write $A\cong B$ and say the
$A$ is \textbf{equipotent} with $B$ if there is some function $f:A \bijection B$.
\end{definition}
\end{slide}


\begin{slide}{Equipotency (2)}
We think of $A\cong B$ as meaning that $A$ and $B$ have the same size, the same
``number'' of elements.

\skipmed

\begin{remark}
We really want to write $|A| = |B|$ instead of $A\cong B$. But this is awkward
because we haven't defined $|A|$. If $A$ is \emph{finite} then $|A|$ means the
number of elements of $A$. But how can we make sense of this if $A$ is infinite?
We need a generalization of the notion of ``number'' to the infinite. We will
use ordinals and cardinals.
\end{remark}

\end{slide}

\begin{slide}{Using injections to compare cardinality}
\begin{definition}
We write $A\preceq B$ if there is an injection from $A$ to $B$.
$A\prec B$ means $A\preceq B$ and  not $A\not\cong B$
\end{definition}

\skipsmall

We think of $A\preceq B$ as meaning that the size of $A$ is less than or equal
to the size of $B$, and $A\prec B$ as meaning that the size of $A$ is strictly less than the
size of $B$. After we have defined cardinality, we will be able to
write $|A|\leq|B|$ and $|A|<|B|$ instead.

\skipsmall

\begin{theorem}{Cantor-Bernstein}
If $A\preceq B$ and $B\preceq A$ then $A\cong B$.
\end{theorem}

\skipsmall

The proof is surprisingly tricky and we omit it.

\skipsmall

\begin{exercise}
Show that if $A\subseteq B$ then $A\preceq B$.
\end{exercise}



\end{slide}

\begin{slide}{Some Properties of Equipotence}
\begin{exercise}
$\cong$ is an equivalence relation. This means it is
\begin{enumerate}
  \item reflexive: $A\cong A$.
  \item symmetric: $A\cong B\implies B\cong A$.
  \item transitive: $A\cong B \AND B\cong C \implies A\cong C$.
\end{enumerate}
\end{exercise}

\skipmed

\begin{exercise}
$\prec$, $\preceq$, and $\cong$ satisfy the following properties:
\begin{enumerate}
  \item transitive: $A\prec B \AND B\prec C \implies A\prec C$ and similarly with
  $\preceq$.
  \item $A\cong B \implies A\preceq B$.
\end{enumerate}
\end{exercise}
\end{slide}

\begin{slide}{The Axiom Of Choice}
In this talk we will not pay particular attention to the axioms we are using.
However there is one axiom that I want to call our attention to.

\skipmed

The Axiom of Choice, abbreviated \AC, is the following statement:

\skipsmall

Suppose that $A$ is an infinite set and $f$ is a function with $A=\dom(f)$.
Suppose that for all $a\in A$, $f(a)$ is a non-empty set. Then there is a
function $g$ with $\dom(g)=A$ such that for all $a \in A$,  $g(a)\in f(a)$.
\end{slide}

\begin{slide}{Using surjections to compare cardinality}
\begin{lemma}
Let $A$ and $B$ be non-empty sets. Then the following two things are equivalent:
\begin{enumerate}
\item There is a function $f:A \injection B$ (i.e. $A\preceq B$)
\item There is a function $g:B \surjection A$
\end{enumerate}
\end{lemma}

\skipsmall

\begin{proof}
(1) $\implies$ (2): Suppose $f:A \injection B$. Let $a_0$ be any element of $A$.
Define $g:B\surjection A$ by $g(b)\defeq f^{-1}(b)$ if $b\in\ran(f)$ or
$g(b)=a_0$ otherwise.

(2) $\implies$ (1): Suppose $g:B \surjection A$. By \AC, there
is a function $f$ such that for all $a\in A$, $f(a)\in g^{-1}[\singleton{a}]$.
Then $f:A \injection B$.
\end{proof}
\end{slide}

\begin{slide}{Powersets}
\begin{definition}
Let $A$ be a set. Then the \textbf{powerset} of $A$, written $\Powerset(A)$, is the set
of all subsets of $A$.
\end{definition}

\skipmed

\begin{example}


\begin{enumerate}
  \item Let $A=\singleton{a,b}$. Then 
  $\Powerset(A)=\singleton{\emptyset,\singleton{a},\singleton{b},\singleton{a,b}}$ 
  \item $\Powerset(\emptyset)=\singleton{\emptyset}$
  \item 
  $\Powerset(\singleton{\emptyset})=\singleton{\emptyset,\singleton{\emptyset}}$
  \item $\Powerset(\R) = $ the set consisting of every set of real numbers
\end{enumerate}


\end{example}


\end{slide}

\begin{slide}{The Cantor Diagonal Argument}
\begin{theorem}
For any set $A$, $A\prec \Powerset(A)$.
\end{theorem}

\skipsmall

\begin{proof}
To see that $A\preceq \Powerset(A)$, let $f:A\injection \Powerset(A)$ be defined
by $f(a)=\singleton{a}$ for each $a\in A$. To see that $\Powerset(A)\not\cong A$, suppose
that $g:A\bijection \Powerset(A)$ and we will derive a contradiction. Let
$X\defeq \setof{a\in A}{a \not\in g(a)}$. Notice that $X\subseteq A$, so
$X\in\Powerset(A)$, so $X\in\ran(g)$. So fix some element $a_0\in A$ such that
$g(a_0)=X$. Now we ask the question, is $a_0\in X$? If $a_0\in X=g(a_0)$ then
$a_0\not\in X$ by definition of $X$. Conversely, if $a_0\not\in X = g(a_0)$
then $a_0 \in X$ by definition of $X$. In either case we get a contradiction. So
it must not be true that $\Powerset(A)\not\cong A$. Thus $A\prec \Powerset(A)$.
\end{proof}


\end{slide}

\begin{slide}{Real and Natural Numbers}
Let $\N=\singleton{0,1,2,3,\dots}$  be the \textbf{natural numbers}  and let
$\R$ be the \textbf{real numbers}. Notice that $\N\subset\R$ and so $\N\preceq
\R$. In fact $\N\prec\R$.

\skipsmall

\begin{exercise}
Show $\N\prec\R$.
Hint:
\begin{enumerate}
  \item $\R\cong (0,1)$.
  \item $(0,1) \cong \setof{s}{s\text{ is an infinite sequence of bits}}$
  \item $\setof{s}{s\text{ is an infinite sequence of bits}} \cong \Powerset(\N)$
\end{enumerate}
So $\N\prec\Powerset(\N)\cong\R$
\end{exercise}

\end{slide}

\begin{slide}{Infinite and Countable}

\begin{definition}
A set $A$ is \textbf{finite} if there is some natural number $n$ such that
$\singleton{0,1,2,\dots,n-1} \cong A$. In this case we write $|A|=n$. If $A$ is
not finite then it is \textbf{infinite}.
\end{definition}

\skipsmall

\begin{definition}
$A$ is  \textbf{countably infinite} if $A\cong\N$. $A$ is \textbf{countable}
if it is either finite or countably infinite. Otherwise $A$ is \textbf{uncountable}.
\end{definition}

\skipmed

So the exercise on the previous slide showed that the set of real numbers is uncountable.

\skipsmall

\begin{exercise}
Show that the set of rational numbers, $\Q$, is countable.
\end{exercise}

\skipsmall

So $\N\subset\Q\subset\R$ and $\N\cong\Q\prec\R$.

\end{slide}

\begin{slide}{The Continuum Hypothesis: Take 1}
One way to state the \CH is the following:
\skipmed
\begin{ch}
If $A$ is an uncountable set of real numbers, then $A\cong\R$.
\end{ch}

\skipsmall

For a different way to state \CH, we need to talk about cardinality and define
$|A|$ for infinite sets $A$.

\end{slide}

\begin{slide}{Cardinality: Informal Approach}

We want to define $|A|$ for an arbitry set $A$. $|A|$ should be some object that
represents the size of $A$. For finite sets we use natural numbers as
cardinalities. What should we use for infinite sets?

\skipsmall

Based on our experience with finite sets, the main property we want from
cardinality is the following:
\begin{equation}
|A|=|B| \Ifff A\cong B
\end{equation}

\skipsmall

Given any set $A$, let $[[A]]=\setof{B}{B\text{ is a set and } B\cong A}$. So
$[[A]]$ is the \textbf{equivalence class} of all sets that are equipotent with
$A$. 

\end{slide}

\begin{slide}{Cardinality: Informal Approach, 2}

We have to elect some object to be $|A|$.  Let's call this object $\kappa$. We
know that we also want $|B|=\kappa$ for each $B\in[[A]]$.

\skipsmall

A first idea is to define $|A| = [[A]]$. Unfortunately there are some technical
problems with this becuase $[[A]]$  is not itself a set. (It is a proper class
because it has too many elments.)

\skipsmall
A solution to this problem is to instead define $|A|$ to be some
\textbf{canonical} element of $[[A]]$. That is $|A|$ will be some distinguished
member of the equivalence class $[[A]]$.

\skipsmall

So what is a good way of picking this distinguished element? We will use an \textbf{ordinal}.

\end{slide}

\begin{slide}{Ordinals, an informal approach}
Informally, the \textbf{ordinals} are what you get if you keep counting past
infinity. 

\skipsmall

$0,1,2,3,\dots\omega,\omega+1,\omega+2,\omega+3,\dots 2
\omega,2\omega+1,2\omega+2, 2\omega+3,\dots$


\skipsmall
There are two different kinds of ordinals: \textbf{successor ordinals} and
\textbf{limit ordinals}. (And zero, which is neither a limit nor a successor.)

\skipsmall

The successor ordinals are just like the integers: each one is just one more
than the previous one. The limit ordinals are a new kind of thing. They are
points in the ordering that sit above all of the points that are less than them,
but they have no immediate predecessor. In the picture above, the limit ordinals
are $\omega$ and $2\omega$.
\end{slide}

\begin{slide}{Linear Orders}
\begin{definition}
A \textbf{linearly ordered set} is a pair $\angles{X,<}$, where $X$ is a set and
$<$ is a binary relation on $X$ (i.e a subset of $X\times X$) such that $<$ is:
\begin{enumerate}
  \item transitive: $x<y<z\Implies x<z$
  \item asymmetric: $x\not<x$
  \item connected: For all $x\not= y$ in $X$, $x<y$ or $y<x$.
\end{enumerate}
If $x<y$ and there is no $z$ such that $x<z<y$ then $x$ is the
\textbf{predecessor} of $y$	 and $y$ is the \textbf{successor} of $x$. If every
element except possibly the greatest one has a successor and every element
except possibly the least one has a predecessor then $\angles{X,<}$ is said to be a
\textbf{discrete} linear order. At the other end of the spectrum, if for
all $x$ and $y$ there is a $z$
such that $x<y<z$ then $\angles{X,<}$ is said to be a \textbf{dense} linear order.
\end{definition}
\end{slide}

\begin{slide}{Linear Orders (2)}
\begin{example}
Let $\Z=\singleton{\dots,-3,-2,-1,0,1,2,3,\dots}$ be the integers. Then
$\angles{\Z,<}$ and $\angles{\N,<}$ are discrete linear orders.
\end{example}

\skipsmall

\begin{example}
$\angles{\R,<}$ and $\angles{\Q,<}$ are dense linear orders.
\end{example}

\skipsmall

Notice that if $X$ is an infinite set and $<$ is a linear order on $X$ then the
notions of the cardinality of $X$ and the type of ordering of $<$ are distinct.
For example $\N\cong \Q$ even though $\angles{\N,<}$ is discrete and
$\angles{\Q,<}$ is dense. And $\Q\prec\R$ even though $\angles{\R,<}$ is also
dense. 

\end{slide}

\begin{slide}{Well-orderings}
\begin{definition}
A \textbf{well-ordering} is a linear ordering in which there are no infinite
descending chains of elements.
\end{definition}

\skipmed

\begin{example}
$\angles{\N,<}$ is a well-ordering, but $\angles{\Z,<}$ and $\angles{\Q,<}$ are not.
\end{example}

\begin{example}
$\N\union\singleton{\omega}=\singleton{0<1<2<3<\dots<\omega}$ is a well-ordering.
\end{example}

\end{slide}

\begin{slide}{Alternate Def of Well-orderings}

\begin{lemma}
Suppose $\angles{X,<}$ is a linear ordering. Then it is a well-ordering if and
only if it satisfies the following property: Every subset of $X$ contains a
\emph{least} element.
\end{lemma}
\begin{proof}
If there is an infinite descending chain of elements, then that chain consists
of a set of elements that does not have a least element. Conversely, suppose
there is some $A\subseteq X$ such that $A$ does not have a least element. Let
$a_0\in A$. This is not the least element of $A$ so there is some $a_1<a_0$ in
$A$. This is also not the least element of $A$ so there is some $a_2<a_1$ in
$A$. Continuing in this way we get an infinite descending chain.
\end{proof}

\skipsmall

Note that this proof uses the Axiom of Choice.

\end{slide}

\begin{slide}{Some Properties of Well-orderings}
\begin{lemma}
Let $\angles{X,<}$ be a well-ordering. Then
\begin{enumerate}
  \item There is a least element of $X$.
  \item There may or may not be a greatest element of $X$.
  \item If $x$ is any element of $X$ other than the greatest, then $x$ has a
  unique successor, which we will write as $\suc(x)$.
  \item If $x$ is any element other than the least, then $x$ may or may not have
  a predecessor. If it has a predecessor, $x$ is called a successor element. If
  it does not have a predecessor, $x$  is called a \textbf{limit} element.
\end{enumerate}
\end{lemma}
\begin{proof}
1. Every subset of $X$ has a least element, so $X$ itself does.
2. $\N\union\singleton{\omega}$ has a greatest element, but $\N$ doesn't.
3. Let $A=\setof{y\in X}{x<y}$. $A$ has a least element, which is $\suc(x)$.
4. Every element of $\N$ other than 0 is a successor element, but $\omega$ does
not have a predecessor, so it is a limit element.
\end{proof}
\end{slide}

\begin{slide}{Principle of Induction}
\begin{lemma}
Suppose $\angles{X,<}$ is a well-ordering. Suppose $P$ is some property that you
wish to show is true of every element of $X$. Let $x_0$ be the least element of
$\angles{X,<}$. If
\begin{enumerate}
  \item $P(x_0)$ holds; and
  \item $(\forall y<x)P(y)\Implies P(x)$.
\end{enumerate}
Then $P(x)$ holds for every $x\in X$.
\end{lemma}
\begin{proof}
Let $A=\setof{x\in X}{\neg P(x)}$. We want to show that $A=\emptyset$. Suppose
not and let $x_1$ be the least element of $A$. Then for all $y<x_1$, $P(y)$
holds. Then by our assumption $P(x_1)$ holds. Contradiction.
\end{proof}
\end{slide}

\begin{slide}{Comparing Well-Orderings}

\begin{definition}
Suppose  $\angles{X,<_X}$ and $\angles{Y,<_Y}$ are well-orderings. An
\textbf{isomorphism} from $\angles{X,<_X}$ to $\angles{Y,<_Y}$ is
an order-preserving bijection: $f:X\bijection Y$ such that $x_1<x_2 \Iff
f(x_1)<f(x_2)$. We say that $\angles{X,<_X}$ and $\angles{Y,<_Y}$ are \textbf{isomorphic}.
\end{definition}

\skipsmall

\begin{lemma}
If $\angles{X,<_X}$ and $\angles{Y,<_Y}$ are isomorphic then there is a
\textbf{unique} isomorphism from $\angles{X,<_X}$ to $\angles{Y,<_Y}$.
\end{lemma}
\begin{proof}
Suppose $f$ and $g$	are two different isomorphisms. Let $x$ be least such that
$f(x)\not=g(x)$. Let $B=\setof{f(z)}{z<x}=\setof{g(z)}{z<x}$. Let $y$ be the
least element of $Y$ that is not in $B$. Then we must have $f(x)=y$ and
$g(x)=y$. Contradiction.
\end{proof}

\end{slide}

\begin{slide}{Initial Segments}
\begin{definition}
An \textbf{initial segment} of a well-ordering $\angles{X,<_X}$ is a subset
$Y\subseteq X$ such that if $y\in Y$ and $x\in X$ and $x<y$ then $x\in Y$.
A \textbf{proper} initial segment is an initial segment that is not all of $X$.
\end{definition}

\skipsmall

\begin{remark}
Every proper initial segment is of the form 
$I_{<x}=\setof{y\in X}{y<x}$ and possibly of the form $I_x=\setof{y\in X}{y<=x}$.
\end{remark}

\skipsmall

\begin{definition}
The \textbf{order type} of a wellordering $\angles{X,<}$ is the class of all
well-orderings that are isomorphic to $\angles{X,<}$.
\end{definition}
\end{slide}

\begin{slide}{The Order Types are Well-Ordered}
\begin{theorem}
Let $\angles{X,<_X}$ and $\angles{Y,<_Y}$ be two well-orderings. Then either they
are isomorphic, or one of them is isomorphic to a proper initial segment of the other.
\end{theorem}
\begin{proof}[proof sketch]
Suppose that there is no isomorphism from $\angles{Y,<_Y}$ to a proper initial
segment of $\angles{X,<_X}$. We will show that there is an isomorphism from 
$\angles{X,<_X}$ to an initial segment of $\angles{Y,<_Y}$. Let
$A=\setof{x\in X}{I_x \text{ is isomorphic to an initial segment of } Y}$. We
claim that $A=X$.
If not, let $x_0$ be least not in $A$. For every $x<x_0$ there is a unique
isomorphism from $I_x$ onto an initial segment of $Y$. The union of all of these
functions is a an isomorphism $f:I_{<x_0}\map {I_{<y_0}}$ for some $y_0\in Y$.
Extend $f$ one more point by setting $f(x_0)=y_0$. This shows that $A=X$.
Now take the union of all of the isomorphisms for all of the initial segments of
$\angles{X,<_X}$. This gives an isomorphism from all of $\angles{X,<_X}$ into an
initial segment of $\angles{Y,<_Y}$.
\end{proof}
\end{slide}

\begin{slide}{How to define the ordinals?}

Our immediate goal is to give a precise definition of the ordinals.

\skipsmall

The order-types of the well-orderings are arranged just like our intuitive
picture of the ordinals.

\skipsmall

\textbf{Idea:} For each well-ordering type, pick some canonical representative
to be an ordinal.

\skipsmall

But do it in a such a way that there is also some global $<$ relation on the ordinals.


\end{slide}

\begin{slide}{Von Neumann Ordinals}

\textbf{Idea:} An ordinal is equal to the set of smaller ordinals.


\begin{itemize}
  \item $0 = \emptyset$
  \item $1 = \singleton{0}$
  \item $2 = \singleton{0,1}$
  \item $3= \singleton{0,1,2}$
  \item $\omega = \singleton{0,1,2,3,4,5,\dots}$
  \item $\omega+1 = \singleton{0,1,2,3,4,5,\dots,\omega}=\omega\union\singleton{\omega}$
\end{itemize}

Notice that $\in$ is playing the role of $<$. The ordinals are well-ordered by
\textbf{membership}.  Also notice that each ordinal is closed under $\in$.



\end{slide}

\begin{slide}{The Ordinals}

\begin{definition}
A set $X$ is \textbf{transitive} if every member of $X$ is also a subset of $X$.
That is, if $y\in x\in X$ then $y\in X$.
\end{definition}

\skipsmall

\begin{definition}
An \textbf{ordinal} is a set that is transitive and well-ordered by $\in$.
\end{definition}

\skipsmall

\begin{definition}
If $\alpha$ and $\beta$ are ordinals we write $\alpha<\beta$ for
$\alpha\in\beta$. Also $\alpha+1\defeq\alpha\union\singleton{\alpha}$. Notice
that $\alpha+1$ is the least ordinal greater than $\alpha$.
\end{definition}

\skipsmall

\begin{definition}
An ordinal $\beta$ is called a \textbf{successor} ordinal if $\beta=\alpha+1$
for some $\alpha$. If $\beta\neq 0$ is not a successor ordinal then it is
called a \textbf{limit} ordinal.
\end{definition}

\skipsmall

\begin{definition}
$\omega$ is the least limit ordinal.
\end{definition}


\end{slide}

\begin{slide}{Properties of the Ordinals}
\begin{remark}
Every element of an ordinal is also an ordinal. Every set of ordinals is
well-ordered by
$\in$. So an ordinal is the same thing as a transitive set of ordinals.
\end{remark}

\skipsmall

\begin{lemma}
There is no largest ordinal. Every set of ordinals $X$ has a least upper bound,
which we write as $\lambda=\sup(X)$.
\end{lemma}
\begin{proof}
There is no largest ordinal because $\alpha<\alpha+1$. Let $X$
be any set of ordinals. If $X$ has a largest element we are done, so suppose it doesn't.
Let $\lambda=\bigcup X = \setof{\alpha}{(\exists\beta\in X)\alpha\in\beta}$.
$\lambda$ is a set of ordinals and $\lambda$ is transitive, so $\lambda$ is an
ordinal. Every member of $X$ is a member of $\lambda$, so $\lambda$ is an
ordinal greater than every member of $X$. Every member of $\lambda$ is a member
of some member of $X$, so $\lambda$ is the least upper bound of $X$.
\end{proof}


\end{slide}

\begin{slide}{The Ordinals are Universal}
\begin{theorem}
Let $\angles{X,<_X}$	be any well-ordering. Then there is a unique ordinal
$\alpha$ such that $\angles{\alpha,\in}$  is isomorphic to $\angles{X,<_X}$.
We will say for short that $\alpha$ is isomorphic to $\angles{X,<_X}$.
\end{theorem}
\begin{proof}
Let $A$ be the set of ordinals that are isomorphic to some initial segment of
$\angles{X,<_X}$, and for each $\gamma\in A$ let $f_{\gamma}$ be the unique
isomorphism from $\angles{\gamma,\in}$ to an initial segment of
$\angles{X,<_X}$. Let $\alpha=\sup(A)$. Let $f$ be the union of all of the
$f_{\gamma}$. Then $f$ is an isomorphism from $\alpha$ to an initial segment of
$\angles{X,<_X}$. (Which means that $\alpha$ must have been the greatest
element of $A$.) $f$ must be onto all of $X$ becuase otherwise $\alpha+1$ would
be in $A$.
\end{proof}
\end{slide}

\begin{slide}{Every Set is Well-orderable}
\begin{theorem}
Let $X$ be any set. Then $X$ is bijectable with an ordinal. Equivalently, there
is some relation $<_X$ on $X$ such that $\angles{X,<_X}$ is a well-ordering.
\end{theorem}


\begin{remark}
The proof of this theorem requires the Axiom of Choice. In fact, the statement
of the theorem is equivalent to \AC.
\end{remark}


\begin{proof}[sketch of proof]
By \AC there is a function $F$ such that for every non-empty subset $a\subseteq
X$, $F(a)\in a$.
Now, by induction on ordinals $\alpha$, we define a sequence of injections
$f_{\alpha}:\alpha\injection X$, so that $\alpha<\beta\Implies f_{\alpha}\subset
f_{\beta}$. At successor ordinals $\alpha+1$, if $f_{\alpha}$ is not already
onto $X$, we let
$f_{\alpha+1}(\alpha)=F(X-\setof{f(\beta)}{\beta<\alpha})$. At limit ordinals $\lambda$,
we set $f_{\lambda}=\Union{\alpha<\lambda}f_{\alpha}$. Eventually we will use up
all of the elements of $X$, so must reach some $\alpha$ s.t.
$f_{\alpha}$ is onto $X$. 
\end{proof}
\end{slide}

\begin{slide}{The Cardinals}
\begin{definition}
A \textbf{cardinal} is an ordinal $\kappa$ with the property that for all
ordinals $\alpha<\kappa$, $\alpha\not\cong\kappa$. That is, there is no
surjection $f:\alpha\surjection\kappa$.
\end{definition}

\skipmed

\begin{remark}
Each of the finite ordinals are cardinals. $\omega$ is a cardinal---it
is the least infinite cardinal. $\omega+1$ is not a cardinal. $\omega+\omega$ is
not a cardinal.
\end{remark}


\end{slide}

\begin{slide}{Cardinality}
\begin{definition}
For any set $X$, the \textbf{cardinality} of $X$, written $|X|$, is the least
ordinal $\kappa$ such that $\kappa\cong X$. Notice that $\kappa$ must in fact be
a cardinal.
\end{definition}

\skipsmall

\begin{remark}
Recall that $[[X]]$ is the equivalence class of all sets that are
equipotent with $X$. We said we were going to define $|X|$ to be
some canonical element of $[[X]]$. We
have finally done this. The canonical element we have chosen is the
least ordinal in $[[X]]$.
\end{remark}

\skipsmall
Notice that if $\kappa$ is a cardinal then $|\kappa|=\kappa$.

\skipsmall

\begin{definition}
If $\kappa$ is a cardinal then we define $2^{\kappa}$ to be the cardinality of $\Powerset(\kappa)$.
\end{definition}

\skipsmall

\begin{remark} 
$2^{\kappa}>\kappa$.
\end{remark}

\end{slide}

\begin{slide}{Properties of Cardinals}
\begin{lemma}
There is no largest cardinal, and if $X$ is a set of cardinals, then $\sup(X)$
is also a cardinal.
\end{lemma}
\begin{proof}
There is no largest cardinal because $k<2^{\kappa}$. Suppose $X$ is a set of
cardinals and let $\lambda=\sup(X)$. If $X$ has a greatest element, then it is
$\lambda$, and so $\lambda$ is a cardinal and we are done. So suppose $X$ does
not have a greatest element. We want to show that $\lambda$ is a cardinal.
Suppose towards a contradiction that it is not. Let $\alpha<\lambda$ and let
$f:\alpha\surjection\lambda$. Since $\lambda=\sup(X)$ and $X$ doesn't have a
greatest element, there is a $\mu\in X$ such that $\alpha<\mu$. But
$\mu\subset\lambda$ and $f:\alpha\surjection\lambda$, so $|\mu|\leq\alpha$,
which contradicts the fact that $\mu$ is a cardinal.
\end{proof}

\end{slide}

\begin{slide}{The Alephs}

\begin{definition}
If $\kappa$ is a cardinal then $\kappa^{+}$	is the next largest cardinal. If
$\mu=\kappa^{+}$ then $\mu$ is called a \textbf{successor cardinal}. If $\mu$ is
not zero and not a successor cardinal, then it is called a \textbf{limit}
cardinal.
\end{definition}

\skipsmall

Each of the positive finite cardinals are successor cardinals. $\omega$ is the
least limit cardinal, as well as the least limit ordinal.

\skipsmall

\begin{definition}
If $\alpha$ is an ordinal then $\aleph_{\alpha}$ is the $\alpha^{\text{th}}$
infinite cardinal. So $\aleph_0 = \omega$ is the least infinite cardinal.
$\aleph_1 = \aleph_0^{+}$ is the least uncountable cardinal. $\aleph_{\omega}$
is the second limit cardinal.
\end{definition}

\end{slide}

\begin{slide}{The Generalized Continuum Hypothesis}

We can finally give the official version of the Continuum Hypothesis. \CH is the
following statment: 

\skipsmall

$2^{\aleph_0} = \aleph_1$


\skipsmall

The Generalized Continuum Hypothesis, or \textbf{GCH} is the following statment:

\skipsmall

For all infinite cardinals $\kappa$, $2^{\kappa}=\kappa^+$

\end{slide}

\begin{slide}{Meta-Mathematics}
Our next goal is to develop enough theory to be able to precisely state the fact
that \CH is independent. In order to do that, we need to turn to a new topic:
meta-mathematics. 

\skipsmall

In ordinary mathematics, like we have been doing on earlier slides, theorems and
proofs are the language we use to speak to each other, and the objects of our
study are things like sets and functions. 

\skipsmall

In meta-mathematics, theorems and
proofs become the objects of our study. Of course we will still need to use
theorems and proofs in order to talk to each other about these new objects of
our study.

\end{slide}

\begin{slide}{Language and Meta-Language}

In order to be able to use theorems and proofs to study theorems and
proofs, we need to seperate our language into a \textbf{formal} language, which
will be the object of our study, and an informal \textbf{meta-language} which we will use to
talk about the formal language.

\skipmed

Our meta-language will be the language we have been using all along in this
talk:  English with some mathematical symbols.

\end{slide}

\begin{slide}{The Language of Set Theory}
Our formal language is called \textbf{LST}, for ``the language of set theory.''

\skipsmall

\textbf{LST} uses the following symbols: $\in, =,\forall, \exists, \AND,\OR,\neg, \implies,\iff
(, )$, and variables $v_0,v_1,v_2,\dots$.

\skipsmall

A \textbf{well-formed formula} or just formula for short, in LST is a
sequence of
the above symbols that obey certain syntactic rules. Here are some examples:

\begin{enumerate}
  \item $\forall v_0(\neg(v_0\in v_1))$ is a formula that means that
  $v_1=\emptyset$. 
  \item $\forall v_0 \exists v_1 \forall v_2 (v_2\in v_1 \iff (\forall
  v_3(v_3\in v_2\implies v_3\in v_0)))$ is a formula that means the powerset of
  every set exists.
  \item $))\forall\implies v_0\neg()$ is not a formula.
\end{enumerate}

\end{slide}

\begin{slide}{Sentences, Theories and Deductions}
A \textbf{sentence} of LST is a formula with no free variables. On the previous
slide, the second formula is a sentence, but the first formula is not because it
contains the free variable $v_1$.

\skipsmall

A \textbf{theory} is a set of sentences.

\skipsmall

If $\sigma$ is a sentence and and $\Sigma$ is a theory then we say that $\sigma$
is provable from $\Sigma$, written $\Sigma\vdash\sigma$, just in case there is a
sequence of formulas $\angles{\varphi_0,\varphi_1,\dots,\varphi_n}$ such that
$\varphi_n=\sigma$ and each $\varphi_i$ is either
\begin{enumerate}
  \item in $\Sigma$ or
  \item is a \textbf{logical axiom} or
  \item can be obtained from earlier $\varphi_j$ by use of one of the \textbf{rules
  of inference}
\end{enumerate}

\end{slide}

\begin{slide}{ZFC}
The standard set of axioms used in set theory is known as \textbf{ZFC}. This
stands for Zermelo-Frankel set theory with the Axiom of Choice.

\skipsmall

ZFC is a formal theory in the sense described on the previous slide: it is a set
of sentences in LST.

\skipsmall

For example two of the axioms that are included in ZFC are:

\begin{enumerate}
  \item The emptyset axiom: $\exists v_1\forall v_0(\neg(v_0\in v_1))$
  \item The powerset axiom: $\forall v_0 \exists v_1 \forall v_2 (v_2\in v_1 \iff (\forall
  v_3(v_3\in v_2\implies v_3\in v_0)))$
\end{enumerate}

Earlier in this talk we discussed the Axiom of Choice in the meta-language. This
axiom, translated into a sentence of LST, is the ``C'' in ZFC.
\end{slide}

\begin{slide}{Formalizing the Meta-Theory}
It is generally agreed that all of the theorems in standard mathematics can be
formalized into LST and derived from ZFC. In particular,
all of the definitions we gave and theorems we proved earlier in the talk in our
informal meta-language can be formalized and translated into formal sentences
in LST and formal deductions from ZFC.

\skipsmall

For example, there is a sentence $\sigma$ in LST that is the formal version of
the English sentence ``Evey well-ordering is isomorphic to an ordinal.'' The
proof we gave earlier can be formalized into a deduction from ZFC, and so we
know that $\ZFC\vdash\sigma$.

\skipsmall

Also, there is a sentence in LST that is the formal version of \CH. Let
us refer to this sentence as $\CHformal$. 


\end{slide}

\begin{slide}{Consistency}
\begin{definition}
A theory $T$ is said to be \textbf{inconsistent} if
$T\vdash\sigma\AND\neg\sigma$ for some sentence $\sigma$. Otherwise $T$ is \textbf{consistent}.
\end{definition}

\skipsmall

\begin{remark}
The following are equivalent:
\begin{enumerate}
  \item $T$ is inconsistent.
  \item $T\vdash\sigma$ for all sentences $\sigma$.
  \item $T\vdash\sigma_0$ where $\sigma_0$ is the
following sentence: $\exists x(x\neq x)$.
\end{enumerate}

\end{remark}

\skipsmall

\begin{remark}
Let $T$ be a theory and $\sigma$ a sentence. Then the following are equivalent:
\begin{enumerate}
  \item $T\vdash\sigma$
  \item $T\union\singleton{\neg\sigma}$ is inconsistent.
\end{enumerate}
\end{remark}
\end{slide}

\begin{slide}{G\"odel's Second Incompleteness}
Is ZFC consistent? According to G\"odel's Second Incompleteness Theorem, we may never
know. 

\skipsmall

\begin{theorem}
If ZFC is consistent, then $\ZFC\not\vdash\sigma_{\text{CON}}$, where $\sigma_{\text{CON}}$ is the
sentence in LST that asserts that ZFC is consistent.
\end{theorem}

\skipsmall

It follows that if ZFC is consistent, there is probably no way for us to ever
know for certain that it is consistent. At least we probably will never be able
to give a proof that it is consistent, because if we could give such a proof we
would probably be able to formalize this proof into a deduction from ZFC,
and so we would get $\ZFC\vdash\sigma_{\text{CON}}$, contradicting
G\"odel's Second Incompleteness Theorem.


\end{slide}

\begin{slide}{Independence of \CH}

Now we can state the precise version of our earlier claim that $\CH$ is
independent of the axioms of mathematics. That precise version is the following:

\skipsmall

 $\ZFC\not\vdash\CHformal$ and $\ZFC\not\vdash\neg\CHformal$.

\skipsmall
Notice that the above statements
are equivalent to the following statements: 

\skipsmall

$ZFC+\neg\CHformal$ is consistent
and $\ZFC+\CHformal$ is consistent.

\skipsmall

Unfortunately we can't prove this. By G\"odel's Second Incompleteness Theorem,
if $\ZFC$ is
consistent, then it is not possible for us to prove either of the above two claims.

\end{slide}

\begin{slide}{Relative Consistency}

Since G\"odel's Second Incompleteness Theorem rules out the possibility that we
will be able to prove outright that \CH is independent of ZFC, the best we
can do is to prove that it is independent \textbf{if} ZFC is consistent.

\skipsmall

The following two theorems have been proved. They are known as \textbf{relative
consistency} results.

\skipsmall

\begin{theorem}[G\"odel, 1936]
c.
\end{theorem}

\skipsmall

\begin{theorem}[Paul Cohen, 1963]
If ZFC is consistent, then so is ZFC+$\neg\CHformal$.
\end{theorem}

\end{slide}

\begin{slide}{Models of Set Theory}

The technique for proving a relative consistency result of the form: If ZFC is
consistent then so is ZFC+$\sigma$, is to use \textbf{models}.

\skipsmall

According to G\"odel's Completeness Theorem, a theory $T$ is consistent if and
only if there exists a model $M$ of $T$. We write this as $M\models T$. This
means
roughly that $M$ is a set and if we think of $(M,\in)$ as being the universe of
all sets, then in this universe all of the sentences of $T$ are true.

\skipsmall

To prove the relative consistency result we do the following: Start with some
model $M\models\ZFC$. From $M$ produce another model $N$ such that $N\models\ZFC+\sigma$.

\end{slide}

\begin{slide}{The Constructible Universe}
In 1936 Kurt G\"odel proved that \textbf{GCH} is consistent relative to ZFC. In order to
do this he invented the Constructible Universe, $L$.

\skipsmall

\begin{enumerate}
  \item $L\models\ZFC+\text{GCH}$
  \item $L$ is an inner model
  \item $L$ is the minimal inner model
\end{enumerate}

\end{slide}

\begin{slide}{Forcing}
In 1963 Paul Cohen proved that $\neg\CH$ is consistent relative to ZFC. In order
to do this he invented a technique called \textbf{forcing.}

\skipsmall

Given a model $M\models\ZFC$ forcing produces another model $M[G]$ such that
\begin{enumerate}
  \item $M[G]\models\ZFC+\neg\CHformal$
  \item $M\subset M[G]$
\end{enumerate}

\end{slide}


\end{document}
