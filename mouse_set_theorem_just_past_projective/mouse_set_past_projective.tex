\documentclass[oneside,12pt]{amsart}

\usepackage{amsmath,amssymb,latexsym,eucal,amsthm,rotating}
%\usepackage[shortlabels]{enumitem}

%%%%%%%%%%%%%%%%%%%%%%%%%%%%%%%%%%%%%%%%%%%%%
% Common Set Theory Constructs
%%%%%%%%%%%%%%%%%%%%%%%%%%%%%%%%%%%%%%%%%%%%%

\newcommand{\setof}[2]{\left\{ \, #1 \, \left| \, #2 \, \right.\right\}}
\newcommand{\lsetof}[2]{\left\{\left. \, #1 \, \right| \, #2 \,  \right\}}
\newcommand{\bigsetof}[2]{\bigl\{ \, #1 \, \bigm | \, #2 \,\bigr\}}
\newcommand{\Bigsetof}[2]{\Bigl\{ \, #1 \, \Bigm | \, #2 \,\Bigr\}}
\newcommand{\biggsetof}[2]{\biggl\{ \, #1 \, \biggm | \, #2 \,\biggr\}}
\newcommand{\Biggsetof}[2]{\Biggl\{ \, #1 \, \Biggm | \, #2 \,\Biggr\}}
\newcommand{\dotsetof}[2]{\left\{ \, #1 \, : \, #2 \, \right\}}
\newcommand{\bigdotsetof}[2]{\bigl\{ \, #1 \, : \, #2 \,\bigr\}}
\newcommand{\Bigdotsetof}[2]{\Bigl\{ \, #1 \, \Bigm : \, #2 \,\Bigr\}}
\newcommand{\biggdotsetof}[2]{\biggl\{ \, #1 \, \biggm : \, #2 \,\biggr\}}
\newcommand{\Biggdotsetof}[2]{\Biggl\{ \, #1 \, \Biggm : \, #2 \,\Biggr\}}
\newcommand{\sequence}[2]{\left\langle \, #1 \,\left| \, #2 \, \right. \right\rangle}
\newcommand{\lsequence}[2]{\left\langle\left. \, #1 \, \right| \,#2 \,  \right\rangle}
\newcommand{\bigsequence}[2]{\bigl\langle \,#1 \, \bigm | \, #2 \, \bigr\rangle}
\newcommand{\Bigsequence}[2]{\Bigl\langle \,#1 \, \Bigm | \, #2 \, \Bigr\rangle}
\newcommand{\biggsequence}[2]{\biggl\langle \,#1 \, \biggm | \, #2 \, \biggr\rangle}
\newcommand{\Biggsequence}[2]{\Biggl\langle \,#1 \, \Biggm | \, #2 \, \Biggr\rangle}
\newcommand{\singleton}[1]{\left\{#1\right\}}
\newcommand{\angles}[1]{\left\langle #1 \right\rangle}
\newcommand{\bigangles}[1]{\bigl\langle #1 \bigr\rangle}
\newcommand{\Bigangles}[1]{\Bigl\langle #1 \Bigr\rangle}
\newcommand{\biggangles}[1]{\biggl\langle #1 \biggr\rangle}
\newcommand{\Biggangles}[1]{\Biggl\langle #1 \Biggr\rangle}


\newcommand{\force}[1]{\Vert\!\underset{\!\!\!\!\!#1}{\!\!\!\relbar\!\!\!%
\relbar\!\!\relbar\!\!\relbar\!\!\!\relbar\!\!\relbar\!\!\relbar\!\!\!%
\relbar\!\!\relbar\!\!\relbar}}
\newcommand{\nforce}[1]{\Vert\!\underset{\!\!\!\!\!#1}{\!\!\!\relbar\!\!\!%
\relbar\!\!\relbar\!\!\relbar\!\!\!\relbar\!\!\relbar\!\!\relbar\!\!\!%
\relbar\!\!\not\relbar\!\!\relbar}}
\newcommand{\forcein}[2]{\overset{#2}{\Vert\underset{\!\!\!\!\!#1}%
{\!\!\!\relbar\!\!\!\relbar\!\!\relbar\!\!\relbar\!\!\!\relbar\!\!\relbar\!%
\!\relbar\!\!\!\relbar\!\!\relbar\!\!\relbar\!\!\relbar\!\!\!\relbar\!\!%
\relbar\!\!\relbar}}}

\newcommand{\pre}[2]{{}^{#2}\!{#1}}

\newcommand{\restr}{\!\!\upharpoonright\!}

%%%%%%%%%%%%%%%%%%%%%%%%%%%%%%%%%%%%%%%%%%%%%
% Set-Theoretic Connectives
%%%%%%%%%%%%%%%%%%%%%%%%%%%%%%%%%%%%%%%%%%%%%

\newcommand{\intersect}{\cap}
\newcommand{\union}{\cup}
\newcommand{\Intersection}[1]{\bigcap\limits_{#1}}
\newcommand{\Union}[1]{\bigcup\limits_{#1}}
\newcommand{\adjoin}{{}^\frown}
\newcommand{\forces}{\Vdash}

%%%%%%%%%%%%%%%%%%%%%%%%%%%%%%%%%%%%%%%%%%%%%
% Miscellaneous
%%%%%%%%%%%%%%%%%%%%%%%%%%%%%%%%%%%%%%%%%%%%%
\newcommand{\defeq}{=_{\text{def}}}
\newcommand{\Turingleq}{\leq_{\text{T}}}
\newcommand{\Turingless}{<_{\text{T}}}
\newcommand{\lexleq}{\leq_{\text{lex}}}
\newcommand{\lexless}{<_{\text{lex}}}
\newcommand{\Turingequiv}{\equiv_{\text{T}}}

%%%%%%%%%%%%%%%%%%%%%%%%%%%%%%%%%%%%%%%%%%%%%
% Constants
%%%%%%%%%%%%%%%%%%%%%%%%%%%%%%%%%%%%%%%%%%%%%
\newcommand{\R}{\mathbb{R}}
\renewcommand{\P}{\mathbb{P}}
\newcommand{\Q}{\mathbb{Q}}
\newcommand{\Z}{\mathbb{Z}}
\newcommand{\C}{\mathbb{C}}
\newcommand{\N}{\mathbb{N}}
\newcommand{\B}{\mathbb{B}}
\newcommand{\LofR}{L(\R)}
\newcommand{\JofR}[1]{J_{#1}(\R)}
\newcommand{\SofR}[1]{S_{#1}(\R)}
\newcommand{\JalphaR}{\JofR{\alpha}}
\newcommand{\JbetaR}{\JofR{\beta}}
\newcommand{\JlambdaR}{\JofR{\lambda}}
\newcommand{\SalphaR}{\SofR{\alpha}}
\newcommand{\SbetaR}{\SofR{\beta}}
\newcommand{\Pkl}{\mathcal{P}_{\kappa}(\lambda)}
\DeclareMathOperator{\con}{con}
\DeclareMathOperator{\ORD}{OR}
\DeclareMathOperator{\Ord}{OR}
\DeclareMathOperator{\WO}{WO}
\DeclareMathOperator{\OD}{OD}
\DeclareMathOperator{\HOD}{HOD}
\DeclareMathOperator{\HC}{HC}
\DeclareMathOperator{\WF}{WF}
\DeclareMathOperator{\HF}{HF}
\newcommand{\One}{I}
\newcommand{\ONE}{I}
\newcommand{\Two}{II}
\newcommand{\TWO}{II}

%%%%%%%%%%%%%%%%%%%%%%%%%%%%%%%%%%%%%%%%%%%%%
% Commutative Algebra Constants
%%%%%%%%%%%%%%%%%%%%%%%%%%%%%%%%%%%%%%%%%%%%%
\DeclareMathOperator{\dottimes}{\dot{\times}}

%%%%%%%%%%%%%%%%%%%%%%%%%%%%%%%%%%%%%%%%%%%%%
% Theories
%%%%%%%%%%%%%%%%%%%%%%%%%%%%%%%%%%%%%%%%%%%%%
\DeclareMathOperator{\ZFC}{ZFC}
\DeclareMathOperator{\ZF}{ZF}
\DeclareMathOperator{\AD}{AD}
\DeclareMathOperator{\ADR}{AD_{\R}}
\DeclareMathOperator{\KP}{KP}
\DeclareMathOperator{\PD}{PD}
\DeclareMathOperator{\CH}{CH}
\DeclareMathOperator{\HPC}{HPC} % HOD pair capturing
%%%%%%%%%%%%%%%%%%%%%%%%%%%%%%%%%%%%%%%%%%%%%
% Iteration Trees
%%%%%%%%%%%%%%%%%%%%%%%%%%%%%%%%%%%%%%%%%%%%%

\newcommand{\pred}{\text{-pred}}

%%%%%%%%%%%%%%%%%%%%%%%%%%%%%%%%%%%%%%%%%%%%%%%%
% Operator Names
%%%%%%%%%%%%%%%%%%%%%%%%%%%%%%%%%%%%%%%%%%%%%%%%
\DeclareMathOperator{\Det}{Det}
\DeclareMathOperator{\dom}{dom}
\DeclareMathOperator{\ran}{ran}
\DeclareMathOperator{\range}{ran}
\DeclareMathOperator{\image}{image}
\DeclareMathOperator{\crit}{crit}
\DeclareMathOperator{\card}{card}
\DeclareMathOperator{\cf}{cf}
\DeclareMathOperator{\cof}{cof}
\DeclareMathOperator{\rank}{rank}
\DeclareMathOperator{\ot}{o.t.}
\DeclareMathOperator{\ords}{o}
\DeclareMathOperator{\ro}{r.o.}
\DeclareMathOperator{\rud}{rud}
\DeclareMathOperator{\Powerset}{\mathcal{P}}
\DeclareMathOperator{\length}{lh}
\DeclareMathOperator{\lh}{lh}
\DeclareMathOperator{\limit}{lim}
\DeclareMathOperator{\fld}{fld}
\DeclareMathOperator{\projection}{p}
\DeclareMathOperator{\Ult}{Ult}
\DeclareMathOperator{\ULT}{Ult}
\DeclareMathOperator{\Coll}{Coll}
\DeclareMathOperator{\Col}{Col}
\DeclareMathOperator{\Hull}{Hull}
\DeclareMathOperator*{\dirlim}{dir lim}
\DeclareMathOperator{\Scale}{Scale}
\DeclareMathOperator{\supp}{supp}
\DeclareMathOperator{\trancl}{tran.cl.}
\DeclareMathOperator{\trace}{Tr}
\DeclareMathOperator{\diag}{diag}
\DeclareMathOperator{\spn}{span}
\DeclareMathOperator{\sgn}{sgn}
%%%%%%%%%%%%%%%%%%%%%%%%%%%%%%%%%%%%%%%%%%%%%
% Logical Connectives
%%%%%%%%%%%%%%%%%%%%%%%%%%%%%%%%%%%%%%%%%%%%%
\newcommand{\IImplies}{\Longrightarrow}
\newcommand{\SkipImplies}{\quad\Longrightarrow\quad}
\newcommand{\Ifff}{\Longleftrightarrow}
\newcommand{\iimplies}{\longrightarrow}
\newcommand{\ifff}{\longleftrightarrow}
\newcommand{\Implies}{\Rightarrow}
\newcommand{\Iff}{\Leftrightarrow}
\renewcommand{\implies}{\rightarrow}
\renewcommand{\iff}{\leftrightarrow}
\newcommand{\AND}{\wedge}
\newcommand{\OR}{\vee}
\newcommand{\st}{\text{ s.t. }}
\newcommand{\Or}{\text{ or }}

%%%%%%%%%%%%%%%%%%%%%%%%%%%%%%%%%%%%%%%%%%%%%
% Function Arrows
%%%%%%%%%%%%%%%%%%%%%%%%%%%%%%%%%%%%%%%%%%%%%

\newcommand{\injection}{\xrightarrow{\text{1-1}}}
\newcommand{\surjection}{\xrightarrow{\text{onto}}}
\newcommand{\bijection}{\xrightarrow[\text{onto}]{\text{1-1}}}
\newcommand{\cofmap}{\xrightarrow{\text{cof}}}
\newcommand{\map}{\rightarrow}

%%%%%%%%%%%%%%%%%%%%%%%%%%%%%%%%%%%%%%%%%%%%%
% Mouse Comparison Operators
%%%%%%%%%%%%%%%%%%%%%%%%%%%%%%%%%%%%%%%%%%%%%
\newcommand{\initseg}{\trianglelefteq}
\newcommand{\properseg}{\lhd}
\newcommand{\notinitseg}{\ntrianglelefteq}
\newcommand{\notproperseg}{\ntriangleleft}

%%%%%%%%%%%%%%%%%%%%%%%%%%%%%%%%%%%%%%%%%%%%%
%           calligraphic letters
%%%%%%%%%%%%%%%%%%%%%%%%%%%%%%%%%%%%%%%%%%%%%
\newcommand{\cA}{\mathcal{A}}
\newcommand{\cB}{\mathcal{B}}
\newcommand{\cC}{\mathcal{C}}
\newcommand{\cD}{\mathcal{D}}
\newcommand{\cE}{\mathcal{E}}
\newcommand{\cF}{\mathcal{F}}
\newcommand{\cG}{\mathcal{G}}
\newcommand{\cH}{\mathcal{H}}
\newcommand{\cI}{\mathcal{I}}
\newcommand{\cJ}{\mathcal{J}}
\newcommand{\cK}{\mathcal{K}}
\newcommand{\cL}{\mathcal{L}}
\newcommand{\cM}{\mathcal{M}}
\newcommand{\cN}{\mathcal{N}}
\newcommand{\cO}{\mathcal{O}}
\newcommand{\cP}{\mathcal{P}}
\newcommand{\cQ}{\mathcal{Q}}
\newcommand{\cR}{\mathcal{R}}
\newcommand{\cS}{\mathcal{S}}
\newcommand{\cT}{\mathcal{T}}
\newcommand{\cU}{\mathcal{U}}
\newcommand{\cV}{\mathcal{V}}
\newcommand{\cW}{\mathcal{W}}
\newcommand{\cX}{\mathcal{X}}
\newcommand{\cY}{\mathcal{Y}}
\newcommand{\cZ}{\mathcal{Z}}


%%%%%%%%%%%%%%%%%%%%%%%%%%%%%%%%%%%%%%%%%%%%%
%          Primed Letters
%%%%%%%%%%%%%%%%%%%%%%%%%%%%%%%%%%%%%%%%%%%%%
\newcommand{\aprime}{a^{\prime}}
\newcommand{\bprime}{b^{\prime}}
\newcommand{\cprime}{c^{\prime}}
\newcommand{\dprime}{d^{\prime}}
\newcommand{\eprime}{e^{\prime}}
\newcommand{\fprime}{f^{\prime}}
\newcommand{\gprime}{g^{\prime}}
\newcommand{\hprime}{h^{\prime}}
\newcommand{\iprime}{i^{\prime}}
\newcommand{\jprime}{j^{\prime}}
\newcommand{\kprime}{k^{\prime}}
\newcommand{\lprime}{l^{\prime}}
\newcommand{\mprime}{m^{\prime}}
\newcommand{\nprime}{n^{\prime}}
\newcommand{\oprime}{o^{\prime}}
\newcommand{\pprime}{p^{\prime}}
\newcommand{\qprime}{q^{\prime}}
\newcommand{\rprime}{r^{\prime}}
\newcommand{\sprime}{s^{\prime}}
\newcommand{\tprime}{t^{\prime}}
\newcommand{\uprime}{u^{\prime}}
\newcommand{\vprime}{v^{\prime}}
\newcommand{\wprime}{w^{\prime}}
\newcommand{\xprime}{x^{\prime}}
\newcommand{\yprime}{y^{\prime}}
\newcommand{\zprime}{z^{\prime}}
\newcommand{\Aprime}{A^{\prime}}
\newcommand{\Bprime}{B^{\prime}}
\newcommand{\Cprime}{C^{\prime}}
\newcommand{\Dprime}{D^{\prime}}
\newcommand{\Eprime}{E^{\prime}}
\newcommand{\Fprime}{F^{\prime}}
\newcommand{\Gprime}{G^{\prime}}
\newcommand{\Hprime}{H^{\prime}}
\newcommand{\Iprime}{I^{\prime}}
\newcommand{\Jprime}{J^{\prime}}
\newcommand{\Kprime}{K^{\prime}}
\newcommand{\Lprime}{L^{\prime}}
\newcommand{\Mprime}{M^{\prime}}
\newcommand{\Nprime}{N^{\prime}}
\newcommand{\Oprime}{O^{\prime}}
\newcommand{\Pprime}{P^{\prime}}
\newcommand{\Qprime}{Q^{\prime}}
\newcommand{\Rprime}{R^{\prime}}
\newcommand{\Sprime}{S^{\prime}}
\newcommand{\Tprime}{T^{\prime}}
\newcommand{\Uprime}{U^{\prime}}
\newcommand{\Vprime}{V^{\prime}}
\newcommand{\Wprime}{W^{\prime}}
\newcommand{\Xprime}{X^{\prime}}
\newcommand{\Yprime}{Y^{\prime}}
\newcommand{\Zprime}{Z^{\prime}}
\newcommand{\alphaprime}{\alpha^{\prime}}
\newcommand{\betaprime}{\beta^{\prime}}
\newcommand{\gammaprime}{\gamma^{\prime}}
\newcommand{\Gammaprime}{\Gamma^{\prime}}
\newcommand{\deltaprime}{\delta^{\prime}}
\newcommand{\epsilonprime}{\epsilon^{\prime}}
\newcommand{\kappaprime}{\kappa^{\prime}}
\newcommand{\lambdaprime}{\lambda^{\prime}}
\newcommand{\rhoprime}{\rho^{\prime}}
\newcommand{\Sigmaprime}{\Sigma^{\prime}}
\newcommand{\tauprime}{\tau^{\prime}}
\newcommand{\xiprime}{\xi^{\prime}}
\newcommand{\thetaprime}{\theta^{\prime}}
\newcommand{\Omegaprime}{\Omega^{\prime}}
\newcommand{\cMprime}{\cM^{\prime}}
\newcommand{\cNprime}{\cN^{\prime}}
\newcommand{\cPprime}{\cP^{\prime}}
\newcommand{\cQprime}{\cQ^{\prime}}
\newcommand{\cRprime}{\cR^{\prime}}
\newcommand{\cSprime}{\cS^{\prime}}
\newcommand{\cTprime}{\cT^{\prime}}

%%%%%%%%%%%%%%%%%%%%%%%%%%%%%%%%%%%%%%%%%%%%%
%          bar Letters
%%%%%%%%%%%%%%%%%%%%%%%%%%%%%%%%%%%%%%%%%%%%%
\newcommand{\abar}{\bar{a}}
\newcommand{\bbar}{\bar{b}}
\newcommand{\zbar}{\bar{z}}
\newcommand{\phibar}{\bar{\varphi}}
\newcommand{\psibar}{\bar{\psi}}
\newcommand{\thetabar}{\bar{\theta}}
\newcommand{\nubar}{\bar{\nu}}

%%%%%%%%%%%%%%%%%%%%%%%%%%%%%%%%%%%%%%%%%%%%%
%          star Letters
%%%%%%%%%%%%%%%%%%%%%%%%%%%%%%%%%%%%%%%%%%%%%
\newcommand{\phistar}{\phi^{*}}


%%%%%%%%%%%%%%%%%%%%%%%%%%%%%%%%%%%%%%%%%%%%%
%          Formulas
%%%%%%%%%%%%%%%%%%%%%%%%%%%%%%%%%%%%%%%%%%%%%

\newcommand{\formulaphi}{\text{\large $\varphi$}}
\newcommand{\Formulaphi}{\text{\Large $\varphi$}}


%%%%%%%%%%%%%%%%%%%%%%%%%%%%%%%%%%%%%%%%%%%%%
%          Fraktur Letters
%%%%%%%%%%%%%%%%%%%%%%%%%%%%%%%%%%%%%%%%%%%%%

\newcommand{\fa}{\mathfrak{a}}
\newcommand{\fb}{\mathfrak{b}}
\newcommand{\fc}{\mathfrak{c}}
\newcommand{\fk}{\mathfrak{k}}
\newcommand{\fp}{\mathfrak{p}}
\newcommand{\fq}{\mathfrak{q}}
\newcommand{\fr}{\mathfrak{r}}
\newcommand{\fA}{\mathfrak{A}}
\newcommand{\fB}{\mathfrak{B}}
\newcommand{\fC}{\mathfrak{C}}
\newcommand{\fD}{\mathfrak{D}}

%%%%%%%%%%%%%%%%%%%%%%%%%%%%%%%%%%%%%%%%%%%%%
%          Bold Letters
%%%%%%%%%%%%%%%%%%%%%%%%%%%%%%%%%%%%%%%%%%%%%
\newcommand{\ba}{\mathbf{a}}
\newcommand{\bb}{\mathbf{b}}
\newcommand{\bc}{\mathbf{c}}
\newcommand{\bd}{\mathbf{d}}
\newcommand{\be}{\mathbf{e}}
\newcommand{\bu}{\mathbf{u}}
\newcommand{\bv}{\mathbf{v}}
\newcommand{\bw}{\mathbf{w}}
\newcommand{\bx}{\mathbf{x}}
\newcommand{\by}{\mathbf{y}}
\newcommand{\bz}{\mathbf{z}}
\newcommand{\bSigma}{\boldsymbol{\Sigma}}
\newcommand{\bPi}{\boldsymbol{\Pi}}
\newcommand{\bDelta}{\boldsymbol{\Delta}}
\newcommand{\bdelta}{\boldsymbol{\delta}}
\newcommand{\bgamma}{\boldsymbol{\gamma}}
\newcommand{\bGamma}{\boldsymbol{\Gamma}}

%%%%%%%%%%%%%%%%%%%%%%%%%%%%%%%%%%%%%%%%%%%%%
%         Bold numbers
%%%%%%%%%%%%%%%%%%%%%%%%%%%%%%%%%%%%%%%%%%%%%
\newcommand{\bzero}{\mathbf{0}}

%%%%%%%%%%%%%%%%%%%%%%%%%%%%%%%%%%%%%%%%%%%%%
% Projective-Like Pointclasses
%%%%%%%%%%%%%%%%%%%%%%%%%%%%%%%%%%%%%%%%%%%%%
\newcommand{\Sa}[2][\alpha]{\Sigma_{(#1,#2)}}
\newcommand{\Pa}[2][\alpha]{\Pi_{(#1,#2)}}
\newcommand{\Da}[2][\alpha]{\Delta_{(#1,#2)}}
\newcommand{\Aa}[2][\alpha]{A_{(#1,#2)}}
\newcommand{\Ca}[2][\alpha]{C_{(#1,#2)}}
\newcommand{\Qa}[2][\alpha]{Q_{(#1,#2)}}
\newcommand{\da}[2][\alpha]{\delta_{(#1,#2)}}
\newcommand{\leqa}[2][\alpha]{\leq_{(#1,#2)}}
\newcommand{\lessa}[2][\alpha]{<_{(#1,#2)}}
\newcommand{\equiva}[2][\alpha]{\equiv_{(#1,#2)}}


\newcommand{\Sl}[1]{\Sa[\lambda]{#1}}
\newcommand{\Pl}[1]{\Pa[\lambda]{#1}}
\newcommand{\Dl}[1]{\Da[\lambda]{#1}}
\newcommand{\Al}[1]{\Aa[\lambda]{#1}}
\newcommand{\Cl}[1]{\Ca[\lambda]{#1}}
\newcommand{\Ql}[1]{\Qa[\lambda]{#1}}

\newcommand{\San}{\Sa{n}}
\newcommand{\Pan}{\Pa{n}}
\newcommand{\Dan}{\Da{n}}
\newcommand{\Can}{\Ca{n}}
\newcommand{\Qan}{\Qa{n}}
\newcommand{\Aan}{\Aa{n}}
\newcommand{\dan}{\da{n}}
\newcommand{\leqan}{\leqa{n}}
\newcommand{\lessan}{\lessa{n}}
\newcommand{\equivan}{\equiva{n}}

%%%%%%%%%%%%%%%%%%%%%%%%%%%%%%%%%%%%%%%%%%%%%
% Linear Algebra
%%%%%%%%%%%%%%%%%%%%%%%%%%%%%%%%%%%%%%%%%%%%%
\newcommand{\transpose}[1]{{#1}^{\text{T}}}
\newcommand{\norm}[1]{\lVert{#1}\rVert}
\newcommand\aug{\fboxsep=-\fboxrule\!\!\!\fbox{\strut}\!\!\!}

%%%%%%%%%%%%%%%%%%%%%%%%%%%%%%%%%%%%%%%%%%%%%
% Number Theory
%%%%%%%%%%%%%%%%%%%%%%%%%%%%%%%%%%%%%%%%%%%%%
\DeclareMathOperator{\Spec}{Spec}
\newcommand{\av}[1]{\lvert#1\rvert}
\DeclareMathOperator{\divides}{\mid}
\DeclareMathOperator{\ndivides}{\nmid}

%%%%%%%%%%%%%%%%%%%%%%%%%%%%%%%%%%%%%%%%%%%%%%%%%%%%%%%%%%%%%%%%%%%%%%%%%%%
%%  Theorem-Like Declarations
%%%%%%%%%%%%%%%%%%%%%%%%%%%%%%%%%%%%%%%%%%%%%%%%%%%%%%%%%%%%%%%%%%%%%%%%%%

\newtheorem{theorem}{Theorem}[section]
\newtheorem{lemma}[theorem]{Lemma}
\newtheorem{corollary}[theorem]{Corollary}
\newtheorem{proposition}[theorem]{Proposition}


\theoremstyle{definition}

\newtheorem{definition}[theorem]{Definition}
\newtheorem{conjecture}[theorem]{Conjecture}
\newtheorem{remark}[theorem]{Remark}
\newtheorem{remarks}[theorem]{Remarks}
\newtheorem{notation}[theorem]{Notation}

\newtheorem{homework}[theorem]{Exercise}
\newtheorem{numbered_example}[theorem]{Example}

\theoremstyle{remark}

\newtheorem*{aside}{Aside}
\newtheorem*{note}{Note}
\newtheorem*{observation}{Observation}
\newtheorem*{warning}{Warning}
\newtheorem*{question}{Question}
\newtheorem*{example}{Example}
\newtheorem*{in_class_example}{In-Class Example}
\newtheorem*{exercise}{Exercise}
\newtheorem*{fact}{Fact}


\newenvironment*{subproof}[1][Proof]
{\begin{proof}[#1]}{\renewcommand{\qedsymbol}{$\diamondsuit$} \end{proof}}

\newenvironment*{case}[1]
{\textbf{Case #1.  }\itshape }{}

\newenvironment*{claim}[1][Claim]
{\textbf{#1.  }\itshape }{}


%\pagestyle{plain}

\begin{document}

\title{The Mouse Set Theorem Just Past Projective}
\author{Mitch Rudominer}
\author{John R. Steel}
\author{W. Hugh Woodin}


\keywords{large cardinals, descriptive set theory, inner model theory}

\begin{abstract}
We describe a natural mouse $\Mladder$, the minimal ladder mouse,
 and show that $\R\intersect \Mladder = Q_{\omega+1}$
where $Q_{\omega+1}$  is the set of reals that are
$\Delta^1_{\omega+1}$ in a countable ordinal. Thus $Q_{\omega+1}$
is a mouse set.

This is analogous to the fact that $\R\intersect M^{\sharp}_1 = Q_3$ where $M^{\sharp}_1$ is the
the sharp for the minimal inner model with a Woodin cardinal, and $Q_3$ is the set of reals
that are $\Delta^1_3$ in a countable ordinal.

More generally $\R\intersect M^{\sharp}_{2n+1} = Q_{2n+3}$.
The set $Q_{\omega+1}$  we consider in this paper is the next natural
one to consider in this series of results. Thus we are proving the mouse
set theorem just past projective.

Some of this is not new. $\R\intersect \Mladder \subseteq Q_{\omega+1}$ was
know in the 1990's. But a proof that $Q_{\omega+1} \subseteq \Mladder$ wasn't known
until Woodin found it in 2018. The main goal of this paper is to give
Woodin's proof.
\end{abstract}

\maketitle

\tableofcontents

\section{Introduction}
\label{section:intro}

We work in $\ZFC+\AD(\LofR)$ for easy quoting. The determinacy of all games
in $J_3(\R)$ is sufficient for most of the results.

Throughout this paper we write $\R$ to mean $\Bairespace$ and we call elements of $\Bairespace$ \emph{reals}.

In the 1990's Martin, Steel and Woodin proved that, assuming large cardinals or determinacy,
$\R\intersect M_n^{\sharp} = $ the set of reals that are $\Delta^1_n$ in a countable ordinal.
For even $n\geq 2$, $\R\intersect M_n^{\sharp} = C_n$, the largest countable $\Sigma^1_n$ set.
For odd $n\geq 3$, $\R\intersect M_n^{\sharp} = Q_n$, the largest countable $\Pi^1_n$
set closed downwards under $\Delta^1_3$ degrees. See \cite{Proj_WO_In_Mod}.

In 1995 Rudominer extended this work to pointclasses beyond the projective
and achieved partial results, proving the mouse set theorem for some pointclasses
and getting partial results for others. See \cite{My_Thesis} and \cite{Mouse_Sets}.
But even at the very first step past projective the full mouse set theorem was open.

Let $\Sigma^1_{\omega}$
be the pointclass consisting of recursive unions of Projective sets. Consider
the projective hierarchy built over $\Sigma^1_{\omega}$:
$\Pi^1_{\omega} = \neg \Sigma^1_{\omega}$,
$\Sigma^1_{\omega+1} = \exists^{\R} \Pi^1_{\omega}$, etc.
By \cite{Scales_In_LofR}, $\Sigma^1_{\omega+n} = \Powerset(\R)\intersect \Sigma_{n+1}(J_2(\R))$
and we get the Second Periodicity Theorem: $\Sigma^1_{\omega+2n}$ and $\Pi^1_{\omega+2n+1}$ are scaled pointclasses.
$\Sigma^1_{\omega}$ is not closed under $\forall^{\omega}$ and so is not a perfect analog of $\Sigma^1_1$.
But $\Pi^1_{\omega+1}$ is closed under $\exists^{\omega}$ and is a good analog of $\Pi^1_1$ or $\Pi^1_3$.
Let $Q_{\omega+1}$ be the set of reals that are $\Delta^1_{\omega+1}$ in a countable ordinal.
In \cite{My_Thesis}, Rudominer defined the minimal ladder mouse $\Mladder$ and showed that
$\R\intersect\Mladder\subseteq Q_{\omega+1}$. ($\Mladder$ is the minimal iterable mouse $M$ such
that for each $n\in\omega$ there is a cardinal $\delta_n$ of $M$ and an initial segment $P_n \initseg M$
such that $\delta_n$ is Woodin in $P_n$ and $P_n$ is not $n$-small above $\delta_n$.)

But the other direction, $Q_{\omega+1} \subseteq \Mladder$, remained open until
Woodin found a proof in the summer of 2018. Woodin transmitted the proof to Steel who fleshed it out
in some hand-written notes. Rudominer distilled those notes into this paper.

No new technology is involved in the proof and it could have been discovered in the 1990's. The main ingredients
of the proof are stationary tower forcing, genericity iterations, and a theorem of Hjorth stating that
all generic extensions of an appropriate model meet the same equivalence classes of a thin, definable equivalence
relation.

\section{Project-like Pointclasses Just Past Projective}
\label{section:projectlikepointclasses}

\begin{definition}
Let $\PiOneOmega$ be the pointclass of recursive intersections of infinitely many (lightface) projective sets.

In more detail, fix for the remainder of this paper, for $n\in\omega$, $G^n\subset\omega\times\R$
a universal $\Pi^1_{2n+1}$ set, uniformly in $n$. (``Uniformly in $n$'' here and below
means there is a recursive sequence of formula with the $n$-th formula defining the $n$-th relation.)
A $\PiOneOmega$ \emph{code} is a total recursive function $h:\omega\to\omega$. If $h$ is a $\PiOneOmega$ code then
$A^h=\setof{x\in\R}{(\forall n)\, G^n(h(n),x)}$. Say $A\subseteq\R$ is $\PiOneOmega$ iff
$A=A^h$ for some $\PiOneOmega$ code $h$. In this case we will also say that $h$
is a $\PiOneOmega$ code \emph{for} $A$.

Similarly we define $\PiOneOmega$ codes and $\PiOneOmega$ subsets of $\omega^s\times \R^t$ for $s,t\in\omega$.

Then we define $\SigmaOneOmega = \neg\PiOneOmega$, $\SigmaOneOmegaPlusOne=\exists^{\R}\PiOneOmega$,
$\PiOneOmegaPlusOne = \forall^{\R}\SigmaOneOmega$ and
$\DeltaOneOmegaPlusOne = \SigmaOneOmegaPlusOne \intersect \PiOneOmegaPlusOne$.

Finally we define as usual the relativized and bold-face
pointclasses $\PiOneOmega(x)$
and $\bPi^1_{\omega}$ etc.
\end{definition}

The following remarks follow from \cite{Scales_In_LofR}.
\begin{remarks} \
\begin{itemize}
\item $\SigmaOneOmega = \Sigma_1(\JofR{2}) \intersect \Powerset(\R)$
\item $\PiOneOmega = \Pi_1(\JofR{2}) \intersect \Powerset(\R)$
\item $\SigmaOneOmegaPlusOne = \Sigma_2(\JofR{2}) \intersect \Powerset(\R)$
\item $\PiOneOmegaPlusOne = \Pi_2(\JofR{2}) \intersect \Powerset(\R)$
\item $\SigmaOneOmega$ and $\PiOneOmegaPlusOne$
are scaled pointclasses.
\end{itemize}
\end{remarks}


\begin{definition}
We say that $x\in\R$ is $\DeltaOneOmegaPlusOne$ in a countable ordinal iff there is an $\alpha<\omega_1$ such
that for all $w\in\WO$ with $|w|=\alpha$, $\singleton{x}$ is $\DeltaOneOmegaPlusOne(w)$.
Let $Q_{\omega+1} = \setof{x\in\R}{x\text{ is }\DeltaOneOmegaPlusOne\text{ in a countable ordinal}}$.
\end{definition}

By way of intuition, we consider $\PiOneOmegaPlusOne$ to be analogous to
$\Pi^1_1$ or $\Pi^1_3$ and we consider $Q_{\omega+1}$ to be analogous to
$Q_1$ or $Q_3$. See \cite{Q_Theory}.

Recall that $Q_3 = \R \intersect M_1^{\sharp}$. In the next section we will
define a mouse $\Mladder$ such that $Q_{\omega+1} = \R\intersect\Mladder$.

\begin{remarks} \
\begin{enumerate}
\item $x\in Q_{\omega+1}$ iff there is a $\SigmaOneOmegaPlusOne$ relation $R$ and
a countable ordinal $\alpha$ such that for all $w\in\WO$ with $|w|=\alpha$,
and for all reals $y$, $R(w,y) \Iff y=x$.
\item $x\in Q_{\omega+1}$ iff there is a $\SigmaOneOmegaPlusOne$ relation $R$
and a countable ordinal $\alpha$ such
that for all $w\in\WO$ with $|w|=\alpha$, and all $i,j\in\omega$,
$x(i)=j \Iff R(w,i,j)$.
\item $Q_{\omega+1}$ is the largest countable
$\DeltaOneOmegaPlusOne$ set closed downwards under
$\DeltaOneOmegaPlusOne$-degrees.
\end{enumerate}
\end{remarks}


\section{Ladder Mice}
\label{section:laddermice}
\begin{definition}
Let $M$ be a premouse, in the sense of \cite{FSIT}. A \emph{ladder over $M$}
is a sequence of ordinals $\sequence{\delta_n,\gamma_n}{n\in\omega}$ such that
for all $n$:
\begin{enumerate}
\item $\delta_n < \gamma_n < \delta_{n+1} < \ord(M)$,
\item $\delta_n$ is a cardinal of $M$,
\item $\delta_n$ is Woodin in $J^M_{\gamma_n}$,
\item $\gamma_n$ is the least $\gamma$ such that $\cJ^M_{\gamma}$ is active and
$\cJ^M_{\gamma}\models$ there are $n$ Woodin cardinals greater than $\delta_n$.
\end{enumerate}

A ladder $\sequence{\delta_n,\gamma_n}{n\in\omega}$ over $M$ is \emph{cofinal}
iff the $\gamma_n$ are cofinal in $\ord(M)$.

$M$ is a \emph{ladder mouse} iff there is a ladder over $M$ and a
\emph{cofinal ladder mouse} iff there is a cofinal ladder over $M$.

$M$ is a \emph{minimal ladder mouse} iff $M$ is a ladder mouse but no initial
segment of $M$ is.

$\Mladder$ is the least fully-iterable ladder mouse, if it exists. If
$\Mladder$ exists it is obviously a minimal ladder mouse (and so a cofinal
ladder mouse), and $\Mladder$ projects to $\omega$ and so it is an $\omega$-mouse
in the sense of \cite{Proj_WO_In_Mod}.
\end{definition}

\begin{remarks} \
\begin{enumerate}
\item Because we are assuming $\AD(\LofR)$, $\Mladder$ exists.
\item For every $n\in\omega$, $\Mladder$ has a rank initial segment that satisfies
$\ZFC$+$\exists n$ Woodin cardinals.  By \cite{Proj_WO_In_Mod}, $\Mladder$ is projectively correct and so $\PiOneOmega$-correct.
\item Let $\sequence{\delta_n,\gamma_n}{n\in\omega}$ be a ladder over $\Mladder$.
For even $n$, $J^{\Mladder}_{\gamma_n}[g]$ is $\Sigma^1_{n+2}$-correct,
where $g$ is $J^{\Mladder}_{\gamma_n}$-generic over $\Coll(\omega,\delta_n)$.
\item Let $\sequence{\delta_n,\gamma_n}{n\in\omega}$ be a ladder over $M$.
$\delta_n$ is not necessarily fully Woodin in $M$, but $\delta_n$ is Woodin in $M$
with respect to functions in $J^M_{\gamma_n}$.
\item By the previous two items, we can, at lest informally, think of
$\Mladder$ as the least mouse
$M$ such that for every $n\in\omega$ there is a cardinal $\delta$ of $M$ such
that $\delta$ is ``$\Sigma^1_n$-Woodin'' in $M$.
\end{enumerate}
\end{remarks}

We can now state the main theorem of the paper:

\begin{theorem}
$Q_{\omega+1} = \R \intersect \Mladder$.
\end{theorem}

We divide the above theorem up into its two directions:

\begin{theorem}
\label{MRealsAreDefinable}
$\R \intersect \Mladder \subseteq Q_{\omega+1}$.
\end{theorem}

\begin{theorem}
\label{DefinableRealsAreInM}
$Q_{\omega+1} \subseteq \Mladder$.
\end{theorem}

Theorem \ref{MRealsAreDefinable} was proven more than 25 years ago in
\cite{My_Thesis} and \cite{Mouse_Sets}. We give a sketch of the proof later
in section \ref{ProofThatMRealsAreDefinable}. The proof follows the same
line of reasoning as the proof that every real in $L$ is $\Delta^1_2$ in a
countable ordinal and the proof that every real in $M_1$ is $\Delta^1_3$ in
a countable ordinal. Namely we show that every initial segment of $\Mladder$
that projects to $\omega$ is $\PiOneOmega$ definable from its ordinal height.

As mentioned in the introduction, the main goal of this paper is to give Woodin's proof of Theorem \ref{DefinableRealsAreInM}.

Theorem \ref{DefinableRealsAreInM} will follow from a quasi-correctness theorem
for $\Mladder$. $\Mladder$ is $\PiOneOmega$-correct but not $\SigmaOneOmegaPlusOne$-correct.
Let $A\subset\R^2$ be $\PiOneOmega$ and let $x\in\R\intersect\Mladder$. It may
happen that there is a $y\in \R$ such that $A(x,y)$ but there is no such $y$
in $\Mladder$. But we will show that even if $\Mladder\not\models(\exists y) A(x,y)$,
$\Mladder$ can still determine whether or not the statement
``$(\exists y) A(x,y)$'' is true. This is similar to the situation with $M_1$ and
$\Sigma^1_3$. $M_1$ is not $\Sigma^1_3$-correct, but $M_1$ can still tell whether
or not $\Sigma^1_3$ statements are true. Let $A\subset\R^2$ be $\Pi^1_2$ and
let $x\in M_1$. Then $(\exists y\in\R)\, A(x,y)$ iff
$$M_1\models 1\force{\Coll(\omega,\delta)} (\exists y) A(x,y),$$
where $\delta$ is the Woodin of $M_1$.

Next we state the
$\SigmaOneOmegaPlusOne$-quasi-correctness for $\Mladder$ precisely and
we show how
Theorem \ref{DefinableRealsAreInM} follows from it.

If $z$ is a real, then by a \emph{mouse over $z$} we mean a mouse in the
sense of the theory of \cite{FSIT} augmented to use $z$ as an additional predicate.

\begin{theorem}
\label{QuasiCorrectness}
There is a formula $\psi$ in the language of Set Theory such that, for all
reals $z$, for all $M$ a countable, iterable, ladder-mouse over $z$,
there is an ordinal parameter $\theta<\omega_2^M$ such that
for all reals $x$ in $M$,
for all $\PiOneOmega$ codes $h$,
$$(\exists y) A^h(x,y) \Iff J^M_{\omega_2^M} \models \psi[\theta, h,x].$$
\end{theorem}

Now, assuming Theorem \ref{QuasiCorrectness}, we prove Theorem \ref{DefinableRealsAreInM}.

\begin{proof}[proof of Theorem \ref{DefinableRealsAreInM}]
Let $x\in Q_{\omega+1}$. We need to show that $x\in\Mladder$.
Fix a $\PiOneOmega$ code $h$ and a countable ordinal $\alpha$ such that for
all $w\in\WO$ with $|w|=\alpha$, and all $i,j\in\omega$,
$$x(i)=j \Iff (\exists y\in\R)A^h(i,j,w,y).$$
Let $\Mprime$ be the $(\alpha+1)$-th iterate of $\Mladder$ by its least
measurable cardinal. So $\Mprime$ is a countable, iterable ladder mouse with
a ladder whose $\delta_0$ is greater than $\alpha$. Let $g$ be $\Mprime$-generic
for $\Coll(\omega,\alpha$). It suffices to see that $x\in\Mprime[g]$.

We can rearrange $\Mprime[g]$ as a mouse over $z$, where $z$ is some real coding $g$.
Let $M$ be this mouse over $z$. $M$ is a countable, iterable ladder-mouse over $z$ and there
is a real $w\in M\intersect\WO$ with $|w|=\alpha$.

Let $\psi$ and $\theta$ be the formula and parameter given by Theorem \ref{QuasiCorrectness}
for $M$. Then for all
$i,j\in\omega$,
$$x(i)=j \Iff (\exists y\in\R)A^h(i,j,w,y) \Iff J^M_{\omega_2^M} \models \psi[\theta,h,i,j,w].$$
So $x\in M$.
\end{proof}


We now develop the ideas that will allow us to prove theorem \ref{QuasiCorrectness}. We
start in the next section with a Suslin representation for $\PiOneOmega$ sets.

\section{A Suslin Representation for $\PiOneOmega$}
\label{section:suslinrep}

Let $\bdelta^1_{\omega} = \sup_n \bdelta^1_n$. In this section we show how to
express a $\PiOneOmega$ set as the projection
of a tree on $\omega\times \bdelta^1_{\omega}$.

\begin{definition}
Let $h$ be a $\PiOneOmega$ code for a subset of $\R$.
There is a natural tree $T^{h}$ on $\omega\times\bdelta^1_{\omega}$ that projects to $A^h$.

Recall that $G^n\subset\omega\times\R$ is a universal $\Pi^1_{2n+1}$ set, uniformly in $n$.

For $n,e\in\omega$, let $G^n_e = \setof{x\in\R}{G^n(e, x)}$ and
let $\varphi^n_e = \sequence{\varphi^n_{e,i}}{i\in\omega}$ be a $\Pi^1_{2n+1}$ scale on $G^n_e$ uniformly in $n$ and $e$,
with each of the norms $\varphi^n_{e,i}$ being regular.
Let $<^n_{e,i}$ be the prewellorder on $\R$ associated with $\varphi^n_{e,i}$. So ``$x,y\in G^n_e$ and $x <^n_{e,i} y$'' is
$\Pi^1_{2n+1}$, uniformly in $n,e,i$, and $\varphi^n_{e,i}(x) = $ the rank of $x$ in $<^n_{e,i}$.
$\varphi^n_{e,i} : G^n_e \map \bdelta^1_{2n+1}$.

Let $T^n_e$ be the tree for the scale $\varphi^n_e$:
$$T^n_e = \setof{\left(x\restr k,\sequence{\varphi^n_{e,i}(x)}{i<k}\right)}{x\in G^n_e \text{ and } k\in\omega}.$$
$G^n_e=p[T^n_e]$.

Fix some recursive coding of pairs of integers by integers, $i\to\angles{i_0,i_1}$,
with the property that $i_0,i_1\leq i$ for $i\in\omega$.

Now fix a $\PiOneOmega$ code $h$ and we will define the tree $T^h$.
Let $(s,u) \in \omega^n\times(\bdelta^1_{\omega})^n$ for some $n\in\omega$. Then $(s,u)\in T^h$ iff there is an
$x\in\R$ extending $s$ such that $(\forall i<n)\, x\in G^{i}_{h(i)}$ and
$u(i)=\varphi^{i_0}_{h(i_0),i_1}(x)$.

We say that $T$ is a $\PiOneOmega$-tree iff $T=T^h$ for some $\PiOneOmega$ code $h$.

Similarly we define $\PiOneOmega$-trees $T^h$ for $h$ a $\PiOneOmega$ subset of $\R^k$ for $k\in\omega$.
\end{definition}

Our coding of pairs of integers by integers induces a coding of $\omega$-sequences of $\omega$-sequences
by $\omega$-sequences, $f\to\sequence{f_n}{n\in\omega}$ given by $f_n(i) = f(\angles{n,i})$.

\begin{proposition}
$A^h = p[T^h]$.
\end{proposition}
\begin{proof}
Let $x\in A^h = \Intersection{n} G^n_{h(n)}$. Let $f:\omega\to\bdelta^1_{\omega}$
be given by $f(i)=\varphi^{i_0}_{h(i_0),i_1}(x)$. Then $(x,f)\in [T^h]$.

Conversely, suppose $(x,f)\in [T^h]$. We claim that $\forall n \, (x,f_n)\in[T^n_{h(n)}]$.
Let $m > k > n \in \omega$
be such that $\angles{n,i}<m$ for all $i<k$. Let $\xprime$ be a real
such that $\xprime\restr m = x \restr m$ and $\xprime$ witness that
$(x\restr m, f\restr m)\in T^h$. Then $\xprime\in G^n_{h(n)}$
and $(\forall i < k)\, f_n(i)=f(\angles{n,i})=\varphi^{n}_{h(n),i}(\xprime)$.
So $\xprime$ witnesses that $(x\restr k, f_n\restr k)\in T^n_{h(n)}$.
As $n,k$ were arbitrary, $\forall n \, (x,f_n)\in[T^n_{h(n)}]$.

\end{proof}


\begin{remark}
\item The definition of the first $n$ levels of $T^h$ depends only on
 $\sequence{G^i_{h(i)}}{i<n}$ and
the norms $\sequence{\varphi^i_{h(i), j}}{i,j<n}$, all of which are $\Pi^1_{2n+1}$.
\end{remark}

Next we want to consider the version of $T^h$ inside of a countable, transitive model.

Let $M$ be a countable transitive model of $\ZFC$ that is $\Pi^1_{2n+1}$-correct.
Let $A\subset\R$ be $\Pi^1_{2n+1}$, and let $\varphi:A\map\bdelta^1_{2n+1}$ be a regular
$\Pi^1_{2n+1}$ norm on $A$ with associated pre-wellorder $<_\varphi$.
Then there is a natural way to interpret $\varphi$ in $M$.
$\varphi^M:A\intersect M \map (\bdelta^1_{2n+1})^M$
is given by $\varphi^M(x) = $ the rank of $x$ in $<_\varphi \intersect M$.

\begin{definition}
In the situation described in the previous paragraph, we define
$\sigma^M_{\varphi}:\range(\varphi^M)\map\bdelta^1_{2n+1}$ by
$\sigma^M_{\varphi}(\varphi^M(x))=\varphi(x)$ for all $x\in A \intersect M$.
This is well-defined and order preserving since for $x,y\in A\intersect M$,
$\varphi^M(x)=\varphi^M(y)$ iff $\varphi(x)=\varphi(y)$ and
$\varphi^M(x)<\varphi^M(y)$ iff $\varphi(x)<\varphi(y)$.
\end{definition}

If $M$ is projectively correct then $\sigma^M_{\varphi}$ does not depend on the norm $\varphi$.
We don't actually need this fact but we will use it because it simplifies the
notation, allowing us to drop the subscript in $\sigma^M_{\varphi}$.

\begin{lemma}
Let $M$ be a countable, transitive, projectively correct model of $\ZFC$.
Let $A,B$ be two $\Pi^1_{2n+1}$ sets of reals and let $\varphi_A:A\to\bdelta^1_{2n+1}$
and $\varphi_B:B\to\bdelta^1_{2n+1}$ be $\Pi^1_{2n+1}$ norms on $A$ and $B$ respectively.
Let $\alpha < \min(\ran(\varphi^M_A),\ran(\varphi^M_B))$. Then
$\sigma^M_{\varphi_A}(\alpha) = \sigma^M_{\varphi_B}(\alpha)$.
\end{lemma}
\begin{proof}
If $x\in A$ and $y\in B$ and $\varphi^M_A(x) = \varphi^M_B(y)$, then
$\varphi_A(x)=\varphi_B(y)$. This is because,
by Theorem 3.3.2 of \cite{HarringtonKechris}, the relation
``$\varphi_A(x)=\varphi_B(y)$'' is $\Delta^1_{2n+3}$ and so absolute for $M$.
\end{proof}


\begin{definition}
Let $M$ be a countable, transitive, projectively correct model of $\ZFC$.
Then $\sigma^M:(\bdelta^1_{\omega})^M\to\bdelta^1_{\omega}$ is the union of all
functions $\sigma^M_{\varphi}$ where $\varphi$ is a (lightface) projective norm on a
(lightface) projective set.

Let $h$ be a $\Pi^1_{\omega}$ code. We extend $\sigma^M$ to the function
$\sigma^M:(T^h)^M\to T^h$ given by
$$\sigma^M\left( (\angles{n_0,\cdots, n_k}, \angles{\alpha_0, \cdots, \alpha_k}  ) \right) =
(\angles{n_0,\cdots, n_k}, \angles{\sigma^M(\alpha_0), \cdots, \sigma^M(\alpha_k)}  )$$
\end{definition}

\begin{lemma}
Let $M$ be a countable, transitive, projectively correct model of $\ZFC$.
Let $h$ be a $\PiOneOmega$ code.
Let $\sigma^M$ be as defined above. Then $\sigma^M:(T^h)^M\to T^h$ is a tree isomorphism
onto a subtree of $T^h$.
\end{lemma}
\begin{proof}
If $(s,u) = (\angles{n_0,\cdots, n_k}, \angles{\alpha_0, \cdots, \alpha_k}  ) \in (T^h)^M$
then there is a real $x\in M \intersect G^0_{h(0)} \intersect \cdots \intersect G^k_{h(k)}$ such
that $x(i)=n_i$ and $\alpha_i=(\varphi^{i_0}_{h(i_0),i_1}(x))^M$  for $0\leq i \leq k$.
So $\sigma^M(\alpha_i) = \varphi^{i_0}_{h(i_0),i_1}(x)$ for $0\leq i \leq k$ and so
$x$ witnesses that  $\sigma^M(s,u)\in T^h$.

That $\sigma^M$ is injective, preserves the lengths of finite sequences and sequence extension
is obvious.
\end{proof}

\begin{definition}
If $T$ is a tree on $\omega\times U$ and $x$ is a real then $T(x)$ is the tree through $x$:
$$T(x) = \setof{u}{(x\restr \lh(u), u) \in T}.$$

If $T$ is a $\PiOneOmega$ tree for a subset of $\R^2$ and $x\in\R$
we say that $T(x)$ is a $\PiOneOmega(x)$-tree
and a $\bPi^1_{\omega}$-tree.
\end{definition}

\begin{remarks}
Let $M$ be a countable, transitive, projectively correct model of $\ZFC$.
$M$ is $\PiOneOmega$-correct, but not necessarily $\SigmaOneOmegaPlusOne$-correct.
Let $h$ be a $\PiOneOmega$ code for $A\subset\R^2$.
Let $x\in\R\intersect M$.
\begin{enumerate}
\item $\exists y A(x,y) \Iff T^h(x)$ is illfounded.
\item $M\models \exists y A(x,y) \Iff (T^h)^M(x)$ is illfounded.
\item $M\models \exists y A(x,y) \Implies \exists y A(x,y)$.
\item $\exists y A(x,y) \not\Implies M\models \exists y A(x,y)$.
\item $(y,f)\in[(T^h)^M] \Implies (y, \sigma^M[f]) \in [T^h]$.
\end{enumerate}
\end{remarks}

\begin{definition}
Let $M$ be a countable, transitive, projectively correct model of $\ZFC$.
Suppose $T\in M$ and $M\models ``T$ is a $\bPi^1_{\omega}$ tree.''
We say that $T$ is \emph{correctly wellfounded} for $M$  iff $T$ is wellfounded and
$T=(T^h(x))^M$ for some
$\PiOneOmega$ code $h$ and real $x\in M$, and $T^h(x)$ is wellfounded. (In this
case we say $T$ is correctly wellfounded for $M$ with respect to $h$ and $x$.)
We say that $T$ is \emph{incorrectly wellfounded} for $M$ iff $T$ is wellfounded and $T=(T^h(x))^M$ for some
$\PiOneOmega$ code $h$ and real $x\in M$, and $T^h(x)$ is illfounded. (In this
case we say $T$ is incorrectly wellfounded for $M$ with respect to $h$ and $x$.)

We also extend this definition to the case where $M$ itself is not a model of all of $\ZFC$ but
$M$ has a rank initial segment that is a model of $\ZFC$.
\end{definition}

The key to our proof that $\Mladder$ is quasi-$\SigmaOneOmegaPlusOne$-correct is this:
We will show that if $T$ is a $\bPi^1_{\omega}$-tree in $\Mladder$ then
$\Mladder$ is able to determine whether or not $T$ is correctly wellfounded.

\begin{theorem}
\label{CorrectBelowIncorrect}
Let $z$ be a real and $M$ a countable, iterable, ladder-mouse over $z$.
Suppose $M\models ``T_1$ and $T_2$ are $\bPi^1_{\omega}$-trees'' and
suppose $T_1$ is correctly wellfounded for $M$ and
$T_2$ is incorrectly wellfounded for $M$. Then $\rank(T_1) < rank(T_2)$.
\end{theorem}

Now, assuming Theorem \ref{CorrectBelowIncorrect}, we prove Theorem
\ref{QuasiCorrectness}.

\begin{proof}[proof of Theorem \ref{QuasiCorrectness}]
Let $z$ be a real and $M$ a countable, iterable, ladder-mouse over $z$.
Let $\theta$ be the minimum of the ranks of all
$\bPi^1_{\omega}$ trees in $M$ that are  incorrectly wellfounded for $M$.
If all wellfounded $\bPi^1_{\omega}$ trees in $M$ are correctly wellfounded
then $M$ is $\Sigma^1_{\omega+1}$-correct and in this case let $\theta=0$.

$M$ is a model of $\PD$ and also is a strongly acceptable model $\GCH$.
So $(\omega_1)^M<(\bdelta^1_{\omega})^M < (\omega_2)^M$.
So all $\bPi^1_{\omega}$ trees in $M$ are in $J^M_{(\omega_2)^M}$ and either have a branch
in $J^M_{(\omega_2)^M}$ or have rank less than $(\omega_2)^M$.
So $\theta < (\omega_2)^M$.


Let $x\in\R\intersect M$ and let $h$ be a
$\PiOneOmega$ code. Then
$$(\exists y) A^h(x,y) \Iff J^M_{(\omega_2)^M}\models T^h(x) \text{ is illfounded} \,\OR\, (\theta>0 \,\AND\, \rank(T^h(x)) \geq \theta.)$$
\end{proof}

Previously we proved Theorem \ref{DefinableRealsAreInM} based on
Theorem \ref{QuasiCorrectness}. But it is also possible to give a proof
of Theorem \ref{DefinableRealsAreInM} based on Theorem \ref{CorrectBelowIncorrect}
without going through Theorem \ref{DefinableRealsAreInM} and thus avoiding
mentioning the parameter $\theta$.

\begin{proof}[alternate proof of Theorem \ref{DefinableRealsAreInM}]
Let $x\in Q_{\omega+1}$. We need to show that $x\in\Mladder$.

For $i\in\omega$ let $\ibar=\angles{i,0,0,\cdots}\in\R$. Thus we are representing integers as reals.
This allows us to avoid having to define what we mean by the tree through $i$, $T(i)$, when $T$ is a tree and $i$ an integer.


Fix two $\PiOneOmega$ codes $h_1,h_2$ and a countable ordinal $\alpha$ such that for
all $w\in\WO$ with $|w|=\alpha$, and all $i,j\in\omega$,
$$x(i)=j \Iff (\exists y\in\R)A^{h_1}(\ibar,\jbar,w,y) \Iff (\forall y\in\R)\neg A^{h_2}(\ibar,\jbar,w,y).$$
Let $\Mprime$ be the $(\alpha+1)$-th iterate of $\Mladder$ by its least
measurable cardinal. So $\Mprime$ is a countable, iterable ladder mouse with
a ladder whose $\delta_0$ is greater than $\alpha$. Let $g$ be $\Mprime$-generic
for $\Coll(\omega,\alpha$). It suffices to see that $x\in\Mprime[g]$.

We can rearrange $\Mprime[g]$ as a mouse over $z$, where $z$ is some real coding $g$.
Let $M$ be this mouse over $z$. $M$ is a countable, iterable ladder-mouse over $z$ and there
is a real $w\in M\intersect\WO$ with $|w|=\alpha$.

Then for all
$i,j\in\omega$,
$$x(i)=j \Iff  M \models T^{h_1}(\ibar,\jbar,w) \text{ illfounded } \,\OR\, \rank(T^{h_2}(\ibar,\jbar,w)) < \rank(T^{h_1}(\ibar,\jbar,w)).$$
So $x\in M$.

\end{proof}

In the next section we turn to the proof of Theorem \ref{CorrectBelowIncorrect}.


\section{Something D}
\label{section:somethingd}

Suppose that $(S)^{\Mladder}$ is truly-wellfounded and $(T)^{\Mladder}$ is
wellfounded but not truly-wellfounded. We want to see that
$\rank((S)^{\Mladder}) < \rank((T)^{\Mladder})$.

Let $(y,\phi(y))$ be a branch of $T^V$.


We are going to arrange the following situation:

\begin{equation}
\label{big-setup}
\Mladder \xrightarrow{\pi_0} \Mprime \xrightarrow{\pi_1} \Mstar
\end{equation}

where

\begin{enumerate}
\item $\pi_0$ and $\pi_1$ are both $\Sigma_1$-elementary embeddings.
\item $\Mprime$ is a countable ladder mouse.
\item Letting $\alpha=(\omega_1)^{\Mprime}$ and $\delta=\ord(\Mprime)$ we have,
\item $\Mstar$ is is illfounded, $\delta$ is in the wellfounded part of $\Mstar$,
\item $\crit(\pi_1) = \alpha$ and $\pi_1(\alpha)=\delta$
\item $\Mprime \subset \Mstar$.
\item $(\bdelta^1_\omega)^{\Mstar}$ is in the wellfounded part of $\Mstar$.
\end{enumerate}

$(\pi_0, \Mprime)$ is going to come from a sequence of genericity iterations using
the initial segments of $\Mladder$ and $(\pi_1,\Mstar)$
is going to come from a stationary tower generic ultrapower of $\Mprime$.

Since $(S)^{\Mstar} \subset \omega^{<\omega}\times ((\bdelta^1_\omega)^{\Mstar})^{<\omega}$,
and
$(T)^{\Mstar} \subset \omega^{<\omega}\times ((\bdelta^1_\omega)^{\Mstar})^{<\omega}$
$(S)^{\Mstar}$ and $(T)^{\Mstar}$ are both in the wellfounded part of $\Mstar$.

By elementarity it suffices to show that $\Mstar \models \rank(S) < \rank(T)$.

Since $S^V$ is wellfounded, $S^{\Mstar}$ is wellfouned and so $(\rank(S))^{\Mstar}$ is in the
wellfounded part of $\Mstar$. We will show that $T^{\Mstar}$
is illfounded in $V$ and so $(\rank(T))^{\Mstar}$ is not in the wellfounded part of $\Mstar$.

We will show that for all $n$, $\phi_n(y)$ is in the range of $\theta^{\Mstar}$ and,
letting $f\in ((\bdelta^1_\omega)^{\Mstar})^{\omega}$ be such that for all $n$,
$\theta^{\Mstar}(f(n)) = \phi_n(y)$, we have that $(y,f)$ is a branch of $T^{\Mstar}$.

Thus it suffices to prove

\begin{lemma}
For all $n$, there is a $y_n\in \Mstar$ such that $y_n\restr n = y \restr n$ and
for all $i,j < n$, $\psi^i_j(y_n) = \psi^i_j(y)$.
\end{lemma}

\bibliographystyle{amsalpha}
\bibliography{math}

\end{document}
