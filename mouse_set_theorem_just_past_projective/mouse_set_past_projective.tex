\documentclass[oneside,12pt]{amsart}

\usepackage{amsmath,amssymb,latexsym,eucal,amsthm,rotating}
%\usepackage[shortlabels]{enumitem}

%%%%%%%%%%%%%%%%%%%%%%%%%%%%%%%%%%%%%%%%%%%%%
% Common Set Theory Constructs
%%%%%%%%%%%%%%%%%%%%%%%%%%%%%%%%%%%%%%%%%%%%%

\newcommand{\setof}[2]{\left\{ \, #1 \, \left| \, #2 \, \right.\right\}}
\newcommand{\lsetof}[2]{\left\{\left. \, #1 \, \right| \, #2 \,  \right\}}
\newcommand{\bigsetof}[2]{\bigl\{ \, #1 \, \bigm | \, #2 \,\bigr\}}
\newcommand{\Bigsetof}[2]{\Bigl\{ \, #1 \, \Bigm | \, #2 \,\Bigr\}}
\newcommand{\biggsetof}[2]{\biggl\{ \, #1 \, \biggm | \, #2 \,\biggr\}}
\newcommand{\Biggsetof}[2]{\Biggl\{ \, #1 \, \Biggm | \, #2 \,\Biggr\}}
\newcommand{\dotsetof}[2]{\left\{ \, #1 \, : \, #2 \, \right\}}
\newcommand{\bigdotsetof}[2]{\bigl\{ \, #1 \, : \, #2 \,\bigr\}}
\newcommand{\Bigdotsetof}[2]{\Bigl\{ \, #1 \, \Bigm : \, #2 \,\Bigr\}}
\newcommand{\biggdotsetof}[2]{\biggl\{ \, #1 \, \biggm : \, #2 \,\biggr\}}
\newcommand{\Biggdotsetof}[2]{\Biggl\{ \, #1 \, \Biggm : \, #2 \,\Biggr\}}
\newcommand{\sequence}[2]{\left\langle \, #1 \,\left| \, #2 \, \right. \right\rangle}
\newcommand{\lsequence}[2]{\left\langle\left. \, #1 \, \right| \,#2 \,  \right\rangle}
\newcommand{\bigsequence}[2]{\bigl\langle \,#1 \, \bigm | \, #2 \, \bigr\rangle}
\newcommand{\Bigsequence}[2]{\Bigl\langle \,#1 \, \Bigm | \, #2 \, \Bigr\rangle}
\newcommand{\biggsequence}[2]{\biggl\langle \,#1 \, \biggm | \, #2 \, \biggr\rangle}
\newcommand{\Biggsequence}[2]{\Biggl\langle \,#1 \, \Biggm | \, #2 \, \Biggr\rangle}
\newcommand{\singleton}[1]{\left\{#1\right\}}
\newcommand{\angles}[1]{\left\langle #1 \right\rangle}
\newcommand{\bigangles}[1]{\bigl\langle #1 \bigr\rangle}
\newcommand{\Bigangles}[1]{\Bigl\langle #1 \Bigr\rangle}
\newcommand{\biggangles}[1]{\biggl\langle #1 \biggr\rangle}
\newcommand{\Biggangles}[1]{\Biggl\langle #1 \Biggr\rangle}


\newcommand{\force}[1]{\Vert\!\underset{\!\!\!\!\!#1}{\!\!\!\relbar\!\!\!%
\relbar\!\!\relbar\!\!\relbar\!\!\!\relbar\!\!\relbar\!\!\relbar\!\!\!%
\relbar\!\!\relbar\!\!\relbar}}
\newcommand{\longforce}[1]{\Vert\!\underset{\!\!\!\!\!#1}{\!\!\!\relbar\!\!\!%
\relbar\!\!\relbar\!\!\relbar\!\!\!\relbar\!\!\relbar\!\!\relbar\!\!\!%
\relbar\!\!\relbar\!\!\relbar\!\!\relbar\!\!\relbar\!\!\relbar\!\!\relbar\!\!\relbar}}
\newcommand{\nforce}[1]{\Vert\!\underset{\!\!\!\!\!#1}{\!\!\!\relbar\!\!\!%
\relbar\!\!\relbar\!\!\relbar\!\!\!\relbar\!\!\relbar\!\!\relbar\!\!\!%
\relbar\!\!\not\relbar\!\!\relbar}}
\newcommand{\forcein}[2]{\overset{#2}{\Vert\underset{\!\!\!\!\!#1}%
{\!\!\!\relbar\!\!\!\relbar\!\!\relbar\!\!\relbar\!\!\!\relbar\!\!\relbar\!%
\!\relbar\!\!\!\relbar\!\!\relbar\!\!\relbar\!\!\relbar\!\!\!\relbar\!\!%
\relbar\!\!\relbar}}}

\newcommand{\pre}[2]{{}^{#2}{#1}}

\newcommand{\restr}{\!\!\upharpoonright\!}

%%%%%%%%%%%%%%%%%%%%%%%%%%%%%%%%%%%%%%%%%%%%%
% Set-Theoretic Connectives
%%%%%%%%%%%%%%%%%%%%%%%%%%%%%%%%%%%%%%%%%%%%%

\newcommand{\intersect}{\cap}
\newcommand{\union}{\cup}
\newcommand{\Intersection}[1]{\bigcap\limits_{#1}}
\newcommand{\Union}[1]{\bigcup\limits_{#1}}
\newcommand{\adjoin}{{}^\frown}
\newcommand{\forces}{\Vdash}

%%%%%%%%%%%%%%%%%%%%%%%%%%%%%%%%%%%%%%%%%%%%%
% Miscellaneous
%%%%%%%%%%%%%%%%%%%%%%%%%%%%%%%%%%%%%%%%%%%%%
\newcommand{\defeq}{=_{\text{def}}}
\newcommand{\Turingleq}{\leq_{\text{T}}}
\newcommand{\Turingless}{<_{\text{T}}}
\newcommand{\lexleq}{\leq_{\text{lex}}}
\newcommand{\lexless}{<_{\text{lex}}}
\newcommand{\Turingequiv}{\equiv_{\text{T}}}
\newcommand{\isomorphic}{\cong}

%%%%%%%%%%%%%%%%%%%%%%%%%%%%%%%%%%%%%%%%%%%%%
% Constants
%%%%%%%%%%%%%%%%%%%%%%%%%%%%%%%%%%%%%%%%%%%%%
\newcommand{\R}{\mathbb{R}}
\renewcommand{\P}{\mathbb{P}}
\newcommand{\Q}{\mathbb{Q}}
\newcommand{\Z}{\mathbb{Z}}
\newcommand{\Zpos}{\Z^{+}}
\newcommand{\Znonneg}{\Z^{\geq 0}}
\newcommand{\C}{\mathbb{C}}
\newcommand{\N}{\mathbb{N}}
\newcommand{\B}{\mathbb{B}}
\newcommand{\Bairespace}{\pre{\omega}{\omega}}
\newcommand{\LofR}{L(\R)}
\newcommand{\JofR}[1]{J_{#1}(\R)}
\newcommand{\SofR}[1]{S_{#1}(\R)}
\newcommand{\JalphaR}{\JofR{\alpha}}
\newcommand{\JbetaR}{\JofR{\beta}}
\newcommand{\JlambdaR}{\JofR{\lambda}}
\newcommand{\SalphaR}{\SofR{\alpha}}
\newcommand{\SbetaR}{\SofR{\beta}}
\newcommand{\Pkl}{\mathcal{P}_{\kappa}(\lambda)}
\DeclareMathOperator{\con}{con}
\DeclareMathOperator{\ORD}{OR}
\DeclareMathOperator{\Ord}{OR}
\DeclareMathOperator{\WO}{WO}
\DeclareMathOperator{\OD}{OD}
\DeclareMathOperator{\HOD}{HOD}
\DeclareMathOperator{\HC}{HC}
\DeclareMathOperator{\WF}{WF}
\DeclareMathOperator{\wfp}{wfp}
\DeclareMathOperator{\HF}{HF}
\newcommand{\One}{I}
\newcommand{\ONE}{I}
\newcommand{\Two}{II}
\newcommand{\TWO}{II}
\newcommand{\Mladder}{M^{\text{ld}}}

%%%%%%%%%%%%%%%%%%%%%%%%%%%%%%%%%%%%%%%%%%%%%
% Commutative Algebra Constants
%%%%%%%%%%%%%%%%%%%%%%%%%%%%%%%%%%%%%%%%%%%%%
\DeclareMathOperator{\dottimes}{\dot{\times}}
\DeclareMathOperator{\dotminus}{\dot{-}}
\DeclareMathOperator{\Spec}{Spec}

%%%%%%%%%%%%%%%%%%%%%%%%%%%%%%%%%%%%%%%%%%%%%
% Theories
%%%%%%%%%%%%%%%%%%%%%%%%%%%%%%%%%%%%%%%%%%%%%
\DeclareMathOperator{\ZFC}{ZFC}
\DeclareMathOperator{\ZF}{ZF}
\DeclareMathOperator{\AD}{AD}
\DeclareMathOperator{\ADR}{AD_{\R}}
\DeclareMathOperator{\KP}{KP}
\DeclareMathOperator{\PD}{PD}
\DeclareMathOperator{\CH}{CH}
\DeclareMathOperator{\GCH}{GCH}
\DeclareMathOperator{\HPC}{HPC} % HOD pair capturing
%%%%%%%%%%%%%%%%%%%%%%%%%%%%%%%%%%%%%%%%%%%%%
% Iteration Trees
%%%%%%%%%%%%%%%%%%%%%%%%%%%%%%%%%%%%%%%%%%%%%

\newcommand{\pred}{\text{-pred}}

%%%%%%%%%%%%%%%%%%%%%%%%%%%%%%%%%%%%%%%%%%%%%%%%
% Operator Names
%%%%%%%%%%%%%%%%%%%%%%%%%%%%%%%%%%%%%%%%%%%%%%%%
\DeclareMathOperator{\Det}{Det}
\DeclareMathOperator{\dom}{dom}
\DeclareMathOperator{\ran}{ran}
\DeclareMathOperator{\range}{ran}
\DeclareMathOperator{\image}{image}
\DeclareMathOperator{\crit}{crit}
\DeclareMathOperator{\card}{card}
\DeclareMathOperator{\cf}{cf}
\DeclareMathOperator{\cof}{cof}
\DeclareMathOperator{\rank}{rank}
\DeclareMathOperator{\ot}{o.t.}
\DeclareMathOperator{\ords}{o}
\DeclareMathOperator{\ro}{r.o.}
\DeclareMathOperator{\rud}{rud}
\DeclareMathOperator{\Powerset}{\mathcal{P}}
\DeclareMathOperator{\length}{lh}
\DeclareMathOperator{\lh}{lh}
\DeclareMathOperator{\limit}{lim}
\DeclareMathOperator{\fld}{fld}
\DeclareMathOperator{\projection}{p}
\DeclareMathOperator{\Ult}{Ult}
\DeclareMathOperator{\ULT}{Ult}
\DeclareMathOperator{\Coll}{Coll}
\DeclareMathOperator{\Col}{Col}
\DeclareMathOperator{\Hull}{Hull}
\DeclareMathOperator*{\dirlim}{dir lim}
\DeclareMathOperator{\Scale}{Scale}
\DeclareMathOperator{\supp}{supp}
\DeclareMathOperator{\trancl}{tran.cl.}
\DeclareMathOperator{\trace}{Tr}
\DeclareMathOperator{\diag}{diag}
\DeclareMathOperator{\spn}{span}
\DeclareMathOperator{\sgn}{sgn}
%%%%%%%%%%%%%%%%%%%%%%%%%%%%%%%%%%%%%%%%%%%%%
% Logical Connectives
%%%%%%%%%%%%%%%%%%%%%%%%%%%%%%%%%%%%%%%%%%%%%
\newcommand{\IImplies}{\Longrightarrow}
\newcommand{\SkipImplies}{\quad\Longrightarrow\quad}
\newcommand{\Ifff}{\Longleftrightarrow}
\newcommand{\iimplies}{\longrightarrow}
\newcommand{\ifff}{\longleftrightarrow}
\newcommand{\Implies}{\Rightarrow}
\newcommand{\Iff}{\Leftrightarrow}
\renewcommand{\implies}{\rightarrow}
\renewcommand{\iff}{\leftrightarrow}
\newcommand{\AND}{\wedge}
\newcommand{\OR}{\vee}
\newcommand{\st}{\text{ s.t. }}
\newcommand{\Or}{\text{ or }}

%%%%%%%%%%%%%%%%%%%%%%%%%%%%%%%%%%%%%%%%%%%%%
% Function Arrows
%%%%%%%%%%%%%%%%%%%%%%%%%%%%%%%%%%%%%%%%%%%%%

\newcommand{\injection}{\xrightarrow{\text{1-1}}}
\newcommand{\surjection}{\xrightarrow{\text{onto}}}
\newcommand{\bijection}{\xrightarrow[\text{onto}]{\text{1-1}}}
\newcommand{\cofmap}{\xrightarrow{\text{cof}}}
\newcommand{\map}{\rightarrow}

%%%%%%%%%%%%%%%%%%%%%%%%%%%%%%%%%%%%%%%%%%%%%
% Mouse Comparison Operators
%%%%%%%%%%%%%%%%%%%%%%%%%%%%%%%%%%%%%%%%%%%%%
\newcommand{\initseg}{\trianglelefteq}
\newcommand{\properseg}{\lhd}
\newcommand{\notinitseg}{\ntrianglelefteq}
\newcommand{\notproperseg}{\ntriangleleft}

%%%%%%%%%%%%%%%%%%%%%%%%%%%%%%%%%%%%%%%%%%%%%
%           calligraphic letters
%%%%%%%%%%%%%%%%%%%%%%%%%%%%%%%%%%%%%%%%%%%%%
\newcommand{\cA}{\mathcal{A}}
\newcommand{\cB}{\mathcal{B}}
\newcommand{\cC}{\mathcal{C}}
\newcommand{\cD}{\mathcal{D}}
\newcommand{\cE}{\mathcal{E}}
\newcommand{\cF}{\mathcal{F}}
\newcommand{\cG}{\mathcal{G}}
\newcommand{\cH}{\mathcal{H}}
\newcommand{\cI}{\mathcal{I}}
\newcommand{\cJ}{\mathcal{J}}
\newcommand{\cK}{\mathcal{K}}
\newcommand{\cL}{\mathcal{L}}
\newcommand{\cM}{\mathcal{M}}
\newcommand{\cN}{\mathcal{N}}
\newcommand{\cO}{\mathcal{O}}
\newcommand{\cP}{\mathcal{P}}
\newcommand{\cQ}{\mathcal{Q}}
\newcommand{\cR}{\mathcal{R}}
\newcommand{\cS}{\mathcal{S}}
\newcommand{\cT}{\mathcal{T}}
\newcommand{\cU}{\mathcal{U}}
\newcommand{\cV}{\mathcal{V}}
\newcommand{\cW}{\mathcal{W}}
\newcommand{\cX}{\mathcal{X}}
\newcommand{\cY}{\mathcal{Y}}
\newcommand{\cZ}{\mathcal{Z}}


%%%%%%%%%%%%%%%%%%%%%%%%%%%%%%%%%%%%%%%%%%%%%
%          Primed Letters
%%%%%%%%%%%%%%%%%%%%%%%%%%%%%%%%%%%%%%%%%%%%%
\newcommand{\aprime}{a^{\prime}}
\newcommand{\bprime}{b^{\prime}}
\newcommand{\cprime}{c^{\prime}}
\newcommand{\dprime}{d^{\prime}}
\newcommand{\eprime}{e^{\prime}}
\newcommand{\fprime}{f^{\prime}}
\newcommand{\gprime}{g^{\prime}}
\newcommand{\hprime}{h^{\prime}}
\newcommand{\iprime}{i^{\prime}}
\newcommand{\jprime}{j^{\prime}}
\newcommand{\kprime}{k^{\prime}}
\newcommand{\lprime}{l^{\prime}}
\newcommand{\mprime}{m^{\prime}}
\newcommand{\nprime}{n^{\prime}}
\newcommand{\oprime}{o^{\prime}}
\newcommand{\pprime}{p^{\prime}}
\newcommand{\qprime}{q^{\prime}}
\newcommand{\rprime}{r^{\prime}}
\newcommand{\sprime}{s^{\prime}}
\newcommand{\tprime}{t^{\prime}}
\newcommand{\uprime}{u^{\prime}}
\newcommand{\vprime}{v^{\prime}}
\newcommand{\wprime}{w^{\prime}}
\newcommand{\xprime}{x^{\prime}}
\newcommand{\yprime}{y^{\prime}}
\newcommand{\zprime}{z^{\prime}}
\newcommand{\Aprime}{A^{\prime}}
\newcommand{\Bprime}{B^{\prime}}
\newcommand{\Cprime}{C^{\prime}}
\newcommand{\Dprime}{D^{\prime}}
\newcommand{\Eprime}{E^{\prime}}
\newcommand{\Fprime}{F^{\prime}}
\newcommand{\Gprime}{G^{\prime}}
\newcommand{\Hprime}{H^{\prime}}
\newcommand{\Iprime}{I^{\prime}}
\newcommand{\Jprime}{J^{\prime}}
\newcommand{\Kprime}{K^{\prime}}
\newcommand{\Lprime}{L^{\prime}}
\newcommand{\Mprime}{M^{\prime}}
\newcommand{\Nprime}{N^{\prime}}
\newcommand{\Oprime}{O^{\prime}}
\newcommand{\Pprime}{P^{\prime}}
\newcommand{\Qprime}{Q^{\prime}}
\newcommand{\Rprime}{R^{\prime}}
\newcommand{\Sprime}{S^{\prime}}
\newcommand{\Tprime}{T^{\prime}}
\newcommand{\Uprime}{U^{\prime}}
\newcommand{\Vprime}{V^{\prime}}
\newcommand{\Wprime}{W^{\prime}}
\newcommand{\Xprime}{X^{\prime}}
\newcommand{\Yprime}{Y^{\prime}}
\newcommand{\Zprime}{Z^{\prime}}
\newcommand{\alphaprime}{\alpha^{\prime}}
\newcommand{\betaprime}{\beta^{\prime}}
\newcommand{\gammaprime}{\gamma^{\prime}}
\newcommand{\Gammaprime}{\Gamma^{\prime}}
\newcommand{\deltaprime}{\delta^{\prime}}
\newcommand{\epsilonprime}{\epsilon^{\prime}}
\newcommand{\kappaprime}{\kappa^{\prime}}
\newcommand{\lambdaprime}{\lambda^{\prime}}
\newcommand{\rhoprime}{\rho^{\prime}}
\newcommand{\Sigmaprime}{\Sigma^{\prime}}
\newcommand{\tauprime}{\tau^{\prime}}
\newcommand{\xiprime}{\xi^{\prime}}
\newcommand{\thetaprime}{\theta^{\prime}}
\newcommand{\Omegaprime}{\Omega^{\prime}}
\newcommand{\cMprime}{\cM^{\prime}}
\newcommand{\cNprime}{\cN^{\prime}}
\newcommand{\cPprime}{\cP^{\prime}}
\newcommand{\cQprime}{\cQ^{\prime}}
\newcommand{\cRprime}{\cR^{\prime}}
\newcommand{\cSprime}{\cS^{\prime}}
\newcommand{\cTprime}{\cT^{\prime}}

%%%%%%%%%%%%%%%%%%%%%%%%%%%%%%%%%%%%%%%%%%%%%
%          bar Letters
%%%%%%%%%%%%%%%%%%%%%%%%%%%%%%%%%%%%%%%%%%%%%
\newcommand{\abar}{\bar{a}}
\newcommand{\bbar}{\bar{b}}
\newcommand{\cbar}{\bar{c}}
\newcommand{\ibar}{\bar{i}}
\newcommand{\jbar}{\bar{j}}
\newcommand{\nbar}{\bar{n}}
\newcommand{\xbar}{\bar{x}}
\newcommand{\ybar}{\bar{y}}
\newcommand{\zbar}{\bar{z}}
\newcommand{\pibar}{\bar{\pi}}
\newcommand{\phibar}{\bar{\varphi}}
\newcommand{\psibar}{\bar{\psi}}
\newcommand{\thetabar}{\bar{\theta}}
\newcommand{\nubar}{\bar{\nu}}

%%%%%%%%%%%%%%%%%%%%%%%%%%%%%%%%%%%%%%%%%%%%%
%          star Letters
%%%%%%%%%%%%%%%%%%%%%%%%%%%%%%%%%%%%%%%%%%%%%
\newcommand{\phistar}{\phi^{*}}
\newcommand{\Mstar}{M^{*}}

%%%%%%%%%%%%%%%%%%%%%%%%%%%%%%%%%%%%%%%%%%%%%
%          dotletters Letters
%%%%%%%%%%%%%%%%%%%%%%%%%%%%%%%%%%%%%%%%%%%%%
\newcommand{\Gdot}{\dot{G}}

%%%%%%%%%%%%%%%%%%%%%%%%%%%%%%%%%%%%%%%%%%%%%
%         check Letters
%%%%%%%%%%%%%%%%%%%%%%%%%%%%%%%%%%%%%%%%%%%%%
\newcommand{\deltacheck}{\check{\delta}}
\newcommand{\gammacheck}{\check{\gamma}}


%%%%%%%%%%%%%%%%%%%%%%%%%%%%%%%%%%%%%%%%%%%%%
%          Formulas
%%%%%%%%%%%%%%%%%%%%%%%%%%%%%%%%%%%%%%%%%%%%%

\newcommand{\formulaphi}{\text{\large $\varphi$}}
\newcommand{\Formulaphi}{\text{\Large $\varphi$}}


%%%%%%%%%%%%%%%%%%%%%%%%%%%%%%%%%%%%%%%%%%%%%
%          Fraktur Letters
%%%%%%%%%%%%%%%%%%%%%%%%%%%%%%%%%%%%%%%%%%%%%

\newcommand{\fa}{\mathfrak{a}}
\newcommand{\fb}{\mathfrak{b}}
\newcommand{\fc}{\mathfrak{c}}
\newcommand{\fk}{\mathfrak{k}}
\newcommand{\fp}{\mathfrak{p}}
\newcommand{\fq}{\mathfrak{q}}
\newcommand{\fr}{\mathfrak{r}}
\newcommand{\fA}{\mathfrak{A}}
\newcommand{\fB}{\mathfrak{B}}
\newcommand{\fC}{\mathfrak{C}}
\newcommand{\fD}{\mathfrak{D}}

%%%%%%%%%%%%%%%%%%%%%%%%%%%%%%%%%%%%%%%%%%%%%
%          Bold Letters
%%%%%%%%%%%%%%%%%%%%%%%%%%%%%%%%%%%%%%%%%%%%%
\newcommand{\ba}{\mathbf{a}}
\newcommand{\bb}{\mathbf{b}}
\newcommand{\bc}{\mathbf{c}}
\newcommand{\bd}{\mathbf{d}}
\newcommand{\be}{\mathbf{e}}
\newcommand{\bu}{\mathbf{u}}
\newcommand{\bv}{\mathbf{v}}
\newcommand{\bw}{\mathbf{w}}
\newcommand{\bx}{\mathbf{x}}
\newcommand{\by}{\mathbf{y}}
\newcommand{\bz}{\mathbf{z}}
\newcommand{\bSigma}{\boldsymbol{\Sigma}}
\newcommand{\bPi}{\boldsymbol{\Pi}}
\newcommand{\bDelta}{\boldsymbol{\Delta}}
\newcommand{\bdelta}{\boldsymbol{\delta}}
\newcommand{\bgamma}{\boldsymbol{\gamma}}
\newcommand{\bGamma}{\boldsymbol{\Gamma}}

%%%%%%%%%%%%%%%%%%%%%%%%%%%%%%%%%%%%%%%%%%%%%
%         Bold numbers
%%%%%%%%%%%%%%%%%%%%%%%%%%%%%%%%%%%%%%%%%%%%%
\newcommand{\bzero}{\mathbf{0}}

%%%%%%%%%%%%%%%%%%%%%%%%%%%%%%%%%%%%%%%%%%%%%
% Projective-Like Pointclasses
%%%%%%%%%%%%%%%%%%%%%%%%%%%%%%%%%%%%%%%%%%%%%
\newcommand{\Sa}[2][\alpha]{\Sigma_{(#1,#2)}}
\newcommand{\Pa}[2][\alpha]{\Pi_{(#1,#2)}}
\newcommand{\Da}[2][\alpha]{\Delta_{(#1,#2)}}
\newcommand{\Aa}[2][\alpha]{A_{(#1,#2)}}
\newcommand{\Ca}[2][\alpha]{C_{(#1,#2)}}
\newcommand{\Qa}[2][\alpha]{Q_{(#1,#2)}}
\newcommand{\da}[2][\alpha]{\delta_{(#1,#2)}}
\newcommand{\leqa}[2][\alpha]{\leq_{(#1,#2)}}
\newcommand{\lessa}[2][\alpha]{<_{(#1,#2)}}
\newcommand{\equiva}[2][\alpha]{\equiv_{(#1,#2)}}


\newcommand{\Sl}[1]{\Sa[\lambda]{#1}}
\newcommand{\Pl}[1]{\Pa[\lambda]{#1}}
\newcommand{\Dl}[1]{\Da[\lambda]{#1}}
\newcommand{\Al}[1]{\Aa[\lambda]{#1}}
\newcommand{\Cl}[1]{\Ca[\lambda]{#1}}
\newcommand{\Ql}[1]{\Qa[\lambda]{#1}}

\newcommand{\San}{\Sa{n}}
\newcommand{\Pan}{\Pa{n}}
\newcommand{\Dan}{\Da{n}}
\newcommand{\Can}{\Ca{n}}
\newcommand{\Qan}{\Qa{n}}
\newcommand{\Aan}{\Aa{n}}
\newcommand{\dan}{\da{n}}
\newcommand{\leqan}{\leqa{n}}
\newcommand{\lessan}{\lessa{n}}
\newcommand{\equivan}{\equiva{n}}

\newcommand{\SigmaOneOmega}{\Sigma^1_{\omega}}
\newcommand{\SigmaOneOmegaPlusOne}{\Sigma^1_{\omega+1}}
\newcommand{\PiOneOmega}{\Pi^1_{\omega}}
\newcommand{\PiOneOmegaPlusOne}{\Pi^1_{\omega+1}}
\newcommand{\DeltaOneOmegaPlusOne}{\Delta^1_{\omega+1}}

%%%%%%%%%%%%%%%%%%%%%%%%%%%%%%%%%%%%%%%%%%%%%
% Linear Algebra
%%%%%%%%%%%%%%%%%%%%%%%%%%%%%%%%%%%%%%%%%%%%%
\newcommand{\transpose}[1]{{#1}^{\text{T}}}
\newcommand{\norm}[1]{\lVert{#1}\rVert}
\newcommand\aug{\fboxsep=-\fboxrule\!\!\!\fbox{\strut}\!\!\!}

%%%%%%%%%%%%%%%%%%%%%%%%%%%%%%%%%%%%%%%%%%%%%
% Number Theory
%%%%%%%%%%%%%%%%%%%%%%%%%%%%%%%%%%%%%%%%%%%%%
\newcommand{\av}[1]{\lvert#1\rvert}
\DeclareMathOperator{\divides}{\mid}
\DeclareMathOperator{\ndivides}{\nmid}
\DeclareMathOperator{\lcm}{lcm}
\DeclareMathOperator{\sign}{sign}
\newcommand{\floor}[1]{\left\lfloor{#1}\right\rfloor}
\DeclareMathOperator{\ConCl}{CC}
\DeclareMathOperator{\ord}{ord}


%%%%%%%%%%%%%%%%%%%%%%%%%%%%%%%%%%%%%%%%%%%%%%%%%%%%%%%%%%%%%%%%%%%%%%%%%%%
%%  Theorem-Like Declarations
%%%%%%%%%%%%%%%%%%%%%%%%%%%%%%%%%%%%%%%%%%%%%%%%%%%%%%%%%%%%%%%%%%%%%%%%%%

\newtheorem{theorem}{Theorem}[section]
\newtheorem{lemma}[theorem]{Lemma}
\newtheorem{corollary}[theorem]{Corollary}
\newtheorem{proposition}[theorem]{Proposition}


\theoremstyle{definition}

\newtheorem{definition}[theorem]{Definition}
\newtheorem{conjecture}[theorem]{Conjecture}
\newtheorem{remark}[theorem]{Remark}
\newtheorem{remarks}[theorem]{Remarks}
\newtheorem{notation}[theorem]{Notation}

\theoremstyle{remark}

\newtheorem*{note}{Note}
\newtheorem*{warning}{Warning}
\newtheorem*{question}{Question}
\newtheorem*{example}{Example}
\newtheorem*{fact}{Fact}


\newenvironment*{subproof}[1][Proof]
{\begin{proof}[#1]}{\renewcommand{\qedsymbol}{$\diamondsuit$} \end{proof}}

\newenvironment*{case}[1]
{\textbf{Case #1.  }\itshape }{}

\newenvironment*{claim}[1][Claim]
{\textbf{#1.  }\itshape }{}


%\pagestyle{plain}

\begin{document}

\title{The Mouse Set Theorem Just Past Projective}
\author{Mitch Rudominer}
\author{John R. Steel}
\author{W. Hugh Woodin}


\keywords{large cardinals, descriptive set theory, inner model theory}

\begin{abstract}
We will describe a natural mouse $M$ and show that $\R\intersect M = Q_{\omega+1}$
where $Q_{\omega+1}$  is the set of reals that are
$\Delta^1_{\omega+1}$ in a countable ordinal. Thus $Q_{\omega+1}$
is a mouse set.

This is analogous to the fact that $\R\intersect M_1 = Q_3$ where $M_1$ is the
canonical inner model with a Woodin cardinal, and $Q_3$ is the set of reals
that are $\Delta^1_3$ in a countable ordinal.

More generally $\R\intersect M_{2n+1} = Q_{2n+3}$.
The pointclass $\Delta_{\omega+1}$  we consider in this paper is the next natural
pointclass to consider in this series of results. It is the analogue of
the $\Delta^1_{2n+3}$ just past projective. Thus we are proving the mouse
set theorem just past projective.
\end{abstract}

\maketitle

\tableofcontents

\section{Introduction}
\label{section:intro}

We work in $\ZFC+\AD(\LofR)$ for convenience. We need much less determinacy
than this, just a bit past $\PD$.

Throughout this paper we write $\R$ to mean $\Bairespace$ and we call elements of $\Bairespace$ \emph{reals}.

\section{Project-like Pointclasses Just Past Projective}
\label{section:projectlikepointclasses}

\begin{definition}
Let $\PiOneOmega$ be the pointclass of recursive intersections of infinitely many (lightface) projective sets.

In more detail, fix for the remainder of this paper, for $n\in\omega$, $G^n\subset\omega\times\R$
a universal $\Pi^1_{2n+1}$ set, uniformly in $n$.
A $\PiOneOmega$ \emph{code} is a total recursive function $h:\omega\to\omega$. If $h$ is a $\PiOneOmega$ code then
$A^h=\setof{x\in\R}{(\forall n)\, G^n(h(n),x)}$. Say $A\subseteq\R$ is $\PiOneOmega$ iff
$A=A^h$ for some $\PiOneOmega$ code $h$.

Similarly we define $\PiOneOmega$ subsets of $\omega^s\times \R^t$ for $s,t\in\omega$.

Then we define $\SigmaOneOmega = \neg\PiOneOmega$, $\SigmaOneOmegaPlusOne=\exists^{\R}\PiOneOmega$,
$\PiOneOmegaPlusOne = \forall^{\R}\SigmaOneOmega$ and
$\DeltaOneOmegaPlusOne = \SigmaOneOmegaPlusOne \intersect \PiOneOmegaPlusOne$.

Finally we define as usual the relativized and bold-face
pointclasses $\PiOneOmega(x)$
and $\bPi^1_{\omega}$ etc.
\end{definition}

The following remarks follow from \cite{Scales_In_LofR}.
\begin{remarks} \
\begin{itemize}
\item $\SigmaOneOmega = \Sigma_1(\JofR{2}) \intersect \Powerset(\R)$
\item $\PiOneOmega = \Pi_1(\JofR{2}) \intersect \Powerset(\R)$
\item $\SigmaOneOmegaPlusOne = \Sigma_2(\JofR{2}) \intersect \Powerset(\R)$
\item $\PiOneOmegaPlusOne = \Pi_2(\JofR{2}) \intersect \Powerset(\R)$
\item $\SigmaOneOmega$ and $\PiOneOmegaPlusOne$
are scaled pointclasses.
(Recall we are assuming $\AD(\LofR)$. $\Det(\JofR{3})$ suffices here. )
\end{itemize}
\end{remarks}


\begin{definition}
We say that $x\in\R$ is $\DeltaOneOmegaPlusOne$ in a countable ordinal iff there is an $\alpha<\omega_1$ such
that for all $w\in\WO$ with $|w|=\alpha$, $\singleton{x}$ is $\DeltaOneOmegaPlusOne(w)$.
Let $Q_{\omega+1} = \setof{x\in\R}{x\text{ is }\DeltaOneOmegaPlusOne\text{ in a countable ordinal}}$.
\end{definition}

By way of intuition, we consider $\PiOneOmegaPlusOne$ to be analogous to
$\Pi^1_1$ or $\Pi^1_3$ and we consider $Q_{\omega+1}$ to be analogous to
$Q_1$ or $Q_3$. See \cite{Q_Theory}.

Recall that $Q_3 = \R \intersect M_1^{\sharp}$. In the next section we will
define a mouse $\Mladder$ such that $Q_{\omega+1} = \R\intersect\Mladder$.

\begin{remarks} \
\begin{enumerate}
\item $x\in Q_{\omega+1}$ iff there is a $\SigmaOneOmegaPlusOne$ relation $R$ and
a countable ordinal $\alpha$ such that for all $w\in\WO$ with $|w|=\alpha$,
and for all reals $y$, $R(w,y) \Iff y=x$.
\item $x\in Q_{\omega+1}$ iff there is a $\SigmaOneOmegaPlusOne$ relation $R$
and a countable ordinal $\alpha$ such
that for all $w\in\WO$ with $|w|=\alpha$, and all $i,j\in\omega$,
$x(i)=j \Iff R(w,i,j)$.
\item $Q_{\omega+1}$ is the largest countable
$\DeltaOneOmegaPlusOne$ set closed downwards under
$\DeltaOneOmegaPlusOne$-degrees.
\end{enumerate}
\end{remarks}


\section{Ladder Mice}
\label{section:laddermice}
\begin{definition}
Let $M$ be a premouse, in the sense of \cite{FSIT}. A \emph{ladder over $M$}
is a sequence of ordinals $\sequence{\delta_n,\gamma_n}{n\in\omega}$ such that
for all $n$:
\begin{enumerate}
\item $\delta_n < \gamma_n < \delta_{n+1} < \ord(M)$,
\item $\delta_n$ is a cardinal of $M$,
\item $\delta_n$ is Woodin in $J^M_{\gamma_n}$,
\item $\gamma_n$ is the least $\gamma$ such that $\cJ^M_{\gamma}$ is active and
$\cJ^M_{\gamma}\models$ there are $n$ Woodin cardinals greater than $\delta_n$.
\end{enumerate}

A ladder $\sequence{\delta_n,\gamma_n}{n\in\omega}$ over $M$ is \emph{cofinal}
iff the $\gamma_n$ are cofinal in $\ord(M)$.

$M$ is a \emph{ladder mouse} iff there is a ladder over $M$ and a
\emph{cofinal ladder mouse} iff there is a cofinal ladder over $M$.

$M$ is a \emph{minimal ladder mouse} iff $M$ is a ladder mouse but no initial
segment of $M$ is.

$\Mladder$ is the least fully-iterable ladder mouse, if it exists. If
$\Mladder$ exists it is obviously a minimal ladder mouse (and so a cofinal
ladder mouse), and $\Mladder$ projects to $\omega$ and so it is an $\omega$-mouse
in the sense of \cite{Proj_WO_In_Mod}.
\end{definition}

\begin{remarks} \
\begin{enumerate}
\item Because we are assuming $\AD(\LofR)$, $\Mladder$ exists.
\item For every $n\in\omega$, $\Mladder$ has a rank initial segment that satisfies
$\ZFC$+$\exists n$ Woodin cardinals.  By \cite{Proj_WO_In_Mod}, $\Mladder$ is projectively correct and so $\PiOneOmega$-correct.
\item Let $\sequence{\delta_n,\gamma_n}{n\in\omega}$ be a ladder over $\Mladder$.
For even $n$, $J^{\Mladder}_{\gamma_n}[g]$ is $\Sigma^1_{n+2}$-correct,
where $g$ is $J^{\Mladder}_{\gamma_n}$-generic over $\Coll(\omega,\delta_n)$.
\item Let $\sequence{\delta_n,\gamma_n}{n\in\omega}$ be a ladder over $M$.
$\delta_n$ is not necessarily fully Woodin in $M$, but $\delta_n$ is Woodin in $M$
with respect to functions in $J^M_{\gamma_n}$.
\item By the previous two items, we can, at lest informally, think of
$\Mladder$ as the least mouse
$M$ such that for every $n\in\omega$ there is a cardinal $\delta$ of $M$ such
that $\delta$ is ``$\Sigma^1_n$-Woodin'' in $M$.
\end{enumerate}
\end{remarks}

We can now state the main theorem of the paper:

\begin{theorem}
$Q_{\omega+1} = \R \intersect \Mladder$.
\end{theorem}

We divide the above theorem up into its two directions:

\begin{theorem}
\label{MRealsAreDefinable}
$\R \intersect \Mladder \subseteq Q_{\omega+1}$.
\end{theorem}

\begin{theorem}
\label{DefinableRealsAreInM}
$Q_{\omega+1} \subseteq \Mladder$.
\end{theorem}

Theorem \ref{MRealsAreDefinable} was proven more than 25 years ago in
\cite{My_Thesis} and \cite{Mouse_Sets}. We give a sketch of the proof later
in section \ref{ProofThatMRealsAreDefinable}. The proof follows the same
line of reasoning as the proof that every real in $L$ is $\Delta^1_2$ in a
countable ordinal and the proof that every real in $M_1$ is $\Delta^1_3$ in
a countable ordinal. Namely we show that every initial segment of $\Mladder$
that projects to $\omega$ is $\PiOneOmega$ definable from its ordinal height.

But 25 years ago the first author was not able to prove Theorem \ref{DefinableRealsAreInM}.
That was proven in the summer of 2019 by Woodin, and giving that proof is the
main goal of this paper.

Theorem \ref{DefinableRealsAreInM} will follow from a quasi-correctness theorem
for $\Mladder$. $\Mladder$ is $\PiOneOmega$-correct but not $\SigmaOneOmegaPlusOne$-correct.
Let $A\subset\R^2$ be $\PiOneOmega$ and let $x\in\R\intersect\Mladder$. It may
happen that there is a $y\in \R$ such that $A(x,y)$ but there is no such $y$
in $\Mladder$. But we will show that even if $\Mladder\not\models(\exists y) A(x,y)$,
$\Mladder$ can still determine whether or not the statement
``$(\exists y) A(x,y)$'' is true. This is similar to the situation with $M_1$ and
$\Sigma^1_3$. $M_1$ is not $\Sigma^1_3$-correct, but $M_1$ can still tell whether
or not $\Sigma^1_3$ statements are true. Let $A\subset\R^2$ be $\Pi^1_2$ and
let $x\in M_1$. Then $(\exists y\in\R)\, A(x,y)$ iff
$$M_1\models 1\force{\Coll(\omega,\delta)} (\exists y) A(x,y),$$
where $\delta$ is the Woodin of $M_1$.

Next we state the
$\SigmaOneOmegaPlusOne$-quasi-correctness for $\Mladder$ precisely and
we show how
Theorem \ref{DefinableRealsAreInM} follows from it.

If $z$ is a real, then by a \emph{mouse over $z$} we mean a mouse in the
sense of the theory of \cite{FSIT} augmented to use $z$ as an additional predicate.

\begin{theorem}
\label{QuasiCorrectness}
There is a formula $\psi$ in the language of Set Theory such that, for all
reals $z$, for all $M$ a countable, iterable, ladder-mouse over $z$,
for all reals $x$ in $M$,
for all $\PiOneOmega$ codes $h$,
$$(\exists y) A^h(x,y) \Iff J^M_{\omega_2^M} \models \psi[h,x].$$
\end{theorem}

Now, assuming Theorem \ref{QuasiCorrectness}, we prove Theorem \ref{DefinableRealsAreInM}.

\begin{proof}[proof of Theorem \ref{DefinableRealsAreInM}]
Let $x\in Q_{\omega+1}$. We need to show that $x\in\Mladder$.
Fix a $\PiOneOmega$ code $h$ and a countable ordinal $\alpha$ such that for
all $w\in\WO$ with $|w|=\alpha$, and all $i,j\in\omega$,
$$x(i)=j \Iff (\exists y\in\R)A^h(i,j,w,y).$$
Let $\Mprime$ be the $(\alpha+1)$-th iterate of $\Mladder$ by its least
measurable cardinal. So $\Mprime$ is a countable, iterable ladder mouse with
a ladder whose $\delta_0$ is greater than $\alpha$. Let $g$ be $\Mprime$-generic
for $\Coll(\omega,\alpha$). It suffices to see that $x\in\Mprime[g]$.

We can rearrange $\Mprime[g]$ as a mouse over $z$, where $z$ is some real coding $g$.
Let $M$ be this mouse over $z$. $M$ is a countable, iterable ladder-mouse over $z$ and there
is a real $w\in M\intersect\WO$ with $|w|=\alpha$.

Let $\psi$ be the formula given by Theorem \ref{QuasiCorrectness}. Then for all
$i,j\in\omega$,
$$x(i)=j \Iff (\exists y\in\R)A^h(i,j,w,y) \Iff J^M_{\omega_2^M} \models \psi[h,i,j,w].$$
So $x\in M$.
\end{proof}


We now develop the ideas that will allow us to prove theorem \ref{QuasiCorrectness}. We
start in the next section with a Suslin representation for $\PiOneOmega$ sets.

\section{A Suslin Representation for $\PiOneOmega$}
\label{section:suslinrep}

Let $\bdelta^1_{\omega} = \sup_n \bdelta^1_n$. In this section we show how to
express a $\PiOneOmega$ set as the projection
of a tree on $\omega\times \bdelta^1_{\omega}$.

\begin{definition}
Let $h$ be a $\PiOneOmega$ code for a subset of $\R$.
There is a natural tree $T^{h}$ on $\omega\times\bdelta^1_{\omega}$ that projects to $A^h$.

Recall that $G^n\subset\omega\times\R$ is a universal $\Pi^1_{2n+1}$ set, uniformly in $n$.

For $n,e\in\omega$, let $G^n_e = \setof{x\in\R}{G^n(e, x)}$ and
let $\varphi^n_e = \sequence{\varphi^n_{e,i}}{i\in\omega}$ be a $\Pi^1_{2n+1}$ scale on $G^n_e$ uniformly in $n$ and $e$,
with each of the norms $\varphi^n_{e,i}$ being regular.
Let $<^n_{e,i}$ be the prewellorder on $\R$ associated with $\varphi^n_{e,i}$. So ``$x,y\in G^n_e$ and $x <^n_{e,i} y$'' is
$\Pi^1_{2n+1}$, uniformly in $n,e,i$, and $\varphi^n_{e,i}(x) = $ the rank of $x$ in $<^n_{e,i}$.
$\varphi^n_{e,i} : G^n_e \map \bdelta^1_{2n+1}$.

Let $T^n_e$ be the tree for the scale $\varphi^n_e$. So
$$[T^n_e] = \setof{\left(x,\sequence{\varphi^n_{e,i}(x)}{i\in\omega}\right)}{x\in G^n_e},$$
$T^n_e$ is the set of all initial segments of its branches, and $G^n_e=p[T^n_e]$.

Fix some recursive coding of pairs of integers by integers, $i\to\angles{i_0,i_1}$,
with the property that $i_0,i_1\leq i$ for $i\in\omega$.

Now fix a $\PiOneOmega$ code $h$ and we will define the tree $T^h$.
Let $(s,u) \in \omega^n\times(\bdelta^1_{\omega})^n$ for some $n\in\omega$. Then $(s,u)\in T^h$ iff there is an
$x\in\R$ extending $s$ such that for all $i<n$, $x\in G^i_{h(i)}$ and $u(i)=\varphi^{i_0}_{h(i_0),i_1}(x)$.

\begin{remark}
$p[T^h] = A^h$.
\end{remark}
\begin{proof}
Let $(x,f)\in[T^h]$ and fix $n > k > m\in\omega$ such that $\angles{m,i} < n$ for all $i<k$.
Since $(x\restr n, f\restr n) \in T^h$,
there is an $\xprime\in\R$ such that $\xprime\restr n = x\restr n$, $\xprime\in G^m_{h(m)}$,
and for $i<k$, $f(\angles{m,i}) = \varphi^{m}_{h(m),i}(\xprime)$.
So $\left(x\restr k, \sequence{f(\angles{m,i})}{i<k}\right)
=\left(\xprime\restr k, \sequence{\varphi^m_{h(m),i}(\xprime)}{i<k}\right)\in T^m_{h(m)}$.
Since $k>m$ was arbitrary, $x\in p[T^m_{h(m)}] = G^m_{h(m)}$.
Since $m\in\omega$ was arbitrary, $x\in A^h = \Intersection{m} G^m_{h(m)}$.

For $n > m$ let $x_n$ extending $x\restr n$ witness that
$(x\restr n, f\restr n) \in T^h$. Then each $x_n\in G^m_{h(m)}$, and $x_n\rightarrow x_m$ and


Then $(x,f) \in \R \times \pre{(\bdelta^1_{\omega})}{\omega}$
and for all $n,k\in\omega$, $x\in G^n_{h(n)}$ and $f(\angles{n,k}) = \varphi^n_{h(n),k}(x)$.
\end{proof}




Say that $T$ is a $\PiOneOmega$ tree iff $T=T^h$ for some $\PiOneOmega$ code $h$.
\end{definition}

The properties of $T^h$ that we need are
\begin{enumerate}
\item $T^h$ is a tree on $\omega\times\bdelta^1_{\omega}$ that projects to $A^h$
\item The first $n$ levels of $T^h$ are simply definable from $\sequence{G^i_{h(i)}}{i<n}$ and
the norms $\sequence{\varphi^i_{h(i), j}}{i,j<n}$, all of which are $\Pi^1_{2n+1}$.
\end{enumerate}

Let $M$ be a countable transitive model of $\ZFC$ that is $\Pi^1_{2n+1}$-correct.
Let $A\subset\R$ be $\Pi^1_{2n+1}$, and let $\varphi:A\map\bdelta^1_{2n+1}$ be a regular
$\Pi^1_{2n+1}$ norm on $A$ with associated pre-wellorder $<_\varphi$.
Then there is a natural way to interpret $\varphi$ in $M$.
$\varphi^M:A\intersect M \map (\bdelta^1_{2n+1})^M$
is given by $\varphi^M(x) = $ the rank of $x$ in $<_\varphi \intersect M$.

\begin{definition}
In the situation described in the previous paragraph, we define
$\sigma^M_{\varphi}:\range(\varphi^M)\map\bdelta^1_{2n+1}$ by
$\sigma^M_{\varphi}(\varphi^M(x))=\varphi(x)$ for all $x\in A \intersect M$.
This is well-defined and order preserving since for $x,y\in A\intersect M$,
$\varphi^M(x)=\varphi^M(y)$ iff $\varphi(x)=\varphi(y)$ and
$\varphi^M(x)<\varphi^M(y)$ iff $\varphi(x)<\varphi(y)$.
\end{definition}

Now let $M$ be a countable transitive model of $\ZFC$ that is projectively correct.
Let $h$ be a $\PiOneOmega$ code. We want to study the relationship between
$(T^h)^M$ and $T^h$.

\section{Something D}
\label{section:somethingd}

Suppose that $(S)^{\Mladder}$ is truly-wellfounded and $(T)^{\Mladder}$ is
wellfounded but not truly-wellfounded. We want to see that
$\rank((S)^{\Mladder}) < \rank((T)^{\Mladder})$.

Let $(y,\phi(y))$ be a branch of $T^V$.


We are going to arrange the following situation:

\begin{equation}
\label{big-setup}
\Mladder \xrightarrow{\pi_0} \Mprime \xrightarrow{\pi_1} \Mstar
\end{equation}

where

\begin{enumerate}
\item $\pi_0$ and $\pi_1$ are both $\Sigma_1$-elementary embeddings.
\item $\Mprime$ is a countable ladder mouse.
\item Letting $\alpha=(\omega_1)^{\Mprime}$ and $\delta=\ord(\Mprime)$ we have,
\item $\Mstar$ is is illfounded, $\delta$ is in the wellfounded part of $\Mstar$,
\item $\crit(\pi_1) = \alpha$ and $\pi_1(\alpha)=\delta$
\item $\Mprime \subset \Mstar$.
\item $(\bdelta^1_\omega)^{\Mstar}$ is in the wellfounded part of $\Mstar$.
\end{enumerate}

$(\pi_0, \Mprime)$ is going to come from a sequence of genericity iterations using
the initial segments of $\Mladder$ and $(\pi_1,\Mstar)$
is going to come from a stationary tower generic ultrapower of $\Mprime$.

Since $(S)^{\Mstar} \subset \omega^{<\omega}\times ((\bdelta^1_\omega)^{\Mstar})^{<\omega}$,
and
$(T)^{\Mstar} \subset \omega^{<\omega}\times ((\bdelta^1_\omega)^{\Mstar})^{<\omega}$
$(S)^{\Mstar}$ and $(T)^{\Mstar}$ are both in the wellfounded part of $\Mstar$.

By elementarity it suffices to show that $\Mstar \models \rank(S) < \rank(T)$.

Since $S^V$ is wellfounded, $S^{\Mstar}$ is wellfouned and so $(\rank(S))^{\Mstar}$ is in the
wellfounded part of $\Mstar$. We will show that $T^{\Mstar}$
is illfounded in $V$ and so $(\rank(T))^{\Mstar}$ is not in the wellfounded part of $\Mstar$.

We will show that for all $n$, $\phi_n(y)$ is in the range of $\theta^{\Mstar}$ and,
letting $f\in ((\bdelta^1_\omega)^{\Mstar})^{\omega}$ be such that for all $n$,
$\theta^{\Mstar}(f(n)) = \phi_n(y)$, we have that $(y,f)$ is a branch of $T^{\Mstar}$.

Thus it suffices to prove

\begin{lemma}
For all $n$, there is a $y_n\in \Mstar$ such that $y_n\restr n = y \restr n$ and
for all $i,j < n$, $\psi^i_j(y_n) = \psi^i_j(y)$.
\end{lemma}

\bibliographystyle{amsalpha}
\bibliography{math}

\end{document}
