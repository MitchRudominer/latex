\documentclass[oneside,12pt]{amsart}

\usepackage{amsmath,amssymb,latexsym,eucal,amsthm,rotating,enumitem}
%\usepackage[shortlabels]{enumitem}

%%%%%%%%%%%%%%%%%%%%%%%%%%%%%%%%%%%%%%%%%%%%%
% Common Set Theory Constructs
%%%%%%%%%%%%%%%%%%%%%%%%%%%%%%%%%%%%%%%%%%%%%

\newcommand{\setof}[2]{\left\{ \, #1 \, \left| \, #2 \, \right.\right\}}
\newcommand{\lsetof}[2]{\left\{\left. \, #1 \, \right| \, #2 \,  \right\}}
\newcommand{\bigsetof}[2]{\bigl\{ \, #1 \, \bigm | \, #2 \,\bigr\}}
\newcommand{\Bigsetof}[2]{\Bigl\{ \, #1 \, \Bigm | \, #2 \,\Bigr\}}
\newcommand{\biggsetof}[2]{\biggl\{ \, #1 \, \biggm | \, #2 \,\biggr\}}
\newcommand{\Biggsetof}[2]{\Biggl\{ \, #1 \, \Biggm | \, #2 \,\Biggr\}}
\newcommand{\dotsetof}[2]{\left\{ \, #1 \, : \, #2 \, \right\}}
\newcommand{\bigdotsetof}[2]{\bigl\{ \, #1 \, : \, #2 \,\bigr\}}
\newcommand{\Bigdotsetof}[2]{\Bigl\{ \, #1 \, \Bigm : \, #2 \,\Bigr\}}
\newcommand{\biggdotsetof}[2]{\biggl\{ \, #1 \, \biggm : \, #2 \,\biggr\}}
\newcommand{\Biggdotsetof}[2]{\Biggl\{ \, #1 \, \Biggm : \, #2 \,\Biggr\}}
\newcommand{\sequence}[2]{\left\langle \, #1 \,\left| \, #2 \, \right. \right\rangle}
\newcommand{\lsequence}[2]{\left\langle\left. \, #1 \, \right| \,#2 \,  \right\rangle}
\newcommand{\bigsequence}[2]{\bigl\langle \,#1 \, \bigm | \, #2 \, \bigr\rangle}
\newcommand{\Bigsequence}[2]{\Bigl\langle \,#1 \, \Bigm | \, #2 \, \Bigr\rangle}
\newcommand{\biggsequence}[2]{\biggl\langle \,#1 \, \biggm | \, #2 \, \biggr\rangle}
\newcommand{\Biggsequence}[2]{\Biggl\langle \,#1 \, \Biggm | \, #2 \, \Biggr\rangle}
\newcommand{\singleton}[1]{\left\{#1\right\}}
\newcommand{\angles}[1]{\left\langle #1 \right\rangle}
\newcommand{\bigangles}[1]{\bigl\langle #1 \bigr\rangle}
\newcommand{\Bigangles}[1]{\Bigl\langle #1 \Bigr\rangle}
\newcommand{\biggangles}[1]{\biggl\langle #1 \biggr\rangle}
\newcommand{\Biggangles}[1]{\Biggl\langle #1 \Biggr\rangle}


\newcommand{\force}[1]{\Vert\!\underset{\!\!\!\!\!#1}{\!\!\!\relbar\!\!\!%
\relbar\!\!\relbar\!\!\relbar\!\!\!\relbar\!\!\relbar\!\!\relbar\!\!\!%
\relbar\!\!\relbar\!\!\relbar}}
\newcommand{\nforce}[1]{\Vert\!\underset{\!\!\!\!\!#1}{\!\!\!\relbar\!\!\!%
\relbar\!\!\relbar\!\!\relbar\!\!\!\relbar\!\!\relbar\!\!\relbar\!\!\!%
\relbar\!\!\not\relbar\!\!\relbar}}
\newcommand{\forcein}[2]{\overset{#2}{\Vert\underset{\!\!\!\!\!#1}%
{\!\!\!\relbar\!\!\!\relbar\!\!\relbar\!\!\relbar\!\!\!\relbar\!\!\relbar\!%
\!\relbar\!\!\!\relbar\!\!\relbar\!\!\relbar\!\!\relbar\!\!\!\relbar\!\!%
\relbar\!\!\relbar}}}

\newcommand{\pre}[2]{{}^{#2}\!{#1}}

\newcommand{\restr}{\!\!\upharpoonright\!}

%%%%%%%%%%%%%%%%%%%%%%%%%%%%%%%%%%%%%%%%%%%%%
% Set-Theoretic Connectives
%%%%%%%%%%%%%%%%%%%%%%%%%%%%%%%%%%%%%%%%%%%%%

\newcommand{\intersect}{\cap}
\newcommand{\union}{\cup}
\newcommand{\Intersection}[1]{\bigcap\limits_{#1}}
\newcommand{\Union}[1]{\bigcup\limits_{#1}}
\newcommand{\adjoin}{{}^\frown}
\newcommand{\forces}{\Vdash}

%%%%%%%%%%%%%%%%%%%%%%%%%%%%%%%%%%%%%%%%%%%%%
% Miscellaneous
%%%%%%%%%%%%%%%%%%%%%%%%%%%%%%%%%%%%%%%%%%%%%
\newcommand{\defeq}{=_{\text{def}}}
\newcommand{\Turingleq}{\leq_{\text{T}}}
\newcommand{\Turingless}{<_{\text{T}}}
\newcommand{\lexleq}{\leq_{\text{lex}}}
\newcommand{\lexless}{<_{\text{lex}}}
\newcommand{\Turingequiv}{\equiv_{\text{T}}}

%%%%%%%%%%%%%%%%%%%%%%%%%%%%%%%%%%%%%%%%%%%%%
% Constants
%%%%%%%%%%%%%%%%%%%%%%%%%%%%%%%%%%%%%%%%%%%%%
\newcommand{\R}{\mathbb{R}}
\renewcommand{\P}{\mathbb{P}}
\newcommand{\Q}{\mathbb{Q}}
\newcommand{\Z}{\mathbb{Z}}
\newcommand{\C}{\mathbb{C}}
\newcommand{\N}{\mathbb{N}}
\newcommand{\B}{\mathbb{B}}
\newcommand{\LofR}{L(\R)}
\newcommand{\JofR}[1]{J_{#1}(\R)}
\newcommand{\SofR}[1]{S_{#1}(\R)}
\newcommand{\JalphaR}{\JofR{\alpha}}
\newcommand{\JbetaR}{\JofR{\beta}}
\newcommand{\JlambdaR}{\JofR{\lambda}}
\newcommand{\SalphaR}{\SofR{\alpha}}
\newcommand{\SbetaR}{\SofR{\beta}}
\newcommand{\Pkl}{\mathcal{P}_{\kappa}(\lambda)}
\DeclareMathOperator{\con}{con}
\DeclareMathOperator{\ORD}{OR}
\DeclareMathOperator{\Ord}{OR}
\DeclareMathOperator{\WO}{WO}
\DeclareMathOperator{\OD}{OD}
\DeclareMathOperator{\HOD}{HOD}
\DeclareMathOperator{\HC}{HC}
\DeclareMathOperator{\WF}{WF}
\DeclareMathOperator{\HF}{HF}
\newcommand{\One}{I}
\newcommand{\ONE}{I}
\newcommand{\Two}{II}
\newcommand{\TWO}{II}

%%%%%%%%%%%%%%%%%%%%%%%%%%%%%%%%%%%%%%%%%%%%%
% Commutative Algebra Constants
%%%%%%%%%%%%%%%%%%%%%%%%%%%%%%%%%%%%%%%%%%%%%
\DeclareMathOperator{\dottimes}{\dot{\times}}

%%%%%%%%%%%%%%%%%%%%%%%%%%%%%%%%%%%%%%%%%%%%%
% Theories
%%%%%%%%%%%%%%%%%%%%%%%%%%%%%%%%%%%%%%%%%%%%%
\DeclareMathOperator{\ZFC}{ZFC}
\DeclareMathOperator{\ZF}{ZF}
\DeclareMathOperator{\AD}{AD}
\DeclareMathOperator{\ADR}{AD_{\R}}
\DeclareMathOperator{\KP}{KP}
\DeclareMathOperator{\PD}{PD}
\DeclareMathOperator{\CH}{CH}
\DeclareMathOperator{\HPC}{HPC} % HOD pair capturing
%%%%%%%%%%%%%%%%%%%%%%%%%%%%%%%%%%%%%%%%%%%%%
% Iteration Trees
%%%%%%%%%%%%%%%%%%%%%%%%%%%%%%%%%%%%%%%%%%%%%

\newcommand{\pred}{\text{-pred}}

%%%%%%%%%%%%%%%%%%%%%%%%%%%%%%%%%%%%%%%%%%%%%%%%
% Operator Names
%%%%%%%%%%%%%%%%%%%%%%%%%%%%%%%%%%%%%%%%%%%%%%%%
\DeclareMathOperator{\Det}{Det}
\DeclareMathOperator{\dom}{dom}
\DeclareMathOperator{\ran}{ran}
\DeclareMathOperator{\range}{ran}
\DeclareMathOperator{\image}{image}
\DeclareMathOperator{\crit}{crit}
\DeclareMathOperator{\card}{card}
\DeclareMathOperator{\cf}{cf}
\DeclareMathOperator{\cof}{cof}
\DeclareMathOperator{\rank}{rank}
\DeclareMathOperator{\ot}{o.t.}
\DeclareMathOperator{\ords}{o}
\DeclareMathOperator{\ro}{r.o.}
\DeclareMathOperator{\rud}{rud}
\DeclareMathOperator{\Powerset}{\mathcal{P}}
\DeclareMathOperator{\length}{lh}
\DeclareMathOperator{\lh}{lh}
\DeclareMathOperator{\limit}{lim}
\DeclareMathOperator{\fld}{fld}
\DeclareMathOperator{\projection}{p}
\DeclareMathOperator{\Ult}{Ult}
\DeclareMathOperator{\ULT}{Ult}
\DeclareMathOperator{\Coll}{Coll}
\DeclareMathOperator{\Col}{Col}
\DeclareMathOperator{\Hull}{Hull}
\DeclareMathOperator*{\dirlim}{dir lim}
\DeclareMathOperator{\Scale}{Scale}
\DeclareMathOperator{\supp}{supp}
\DeclareMathOperator{\trancl}{tran.cl.}
\DeclareMathOperator{\trace}{Tr}
\DeclareMathOperator{\diag}{diag}
\DeclareMathOperator{\spn}{span}
\DeclareMathOperator{\sgn}{sgn}
%%%%%%%%%%%%%%%%%%%%%%%%%%%%%%%%%%%%%%%%%%%%%
% Logical Connectives
%%%%%%%%%%%%%%%%%%%%%%%%%%%%%%%%%%%%%%%%%%%%%
\newcommand{\IImplies}{\Longrightarrow}
\newcommand{\SkipImplies}{\quad\Longrightarrow\quad}
\newcommand{\Ifff}{\Longleftrightarrow}
\newcommand{\iimplies}{\longrightarrow}
\newcommand{\ifff}{\longleftrightarrow}
\newcommand{\Implies}{\Rightarrow}
\newcommand{\Iff}{\Leftrightarrow}
\renewcommand{\implies}{\rightarrow}
\renewcommand{\iff}{\leftrightarrow}
\newcommand{\AND}{\wedge}
\newcommand{\OR}{\vee}
\newcommand{\st}{\text{ s.t. }}
\newcommand{\Or}{\text{ or }}

%%%%%%%%%%%%%%%%%%%%%%%%%%%%%%%%%%%%%%%%%%%%%
% Function Arrows
%%%%%%%%%%%%%%%%%%%%%%%%%%%%%%%%%%%%%%%%%%%%%

\newcommand{\injection}{\xrightarrow{\text{1-1}}}
\newcommand{\surjection}{\xrightarrow{\text{onto}}}
\newcommand{\bijection}{\xrightarrow[\text{onto}]{\text{1-1}}}
\newcommand{\cofmap}{\xrightarrow{\text{cof}}}
\newcommand{\map}{\rightarrow}

%%%%%%%%%%%%%%%%%%%%%%%%%%%%%%%%%%%%%%%%%%%%%
% Mouse Comparison Operators
%%%%%%%%%%%%%%%%%%%%%%%%%%%%%%%%%%%%%%%%%%%%%
\newcommand{\initseg}{\trianglelefteq}
\newcommand{\properseg}{\lhd}
\newcommand{\notinitseg}{\ntrianglelefteq}
\newcommand{\notproperseg}{\ntriangleleft}

%%%%%%%%%%%%%%%%%%%%%%%%%%%%%%%%%%%%%%%%%%%%%
%           calligraphic letters
%%%%%%%%%%%%%%%%%%%%%%%%%%%%%%%%%%%%%%%%%%%%%
\newcommand{\cA}{\mathcal{A}}
\newcommand{\cB}{\mathcal{B}}
\newcommand{\cC}{\mathcal{C}}
\newcommand{\cD}{\mathcal{D}}
\newcommand{\cE}{\mathcal{E}}
\newcommand{\cF}{\mathcal{F}}
\newcommand{\cG}{\mathcal{G}}
\newcommand{\cH}{\mathcal{H}}
\newcommand{\cI}{\mathcal{I}}
\newcommand{\cJ}{\mathcal{J}}
\newcommand{\cK}{\mathcal{K}}
\newcommand{\cL}{\mathcal{L}}
\newcommand{\cM}{\mathcal{M}}
\newcommand{\cN}{\mathcal{N}}
\newcommand{\cO}{\mathcal{O}}
\newcommand{\cP}{\mathcal{P}}
\newcommand{\cQ}{\mathcal{Q}}
\newcommand{\cR}{\mathcal{R}}
\newcommand{\cS}{\mathcal{S}}
\newcommand{\cT}{\mathcal{T}}
\newcommand{\cU}{\mathcal{U}}
\newcommand{\cV}{\mathcal{V}}
\newcommand{\cW}{\mathcal{W}}
\newcommand{\cX}{\mathcal{X}}
\newcommand{\cY}{\mathcal{Y}}
\newcommand{\cZ}{\mathcal{Z}}


%%%%%%%%%%%%%%%%%%%%%%%%%%%%%%%%%%%%%%%%%%%%%
%          Primed Letters
%%%%%%%%%%%%%%%%%%%%%%%%%%%%%%%%%%%%%%%%%%%%%
\newcommand{\aprime}{a^{\prime}}
\newcommand{\bprime}{b^{\prime}}
\newcommand{\cprime}{c^{\prime}}
\newcommand{\dprime}{d^{\prime}}
\newcommand{\eprime}{e^{\prime}}
\newcommand{\fprime}{f^{\prime}}
\newcommand{\gprime}{g^{\prime}}
\newcommand{\hprime}{h^{\prime}}
\newcommand{\iprime}{i^{\prime}}
\newcommand{\jprime}{j^{\prime}}
\newcommand{\kprime}{k^{\prime}}
\newcommand{\lprime}{l^{\prime}}
\newcommand{\mprime}{m^{\prime}}
\newcommand{\nprime}{n^{\prime}}
\newcommand{\oprime}{o^{\prime}}
\newcommand{\pprime}{p^{\prime}}
\newcommand{\qprime}{q^{\prime}}
\newcommand{\rprime}{r^{\prime}}
\newcommand{\sprime}{s^{\prime}}
\newcommand{\tprime}{t^{\prime}}
\newcommand{\uprime}{u^{\prime}}
\newcommand{\vprime}{v^{\prime}}
\newcommand{\wprime}{w^{\prime}}
\newcommand{\xprime}{x^{\prime}}
\newcommand{\yprime}{y^{\prime}}
\newcommand{\zprime}{z^{\prime}}
\newcommand{\Aprime}{A^{\prime}}
\newcommand{\Bprime}{B^{\prime}}
\newcommand{\Cprime}{C^{\prime}}
\newcommand{\Dprime}{D^{\prime}}
\newcommand{\Eprime}{E^{\prime}}
\newcommand{\Fprime}{F^{\prime}}
\newcommand{\Gprime}{G^{\prime}}
\newcommand{\Hprime}{H^{\prime}}
\newcommand{\Iprime}{I^{\prime}}
\newcommand{\Jprime}{J^{\prime}}
\newcommand{\Kprime}{K^{\prime}}
\newcommand{\Lprime}{L^{\prime}}
\newcommand{\Mprime}{M^{\prime}}
\newcommand{\Nprime}{N^{\prime}}
\newcommand{\Oprime}{O^{\prime}}
\newcommand{\Pprime}{P^{\prime}}
\newcommand{\Qprime}{Q^{\prime}}
\newcommand{\Rprime}{R^{\prime}}
\newcommand{\Sprime}{S^{\prime}}
\newcommand{\Tprime}{T^{\prime}}
\newcommand{\Uprime}{U^{\prime}}
\newcommand{\Vprime}{V^{\prime}}
\newcommand{\Wprime}{W^{\prime}}
\newcommand{\Xprime}{X^{\prime}}
\newcommand{\Yprime}{Y^{\prime}}
\newcommand{\Zprime}{Z^{\prime}}
\newcommand{\alphaprime}{\alpha^{\prime}}
\newcommand{\betaprime}{\beta^{\prime}}
\newcommand{\gammaprime}{\gamma^{\prime}}
\newcommand{\Gammaprime}{\Gamma^{\prime}}
\newcommand{\deltaprime}{\delta^{\prime}}
\newcommand{\epsilonprime}{\epsilon^{\prime}}
\newcommand{\kappaprime}{\kappa^{\prime}}
\newcommand{\lambdaprime}{\lambda^{\prime}}
\newcommand{\rhoprime}{\rho^{\prime}}
\newcommand{\Sigmaprime}{\Sigma^{\prime}}
\newcommand{\tauprime}{\tau^{\prime}}
\newcommand{\xiprime}{\xi^{\prime}}
\newcommand{\thetaprime}{\theta^{\prime}}
\newcommand{\Omegaprime}{\Omega^{\prime}}
\newcommand{\cMprime}{\cM^{\prime}}
\newcommand{\cNprime}{\cN^{\prime}}
\newcommand{\cPprime}{\cP^{\prime}}
\newcommand{\cQprime}{\cQ^{\prime}}
\newcommand{\cRprime}{\cR^{\prime}}
\newcommand{\cSprime}{\cS^{\prime}}
\newcommand{\cTprime}{\cT^{\prime}}

%%%%%%%%%%%%%%%%%%%%%%%%%%%%%%%%%%%%%%%%%%%%%
%          bar Letters
%%%%%%%%%%%%%%%%%%%%%%%%%%%%%%%%%%%%%%%%%%%%%
\newcommand{\abar}{\bar{a}}
\newcommand{\bbar}{\bar{b}}
\newcommand{\zbar}{\bar{z}}
\newcommand{\phibar}{\bar{\varphi}}
\newcommand{\psibar}{\bar{\psi}}
\newcommand{\thetabar}{\bar{\theta}}
\newcommand{\nubar}{\bar{\nu}}

%%%%%%%%%%%%%%%%%%%%%%%%%%%%%%%%%%%%%%%%%%%%%
%          star Letters
%%%%%%%%%%%%%%%%%%%%%%%%%%%%%%%%%%%%%%%%%%%%%
\newcommand{\phistar}{\phi^{*}}


%%%%%%%%%%%%%%%%%%%%%%%%%%%%%%%%%%%%%%%%%%%%%
%          Formulas
%%%%%%%%%%%%%%%%%%%%%%%%%%%%%%%%%%%%%%%%%%%%%

\newcommand{\formulaphi}{\text{\large $\varphi$}}
\newcommand{\Formulaphi}{\text{\Large $\varphi$}}


%%%%%%%%%%%%%%%%%%%%%%%%%%%%%%%%%%%%%%%%%%%%%
%          Fraktur Letters
%%%%%%%%%%%%%%%%%%%%%%%%%%%%%%%%%%%%%%%%%%%%%

\newcommand{\fa}{\mathfrak{a}}
\newcommand{\fb}{\mathfrak{b}}
\newcommand{\fc}{\mathfrak{c}}
\newcommand{\fk}{\mathfrak{k}}
\newcommand{\fp}{\mathfrak{p}}
\newcommand{\fq}{\mathfrak{q}}
\newcommand{\fr}{\mathfrak{r}}
\newcommand{\fA}{\mathfrak{A}}
\newcommand{\fB}{\mathfrak{B}}
\newcommand{\fC}{\mathfrak{C}}
\newcommand{\fD}{\mathfrak{D}}

%%%%%%%%%%%%%%%%%%%%%%%%%%%%%%%%%%%%%%%%%%%%%
%          Bold Letters
%%%%%%%%%%%%%%%%%%%%%%%%%%%%%%%%%%%%%%%%%%%%%
\newcommand{\ba}{\mathbf{a}}
\newcommand{\bb}{\mathbf{b}}
\newcommand{\bc}{\mathbf{c}}
\newcommand{\bd}{\mathbf{d}}
\newcommand{\be}{\mathbf{e}}
\newcommand{\bu}{\mathbf{u}}
\newcommand{\bv}{\mathbf{v}}
\newcommand{\bw}{\mathbf{w}}
\newcommand{\bx}{\mathbf{x}}
\newcommand{\by}{\mathbf{y}}
\newcommand{\bz}{\mathbf{z}}
\newcommand{\bSigma}{\boldsymbol{\Sigma}}
\newcommand{\bPi}{\boldsymbol{\Pi}}
\newcommand{\bDelta}{\boldsymbol{\Delta}}
\newcommand{\bdelta}{\boldsymbol{\delta}}
\newcommand{\bgamma}{\boldsymbol{\gamma}}
\newcommand{\bGamma}{\boldsymbol{\Gamma}}

%%%%%%%%%%%%%%%%%%%%%%%%%%%%%%%%%%%%%%%%%%%%%
%         Bold numbers
%%%%%%%%%%%%%%%%%%%%%%%%%%%%%%%%%%%%%%%%%%%%%
\newcommand{\bzero}{\mathbf{0}}

%%%%%%%%%%%%%%%%%%%%%%%%%%%%%%%%%%%%%%%%%%%%%
% Projective-Like Pointclasses
%%%%%%%%%%%%%%%%%%%%%%%%%%%%%%%%%%%%%%%%%%%%%
\newcommand{\Sa}[2][\alpha]{\Sigma_{(#1,#2)}}
\newcommand{\Pa}[2][\alpha]{\Pi_{(#1,#2)}}
\newcommand{\Da}[2][\alpha]{\Delta_{(#1,#2)}}
\newcommand{\Aa}[2][\alpha]{A_{(#1,#2)}}
\newcommand{\Ca}[2][\alpha]{C_{(#1,#2)}}
\newcommand{\Qa}[2][\alpha]{Q_{(#1,#2)}}
\newcommand{\da}[2][\alpha]{\delta_{(#1,#2)}}
\newcommand{\leqa}[2][\alpha]{\leq_{(#1,#2)}}
\newcommand{\lessa}[2][\alpha]{<_{(#1,#2)}}
\newcommand{\equiva}[2][\alpha]{\equiv_{(#1,#2)}}


\newcommand{\Sl}[1]{\Sa[\lambda]{#1}}
\newcommand{\Pl}[1]{\Pa[\lambda]{#1}}
\newcommand{\Dl}[1]{\Da[\lambda]{#1}}
\newcommand{\Al}[1]{\Aa[\lambda]{#1}}
\newcommand{\Cl}[1]{\Ca[\lambda]{#1}}
\newcommand{\Ql}[1]{\Qa[\lambda]{#1}}

\newcommand{\San}{\Sa{n}}
\newcommand{\Pan}{\Pa{n}}
\newcommand{\Dan}{\Da{n}}
\newcommand{\Can}{\Ca{n}}
\newcommand{\Qan}{\Qa{n}}
\newcommand{\Aan}{\Aa{n}}
\newcommand{\dan}{\da{n}}
\newcommand{\leqan}{\leqa{n}}
\newcommand{\lessan}{\lessa{n}}
\newcommand{\equivan}{\equiva{n}}

%%%%%%%%%%%%%%%%%%%%%%%%%%%%%%%%%%%%%%%%%%%%%
% Linear Algebra
%%%%%%%%%%%%%%%%%%%%%%%%%%%%%%%%%%%%%%%%%%%%%
\newcommand{\transpose}[1]{{#1}^{\text{T}}}
\newcommand{\norm}[1]{\lVert{#1}\rVert}
\newcommand\aug{\fboxsep=-\fboxrule\!\!\!\fbox{\strut}\!\!\!}

%%%%%%%%%%%%%%%%%%%%%%%%%%%%%%%%%%%%%%%%%%%%%
% Number Theory
%%%%%%%%%%%%%%%%%%%%%%%%%%%%%%%%%%%%%%%%%%%%%
\DeclareMathOperator{\Spec}{Spec}
\newcommand{\av}[1]{\lvert#1\rvert}
\DeclareMathOperator{\divides}{\mid}
\DeclareMathOperator{\ndivides}{\nmid}

\DeclareMathOperator{\NSat}{NSat}
%%%%%%%%%%%%%%%%%%%%%%%%%%%%%%%%%%%%%%%%%%%%%%%%%%%%%%%%%%%%%%%%%%%%%%%%%%%
%%  Theorem-Like Declarations
%%%%%%%%%%%%%%%%%%%%%%%%%%%%%%%%%%%%%%%%%%%%%%%%%%%%%%%%%%%%%%%%%%%%%%%%%%

\newtheorem{theorem}{Theorem}[section]
\newtheorem{lemma}[theorem]{Lemma}
\newtheorem{corollary}[theorem]{Corollary}
\newtheorem{proposition}[theorem]{Proposition}


\theoremstyle{definition}

\newtheorem{definition}[theorem]{Definition}
\newtheorem{conjecture}[theorem]{Conjecture}
\newtheorem{remark}[theorem]{Remark}
\newtheorem{remarks}[theorem]{Remarks}
\newtheorem{notation}[theorem]{Notation}

\newtheorem{homework}[theorem]{Exercise}
\newtheorem{numbered_example}[theorem]{Example}

\theoremstyle{remark}

\newtheorem*{aside}{Aside}
\newtheorem*{note}{Note}
\newtheorem*{observation}{Observation}
\newtheorem*{warning}{Warning}
\newtheorem*{question}{Question}
\newtheorem*{example}{Example}
\newtheorem*{in_class_example}{In-Class Example}
\newtheorem*{exercise}{Exercise}
\newtheorem*{fact}{Fact}


\newenvironment*{subproof}[1][Proof]
{\begin{proof}[#1]}{\renewcommand{\qedsymbol}{$\diamondsuit$} \end{proof}}

\newenvironment*{case}[1]
{\textbf{Case #1.  }\itshape }{}

\newenvironment*{claim}[1][Claim]
{\textbf{#1.  }\itshape }{}


%\pagestyle{plain}

\begin{document}

\title{Response to Referee Report for: The Mouse Set Theorem Just Past Projective}
\author{Mitch Rudominer}

\maketitle

We thank the referee for a thorough review of the paper and great feedback. We have addressed all of the comments.
Below we enumerate responses to each of the comments.


\begin{enumerate} [label=\alph*)]
\item  \emph{The concept of correctly well-founded trees could have been isolated more clearly, and there could have been a discussion on why it is defined as $\exists h,x$ rather than $\forall h,x$.} Thanks, I have cleaned up the definition of Correctly Wellfounded to more clearly explain the role of $h$ and $x$. See Definition 4.10 and Remark 4.12.
\end{enumerate}

\begin{enumerate} [label=\arabic*)]
\item \emph{It would be nice to remind the reader what exactly $n$-small means. This is relevant to the 5th paragraph of page 2.}
 I have inserted ``Recall the definition of $n$-small...''  in the third paragraph of section 1.
 
 \item  \emph{Remarks 2.2, exactly how these things follow from [Ste83]? Are they stated there, what theorems or lemma must the reader look at?}
 I have now added a proof sketch.
 
 \item \emph{Is Definition 2.3 correct? Does the formula stay the same for all w or it can change?}
 The definition is correct as stated. This is the standard way of defining a “$Q$” set. See for example definition 2.4 in “Introduction to Q-Theory” \cite{Q_Theory}. However it can be shown that there is a single definition that works for all $w$. 
 This can be shown using pure descriptive set-theoretic techniques. But also it actually follows from my paper: With the definition as stated, Theorem 3.6 gives us that  $Q_{\omega+1}\subset \Mladder$. But the proof of Theorem 3.5  ($\Mladder\intersect \R \subset Q_{\omega+1}$) actually gives a single definition that works for all $w$. I have now slightly modified the wording in the proof of Theorem 3.6 to make it clear that we do not need to assume that there is a uniform definition that works for all $w$. We only need that for each $w$ there is some definition. Previously my wording made it seem as if we were using a single definition that worked for all $w$ but we in fact do not need this.
 
 \item \emph{Remark 2.4? Is this a lemma or a proposition? Please give references, or give a short outline of the proof.} I have now deleted the Remarks---I don’t need them. Remark (1) concerns the ability to choose a  single uniform definition that works for all 
 $w$. That was the topic of your previous comment above–-the fact is true but we don’t need to assume it and in fact it follows from my paper. Remark (2) concerns the fact that if a real is $\SigmaOneOmegaPlusOne$ as a singleton, then it is also 
 $\SigmaOneOmegaPlusOne$ as a subset of $\omega \times \omega$. This is a very easy fact to see and I now use it directly in the proof Theorem 3.6. Remark (3) is a culturally interesting fact that is unrelated to the paper. I have now moved it up to the end of the paragraph (two paragraphs up) but I reference it in terms of $\Delta^1_3$ instead of $\Delta^1_{\omega+1}$ and I refer the reader to \cite{Q_Theory}.
 
 \item  \emph{Def 3.1, it would help to remind the reader what $n$ small is. For example, is there an extender indexed at 
 $\gamma_n$, if yes the notation you use is confusing. Is $M\vert\vert \gamma$ (i.e. keep the last extender, I think Steel uses 
 $M\vert\gamma$ for this). Also, is the $J$ in (3) the same as $\cJ$ in (4), or it is some notation not introduced yet?} 
  Based on your comment (1) above I reminded the reader of the definition of $n$-small. I have now added a paragraph after definition 3.1 reminding the reader of the distinction between $J^M_{\gamma}$ and $\cJ^M_{\gamma}$. 
  The least mouse that is not $n$-small is active. So, yes, there is an extender indexed at $\gamma_n$.

  \item \emph{Remarks 3.2, why is (1) true? In (2), what does it mean to say that something is projectively correct? Basically, it wouldn’t hurt to say a few more words about the proofs, it is
all standard, but people do not know these things.} These are now Remarks 3.3 and I have refactored them. Previously (1) read ``Because we are assuming $\AD^{\LofR}$, $\Mladder$ exists.'' For simplicity I have decided to change the official hypothesis of the paper to $\ZFC + $ ``there exists $\omega$ Woodin cardinals with a measurable cardinal above them all.''
This hypothesis as well as the consequences of it that we need are now described at the end of Section 1 along with references
to proofs. For (2), I added definition 3.2 defining several notions of correctness.

\item \emph{Theorem 3.3, what is the hypo? just that $\Mladder$ exists? Same for Theorem 3.4 and 3.5.}
These are now Theorems 3.4, 3.5, 3.6. I added explicit hypotheses. Also I added two additional paragraphs at the end of the introduction discussing the hypotheses of the paper.

\item \emph{Page 5, the paragraph that starts ``if $z$ is a real...'', what does it mean to say that $z$ is an
additional predicate? Do we start the constructibility order with $z$? One can also explain
this in terms of the difference between $L$ and $L[z]$.}  I added the sentence 
``(So this will be a model of the form $J_{\alpha}[\vec{E},z]$.)'' This is now the top of page 7.

\item  \emph{Page 6, shouldn’t you say something about the $\varphi$ sequence being uniform in $(n,e)$? Isn’t this Moskovakis’ Periodicity Theorems? At any rate, a few words would be nice.} I’m sorry I’m having trouble responding to this. Could you point to the exact line you are talking about and say more about what you want me to say a few more words about? I believe you are
referring to Definition 4.1. As the text is written I do already write ``uniformly in $n$ and $e$''. Moskovakis’s  second Periodicity Theorem does play an indirect role in this paper. I mention in Remark 2.2 (5) that $\Pi^1_{\omega+1}$ is a scaled point class. 

\item \emph{It says $G^n_e = p[T^n_e]$. What is this? a fact? a claim? Truth?}  I  added some more words: This is part of what it means that $\varphi$ is a scale.

\item \emph{Proof of theorem 3.5, the reader might wonder why $\Mladder$ has a measurable cardinal.} OK I added a short explanation in parentheses.

\item \emph{It might be useful to the reader to note that $T^h$ is building a real and proving that it belongs to each $T^n_{h(n)}$}
 Ok, I added the following sentence: ``Intuitively, one can think of $T^h$ as building a pair $(x, f)$ where $x$ is a real and $f$
  is a certificate verifying that $x \in p[T^n_{h(n)} ]$ for all $n\in\omega$.''

 \item \emph{Page 7, $\omega$-sequences of what?} I changed the wording to: ``Our coding of pairs of integers by integers induces a mapping from $\omega^\omega$ to $(\omega^\omega) ^\omega$.'' This is now towards the bottom of page 8.

 \item \emph{Proof of Theorem 3.6, the last sentence of the first paragraph, this sounds like an important fact to note somewhere.}
 I think you are referring to this statement: ``If all wellfounded $\bPiOneOmega$ trees in $M$ are correctly wellfounded
then $M$ is $\Sigma^1_{\omega+1}$-correct.'' This is now the proof of Theorem 3.7.
I changed the sentence to read instead
``If there are no incorrectly wellfounded $\bPiOneOmega$ trees in $M$
then $M$ is $\Sigma^1_{\omega+1}$-correct.'' This is really what I wanted to say.
In this form the sentences is really immediate. An incorrectly wellfounded tree is a witness of a 
$\Sigma^1_{\omega+1}$ statement that is true in $V$ and false in $M$. So if there are no incorrectly wellfounded trees then there are no such statements and so $M$ is $\Sigma^1_{\omega+1}$-correct.

\item \emph{Just after the proof of Theorem 3.6, should be go through Theorem 3.6.}
Changed, thank you. (But now it is Theorem 3.7 instead of Theorem 3.6.)

\item \emph{ It seems that the equivalence displayed at the end of ``alternative proof of theorem 3.5''
needs that if $M\models T^h(x) = T^{\hprime}(\xprime)$ then $T^h(x) = T^{\hprime}(\xprime)$
which seems to hold as $M$ is
projectively correct. If I am correct, please make a note of this.} This is now the
alternate proof of Theorem 3.6. I expanded the explanation there into the Claim 
at the bottom of page 11, with a proof.

\item \emph{lemma 5.2, what does $\Gamma$-correct mean?}
You are right. I don’t know how to make sense of $\Gamma$-correct for an arbitrary pointclass $\Gamma$. I changed the text to restrict the statement of the Lemma to 
$\Gamma = \Sigma^1_n$ and added a note that the Lemma can be generalized to other pointclasses $\Gamma$ for which we can make sense of the notion of $\Gamma$-correct.

\item \emph{the proof of Cor 5.3, the fact that $E$ is projective uses the Harrington-Kechris thingy, right?} It is prima facie projective because if $\varphi$ is a projective norm then 
``$\varphi(x) = \varphi(y)$'' is projetive by definition.

\item \emph{sketch of the proof of Lemma 5.1, $\gamma_0$ is not a cardinal, you probably meant $\delta_0$.} Thank you. Changed.

\item \emph{Claim 1 on Page 18, can you remind the reader what ``$x$ captures $D$'' means?}
OK, I have now added Definitions 5.6 - 5.10 (defining captures, end-extends, and semi-proper) and Lemmas 5.11 and 5.12 (reminding the reader of some facts from Stationary Tower forcing that we heavily rely on.)

\item \emph{I don’t see that $D = D_j$
in the proof of Claim 1 on page 19. Presumably, the least not
captured dense set for $x_j$ could have smaller. Shouldn’t more bookkeeping be used in
this argument? Clearly the claim is correct, there are countably many conditions and
countably many steps, so we can of course do it.} Thank you for catching this. I have 
substantively reorganized the proof of Lemma 5.5. Now what we do is define a function $\pi$
from $\omega$ to 
$\omega \times \omega$ and we define $x_{t+1}$ so that if $\pi(t)=(i,j)$
 then $x_{t+1}$ captures the the $j$th dense set from $x_i$. This argument is now on page 22.

 \item \emph{Page 20, second paragraph from below. What does it mean to say that there is a comparison theorem for some type of mice? There is a comparison theorem for all mice. Do
you mean assuming some weak form of iterability?}
I really just meant the statement that I have restated as Lemma 6.8 later on this same page. I have just removed the sentence that you found confusing.

\item \emph{It is unclear why eliminating the ordinal parameter $\theta$ is important or useful. Can you give some info on this?}
OK, I have now added some new material to the top of section 7 giving the motivation.

\item \emph{The last section of the paper is unreadable, perhaps consider giving the definition of $n$-iterability or what consequences it might have. You don’t have to produce all the theorem
or arguments about $n$-iterability, but a quick review would do a lot of good.}
 I think it was not clear that I was expecting the reader to know some of the material from Section 6 in order to read Section 7. I added a note to this effect at the top of Section 7. You specifically call out the definition of $n$-iterability. That definition and its important properties are given in Section 6, but actually this is not used directly in Section 7. It is the notions of ladder-small and $\Pi^1_{\omega}$ iterability from Section 6 that are used in Section 7.

 \item \emph{ The definition of $S_{n,w}$, is it true that the reals in it all code the same mouse? if so, please make a comment.}
 What you are saying is true if $|w| < (\omega_1)^{\Mladder}$. I have added Remark 7.5 to this effect.

 \item \emph{Stationary Tower, once in capitals then in lower cases.}
  Thanks. Changed to lower-case everywhere.


\item \emph{What does above $g$ mean?} It’s defined in the last sentence of Definition 7.9.

\item \emph{Please explain the first line of the first paragraph of page 24, recall that correct wellfoundedness is defined for any real parameter, so I suppose there is a lot of correctness
used in this argument. probably it is better to just isolate a lemma summarizing the
bottom paragraph on page 12. There are many such spots in the paper that could have
been isolated and stated more clearly}
 
 I think the sentence you are referring to is now near the top of page 29 and reads ``By Lemma 7.6, all of the $\NSat$ trees of $\Mladder$
are correctly wellfounded.'' I think this was confusing because it was not clear what I meant by saying that a tree is correctly wellfounded without reference to the $\PiOneOmega$ code and real parameter used to define the tree. To correct this problem I have now added in parentheses  ``(See Remark 4.12 for what it means to say that a tree is correctly or incorrectly wellfounded without reference to a a $\PiOneOmega$-code or real parameter.)'' With this clarification, I think that confusion goes away. 
Let $T$ be an $\NSat$ tree of $\Mladder$. By definition this means that 
$T = T^{\Mladder}_{(n,w)}$ for $(n,w) \in \NSat^{\Mladder}$. Lemma 7.6 says that 
$T^V_{(n,w)}$ is wellfounded which means that 
$T^{\Mladder}_{(n,w)}$ is correctly wellfounded.

\item \emph{Page 25, the top paragraph, there are two ”and”s.} Thanks. Fixed.

\item \emph{Page 25, the last paragraph, I feel this should have been done much earlier than here.
Also, this is not too clear. For some $x$ or for all $x$, or are these equivalent? in general,
it will be nice to talk about the dependence or independence of the concept of correctly
well-founded on the parameters involved.}
OK, I added Definition 7.10 and the paragraph above it, and Remark 7.11.

\item  \emph{In the definition of $X^{M,\delta}$, 
$g \subseteq \Coll$...,the way it is written looks somewhat strange.} Thanks. Fixed






















\end{enumerate}


\bibliographystyle{amsalpha}
\bibliography{math}

\end{document}
