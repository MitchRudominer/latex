% $Header$

\documentclass{beamer}
%\documentclass[handout]{beamer}

\usepackage{amsmath,amssymb,latexsym,eucal,amsthm,graphicx,hyperref}
%%%%%%%%%%%%%%%%%%%%%%%%%%%%%%%%%%%%%%%%%%%%%
% Common Set Theory Constructs
%%%%%%%%%%%%%%%%%%%%%%%%%%%%%%%%%%%%%%%%%%%%%

\newcommand{\setof}[2]{\left\{ \, #1 \, \left| \, #2 \, \right.\right\}}
\newcommand{\lsetof}[2]{\left\{\left. \, #1 \, \right| \, #2 \,  \right\}}
\newcommand{\bigsetof}[2]{\bigl\{ \, #1 \, \bigm | \, #2 \,\bigr\}}
\newcommand{\Bigsetof}[2]{\Bigl\{ \, #1 \, \Bigm | \, #2 \,\Bigr\}}
\newcommand{\biggsetof}[2]{\biggl\{ \, #1 \, \biggm | \, #2 \,\biggr\}}
\newcommand{\Biggsetof}[2]{\Biggl\{ \, #1 \, \Biggm | \, #2 \,\Biggr\}}
\newcommand{\dotsetof}[2]{\left\{ \, #1 \, : \, #2 \, \right\}}
\newcommand{\bigdotsetof}[2]{\bigl\{ \, #1 \, : \, #2 \,\bigr\}}
\newcommand{\Bigdotsetof}[2]{\Bigl\{ \, #1 \, \Bigm : \, #2 \,\Bigr\}}
\newcommand{\biggdotsetof}[2]{\biggl\{ \, #1 \, \biggm : \, #2 \,\biggr\}}
\newcommand{\Biggdotsetof}[2]{\Biggl\{ \, #1 \, \Biggm : \, #2 \,\Biggr\}}
\newcommand{\sequence}[2]{\left\langle \, #1 \,\left| \, #2 \, \right. \right\rangle}
\newcommand{\lsequence}[2]{\left\langle\left. \, #1 \, \right| \,#2 \,  \right\rangle}
\newcommand{\bigsequence}[2]{\bigl\langle \,#1 \, \bigm | \, #2 \, \bigr\rangle}
\newcommand{\Bigsequence}[2]{\Bigl\langle \,#1 \, \Bigm | \, #2 \, \Bigr\rangle}
\newcommand{\biggsequence}[2]{\biggl\langle \,#1 \, \biggm | \, #2 \, \biggr\rangle}
\newcommand{\Biggsequence}[2]{\Biggl\langle \,#1 \, \Biggm | \, #2 \, \Biggr\rangle}
\newcommand{\singleton}[1]{\left\{#1\right\}}
\newcommand{\angles}[1]{\left\langle #1 \right\rangle}
\newcommand{\bigangles}[1]{\bigl\langle #1 \bigr\rangle}
\newcommand{\Bigangles}[1]{\Bigl\langle #1 \Bigr\rangle}
\newcommand{\biggangles}[1]{\biggl\langle #1 \biggr\rangle}
\newcommand{\Biggangles}[1]{\Biggl\langle #1 \Biggr\rangle}


\newcommand{\force}[1]{\Vert\!\underset{\!\!\!\!\!#1}{\!\!\!\relbar\!\!\!%
\relbar\!\!\relbar\!\!\relbar\!\!\!\relbar\!\!\relbar\!\!\relbar\!\!\!%
\relbar\!\!\relbar\!\!\relbar}}
\newcommand{\nforce}[1]{\Vert\!\underset{\!\!\!\!\!#1}{\!\!\!\relbar\!\!\!%
\relbar\!\!\relbar\!\!\relbar\!\!\!\relbar\!\!\relbar\!\!\relbar\!\!\!%
\relbar\!\!\not\relbar\!\!\relbar}}
\newcommand{\forcein}[2]{\overset{#2}{\Vert\underset{\!\!\!\!\!#1}%
{\!\!\!\relbar\!\!\!\relbar\!\!\relbar\!\!\relbar\!\!\!\relbar\!\!\relbar\!%
\!\relbar\!\!\!\relbar\!\!\relbar\!\!\relbar\!\!\relbar\!\!\!\relbar\!\!%
\relbar\!\!\relbar}}}

\newcommand{\pre}[2]{{}^{#2}\!{#1}}

\newcommand{\restr}{\!\!\upharpoonright\!}

%%%%%%%%%%%%%%%%%%%%%%%%%%%%%%%%%%%%%%%%%%%%%
% Set-Theoretic Connectives
%%%%%%%%%%%%%%%%%%%%%%%%%%%%%%%%%%%%%%%%%%%%%

\newcommand{\intersect}{\cap}
\newcommand{\union}{\cup}
\newcommand{\Intersection}[1]{\bigcap\limits_{#1}}
\newcommand{\Union}[1]{\bigcup\limits_{#1}}
\newcommand{\adjoin}{{}^\frown}
\newcommand{\forces}{\Vdash}

%%%%%%%%%%%%%%%%%%%%%%%%%%%%%%%%%%%%%%%%%%%%%
% Miscellaneous
%%%%%%%%%%%%%%%%%%%%%%%%%%%%%%%%%%%%%%%%%%%%%
\newcommand{\defeq}{=_{\text{def}}}
\newcommand{\Turingleq}{\leq_{\text{T}}}
\newcommand{\Turingless}{<_{\text{T}}}
\newcommand{\lexleq}{\leq_{\text{lex}}}
\newcommand{\lexless}{<_{\text{lex}}}
\newcommand{\Turingequiv}{\equiv_{\text{T}}}

%%%%%%%%%%%%%%%%%%%%%%%%%%%%%%%%%%%%%%%%%%%%%
% Constants
%%%%%%%%%%%%%%%%%%%%%%%%%%%%%%%%%%%%%%%%%%%%%
\newcommand{\R}{\mathbb{R}}
\renewcommand{\P}{\mathbb{P}}
\newcommand{\Q}{\mathbb{Q}}
\newcommand{\Z}{\mathbb{Z}}
\newcommand{\C}{\mathbb{C}}
\newcommand{\N}{\mathbb{N}}
\newcommand{\B}{\mathbb{B}}
\newcommand{\LofR}{L(\R)}
\newcommand{\JofR}[1]{J_{#1}(\R)}
\newcommand{\SofR}[1]{S_{#1}(\R)}
\newcommand{\JalphaR}{\JofR{\alpha}}
\newcommand{\JbetaR}{\JofR{\beta}}
\newcommand{\JlambdaR}{\JofR{\lambda}}
\newcommand{\SalphaR}{\SofR{\alpha}}
\newcommand{\SbetaR}{\SofR{\beta}}
\newcommand{\Pkl}{\mathcal{P}_{\kappa}(\lambda)}
\DeclareMathOperator{\con}{con}
\DeclareMathOperator{\ORD}{OR}
\DeclareMathOperator{\Ord}{OR}
\DeclareMathOperator{\WO}{WO}
\DeclareMathOperator{\OD}{OD}
\DeclareMathOperator{\HOD}{HOD}
\DeclareMathOperator{\HC}{HC}
\DeclareMathOperator{\WF}{WF}
\DeclareMathOperator{\HF}{HF}
\newcommand{\One}{I}
\newcommand{\ONE}{I}
\newcommand{\Two}{II}
\newcommand{\TWO}{II}

%%%%%%%%%%%%%%%%%%%%%%%%%%%%%%%%%%%%%%%%%%%%%
% Commutative Algebra Constants
%%%%%%%%%%%%%%%%%%%%%%%%%%%%%%%%%%%%%%%%%%%%%
\DeclareMathOperator{\dottimes}{\dot{\times}}

%%%%%%%%%%%%%%%%%%%%%%%%%%%%%%%%%%%%%%%%%%%%%
% Theories
%%%%%%%%%%%%%%%%%%%%%%%%%%%%%%%%%%%%%%%%%%%%%
\DeclareMathOperator{\ZFC}{ZFC}
\DeclareMathOperator{\ZF}{ZF}
\DeclareMathOperator{\AD}{AD}
\DeclareMathOperator{\ADR}{AD_{\R}}
\DeclareMathOperator{\KP}{KP}
\DeclareMathOperator{\PD}{PD}
\DeclareMathOperator{\CH}{CH}
\DeclareMathOperator{\HPC}{HPC} % HOD pair capturing
%%%%%%%%%%%%%%%%%%%%%%%%%%%%%%%%%%%%%%%%%%%%%
% Iteration Trees
%%%%%%%%%%%%%%%%%%%%%%%%%%%%%%%%%%%%%%%%%%%%%

\newcommand{\pred}{\text{-pred}}

%%%%%%%%%%%%%%%%%%%%%%%%%%%%%%%%%%%%%%%%%%%%%%%%
% Operator Names
%%%%%%%%%%%%%%%%%%%%%%%%%%%%%%%%%%%%%%%%%%%%%%%%
\DeclareMathOperator{\Det}{Det}
\DeclareMathOperator{\dom}{dom}
\DeclareMathOperator{\ran}{ran}
\DeclareMathOperator{\range}{ran}
\DeclareMathOperator{\image}{image}
\DeclareMathOperator{\crit}{crit}
\DeclareMathOperator{\card}{card}
\DeclareMathOperator{\cf}{cf}
\DeclareMathOperator{\cof}{cof}
\DeclareMathOperator{\rank}{rank}
\DeclareMathOperator{\ot}{o.t.}
\DeclareMathOperator{\ords}{o}
\DeclareMathOperator{\ro}{r.o.}
\DeclareMathOperator{\rud}{rud}
\DeclareMathOperator{\Powerset}{\mathcal{P}}
\DeclareMathOperator{\length}{lh}
\DeclareMathOperator{\lh}{lh}
\DeclareMathOperator{\limit}{lim}
\DeclareMathOperator{\fld}{fld}
\DeclareMathOperator{\projection}{p}
\DeclareMathOperator{\Ult}{Ult}
\DeclareMathOperator{\ULT}{Ult}
\DeclareMathOperator{\Coll}{Coll}
\DeclareMathOperator{\Col}{Col}
\DeclareMathOperator{\Hull}{Hull}
\DeclareMathOperator*{\dirlim}{dir lim}
\DeclareMathOperator{\Scale}{Scale}
\DeclareMathOperator{\supp}{supp}
\DeclareMathOperator{\trancl}{tran.cl.}
\DeclareMathOperator{\trace}{Tr}
\DeclareMathOperator{\diag}{diag}
\DeclareMathOperator{\spn}{span}
\DeclareMathOperator{\sgn}{sgn}
%%%%%%%%%%%%%%%%%%%%%%%%%%%%%%%%%%%%%%%%%%%%%
% Logical Connectives
%%%%%%%%%%%%%%%%%%%%%%%%%%%%%%%%%%%%%%%%%%%%%
\newcommand{\IImplies}{\Longrightarrow}
\newcommand{\SkipImplies}{\quad\Longrightarrow\quad}
\newcommand{\Ifff}{\Longleftrightarrow}
\newcommand{\iimplies}{\longrightarrow}
\newcommand{\ifff}{\longleftrightarrow}
\newcommand{\Implies}{\Rightarrow}
\newcommand{\Iff}{\Leftrightarrow}
\renewcommand{\implies}{\rightarrow}
\renewcommand{\iff}{\leftrightarrow}
\newcommand{\AND}{\wedge}
\newcommand{\OR}{\vee}
\newcommand{\st}{\text{ s.t. }}
\newcommand{\Or}{\text{ or }}

%%%%%%%%%%%%%%%%%%%%%%%%%%%%%%%%%%%%%%%%%%%%%
% Function Arrows
%%%%%%%%%%%%%%%%%%%%%%%%%%%%%%%%%%%%%%%%%%%%%

\newcommand{\injection}{\xrightarrow{\text{1-1}}}
\newcommand{\surjection}{\xrightarrow{\text{onto}}}
\newcommand{\bijection}{\xrightarrow[\text{onto}]{\text{1-1}}}
\newcommand{\cofmap}{\xrightarrow{\text{cof}}}
\newcommand{\map}{\rightarrow}

%%%%%%%%%%%%%%%%%%%%%%%%%%%%%%%%%%%%%%%%%%%%%
% Mouse Comparison Operators
%%%%%%%%%%%%%%%%%%%%%%%%%%%%%%%%%%%%%%%%%%%%%
\newcommand{\initseg}{\trianglelefteq}
\newcommand{\properseg}{\lhd}
\newcommand{\notinitseg}{\ntrianglelefteq}
\newcommand{\notproperseg}{\ntriangleleft}

%%%%%%%%%%%%%%%%%%%%%%%%%%%%%%%%%%%%%%%%%%%%%
%           calligraphic letters
%%%%%%%%%%%%%%%%%%%%%%%%%%%%%%%%%%%%%%%%%%%%%
\newcommand{\cA}{\mathcal{A}}
\newcommand{\cB}{\mathcal{B}}
\newcommand{\cC}{\mathcal{C}}
\newcommand{\cD}{\mathcal{D}}
\newcommand{\cE}{\mathcal{E}}
\newcommand{\cF}{\mathcal{F}}
\newcommand{\cG}{\mathcal{G}}
\newcommand{\cH}{\mathcal{H}}
\newcommand{\cI}{\mathcal{I}}
\newcommand{\cJ}{\mathcal{J}}
\newcommand{\cK}{\mathcal{K}}
\newcommand{\cL}{\mathcal{L}}
\newcommand{\cM}{\mathcal{M}}
\newcommand{\cN}{\mathcal{N}}
\newcommand{\cO}{\mathcal{O}}
\newcommand{\cP}{\mathcal{P}}
\newcommand{\cQ}{\mathcal{Q}}
\newcommand{\cR}{\mathcal{R}}
\newcommand{\cS}{\mathcal{S}}
\newcommand{\cT}{\mathcal{T}}
\newcommand{\cU}{\mathcal{U}}
\newcommand{\cV}{\mathcal{V}}
\newcommand{\cW}{\mathcal{W}}
\newcommand{\cX}{\mathcal{X}}
\newcommand{\cY}{\mathcal{Y}}
\newcommand{\cZ}{\mathcal{Z}}


%%%%%%%%%%%%%%%%%%%%%%%%%%%%%%%%%%%%%%%%%%%%%
%          Primed Letters
%%%%%%%%%%%%%%%%%%%%%%%%%%%%%%%%%%%%%%%%%%%%%
\newcommand{\aprime}{a^{\prime}}
\newcommand{\bprime}{b^{\prime}}
\newcommand{\cprime}{c^{\prime}}
\newcommand{\dprime}{d^{\prime}}
\newcommand{\eprime}{e^{\prime}}
\newcommand{\fprime}{f^{\prime}}
\newcommand{\gprime}{g^{\prime}}
\newcommand{\hprime}{h^{\prime}}
\newcommand{\iprime}{i^{\prime}}
\newcommand{\jprime}{j^{\prime}}
\newcommand{\kprime}{k^{\prime}}
\newcommand{\lprime}{l^{\prime}}
\newcommand{\mprime}{m^{\prime}}
\newcommand{\nprime}{n^{\prime}}
\newcommand{\oprime}{o^{\prime}}
\newcommand{\pprime}{p^{\prime}}
\newcommand{\qprime}{q^{\prime}}
\newcommand{\rprime}{r^{\prime}}
\newcommand{\sprime}{s^{\prime}}
\newcommand{\tprime}{t^{\prime}}
\newcommand{\uprime}{u^{\prime}}
\newcommand{\vprime}{v^{\prime}}
\newcommand{\wprime}{w^{\prime}}
\newcommand{\xprime}{x^{\prime}}
\newcommand{\yprime}{y^{\prime}}
\newcommand{\zprime}{z^{\prime}}
\newcommand{\Aprime}{A^{\prime}}
\newcommand{\Bprime}{B^{\prime}}
\newcommand{\Cprime}{C^{\prime}}
\newcommand{\Dprime}{D^{\prime}}
\newcommand{\Eprime}{E^{\prime}}
\newcommand{\Fprime}{F^{\prime}}
\newcommand{\Gprime}{G^{\prime}}
\newcommand{\Hprime}{H^{\prime}}
\newcommand{\Iprime}{I^{\prime}}
\newcommand{\Jprime}{J^{\prime}}
\newcommand{\Kprime}{K^{\prime}}
\newcommand{\Lprime}{L^{\prime}}
\newcommand{\Mprime}{M^{\prime}}
\newcommand{\Nprime}{N^{\prime}}
\newcommand{\Oprime}{O^{\prime}}
\newcommand{\Pprime}{P^{\prime}}
\newcommand{\Qprime}{Q^{\prime}}
\newcommand{\Rprime}{R^{\prime}}
\newcommand{\Sprime}{S^{\prime}}
\newcommand{\Tprime}{T^{\prime}}
\newcommand{\Uprime}{U^{\prime}}
\newcommand{\Vprime}{V^{\prime}}
\newcommand{\Wprime}{W^{\prime}}
\newcommand{\Xprime}{X^{\prime}}
\newcommand{\Yprime}{Y^{\prime}}
\newcommand{\Zprime}{Z^{\prime}}
\newcommand{\alphaprime}{\alpha^{\prime}}
\newcommand{\betaprime}{\beta^{\prime}}
\newcommand{\gammaprime}{\gamma^{\prime}}
\newcommand{\Gammaprime}{\Gamma^{\prime}}
\newcommand{\deltaprime}{\delta^{\prime}}
\newcommand{\epsilonprime}{\epsilon^{\prime}}
\newcommand{\kappaprime}{\kappa^{\prime}}
\newcommand{\lambdaprime}{\lambda^{\prime}}
\newcommand{\rhoprime}{\rho^{\prime}}
\newcommand{\Sigmaprime}{\Sigma^{\prime}}
\newcommand{\tauprime}{\tau^{\prime}}
\newcommand{\xiprime}{\xi^{\prime}}
\newcommand{\thetaprime}{\theta^{\prime}}
\newcommand{\Omegaprime}{\Omega^{\prime}}
\newcommand{\cMprime}{\cM^{\prime}}
\newcommand{\cNprime}{\cN^{\prime}}
\newcommand{\cPprime}{\cP^{\prime}}
\newcommand{\cQprime}{\cQ^{\prime}}
\newcommand{\cRprime}{\cR^{\prime}}
\newcommand{\cSprime}{\cS^{\prime}}
\newcommand{\cTprime}{\cT^{\prime}}

%%%%%%%%%%%%%%%%%%%%%%%%%%%%%%%%%%%%%%%%%%%%%
%          bar Letters
%%%%%%%%%%%%%%%%%%%%%%%%%%%%%%%%%%%%%%%%%%%%%
\newcommand{\abar}{\bar{a}}
\newcommand{\bbar}{\bar{b}}
\newcommand{\zbar}{\bar{z}}
\newcommand{\phibar}{\bar{\varphi}}
\newcommand{\psibar}{\bar{\psi}}
\newcommand{\thetabar}{\bar{\theta}}
\newcommand{\nubar}{\bar{\nu}}

%%%%%%%%%%%%%%%%%%%%%%%%%%%%%%%%%%%%%%%%%%%%%
%          star Letters
%%%%%%%%%%%%%%%%%%%%%%%%%%%%%%%%%%%%%%%%%%%%%
\newcommand{\phistar}{\phi^{*}}


%%%%%%%%%%%%%%%%%%%%%%%%%%%%%%%%%%%%%%%%%%%%%
%          Formulas
%%%%%%%%%%%%%%%%%%%%%%%%%%%%%%%%%%%%%%%%%%%%%

\newcommand{\formulaphi}{\text{\large $\varphi$}}
\newcommand{\Formulaphi}{\text{\Large $\varphi$}}


%%%%%%%%%%%%%%%%%%%%%%%%%%%%%%%%%%%%%%%%%%%%%
%          Fraktur Letters
%%%%%%%%%%%%%%%%%%%%%%%%%%%%%%%%%%%%%%%%%%%%%

\newcommand{\fa}{\mathfrak{a}}
\newcommand{\fb}{\mathfrak{b}}
\newcommand{\fc}{\mathfrak{c}}
\newcommand{\fk}{\mathfrak{k}}
\newcommand{\fp}{\mathfrak{p}}
\newcommand{\fq}{\mathfrak{q}}
\newcommand{\fr}{\mathfrak{r}}
\newcommand{\fA}{\mathfrak{A}}
\newcommand{\fB}{\mathfrak{B}}
\newcommand{\fC}{\mathfrak{C}}
\newcommand{\fD}{\mathfrak{D}}

%%%%%%%%%%%%%%%%%%%%%%%%%%%%%%%%%%%%%%%%%%%%%
%          Bold Letters
%%%%%%%%%%%%%%%%%%%%%%%%%%%%%%%%%%%%%%%%%%%%%
\newcommand{\ba}{\mathbf{a}}
\newcommand{\bb}{\mathbf{b}}
\newcommand{\bc}{\mathbf{c}}
\newcommand{\bd}{\mathbf{d}}
\newcommand{\be}{\mathbf{e}}
\newcommand{\bu}{\mathbf{u}}
\newcommand{\bv}{\mathbf{v}}
\newcommand{\bw}{\mathbf{w}}
\newcommand{\bx}{\mathbf{x}}
\newcommand{\by}{\mathbf{y}}
\newcommand{\bz}{\mathbf{z}}
\newcommand{\bSigma}{\boldsymbol{\Sigma}}
\newcommand{\bPi}{\boldsymbol{\Pi}}
\newcommand{\bDelta}{\boldsymbol{\Delta}}
\newcommand{\bdelta}{\boldsymbol{\delta}}
\newcommand{\bgamma}{\boldsymbol{\gamma}}
\newcommand{\bGamma}{\boldsymbol{\Gamma}}

%%%%%%%%%%%%%%%%%%%%%%%%%%%%%%%%%%%%%%%%%%%%%
%         Bold numbers
%%%%%%%%%%%%%%%%%%%%%%%%%%%%%%%%%%%%%%%%%%%%%
\newcommand{\bzero}{\mathbf{0}}

%%%%%%%%%%%%%%%%%%%%%%%%%%%%%%%%%%%%%%%%%%%%%
% Projective-Like Pointclasses
%%%%%%%%%%%%%%%%%%%%%%%%%%%%%%%%%%%%%%%%%%%%%
\newcommand{\Sa}[2][\alpha]{\Sigma_{(#1,#2)}}
\newcommand{\Pa}[2][\alpha]{\Pi_{(#1,#2)}}
\newcommand{\Da}[2][\alpha]{\Delta_{(#1,#2)}}
\newcommand{\Aa}[2][\alpha]{A_{(#1,#2)}}
\newcommand{\Ca}[2][\alpha]{C_{(#1,#2)}}
\newcommand{\Qa}[2][\alpha]{Q_{(#1,#2)}}
\newcommand{\da}[2][\alpha]{\delta_{(#1,#2)}}
\newcommand{\leqa}[2][\alpha]{\leq_{(#1,#2)}}
\newcommand{\lessa}[2][\alpha]{<_{(#1,#2)}}
\newcommand{\equiva}[2][\alpha]{\equiv_{(#1,#2)}}


\newcommand{\Sl}[1]{\Sa[\lambda]{#1}}
\newcommand{\Pl}[1]{\Pa[\lambda]{#1}}
\newcommand{\Dl}[1]{\Da[\lambda]{#1}}
\newcommand{\Al}[1]{\Aa[\lambda]{#1}}
\newcommand{\Cl}[1]{\Ca[\lambda]{#1}}
\newcommand{\Ql}[1]{\Qa[\lambda]{#1}}

\newcommand{\San}{\Sa{n}}
\newcommand{\Pan}{\Pa{n}}
\newcommand{\Dan}{\Da{n}}
\newcommand{\Can}{\Ca{n}}
\newcommand{\Qan}{\Qa{n}}
\newcommand{\Aan}{\Aa{n}}
\newcommand{\dan}{\da{n}}
\newcommand{\leqan}{\leqa{n}}
\newcommand{\lessan}{\lessa{n}}
\newcommand{\equivan}{\equiva{n}}

%%%%%%%%%%%%%%%%%%%%%%%%%%%%%%%%%%%%%%%%%%%%%
% Linear Algebra
%%%%%%%%%%%%%%%%%%%%%%%%%%%%%%%%%%%%%%%%%%%%%
\newcommand{\transpose}[1]{{#1}^{\text{T}}}
\newcommand{\norm}[1]{\lVert{#1}\rVert}
\newcommand\aug{\fboxsep=-\fboxrule\!\!\!\fbox{\strut}\!\!\!}

%%%%%%%%%%%%%%%%%%%%%%%%%%%%%%%%%%%%%%%%%%%%%
% Number Theory
%%%%%%%%%%%%%%%%%%%%%%%%%%%%%%%%%%%%%%%%%%%%%
\DeclareMathOperator{\Spec}{Spec}
\newcommand{\av}[1]{\lvert#1\rvert}
\DeclareMathOperator{\divides}{\mid}
\DeclareMathOperator{\ndivides}{\nmid}


\graphicspath{{images/}}

\newtheorem*{claim}{claim}
\newtheorem*{observation}{Observation}
\newtheorem*{warning}{Warning}
\newtheorem*{question}{Question}
\newtheorem{remark}[theorem]{Remark}

\newenvironment*{subproof}[1][Proof]
{\begin{proof}[#1]}{\renewcommand{\qedsymbol}{$\diamondsuit$} \end{proof}}

\mode<presentation>
{
  \usetheme{Singapore}
  % or ...

  \setbeamercovered{invisible}
  % or whatever (possibly just delete it)
}


\usepackage[english]{babel}
% or whatever

\usepackage[latin1]{inputenc}
% or whatever

\usepackage{times}
\usepackage[T1]{fontenc}
% Or whatever. Note that the encoding and the font should match. If T1
% does not look nice, try deleting the line with the fontenc.

\title{Ideals, Varieties and Algorithms \\ Chapter 2}
\subtitle{Bloom Math Reading Group \\ July 11, 2022 \\ Google}
\author{Mitch Rudominer}
\date{}



% If you wish to uncover everything in a step-wise fashion, uncomment
% the following command:

\beamerdefaultoverlayspecification{<+->}

\begin{document}

\begin{frame}
  \titlepage
\end{frame}



%%%%%%%%%%%%%%%%%%%%%%%%%%%%%%%%%%%%%%%%%%%%%%%%%%%%%%%%%%%%%%%%%

\begin{frame}{Review: Different types of rings}

\begin{center}
\includegraphics[width=\textwidth]{summary}
\end{center}

\end{frame}


%%%%%%%%%%%%%%%%%%%%%%%%%%%%%%%%%%%%%%%%%%%%%%%%%%%%%%%%%%%%%%%%%

\begin{frame}{Review: The status of $\R[x,y,z]$}

\begin{itemize}
  \item Last time we explained why $\R[x,y,z]$ is harder to work with than $\R[x]$.
  \item $\R[x,y,z]$ is not a PID so we don't have the Euclidean algorithm or the B\'{e}zout property.
  \item Now let's turn to what we do have. Let's see how we can work with $\R[x,y,z]$ despite it not being as nice as $\R[x]$.
  \item We will study Gr\"{o}bner bases which are especially nice bases for ideals of $\R[x,y,z]$.
  \item Let's start by seeing that in $\R[x,y,z]$ ideals have a finite basis at all.
\end{itemize}

\end{frame}


%%%%%%%%%%%%%%%%%%%%%%%%%%%%%%%%%%%%%%%%%%%%%%%%%%%%%%%%%%%%%%%%%

\begin{frame}{Noetherian Rings}

\begin{itemize}
  \item \textbf{Definition.} A ring $R$ is \emph{Noetherian} iff every ideal of $R$ is finitely generated.
  \item In particular every PID is Noetherian.
  \item \textbf{Hilbert Basis Theorem.} Let $R$ be a Noetherian ring. Then $R[x]$ is Noetherian.
  \item \textbf{Corollary.} $\R[x,y,z]$ is Noetherian.
  \item An example of a UFD that is non-Noetherian: $\R[x_1, x_2, \cdots]$.
\end{itemize}

\end{frame}


%%%%%%%%%%%%%%%%%%%%%%%%%%%%%%%%%%%%%%%%%%%%%%%%%%%%%%%%%%%%%%%%%

\begin{frame}{Diagram: Noetherian Rings}

\begin{center}
\includegraphics[width=\textwidth]{noetherian}
\end{center}

\end{frame}


%%%%%%%%%%%%%%%%%%%%%%%%%%%%%%%%%%%%%%%%%%%%%%%%%%%%%%%%%%%%%%%%%

\begin{frame}{The Ascending Chain Condition}

\begin{itemize}
  \item \textbf{Definition.} A ring $R$ has the \emph{ascending chain condition} iff it does not have an infinite
  ascending chain of ideals:
  \item $I_0 \subset  I_1 \subset I_2 \subset \cdots$
  \item such that each $I_n$ is a proper subset of $I_{n+1}$.
  \item \textbf{Theorem.} $R$ is Noetherian iff it has the ascending chain condition.
  \item \textbf{proof.} First suppose that $R$ is Noetherian and we will show it has the ACC.
  \item  Suppose $I_0 \subseteq  I_1 \subseteq I_2 \subseteq \cdots$ is an ascending chain of ideals in $R$.
  \item Let $I=\union_n I_n$. Since $R$ is Noetherian $I$ is finitely-generated.
  \item Let $G$ be a finite set of generators of $I$. Let $n$ be large enough that every element of $G$ is in $I_n$.
  \item Then $I_n=I$ so it is not the case that $I_n$ is a proper subset of $I_{n+1}$.
\end{itemize}

\end{frame}


%%%%%%%%%%%%%%%%%%%%%%%%%%%%%%%%%%%%%%%%%%%%%%%%%%%%%%%%%%%%%%%%%

\begin{frame}{proof continued}

\begin{itemize}
  \item Conversely, suppose that $R$ is not Noetherian and let $I$ be an ideal that is not finitely-generated.
  \item Let $p_0$ be any elelemnt of $I$ and let $I_0=(p_0)\subset I$.
  \item Since $I$ is not finitely-generated, $I_0\not= I$, so let $p_1 \in I\setminus I_0$ and let $I_1 = (p_0,p_1)$.
  \item We have $I_0 \subset I_1 \subset I$ and since $I$ is not finitely-generated there is a $p_2\in I\setminus I_1$. Let $I_2=(p_0,p_1,p_2)$.
  \item Continuing in this way we build an ascending chain $I_0 \subset  I_1 \subset I_2 \subset \cdots$
  \item such that each $I_n$ is a proper subset of $I_{n+1}$. So $R$ does not have the ACC. $\qed$.

\end{itemize}

\end{frame}


%%%%%%%%%%%%%%%%%%%%%%%%%%%%%%%%%%%%%%%%%%%%%%%%%%%%%%%%%%%%%%%%%

\begin{frame}{Hilbert's Basis Theorem}

\begin{itemize}
  \item \textbf{Theorem} Let $R$ be a Noetherian ring. Then $R[x]$ is Noetherian.
  \item \textbf{proof.} Suppose $R[x]$ is not Noetherian and we will show that $R$ is not.
  \item Let $I$ be an ideal of $R[x]$ that is not finitely-generated.
  \item Let $f_0$ be an element of $I$ of least possible degree.
  \item Given $f_0,\cdots, f_k$, let $f_{k+1} \in I\setminus (f_1,\cdots f_k)$ be an element of least possible degree.
  \item For each $k$ let $n_k$ be the degree and $a_k$ be the leading coefficient of $f_k$. $n_1 \leq n_2 \leq \cdots$ is a non-decreasing sequence of integers.
  \item We claim that $(a_1) \subset (a_1, a_2) \subset (a_1, a_2, a_3) \subset \cdots$ is an ascending sequence of ideals of $R$
  \item such that each ideal is a proper subset of the next one.
  \item Proving this will show that $R$ does not have the ACC.
\end{itemize}
\end{frame}


%%%%%%%%%%%%%%%%%%%%%%%%%%%%%%%%%%%%%%%%%%%%%%%%%%%%%%%%%%%%%%%%%

\begin{frame}{proof continued}

\begin{itemize}
  \item For suppose that $(a_1,\cdots, a_k) = (a_1,\cdots, a_k, a_{k+1})$.
  \item Then $a_{k+1} = \sum_{i=1}^k b_i a_i$ for some $b_i\in R$.
  \item Let $g=f_{k+1} - \sum_{i=1}^kb_i x^{n_{k+1} - n_i} f_i$.
  \item $g$ is a linear combination of the $f_i$ so $g\in I$.
  \item $g \in I \setminus (f_1,\cdots,f_k)$, because $f_{k+1} \notin (f_1,\cdots,f_k)$.
  \item But $g$ is of
  lower degree than $f_{k+1}$, contradicting the choice of $f_{k+1}$.
  \item For example suppose $n_1 = 1, n_2 =2, n_3 = 3$ and $(a_1,a_2,a_3) = (a_1, a_2)$.
  \item Write $a_3 = b_1 a_1 + b_2 a_2$.
  \item Let $g=f_3 - (b_1 x^2 f_1 + b_2 x f_2)$.
  \item The $x^3$ terms cancel so $g$ is of degree 2.
  \item But $g \in I \setminus (f_1, f_2)$ and $f_3$ was chosen to have least possible degree.
  \item Contradiction. $\qed$.
\end{itemize}
\end{frame}

%%%%%%%%%%%%%%%%%%%%%%%%%%%%%%%%%%%%%%%%%%%%%%%%%%%%%%%%%%%%%%%%%

\begin{frame}{Varieties}

\begin{itemize}
  \item A quick detour into the geometric side of algebraic geometry.
  \item We just proved that $\R[x,y,z]$ is Noetherian so every ideal is finitely generated.
  \item But the problem is that we have not given any examples of ideals besides $(f_1,\cdots, f_n)$. These are obviously finitely generated,
  so the Hilbert Basis Theorem is not that interesting yet.
  \item I want to give more examples of ideals and getting a little bit into algebraic geometry is a good way to do that.
  \item \textbf{Definition.} Let $f_1,\cdots f_n \in \R[x,y,z]$. Then the \emph{variety} of $f_1,\cdots f_n$ is
  \item $\mathbf{V}(f_1,\cdots,f_n) = \setof{p\in\R^3}{f_1(p)=0 \text{ and } f_2(p)=0 \text{ and } \cdots f_n(p)=0}$.
  \item This is a surface in $\R^3$ or a curve or a finite set of points or the empty set or all of $\R^3$.
\end{itemize}
\end{frame}


%%%%%%%%%%%%%%%%%%%%%%%%%%%%%%%%%%%%%%%%%%%%%%%%%%%%%%%%%%%%%%%%%

\begin{frame}{More abut Varieties}

\begin{itemize}
  \item $\mathbf{V}(z-x^2-y)$ is a paraboloid.
  \item $\mathbf{V}(x,y,z)$ is the origin.
  \item $\mathbf{V}(x^2-1,y^2-1,z^2-1)$ consists of eight points.
  \item $\mathbf{V}(x,y)$ is the $z$-axis
  \item $\mathbf{V}(x^2 - 1,x^2 - 2)$ is the empty set
  \item $\mathbf{V}(1)$ is the empty set
  \item $\mathbf{V}(0)$ is all of $\R^3$.
  \item The union and the intersection of two varieties is a variety.
  \item In general, if someone gives us a list of polynomials, it is not trivial to look at the polynomials and
  answer questions about the variety of those polynomials.
  \item e.g. Is the variety infinite? If so what is its dimension?
  \item Is the variety finite? If so what is its cardinality?

\end{itemize}
\end{frame}

%%%%%%%%%%%%%%%%%%%%%%%%%%%%%%%%%%%%%%%%%%%%%%%%%%%%%%%%%%%%%%%%%

\begin{frame}{Varieties and Ideals}

\begin{itemize}
  \item Let $I$ be an ideal of $\R[x,y,z]$. The variety of $I$ is
  \item $\mathbf{V}(I) = \setof{p\in\R^3}{f(p)=0\text{ for all } f\in I}$.
  \item \textbf{Claim.} $\mathbf{V}(f_1,\cdots,f_n) = \mathbf{V}(I)$ where $I = (f_1,\cdots,f_n)$.
  \item In other words, the variety of a set of polynomials is the same as the variety of the ideal generated by those polynomials.
  So we should really think about varieties as being associated with ideals as opposed to being associated with a particular finite list of polynomials.
  \item \textbf{proof of claim.} Let $p\in\R^3$.
  \item $p\in \textbf{V}(f_1,\cdots,f_n)$ iff $f_1(p) = \cdots = f_n(p) = 0$.
  \item $p\in \textbf{V}(I)$ iff $f(p)=0$ whenever $f=c_1f_1 + c_2f_2 + \cdots + c_nf_n$ for any values of $c_1,\cdots,c_n$.
  \item These two are obviously equivalent. $\qed$.
\end{itemize}
\end{frame}


%%%%%%%%%%%%%%%%%%%%%%%%%%%%%%%%%%%%%%%%%%%%%%%%%%%%%%%%%%%%%%%%%

\begin{frame}{Varieties and Gr\"{o}bner bases}

\begin{itemize}
  \item If $(f_1,\cdots,f_n) = (g_1,\cdots,g_m)$ then $\mathbf{V}(f_1,\cdots,f_n) = \mathbf{V}(g_1,\cdots,g_m)$.
  \item i.e. Given $f_1,\cdots,f_n$, if we want to answer some questions about $\mathbf{V}(f_1,\cdots,f_n)$,
  we can replace $f_1,\cdots,f_n$ with a different basis for the same ideal, $g_1,\cdots,g_m$, and use
  $g_1,\cdots,g_m$ to answer questions about the variety.
  \item It turns out that a Gr\"{o}bner basis is a good basis to use for understanding ideals.
  \item This explains why Gr\"{o}bner bases are useful in algebraic geometry and gives another motivation for studying Gr\"{o}bner bases.
  \item If $g_1,\cdots,g_m$ is a Gr\"{o}bner then it is easier to look at $g_1,\cdots,g_m$ and answer those questions I mentioned earlier
  about $\mathbf{V}(g_1,\cdots,g_m)$.

\end{itemize}
\end{frame}


%%%%%%%%%%%%%%%%%%%%%%%%%%%%%%%%%%%%%%%%%%%%%%%%%%%%%%%%%%%%%%%%%

\begin{frame}{Towards Gr\"{o}bner}

\begin{itemize}
  \item We want to describe remainders in a more sophisticated way.
  \item \textbf{Theorem.} Let $a,b\in\Z$, $b\not=0$.
  \item Then there exists unique integers $q$ and $r$ such that
  \item $a = qb +r$ and $0\leq r < b$.
  \item Example: $17 = 3 \cdot 5 + 2$.
  \item \textbf{Theorem.} Let $f,g\in\R[x]$, $g\not=0$.
  \item Then there exists unique polynomials $q,r\in\R[x]$ such that
  \item $f = qg +r$ and $0\leq \deg(r) < \deg(b)$.
  \item Example: $x^3 + x^2 + 2 = (x+1)(x^2+1) + ( -x + 1)$.
\end{itemize}
\end{frame}

%%%%%%%%%%%%%%%%%%%%%%%%%%%%%%%%%%%%%%%%%%%%%%%%%%%%%%%%%%%%%%%%%
\begin{frame}{Congruence Classes}

\begin{itemize}
  \item Let $I$ be an ideal of the ring $R$. Then $I$ induces an equivalence relation and thus a partition of $R$.
  \item \textbf{Definition.} Let $a,b\in R$. Then $a \equiv b \pmod I$ iff $a - b \in I$.
  \item We say that $a$ and $b$ are congruent mod $I$. Like every equivalence relation, this induces a partition of
  $R$. We call the equivalence classes \emph{congruence classes.}
  \item Example: let $I\subset \Z$ be the ideal of multiples of 5. $I=(5)$.
  \item If $a,b\in \Z$ then $a\equiv b \pmod I$ iff $a-b$ is divisible by 5.
  \item Normally we write $a\equiv b \pmod 5$.
  \item $a\bmod 5 = b \bmod 5$.
  \item Example: because $17 = 3 \cdot 5 + 2$ we have $17 \equiv 2 \bmod 5$.
  \item 2 is the least positive integer that is congruent to 17 mod 5.
  \item Point: The remainder selects a canonical representative of each congruence class.
\end{itemize}
\end{frame}

%%%%%%%%%%%%%%%%%%%%%%%%%%%%%%%%%%%%%%%%%%%%%%%%%%%%%%%%%%%%%%%%%

\begin{frame}{Diagram: Congruence Classes Mod 5}

\begin{center}
\includegraphics[width=\textwidth]{congruence-mod-5}
\end{center}

\end{frame}


%%%%%%%%%%%%%%%%%%%%%%%%%%%%%%%%%%%%%%%%%%%%%%%%%%%%%%%%%%%%%%%%%

\begin{frame}{Congruence Classes in $\R[x]$}

\begin{itemize}
  \item Let $I \subset \R[x]$ be the ideal of multiples of $h = x^2+1$.
  \item $I = (x^2+1)$.
  \item If $f,g\in\R[x]$ then $f\equiv g \pmod I$ iff $f-g$ is divisible by $h$.
  \item Example: Because $x^3 + x^2 + 2 = (x+1)(x^2+1) + ( -x + 1)$, we have
  \item $x^3 + x^2 + 2 \equiv -x +1 \pmod {h}$.
  \item $(-x+1)$ is a polynomial of least degree that is congruent to $x^3 + x^2 + 2$ mod $h$.
  \item Point: The remainder selects a canonical representative of each congruence class.
\end{itemize}
\end{frame}

%%%%%%%%%%%%%%%%%%%%%%%%%%%%%%%%%%%%%%%%%%%%%%%%%%%%%%%%%%%%%%%%%
\begin{frame}{Diagram: Congruence Classes Mod $x^2 + 1$}

\begin{center}
\includegraphics[width=\textwidth]{congruence-mod-h}
\end{center}

\end{frame}


%%%%%%%%%%%%%%%%%%%%%%%%%%%%%%%%%%%%%%%%%%%%%%%%%%%%%%%%%%%%%%%%%

\begin{frame}{Canonical representative of each congruence class}

\begin{itemize}
  \item Notice that in the example of $(5)$, the remainder is the unique element of its congruence class that is in the set $\singleton{0,1,2,3,4}$.
  \item \textbf{proof.} Suppose $r,s\in \singleton{0,1,2,3,4}$ and $r> s$. Then $r-s\in \singleton{1,2,3,4}$. So $r-s$ is not divisible by 5. $\qed$.
  \item In the example of $(x^2+1)$, the remainder is the unique element of its congruence class whose degree is less than 2.
  \item \textbf{proof.} Suppose $r\not=s\in\R[x]$ and have degree less than 2. Then $r-s$ has degree less than 2 and $r-s\not=0$. So $r-s$ is not divisible
  by $(x^2+1)$.
\end{itemize}

\end{frame}


%%%%%%%%%%%%%%%%%%%%%%%%%%%%%%%%%%%%%%%%%%%%%%%%%%%%%%%%%%%%%%%%%

\begin{frame}{How to select a canonical representatives in $\R[x,y,z]$?}

\begin{itemize}
  \item Now let's consider the ring $\R[x,y,z]$. Let $I=(f_1,\cdots f_n)$ be an ideal. Consider this problem:
  \item Let $g\in\R[x,y,z]$. We want some notion of ``the unique remainder when we divide $g$ by $f_1,\cdots, f_n$.".
  \item $g = h_1 f_1 + h_2 f_2 + \cdots h_n f_n + r$.
  \item How should we define the ``unique remainder'' $r$? How should we pick the canonical representative of each congruence class?
  \item The remainder should satisfy some property such that if $r \not= s \in \R[x,y,z]$ and both satisfy the property, then $r-s\notin I$.
  \item What should this property be?
\end{itemize}

\end{frame}

%%%%%%%%%%%%%%%%%%%%%%%%%%%%%%%%%%%%%%%%%%%%%%%%%%%%%%%%%%%%%%%%%

\begin{frame}{Diagram: Congruence Classes Mod $f_1,f_2,f_3,f_4$ in $\R[x,y,z]$}

\begin{center}
\includegraphics[width=\textwidth]{congruence-mod-fn}
\end{center}

\end{frame}


%%%%%%%%%%%%%%%%%%%%%%%%%%%%%%%%%%%%%%%%%%%%%%%%%%%%%%%%%%%%%%%%%

\begin{frame}{Leading Terms}

\begin{itemize}
  \item Look again at remainders in $\R[x]$.
  \item $g=hf + r$
  \item Remainder uniqueness property: degree of $r$ is less than the degree of $f$.
  \item Rephrase (1): Each term of $r$ is not divisible by the leading term of $f$.
  \item Example: $x^3 + x^2 + 2 = (x+1)(x^2+1) + ( -x + 1)$.
  \item Each term of $-x+1$ is not divisible by $x^2$.
  \item Rephrase (2): Each term of $r$ is not divisible by the leading term of any non-zero polynomial in the ideal $(f)$.
  \item Example: Each term of $-x+1$ is not divisible by the leading term of any non-zero polynomial in the ideal $(x^2)$.
  \item Rephrase (2) generalizes to $\R[x,y,z]$.
  \item Rephrase (1) generalizes to Gr\"{o}bner bases.
\end{itemize}

\end{frame}

%%%%%%%%%%%%%%%%%%%%%%%%%%%%%%%%%%%%%%%%%%%%%%%%%%%%%%%%%%%%%%%%%
\begin{frame}{Unique Remainders in $\R[x,y,z]$}

\begin{itemize}
  \item \textbf{Theorem.} Let $g, f_1,\cdots f_n \in \R[x,y,z]$.
  \item There is a unique $r\in\R[x,y,z]$ such that
  \item $g = h_1 f_1 + \cdots h_n f_n + r$ for some $h_1\cdots h_n$ and
  \item each term of $r$ is not divisible by the leading term of any non-zero polynomial in the ideal $(f_1,\cdots,f_n)$.
  \item \textbf{Definition.} $f_1,\cdots f_n \in \R[x,y,z]$ is a Gr\"{o}bner basis iff
  \item for every non-zero $g\in (f_1,\cdots f_n)$, the leading term of $g$ is divisible
  by the leading term of one of $f_1,\cdots,f_n$.
  \item If $f_1,\cdots f_n$ is a Gr\"{o}bner basis then the uniqueness condition for $r$ in the above theorem can be expressed as:
  \item each term of $r$ is not divisible by the leading term of any of $f_1,\cdots,f_n$.
  \item This will give us an algorithm for computing $r$.
\end{itemize}

\end{frame}

%%%%%%%%%%%%%%%%%%%%%%%%%%%%%%%%%%%%%%%%%%%%%%%%%%%%%%%%%%%%%%%%%

\begin{frame}{Monomial Orderings}

\begin{itemize}
  \item But which is the leading term?
  \item Lex: $x^4yz^2 + x^3y^3z^2 + x^2y^5z$
  \item GrLex: $x^3y^3z^2 + x^2y^5z + x^4yz^2$
  \item GRevLex: $x^2y^5z +x^3y^3z^2 +  x^4yz^2$
  \item \textbf{Definition} A \emph{monomial ordering} for $\R[x,y,z]$ is a
  well-ordering $<$ of 3-tuples of non-negative integers such that $\alpha < \beta \Implies \alpha + \gamma < \beta + \gamma$
  \item A monomial ordering on 3-tuples of exponents $\alpha=(\alpha_0,\alpha_1,\alpha_2)$ naturally gives us an ordering on
  monomials $u = x^{\alpha_0}y^{\alpha_1}z^{\alpha_2}$ with the property that $u<v \Implies uw < vw$.
  \item All discussions of leading terms and Gr\"{o}bner bases are relative to some fixed choice of a monomial ordering.
  \item We'll use GrLex for the remainder of this presentation.
\end{itemize}

\end{frame}

%%%%%%%%%%%%%%%%%%%%%%%%%%%%%%%%%%%%%%%%%%%%%%%%%%%%%%%%%%%%%%%%%
\begin{frame}{Gr\"{o}bner bases existence}

\begin{itemize}
  \item \textbf{Theorem.} Let $I$ be an ideal of $\R[x,y,z]$. Then $I$ has a Gr\"{o}bner basis.
  \item \textbf{proof.} Let $J$ be the ideal generated by the leading terms of the elements if $I$.
  \item Since $\R[x,y,z]$ is Noetherian, $J$ is finitely-generated.
  \item Let $f_1,\cdots,f_n\in I$ be such that their leading terms generate $J$.
  \item Then $f_1,\cdots, f_n$ is a Gr\"{o}bner basis for $I$. This follows from the following
  \item \textbf{Lemma.} Let $J$ be an ideal of $\R[x,y,z]$ that is generated by monomials $u_1,\cdots u_n$.
  \item Then every monomial in $J$ is actually a multiple of one of $u_1,\cdots u_n$.
  \item\textbf{proof.} If $u$ is a monomial and $u=h_1u_1 + \cdots h_n u_n$ then, after collecting terms, each
  of the terms on the right-hand side is divisible by one of $u_1,\cdots u_n$, so that is also true of the left-hand-side. $\qed$.
\end{itemize}

\end{frame}

%%%%%%%%%%%%%%%%%%%%%%%%%%%%%%%%%%%%%%%%%%%%%%%%%%%%%%%%%%%%%%%%%
\begin{frame}{Proof of the Unique Remainder Theorem}

\begin{itemize}
  \item \textbf{Theorem.} Let $g, f_1,\cdots f_n \in \R[x,y,z]$.
  \item There is a unique $r\in\R[x,y,z]$ such that
  \item $g = h_1 f_1 + \cdots h_n f_n + r$ for some $h_1\cdots h_n$ and
  \item each term of $r$ is not divisible by the leading term of any non-zero polynomial in the ideal $(f_1,\cdots,f_n)$.
  \item \textbf{proof.} Uniqueness is easy. Suppose $r_1$ and $r_2$ were two different remainders.
  \item The leading term of $r_1 - r_2$ is a scalar multiple of one of the terms of $r_1$ or of $r_2$.
  \item So the leading term of $r_1 - r_2$ is not divisible by the leading term of any non-zero polynomial in the ideal $(f_1,\cdots,f_n)$.
  \item In particular $r_1 - r_2$ is not in the ideal $(f_1\cdots f_n)$.
  \item This is a contradiction since $r_1 \equiv r_2 \equiv g \pmod {(f_1,\cdots,f_n)}$.
\end{itemize}

\end{frame}

%%%%%%%%%%%%%%%%%%%%%%%%%%%%%%%%%%%%%%%%%%%%%%%%%%%%%%%%%%%%%%%%%
\begin{frame}{Division Algorithm for $\R[x,y,z]$}

\begin{itemize}
  \item For proof of existence, by the previous theorem we may assume $f_1,\cdots, f_n$ is a Gr\"{o}bner basis.
  \item We will describe an algorithm by which we will produce $h_1,\cdots h_n$ and $r$ such that
  \item $g = h_1 f_1 + \cdots h_n f_n + r$ and
  \item each term of $r$ is not divisible by the leading term of any of $f_1,\cdots,f_n$.
  \item Let $h_1^0 = \cdots = h_n^0 = 0$.
  \item Let $s^0 = g$. Let $r^0 = 0$.
  \item Then $g = h_1^0 f_1 + \cdots + h_n^0 f_n + s^0 + r^0$
  \item and (vacuously) every term of $r^0$ is not divisible by any of the leading terms of $f_1,\cdots, f_n$.
\end{itemize}

\end{frame}

%%%%%%%%%%%%%%%%%%%%%%%%%%%%%%%%%%%%%%%%%%%%%%%%%%%%%%%%%%%%%%%%%
\begin{frame}{Division Algorithm for $\R[x,y,z]$}

\begin{itemize}
  \item Suppose we $h_1^k, \cdots, h_n^k$ and $s^k$ and $r^k$ such that
  \item $g = h_1^k f_1 + \cdots + h_n^k f_n + s^k + r^k$
  \item and every term of $r^k$ is not divisible by any of the leading terms of $f_1,\cdots, f_n$.
  \item Step $k+1$ of the algorithm will produce $h_1^{k+1} , \cdots,  h_n^{k+1}$ and $s^{k+1}$ and $r^{k+1}$ such that
  \item $g = h_1^{k+1} f_1 + \cdots + h_n^{k+1} f_n + s^{k+1} + r^{k+1}$ and
  \item every term of $r^{k+1}$ is not divisible by any of the leading terms of $f_1,\cdots, f_n$ and
  \item the leading term of $s^{k+1}$ is less than the leading term of $s^k$ in the monomial ordering.
  \item Since the monomial ordering is a well-ordering, the algorithm must terminate with some $k$ such that $s^k = 0$.
  \item For this $k$ we have $g = h_1^k f_1 + \cdots + h_n^k f_n + r^k$ and we are done.
\end{itemize}

\end{frame}

%%%%%%%%%%%%%%%%%%%%%%%%%%%%%%%%%%%%%%%%%%%%%%%%%%%%%%%%%%%%%%%%%
\begin{frame}{Step $k+1$ of the algorithm}

\begin{itemize}
  \item Let $u$ be the leading term of $s_k$.
  \item If $u$ is not divisible by the leading term of any of $f_1,\cdots,f_n$ then we move $u$ from $s$ to $r$.
  \item We let $s^{k+1} = s^k - u$ and $r^{k+1} = r^k + u$ and $h_i^{k+1} = h_i^k$ for $i=1,\cdots,n$.
  \item Then our inductive hypothesis remains true:
  \item $g = h_1^{k+1} f_1 + \cdots + h_n^{k+1} f_n + s^{k+1} + r^{k+1}$ and
  \item every term of $r^{k+1}$ is not divisible by any of the leading terms of $f_1,\cdots, f_n$ and
  \item the leading term of $s^{k+1}$ is less than the leading term of $s^k$ in the monomial ordering since the leading
  term of $s^{k+1}$ is the second leading term of $s^k$
\end{itemize}

\end{frame}

%%%%%%%%%%%%%%%%%%%%%%%%%%%%%%%%%%%%%%%%%%%%%%%%%%%%%%%%%%%%%%%%%
\begin{frame}{Step $k+1$ continued}

\begin{itemize}
  \item Now suppose that $u$ is divisible by the leading term of one of the $f_i$.
  \item For simplicity suppose $u$ is divisible by $v$, the leading term of $f_1$.
  \item We will move $u$ out of $s$ and into $h_1 f_1$.
  \item Write $u = vw$. We will add $w$ to $h_1$ and subtract terms from $s^k$ to compensate.
  \item Let $h_1^{k+1} = h_k + w$ and $s_{k+1} = s_k - wf_1$. Everything else stays the same:
  \item $r_{k+1} = r_k$ and $h_i^{k+1} = h_i^k$ for $i = 2\cdots n$.
  \item Then our inductive hypothesis remains true.
  \item $g = (h_1^k + w) f_1 + \cdots + h_n^k f_n + s^k - wf_1 + r^k$.
  \item $g = h_1^{k+1} f_1 + \cdots + h_n^{k+1} f_n + s^{k+1} + r^{k+1}$.
\end{itemize}

\end{frame}

%%%%%%%%%%%%%%%%%%%%%%%%%%%%%%%%%%%%%%%%%%%%%%%%%%%%%%%%%%%%%%%%%
\begin{frame}{finishing the proof}

\begin{itemize}
  \item  Every term of $r^{k+1}$ is not divisible by any of the leading terms of $f_1,\cdots, f_n$ and
  \item \textbf{Claim.} The leading term of $s^{k+1}$ is less than the leading term of $s^k$ in the monomial ordering.
  \item \textbf{proof of claim.}  $s_{k+1} = s_k - wf_1 = s_k - wv + w(v - f_1)$.
  \item $wv = u = $ the leading term of $s_k$.
  \item Since $v$ is the leading term of $f_1$, all of the terms of $v -f_1$ are less than $v$ in the monomial ordering.
  \item So all of the terms of $w(v-f_1)$ are less than $wv = u$.
  \item To form $s_{k+1}$ we removed the leading term $u$ from $s_k$ and added in other terms that were all less than $u$.
  \item So all of the terms of $s_{k+1}$ are less than $u$. $\qed$.
\end{itemize}
\end{frame}

%%%%%%%%%%%%%%%%%%%%%%%%%%%%%%%%%%%%%%%%%%%%%%%%%%%%%%%%%%%%%%%%%




\end{document}
