%%Prosper slide presentation template
%%Build using latex+dvipdf
%%Note that the prosper package must be separately installed
%%See http://prosper.sourceforge.net/
\documentclass[pdf,final]{prosper}

\usepackage{amsmath,amssymb,latexsym,eucal,amsthm}
\include{MathDefs}
%%%%%%%%%%%%%%%%%%%%%%%%%%%%%%%%%%%%%%%%%%%%%%%%%%%%%%%%%%%%%%%%%%%%%%%%%%%
%%  Theorem-Like Declarations
%%%%%%%%%%%%%%%%%%%%%%%%%%%%%%%%%%%%%%%%%%%%%%%%%%%%%%%%%%%%%%%%%%%%%%%%%%

\newtheorem{theorem}{Theorem}[section]
\newtheorem{lemma}[theorem]{Lemma}
\newtheorem{corollary}[theorem]{Corollary}
\newtheorem{proposition}[theorem]{Proposition}


\theoremstyle{definition}

\newtheorem{definition}[theorem]{Definition}
\newtheorem{conjecture}[theorem]{Conjecture}
\newtheorem{remark}[theorem]{Remark}
\newtheorem{example}[theorem]{Example}
\newtheorem{remarks}[theorem]{Remarks}
\newtheorem{notation}[theorem]{Notation}

\theoremstyle{remark}

%\newtheorem*{note}{Note}
\newtheorem*{warning}{Warning}
\newtheorem*{question}{Question}
\newtheorem*{fact}{Fact}
\newtheorem*{problem}{Problem}

\newenvironment*{subproof}[1][Proof]
{\begin{proof}[#1]}{\renewcommand{\qedsymbol}{$\diamondsuit$} \end{proof}}
 
 \newenvironment*{case}[1]
{\textbf{Case #1.  }\itshape }

\newenvironment*{claim}[1][Claim]
{\textbf{#1.  }\itshape }



\newcommand{\skipsmall}{\vspace{1em}}
\newcommand{\skipmed}{\vspace{2em}}
\newcommand{\skipbig}{\vspace{3em}}
\newcommand{\skipsmallminus}{\vspace{-1em}}

%\usepackage[T1]{fontenc}
%\usepackage[latin1]{inputenc}
%\usepackage{moreverb}
%\usepackage{graphicx}

\title{The Determinacy of Infinitely Long Two-Player Games}
\subtitle{Search/Pathways Friday Colloquium}

\author{Mitch Rudominer}
\institution{BEA}
\email{Mitch.Rudominer@bea.com}

% You can use this macro to put a caption at the bottom of each slide.
%\slideCaption{}
% \Logo This allows you to place a logo on each slide at a specified position.

% This defines the type of transition that should occur between slides.
% \DefaultTransition

\begin{document}

\maketitle

\begin{slide}{Abstract}
We discuss games in which two players, 
called $\ONE$ and  $\TWO$, alternately pick an integer at each move, and the
play of the game lasts for infinitely many moves. 
$\ONE$ wins the game if the infinite sequence of integers produced by the play of the game is an element of a given set, 
called the payoff set. Otherwise $\TWO$ wins. 

\skipsmall

We will define what it means for the players to have a winning strategy. 
The game is called determined if one of the two players has a winning strategy. 

\skipsmall

Whether or not the game is determined is related to the complexity of the payoff set. 
If the payoff set is Borel, then the game is determined. For more complex sets, the determinacy of the game is closely related to the 
existence of large cardinals.
\end{slide}

\begin{slide}{Baire Space}
\begin{definition}
$\omega=\singleton{0,1,2,3,\dots}$ is the set of non-negative integers.
\newline
$\pre{\omega}{\omega}$ is the set of infinite sequences of integers.
\newline
$\pre{\omega}{\omega}$ is known as \emph{Baire Space}.
\newline
If $x\in\BaireSpace$ and $n\in\omega$ then $x(n)$ means the integer at position
$n$ in $x$. Thus $x=\angles{x(0),x(1),x(2),\dots}$.
\end{definition}

\skipsmall

\begin{example}
Let $x=\angles{1,0,1,0,1,0,\dots}$.
\newline
Let $y=\angles{1,0,2,0,0,3,0,0,0,4,0,0,0,0,5,\dots}$.
\newline
Let $z$ be the sequence consiting of the digits of $\pi$.
\newline
Then $x$, $y$, and $z$ are all elements of Baire Space.
\end{example}

\skipsmall

\begin{example}
Let $A=\setof{s\in\BaireSpace}{s\text{ starts with a 1}}$.
\newline
Let $B=\setof{s\in\BaireSpace}{s\text{ has infinitely many 1's}}$.
\newline
Then $A\subset\BaireSpace$ and $B\subset\BaireSpace$ and $x\in A$ and $x\in B$
and $y\in A$ but $y\notin B$.
\end{example}


\end{slide}

\begin{slide}{Trees}
\begin{definition}
$\FiniteSeq$ is the set of \textbf{finite} sequences from $\omega$.
\newline
$\emptyset$ is the unique element of $\FiniteSeq$ of length zero.
\newline
If $s$ and $t$ are elements of $\FiniteSeq$ then we write $s\subset t$ to mean
that $s$ is a subsequence $t$.
\end{definition}

\skipsmall

\begin{example}
Let $s=\angles{128,256}$, $t=\angles{128,256,512,1024}$, then $s\in\FiniteSeq$
and $t\in\FiniteSeq$ and $\emptyset\subset s\subset t$.
\end{example}

\skipsmall

\begin{definition}
A \textbf{tree} is a set $T\subseteq\FiniteSeq$ such that $T$ is closed under
subsequence. That is, if $s\in T$ and $t\subset s$ then $t\in T$.
\end{definition}

\skipsmall

\begin{example}
Let
$T=\singleton{\emptyset,\angles{0},\angles{0,0},\angles{0,1},\angles{0,1,0},\angles{0,1,2},\angles{0,1,3}}$. 
Then $T$ is a tree.
\end{example}
\end{slide}

\begin{slide}{Branches of Trees}
\begin{definition}
If $s\in\FiniteSeq$ then $T_s=\setof{t\in\FiniteSeq}{t\subseteq s \text{ or }
s\subseteq t}$. $T_s$ is the \textbf{tree through $s$}.
\end{definition}

\skipsmall

\begin{definition}
If $x\in\BaireSpace$ and $n\in\omega$ then $x\restr n =
\angles{x(0),x(1),\dots,x(n-1)}$.
\newline$x\restr 0=\emptyset$.
\end{definition}

\skipsmall

\begin{definition}
If $T$ is a tree then a \textbf{branch} of $T$ is an $x\in\BaireSpace$ such that
$x\restr n \in T$ for each $n\in\omega$.
\newline
$[T]$ is the set of all branches of $T$.
\end{definition}

\skipsmall

So $[T_s]$ is the set of all $x\in\BaireSpace$ that extend $s$.


\end{slide}


\begin{slide}{Games}
\begin{definition}
Let $A\subseteq\BaireSpace$. The \textbf{game} $G(A)$ is defined as
follows.

\skipsmall

There are two players named $\ONE$ and $\TWO$. The game has $\omega$
rounds. 

\skipsmall

At round $n$, if $n$ is even, then $\ONE$ must play some integer
$x(n)\in\omega$, if $n$ is odd then $\TWO$ must play $x(n)$. 

\skipsmall

\begin{tabular}{lccccccr}
$\ONE$: & $x(0)$ & & $x(2)$ & & $x(4)$ & \dots &\\
$\TWO$: & & $x(1)$ & &$x(3)$ & &$x(5)$ & \dots
\end{tabular}

\skipsmall

After $\omega$ rounds, the
play of the game has produced an element $x\in\BaireSpace$ given by
$x=\angles{x(0),x(1),x(2),x(3),\dots}$. 

\skipsmall

If $x\in A$ then $\ONE$ wins this play
of the game. Otherwise $\TWO$ wins.
\end{definition}


\end{slide}

\begin{slide}{Strategies}

\begin{definition}
A \textbf{strategy} is a function $\sigma:\FiniteSeq\map\omega$. 
\end{definition}

\skipsmall

The players may use a given strategy during a play of the game. Suppose $\sigma$
is a strategy. Player $\ONE$ uses $\sigma$ while playing $G(A)$ as follows.

\skipsmall

\begin{tabular}{lccccr}
$\ONE$: & $\sigma(\emptyset)$ & & $\sigma\bigl(\angles{x(0),x(1)}\bigr)$ &   \dots &\\
$\TWO$: & & $x(1)$ & &$x(3)$  & \dots
\end{tabular}

\skipsmall

Similarly player $\TWO$ uses $\sigma$ while
playing $G(A)$ as follows.

\skipsmall

\begin{tabular}{lccccr}
$\ONE$: & $x(0)$ & & $x(2)$ &   \dots &\\
$\TWO$: & & $\sigma\bigl(\angles{x(0)}\bigr)$ & &$\sigma\bigl(\angles{x(0),x(1),x(2)}\bigr)$  & \dots
\end{tabular}

\end{slide}

\begin{slide}{Winning Strategies}

\begin{definition}
If $\sigma$ is a strategy and $y\in\BaireSpace$ then $\sigma*y$ is the element $z$
of $\BaireSpace$ that results when $\ONE$ plays $\sigma$ and $\TWO$ plays $y$:

\skipsmall

$z(0)=\sigma(\emptyset)$, $z(1)=y(0)$,  $z(2)=\sigma(\angles{z(0),z(1)})$.
$z(3)=y(1)$. etc. 

\skipsmall

Similarly, $y*\sigma$ is the element
of $\BaireSpace$ that results when $\ONE$ plays $y$ and $\TWO$ plays
$\sigma$.

\end{definition}

\skipsmall

\begin{definition}
Given a set $A$, a strategy $\sigma$ is \textbf{winning} for $\ONE$ in $G(A)$
if, by using $\sigma$ $\ONE$ will win $G(A)$ regardless of how $\TWO$ plays. 

\skipsmall
More preciesely, $\sigma$ is winning for $\ONE$ if for
all $y\in\BaireSpace$, $\sigma*y\in A$.

\skipsmall

Similary, $\sigma$ is winning for $\TWO$ in $G(A)$ if for all $y\in\BaireSpace$,
$y*\sigma\notin A$.
\end{definition}

\end{slide}

\begin{slide}{Winning Strategies (2)}

Notice that it is not possible for both players to have a winning strategy in
the same game $G(A)$. For if they did, we could play them against each other.
Suppose $\sigma$ was winning for $\ONE$ and $\tau$ was winning for $\TWO$:

\skipsmall

\begin{tabular}{lccc}
$\ONE$: & $x(0)=\sigma(\emptyset)$ & & $\sigma\bigl(\angles{x(0),x(1)}\bigr)\dots$\\
$\TWO$: & & $x(1)=\tau\bigl(\angles{x(0)}\bigr)$ &\dots
\end{tabular}

\skipsmall

Since $\sigma$ is winning for $\ONE$, $x\in A$. Since $\tau$ is winning for
$\TWO$, $x\notin A$. Contradiction!

\end{slide}

\begin{slide}{Determinacy}
\begin{definition}
Let $A\subseteq\BaireSpace$. We say that $G(A)$ is \textbf{determined}, or
sometimes just that $A$ is determined, and we
write $\Det(A)$, if either player $\ONE$ or player $\TWO$ has a winning strategy
in $G(A)$.
\end{definition}

\skipsmall

\begin{example}
Let $A=\BaireSpace$. Then every strategy is winning for $\ONE$ in $G(A)$. So $\Det(A)$.


\skipsmall

Let $A=\emptyset$. Then every strategy is winning for $\TWO$ in $G(A)$. So
$\Det(A)$.


\skipsmall

Let $A=\setof{x\in\BaireSpace}{x\text{ has infinitely many 1's}}$. Then $\ONE$
has a winning strategy in $G(A)$, so $\Det(A)$.


\skipsmall

Let $A$ be any countable subset of $\BaireSpace$. Then $\TWO$ has a winning
strategy in $G(A)$, so $\Det(A)$. (proof: In round $2n+1$ player $\TWO$ plays any
integer that is not in position $2n+1$ in the $n^{\text{th}}$ element of $A$.)
\end{example}

\end{slide}

\begin{slide}{An Undetermined Game}

\begin{theorem}
There is a set $A\subset\BaireSpace$ such that $G(A)$ is not determined.
\end{theorem}

\begin{proof}[proof sketch]
Using the axiom of choice we can enumerate all strategies: $\sequence{\sigma_{\alpha}}{\alpha<2^{\aleph_0}}$.
There are $2^{\aleph_0}$ strategies, and also $|\BaireSpace|=2^{\aleph_0}$.
By induction on ordinals
$\alpha<2^{\aleph_0}$, we build a payoff set $A$ that avoids every strategy. At
stage $\alpha$ we ensure that $\sigma_{\alpha}$ is not winning for $\ONE$ by
throwing some appropriate $x\in\BaireSpace$ into the complement of $A$, and we
ensure $\sigma_{\alpha}$ is not winning for $\TWO$ by
throwing some appropriate $x\in\BaireSpace$ into $A$.
\end{proof}

\skipsmall

Notice that the proof was not constructive---we did not give an explicit
definition for the non-determined set $A$.  There is no known proof that does.

\end{slide}

\begin{slide}{A Metric on Baire Space}
Baire Space is a metric space under the metric:

\skipsmall

$d(x,y) = \frac{1}{n+1}$ where $n$ is greatest such that $x\restr n = y\restr n$.

\skipsmall

This gives us a notion of a convergent sequence in Baire Space:

\skipsmall
Suppose that $\sequence{x_n}{n\in\omega}$ is a sequence of elments of
$\BaireSpace$ and $x\in\BaireSpace$. We say that $x_n\implies x$ if for all $k\in\omega$
there is an $N\in\omega$ such that for all $n>N$, $x_n\restr k = x\restr k$.


\end{slide}

\begin{slide}{Topology on Baire Space}
\begin{definition}
The \textbf{basic open sets} in Baire Space are the sets of the form $[T_s]$ for
$s\in\FiniteSeq$. 

\skipsmall

If $A\subseteq\BaireSpace$, $A$ is \textbf{open} if $A$ is the union of a
collection of basic open sets. 

\skipsmall

Equivalently, $A$ is open if, whenever $x\in A$
there is an $s\in\FiniteSeq$ such that $s\subset x$  and
$x\in[T_s]\subseteq A$.

\skipsmall
 $A$ is
\textbf{closed} if $\neg A$ is open, where $\neg A = \BaireSpace - A$.
\end{definition}

\skipsmall

\begin{enumerate}
  \item $\BaireSpace$ and $\emptyset$ are both open and closed.
  \item The union of a collection of open sets is open.
  \item The intersection of a collection of closed sets is closed.
\end{enumerate}

\end{slide}


\begin{slide}{Closed Sets in Any Metric Space}

\begin{lemma}
The following are equivalent:
\begin{enumerate}
  \item $A$ is closed
  \item Whenever $x_n\implies x$ and each $x_n\in A$ then $x\in A$.
\end{enumerate}
\end{lemma}
\begin{proof}
First assume (1).
Let $x_n\implies x$ with each $x_n\in A$. Suppose towards a
contradiction that $x\in\neg A$. Then there is an $s\subseteq x$ such that
$[T_s]\subseteq \neg A$. Now $s=x\restr k$ for some $k$. There is an $N$ such
that for
all $n>N$ $x_n\restr k = x\restr k = s$. But then $x_n\in [T_s]$  so $x_n\in
\neg A$. Contradiction.

\skipsmall

Conversely suppose (2) holds and we will show that $A$ is closed, by showing
that $\neg A$ is open. Let $x\in\neg A$. We want to see that there is some $n$
such that, letting $s=x\restr n$, we have that $[T_s]\subseteq \neg A$. Suppose
not. Then for each $n$, there is some $x_n\in[T_s]$ where $s=x\restr n$ such
that $x_n\in A$. So $x_n\implies x$. By (2), $x\in A$. Contradiction.
\end{proof}
\end{slide}

\begin{slide}{Closed Sets in Baire Space}
\begin{lemma}
$A$ is closed $\Ifff$ $\exists$ a tree $T$ such that $A=[T]$.
\end{lemma}
\begin{proof}
First suppose that $A=[T]$. We will show that $\neg A$ is
open. Let $x\in \neg A=\neg[T]$. There is some $n$ such that
$x\restr
n\notin T$. Let $s=x\restr n$. Then $x\in[T_s]\subseteq\neg A$. So $\neg A$ is
open, so $A$ is closed. 

\skipsmall

Now suppose $A$ is closed. Let $T=\setof{x\restr n}{x\in A,
n\in\omega}$. $T$ is a tree. We will show that $A=[T]$. 

\skipsmall
$A\subseteq[T]$ is
obvious and doesn't require $A$ to be closed. 

\skipsmall

So we must see that $[T]\subseteq
A$. Let $x\in[T]$. This means that $x\restr n \in T$ for each $n$. This means
that for each $n$ there is some $x_n\in A$ such that $x\restr n = x_n \restr n$.
This means that $x_n\implies x$. So $x\in A$.
\end{proof}
\end{slide}

\begin{slide}{Open Games}
\begin{theorem}[Gale,Stewart 1953]
Let $A\subseteq\BaireSpace$ be open or closed. Then $G(A)$ is determined.
\end{theorem}


\skipsmall

We will prove this theorem over the next few slides.

\skipmed

\begin{definition}
If $s\in\FiniteSeq$ and $x\in\BaireSpace$ or $x\in\FiniteSeq$, then $s\frown x$
is $s$ concatenated with $x$. 
\newline
$s\frown x = \angles{s(0),\dots,s(n),x(0),x(1),\dots}$ 
\newline
where $s=\angles{s(0),\dots,s(n)}$.
\end{definition}

\skipsmall



\end{slide}

\begin{slide}{Winning Strategies Relative to $s$}

\begin{definition}
Let $s\in\FiniteSeq$. We say that $\sigma$ is a winning strategy for a player
($\ONE$ or $\TWO$) in $G(A)$ \textbf{relative to $s$}
if, whenever the player uses $\sigma$ in a play of $G(A)$ that starts with $s$,
then he will win. More precisely:

\skipsmall

First suppose the length of $s$ is even. 

\skipsmall

 $\sigma$ is winning for $\ONE$ in
$G(A)$ relative to $s$ if for all $y\in\BaireSpace$, $s\frown\sigma*y\in A$.

\skipsmall

 $\sigma$ is winning for $\TWO$ in
$G(A)$ relative to $s$ if for  $y\in\BaireSpace$,
$s\frown y*\sigma\in \neg A$.

\skipsmall

Now suppose the length of $s$ is odd.

\skipsmall

 $\sigma$ is winning for $\ONE$ in
$G(A)$ relative to $s$ if for all $a\in\omega$ and all $y\in\BaireSpace$,
$s\frown\angles{a}\frown\sigma*y\in A$.

\skipsmall

 $\sigma$ is winning for $\TWO$ in
$G(A)$ relative to $s$ if there exists $a\in\omega$ such that for all
$y\in\BaireSpace$, $s\frown\angles{a}\frown y*\sigma\in \neg A$.

\end{definition}

\end{slide}

\begin{slide}{Winning Positions}

\begin{definition}
We say that $s$ is a \textbf{winning position} for player $\ONE$ in $G(A)$ if
there exists a strategy $\sigma$ such that $\sigma$ is winning in $G(A)$
relative to $s$. Similarly for player $\TWO$.
\end{definition}

\skipsmall

\begin{lemma}
Suppose $s$ has even length and $s$ is not a winning position for
player $\TWO$ in $G(A)$. Then there exists an $a\in\omega$ such that for all
$b\in\omega$,
$s\frown\angles{a}\frown\angles{b}$ is also not a winning position for $\TWO$.
\end{lemma}
\begin{proof}
If for all $a\in\omega$ there exists a $b\in\omega$ such that
$s\frown\angles{a}\frown\angles{b}$ is a winning position for $\TWO$, we could
put all the strategies for the different $a$'s together to get a winning
strategy for $\TWO$ relative to $s$.
\end{proof}

\skipsmall

\begin{lemma}
Suppose $s$ has odd length and $s$ is not a winning position for
player $\ONE$ in $G(A)$. Then there exists an $a\in\omega$ such that for all
$b\in\omega$,
$s\frown\angles{a}\frown\angles{b}$ is also not a winning position for $\ONE$.
\end{lemma}
\end{slide}

\begin{slide}{Keep Not Losing}
\begin{definition}
Suppose that player $\TWO$ does not have a winning strategy in $G(A)$. A
\textbf{keep-not-losing} strategy for player $\ONE$ is the following. Given any
position in the game $s$ of even length that is not winning for $\TWO$, play
some integer $a$ such that for all integers $b$,
$s\frown\angles{a}\frown\angles{b}$ is still not a winning position for $\TWO$.

\skipsmall

In a game in which $\ONE$ does not have a winning strategy, a keep-not-losing
strategy for player $\TWO$ is defined similarly.
\end{definition}

\end{slide}

\begin{slide}{Proof of Open and Closed Determinacy}

\begin{proof}
Suppose $A$ is closed and $\TWO$ does not have a winning strategy. Let $\sigma$
by a keep-not-losing strategy for $\ONE$. Then $\sigma$ is actually a winning
strategy for $\ONE$. For let $y$ be $\TWO$'s play and let $x=\sigma*y$ be the
outcome of the game. We want to see that $x\in A$. Suppose towards a
contradiction tht $x\in\neg A$. Then
there is an $s\in\FiniteSeq$ such that $x\in[T_s]\subseteq\neg A$. Since every
$z\in\BaireSpace$ such that $s\subset z$ is in $\neg A$,  $s$ is a winning
position for $\TWO$ in $G(A)$. But this contradicts the fact that $\ONE$ was
using a keep-not-losing strategy. Thus $x\in A$. So $\sigma$ is a winning
strategy for $\ONE$. So $G(A)$ is determined. This proves the determinacy of all
closed games.

\skipsmall

The determinacy of all open games is proved similarly with the roles of $\ONE$
and $\TWO$ reversed.
\end{proof}
\end{slide}

\begin{slide}{More Complicated Sets}
\begin{definition}
A set $A\subseteq\BaireSpace$ is called $\mathbf{\Sigma^0_2}$ if
$A=\Union{k\in\omega}F_k$ where each $F_k$ is closed.

\skipsmall

A set $A\subseteq\BaireSpace$ is called $\mathbf{\Pi^0_2}$ if
$A=\Intersection{k\in\omega}G_k$ where each $G_k$ is open.
\end{definition}

\skipsmall

\begin{definition}
Let us use $\mathbf{\Sigma^0_1}$ as a synonym for open and $\mathbf{\Pi^0_1}$ as
a synonym for closed.

\skipsmall

 A set $A\subseteq\BaireSpace$ is called $\mathbf{\Sigma^0_{n+1}}$
if $A=\Union{k\in\omega}B_k$ where each $B_k$ is $\mathbf{\Pi^0_{n}}$

\skipsmall

A set $A\subseteq\BaireSpace$ is called $\mathbf{\Pi^0_{n+1}}$ if
$A=\Intersection{k\in\omega}B_k$ where each $B_k$ is $\mathbf{\Sigma^0_{n}}$.
\end{definition}

\end{slide}

\begin{slide}{Logical Complexity}
A different point of view is to consider the \textbf{formula} that defines the
set $A$.

\skipsmall

$A$ is  $\mathbf{\Sigma^0_4}$ just in case 
$A=\setof{x\in\BaireSpace}{\exists a\in\omega\forall b\in\omega\exists c\in\omega}R(a,b,c,x)$,
where $R(a,b,c,x)$ means that $x\in F_{(a,b,c)}$, where each $F_{(a,b,c)}$ is closed.

\skipsmall

So a $\mathbf{\Sigma^0_{n+1}}$ set is more complex than a
$\mathbf{\Sigma^0_{n}}$ set
in the sense that the formula used to define it it more complex---it has more quantifiers.

\end{slide}

\begin{slide}{More Determinacy}

\begin{theorem}[Wolfe 1955]
All $\mathbf{\Sigma^0_2}$ and $\mathbf{\Pi^0_2}$ games are determined.
\end{theorem}

\skipmed

\begin{theorem}[Davis 1964]
All $\mathbf{\Sigma^0_3}$ and $\mathbf{\Pi^0_3}$ games are determined.
\end{theorem}

\end{slide}

\begin{slide}{Borel Determinacy}
\begin{definition}
The \textbf{Borel} $\Sigma$-algebra is the smallest subset of
$\Powerset(\BaireSpace)$ that contains the open sets and is closed under
compliment and countable intersection.

\skipsmall

A set $A\subseteq \BaireSpace$ is called Borel if it is in the Borel $\Sigma$-algebra.
\end{definition}

\skipmed

\begin{theorem}[Martin 1975]
All Borel games are determined.
\end{theorem}

\end{slide}



\end{document}
