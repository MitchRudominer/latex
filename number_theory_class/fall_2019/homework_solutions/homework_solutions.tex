\documentclass[oneside,12pt]{amsart}

\usepackage{amsmath,amssymb,latexsym,eucal,amsthm}

%%%%%%%%%%%%%%%%%%%%%%%%%%%%%%%%%%%%%%%%%%%%%
% Common Set Theory Constructs
%%%%%%%%%%%%%%%%%%%%%%%%%%%%%%%%%%%%%%%%%%%%%

\newcommand{\setof}[2]{\left\{ \, #1 \, \left| \, #2 \, \right.\right\}}
\newcommand{\lsetof}[2]{\left\{\left. \, #1 \, \right| \, #2 \,  \right\}}
\newcommand{\bigsetof}[2]{\bigl\{ \, #1 \, \bigm | \, #2 \,\bigr\}}
\newcommand{\Bigsetof}[2]{\Bigl\{ \, #1 \, \Bigm | \, #2 \,\Bigr\}}
\newcommand{\biggsetof}[2]{\biggl\{ \, #1 \, \biggm | \, #2 \,\biggr\}}
\newcommand{\Biggsetof}[2]{\Biggl\{ \, #1 \, \Biggm | \, #2 \,\Biggr\}}
\newcommand{\dotsetof}[2]{\left\{ \, #1 \, : \, #2 \, \right\}}
\newcommand{\bigdotsetof}[2]{\bigl\{ \, #1 \, : \, #2 \,\bigr\}}
\newcommand{\Bigdotsetof}[2]{\Bigl\{ \, #1 \, \Bigm : \, #2 \,\Bigr\}}
\newcommand{\biggdotsetof}[2]{\biggl\{ \, #1 \, \biggm : \, #2 \,\biggr\}}
\newcommand{\Biggdotsetof}[2]{\Biggl\{ \, #1 \, \Biggm : \, #2 \,\Biggr\}}
\newcommand{\sequence}[2]{\left\langle \, #1 \,\left| \, #2 \, \right. \right\rangle}
\newcommand{\lsequence}[2]{\left\langle\left. \, #1 \, \right| \,#2 \,  \right\rangle}
\newcommand{\bigsequence}[2]{\bigl\langle \,#1 \, \bigm | \, #2 \, \bigr\rangle}
\newcommand{\Bigsequence}[2]{\Bigl\langle \,#1 \, \Bigm | \, #2 \, \Bigr\rangle}
\newcommand{\biggsequence}[2]{\biggl\langle \,#1 \, \biggm | \, #2 \, \biggr\rangle}
\newcommand{\Biggsequence}[2]{\Biggl\langle \,#1 \, \Biggm | \, #2 \, \Biggr\rangle}
\newcommand{\singleton}[1]{\left\{#1\right\}}
\newcommand{\angles}[1]{\left\langle #1 \right\rangle}
\newcommand{\bigangles}[1]{\bigl\langle #1 \bigr\rangle}
\newcommand{\Bigangles}[1]{\Bigl\langle #1 \Bigr\rangle}
\newcommand{\biggangles}[1]{\biggl\langle #1 \biggr\rangle}
\newcommand{\Biggangles}[1]{\Biggl\langle #1 \Biggr\rangle}


\newcommand{\force}[1]{\Vert\!\underset{\!\!\!\!\!#1}{\!\!\!\relbar\!\!\!%
\relbar\!\!\relbar\!\!\relbar\!\!\!\relbar\!\!\relbar\!\!\relbar\!\!\!%
\relbar\!\!\relbar\!\!\relbar}}
\newcommand{\longforce}[1]{\Vert\!\underset{\!\!\!\!\!#1}{\!\!\!\relbar\!\!\!%
\relbar\!\!\relbar\!\!\relbar\!\!\!\relbar\!\!\relbar\!\!\relbar\!\!\!%
\relbar\!\!\relbar\!\!\relbar\!\!\relbar\!\!\relbar\!\!\relbar\!\!\relbar\!\!\relbar}}
\newcommand{\nforce}[1]{\Vert\!\underset{\!\!\!\!\!#1}{\!\!\!\relbar\!\!\!%
\relbar\!\!\relbar\!\!\relbar\!\!\!\relbar\!\!\relbar\!\!\relbar\!\!\!%
\relbar\!\!\not\relbar\!\!\relbar}}
\newcommand{\forcein}[2]{\overset{#2}{\Vert\underset{\!\!\!\!\!#1}%
{\!\!\!\relbar\!\!\!\relbar\!\!\relbar\!\!\relbar\!\!\!\relbar\!\!\relbar\!%
\!\relbar\!\!\!\relbar\!\!\relbar\!\!\relbar\!\!\relbar\!\!\!\relbar\!\!%
\relbar\!\!\relbar}}}

\newcommand{\pre}[2]{{}^{#2}{#1}}

\newcommand{\restr}{\!\!\upharpoonright\!}

%%%%%%%%%%%%%%%%%%%%%%%%%%%%%%%%%%%%%%%%%%%%%
% Set-Theoretic Connectives
%%%%%%%%%%%%%%%%%%%%%%%%%%%%%%%%%%%%%%%%%%%%%

\newcommand{\intersect}{\cap}
\newcommand{\union}{\cup}
\newcommand{\Intersection}[1]{\bigcap\limits_{#1}}
\newcommand{\Union}[1]{\bigcup\limits_{#1}}
\newcommand{\adjoin}{{}^\frown}
\newcommand{\forces}{\Vdash}

%%%%%%%%%%%%%%%%%%%%%%%%%%%%%%%%%%%%%%%%%%%%%
% Miscellaneous
%%%%%%%%%%%%%%%%%%%%%%%%%%%%%%%%%%%%%%%%%%%%%
\newcommand{\defeq}{=_{\text{def}}}
\newcommand{\Turingleq}{\leq_{\text{T}}}
\newcommand{\Turingless}{<_{\text{T}}}
\newcommand{\lexleq}{\leq_{\text{lex}}}
\newcommand{\lexless}{<_{\text{lex}}}
\newcommand{\Turingequiv}{\equiv_{\text{T}}}
\newcommand{\isomorphic}{\cong}

%%%%%%%%%%%%%%%%%%%%%%%%%%%%%%%%%%%%%%%%%%%%%
% Constants
%%%%%%%%%%%%%%%%%%%%%%%%%%%%%%%%%%%%%%%%%%%%%
\newcommand{\R}{\mathbb{R}}
\renewcommand{\P}{\mathbb{P}}
\newcommand{\Q}{\mathbb{Q}}
\newcommand{\Z}{\mathbb{Z}}
\newcommand{\Zpos}{\Z^{+}}
\newcommand{\Znonneg}{\Z^{\geq 0}}
\newcommand{\C}{\mathbb{C}}
\newcommand{\N}{\mathbb{N}}
\newcommand{\B}{\mathbb{B}}
\newcommand{\Bairespace}{\pre{\omega}{\omega}}
\newcommand{\LofR}{L(\R)}
\newcommand{\JofR}[1]{J_{#1}(\R)}
\newcommand{\SofR}[1]{S_{#1}(\R)}
\newcommand{\JalphaR}{\JofR{\alpha}}
\newcommand{\JbetaR}{\JofR{\beta}}
\newcommand{\JlambdaR}{\JofR{\lambda}}
\newcommand{\SalphaR}{\SofR{\alpha}}
\newcommand{\SbetaR}{\SofR{\beta}}
\newcommand{\Pkl}{\mathcal{P}_{\kappa}(\lambda)}
\DeclareMathOperator{\con}{con}
\DeclareMathOperator{\ORD}{OR}
\DeclareMathOperator{\Ord}{OR}
\DeclareMathOperator{\WO}{WO}
\DeclareMathOperator{\OD}{OD}
\DeclareMathOperator{\HOD}{HOD}
\DeclareMathOperator{\HC}{HC}
\DeclareMathOperator{\WF}{WF}
\DeclareMathOperator{\wfp}{wfp}
\DeclareMathOperator{\HF}{HF}
\newcommand{\One}{I}
\newcommand{\ONE}{I}
\newcommand{\Two}{II}
\newcommand{\TWO}{II}
\newcommand{\Mladder}{M^{\text{ld}}}

%%%%%%%%%%%%%%%%%%%%%%%%%%%%%%%%%%%%%%%%%%%%%
% Commutative Algebra Constants
%%%%%%%%%%%%%%%%%%%%%%%%%%%%%%%%%%%%%%%%%%%%%
\DeclareMathOperator{\dottimes}{\dot{\times}}
\DeclareMathOperator{\dotminus}{\dot{-}}
\DeclareMathOperator{\Spec}{Spec}

%%%%%%%%%%%%%%%%%%%%%%%%%%%%%%%%%%%%%%%%%%%%%
% Theories
%%%%%%%%%%%%%%%%%%%%%%%%%%%%%%%%%%%%%%%%%%%%%
\DeclareMathOperator{\ZFC}{ZFC}
\DeclareMathOperator{\ZF}{ZF}
\DeclareMathOperator{\AD}{AD}
\DeclareMathOperator{\ADR}{AD_{\R}}
\DeclareMathOperator{\KP}{KP}
\DeclareMathOperator{\PD}{PD}
\DeclareMathOperator{\CH}{CH}
\DeclareMathOperator{\GCH}{GCH}
\DeclareMathOperator{\HPC}{HPC} % HOD pair capturing
%%%%%%%%%%%%%%%%%%%%%%%%%%%%%%%%%%%%%%%%%%%%%
% Iteration Trees
%%%%%%%%%%%%%%%%%%%%%%%%%%%%%%%%%%%%%%%%%%%%%

\newcommand{\pred}{\text{-pred}}

%%%%%%%%%%%%%%%%%%%%%%%%%%%%%%%%%%%%%%%%%%%%%%%%
% Operator Names
%%%%%%%%%%%%%%%%%%%%%%%%%%%%%%%%%%%%%%%%%%%%%%%%
\DeclareMathOperator{\Det}{Det}
\DeclareMathOperator{\dom}{dom}
\DeclareMathOperator{\ran}{ran}
\DeclareMathOperator{\range}{ran}
\DeclareMathOperator{\image}{image}
\DeclareMathOperator{\crit}{crit}
\DeclareMathOperator{\card}{card}
\DeclareMathOperator{\cf}{cf}
\DeclareMathOperator{\cof}{cof}
\DeclareMathOperator{\rank}{rank}
\DeclareMathOperator{\ot}{o.t.}
\DeclareMathOperator{\ords}{o}
\DeclareMathOperator{\ro}{r.o.}
\DeclareMathOperator{\rud}{rud}
\DeclareMathOperator{\Powerset}{\mathcal{P}}
\DeclareMathOperator{\length}{lh}
\DeclareMathOperator{\lh}{lh}
\DeclareMathOperator{\limit}{lim}
\DeclareMathOperator{\fld}{fld}
\DeclareMathOperator{\projection}{p}
\DeclareMathOperator{\Ult}{Ult}
\DeclareMathOperator{\ULT}{Ult}
\DeclareMathOperator{\Coll}{Coll}
\DeclareMathOperator{\Col}{Col}
\DeclareMathOperator{\Hull}{Hull}
\DeclareMathOperator*{\dirlim}{dir lim}
\DeclareMathOperator{\Scale}{Scale}
\DeclareMathOperator{\supp}{supp}
\DeclareMathOperator{\trancl}{tran.cl.}
\DeclareMathOperator{\trace}{Tr}
\DeclareMathOperator{\diag}{diag}
\DeclareMathOperator{\spn}{span}
\DeclareMathOperator{\sgn}{sgn}
%%%%%%%%%%%%%%%%%%%%%%%%%%%%%%%%%%%%%%%%%%%%%
% Logical Connectives
%%%%%%%%%%%%%%%%%%%%%%%%%%%%%%%%%%%%%%%%%%%%%
\newcommand{\IImplies}{\Longrightarrow}
\newcommand{\SkipImplies}{\quad\Longrightarrow\quad}
\newcommand{\Ifff}{\Longleftrightarrow}
\newcommand{\iimplies}{\longrightarrow}
\newcommand{\ifff}{\longleftrightarrow}
\newcommand{\Implies}{\Rightarrow}
\newcommand{\Iff}{\Leftrightarrow}
\renewcommand{\implies}{\rightarrow}
\renewcommand{\iff}{\leftrightarrow}
\newcommand{\AND}{\wedge}
\newcommand{\OR}{\vee}
\newcommand{\st}{\text{ s.t. }}
\newcommand{\Or}{\text{ or }}

%%%%%%%%%%%%%%%%%%%%%%%%%%%%%%%%%%%%%%%%%%%%%
% Function Arrows
%%%%%%%%%%%%%%%%%%%%%%%%%%%%%%%%%%%%%%%%%%%%%

\newcommand{\injection}{\xrightarrow{\text{1-1}}}
\newcommand{\surjection}{\xrightarrow{\text{onto}}}
\newcommand{\bijection}{\xrightarrow[\text{onto}]{\text{1-1}}}
\newcommand{\cofmap}{\xrightarrow{\text{cof}}}
\newcommand{\map}{\rightarrow}

%%%%%%%%%%%%%%%%%%%%%%%%%%%%%%%%%%%%%%%%%%%%%
% Mouse Comparison Operators
%%%%%%%%%%%%%%%%%%%%%%%%%%%%%%%%%%%%%%%%%%%%%
\newcommand{\initseg}{\trianglelefteq}
\newcommand{\properseg}{\lhd}
\newcommand{\notinitseg}{\ntrianglelefteq}
\newcommand{\notproperseg}{\ntriangleleft}

%%%%%%%%%%%%%%%%%%%%%%%%%%%%%%%%%%%%%%%%%%%%%
%           calligraphic letters
%%%%%%%%%%%%%%%%%%%%%%%%%%%%%%%%%%%%%%%%%%%%%
\newcommand{\cA}{\mathcal{A}}
\newcommand{\cB}{\mathcal{B}}
\newcommand{\cC}{\mathcal{C}}
\newcommand{\cD}{\mathcal{D}}
\newcommand{\cE}{\mathcal{E}}
\newcommand{\cF}{\mathcal{F}}
\newcommand{\cG}{\mathcal{G}}
\newcommand{\cH}{\mathcal{H}}
\newcommand{\cI}{\mathcal{I}}
\newcommand{\cJ}{\mathcal{J}}
\newcommand{\cK}{\mathcal{K}}
\newcommand{\cL}{\mathcal{L}}
\newcommand{\cM}{\mathcal{M}}
\newcommand{\cN}{\mathcal{N}}
\newcommand{\cO}{\mathcal{O}}
\newcommand{\cP}{\mathcal{P}}
\newcommand{\cQ}{\mathcal{Q}}
\newcommand{\cR}{\mathcal{R}}
\newcommand{\cS}{\mathcal{S}}
\newcommand{\cT}{\mathcal{T}}
\newcommand{\cU}{\mathcal{U}}
\newcommand{\cV}{\mathcal{V}}
\newcommand{\cW}{\mathcal{W}}
\newcommand{\cX}{\mathcal{X}}
\newcommand{\cY}{\mathcal{Y}}
\newcommand{\cZ}{\mathcal{Z}}


%%%%%%%%%%%%%%%%%%%%%%%%%%%%%%%%%%%%%%%%%%%%%
%          Primed Letters
%%%%%%%%%%%%%%%%%%%%%%%%%%%%%%%%%%%%%%%%%%%%%
\newcommand{\aprime}{a^{\prime}}
\newcommand{\bprime}{b^{\prime}}
\newcommand{\cprime}{c^{\prime}}
\newcommand{\dprime}{d^{\prime}}
\newcommand{\eprime}{e^{\prime}}
\newcommand{\fprime}{f^{\prime}}
\newcommand{\gprime}{g^{\prime}}
\newcommand{\hprime}{h^{\prime}}
\newcommand{\iprime}{i^{\prime}}
\newcommand{\jprime}{j^{\prime}}
\newcommand{\kprime}{k^{\prime}}
\newcommand{\lprime}{l^{\prime}}
\newcommand{\mprime}{m^{\prime}}
\newcommand{\nprime}{n^{\prime}}
\newcommand{\oprime}{o^{\prime}}
\newcommand{\pprime}{p^{\prime}}
\newcommand{\qprime}{q^{\prime}}
\newcommand{\rprime}{r^{\prime}}
\newcommand{\sprime}{s^{\prime}}
\newcommand{\tprime}{t^{\prime}}
\newcommand{\uprime}{u^{\prime}}
\newcommand{\vprime}{v^{\prime}}
\newcommand{\wprime}{w^{\prime}}
\newcommand{\xprime}{x^{\prime}}
\newcommand{\yprime}{y^{\prime}}
\newcommand{\zprime}{z^{\prime}}
\newcommand{\Aprime}{A^{\prime}}
\newcommand{\Bprime}{B^{\prime}}
\newcommand{\Cprime}{C^{\prime}}
\newcommand{\Dprime}{D^{\prime}}
\newcommand{\Eprime}{E^{\prime}}
\newcommand{\Fprime}{F^{\prime}}
\newcommand{\Gprime}{G^{\prime}}
\newcommand{\Hprime}{H^{\prime}}
\newcommand{\Iprime}{I^{\prime}}
\newcommand{\Jprime}{J^{\prime}}
\newcommand{\Kprime}{K^{\prime}}
\newcommand{\Lprime}{L^{\prime}}
\newcommand{\Mprime}{M^{\prime}}
\newcommand{\Nprime}{N^{\prime}}
\newcommand{\Oprime}{O^{\prime}}
\newcommand{\Pprime}{P^{\prime}}
\newcommand{\Qprime}{Q^{\prime}}
\newcommand{\Rprime}{R^{\prime}}
\newcommand{\Sprime}{S^{\prime}}
\newcommand{\Tprime}{T^{\prime}}
\newcommand{\Uprime}{U^{\prime}}
\newcommand{\Vprime}{V^{\prime}}
\newcommand{\Wprime}{W^{\prime}}
\newcommand{\Xprime}{X^{\prime}}
\newcommand{\Yprime}{Y^{\prime}}
\newcommand{\Zprime}{Z^{\prime}}
\newcommand{\alphaprime}{\alpha^{\prime}}
\newcommand{\betaprime}{\beta^{\prime}}
\newcommand{\gammaprime}{\gamma^{\prime}}
\newcommand{\Gammaprime}{\Gamma^{\prime}}
\newcommand{\deltaprime}{\delta^{\prime}}
\newcommand{\epsilonprime}{\epsilon^{\prime}}
\newcommand{\kappaprime}{\kappa^{\prime}}
\newcommand{\lambdaprime}{\lambda^{\prime}}
\newcommand{\rhoprime}{\rho^{\prime}}
\newcommand{\Sigmaprime}{\Sigma^{\prime}}
\newcommand{\tauprime}{\tau^{\prime}}
\newcommand{\xiprime}{\xi^{\prime}}
\newcommand{\thetaprime}{\theta^{\prime}}
\newcommand{\Omegaprime}{\Omega^{\prime}}
\newcommand{\cMprime}{\cM^{\prime}}
\newcommand{\cNprime}{\cN^{\prime}}
\newcommand{\cPprime}{\cP^{\prime}}
\newcommand{\cQprime}{\cQ^{\prime}}
\newcommand{\cRprime}{\cR^{\prime}}
\newcommand{\cSprime}{\cS^{\prime}}
\newcommand{\cTprime}{\cT^{\prime}}

%%%%%%%%%%%%%%%%%%%%%%%%%%%%%%%%%%%%%%%%%%%%%
%          bar Letters
%%%%%%%%%%%%%%%%%%%%%%%%%%%%%%%%%%%%%%%%%%%%%
\newcommand{\abar}{\bar{a}}
\newcommand{\bbar}{\bar{b}}
\newcommand{\cbar}{\bar{c}}
\newcommand{\ibar}{\bar{i}}
\newcommand{\jbar}{\bar{j}}
\newcommand{\nbar}{\bar{n}}
\newcommand{\xbar}{\bar{x}}
\newcommand{\ybar}{\bar{y}}
\newcommand{\zbar}{\bar{z}}
\newcommand{\pibar}{\bar{\pi}}
\newcommand{\phibar}{\bar{\varphi}}
\newcommand{\psibar}{\bar{\psi}}
\newcommand{\thetabar}{\bar{\theta}}
\newcommand{\nubar}{\bar{\nu}}

%%%%%%%%%%%%%%%%%%%%%%%%%%%%%%%%%%%%%%%%%%%%%
%          star Letters
%%%%%%%%%%%%%%%%%%%%%%%%%%%%%%%%%%%%%%%%%%%%%
\newcommand{\phistar}{\phi^{*}}
\newcommand{\Mstar}{M^{*}}

%%%%%%%%%%%%%%%%%%%%%%%%%%%%%%%%%%%%%%%%%%%%%
%          dotletters Letters
%%%%%%%%%%%%%%%%%%%%%%%%%%%%%%%%%%%%%%%%%%%%%
\newcommand{\Gdot}{\dot{G}}

%%%%%%%%%%%%%%%%%%%%%%%%%%%%%%%%%%%%%%%%%%%%%
%         check Letters
%%%%%%%%%%%%%%%%%%%%%%%%%%%%%%%%%%%%%%%%%%%%%
\newcommand{\deltacheck}{\check{\delta}}
\newcommand{\gammacheck}{\check{\gamma}}


%%%%%%%%%%%%%%%%%%%%%%%%%%%%%%%%%%%%%%%%%%%%%
%          Formulas
%%%%%%%%%%%%%%%%%%%%%%%%%%%%%%%%%%%%%%%%%%%%%

\newcommand{\formulaphi}{\text{\large $\varphi$}}
\newcommand{\Formulaphi}{\text{\Large $\varphi$}}


%%%%%%%%%%%%%%%%%%%%%%%%%%%%%%%%%%%%%%%%%%%%%
%          Fraktur Letters
%%%%%%%%%%%%%%%%%%%%%%%%%%%%%%%%%%%%%%%%%%%%%

\newcommand{\fa}{\mathfrak{a}}
\newcommand{\fb}{\mathfrak{b}}
\newcommand{\fc}{\mathfrak{c}}
\newcommand{\fk}{\mathfrak{k}}
\newcommand{\fp}{\mathfrak{p}}
\newcommand{\fq}{\mathfrak{q}}
\newcommand{\fr}{\mathfrak{r}}
\newcommand{\fA}{\mathfrak{A}}
\newcommand{\fB}{\mathfrak{B}}
\newcommand{\fC}{\mathfrak{C}}
\newcommand{\fD}{\mathfrak{D}}

%%%%%%%%%%%%%%%%%%%%%%%%%%%%%%%%%%%%%%%%%%%%%
%          Bold Letters
%%%%%%%%%%%%%%%%%%%%%%%%%%%%%%%%%%%%%%%%%%%%%
\newcommand{\ba}{\mathbf{a}}
\newcommand{\bb}{\mathbf{b}}
\newcommand{\bc}{\mathbf{c}}
\newcommand{\bd}{\mathbf{d}}
\newcommand{\be}{\mathbf{e}}
\newcommand{\bu}{\mathbf{u}}
\newcommand{\bv}{\mathbf{v}}
\newcommand{\bw}{\mathbf{w}}
\newcommand{\bx}{\mathbf{x}}
\newcommand{\by}{\mathbf{y}}
\newcommand{\bz}{\mathbf{z}}
\newcommand{\bSigma}{\boldsymbol{\Sigma}}
\newcommand{\bPi}{\boldsymbol{\Pi}}
\newcommand{\bDelta}{\boldsymbol{\Delta}}
\newcommand{\bdelta}{\boldsymbol{\delta}}
\newcommand{\bgamma}{\boldsymbol{\gamma}}
\newcommand{\bGamma}{\boldsymbol{\Gamma}}

%%%%%%%%%%%%%%%%%%%%%%%%%%%%%%%%%%%%%%%%%%%%%
%         Bold numbers
%%%%%%%%%%%%%%%%%%%%%%%%%%%%%%%%%%%%%%%%%%%%%
\newcommand{\bzero}{\mathbf{0}}

%%%%%%%%%%%%%%%%%%%%%%%%%%%%%%%%%%%%%%%%%%%%%
% Projective-Like Pointclasses
%%%%%%%%%%%%%%%%%%%%%%%%%%%%%%%%%%%%%%%%%%%%%
\newcommand{\Sa}[2][\alpha]{\Sigma_{(#1,#2)}}
\newcommand{\Pa}[2][\alpha]{\Pi_{(#1,#2)}}
\newcommand{\Da}[2][\alpha]{\Delta_{(#1,#2)}}
\newcommand{\Aa}[2][\alpha]{A_{(#1,#2)}}
\newcommand{\Ca}[2][\alpha]{C_{(#1,#2)}}
\newcommand{\Qa}[2][\alpha]{Q_{(#1,#2)}}
\newcommand{\da}[2][\alpha]{\delta_{(#1,#2)}}
\newcommand{\leqa}[2][\alpha]{\leq_{(#1,#2)}}
\newcommand{\lessa}[2][\alpha]{<_{(#1,#2)}}
\newcommand{\equiva}[2][\alpha]{\equiv_{(#1,#2)}}


\newcommand{\Sl}[1]{\Sa[\lambda]{#1}}
\newcommand{\Pl}[1]{\Pa[\lambda]{#1}}
\newcommand{\Dl}[1]{\Da[\lambda]{#1}}
\newcommand{\Al}[1]{\Aa[\lambda]{#1}}
\newcommand{\Cl}[1]{\Ca[\lambda]{#1}}
\newcommand{\Ql}[1]{\Qa[\lambda]{#1}}

\newcommand{\San}{\Sa{n}}
\newcommand{\Pan}{\Pa{n}}
\newcommand{\Dan}{\Da{n}}
\newcommand{\Can}{\Ca{n}}
\newcommand{\Qan}{\Qa{n}}
\newcommand{\Aan}{\Aa{n}}
\newcommand{\dan}{\da{n}}
\newcommand{\leqan}{\leqa{n}}
\newcommand{\lessan}{\lessa{n}}
\newcommand{\equivan}{\equiva{n}}

\newcommand{\SigmaOneOmega}{\Sigma^1_{\omega}}
\newcommand{\SigmaOneOmegaPlusOne}{\Sigma^1_{\omega+1}}
\newcommand{\PiOneOmega}{\Pi^1_{\omega}}
\newcommand{\PiOneOmegaPlusOne}{\Pi^1_{\omega+1}}
\newcommand{\DeltaOneOmegaPlusOne}{\Delta^1_{\omega+1}}

%%%%%%%%%%%%%%%%%%%%%%%%%%%%%%%%%%%%%%%%%%%%%
% Linear Algebra
%%%%%%%%%%%%%%%%%%%%%%%%%%%%%%%%%%%%%%%%%%%%%
\newcommand{\transpose}[1]{{#1}^{\text{T}}}
\newcommand{\norm}[1]{\lVert{#1}\rVert}
\newcommand\aug{\fboxsep=-\fboxrule\!\!\!\fbox{\strut}\!\!\!}

%%%%%%%%%%%%%%%%%%%%%%%%%%%%%%%%%%%%%%%%%%%%%
% Number Theory
%%%%%%%%%%%%%%%%%%%%%%%%%%%%%%%%%%%%%%%%%%%%%
\newcommand{\av}[1]{\lvert#1\rvert}
\DeclareMathOperator{\divides}{\mid}
\DeclareMathOperator{\ndivides}{\nmid}
\DeclareMathOperator{\lcm}{lcm}
\DeclareMathOperator{\sign}{sign}
\newcommand{\floor}[1]{\left\lfloor{#1}\right\rfloor}
\DeclareMathOperator{\ConCl}{CC}
\DeclareMathOperator{\ord}{ord}


%%%%%%%%%%%%%%%%%%%%%%%%%%%%%%%%%%%%%%%%%%%%%%%%%%%%%%%%%%%%%%%%%%%%%%%%%%%
%%  Theorem-Like Declarations
%%%%%%%%%%%%%%%%%%%%%%%%%%%%%%%%%%%%%%%%%%%%%%%%%%%%%%%%%%%%%%%%%%%%%%%%%%

\newtheorem{theorem}{Theorem}[section]
\newtheorem{lemma}[theorem]{Lemma}
\newtheorem{corollary}[theorem]{Corollary}
\newtheorem{proposition}[theorem]{Proposition}


\theoremstyle{definition}

\newtheorem{definition}[theorem]{Definition}
\newtheorem{conjecture}[theorem]{Conjecture}
\newtheorem{remark}[theorem]{Remark}
\newtheorem{remarks}[theorem]{Remarks}
\newtheorem{notation}[theorem]{Notation}

\theoremstyle{remark}

\newtheorem*{note}{Note}
\newtheorem*{warning}{Warning}
\newtheorem*{question}{Question}
\newtheorem*{example}{Example}
\newtheorem*{fact}{Fact}


\newenvironment*{subproof}[1][Proof]
{\begin{proof}[#1]}{\renewcommand{\qedsymbol}{$\diamondsuit$} \end{proof}}

\newenvironment*{case}[1]
{\textbf{Case #1.  }\itshape }{}

\newenvironment*{claim}[1][Claim]
{\textbf{#1.  }\itshape }{}


\pagestyle{plain}

\begin{document}

\title{Homework Solutions \\ Math 310, Elementary Number Theory \\ Fall 2019}
\author{Mitch Rudominer}

\maketitle

\textbf{Solution to Exercise 2.3} Suppose $a+b = c$. We need to show that $a=c-b$.
By Axiom (4) $b$ has a unique additive inverse $-b$. Adding $-b$
to both sides of the equation, and
using axioms (1), (2) and (3) we get
$a= a+0 = a + (b + (-b)) = (a+b) + (-b) = c + (-b) = c-b$.

Now suppose that $a-b = c$. We want to show that $a=b+c$.
Adding $b$ to both sides of the equation and using
the axioms we get
$a = a + 0 = a + (-b + b) = (a - b) +b  = c + b = b+c$.

\bigskip


\textbf{Solution to Exercise 2.4} Fix integers $n,m$ and suppose that $n+m=n$.
We want to show that this implies that $m=0$.
By the previous exercise we  can subtract $n$ from both sides of the equation
to yield $m = n - n = 0$.


\bigskip


\textbf{Solution to Exercise 2.5} Fix integers $n,m$ and suppose that $n+m=0$.
We want to show that this implies that $n=-m$ and $m=-n$.
By Exercise 2.3 we can subtract $m$ from both sides of the equation to yield
 $n = 0+ (-m) = -m$ . Similarly we can subtract $n$ from both sides of the
 equation to yield $m=0+(-n) = -n$.

\bigskip


\textbf{Solution to Exercise 2.6} Since $0+0 = 0$, by the previous exercise
$0=-0$.


\bigskip


\textbf{Solution to Exercise 2.7} Since $(-n) + n = 0$, by exercise 2.5,
$-(-n) = n$.

\bigskip


\textbf{Solution to Exercise 2.9} Let $m=n\times 0$. To see that $m=0$, by
Excercise 2.4 it is enough to show that $n+m=n$.
$n+m = n + n\times 0 = n\times 1 + n\times 0
= n (1 + 0) = n \times 1 = n$.


\bigskip


\textbf{Solution to Exercise 2.13} Fix integers integers $a, b$.
We want to show that $(-a)b = a(-b) = -(ab)$. Using Lemma 2.12
we see that all three of these expressions are equal to $-1(ab)$.
$(-a)b=(-1\cdot a)b = -1(ab)$, $a(-b)=a(-1\cdot b) = -1(ab)$,
$-(ab) = -1(ab)$.


\bigskip


\textbf{Solution to Exercise 2.14} Let $m=n-1$. We need to show tht $m$ is
the predecessor of $n$. This means that $m<n$ and no integer lies between
$m$ and $n$. But this is true because $n=m+1$ so $n$ is the successor of $m$.


\bigskip


\textbf{Solution to Exercise 2.16} We are asked to show that $1$ is the least
positive integer. This is true because $1=0+1$ so $1$ is the successor of 0,
meaning $0 < 1$ and there is no integer between 0 and 1. But this means
that 1 is positive and there is no positive integer less than 1.

\bigskip


\textbf{Solution to Exercise 2.18} Let $S$ be the set of all positive real
numbers. This set does not have a minimal element.


\bigskip


\textbf{Solution to Exercise 2.20} We are asked to show that Axiom (2) follows
from Lemma 2.19. So assume that Lemma 2.19 is true and we will prove
Axiom (2). Let $n$ be negative. We need to show that $-n$ is positive.
By Lemma 2.19 we can add $-n$ to both sides of the inequality $n < 0$ to get
$n-n < 0 -n$ or $0 < -n$. So $-n$ is positive.

\bigskip


\textbf{Solution to Exercise 2.21} Suppose $n$ is positive. We will show $-n$
is negative. By Lemma 2.19 we can add $-n$ to both sides of the
ineqality $0<n$ to get $-n < n-n = 0$, so $-n$ is negative.


\bigskip


\textbf{Solution to Exercise 2.24} This exercise was not in the right place.
It is more natural for this exercise to come after Exercise 2.29 because we
want to use Exercises 2.28 and 2.29 in the proof. Fix integers $a,b$.
We need to show that $\av{ab} = \av{a}\av{b}$. We consider several cases.
If $a=0$ or $b=0$ then both sides of the equation
$\av{ab} = \av{a}\av{b}$ equal zero. If $a>0$ and $b>0$ then both sides of
the equation $\av{ab} = \av{a}\av{b}$ equal $ab$. If $a$ is positive and
$b$ negative then, then by Exercise 2.29, $ab$ is negative and so
$\av{ab} = -(ab) = a (-b) = \av{a}\av{b}$. The proof is similar if
$a$ is negative and $b$ is positive.

\bigskip


\textbf{Solution to Exercise 2.28} We are asked to show that the product of two
positive integers is positive. But this follows immediately from
Lemma  2.27 because if $a,b>0$ then $ab\geq a > 0$.

\bigskip


\textbf{Solution to Exercise 2.29} We need to prove that the product of a
positive integer and a  negative integer is negative. Suppose that $a>0$
and $b<0$. Then $-b>0$. By the previous exercise $a(-b) > 0$ and so
$ab = -(a(-b)) < 0$.

Now we need to prove that the product of two negative integers is positive.
Suppose that $a,b<0$. Then $-a,-b > 0$ so $ab = 1\cdot ab = -1\cdot-1\cdot ab
=-a\cdot-b$ is positive.

\bigskip


\textbf{Solution to Exercise 2.32} Let $a<b$ and $x>0$. We need to show
that $ax < bx$. By Lemma 2.25, $b-a$ is positive.
Since $x$ is also positive, we have $bx - ax = (b-a)x$,
which is the product of two positives and so is
positive. Thus by Lemma 2.24 again $ax < bx$.

Now, let $a<b$ and $x<0$. We need to show that $bx < ax$. By Lemma 2.25,
$b-a$ is positive. Since $x$ is negative we have $bx - ax = (b-a)x$, which
is the product of a positive and a negative and so is negative. Thus
$-(bx - ax) = (ax -bx)$ is positive. So $ax < bx$.

\bigskip


\textbf{Solution to Exercise 2.33} Fix integers $n$ and $m$ with $n\not=0$.
Suppose $n\times m=n$. We need to show that $m=1$. This follows from
the Cancellation Law, Lemma 2.31. Since $m\times n = 1 \times n$ and $n\not=0$,
we can cancel $n$ to get $m=1$.

\bigskip


\textbf{Solution to Exercise 3.4}

\textbf{(a)} We are asked to show that
$a\divides b$ iff $-a\divides b$ iff $a\divides -b$. This follows from the
fact that $b=ac$ iff $b = (-a)(-c)$ iff $-b = a(-c)$.

\bigskip

\textbf{(b)}Suppse  $a\divides b \And a\divides c$. We must show that
$a\divides (b+c)$. Let $d$ and $e$ be such that $b=ad$ and $c=ae$.
Let $f=d+e$.
Then $b+c = ad + ae = a(d+e) =af$. So $a\divides (b+c)$.

\bigskip

\textbf{(c)} Suppose $a\divides b \And b\divides c$. We must show that
$a\divides c$. Let $d$ and $e$ be such that
$b=ad$ and $c=be$. Let f = $de$. Then $c=be=ade = af$. So $a\divides c$.

\bigskip

\textbf{Solution to Exercise 3.14}
\begin{itemize}
\item $17 \mod 100 = 17$
\item $100 \mod 17 = 15$
\item $-17 \bmod 100 = 83$
\item $-100 \bmod 17 = 2$
\item $27 \bmod 3 = 0$
\item $4 \bmod 3 = 1$
\item $3 \bmod 3 = 0$
\item $2 \bmod 3 = 2$
\item $1 \bmod 3 = 1$
\item $0 \bmod  = 0$
\item $-1 \bmod 3 = 2$
\item $-2 \bmod 3 = 1$
\item $-3 \bmod 3 = 0$
\item $-4 \bmod 3 = 2$
\end{itemize}

\bigskip

\textbf{Solution to Exercise 1.1 on page 4 of \emph{Shoup}}.
Let $a,b,d\in\Z$ with $d\not=0$. Show that $a\divides b$ if and only if
$da\divides db$.
\begin{proof}
Suppose $a\divides b$ and let $c$ be such that $b=ac$. Then $bd=adc$ so
$ad \divides bd$. Conversely suppose $ad \divides bd$ and let $c$ be
such that $bd = adc$. Since $d\not=0$ we can cancel it to yield
$b=ac$. So $a\divides b$.
\end{proof}

\bigskip

\textbf{Solution to Exercise 4.5} Suppose $I_1$ and $I_2$ are ideals. Show that $I_1 \intersect I_2$ is an ideal.
\begin{proof}
We must show that  $I_1 \intersect I_2$  satisfies the the three clauses in the definition of an ideal:
\begin{itemize}
\item It is non-empty.
\item It is closed under addition.
\item It is closed under scalar multiplication.
\end{itemize}
Since 0 is in both $I_1$ and $I_2$, 0 is in  $I_1 \intersect I_2$ and so  $I_1 \intersect I_2$  is non-empty.
To see that $I_1 \intersect I_2$ is closed under addition, let $a, b\in I_1 \intersect I_2$ and we will show that $a+b\in  I_1 \intersect I_2$.
Since  $a, b\in I_1 \intersect I_2$,  $a, b\in I_1$. Since $I_1$ is an ideal, $a+b\in I_1$. By the same argument $a+b\in I_2$.
So $a+b\in I_1\intersect I_2$.

To see that $I_1 \intersect I_2$ is closed under scalar multiplication, let $a \in I_1 \intersect I_2$ and let $x\in\Z$ and we will show that $xa\in  I_1 \intersect I_2$.
Since  $a\in I_1 \intersect I_2$,  $a\in I_1$. Since $I_1$ is an ideal, $xa\in I_1$. By the same argument $xa\in I_2$.
So $xa\in I_1\intersect I_2$.
\end{proof}

Now suppose that $\cS$ is a non-empty collection of ideals. Prove that the intersection
of all ideals in $\cS$ is also an ideal.

\begin{proof}
This proof is very similar to the previous one except it is a little more abstract because rather than having two ideals $I_1$ and $I_2$ we now have
a set $\cS$ of ideals. In order to follow this proof, think about the fact that if $\cS = \singleton{I_1, I_2}$ then this proof reduces to the previous proof.

Let $I$ be the intersection of all ideals in $\cS$. To see that $I$ is an ideal we must show that it is non-empty, closed under addition, and closed under scalar multiplication. It is non-empty because $0\in I$ because 0 is in every ideal in $\cS$. Let $a, b \in I$. Then $a$ and $b$ are in every ideal in $\cS$ so $a+b$ is in every ideal in $\cS$ so $a+b\in I$. Let $a\in I$ and $x\in\Z$. Then $a$ is in every ideal in $\cS$ so $xa$ is in every ideal in $\cS$ so $xa\in I$.
\end{proof}

\bigskip

\textbf{Solution to Exercise 4.8} Show that $a\Z$ is an ideal of $\Z$ that contains $a$.
\begin{proof}
$a\Z$ contains $a$ because $a=1\cdot a$ is a multiple of $a$. Thus $a\Z$ is not empty. To see that $a\Z$ is closed under addition, note that
$xa + ya = (x+y)a$ so that the sum of two multiples of $a$ is again a multiple of $a$. To see that $a\Z$ is closed under scalar multiplication,
note that $y(xa) = (yx)a$ so that a scalar multiple of a multiple of $a$ is also a multiple of $a$.
\end{proof}

\bigskip

\textbf{Solution to Exercise 4.10}  Note that we will use Lemma 4.9 so we know that $(a) = a\Z$.

$b\in (a)$ iff $a\divides b$.
\begin{proof}
This is easy because $b\in (a)$ iff $b\in a\Z$ iff $b$ is a multiple of $a$ iff $a\divides b$.
\end{proof}

For any ideal $I$, $b\in I$ iff $(b)\subseteq I$.
\begin{proof}
If $(b)\subseteq I$ then obviously $b\in I$. For the other direction, suppose $b\in I$. The reason we get $(b)\subseteq I$ is that $I$ is closed under scalar multiplication and so every multiple of $b$ is also in $I$.
\end{proof}

$(b)\subseteq (a)$ iff $a\divides b$.
\begin{proof}
If $(b)\subseteq (a)$ then $b\in (a)$ so $a\divides b$ by the first part of this exercise. If $a\divides b$ then $b\in (a)$ so by the second part of
this exercise $(b)\subseteq (a)$.
\end{proof}

\bigskip

\textbf{Solution to Exercise 4.12}  Show that $I_1+I_2$ is an ideal.
\begin{proof}
Let $I=I_1+I_2$. $0\in I$ because $0=0+0$. So $I$ is not empty. To see that $I$ is closed under addition,
let $a,b\in I$. Then there are $a_1,b_1\in I_1$ and $a_2,b_2\in I_2$ such that $a=a_1+a_2$ and $b=b_1+b_2$.
Then $a+b = (a_1+b_1) + (a_2 + b_2)$ and $(a_1+b_1)\in I_1$ and $(a_2+b_2)\in I_2$ so $a+b\in I$.
To see that $I$ is closed under scalar multiplication, let $a=a_1+a_2$ be as above and let $x\in\Z$.
Then $xa=xa_1+xa_2\in I$.
\end{proof}

\bigskip

\textbf{Solution to Exercise 4.13}  Let $(a)$ and $(b)$ be two principal ideals. Show that $(a)+(b)$ is equal to the set of all integer linear combinations of $a$ and $b$.
\begin{proof}
This follows just by unravelling definitions. An element of $(a)+(b)$ looks like $xa+yb$ for some $x,y\in\Z$.
But this is exactly an integer linear combination of $a$ and $b$ by definition.
\end{proof}

\bigskip

\textbf{Solution to Exercise 4.15}  Show that $(a,b)=(a)+(b)$.
\begin{proof}
$(a)+(b)$ is an ideal that contains $a$ and $b$ so $(a,b)\subseteq(a)+(b)$. By Lemma 4.3 every ideal
is closed under integer linear combinations, so $(a,b)$ contains all integer linear combinations of $a$ and $b$
so by the previous exercise $(a)+(b)\subseteq (a,b)$.
\end{proof}

\bigskip

\textbf{Solution to Exercise 4.22} In each case below, exprecss $\gcd(a,b)$ as a linear combination of $a$ and $b$.
\begin{enumerate}
\item $a=8,b=12$. $\gcd(8,12)=4$. $4=12-8$.
\item $a=20,b=12$. $\gcd(20,12)=4$. $4=2\cdot 12 - 20$.
\item $a=10,b=7$. $\gcd(10,7)=1$. $1=3\cdot 7 - 2 \cdot 10$.
\item $a=20,b=14$. $\gcd(20,14)=2$. $2=3 \cdot 14-2\cdot20$.
\end{enumerate}

\bigskip

\textbf{Solution to Shoup Exercise 1.8} Let $I$ be a non-empty set of integers that is closed under addition. Then $I$ is an ideal iff $-a\in I$ for all $a\in I$.
\begin{proof}
If $I$ is an ideal and $a\in I$ then $-a\in I$ because $-a = -1 \cdot a$ and $I$ is closed under scalar multiplication.

Conversely, suppose that for all $a\in I$, $-a\in I$. To see that $I$ is an ideal we need to see that $I$ is closed under scalar multiplication.
Let $a\in I$ and let $n\in \Z$. We need to see that $na\in I$.

First consider the case $n=0$. Then $na=0$ so we need to show that $0\in I$. This is true because $a$ and $-a$ are in $I$ and $I$ is closed under addition.

Next consider the case $n>0$. We prove by induction on $n$ that $na\in I$. For $n=1$ this is true. For the inductive step, assume $na\in I$ and we will show that $(n+1)a\in I$. This is true because $(n+1)a= na +a$ and both $na$ and $a$ are in $I$.

Finally consider the case $n<0$. We need to see that $na\in I$. Since $-n$ is positive, we now know that $-na\in I$. But then our hypothesis implies $-(-na)=na$ is also in $I$.
\end{proof}

\bigskip

\textbf{Solution to Shoup Exercise 1.9} Show that for all integers $a,b,c$ we have:
\begin{enumerate}
\item[(a)] $\gcd(a,b)=\gcd(b,a)$.
\begin{proof}
This follows immediately from several of our different characterizations of $\gcd$ which are easily seen to be symmetric in $a$ and $b$. For example we know that $\gcd$ is the greatest of the common divisors and the common divisors of $a$ and $b$ are the same as the common divisors of $b$ and $a$.
\end{proof}

\item[(b)] $\gcd(a,b) = \av{a} \Iff a\divides b$.
\begin{proof}
If $\gcd(a,b) = \av{a}$ then $\av{a}$ is a common divisor of $a$ and $b$
so in particular it is divisor of $b$. So $a\divides b$ or $-a\divides b$. But if $-a\divides b$ then also $a\divides b$.

Conversely, suppose $a\divides b$. Then also $-a\divides b$ so $\av{a} \divides b$. So $\av{a}$ is a non-negative common divisor of $a$ and $b$. To see it is the greatest common divisor, it suffices to see that every common
divisor divides it. Suppose $c$ is a common divisor of $a$ and $b$. In 
particular $c\divides a$ so $c$ divides $\av{a}$.
\end{proof}

\item[(c)] $\gcd(a,0)=\gcd(a,a) = \av{a}$ and $\gcd(a,1) = 1$.
\begin{proof}
That $\gcd(a,0)=\gcd(a,a) = \av{a}$ is easy to see if we think about ideals.
The principal ideal generated by $a$ and $0$ is the same as the principal
ideal generated by $a$ (because 0 is in that ideal anyway) which is the
same as the principal ideal generated by $-a$. In other words
$(a,0)=(a)=(-a) = (\av{a})$.

That $\gcd(a,1)=1$ may also be seen by thinking about ideals.
$(a,1) = (1)$ because $1\in (a,1)$ and $1\in (1)$ so $(a,1)=(1)=\Z$.
\end{proof}

\item[(d)] $\gcd(ca,cb) = \av{c} \gcd(a,b)$.
\begin{proof}
Let $d=\gcd(a,b)$ and let $\bar{c}=\av{c}$. We want to see that $\bar{c}d=\gcd(ca,cb)$. Since $d\divides a$, $cd \divides ca$, so also $-cd \divides ca$ so $\bar{c}d\divides ca$. Similarly $\bar{c}d \divides cb$. So $\bar{c}d$ is a non-negative
common divisor of $a$ and $b$. To see that it is the greatest common divisor
it suffices to see that $\bar{c}d$ can be written as a linear combination
of $ca$ and $cb$. Let $x$ and $y$ be integers such that $d=xa+yb$.
Then $cd=x(ca) + y(cb)$. Then $-cd=-x(ca) -y(cb)$. So $\bar{c}d$ can
be written as an integer linear combination of $ca$ and $cb$.
\end{proof}
\end{enumerate}

\bigskip

\textbf{Solution to Shoup Exercise 1.10} Show that for al integers $a,b$ with $d:= \gcd(a,b) \not= 0$, we have $\gcd(a/d,b/d)=1$.
\begin{proof}
Let $a,b,d\in \Z$ with $d=\gcd(a,b)$. Suppose $d\not=0$.
Since $d\divides a$ and $d\divides b$ and $d\not=0$, $a/d$ and $b/d$ are
integers. Let $\bar{a}=a/d$ and $\bar{b}=b/d$. We must show that $\bar{a}$
and $\bar{b}$ are relatively prime.
Since $d=\gcd(a,b)$ there are integers $x,y$ such that $d=xa+yb$. Dividing
both sides by $d$ we get $1=x\bar{a}+y\bar{b}$. So $\bar{a}$ and $\bar{b}$ are relatively prime.
\end{proof}

\bigskip

\textbf{Solution to Shoup Exercise 1.11} Let $n$ be an integer. Show that if $a,b$ are relatively prime integers, each of which divides $n$, then $ab$ divides $n$.
\begin{proof}
Suppose $a,b$ are relatively prime and $a\divides n$ and $b\divides n$.
We must show that $ab\divides n$.

Let $x,y$ be integers such that $xa+yb=1$. Then $xan + ybn = n$.
To see that $ab\divides n$, it suffices to see that $ab\divides xan$ and
$ab\divides ybn$. Since $b\divides n$ there is an integer 
$c$ such that $n=bc$.
Thus $xan=xabc$. Thus $ab\divides xan$. Similarly since $a\divides n$ there
is an integer $d$ such that $n=ad$. Thus $ybn=ybad$ so $ab\divides ybn$.
So $ab\divides n$.
\end{proof}

\bigskip

\textbf{Solution to Exercise 5.1} Let $p$ be a prime and let $a$ be any integer. Then either
\begin{itemize}
\item $p\divides a$ and $\gcd(p,a)=p$ or
\item $p\ndivides a$ and $\gcd(p,a)=1$.
\end{itemize}
\begin{proof}
Let $d=\gcd(p,a)$. Then in particular $d\divides p$ and $d$ is positive (since $p$ is positive $d\not=0$) so $d=1$ or $d=p$.

If $p\divides a$ then $p$ is a common divisor of $p$ and $a$ so we must have $d=p$.

If $p\ndivides a$ then $p$ is not a common divisor of $p$ and $a$ so
$d$ must be 1.
\end{proof}

\bigskip

\textbf{Solution to Exercise 1.2 from Shoup} Let $n$ be a composite integer. Show that there exists a prime $p$ dividing $n$ with $p\leq n^{1/2}$.
\begin{proof}
Since $n$ is composite there are $a,b$ with $1<a,b<n$ such that $n=ab$. Without loss of generality, assume that $a\leq b$. Since $a>1$, there is some prime $p$ such that $p\divides a$. Then $p\divides n$. Also $1<p\leq a$.
Thus $p^2\leq a^2 \leq ab=n$. So $p\leq n^{1/2}$.
\end{proof}

\bigskip

\textbf{Solution to Exercise 1.12 from Shoup} Show that two integers are relatively prime iff there is no one prime that divides both of them.
\begin{proof}
Let $a,b\in \Z$. First assume that $a$ and $b$ are relatively prime.
Then the only positive common divisor of $a$ and $b$ is 1, so in particular $a$ and $b$ have no common prime divisor.

Conversely, suppose that $a$ and $b$ are not relatively prime. Then they have some positive common divisor $d$. Let $p$ be any prime that divides $d$. Then $d$ divides both $a$ and $b$.
\end{proof}

\bigskip

\textbf{Solution to Exercise 1.15 from Shoup} An integer $a$ is called \textbf{square-free} if it is not divisible by the square of any integer greater than 1.

(a) $a$ is square-free iff $a=\pm p_1 \cdots p_r$, where the $p_i$'s are distinct primes.

\begin{proof}
Let $a=\pm p_1^{e_1} \cdots p_r^{e_r}$ be the prime factorization of $a$.
The condition on the right-hand-side is equivalent to $e_1=\cdots=e_r=1$.

Suppose $e_i\geq 2$ for some $i$. Then $p_i^2\divides a$ so $a$ is not square-free.

Conversely, suppose $d$ is positive and $d^2\divides a$. Let $p$ be a prime
divisor of $d$. Then $p$ is a prime divisor of $a$ so $p=p_i$ for some $i$.
Since $d^2\divides a$, $p^2\divides a$,so $e_i\geq 2$.
\end{proof}

\medskip

(b) Every positive integer $n$ can be expressed uniquely as $n=ab^2$, where $a$ and $b$ are positive integers, and $a$ is square-free.

\begin{proof}
Let $n$ be a positive integer and let $n=p_1^{e_1}\cdots p_r^{e_r}$
be the prime factorization of $n$, with the $p_i$ distinct. For each $i$,
if $e_i$ is even let $f_i$ be such that $e_i=2f_i$ and if $e_i$ is odd
let $f_i$ be such that $e_i=2f_i + 1$.

Let $a$ be the product of all of those $p_i$ such that $e_i$ is odd. Since
$a$ is a product of distinct primes, $a$ is square free.

Let $b=p_1^{f_1}\cdots p_r^{f_r}$.

Then $b^2=p_1^{2f_1}\cdots p_r^{2f_r}$ and 
$ab^2=p_1^{e_1}\cdots p_r^{e_r} = n$.

To show uniqueness, suppose that $n=ab^2$ where $a$ and $b$ are positive integers and $a$ is square-free. We will show that $a$ and $b$ must be as we just constructed above. Let $p_i$ be a prime factor of $n$ and let $e_i$ be the exponent of $p_i$ in the prime factorization of $n$. Let $f_i$ be the exponent of $p_i$ in the prime factorization of $b$ and let $g_i$ be the exponent of $p_i$ in the prime factorization of $a$. Here $f_i$ and $g_i$ are both allowed to be 0. Since $a$ is square free, $g_i$ is either zero or one.
We see that $e_i = 2f_i + g_i$.  Thus $g_i=1$ iff $e_i$ is odd. Thus $a$ is
the product of the prime factors of $n$ whose exponents are odd. In
other words, $a$ is the same $a$ that we constructed above. But then $b^2=n/a$ so $b$ is the same $b$ that we constructed above. This proves uniqueness.
\end{proof}

\bigskip

\textbf{Solution to Exercise 6.6}

\begin{enumerate}
\item[(a)] $\nu_5(100) = 2$ because $100=2^2 \cdot 5^2$.
\item[(b)] $\nu_5(-100) = 2$ because $-100=-1 \cdot 2^2 \cdot 5^2$.
\item[(c)] We are asked to find the smallest positive integers $a,b,c$ such that 
$\nu_2(a)=b$ and $\nu_3(b)=c$ and $\nu_2(c) > \nu_3(c)$. The smallest positive $c$ such
that $\nu_2(c) > \nu_3(c)$ is $c=2$. For this $c$ we have 
$\nu_2(c) = 1, \nu_3(c)=0$. The smallest positive integer $b$ such that 
$\nu_3(b)=2$ is $b=3^2=9$. The smallest positive integer $a$ such that
$\nu_2(a)=9$  is $a=2^9=512$.
\end{enumerate}

\bigskip

\textbf{Solution to Exercise 1.20 from Shoup} Let $a,b,c\in\Z$.
\begin{enumerate}
\item[(a)] $\lcm(a,b)=\lcm(b,a)$
\begin{subproof}
This is trivial because being a ``common multiple of $a$ and $b$'' is the same thing
as being a ``common multiple of $b$ and $a$.''
\end{subproof}
\item[(b)] $\lcm(a,b)=\av{a} \Iff b\divides a$.
\begin{subproof}
First consider the case $a\geq 0$ so we can ignore the absolute value signs. 

We will give two proofs of the fact that $\lcm(a,b) = a \Iff b\divides a$.

For the first proof we use the $\nu$ function.

\begin{align*}
\lcm(a,b) = a &\Iff \prod_p p^{\max\left(\nu_p(a)\nu_p(b)\right)} = \prod_p p^{\nu_p(a)} \\
              &\Iff \forall p \, \nu_p(a)\geq \nu_p(b) \\
              &\Iff b\divides a
\end{align*}


For the second proof of the fact $\lcm(a,b) = a \Iff b\divides a$ we directly use
divisibility.

If $a=\lcm(a,b)$ then in particular $a$ is a multiple of $b$ so $b\divides a$. Conversely if
$b\divides a$ then $a$ is a common multiple of $a$ and $b$ and $a$ divides every common
multiple of $a$ and $b$ so $a=\lcm(a,b)$.
 
Now suppose $a<0$ so $\av{a}=-a$. Notice $a$ and $-a$ have the same multiples. 
So $\lcm(a,b) = \av{a} \Iff \lcm(\av{a},b) = \av{a} \Iff b\divides\av{a} \Iff b\divides a$.
\end{subproof}

\item[(c)] $\lcm(a,a)=\lcm(a,1)=\av{a}$.
\begin{subproof}
The first equality is trivial becase a ``common multiple of $a$ and $a$'' is the same
thing as a multiple of $a$ which is the same thing as a ``common multiple of $a$ and 1'' 
since every integer is a multiple of 1.

For the second equality, first take the case that $a\geq 0$ so we can ignore the absolute
value signs. Then clearly $a$ itself is
the least non-negative multiple of $a$ so $a = \lcm(a,a)$.

Now take the case that $a$ is negative. Then clearly $\av{a}$ is the least positive
multiple of $a$ and so the least common multiple of $a$ and 1.
\end{subproof}

\item[(d)] $\lcm(ca,cb) = \av{c}\lcm(a,b)$.
\begin{subproof}
First lets consider the case $c\geq 0$ so we can ignore the absolute value signs.

We give two proofs of the fact that $\lcm(ca,cb) = c\lcm(a,b)$.

For the first proof we use the $\nu$ function.

\begin{align*}
\lcm(ca,cb)  &=  \prod_p p^{\max\left(\nu_p(ca),\nu_p(cb)\right)} \\
             &=  \prod_p p^{\max\left(\nu_p(c)+\nu_p(a),\nu_p(c) + \nu_p(b)\right)} \\
             &=  \prod_p p^{\nu_p(c) + \max(\nu_p(a),\nu_p(b))} \\
             &=  \prod_p p^{\nu_p(c)} \cdot \prod_p p^{\max(\nu_p(a),\nu_p(b))}\\
             &=  c \lcm(a,b) \\
\end{align*}

For the second proof of the fact that $\lcm(ca,cb) = c\lcm(a,b)$ we use straight divisibility.

Let $m=\lcm(a,b)$. So $ca\divides cm$ and $cb\divides cm$ and so $cm$ is a common multiple of $ca$ and $cb$.

Let $k$ be any common multiple of $ca$ and $cb$. So $k=cas=cbt$, for some integers $s,t$. 
Cancelling $c$ we get $as=bt$ is a common multiple of $a$ and $b$ and so a multiple of $m$: $as=bt=mr$
for some $r$. So $k=cmr$ so $cm\divides k$. We have shown that $m$ is a common multiple of $a$ and $b$ and that every common multiple of 
$a$ and $b$ divides m, so $m=\lcm(a,b)$.

Now suppose $c$ is negative. Then $\lcm(ca,cb) = \lcm(-ca,-cb) = -c \lcm(a,b) = \av{c} \lcm(a,b)$.
\end{subproof}

\end{enumerate}


\bigskip

\textbf{Solution to Exercise 1.21 from Shoup} Show that for all $a,b\in\Z$ we have:
\begin{enumerate}
\item[(a)] $\gcd(a,b) \cdot \lcm(a,b) = \av{ab}$.

If $a$ or $b$ are zero then both sides of the equation are zero. So now assume that $a$ and $b$ are non-zero.

First assume that $a$ and $b$ are positive so we can ignore the absolute value sign.

We will give two proofs of the fact that $\gcd(a,b) \cdot \lcm(a,b) = ab$.

For the first proof we use the $\nu$ function.

\begin{align*}
\gcd(a,c) \cdot \lcm(a,b)  &=  \prod_p p^{\min\left(\nu_p(a),\nu_p(b)\right)}\cdot \prod_p p^{\max\left(\nu_p(a),\nu_p(b)\right)} \\
           &= \prod_p p^{\min\left(\nu_p(a),\nu_p(b)\right)+\max\left(\nu_p(a),\nu_p(b)\right)} \\
           &= \prod_p p^{\nu_p(a) + \nu_p(b)} \\
           &= ab \\
\end{align*}


For the second proof of the fact that $\gcd(a,b) \cdot \lcm(a,b) = ab$ when $a,b>0$ we will use straight divisibility.
First we handle the special case that $a$ and $b$ are relatively prime.

\textbf{Lemma} Suppose $a$ and $b$ are relatively prime and positive. Then $\lcm(a,b) = ab$.
\begin{subproof}
Clearly $ab$ is a common multiple of $a$ and $b$. To see that it is the least common multiple, let
$k$ be any common multiple of $a$ and $b$. So there are $s$ and $t$ such that $k=sa=tb$.
Since $a\divides (sa)$, $a\divides (tb)$. Since $a$ is relatively prime to $b$, $a\divides t$. 
So write $t=ar$. Then $k=arb$. So $ab\divides k$. We have shown that $ab$ divides every common multiple of 
$a$ and $b$ so $ab=\lcm(a,b)$.
\end{subproof}

Now we consider the general case where $a$ and $b$ are not necessarily relatively prime. We want to show that
$\gcd(a,b) \cdot \lcm(a,b) = ab$ when $a,b>0$.

Let $d=\gcd(a,b)$. Since $a,b>0$, $d>0$.


Since $d\divides a$ and $d\divides b$ let $\abar, \bbar$ be such that
$a=d\abar$, $b=d\bbar$. 

By Exercise 1.10 from Shoup, $\abar$ and $\bbar$ are relatively prime.
Let $m=d\abar\bbar = ab/d$. We must show that $m=\lcm(a,b)$.

By the Lemma we just proved $\lcm(\abar,\bbar) = \abar \bbar$.

By part (d) of the previous exercise $\lcm(d\abar,d\bbar) = d\lcm(\abar,\bbar)$.
So $\lcm(a,b) = d\abar\bbar=m$.

If $a<0$ and $b\geq 0$ then $\lcm(a,b) = \lcm(-a,b) = \gcd(-a,b) (-ab) = \gcd(a,b) \av{ab}$.

Similarly if $b<0$ and $a\geq 0$.

If $a<0$ and $b<0$ then $\lcm(a,b) = \lcm(-a,-b) = \gcd(-a,-b) (-a\cdot -b) = \gcd(a,b) \av{ab}$.

\item[(b)] This part follows immediately from part (a).
\end{enumerate}

\bigskip

\textbf{Solution to Exercise 1.27 from Shoup.} For all $a,b\in\Z$,
$$\gcd(a+b,\lcm(a,b)) = \gcd(a,b).$$
\begin{proof}
First we handle the case that $a$ and $b$ are relatively prime. In this case we must show that
$a+b$ and $ab$ are relativley prime. This is true because if $p$ is a prime and $p$ divides
both $ab$ and $a+b$ then $p$ divides one of $a$ or $b$, but then since $p$ divides $a+b$ it
divides both $a$ and $b$, which is impossible since $a$ and $b$ are relatively prime.

Now let $a,b$ be arbitrary integers. Let $d=\gcd(a,b)$ and let $\abar,\bbar$ be such that
$a=d\abar,b=d\bbar$. Then $\abar,\bbar$ are relatively prime and letting $m=d\abar\bbar$ we have
that $m=\lcm(a,b)$. By what we have just shown,
$\gcd(\abar+\bbar,\abar\bbar) = 1$. Multiplying by $d$ and using a fact from an exercise we proved earlier,
$\gcd(d\abar+d\bbar,d\abar\bbar)=d$. So $\gcd(a+b,m)=d$ as was to be shown.
\end{proof}

\bigskip

\textbf{Solution to Exercise 1.28 from Shoup.} Show that for every positive integer $k$, there exists $k$
consecutive composite integers. Thus, there are arbitrarily loarge gaps
between primes.

\begin{note}
Do not feel bad if you were not able to solve this problem on your own--it required a flash of insight and
was not at all obvious. I expected you would ask for a hint. However it is important that you do understand
the proof now.
\end{note}

\begin{proof}
For every integer $k\geq 3$, consider the interval of integers ${[k!+2, k!+k]}$. This interval has length $k-1$.
We claim that every integer in that interval is composite. In fact ever integer in that interval is greater than $k$,
and we will show that every integer in that interval is divisible by an integer less than $k$ and greater than 1.
Every integer in that interval is of the form $k!+n$ for some $n$ with $1< n \leq k$. Since $n\divides (k!)$
and $n\divides n$, $n\divides (k!+n)$. Thus $(k!+n)$ is composite.
\end{proof}

\bigskip

\textbf{Solution to Exercise 1.29 from Shoup.} Let $p$ be prime. Show that for all $a,b\in\Z$ we have
$\nu_p(a+b)\geq\min\left\{\nu_p(a),\nu_p(b)\right\}$ and $\nu_p(a+b)=\nu_p(a)$ if $\nu_p(a)<\nu_p(b)$.

\begin{proof}
Let $N=\nu_p(a)$ and $M=\nu_p(b)$ and suppose without loss of generality that $N\leq M$.
Let $\abar,\bbar$ be such that $a=p^N\abar$, $b=p^M\bbar$, $p\ndivides\abar$, $p\ndivides\bbar$.
Then 
$$a+b=p^N\left(\abar+p^{M-N}\bbar\right).$$

 So $p^N\divides(a+b)$. So $\nu_p(a+b)\geq N = \min\left\{\nu_p(a),\nu_p(b)\right\}$.

Now suppose that $N<M$. This implies that $p\divides(p^{M-N}\bbar)$. But $p\ndivides\abar$.
Therefore $p\ndivides(\abar+p^{M-N}\bbar)$. This means that $\nu_p(a+b)=N=\nu_p(a)$.
Notice the point here is that if $N=M$ then we would have that $a+b=p^N(\abar+\bbar)$
and we would not know whether or not $p\divides(\abar+\bbar)$ (it might or it might not).
If $p\divides(\abar+\bbar)$ then $\nu_p(a+b) > N$.

\bigskip

\textbf{Solution to Exercise 1.34 from Shoup.}

\textbf{(a)} That the intersection of two ideals is an ideal was a homework problem form an earlier week.

\bigskip

\textbf{(b)}. Let $m=\lcm(a,b)$. We are supposed to show that $(m) = (a) \intersect (b)$.
$(a) \intersect (b)$ is precisely the set of common multiples of $a$ and $b$. So $m$ is, by definition
of $\lcm$, the least non-negative element of $(a)\intersect(b)$. But from the proof that
$\Z$ is a PID, this tells us that $m$ generates the ideal $(a)\intersect(b)$, i.e.
$(m)=(a)\intersect(b)$.

\end{proof}

\bigskip

\textbf{Solution to Exercise 1 from section 2-2 of Andrews.} Find the gcd of the
given pairs of integers using the Euclidean algorithm.

\bigskip

\textbf{(d)}  108, 243

\bigskip

$243 = 2\cdot 108 + 27$ \\
$108 = 4\cdot27 + 0$

\bigskip

So $\gcd(108, 243) = 27$.

\bigskip

\textbf{(e)} 132, 473

\bigskip

$473 = 3\cdot 132 + 77$ \\
$132 = 1 \cdot 77 + 55$ \\
$ 77 = 1 \cdot 55 + 22 \\
$ 55 = 2 \cdot 22 + 11 \\
$ 22 = 2 \cdot 11$.

\bigskip

So $\gcd(132,473) = 11$.



\end{document}

