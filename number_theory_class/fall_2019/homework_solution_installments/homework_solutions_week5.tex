\documentclass[oneside,12pt]{amsart}

\usepackage{amsmath,amssymb,latexsym,eucal,amsthm}

%%%%%%%%%%%%%%%%%%%%%%%%%%%%%%%%%%%%%%%%%%%%%
% Common Set Theory Constructs
%%%%%%%%%%%%%%%%%%%%%%%%%%%%%%%%%%%%%%%%%%%%%

\newcommand{\setof}[2]{\left\{ \, #1 \, \left| \, #2 \, \right.\right\}}
\newcommand{\lsetof}[2]{\left\{\left. \, #1 \, \right| \, #2 \,  \right\}}
\newcommand{\bigsetof}[2]{\bigl\{ \, #1 \, \bigm | \, #2 \,\bigr\}}
\newcommand{\Bigsetof}[2]{\Bigl\{ \, #1 \, \Bigm | \, #2 \,\Bigr\}}
\newcommand{\biggsetof}[2]{\biggl\{ \, #1 \, \biggm | \, #2 \,\biggr\}}
\newcommand{\Biggsetof}[2]{\Biggl\{ \, #1 \, \Biggm | \, #2 \,\Biggr\}}
\newcommand{\dotsetof}[2]{\left\{ \, #1 \, : \, #2 \, \right\}}
\newcommand{\bigdotsetof}[2]{\bigl\{ \, #1 \, : \, #2 \,\bigr\}}
\newcommand{\Bigdotsetof}[2]{\Bigl\{ \, #1 \, \Bigm : \, #2 \,\Bigr\}}
\newcommand{\biggdotsetof}[2]{\biggl\{ \, #1 \, \biggm : \, #2 \,\biggr\}}
\newcommand{\Biggdotsetof}[2]{\Biggl\{ \, #1 \, \Biggm : \, #2 \,\Biggr\}}
\newcommand{\sequence}[2]{\left\langle \, #1 \,\left| \, #2 \, \right. \right\rangle}
\newcommand{\lsequence}[2]{\left\langle\left. \, #1 \, \right| \,#2 \,  \right\rangle}
\newcommand{\bigsequence}[2]{\bigl\langle \,#1 \, \bigm | \, #2 \, \bigr\rangle}
\newcommand{\Bigsequence}[2]{\Bigl\langle \,#1 \, \Bigm | \, #2 \, \Bigr\rangle}
\newcommand{\biggsequence}[2]{\biggl\langle \,#1 \, \biggm | \, #2 \, \biggr\rangle}
\newcommand{\Biggsequence}[2]{\Biggl\langle \,#1 \, \Biggm | \, #2 \, \Biggr\rangle}
\newcommand{\singleton}[1]{\left\{#1\right\}}
\newcommand{\angles}[1]{\left\langle #1 \right\rangle}
\newcommand{\bigangles}[1]{\bigl\langle #1 \bigr\rangle}
\newcommand{\Bigangles}[1]{\Bigl\langle #1 \Bigr\rangle}
\newcommand{\biggangles}[1]{\biggl\langle #1 \biggr\rangle}
\newcommand{\Biggangles}[1]{\Biggl\langle #1 \Biggr\rangle}


\newcommand{\force}[1]{\Vert\!\underset{\!\!\!\!\!#1}{\!\!\!\relbar\!\!\!%
\relbar\!\!\relbar\!\!\relbar\!\!\!\relbar\!\!\relbar\!\!\relbar\!\!\!%
\relbar\!\!\relbar\!\!\relbar}}
\newcommand{\longforce}[1]{\Vert\!\underset{\!\!\!\!\!#1}{\!\!\!\relbar\!\!\!%
\relbar\!\!\relbar\!\!\relbar\!\!\!\relbar\!\!\relbar\!\!\relbar\!\!\!%
\relbar\!\!\relbar\!\!\relbar\!\!\relbar\!\!\relbar\!\!\relbar\!\!\relbar\!\!\relbar}}
\newcommand{\nforce}[1]{\Vert\!\underset{\!\!\!\!\!#1}{\!\!\!\relbar\!\!\!%
\relbar\!\!\relbar\!\!\relbar\!\!\!\relbar\!\!\relbar\!\!\relbar\!\!\!%
\relbar\!\!\not\relbar\!\!\relbar}}
\newcommand{\forcein}[2]{\overset{#2}{\Vert\underset{\!\!\!\!\!#1}%
{\!\!\!\relbar\!\!\!\relbar\!\!\relbar\!\!\relbar\!\!\!\relbar\!\!\relbar\!%
\!\relbar\!\!\!\relbar\!\!\relbar\!\!\relbar\!\!\relbar\!\!\!\relbar\!\!%
\relbar\!\!\relbar}}}

\newcommand{\pre}[2]{{}^{#2}{#1}}

\newcommand{\restr}{\!\!\upharpoonright\!}

%%%%%%%%%%%%%%%%%%%%%%%%%%%%%%%%%%%%%%%%%%%%%
% Set-Theoretic Connectives
%%%%%%%%%%%%%%%%%%%%%%%%%%%%%%%%%%%%%%%%%%%%%

\newcommand{\intersect}{\cap}
\newcommand{\union}{\cup}
\newcommand{\Intersection}[1]{\bigcap\limits_{#1}}
\newcommand{\Union}[1]{\bigcup\limits_{#1}}
\newcommand{\adjoin}{{}^\frown}
\newcommand{\forces}{\Vdash}

%%%%%%%%%%%%%%%%%%%%%%%%%%%%%%%%%%%%%%%%%%%%%
% Miscellaneous
%%%%%%%%%%%%%%%%%%%%%%%%%%%%%%%%%%%%%%%%%%%%%
\newcommand{\defeq}{=_{\text{def}}}
\newcommand{\Turingleq}{\leq_{\text{T}}}
\newcommand{\Turingless}{<_{\text{T}}}
\newcommand{\lexleq}{\leq_{\text{lex}}}
\newcommand{\lexless}{<_{\text{lex}}}
\newcommand{\Turingequiv}{\equiv_{\text{T}}}
\newcommand{\isomorphic}{\cong}

%%%%%%%%%%%%%%%%%%%%%%%%%%%%%%%%%%%%%%%%%%%%%
% Constants
%%%%%%%%%%%%%%%%%%%%%%%%%%%%%%%%%%%%%%%%%%%%%
\newcommand{\R}{\mathbb{R}}
\renewcommand{\P}{\mathbb{P}}
\newcommand{\Q}{\mathbb{Q}}
\newcommand{\Z}{\mathbb{Z}}
\newcommand{\Zpos}{\Z^{+}}
\newcommand{\Znonneg}{\Z^{\geq 0}}
\newcommand{\C}{\mathbb{C}}
\newcommand{\N}{\mathbb{N}}
\newcommand{\B}{\mathbb{B}}
\newcommand{\Bairespace}{\pre{\omega}{\omega}}
\newcommand{\LofR}{L(\R)}
\newcommand{\JofR}[1]{J_{#1}(\R)}
\newcommand{\SofR}[1]{S_{#1}(\R)}
\newcommand{\JalphaR}{\JofR{\alpha}}
\newcommand{\JbetaR}{\JofR{\beta}}
\newcommand{\JlambdaR}{\JofR{\lambda}}
\newcommand{\SalphaR}{\SofR{\alpha}}
\newcommand{\SbetaR}{\SofR{\beta}}
\newcommand{\Pkl}{\mathcal{P}_{\kappa}(\lambda)}
\DeclareMathOperator{\con}{con}
\DeclareMathOperator{\ORD}{OR}
\DeclareMathOperator{\Ord}{OR}
\DeclareMathOperator{\WO}{WO}
\DeclareMathOperator{\OD}{OD}
\DeclareMathOperator{\HOD}{HOD}
\DeclareMathOperator{\HC}{HC}
\DeclareMathOperator{\WF}{WF}
\DeclareMathOperator{\wfp}{wfp}
\DeclareMathOperator{\HF}{HF}
\newcommand{\One}{I}
\newcommand{\ONE}{I}
\newcommand{\Two}{II}
\newcommand{\TWO}{II}
\newcommand{\Mladder}{M^{\text{ld}}}

%%%%%%%%%%%%%%%%%%%%%%%%%%%%%%%%%%%%%%%%%%%%%
% Commutative Algebra Constants
%%%%%%%%%%%%%%%%%%%%%%%%%%%%%%%%%%%%%%%%%%%%%
\DeclareMathOperator{\dottimes}{\dot{\times}}
\DeclareMathOperator{\dotminus}{\dot{-}}
\DeclareMathOperator{\Spec}{Spec}

%%%%%%%%%%%%%%%%%%%%%%%%%%%%%%%%%%%%%%%%%%%%%
% Theories
%%%%%%%%%%%%%%%%%%%%%%%%%%%%%%%%%%%%%%%%%%%%%
\DeclareMathOperator{\ZFC}{ZFC}
\DeclareMathOperator{\ZF}{ZF}
\DeclareMathOperator{\AD}{AD}
\DeclareMathOperator{\ADR}{AD_{\R}}
\DeclareMathOperator{\KP}{KP}
\DeclareMathOperator{\PD}{PD}
\DeclareMathOperator{\CH}{CH}
\DeclareMathOperator{\GCH}{GCH}
\DeclareMathOperator{\HPC}{HPC} % HOD pair capturing
%%%%%%%%%%%%%%%%%%%%%%%%%%%%%%%%%%%%%%%%%%%%%
% Iteration Trees
%%%%%%%%%%%%%%%%%%%%%%%%%%%%%%%%%%%%%%%%%%%%%

\newcommand{\pred}{\text{-pred}}

%%%%%%%%%%%%%%%%%%%%%%%%%%%%%%%%%%%%%%%%%%%%%%%%
% Operator Names
%%%%%%%%%%%%%%%%%%%%%%%%%%%%%%%%%%%%%%%%%%%%%%%%
\DeclareMathOperator{\Det}{Det}
\DeclareMathOperator{\dom}{dom}
\DeclareMathOperator{\ran}{ran}
\DeclareMathOperator{\range}{ran}
\DeclareMathOperator{\image}{image}
\DeclareMathOperator{\crit}{crit}
\DeclareMathOperator{\card}{card}
\DeclareMathOperator{\cf}{cf}
\DeclareMathOperator{\cof}{cof}
\DeclareMathOperator{\rank}{rank}
\DeclareMathOperator{\ot}{o.t.}
\DeclareMathOperator{\ords}{o}
\DeclareMathOperator{\ro}{r.o.}
\DeclareMathOperator{\rud}{rud}
\DeclareMathOperator{\Powerset}{\mathcal{P}}
\DeclareMathOperator{\length}{lh}
\DeclareMathOperator{\lh}{lh}
\DeclareMathOperator{\limit}{lim}
\DeclareMathOperator{\fld}{fld}
\DeclareMathOperator{\projection}{p}
\DeclareMathOperator{\Ult}{Ult}
\DeclareMathOperator{\ULT}{Ult}
\DeclareMathOperator{\Coll}{Coll}
\DeclareMathOperator{\Col}{Col}
\DeclareMathOperator{\Hull}{Hull}
\DeclareMathOperator*{\dirlim}{dir lim}
\DeclareMathOperator{\Scale}{Scale}
\DeclareMathOperator{\supp}{supp}
\DeclareMathOperator{\trancl}{tran.cl.}
\DeclareMathOperator{\trace}{Tr}
\DeclareMathOperator{\diag}{diag}
\DeclareMathOperator{\spn}{span}
\DeclareMathOperator{\sgn}{sgn}
%%%%%%%%%%%%%%%%%%%%%%%%%%%%%%%%%%%%%%%%%%%%%
% Logical Connectives
%%%%%%%%%%%%%%%%%%%%%%%%%%%%%%%%%%%%%%%%%%%%%
\newcommand{\IImplies}{\Longrightarrow}
\newcommand{\SkipImplies}{\quad\Longrightarrow\quad}
\newcommand{\Ifff}{\Longleftrightarrow}
\newcommand{\iimplies}{\longrightarrow}
\newcommand{\ifff}{\longleftrightarrow}
\newcommand{\Implies}{\Rightarrow}
\newcommand{\Iff}{\Leftrightarrow}
\renewcommand{\implies}{\rightarrow}
\renewcommand{\iff}{\leftrightarrow}
\newcommand{\AND}{\wedge}
\newcommand{\OR}{\vee}
\newcommand{\st}{\text{ s.t. }}
\newcommand{\Or}{\text{ or }}

%%%%%%%%%%%%%%%%%%%%%%%%%%%%%%%%%%%%%%%%%%%%%
% Function Arrows
%%%%%%%%%%%%%%%%%%%%%%%%%%%%%%%%%%%%%%%%%%%%%

\newcommand{\injection}{\xrightarrow{\text{1-1}}}
\newcommand{\surjection}{\xrightarrow{\text{onto}}}
\newcommand{\bijection}{\xrightarrow[\text{onto}]{\text{1-1}}}
\newcommand{\cofmap}{\xrightarrow{\text{cof}}}
\newcommand{\map}{\rightarrow}

%%%%%%%%%%%%%%%%%%%%%%%%%%%%%%%%%%%%%%%%%%%%%
% Mouse Comparison Operators
%%%%%%%%%%%%%%%%%%%%%%%%%%%%%%%%%%%%%%%%%%%%%
\newcommand{\initseg}{\trianglelefteq}
\newcommand{\properseg}{\lhd}
\newcommand{\notinitseg}{\ntrianglelefteq}
\newcommand{\notproperseg}{\ntriangleleft}

%%%%%%%%%%%%%%%%%%%%%%%%%%%%%%%%%%%%%%%%%%%%%
%           calligraphic letters
%%%%%%%%%%%%%%%%%%%%%%%%%%%%%%%%%%%%%%%%%%%%%
\newcommand{\cA}{\mathcal{A}}
\newcommand{\cB}{\mathcal{B}}
\newcommand{\cC}{\mathcal{C}}
\newcommand{\cD}{\mathcal{D}}
\newcommand{\cE}{\mathcal{E}}
\newcommand{\cF}{\mathcal{F}}
\newcommand{\cG}{\mathcal{G}}
\newcommand{\cH}{\mathcal{H}}
\newcommand{\cI}{\mathcal{I}}
\newcommand{\cJ}{\mathcal{J}}
\newcommand{\cK}{\mathcal{K}}
\newcommand{\cL}{\mathcal{L}}
\newcommand{\cM}{\mathcal{M}}
\newcommand{\cN}{\mathcal{N}}
\newcommand{\cO}{\mathcal{O}}
\newcommand{\cP}{\mathcal{P}}
\newcommand{\cQ}{\mathcal{Q}}
\newcommand{\cR}{\mathcal{R}}
\newcommand{\cS}{\mathcal{S}}
\newcommand{\cT}{\mathcal{T}}
\newcommand{\cU}{\mathcal{U}}
\newcommand{\cV}{\mathcal{V}}
\newcommand{\cW}{\mathcal{W}}
\newcommand{\cX}{\mathcal{X}}
\newcommand{\cY}{\mathcal{Y}}
\newcommand{\cZ}{\mathcal{Z}}


%%%%%%%%%%%%%%%%%%%%%%%%%%%%%%%%%%%%%%%%%%%%%
%          Primed Letters
%%%%%%%%%%%%%%%%%%%%%%%%%%%%%%%%%%%%%%%%%%%%%
\newcommand{\aprime}{a^{\prime}}
\newcommand{\bprime}{b^{\prime}}
\newcommand{\cprime}{c^{\prime}}
\newcommand{\dprime}{d^{\prime}}
\newcommand{\eprime}{e^{\prime}}
\newcommand{\fprime}{f^{\prime}}
\newcommand{\gprime}{g^{\prime}}
\newcommand{\hprime}{h^{\prime}}
\newcommand{\iprime}{i^{\prime}}
\newcommand{\jprime}{j^{\prime}}
\newcommand{\kprime}{k^{\prime}}
\newcommand{\lprime}{l^{\prime}}
\newcommand{\mprime}{m^{\prime}}
\newcommand{\nprime}{n^{\prime}}
\newcommand{\oprime}{o^{\prime}}
\newcommand{\pprime}{p^{\prime}}
\newcommand{\qprime}{q^{\prime}}
\newcommand{\rprime}{r^{\prime}}
\newcommand{\sprime}{s^{\prime}}
\newcommand{\tprime}{t^{\prime}}
\newcommand{\uprime}{u^{\prime}}
\newcommand{\vprime}{v^{\prime}}
\newcommand{\wprime}{w^{\prime}}
\newcommand{\xprime}{x^{\prime}}
\newcommand{\yprime}{y^{\prime}}
\newcommand{\zprime}{z^{\prime}}
\newcommand{\Aprime}{A^{\prime}}
\newcommand{\Bprime}{B^{\prime}}
\newcommand{\Cprime}{C^{\prime}}
\newcommand{\Dprime}{D^{\prime}}
\newcommand{\Eprime}{E^{\prime}}
\newcommand{\Fprime}{F^{\prime}}
\newcommand{\Gprime}{G^{\prime}}
\newcommand{\Hprime}{H^{\prime}}
\newcommand{\Iprime}{I^{\prime}}
\newcommand{\Jprime}{J^{\prime}}
\newcommand{\Kprime}{K^{\prime}}
\newcommand{\Lprime}{L^{\prime}}
\newcommand{\Mprime}{M^{\prime}}
\newcommand{\Nprime}{N^{\prime}}
\newcommand{\Oprime}{O^{\prime}}
\newcommand{\Pprime}{P^{\prime}}
\newcommand{\Qprime}{Q^{\prime}}
\newcommand{\Rprime}{R^{\prime}}
\newcommand{\Sprime}{S^{\prime}}
\newcommand{\Tprime}{T^{\prime}}
\newcommand{\Uprime}{U^{\prime}}
\newcommand{\Vprime}{V^{\prime}}
\newcommand{\Wprime}{W^{\prime}}
\newcommand{\Xprime}{X^{\prime}}
\newcommand{\Yprime}{Y^{\prime}}
\newcommand{\Zprime}{Z^{\prime}}
\newcommand{\alphaprime}{\alpha^{\prime}}
\newcommand{\betaprime}{\beta^{\prime}}
\newcommand{\gammaprime}{\gamma^{\prime}}
\newcommand{\Gammaprime}{\Gamma^{\prime}}
\newcommand{\deltaprime}{\delta^{\prime}}
\newcommand{\epsilonprime}{\epsilon^{\prime}}
\newcommand{\kappaprime}{\kappa^{\prime}}
\newcommand{\lambdaprime}{\lambda^{\prime}}
\newcommand{\rhoprime}{\rho^{\prime}}
\newcommand{\Sigmaprime}{\Sigma^{\prime}}
\newcommand{\tauprime}{\tau^{\prime}}
\newcommand{\xiprime}{\xi^{\prime}}
\newcommand{\thetaprime}{\theta^{\prime}}
\newcommand{\Omegaprime}{\Omega^{\prime}}
\newcommand{\cMprime}{\cM^{\prime}}
\newcommand{\cNprime}{\cN^{\prime}}
\newcommand{\cPprime}{\cP^{\prime}}
\newcommand{\cQprime}{\cQ^{\prime}}
\newcommand{\cRprime}{\cR^{\prime}}
\newcommand{\cSprime}{\cS^{\prime}}
\newcommand{\cTprime}{\cT^{\prime}}

%%%%%%%%%%%%%%%%%%%%%%%%%%%%%%%%%%%%%%%%%%%%%
%          bar Letters
%%%%%%%%%%%%%%%%%%%%%%%%%%%%%%%%%%%%%%%%%%%%%
\newcommand{\abar}{\bar{a}}
\newcommand{\bbar}{\bar{b}}
\newcommand{\cbar}{\bar{c}}
\newcommand{\ibar}{\bar{i}}
\newcommand{\jbar}{\bar{j}}
\newcommand{\nbar}{\bar{n}}
\newcommand{\xbar}{\bar{x}}
\newcommand{\ybar}{\bar{y}}
\newcommand{\zbar}{\bar{z}}
\newcommand{\pibar}{\bar{\pi}}
\newcommand{\phibar}{\bar{\varphi}}
\newcommand{\psibar}{\bar{\psi}}
\newcommand{\thetabar}{\bar{\theta}}
\newcommand{\nubar}{\bar{\nu}}

%%%%%%%%%%%%%%%%%%%%%%%%%%%%%%%%%%%%%%%%%%%%%
%          star Letters
%%%%%%%%%%%%%%%%%%%%%%%%%%%%%%%%%%%%%%%%%%%%%
\newcommand{\phistar}{\phi^{*}}
\newcommand{\Mstar}{M^{*}}

%%%%%%%%%%%%%%%%%%%%%%%%%%%%%%%%%%%%%%%%%%%%%
%          dotletters Letters
%%%%%%%%%%%%%%%%%%%%%%%%%%%%%%%%%%%%%%%%%%%%%
\newcommand{\Gdot}{\dot{G}}

%%%%%%%%%%%%%%%%%%%%%%%%%%%%%%%%%%%%%%%%%%%%%
%         check Letters
%%%%%%%%%%%%%%%%%%%%%%%%%%%%%%%%%%%%%%%%%%%%%
\newcommand{\deltacheck}{\check{\delta}}
\newcommand{\gammacheck}{\check{\gamma}}


%%%%%%%%%%%%%%%%%%%%%%%%%%%%%%%%%%%%%%%%%%%%%
%          Formulas
%%%%%%%%%%%%%%%%%%%%%%%%%%%%%%%%%%%%%%%%%%%%%

\newcommand{\formulaphi}{\text{\large $\varphi$}}
\newcommand{\Formulaphi}{\text{\Large $\varphi$}}


%%%%%%%%%%%%%%%%%%%%%%%%%%%%%%%%%%%%%%%%%%%%%
%          Fraktur Letters
%%%%%%%%%%%%%%%%%%%%%%%%%%%%%%%%%%%%%%%%%%%%%

\newcommand{\fa}{\mathfrak{a}}
\newcommand{\fb}{\mathfrak{b}}
\newcommand{\fc}{\mathfrak{c}}
\newcommand{\fk}{\mathfrak{k}}
\newcommand{\fp}{\mathfrak{p}}
\newcommand{\fq}{\mathfrak{q}}
\newcommand{\fr}{\mathfrak{r}}
\newcommand{\fA}{\mathfrak{A}}
\newcommand{\fB}{\mathfrak{B}}
\newcommand{\fC}{\mathfrak{C}}
\newcommand{\fD}{\mathfrak{D}}

%%%%%%%%%%%%%%%%%%%%%%%%%%%%%%%%%%%%%%%%%%%%%
%          Bold Letters
%%%%%%%%%%%%%%%%%%%%%%%%%%%%%%%%%%%%%%%%%%%%%
\newcommand{\ba}{\mathbf{a}}
\newcommand{\bb}{\mathbf{b}}
\newcommand{\bc}{\mathbf{c}}
\newcommand{\bd}{\mathbf{d}}
\newcommand{\be}{\mathbf{e}}
\newcommand{\bu}{\mathbf{u}}
\newcommand{\bv}{\mathbf{v}}
\newcommand{\bw}{\mathbf{w}}
\newcommand{\bx}{\mathbf{x}}
\newcommand{\by}{\mathbf{y}}
\newcommand{\bz}{\mathbf{z}}
\newcommand{\bSigma}{\boldsymbol{\Sigma}}
\newcommand{\bPi}{\boldsymbol{\Pi}}
\newcommand{\bDelta}{\boldsymbol{\Delta}}
\newcommand{\bdelta}{\boldsymbol{\delta}}
\newcommand{\bgamma}{\boldsymbol{\gamma}}
\newcommand{\bGamma}{\boldsymbol{\Gamma}}

%%%%%%%%%%%%%%%%%%%%%%%%%%%%%%%%%%%%%%%%%%%%%
%         Bold numbers
%%%%%%%%%%%%%%%%%%%%%%%%%%%%%%%%%%%%%%%%%%%%%
\newcommand{\bzero}{\mathbf{0}}

%%%%%%%%%%%%%%%%%%%%%%%%%%%%%%%%%%%%%%%%%%%%%
% Projective-Like Pointclasses
%%%%%%%%%%%%%%%%%%%%%%%%%%%%%%%%%%%%%%%%%%%%%
\newcommand{\Sa}[2][\alpha]{\Sigma_{(#1,#2)}}
\newcommand{\Pa}[2][\alpha]{\Pi_{(#1,#2)}}
\newcommand{\Da}[2][\alpha]{\Delta_{(#1,#2)}}
\newcommand{\Aa}[2][\alpha]{A_{(#1,#2)}}
\newcommand{\Ca}[2][\alpha]{C_{(#1,#2)}}
\newcommand{\Qa}[2][\alpha]{Q_{(#1,#2)}}
\newcommand{\da}[2][\alpha]{\delta_{(#1,#2)}}
\newcommand{\leqa}[2][\alpha]{\leq_{(#1,#2)}}
\newcommand{\lessa}[2][\alpha]{<_{(#1,#2)}}
\newcommand{\equiva}[2][\alpha]{\equiv_{(#1,#2)}}


\newcommand{\Sl}[1]{\Sa[\lambda]{#1}}
\newcommand{\Pl}[1]{\Pa[\lambda]{#1}}
\newcommand{\Dl}[1]{\Da[\lambda]{#1}}
\newcommand{\Al}[1]{\Aa[\lambda]{#1}}
\newcommand{\Cl}[1]{\Ca[\lambda]{#1}}
\newcommand{\Ql}[1]{\Qa[\lambda]{#1}}

\newcommand{\San}{\Sa{n}}
\newcommand{\Pan}{\Pa{n}}
\newcommand{\Dan}{\Da{n}}
\newcommand{\Can}{\Ca{n}}
\newcommand{\Qan}{\Qa{n}}
\newcommand{\Aan}{\Aa{n}}
\newcommand{\dan}{\da{n}}
\newcommand{\leqan}{\leqa{n}}
\newcommand{\lessan}{\lessa{n}}
\newcommand{\equivan}{\equiva{n}}

\newcommand{\SigmaOneOmega}{\Sigma^1_{\omega}}
\newcommand{\SigmaOneOmegaPlusOne}{\Sigma^1_{\omega+1}}
\newcommand{\PiOneOmega}{\Pi^1_{\omega}}
\newcommand{\PiOneOmegaPlusOne}{\Pi^1_{\omega+1}}
\newcommand{\DeltaOneOmegaPlusOne}{\Delta^1_{\omega+1}}

%%%%%%%%%%%%%%%%%%%%%%%%%%%%%%%%%%%%%%%%%%%%%
% Linear Algebra
%%%%%%%%%%%%%%%%%%%%%%%%%%%%%%%%%%%%%%%%%%%%%
\newcommand{\transpose}[1]{{#1}^{\text{T}}}
\newcommand{\norm}[1]{\lVert{#1}\rVert}
\newcommand\aug{\fboxsep=-\fboxrule\!\!\!\fbox{\strut}\!\!\!}

%%%%%%%%%%%%%%%%%%%%%%%%%%%%%%%%%%%%%%%%%%%%%
% Number Theory
%%%%%%%%%%%%%%%%%%%%%%%%%%%%%%%%%%%%%%%%%%%%%
\newcommand{\av}[1]{\lvert#1\rvert}
\DeclareMathOperator{\divides}{\mid}
\DeclareMathOperator{\ndivides}{\nmid}
\DeclareMathOperator{\lcm}{lcm}
\DeclareMathOperator{\sign}{sign}
\newcommand{\floor}[1]{\left\lfloor{#1}\right\rfloor}
\DeclareMathOperator{\ConCl}{CC}
\DeclareMathOperator{\ord}{ord}


%%%%%%%%%%%%%%%%%%%%%%%%%%%%%%%%%%%%%%%%%%%%%%%%%%%%%%%%%%%%%%%%%%%%%%%%%%%
%%  Theorem-Like Declarations
%%%%%%%%%%%%%%%%%%%%%%%%%%%%%%%%%%%%%%%%%%%%%%%%%%%%%%%%%%%%%%%%%%%%%%%%%%

\newtheorem{theorem}{Theorem}[section]
\newtheorem{lemma}[theorem]{Lemma}
\newtheorem{corollary}[theorem]{Corollary}
\newtheorem{proposition}[theorem]{Proposition}


\theoremstyle{definition}

\newtheorem{definition}[theorem]{Definition}
\newtheorem{conjecture}[theorem]{Conjecture}
\newtheorem{remark}[theorem]{Remark}
\newtheorem{remarks}[theorem]{Remarks}
\newtheorem{notation}[theorem]{Notation}

\theoremstyle{remark}

\newtheorem*{note}{Note}
\newtheorem*{warning}{Warning}
\newtheorem*{question}{Question}
\newtheorem*{example}{Example}
\newtheorem*{fact}{Fact}


\newenvironment*{subproof}[1][Proof]
{\begin{proof}[#1]}{\renewcommand{\qedsymbol}{$\diamondsuit$} \end{proof}}

\newenvironment*{case}[1]
{\textbf{Case #1.  }\itshape }{}

\newenvironment*{claim}[1][Claim]
{\textbf{#1.  }\itshape }{}


\pagestyle{plain}

\begin{document}

\title{Homework Solutions, Week 5 \\ Math 310, Elementary Number Theory \\ Fall 2019}
\author{Mitch Rudominer}

\maketitle

\textbf{Solution to Exercise 1.12 from Shoup} Show that two integers are relatively prime iff there is no one prime that divides both of them.
\begin{proof}
Let $a,b\in \Z$. First assume that $a$ and $b$ are relatively prime.
Then the only positive common divisor of $a$ and $b$ is 1, so in particular $a$ and $b$ have no common prime divisor.

Conversely, suppose that $a$ and $b$ are not relatively prime. Then they have some positive common divisor $d$. Let $p$ be any prime that divides $d$. Then $d$ divides both $a$ and $b$.
\end{proof}

\bigskip

\textbf{Solution to Exercise 1.15 from Shoup} An integer $a$ is called \textbf{square-free} if it is not divisible by the square of any integer greater than 1.

(a) $a$ is square-free iff $a=\pm p_1 \cdots p_r$, where the $p_i$'s are distinct primes.

\begin{proof}
Let $a=\pm p_1^{e_1} \cdots p_r^{e_r}$ be the prime factorization of $a$.
The condition on the right-hand-side is equivalent to $e_1=\cdots=e_r=1$.

Suppose $e_i\geq 2$ for some $i$. Then $p_i^2\divides a$ so $a$ is not square-free.

Conversely, suppose $d$ is positive and $d^2\divides a$. Let $p$ be a prime
divisor of $d$. Then $p$ is a prime divisor of $a$ so $p=p_i$ for some $i$.
Since $d^2\divides a$, $p^2\divides a$,so $e_i\geq 2$.
\end{proof}

\medskip

(b) Every positive integer $n$ can be expressed uniquely as $n=ab^2$, where $a$ and $b$ are positive integers, and $a$ is square-free.

\begin{proof}
Let $n$ be a positive integer and let $n=p_1^{e_1}\cdots p_r^{e_r}$
be the prime factorization of $n$, with the $p_i$ distinct. For each $i$,
if $e_i$ is even let $f_i$ be such that $e_i=2f_i$ and if $e_i$ is odd
let $f_i$ be such that $e_i=2f_i + 1$.

Let $a$ be the product of all of those $p_i$ such that $e_i$ is odd. Since
$a$ is a product of distinct primes, $a$ is square free.

Let $b=p_1^{f_1}\cdots p_r^{f_r}$.

Then $b^2=p_1^{2f_1}\cdots p_r^{2f_r}$ and 
$ab^2=p_1^{e_1}\cdots p_r^{e_r} = n$.

To show uniqueness, suppose that $n=ab^2$ where $a$ and $b$ are positive integers and $a$ is square-free. We will show that $a$ and $b$ must be as we just constructed above. Let $p_i$ be a prime factor of $n$ and let $e_i$ be the exponent of $p_i$ in the prime factorization of $n$. Let $f_i$ be the exponent of $p_i$ in the prime factorization of $b$ and let $g_i$ be the exponent of $p_i$ in the prime factorization of $a$. Here $f_i$ and $g_i$ are both allowed to be 0. Since $a$ is square free, $g_i$ is either zero or one.
We see that $e_i = 2f_i + g_i$.  Thus $g_i=1$ iff $e_i$ is odd. Thus $a$ is
the product of the prime factors of $n$ whose exponents are odd. In
other words, $a$ is the same $a$ that we constructed above. But then $b^2=n/a$ so $b$ is the same $b$ that we constructed above. This proves uniqueness.
\end{proof}

\bigskip

\textbf{Solution to Exercise 6.6}

\begin{enumerate}
\item[(a)] $\nu_5(100) = 2$ because $100=2^2 \cdot 5^2$.
\item[(b)] $\nu_5(-100) = 2$ because $-100=-1 \cdot 2^2 \cdot 5^2$.
\item[(c)] We are asked to find the smallest positive integers $a,b,c$ such that 
$\nu_2(a)=b$ and $\nu_3(b)=c$ and $\nu_2(c) > \nu_3(c)$. The smallest positive $c$ such
that $\nu_2(c) > \nu_3(c)$ is $c=2$. For this $c$ we have 
$\nu_2(c) = 1, \nu_3(c)=0$. The smallest positive integer $b$ such that 
$\nu_3(b)=2$ is $b=3^2=9$. The smallest positive integer $a$ such that
$\nu_2(a)=9$  is $a=2^9=512$.
\end{enumerate}

\bigskip

\textbf{Solution to Exercise 1.20 from Shoup} Let $a,b,c\in\Z$.
\begin{enumerate}
\item[(a)] $\lcm(a,b)=\lcm(b,a)$
\begin{subproof}
This is trivial because being a ``common multiple of $a$ and $b$'' is the same thing
as being a ``common multiple of $b$ and $a$.''
\end{subproof}
\item[(b)] $\lcm(a,b)=\av{a} \Iff b\divides a$.
\begin{subproof}
First consider the case $a\geq 0$ so we can ignore the absolute value signs. 

We will give two proofs of the fact that $\lcm(a,b) = a \Iff b\divides a$.

For the first proof we use the $\nu$ function.

\begin{align*}
\lcm(a,b) = a &\Iff \prod_p p^{\max\left(\nu_p(a)\nu_p(b)\right)} = \prod_p p^{\nu_p(a)} \\
              &\Iff \forall p \, \nu_p(a)\geq \nu_p(b) \\
              &\Iff b\divides a
\end{align*}


For the second proof of the fact $\lcm(a,b) = a \Iff b\divides a$ we directly use
divisibility.

If $a=\lcm(a,b)$ then in particular $a$ is a multiple of $b$ so $b\divides a$. Conversely if
$b\divides a$ then $a$ is a common multiple of $a$ and $b$ and $a$ divides every common
multiple of $a$ and $b$ so $a=\lcm(a,b)$.
 
Now suppose $a<0$ so $\av{a}=-a$. Notice $a$ and $-a$ have the same multiples. 
So $\lcm(a,b) = \av{a} \Iff \lcm(\av{a},b) = \av{a} \Iff b\divides\av{a} \Iff b\divides a$.
\end{subproof}

\item[(c)] $\lcm(a,a)=\lcm(a,1)=\av{a}$.
\begin{subproof}
The first equality is trivial becase a ``common multiple of $a$ and $a$'' is the same
thing as a multiple of $a$ which is the same thing as a ``common multiple of $a$ and 1'' 
since every integer is a multiple of 1.

For the second equality, first take the case that $a\geq 0$ so we can ignore the absolute
value signs. Then clearly $a$ itself is
the least non-negative multiple of $a$ so $a = \lcm(a,a)$.

Now take the case that $a$ is negative. Then clearly $\av{a}$ is the least positive
multiple of $a$ and so the least common multiple of $a$ and 1.
\end{subproof}

\item[(d)] $\lcm(ca,cb) = \av{c}\lcm(a,b)$.
\begin{subproof}
First lets consider the case $c\geq 0$ so we can ignore the absolute value signs.

We give two proofs of the fact that $\lcm(ca,cb) = c\lcm(a,b)$.

For the first proof we use the $\nu$ function.

\begin{align*}
\lcm(ca,cb)  &=  \prod_p p^{\max\left(\nu_p(ca),\nu_p(cb)\right)} \\
             &=  \prod_p p^{\max\left(\nu_p(c)+\nu_p(a),\nu_p(c) + \nu_p(b)\right)} \\
             &=  \prod_p p^{\nu_p(c) + \max(\nu_p(a),\nu_p(b))} \\
             &=  \prod_p p^{\nu_p(c)} \cdot \prod_p p^{\max(\nu_p(a),\nu_p(b))}\\
             &=  c \lcm(a,b) \\
\end{align*}

For the second proof of the fact that $\lcm(ca,cb) = c\lcm(a,b)$ we use straight divisibility.

Let $m=\lcm(a,b)$. So $ca\divides cm$ and $cb\divides cm$ and so $cm$ is a common multiple of $ca$ and $cb$.

Let $k$ be any common multiple of $ca$ and $cb$. So $k=cas=cbt$, for some integers $s,t$. 
Cancelling $c$ we get $as=bt$ is a common multiple of $a$ and $b$ and so a multiple of $m$: $as=bt=mr$
for some $r$. So $k=cmr$ so $cm\divides k$. We have shown that $m$ is a common multiple of $a$ and $b$ and that every common multiple of 
$a$ and $b$ divides m, so $m=\lcm(a,b)$.

Now suppose $c$ is negative. Then $\lcm(ca,cb) = \lcm(-ca,-cb) = -c \lcm(a,b) = \av{c} \lcm(a,b)$.
\end{subproof}

\end{enumerate}


\bigskip

\textbf{Solution to Exercise 1.21 from Shoup} Show that for all $a,b\in\Z$ we have:
\begin{enumerate}
\item[(a)] $\gcd(a,b) \cdot \lcm(a,b) = \av{ab}$.

If $a$ or $b$ are zero then both sides of the equation are zero. So now assume that $a$ and $b$ are non-zero.

First assume that $a$ and $b$ are positive so we can ignore the absolute value sign.

We will give two proofs of the fact that $\gcd(a,b) \cdot \lcm(a,b) = ab$.

For the first proof we use the $\nu$ function.

\begin{align*}
\gcd(a,c) \cdot \lcm(a,b)  &=  \prod_p p^{\min\left(\nu_p(a),\nu_p(b)\right)}\cdot \prod_p p^{\max\left(\nu_p(a),\nu_p(b)\right)} \\
           &= \prod_p p^{\min\left(\nu_p(a),\nu_p(b)\right)+\max\left(\nu_p(a),\nu_p(b)\right)} \\
           &= \prod_p p^{\nu_p(a) + \nu_p(b)} \\
           &= ab \\
\end{align*}


For the second proof of the fact that $\gcd(a,b) \cdot \lcm(a,b) = ab$ when $a,b>0$ we will use straight divisibility.
First we handle the special case that $a$ and $b$ are relatively prime.

\textbf{Lemma} Suppose $a$ and $b$ are relatively prime and positive. Then $\lcm(a,b) = ab$.
\begin{subproof}
Clearly $ab$ is a common multiple of $a$ and $b$. To see that it is the least common multiple, let
$k$ be any common multiple of $a$ and $b$. So there are $s$ and $t$ such that $k=sa=tb$.
Since $a\divides (sa)$, $a\divides (tb)$. Since $a$ is relatively prime to $b$, $a\divides t$. 
So write $t=ar$. Then $k=arb$. So $ab\divides k$. We have shown that $ab$ divides every common multiple of 
$a$ and $b$ so $ab=\lcm(a,b)$.
\end{subproof}

Now we consider the general case where $a$ and $b$ are not necessarily relatively prime. We want to show that
$\gcd(a,b) \cdot \lcm(a,b) = ab$ when $a,b>0$.

Let $d=\gcd(a,b)$. Since $a,b>0$, $d>0$.


Since $d\divides a$ and $d\divides b$ let $\abar, \bbar$ be such that
$a=d\abar$, $b=d\bbar$. 

By Exercise 1.10 from Shoup, $\abar$ and $\bbar$ are relatively prime.
Let $m=d\abar\bbar = ab/d$. We must show that $m=\lcm(a,b)$.

By the Lemma we just proved $\lcm(\abar,\bbar) = \abar \bbar$.

By part (d) of the previous exercise $\lcm(d\abar,d\bbar) = d\lcm(\abar,\bbar)$.
So $\lcm(a,b) = d\abar\bbar=m$.

If $a<0$ and $b\geq 0$ then $\lcm(a,b) = \lcm(-a,b) = \gcd(-a,b) (-ab) = \gcd(a,b) \av{ab}$.

Similarly if $b<0$ and $a\geq 0$.

If $a<0$ and $b<0$ then $\lcm(a,b) = \lcm(-a,-b) = \gcd(-a,-b) (-a\cdot -b) = \gcd(a,b) \av{ab}$.

\item[(b)] This part follows immediately from part (a).
\end{enumerate}

\bigskip

\textbf{Solution to Exercise 1.27 from Shoup.} For all $a,b\in\Z$,
$$\gcd(a+b,\lcm(a,b)) = \gcd(a,b).$$
\begin{proof}
First we handle the case that $a$ and $b$ are relatively prime. In this case we must show that
$a+b$ and $ab$ are relativley prime. This is true because if $p$ is a prime and $p$ divides
both $ab$ and $a+b$ then $p$ divides one of $a$ or $b$, but then since $p$ divides $a+b$ it
divides both $a$ and $b$, which is impossible since $a$ and $b$ are relatively prime.

Now let $a,b$ be arbitrary integers. Let $d=\gcd(a,b)$ and let $\abar,\bbar$ be such that
$a=d\abar,b=d\bbar$. Then $\abar,\bbar$ are relatively prime and letting $m=d\abar\bbar$ we have
that $m=\lcm(a,b)$. By what we have just shown,
$\gcd(\abar+\bbar,\abar\bbar) = 1$. Multiplying by $d$ and using a fact from an exercise we proved earlier,
$\gcd(d\abar+d\bbar,d\abar\bbar)=d$. So $\gcd(a+b,m)=d$ as was to be shown.
\end{proof}

\bigskip

\textbf{Solution to Exercise 1.28 from Shoup.} Show that for every positive integer $k$, there exists $k$
consecutive composite integers. Thus, there are arbitrarily loarge gaps
between primes.

\begin{note}
Do not feel bad if you were not able to solve this problem on your own--it required a flash of insight and
was not at all obvious. I expected you would ask for a hint. However it is important that you do understand
the proof now.
\end{note}

\begin{proof}
For every integer $k\geq 3$, consider the interval of integers ${[k!+2, k!+k]}$. This interval has length $k-1$.
We claim that every integer in that interval is composite. In fact ever integer in that interval is greater than $k$,
and we will show that every integer in that interval is divisible by an integer less than $k$ and greater than 1.
Every integer in that interval is of the form $k!+n$ for some $n$ with $1< n \leq k$. Since $n\divides (k!)$
and $n\divides n$, $n\divides (k!+n)$. Thus $(k!+n)$ is composite.
\end{proof}

\bigskip

\textbf{Solution to Exercise 1.29 from Shoup.} Let $p$ be prime. Show that for all $a,b\in\Z$ we have
$\nu_p(a+b)\geq\min\left\{\nu_p(a),\nu_p(b)\right\}$ and $\nu_p(a+b)=\nu_p(a)$ if $\nu_p(a)<\nu_p(b)$.

\begin{proof}
Let $N=\nu_p(a)$ and $M=\nu_p(b)$ and suppose without loss of generality that $N\leq M$.
Let $\abar,\bbar$ be such that $a=p^N\abar$, $b=p^M\bbar$, $p\ndivides\abar$, $p\ndivides\bbar$.
Then 
$$a+b=p^N\left(\abar+p^{M-N}\bbar\right).$$

 So $p^N\divides(a+b)$. So $\nu_p(a+b)\geq N = \min\left\{\nu_p(a),\nu_p(b)\right\}$.

Now suppose that $N<M$. This implies that $p\divides(p^{M-N}\bbar)$. But $p\ndivides\abar$.
Therefore $p\ndivides(\abar+p^{M-N}\bbar)$. This means that $\nu_p(a+b)=N=\nu_p(a)$.
Notice the point here is that if $N=M$ then we would have that $a+b=p^N(\abar+\bbar)$
and we would not know whether or not $p\divides(\abar+\bbar)$ (it might or it might not).
If $p\divides(\abar+\bbar)$ then $\nu_p(a+b) > N$.

\bigskip

\textbf{Solution to Exercise 1.34 from Shoup.}

\textbf{(a)} That the intersection of two ideals is an ideal was a homework problem form an earlier week.

\bigskip

\textbf{(b)}. Let $m=\lcm(a,b)$. We are supposed to show that $(m) = (a) \intersect (b)$.
$(a) \intersect (b)$ is precisely the set of common multiples of $a$ and $b$. So $m$ is, by definition
of $\lcm$, the least non-negative element of $(a)\intersect(b)$. But from the proof that
$\Z$ is a PID, this tells us that $m$ generates the ideal $(a)\intersect(b)$, i.e.
$(m)=(a)\intersect(b)$.

\end{proof}



\end{document}

