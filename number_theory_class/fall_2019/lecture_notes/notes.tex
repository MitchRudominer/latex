\documentclass[oneside,12pt]{amsart}

\usepackage{amsmath,amssymb,latexsym,eucal,amsthm}

%%%%%%%%%%%%%%%%%%%%%%%%%%%%%%%%%%%%%%%%%%%%%
% Common Set Theory Constructs
%%%%%%%%%%%%%%%%%%%%%%%%%%%%%%%%%%%%%%%%%%%%%

\newcommand{\setof}[2]{\left\{ \, #1 \, \left| \, #2 \, \right.\right\}}
\newcommand{\lsetof}[2]{\left\{\left. \, #1 \, \right| \, #2 \,  \right\}}
\newcommand{\bigsetof}[2]{\bigl\{ \, #1 \, \bigm | \, #2 \,\bigr\}}
\newcommand{\Bigsetof}[2]{\Bigl\{ \, #1 \, \Bigm | \, #2 \,\Bigr\}}
\newcommand{\biggsetof}[2]{\biggl\{ \, #1 \, \biggm | \, #2 \,\biggr\}}
\newcommand{\Biggsetof}[2]{\Biggl\{ \, #1 \, \Biggm | \, #2 \,\Biggr\}}
\newcommand{\dotsetof}[2]{\left\{ \, #1 \, : \, #2 \, \right\}}
\newcommand{\bigdotsetof}[2]{\bigl\{ \, #1 \, : \, #2 \,\bigr\}}
\newcommand{\Bigdotsetof}[2]{\Bigl\{ \, #1 \, \Bigm : \, #2 \,\Bigr\}}
\newcommand{\biggdotsetof}[2]{\biggl\{ \, #1 \, \biggm : \, #2 \,\biggr\}}
\newcommand{\Biggdotsetof}[2]{\Biggl\{ \, #1 \, \Biggm : \, #2 \,\Biggr\}}
\newcommand{\sequence}[2]{\left\langle \, #1 \,\left| \, #2 \, \right. \right\rangle}
\newcommand{\lsequence}[2]{\left\langle\left. \, #1 \, \right| \,#2 \,  \right\rangle}
\newcommand{\bigsequence}[2]{\bigl\langle \,#1 \, \bigm | \, #2 \, \bigr\rangle}
\newcommand{\Bigsequence}[2]{\Bigl\langle \,#1 \, \Bigm | \, #2 \, \Bigr\rangle}
\newcommand{\biggsequence}[2]{\biggl\langle \,#1 \, \biggm | \, #2 \, \biggr\rangle}
\newcommand{\Biggsequence}[2]{\Biggl\langle \,#1 \, \Biggm | \, #2 \, \Biggr\rangle}
\newcommand{\singleton}[1]{\left\{#1\right\}}
\newcommand{\angles}[1]{\left\langle #1 \right\rangle}
\newcommand{\bigangles}[1]{\bigl\langle #1 \bigr\rangle}
\newcommand{\Bigangles}[1]{\Bigl\langle #1 \Bigr\rangle}
\newcommand{\biggangles}[1]{\biggl\langle #1 \biggr\rangle}
\newcommand{\Biggangles}[1]{\Biggl\langle #1 \Biggr\rangle}


\newcommand{\force}[1]{\Vert\!\underset{\!\!\!\!\!#1}{\!\!\!\relbar\!\!\!%
\relbar\!\!\relbar\!\!\relbar\!\!\!\relbar\!\!\relbar\!\!\relbar\!\!\!%
\relbar\!\!\relbar\!\!\relbar}}
\newcommand{\longforce}[1]{\Vert\!\underset{\!\!\!\!\!#1}{\!\!\!\relbar\!\!\!%
\relbar\!\!\relbar\!\!\relbar\!\!\!\relbar\!\!\relbar\!\!\relbar\!\!\!%
\relbar\!\!\relbar\!\!\relbar\!\!\relbar\!\!\relbar\!\!\relbar\!\!\relbar\!\!\relbar}}
\newcommand{\nforce}[1]{\Vert\!\underset{\!\!\!\!\!#1}{\!\!\!\relbar\!\!\!%
\relbar\!\!\relbar\!\!\relbar\!\!\!\relbar\!\!\relbar\!\!\relbar\!\!\!%
\relbar\!\!\not\relbar\!\!\relbar}}
\newcommand{\forcein}[2]{\overset{#2}{\Vert\underset{\!\!\!\!\!#1}%
{\!\!\!\relbar\!\!\!\relbar\!\!\relbar\!\!\relbar\!\!\!\relbar\!\!\relbar\!%
\!\relbar\!\!\!\relbar\!\!\relbar\!\!\relbar\!\!\relbar\!\!\!\relbar\!\!%
\relbar\!\!\relbar}}}

\newcommand{\pre}[2]{{}^{#2}{#1}}

\newcommand{\restr}{\!\!\upharpoonright\!}

%%%%%%%%%%%%%%%%%%%%%%%%%%%%%%%%%%%%%%%%%%%%%
% Set-Theoretic Connectives
%%%%%%%%%%%%%%%%%%%%%%%%%%%%%%%%%%%%%%%%%%%%%

\newcommand{\intersect}{\cap}
\newcommand{\union}{\cup}
\newcommand{\Intersection}[1]{\bigcap\limits_{#1}}
\newcommand{\Union}[1]{\bigcup\limits_{#1}}
\newcommand{\adjoin}{{}^\frown}
\newcommand{\forces}{\Vdash}

%%%%%%%%%%%%%%%%%%%%%%%%%%%%%%%%%%%%%%%%%%%%%
% Miscellaneous
%%%%%%%%%%%%%%%%%%%%%%%%%%%%%%%%%%%%%%%%%%%%%
\newcommand{\defeq}{=_{\text{def}}}
\newcommand{\Turingleq}{\leq_{\text{T}}}
\newcommand{\Turingless}{<_{\text{T}}}
\newcommand{\lexleq}{\leq_{\text{lex}}}
\newcommand{\lexless}{<_{\text{lex}}}
\newcommand{\Turingequiv}{\equiv_{\text{T}}}
\newcommand{\isomorphic}{\cong}

%%%%%%%%%%%%%%%%%%%%%%%%%%%%%%%%%%%%%%%%%%%%%
% Constants
%%%%%%%%%%%%%%%%%%%%%%%%%%%%%%%%%%%%%%%%%%%%%
\newcommand{\R}{\mathbb{R}}
\renewcommand{\P}{\mathbb{P}}
\newcommand{\Q}{\mathbb{Q}}
\newcommand{\Z}{\mathbb{Z}}
\newcommand{\Zpos}{\Z^{+}}
\newcommand{\Znonneg}{\Z^{\geq 0}}
\newcommand{\C}{\mathbb{C}}
\newcommand{\N}{\mathbb{N}}
\newcommand{\B}{\mathbb{B}}
\newcommand{\Bairespace}{\pre{\omega}{\omega}}
\newcommand{\LofR}{L(\R)}
\newcommand{\JofR}[1]{J_{#1}(\R)}
\newcommand{\SofR}[1]{S_{#1}(\R)}
\newcommand{\JalphaR}{\JofR{\alpha}}
\newcommand{\JbetaR}{\JofR{\beta}}
\newcommand{\JlambdaR}{\JofR{\lambda}}
\newcommand{\SalphaR}{\SofR{\alpha}}
\newcommand{\SbetaR}{\SofR{\beta}}
\newcommand{\Pkl}{\mathcal{P}_{\kappa}(\lambda)}
\DeclareMathOperator{\con}{con}
\DeclareMathOperator{\ORD}{OR}
\DeclareMathOperator{\Ord}{OR}
\DeclareMathOperator{\WO}{WO}
\DeclareMathOperator{\OD}{OD}
\DeclareMathOperator{\HOD}{HOD}
\DeclareMathOperator{\HC}{HC}
\DeclareMathOperator{\WF}{WF}
\DeclareMathOperator{\wfp}{wfp}
\DeclareMathOperator{\HF}{HF}
\newcommand{\One}{I}
\newcommand{\ONE}{I}
\newcommand{\Two}{II}
\newcommand{\TWO}{II}
\newcommand{\Mladder}{M^{\text{ld}}}

%%%%%%%%%%%%%%%%%%%%%%%%%%%%%%%%%%%%%%%%%%%%%
% Commutative Algebra Constants
%%%%%%%%%%%%%%%%%%%%%%%%%%%%%%%%%%%%%%%%%%%%%
\DeclareMathOperator{\dottimes}{\dot{\times}}
\DeclareMathOperator{\dotminus}{\dot{-}}
\DeclareMathOperator{\Spec}{Spec}

%%%%%%%%%%%%%%%%%%%%%%%%%%%%%%%%%%%%%%%%%%%%%
% Theories
%%%%%%%%%%%%%%%%%%%%%%%%%%%%%%%%%%%%%%%%%%%%%
\DeclareMathOperator{\ZFC}{ZFC}
\DeclareMathOperator{\ZF}{ZF}
\DeclareMathOperator{\AD}{AD}
\DeclareMathOperator{\ADR}{AD_{\R}}
\DeclareMathOperator{\KP}{KP}
\DeclareMathOperator{\PD}{PD}
\DeclareMathOperator{\CH}{CH}
\DeclareMathOperator{\GCH}{GCH}
\DeclareMathOperator{\HPC}{HPC} % HOD pair capturing
%%%%%%%%%%%%%%%%%%%%%%%%%%%%%%%%%%%%%%%%%%%%%
% Iteration Trees
%%%%%%%%%%%%%%%%%%%%%%%%%%%%%%%%%%%%%%%%%%%%%

\newcommand{\pred}{\text{-pred}}

%%%%%%%%%%%%%%%%%%%%%%%%%%%%%%%%%%%%%%%%%%%%%%%%
% Operator Names
%%%%%%%%%%%%%%%%%%%%%%%%%%%%%%%%%%%%%%%%%%%%%%%%
\DeclareMathOperator{\Det}{Det}
\DeclareMathOperator{\dom}{dom}
\DeclareMathOperator{\ran}{ran}
\DeclareMathOperator{\range}{ran}
\DeclareMathOperator{\image}{image}
\DeclareMathOperator{\crit}{crit}
\DeclareMathOperator{\card}{card}
\DeclareMathOperator{\cf}{cf}
\DeclareMathOperator{\cof}{cof}
\DeclareMathOperator{\rank}{rank}
\DeclareMathOperator{\ot}{o.t.}
\DeclareMathOperator{\ords}{o}
\DeclareMathOperator{\ro}{r.o.}
\DeclareMathOperator{\rud}{rud}
\DeclareMathOperator{\Powerset}{\mathcal{P}}
\DeclareMathOperator{\length}{lh}
\DeclareMathOperator{\lh}{lh}
\DeclareMathOperator{\limit}{lim}
\DeclareMathOperator{\fld}{fld}
\DeclareMathOperator{\projection}{p}
\DeclareMathOperator{\Ult}{Ult}
\DeclareMathOperator{\ULT}{Ult}
\DeclareMathOperator{\Coll}{Coll}
\DeclareMathOperator{\Col}{Col}
\DeclareMathOperator{\Hull}{Hull}
\DeclareMathOperator*{\dirlim}{dir lim}
\DeclareMathOperator{\Scale}{Scale}
\DeclareMathOperator{\supp}{supp}
\DeclareMathOperator{\trancl}{tran.cl.}
\DeclareMathOperator{\trace}{Tr}
\DeclareMathOperator{\diag}{diag}
\DeclareMathOperator{\spn}{span}
\DeclareMathOperator{\sgn}{sgn}
%%%%%%%%%%%%%%%%%%%%%%%%%%%%%%%%%%%%%%%%%%%%%
% Logical Connectives
%%%%%%%%%%%%%%%%%%%%%%%%%%%%%%%%%%%%%%%%%%%%%
\newcommand{\IImplies}{\Longrightarrow}
\newcommand{\SkipImplies}{\quad\Longrightarrow\quad}
\newcommand{\Ifff}{\Longleftrightarrow}
\newcommand{\iimplies}{\longrightarrow}
\newcommand{\ifff}{\longleftrightarrow}
\newcommand{\Implies}{\Rightarrow}
\newcommand{\Iff}{\Leftrightarrow}
\renewcommand{\implies}{\rightarrow}
\renewcommand{\iff}{\leftrightarrow}
\newcommand{\AND}{\wedge}
\newcommand{\OR}{\vee}
\newcommand{\st}{\text{ s.t. }}
\newcommand{\Or}{\text{ or }}

%%%%%%%%%%%%%%%%%%%%%%%%%%%%%%%%%%%%%%%%%%%%%
% Function Arrows
%%%%%%%%%%%%%%%%%%%%%%%%%%%%%%%%%%%%%%%%%%%%%

\newcommand{\injection}{\xrightarrow{\text{1-1}}}
\newcommand{\surjection}{\xrightarrow{\text{onto}}}
\newcommand{\bijection}{\xrightarrow[\text{onto}]{\text{1-1}}}
\newcommand{\cofmap}{\xrightarrow{\text{cof}}}
\newcommand{\map}{\rightarrow}

%%%%%%%%%%%%%%%%%%%%%%%%%%%%%%%%%%%%%%%%%%%%%
% Mouse Comparison Operators
%%%%%%%%%%%%%%%%%%%%%%%%%%%%%%%%%%%%%%%%%%%%%
\newcommand{\initseg}{\trianglelefteq}
\newcommand{\properseg}{\lhd}
\newcommand{\notinitseg}{\ntrianglelefteq}
\newcommand{\notproperseg}{\ntriangleleft}

%%%%%%%%%%%%%%%%%%%%%%%%%%%%%%%%%%%%%%%%%%%%%
%           calligraphic letters
%%%%%%%%%%%%%%%%%%%%%%%%%%%%%%%%%%%%%%%%%%%%%
\newcommand{\cA}{\mathcal{A}}
\newcommand{\cB}{\mathcal{B}}
\newcommand{\cC}{\mathcal{C}}
\newcommand{\cD}{\mathcal{D}}
\newcommand{\cE}{\mathcal{E}}
\newcommand{\cF}{\mathcal{F}}
\newcommand{\cG}{\mathcal{G}}
\newcommand{\cH}{\mathcal{H}}
\newcommand{\cI}{\mathcal{I}}
\newcommand{\cJ}{\mathcal{J}}
\newcommand{\cK}{\mathcal{K}}
\newcommand{\cL}{\mathcal{L}}
\newcommand{\cM}{\mathcal{M}}
\newcommand{\cN}{\mathcal{N}}
\newcommand{\cO}{\mathcal{O}}
\newcommand{\cP}{\mathcal{P}}
\newcommand{\cQ}{\mathcal{Q}}
\newcommand{\cR}{\mathcal{R}}
\newcommand{\cS}{\mathcal{S}}
\newcommand{\cT}{\mathcal{T}}
\newcommand{\cU}{\mathcal{U}}
\newcommand{\cV}{\mathcal{V}}
\newcommand{\cW}{\mathcal{W}}
\newcommand{\cX}{\mathcal{X}}
\newcommand{\cY}{\mathcal{Y}}
\newcommand{\cZ}{\mathcal{Z}}


%%%%%%%%%%%%%%%%%%%%%%%%%%%%%%%%%%%%%%%%%%%%%
%          Primed Letters
%%%%%%%%%%%%%%%%%%%%%%%%%%%%%%%%%%%%%%%%%%%%%
\newcommand{\aprime}{a^{\prime}}
\newcommand{\bprime}{b^{\prime}}
\newcommand{\cprime}{c^{\prime}}
\newcommand{\dprime}{d^{\prime}}
\newcommand{\eprime}{e^{\prime}}
\newcommand{\fprime}{f^{\prime}}
\newcommand{\gprime}{g^{\prime}}
\newcommand{\hprime}{h^{\prime}}
\newcommand{\iprime}{i^{\prime}}
\newcommand{\jprime}{j^{\prime}}
\newcommand{\kprime}{k^{\prime}}
\newcommand{\lprime}{l^{\prime}}
\newcommand{\mprime}{m^{\prime}}
\newcommand{\nprime}{n^{\prime}}
\newcommand{\oprime}{o^{\prime}}
\newcommand{\pprime}{p^{\prime}}
\newcommand{\qprime}{q^{\prime}}
\newcommand{\rprime}{r^{\prime}}
\newcommand{\sprime}{s^{\prime}}
\newcommand{\tprime}{t^{\prime}}
\newcommand{\uprime}{u^{\prime}}
\newcommand{\vprime}{v^{\prime}}
\newcommand{\wprime}{w^{\prime}}
\newcommand{\xprime}{x^{\prime}}
\newcommand{\yprime}{y^{\prime}}
\newcommand{\zprime}{z^{\prime}}
\newcommand{\Aprime}{A^{\prime}}
\newcommand{\Bprime}{B^{\prime}}
\newcommand{\Cprime}{C^{\prime}}
\newcommand{\Dprime}{D^{\prime}}
\newcommand{\Eprime}{E^{\prime}}
\newcommand{\Fprime}{F^{\prime}}
\newcommand{\Gprime}{G^{\prime}}
\newcommand{\Hprime}{H^{\prime}}
\newcommand{\Iprime}{I^{\prime}}
\newcommand{\Jprime}{J^{\prime}}
\newcommand{\Kprime}{K^{\prime}}
\newcommand{\Lprime}{L^{\prime}}
\newcommand{\Mprime}{M^{\prime}}
\newcommand{\Nprime}{N^{\prime}}
\newcommand{\Oprime}{O^{\prime}}
\newcommand{\Pprime}{P^{\prime}}
\newcommand{\Qprime}{Q^{\prime}}
\newcommand{\Rprime}{R^{\prime}}
\newcommand{\Sprime}{S^{\prime}}
\newcommand{\Tprime}{T^{\prime}}
\newcommand{\Uprime}{U^{\prime}}
\newcommand{\Vprime}{V^{\prime}}
\newcommand{\Wprime}{W^{\prime}}
\newcommand{\Xprime}{X^{\prime}}
\newcommand{\Yprime}{Y^{\prime}}
\newcommand{\Zprime}{Z^{\prime}}
\newcommand{\alphaprime}{\alpha^{\prime}}
\newcommand{\betaprime}{\beta^{\prime}}
\newcommand{\gammaprime}{\gamma^{\prime}}
\newcommand{\Gammaprime}{\Gamma^{\prime}}
\newcommand{\deltaprime}{\delta^{\prime}}
\newcommand{\epsilonprime}{\epsilon^{\prime}}
\newcommand{\kappaprime}{\kappa^{\prime}}
\newcommand{\lambdaprime}{\lambda^{\prime}}
\newcommand{\rhoprime}{\rho^{\prime}}
\newcommand{\Sigmaprime}{\Sigma^{\prime}}
\newcommand{\tauprime}{\tau^{\prime}}
\newcommand{\xiprime}{\xi^{\prime}}
\newcommand{\thetaprime}{\theta^{\prime}}
\newcommand{\Omegaprime}{\Omega^{\prime}}
\newcommand{\cMprime}{\cM^{\prime}}
\newcommand{\cNprime}{\cN^{\prime}}
\newcommand{\cPprime}{\cP^{\prime}}
\newcommand{\cQprime}{\cQ^{\prime}}
\newcommand{\cRprime}{\cR^{\prime}}
\newcommand{\cSprime}{\cS^{\prime}}
\newcommand{\cTprime}{\cT^{\prime}}

%%%%%%%%%%%%%%%%%%%%%%%%%%%%%%%%%%%%%%%%%%%%%
%          bar Letters
%%%%%%%%%%%%%%%%%%%%%%%%%%%%%%%%%%%%%%%%%%%%%
\newcommand{\abar}{\bar{a}}
\newcommand{\bbar}{\bar{b}}
\newcommand{\cbar}{\bar{c}}
\newcommand{\ibar}{\bar{i}}
\newcommand{\jbar}{\bar{j}}
\newcommand{\nbar}{\bar{n}}
\newcommand{\xbar}{\bar{x}}
\newcommand{\ybar}{\bar{y}}
\newcommand{\zbar}{\bar{z}}
\newcommand{\pibar}{\bar{\pi}}
\newcommand{\phibar}{\bar{\varphi}}
\newcommand{\psibar}{\bar{\psi}}
\newcommand{\thetabar}{\bar{\theta}}
\newcommand{\nubar}{\bar{\nu}}

%%%%%%%%%%%%%%%%%%%%%%%%%%%%%%%%%%%%%%%%%%%%%
%          star Letters
%%%%%%%%%%%%%%%%%%%%%%%%%%%%%%%%%%%%%%%%%%%%%
\newcommand{\phistar}{\phi^{*}}
\newcommand{\Mstar}{M^{*}}

%%%%%%%%%%%%%%%%%%%%%%%%%%%%%%%%%%%%%%%%%%%%%
%          dotletters Letters
%%%%%%%%%%%%%%%%%%%%%%%%%%%%%%%%%%%%%%%%%%%%%
\newcommand{\Gdot}{\dot{G}}

%%%%%%%%%%%%%%%%%%%%%%%%%%%%%%%%%%%%%%%%%%%%%
%         check Letters
%%%%%%%%%%%%%%%%%%%%%%%%%%%%%%%%%%%%%%%%%%%%%
\newcommand{\deltacheck}{\check{\delta}}
\newcommand{\gammacheck}{\check{\gamma}}


%%%%%%%%%%%%%%%%%%%%%%%%%%%%%%%%%%%%%%%%%%%%%
%          Formulas
%%%%%%%%%%%%%%%%%%%%%%%%%%%%%%%%%%%%%%%%%%%%%

\newcommand{\formulaphi}{\text{\large $\varphi$}}
\newcommand{\Formulaphi}{\text{\Large $\varphi$}}


%%%%%%%%%%%%%%%%%%%%%%%%%%%%%%%%%%%%%%%%%%%%%
%          Fraktur Letters
%%%%%%%%%%%%%%%%%%%%%%%%%%%%%%%%%%%%%%%%%%%%%

\newcommand{\fa}{\mathfrak{a}}
\newcommand{\fb}{\mathfrak{b}}
\newcommand{\fc}{\mathfrak{c}}
\newcommand{\fk}{\mathfrak{k}}
\newcommand{\fp}{\mathfrak{p}}
\newcommand{\fq}{\mathfrak{q}}
\newcommand{\fr}{\mathfrak{r}}
\newcommand{\fA}{\mathfrak{A}}
\newcommand{\fB}{\mathfrak{B}}
\newcommand{\fC}{\mathfrak{C}}
\newcommand{\fD}{\mathfrak{D}}

%%%%%%%%%%%%%%%%%%%%%%%%%%%%%%%%%%%%%%%%%%%%%
%          Bold Letters
%%%%%%%%%%%%%%%%%%%%%%%%%%%%%%%%%%%%%%%%%%%%%
\newcommand{\ba}{\mathbf{a}}
\newcommand{\bb}{\mathbf{b}}
\newcommand{\bc}{\mathbf{c}}
\newcommand{\bd}{\mathbf{d}}
\newcommand{\be}{\mathbf{e}}
\newcommand{\bu}{\mathbf{u}}
\newcommand{\bv}{\mathbf{v}}
\newcommand{\bw}{\mathbf{w}}
\newcommand{\bx}{\mathbf{x}}
\newcommand{\by}{\mathbf{y}}
\newcommand{\bz}{\mathbf{z}}
\newcommand{\bSigma}{\boldsymbol{\Sigma}}
\newcommand{\bPi}{\boldsymbol{\Pi}}
\newcommand{\bDelta}{\boldsymbol{\Delta}}
\newcommand{\bdelta}{\boldsymbol{\delta}}
\newcommand{\bgamma}{\boldsymbol{\gamma}}
\newcommand{\bGamma}{\boldsymbol{\Gamma}}

%%%%%%%%%%%%%%%%%%%%%%%%%%%%%%%%%%%%%%%%%%%%%
%         Bold numbers
%%%%%%%%%%%%%%%%%%%%%%%%%%%%%%%%%%%%%%%%%%%%%
\newcommand{\bzero}{\mathbf{0}}

%%%%%%%%%%%%%%%%%%%%%%%%%%%%%%%%%%%%%%%%%%%%%
% Projective-Like Pointclasses
%%%%%%%%%%%%%%%%%%%%%%%%%%%%%%%%%%%%%%%%%%%%%
\newcommand{\Sa}[2][\alpha]{\Sigma_{(#1,#2)}}
\newcommand{\Pa}[2][\alpha]{\Pi_{(#1,#2)}}
\newcommand{\Da}[2][\alpha]{\Delta_{(#1,#2)}}
\newcommand{\Aa}[2][\alpha]{A_{(#1,#2)}}
\newcommand{\Ca}[2][\alpha]{C_{(#1,#2)}}
\newcommand{\Qa}[2][\alpha]{Q_{(#1,#2)}}
\newcommand{\da}[2][\alpha]{\delta_{(#1,#2)}}
\newcommand{\leqa}[2][\alpha]{\leq_{(#1,#2)}}
\newcommand{\lessa}[2][\alpha]{<_{(#1,#2)}}
\newcommand{\equiva}[2][\alpha]{\equiv_{(#1,#2)}}


\newcommand{\Sl}[1]{\Sa[\lambda]{#1}}
\newcommand{\Pl}[1]{\Pa[\lambda]{#1}}
\newcommand{\Dl}[1]{\Da[\lambda]{#1}}
\newcommand{\Al}[1]{\Aa[\lambda]{#1}}
\newcommand{\Cl}[1]{\Ca[\lambda]{#1}}
\newcommand{\Ql}[1]{\Qa[\lambda]{#1}}

\newcommand{\San}{\Sa{n}}
\newcommand{\Pan}{\Pa{n}}
\newcommand{\Dan}{\Da{n}}
\newcommand{\Can}{\Ca{n}}
\newcommand{\Qan}{\Qa{n}}
\newcommand{\Aan}{\Aa{n}}
\newcommand{\dan}{\da{n}}
\newcommand{\leqan}{\leqa{n}}
\newcommand{\lessan}{\lessa{n}}
\newcommand{\equivan}{\equiva{n}}

\newcommand{\SigmaOneOmega}{\Sigma^1_{\omega}}
\newcommand{\SigmaOneOmegaPlusOne}{\Sigma^1_{\omega+1}}
\newcommand{\PiOneOmega}{\Pi^1_{\omega}}
\newcommand{\PiOneOmegaPlusOne}{\Pi^1_{\omega+1}}
\newcommand{\DeltaOneOmegaPlusOne}{\Delta^1_{\omega+1}}

%%%%%%%%%%%%%%%%%%%%%%%%%%%%%%%%%%%%%%%%%%%%%
% Linear Algebra
%%%%%%%%%%%%%%%%%%%%%%%%%%%%%%%%%%%%%%%%%%%%%
\newcommand{\transpose}[1]{{#1}^{\text{T}}}
\newcommand{\norm}[1]{\lVert{#1}\rVert}
\newcommand\aug{\fboxsep=-\fboxrule\!\!\!\fbox{\strut}\!\!\!}

%%%%%%%%%%%%%%%%%%%%%%%%%%%%%%%%%%%%%%%%%%%%%
% Number Theory
%%%%%%%%%%%%%%%%%%%%%%%%%%%%%%%%%%%%%%%%%%%%%
\newcommand{\av}[1]{\lvert#1\rvert}
\DeclareMathOperator{\divides}{\mid}
\DeclareMathOperator{\ndivides}{\nmid}
\DeclareMathOperator{\lcm}{lcm}
\DeclareMathOperator{\sign}{sign}
\newcommand{\floor}[1]{\left\lfloor{#1}\right\rfloor}
\DeclareMathOperator{\ConCl}{CC}
\DeclareMathOperator{\ord}{ord}


%%%%%%%%%%%%%%%%%%%%%%%%%%%%%%%%%%%%%%%%%%%%%%%%%%%%%%%%%%%%%%%%%%%%%%%%%%%
%%  Theorem-Like Declarations
%%%%%%%%%%%%%%%%%%%%%%%%%%%%%%%%%%%%%%%%%%%%%%%%%%%%%%%%%%%%%%%%%%%%%%%%%%

\newtheorem{theorem}{Theorem}[section]
\newtheorem{lemma}[theorem]{Lemma}
\newtheorem{corollary}[theorem]{Corollary}
\newtheorem{proposition}[theorem]{Proposition}


\theoremstyle{definition}

\newtheorem{definition}[theorem]{Definition}
\newtheorem{conjecture}[theorem]{Conjecture}
\newtheorem{remark}[theorem]{Remark}
\newtheorem{remarks}[theorem]{Remarks}
\newtheorem{notation}[theorem]{Notation}

\theoremstyle{remark}

\newtheorem*{note}{Note}
\newtheorem*{warning}{Warning}
\newtheorem*{question}{Question}
\newtheorem*{example}{Example}
\newtheorem*{fact}{Fact}


\newenvironment*{subproof}[1][Proof]
{\begin{proof}[#1]}{\renewcommand{\qedsymbol}{$\diamondsuit$} \end{proof}}

\newenvironment*{case}[1]
{\textbf{Case #1.  }\itshape }{}

\newenvironment*{claim}[1][Claim]
{\textbf{#1.  }\itshape }{}


\pagestyle{plain}

\begin{document}

\title{Lecture Notes \\ Math 310, Elementary Number Theory \\ Fall 2019}
\author{Mitch Rudominer}

\maketitle

\tableofcontents

%%%%%%%%%%%%%%%%%%%%%%%%%%%%%%%%%%%%%%%%%%%%%%%%%%%%%%%%%%%%%%%%%%%%%%%%%%%%%%%%%%%%%%%%

\newpage

\section{What is this class about? A whirlwind tour}

Elementary number theory is the theory of the integers.

$\Z = \singleton{\cdots, -3, -2, -1, 0, 1, 2, 3, \cdots} = $ the set of \emph{integers}.

Here is a quick peek at some ideas we will learn in this class. We'll zip through this quickly. Don't
worry if you don't fully understand the contents of this section yet (or understand it at all).
We'll cover all of this carefully later on. Here I'm just trying
to give you a flavor of the material you will learn in this class.

\begin{definition}[The $\bmod$ operator]
\label{ModOperatorTake1}
Let $a$ and $b$ be positive integers. Then $a \bmod b$ means the remainder when $a$ is divided by $b$.
\end{definition}

For example $7\bmod 3 = 1$, $8\bmod 2 = 0$, $5\bmod 8 = 5$.

\begin{definition} $\Z_n$ will refer to the ring of integers mod n. \end{definition}

This means that $\Z_n$ consists of the set of non-negative integers less than $n$ along
with two binary operations $\dotplus$, $\dottimes$ defined by

$$a \dotplus b = (a + b) \bmod n \text{ and } a \dottimes b = (a \times b) \bmod n.$$

Note that rather than writing $a\dotplus b$ and $a\dottimes b$ we will often just write
$a+b$ and $a\times b$ or $a\cdot b$ or $a b$ and say that we are doing arithmetic \emph{in $\Z_n$}.

For example $\Z_{10} = \singleton{0,1,2,3,4,5,6,7,8,9}$ under addition and multiplication $\bmod$ 10. This means for example that in $\Z_{10}$,
$6 \times 7 = 2$ because $42 \bmod 10 = 2$, and $4 + 9 = 3$ because $13 \bmod 10 = 3$.

In $\Z_{10}$ notice that some numbers have multiplicative inverses and some do not. For
example in $\Z_{10}, 3 \times 7 = 1$ so $3$ and $7$ are multiplicative inverses. But $4$ does not have any
multiplicative inverse in $\Z_{10}$ because the multiples of $4$ in $\Z_{10}$ are $4, 8, 2, 6, 0$ and so
$1$ is not a multiple of $4$ and so there is nothing you can multiply $4$ by to get $1$ and so
$4$ does not have a multiplicative inverse in $\Z_{10}$.

If an element of $\Z_{10}$ has an inverse then it is called a \emph{unit}.
So above we saw that in $\Z_{10}$, 3 and 7 are units but 4 is not.
Since $1\times 1 = 1$, 1 is always a unit in any $\Z_n$.
The set of all units of $\Z_{n}$ is written $\Z^*_{n}$.
You can check that $\Z^*_{10} = \singleton{1,3,7,9}$.

You can check that $\Z^*_{10}$ is closed under multiplication. That is, in $\Z_{10}$, if
$a,b$ are units then so is $ab$. For example 3 and 9 are units and in $\Z_{10}$, $3\times 9 = 7$
and $7$ is a unit. So if you take any $a\in \Z^*_{10}$, then all of the powers of $a$, $a^n$, are
also in $\Z^*_{10}$.

Consider the powers of $9$ in $\Z_{10}$. $9^2=1, 9^3 = 9, 9^4 =1, 9^5 = 9, \cdots$. So the only
powers of $9$ in $\Z_{10}$ are $9$ and $1$.

Now consider the powers of $3$ in $\Z_{10}$. $3^2=9, 3^3 = 7, 3^4 =1, 3^5 = 9, 3^6 = 7, \cdots$.
So the powers of $3$ in $\Z_{10}$ are $\singleton{1,3,7,9}$, i.e. all of $\Z^*_{10}$.

Because $3\in Z^*_{10}$ and all of the elements of $\Z^*_{10}$ can be written as
powers of $3$, we say
that $3$ is a \emph{generator} of $\Z^*_{10}$ and that 3 is a \emph{primitive root} $\bmod$ 10 or a primitive
root of 10. As we saw above, 9 is not a primitive root of 10 but 3 is.

Now let's play that whole game with $\Z_8$ instead of $\Z_{10}$.
$\Z_8 = \singleton{0,1,2,3,4,5,6,7}$ under addition and multiplication $\bmod$ 8.
$\Z^*_{8} = \singleton{1,3,5,7}$. Notice that $1^2 = 1, 3^2 = 1, 5^2 = 1, 7^2 = 1$. This means
that none of $\singleton{1,3,5,7}$ is a generator of $\Z^*_{8}$. $\Z^*_{8}$ does not have any
generators. 8 does not have any primitive roots.

Which integers have primitive roots? With effort you could check the following:
3 yes, 4 yes, 5 yes, 6 yes, 7 yes, 8 no, 9 yes, 10 yes, 11 yes, 12 no, 13 yes, 14 yes, 15 no,
16 no, 17 yes, 18 yes, 19 yes 20 no.

What's going on here? What is the pattern of which integers have primitive roots? After taking
 this
class, you will have a full and deep understanding of this question.

\newpage

\section{Axioms for the Integers}

\textbf{Question} Find the prime factorization of 20.

\textbf{Answer} $20=2^2 \times 5$?

\textbf{Question} Is this the only prime factorization of 20?

\textbf{Answer} Yes.

\textbf{Question} Find the prime factorization of 54.

\textbf{Answer} $54=2\times 3^3$?

\textbf{Question} Is this the only prime factorization of 54?

\textbf{Answer} Yes.

\begin{theorem}[The Fundamental Theorem Of Arithmetic] For all integers $n>1$, $n$ may be expressed
as a product of powers of primes: $n=p_1^{e_1}p_2^{e_2}\cdots p_k^{e_k}$, where each
$p_i$ is a prime number and each $e_i\geq 1$. This expression is unique up to the order of
the primes.
\end{theorem}

Our first major goal of the course will be to prove the Fundamental Theorem of Arithmetic. Think
for a moment about how you would prove it yourself.

One problem that arises immediately is this: The fact that you can factor integers into prime numbers is something we learn
early in school. If we are going to prove this fact then obviously we are not allowed to assume this fact. But this is
one of the ``basic" facts about the integers. If we are not allowed to assume this basic fact then which other basic facts
are we not allowed to assume? If we don't know what we are allowed to assume, then how could we possibly prove anything?

To resolve this problem we will write down exactly what it is we are planning to assume about the integers. We will call these assumptions the Axioms of the Integers. Everything else in
this course will be derived from these axioms.

\subsection{Axioms about addition}
The integers form an abelian group under addition.

\begin{enumerate}
\item $+$ is associative: $\forall n,m,k\in\Z, (n+m)+k=n+(m+k)$
\item $+$ is commutative: $\forall n,m\in\Z, n+m=m+n$
\item 0 is an additive identity: $\forall n\in\Z, n+0 = 0+n = n$
\item Every integer has an additive inverse: For all integers $n$ there exists another
integer called $-n$ such that $n + (-n) = (-n) + n = 0$.
\end{enumerate}

We write $a-b$ instead of $a + (-b)$.

\begin{remark}
The following facts are true and we do not need axioms for them:
\begin{itemize}
\item If $n=m$ then $n+k = m+k$.
\item If $n=m$ then $-n = -m$.
\end{itemize}
The reason we do not need axioms for these facts is that they follow from
the \emph{logical axioms} about the meaning of $=$. If $n$ and $m$ are different
names for the same thing, then, of course $-n$ and $-m$ are different names
for the same thing and $n+k$ and $m+k$ are names for the same thing.
\end{remark}

\begin{homework}[Bringing to the other side]
Using the above axioms, show that for all integers $a,b,c$, if
$a+b = c$ then $a = c - b$. Similarly, if $a-b=c$ then $a=b+c$.
Thus we are allowed to perform the usual operation of bringing a
term to the other side of the equation and changing its sign.
\end{homework}

\begin{homework}
\label{ZeroIsUnique}
Using the above axioms, show that $0$ is the only additive identity.  Even
stronger, show that for all integers $n$ and $m$, if $n+m=n$ then $m=0$.
\end{homework}

\begin{homework}
Using the above axioms, show that additive inverses are unique. Let $n,m\in\Z$.
Suppose that $n+m = 0$. Show that $n=-m$ and $m=-n$.
\end{homework}

\begin{homework}
Show that $-0 = 0$.
\end{homework}

\begin{homework}
Show that for all integers $n$, $-(-n) = n$.
\end{homework}

\subsection{Axioms about multiplication}
The integers form a commutative ring under addition and multiplication.

\begin{enumerate}
\item $\times$ is associative: $\forall n,m,k\in\Z, (n\times m )\times k = n \times (m \times k)$
\item $\times$ is commutative: $\forall n,m\in\Z, n\times m = m\times n$
\item 1 is a multiplicative identity: $\forall n\not= 0, n\times 1 = n$
\item $\times$ distributes over $+$: $\forall n,m,k\in\Z, n\times(m+k) = (n\times m) + (n\times k)$
\end{enumerate}

\begin{remark}
Notice that in $\Z$ there are additive inverses but not multiplicative inverses.
\end{remark}

We use several other notations besides $\times$ to indicate multiplication.
All of the following can mean $a\times b: a*b, a\cdot b, ab$.
Multiplication has higher order of precedence than addition in an expression so
that $ab+cd$ means $(a\times b) + (c\times d)$ and not $a(b+c)d$.

\begin{homework}
Using any of the previous axioms, show that for all integers $n$, $n\times 0 = 0$.
(Hint: Use exercise \ref{ZeroIsUnique}.)
\end{homework}

\begin{lemma}
\label{OneIsUnique}
$1$ is the unique multiplicative identity.
\end{lemma}
\begin{proof}
Suppose $z$ is an integer with the property that for all non-zero integers $n$,
$zn = n$. We will show that $z=1$. Because both $z$ and 1 are multiplicative
identities we have $z\cdot 1 = 1$ and $z\cdot 1 = z$. So $z=1$.
\end{proof}

\begin{remark}
Compare Lemma \ref{OneIsUnique} to exercise \ref{ZeroIsUnique}. You might have expected
us to make the stronger claim that if $z$ and $w$ are any two integers and
$w\not=0$ and $zw = w$ then $z = 1$. This is of course true but we cannot prove
it based on just the axioms listed so far. The reason is that so far our axioms
only tell us that $\Z$ is a commutative ring, and there are some commutative rings
in which this stronger property is not true. We will get this stronger property
in Exercise \ref{StrongerOneIsUnique} below, but that will depend on first proving
Lemma \ref{ZIsAnIntegralDomain} which we cannot do without first adding more axioms.
\end{remark}

\begin{lemma}
For all $n$, $-n=-1\times n$.
\end{lemma}
\begin{proof}
Let $b=-1\times n$. Then $b+n = -1\times n + 1\times n = (-1 + 1)n = 0$.
Since $b+n=0$, $b=-n$.
\end{proof}

\begin{homework}
Prove that for all integers $a,b$, $(-a)b=a(-b)=-(ab)$.
\end{homework}

\subsection{Axioms of linear order}
$(\Z, <)$ is a discrete linear order in which $n+1$ is the successor of $n$ and
$n-1$ is the predecessor of $n$.

\begin{enumerate}
\item $<$ is transitive: $\forall n,m,p \in \Z, n < m \And m < p \implies n < p$
\item $<$ is asymmetric: $\forall n,m \in \Z, n < m \implies m \not< n$.
\item $<$ is connected: $\forall n,m \in \Z, n\not=m \implies (n < m \Or m < n)$
\item For all integers $n$, $n+1$ is the successor of $n$, i.e. the least integer greater than $n$.
That is, $n<n+1$ and there is no $m$ such that $n < m < n+1$.
\end{enumerate}

\begin{homework}
Using any of the previous axioms, show that for all integers
$n$, $n-1$ is the predecessor of $n$, i.e. the greatest integer
less than $n$.
\end{homework}

\begin{definition}
If $n>0$ we say that $n$ is positive and if $n<0$ we say that $n$ is negative.
\end{definition}


\begin{homework}
Use the axioms listed so far to show that 1 is the least positive integer.
\end{homework}

\subsection{Axiom of wellordering}
The set of positive integers is wellorderd by $<$ and the negative numbers are the
reflection of the positive numbers.

\begin{enumerate}
\item Every non-empty set of positive integers contains a least element.
\item If $n$ is negative then $-n$ is positive.
\end{enumerate}

Axiom (1) allows us to use proof by induction on the positive integers,
and axiom (2) assures us that there are no non-standard negative integers.
The only use we make of axiom (2) is in proving Lemma \ref{AddToInequality}
(which actually implies axiom (2)).
\begin{lemma}[Proof by induction] Let P be any property of integers. Suppose
\begin{enumerate}
\item[(i)] $P(1)$ holds, and
\item For all $n>0$, $P(n) \implies P(n+1)$.
\end{enumerate}
Then $P(n)$ holds for all positive integers.
\end{lemma}
\begin{proof}
Let $S$ be the set of positive integers for which $P$ does not hold. We will show that
$S$ is empty. Suppose $S$ is not empty and we will derive a contradiction.
By axiom (1), $S$ has a least element, $m$. By (i), $m\not=1$. Since
$m$ is positive and 1 is the least positive integer, $1 < m$, so $1 \leq m-1$,
so $m-1$ is positive. In other words, $m=n+1$ for some positive integer $n$.
Since $n<m$ and $m$ is the least positive integer in $S$,
$n\notin S$. So $P(n)$ holds. By (ii), $P(n+1)$ holds. So $m\notin S$. Contradiction.
\end{proof}

\begin{homework}
Show that the set of positive \emph{real} numbers is not wellordered by $<$.
Give a non-empty set of positive real numbers that does not contain a least element.
\end{homework}

Now that we have induction at our disposal, we will be able to prove more interesting
things about the integers. We prove a series of lemmas that relate addition and multiplication with the ordering $<$.

\begin{lemma}[Add to both sides of an inequality]
\label{AddToInequality}
Let $a,b,x$ be integers and suppose that $a<b$. Then $a+x < b+x$.
\end{lemma}
\begin{proof}
First we prove by induction that the lemma is true for positive $x$. For $x=1$
we need to prove that
$a<b\implies a+1 < b+1.$
This is true because, since $a<b$ and $a+1$ is the least integer greater than $a$,
$a+1\leq b < b+1$.

Now suppose the lemma is true for $x$ and we will show it is true for $x+1$.
We are assuming by induction that $a+x<b+x$ and we need to show that $a+x+1<b+x+1$.
But this follows from the argument we just gave.

Notice that the lemma is trivially true for $x=0$. So to finish the proof we need to prove that
the lemma is true for all negative $x$.

Suppose $x<0$ and the lemma is not true for $x$ so that $a<b$ but $b+x \leq a+x$.
We will derive a contradiction.
It cannot be true that $b+x=a+x$ because that would imply that $b=a$. So we must have
that $b+x < a+x$. Because $x<0$, axiom (2) says that $-x>0$ and so by the lemma for positive $x$, which
we have already proved, $b+x+(-x) < a+x+(-x)$, i.e. $b<a$. But this contradicts
the hypothesis that $a<b$, and so our proof is complete.
\end{proof}

\begin{homework}
Show that our axiom (2) in this section follows from Lemma \ref{AddToInequality}.
(Hint: Use the lemma with $b=0$.) (But we used axiom (2) in proving the lemma
so we cannot get rid of it.)
\end{homework}

We don't need to assume axiom (2) for positive numbers because we can prove it.

\begin{homework}
If $n$ is positive then $-n$ is negative.
(Hint: Use the previous lemma with $a=0$.)
\end{homework}

Axiom (2) of this section allows us to define \emph{absolute value.}

\begin{definition}
If $a$ is a non-negative integer then $\av{a} = a$. If $a$ is a negative integer
then $\av{a} = -a$.
\end{definition}

\begin{remark}
For all non-zero integers $a$, $\av{a} \geq 1$.
\end{remark}

\begin{homework}
For all integers $a,b$, $\av{ab} = \av{a}\av{b}$.
\end{homework}

\begin{lemma}
\label{OrderInTermsOfPositive}
For all integers $a,b$, $a<b$ iff $b-a$ is positive.
\end{lemma}
\begin{proof}
Adding $a$ to both sides of $0 < b-a$ yields $a<b$.
Adding $-a$ to both sides of $a<b$ yields $0 < b-a$.
\end{proof}

\begin{lemma}
\label{AddingPositiveIncreases}
For all integers $a,x$ with $x>0$, $a-x < a <a+x$.
\end{lemma}
\begin{proof}
The first inequality follows because $a-(a-x) = x$ is positive and
the second inequality follows because $a+x -a = x$ is positive.
\end{proof}

\begin{lemma}
\label{ProductOfTwoPositives}
If $a,b>0$ then $a \leq ab$.
\end{lemma}
\begin{proof}
By induction on $b$. For $b=1$ we have $a=a\cdot 1$. Now suppose the
lemma is true for $b$ and we will prove it is true for $b+1$.
We are assuming that $a\leq ab$. Then, since $a>0$, Lemma \ref{AddingPositiveIncreases}
tells us that $a \leq ab < ab+a = a(b+1)$.
\end{proof}

\begin{homework}
The product of two positive integers is positive.
\end{homework}

\begin{homework}
The product of a positive integer and a negative integer is negative.
The product of two negative integers is positive.
(Hint: Use the fact that $(-a)b = -(ab))$.
\end{homework}

\begin{lemma}[$\Z$ is an integral domain.]
\label{ZIsAnIntegralDomain}
If $n$ and $m$ are non-zero integers then $n\times m \not=0$.
\end{lemma}
\begin{proof}
This follows from the previous two exercises. If $n$ and $m$ are both
positive or both negative then their product is positive. Otherwise there
product is negative. In all cases their product is not zero.
\end{proof}

\begin{lemma}[Cancellation Law]
\label{CancellationLaw}
For all integers $a,b,c$, if $a\not=0$
and $ab = ac$ then $b=c$.
\end{lemma}
\begin{proof}
$ab=ac\IImplies ab-ac=0 \IImplies a(b-c) = 0$. Since $a\not=0$, Lemma
\ref{ZIsAnIntegralDomain} implies that $b-c=0$, so $b=c$.
\end{proof}

\begin{homework}
If $a<b$ and $x>0$ then $ax < bx$.
If $a<b$ and $x<0$ then $bx < ax$.
(Hint: Use Lemma \ref{OrderInTermsOfPositive}.)
\end{homework}

This exercise is a strengthening of Lemma \ref{OneIsUnique}. Compare the statement
to the statement of exercise \ref{ZeroIsUnique}.
\begin{homework}
\label{StrongerOneIsUnique}
For all integers $n$ and $m$, if $n\not=0$
and $n\times m=n$ then $m=1$.
\end{homework}

\textbf{Textbook Reading}
\emph{Preliminaries}, pages xiv-xvii of \emph{Shoup}.

\newpage

\section{Divisibility}

\begin{definition}
If $a$ and $b$ are integers we write $a\divides b$, pronounced $a$ \emph{divides}
$b$, iff there is an integer $c$ such that $b=ac$. If it is not true that
$a\divides b$ we write $a\nmid b$.

If $a\divides$ be we also say that $a$ is a \emph{divisor} of $b$ and that
$b$ is a \emph{multiple} of $a$ and that $b$ is \emph{divisible} by $a$.
\end{definition}

\begin{in_class_example}
Find all of the positive divisors of 12.
\end{in_class_example}


\begin{numbered_example}
Some examples of divisibility
\begin{itemize}
\item $3\divides 6$, $3\divides -6$, $-3\divides 6$, $-3\divides -6$.
\item $3 \ndivides 8$, $6\ndivides 3$.
\item For all integers $b$, $1\divides b$ because $b = 1\cdot b$.
\item For all integers $a$, $a\divides a$ because $a = a\cdot 1$.
\item For all integers $a$, $a\divides 0$ because $0 = a\cdot 0$.
\item In particular $0\divides 0$.
\item If $b\not=0$ then $0\ndivides b$ because for all $c$, $b\not=0\cdot c$.
\end{itemize}

\end{numbered_example}

\begin{lemma}
If $a\divides b$ and $b\not=0$ then $1 \leq \av{a} \leq \av{b}$.
\end{lemma}
\begin{proof}
Suppose $b=ac$ and $b\not=0$. Then $c\not=0$, so $\av{c}\geq 1$, and
$\av{b} = \av{ac} = \av{a}\av{c}$. So $\av{b} \geq \av{a}$ by Lemma
\ref{ProductOfTwoPositives}.
\end{proof}

\begin{homework}
Prove the following. For all integers $a,b,c$
\begin{itemize}
\item $a\divides b$ iff $-a\divides b$ iff $a\divides -b$.
\item $a\divides b \And a\divides c \IImplies a\divides (b+c)$.
\item $a\divides b \And b\divides c \IImplies  a\divides c$.
\end{itemize}
\end{homework}

\begin{definition}
Let $a$ be a nonzero element of a commutative ring. Then $a$ is called a
\emph{zero-divisors} iff there is a nonzero element $b$ such that $ab=0$.
\end{definition}

There are no zero-divisors in $\Z$. But other rings we will consider in this
class do have zero-divisors.

\begin{definition}
An integer $a$ is called a \emph{unit} iff $a\divides 1$.
\end{definition}

\begin{lemma}
$a$ is a unit iff $a$ has a multiplicative inverse.
\end{lemma}
\begin{proof}
$a\divides 1$ iff there is a $c$ such that $ac=1$. This $c$ is a multiplicative
inverse of $a$ by definition.
\end{proof}

\begin{lemma}
If $u$ is a unit and $a=bu$ then there is a unit $t$ such that $b=at$.
\end{lemma}
\begin{proof}
Since $u$ is a unit there is a $t$ such that $ut=1$. Multiply both sides
of $a=bu$ by $t$ to get $at=but=b$.
\end{proof}

\begin{definition}
Two integers $a$ and $b$ are called \emph{associates} iff there
is a unit $u$ such that $a=bu$, or equivalently there is a unit $t$ such
that $b=at$.
\end{definition}

\begin{lemma}
Let $a$ and $b$ be integers. Then $a$ and $b$ are associates iff $a\divides b$
and $b\divides a$.
\end{lemma}
\begin{proof}
If $a$ and $b$ are associates then there are units $u$ and $t$ such that
$a=bu$ and $b=at$. Therefore $a\divides b$ and $b\divides a$.

Conversely, suppose $a\divides b$ and $b\divides a$. Let $c$ and $d$ be such
that $a=bc$ and $b=ad$. Then $a=adc$. By the Cancellation Law Lemma
\ref{CancellationLaw} $1=dc$. This means that $d$ and $c$ are units and so
$a$ and $b$ are associates.
\end{proof}

\begin{lemma}
The units of $\Z$ are 1 and -1.

Therefore in $\Z$, $a$ and $b$ are associates iff $a=\pm b$.

Also $a\divides b$ and $b\divides a$ iff $a=\pm b$.
\end{lemma}
\begin{proof}
Suppose $a\divides 1$ so there is a $b$ such that $1=ab$.
Then $a,b\not=0$ so $\av{a}\geq 1$, $\av{b}\geq 1$.
We have $1=\av{1} = \av{a}\av{b} \geq \av{a} \geq 1$. So $\av{a} = 1$
so $a=1$ or $a=-1$.
\end{proof}

\begin{theorem}[Quotient and Remainder Property. a.k.a. Euclid's Division Lemma]
Let $a,b$ be integers with $b>0$. Then there are unique
integers $q,r$ with $0\leq r < b$
such that $a = qb+r$.

$q$ is called the \emph{quotient} and $r$ is called the remainder.
\end{theorem}
\begin{proof}
Let $S=\setof{a-qb}{q\in\Z\And a-qb\geq 0}$.

\begin{claim}[Claim 1]
$S\not=\emptyset$.
\end{claim}

\begin{subproof}[Proof of Claim 1]
If $a\geq 0$ then let $q=0$. Then $a=a-qb\geq0$ so $a-qb\in S$.
If $a<0$, let $q=a$. Then $a-qb=a-ab=a(1-b)$. Since $a<0$ and $1-b\leq 0$,
$a-qb\geq 0$ so $a-qb\in S$. In either case we have found an element of $S$
so $S\not=\emptyset$.
\end{subproof}

Since $S$ is a non-empty set of non-negative integers, let $r$ be the smallest element of $S$,
and let $q$ be such that $r=a-qb$. Then $a=qb+r$ and $r\geq 0$.

\begin{claim}[Claim 2]
$r<b$.
\end{claim}

\begin{subproof}[Proof of Claim 2]
Suppose $r\geq b$ and let $\rprime = r -b$. Then $0\leq\rprime < r$ and
$\rprime = r - b = a-qb -b = a -(q+1) b \in S$. This contradicts the fact that
$r$ was chosen to be the smallest element of $S$.
\end{subproof}

So we have shown that there are integers $q,r$ with $0\leq r < b$
such that $a = qb+r$. What's left is to show that these are the unique such integers.

Suppose there were another pair of integers $\qprime$ and $\rprime$ such that
$0\leq \rprime < b$ and $a = \qprime b+\rprime$. Then
$qb+r = \qprime b + \rprime$ so $(q - \qprime)b = \rprime-r$. We claim
that $q=\qprime$ and $r=\rprime$. If not then  $q-\qprime\not=0$ and
$r-\rprime\not=0$.  Assume $q>\qprime$ (or else swap the names of $q$ and $\qprime$.)
Then $q-\qprime > 0$ and $b>0$ and $r\geq 0$ so
$\rprime \geq \rprime-r = (q - \qprime)b \geq b$.
This contradicts the fact that $\rprime < b$, so $q=\qprime$ and $r=\rprime$.
\end{proof}

In definition \ref{ModOperatorTake1} we defined the $\bmod$ operator on positive
numbers $a$ and $b$. Now we give a slightly more general definition where we
don't require $a$ to be positive.

\begin{definition}[The $\bmod$ operator]
\label{ModOperator}
Let $a$ be any integer and $b$ a positive integer. Let $q$ and $r$
be the unique quotient and remainder from the previous theorem so that
$a=qb+r$ with $0\leq r < b$.Then we define
$a \bmod b=r$.
\end{definition}

\begin{homework} Compute the following remainders.
\begin{itemize}
\item $17 \bmod 100$
\item $100 \bmod 17$
\item $-17 \bmod 100$
\item $-100 \bmod 17$
\item $27 \bmod 3$
\item $4 \bmod 3$
\item $3 \bmod 3$
\item $2 \bmod 3$
\item $1 \bmod 3$
\item $0 \bmod 3$
\item $-1 \bmod 3$
\item $-2 \bmod 3$
\item $-3 \bmod 3$
\item $-4 \bmod 3$
\end{itemize}
\end{homework}

\textbf{Textbook Reading} Read Section 1.1
\emph{Divisibility and primality}, pages 1 -4 of \emph{Shoup}.

\textbf{Textbook Exercises:} Do exercise 1.1 on page 4 of \emph{Shoup}.

\newpage

\section{Ideals and GCDs}

The $\gcd$ or greatest common divisor of two integers is an easy concept.
$\gcd(a,b)$ is the greatest integer that divides both $a$ and $b$.
For example $\gcd(12,18) = 6$. This is because the positive divisors of 12 are
1,2,3,4,6,12 and the positive divisors of 18 are 1,2,3,6, 9, 18. The common
positive divisors of 12 and 18 are 1,2,3,6 and the greatest of these is 6.
We will eventually get back to this simple understanding of $\gcd$, but it will
pay to start with a more abstract and sophisticated definition of $\gcd$
based on the concept of an ideal.

\begin{definition}
If $a_1,a_2,\cdots, a_n$ are some fixed set of integers,
then an \emph{integer linear combination}
of $a_1,a_2,\cdots a_n$ is any value of the form
$$x_1 a_1 + x_2 a_2 + \cdots + x_n a_n $$
where the $x_i$ are any integers. In this context the $x_i$ are called
\emph{scalars} and they are also called the \emph{coefficients} of the
linear combination.
\end{definition}

Later we will see that we can make sense of integer linear combinations
where the $a_1,a_2,\cdots, a_n$ are objects other than integers. The thing that
makes this an \emph{integer} linear combination is that the \emph{coefficients}
are integers, not that the $a_i$ are integers.

\begin{definition}
An \emph{ideal} of $\Z$ is a non-empty set $I\subset\Z$ such that
\begin{enumerate}
\item $I$ is closed under addition: If $a,b\in I$ then $a+b\in I$.
\item $I$ is closed under scalar multiplication:
If $a\in I$ then for all integers $x$, $xa\in I$.
\end{enumerate}
\end{definition}

\begin{lemma}
Let $I$ be an ideal of $\Z$. Then $0\in I$ and $I$ is closed under subtraction
and integer linear combinations.
\end{lemma}
\begin{proof}
By definition, $I$ is not empty so let $a\in I$. Then $0=0\cdot a \in I$.

Let $a,b\in I$. Then $-1\cdot b = -b \in I$ so $a + (-b) = a-b\in I$.

Suppose $a_1,a_2,\cdots,a_n\in I$ and let
$b=x_1 a_1 + x_2 a_2 + \cdots + x_n a_n$ be any integer linear combination of
the $a_1,a_2,\cdots,a_n$. Then by repeatedly applying closure under addition
and scalar multiplication we can see that $b\in I$. Formally we can arrange this
 as a proof by induction on $n$. For $n=1$, $x_1 a_1 \in I$ by closure under
 scalar multiplication. Now suppose
 $b= x_1 a_1 + x_2 a_2 + \cdots + x_n a_n \in I$ and
 we sill show that $c = x_1 a_1 + x_2 a_2 + \cdots + x_n a_n + x_{n+1}a_{n+1} \in I$.
 $x_{n+1}a_{n+1} \in I$ by closure under scalar multiplication and so
 $c=b+x_{n+1}a_{n+1} \in I$ by closure under addition.
\end{proof}

\begin{lemma}
Let $I$ be an ideal of $\Z$. Then $I=\Z$ iff $1\in I$.
\end{lemma}
\begin{proof}
If $I=\Z$ then of course $1\in I$. Conversely suppose $1\in I$. Then for
all $z\in\Z$, $z=1\cdot z \in I$.
\end{proof}

\begin{homework}
Suppose $I_1$ and $I_2$ are ideals. Show that $I_1\intersect I_2$ is an ideal.

Now suppose that $\cS$ is a non-empty collection of ideals. Prove that the intersection
of all ideals in $\cS$ is also an ideal.
\end{homework}

\begin{definition}
Let $a$ be an integer. Then the \emph{principal ideal generated by $a$},
written $(a)$, is defined as the smallest ideal of $\Z$ containing the element $a$.
(To see that there \emph{is} a smallest ideal containing $a$, consider the
intersection of all ideals containing $a$ and notice that by the previous
exercise, that is an ideal.)
\end{definition}

\begin{definition}
Let $a$ be an integer. Then $a\Z$ is the set of all integer multiples of $a$.
\end{definition}

\begin{homework}
Show that $a\Z$ is an ideal of $\Z$ that contains $a$.
\end{homework}

\begin{lemma}
Let $a$ be an integer. Then $(a)=a\Z$.
\end{lemma}
\begin{proof}
Since $(a)$ is closed under
scalar multiplication and $a\in (a)$, $a\Z\subseteq (a)$. On the other hand,
$(a)$ is the smallest ideal containing $a$ and $a\Z$ is an ideal containing
$a$ so $(a)\subseteq a\Z$.
\end{proof}

\begin{in_class_example}
\quad
\begin{itemize}
\item Letting $a=0$ we have $(0)=\singleton{0}$.
\item Letting $a=1$ we have $(1) = \Z$.
\item Letting $a=2$ we have $(2)$ is the set of all even numbers.
\end{itemize}
\end{in_class_example}

\begin{homework}
\quad
\begin{itemize}
\item $b\in (a)$ iff $a\divides b$.
\item For any ideal $I$, $b\in I$ iff $(b) \subseteq I$.
\item $(b)\subseteq (a)$ iff $a \divides b$.
\end{itemize}
\end{homework}

\begin{definition}
Let $I_1$ and $I_2$ be ideals. Then $I_1+I_2$ is defined by
$$I_1+I_2 = \setof{a_1+a_2}{a_1\in I_1\And a_2 \in I_2}.$$
\end{definition}

\begin{homework}
Show that $I_1 + I_2$ is an ideal.
\end{homework}

\begin{homework}
Let $(a)$ and $(b)$ be two principal ideals.
Show that $(a)+(b)$ is equal to the set of
all integer linear combinations of $a$ and $b$.
\end{homework}

\begin{definition}
If $a$ and $b$ are two integers, then $(a,b)$ is defined to be the smallest
ideal containing $a$ and $b$. (To see that there \emph{is} a smallest
ideal containing $a$ and $b$, consider the intersection of all ideals containing
$a$ and $b$ and notice that, by an earlier exercise, that is itself an ideal.)
\end{definition}

\begin{homework}
Show that $(a,b) = (a) + (b)$.
\end{homework}

\begin{in_class_example}
\quad
\begin{itemize}
\item $(3) + (5) = (3,5) = (1) = \Z$.

\item $(4) + (6) = (4,6) = (2)$ = the set of all even numbers.

\item $(12) + (18) = (12,18) = (6)$ = the set of all multiples of 6.
\end{itemize}
\end{in_class_example}

\begin{theorem}[$\Z$ is a PID]
\label{ZIsAPID}
Let $I$ be an ideal of $\Z$. Then there exists a unique non-negative integer
 $d$
such that $I=(d)$.
Therefore every ideal of $\Z$ is principle. We say that $\Z$ is a principal ideal domain,
or PID.
\end{theorem}
\begin{proof}
Let $I$ be an ideal of $\Z$. If $I=\singleton{0}$ then $I=(0)$ and so $I$ is
principal.

Otherwise
there is a non-zero element of $I$ and by multiplying by $-1$ if necessary
we have that there is a positive element of $I$. Let $d$ be the least positive
element of $I$. We claim that $I=(d)$. Since $d\in I$, $(d)\subseteq I$.
So we need to show that $I\subseteq (d)$. Let $a\in I$. To show that $a\in (d)$
we need to show that $d\divides a$. Let $q$ and $r$ be such that
$a=qd+r$ with $0\leq r < d$. Since $a\in I$ and $d\in I$, we have that
$qd\in I$ and $r=a-qd\in I$. But $0\leq r<d$ and $d$ is the least positive element
of $I$, so $r=0$. This means that $d \divides a$ so $a\in (d)$.

We have shown that $I=(d)$. Now we need to show that $d$ is the unique
non-negative integer with this property. Suppose
that $(e) = (d)$ and $e$ is also non-negative. Since
$e\in (d)$, $d\divides e$. Since $d\in (e)$, $e\divides d$.
Thus $e = \pm d$. But $d$ and $e$ are both non-negative. So
$e=d$.
\end{proof}

\begin{definition}[GCD]
Let $a$ and $b$ be integers. Then the \emph{greatest common divisor} of
$a$ and $b$, written $\gcd(a,b)$, is the unique non-negative integer $d$
such that $(d) = (a,b)$.
\end{definition}

\begin{in_class_example}
\quad
\begin{itemize}
\item Above we saw that $(3,5) = (1)$, so $\gcd(3,5) = 1$.

\item Above bove we saw that $(4,6) = (2)$, so $\gcd(4,6) = 2$.
\end{itemize}
\end{in_class_example}

\begin{definition}
Let $a$ and $b$ be integers. Then $d$ is a \emph{common divisor} of $a$ and $b$
iff $d\divides a$ and $d\divides b$.
\end{definition}

\begin{lemma}
$\gcd(a,b)$ is the unique non-negative
integer $d$ such that
\begin{itemize}
\item $d$ is a common divisor of $a$ and $b$ and
\item if $c$ is any common divisor of $a$ and $b$ then $c\divides d$.
\end{itemize}
\end{lemma}
\begin{proof}
To say that $e$ is a common divisor of $a$ and $b$ is the same as to say that
$a\in (e)$ and $b\in (e)$ which is the same as to say that

\begin{equation}
\label{gcd-equation-1}
(a,b) \subseteq (e).
\end{equation}

To say that every common divisor of $a$ and $b$ also divides $e$ is the same
as to say that $(a,b) \subseteq (c) \implies e \in (c)$ which is to say

\begin{equation}
\label{gcd-equation-2}
(a,b) \subseteq (c) \implies (e) \subseteq (c).
\end{equation}


Both of these are true true for $e=\gcd(a,b)$ because for that $e$ we have
$(a,b) = (e)$.

To show uniqueness, suppose $d=\gcd(a,b)$ and $e$ is any integer satisfying
equations (\ref{gcd-equation-1}) and (\ref{gcd-equation-2}).

Then $(d) = (a,b) \subseteq (e)$ by (\ref{gcd-equation-1}) and,
since $(a,b) \subseteq (d)$,
$(e) \subseteq (d)$ by (\ref{gcd-equation-2}).
So $(d) = (e)$ and since $d$ and $e$ are both non-negative, by
Theorem \ref{ZIsAPID}, $d=e$.
\end{proof}

\begin{in_class_example}
\quad
\begin{itemize}
\item $\gcd(3,5)=1$. Suppose $a\divides 3$ and $a\divides 5$. Then $a\divides 1$.

\item $\gcd(4,6)=2$. Suppose $a\divides 4$ and $a\divides 6$. Then $a\divides 2$.

\item $\gcd(12,18)=6$. Suppose $a\divides 12$ and $a\divides 18$. Then $a\divides 6$.
\end{itemize}
\end{in_class_example}

\begin{lemma}
$\gcd(a,b)$ is the greatest common divisor of $a$ and $b$.
\end{lemma}
\begin{proof}
Let $d=\gcd(a,b)$. We have already seen that $d$ is a common divisor of
$a$ and $b$. Suppose $c$ is any other common divisor. We need to see that
$c\leq d$. By the previous
lemma, $c\divides{d}$. Thus $c\leq\av{c}\leq\av{d}=d$.
\end{proof}

The main advantage of our abstract approach to $\gcd$s via ideals is that
now we know the following:

\begin{theorem}
Let $a$, $b$ be integers. Then $\gcd(a,b)$ can be expressed as an integer
linear combination of $a$ and $b$.
\end{theorem}
\begin{proof}
Let $d=\gcd(a,b)$. The theorem follows immediately from the fact that
$(d) = (a,b)$.
\end{proof}

\begin{in_class_example}
Express $\gcd(6,10)$ as a linear combination of $6$ and $10$.
\end{in_class_example}

\begin{homework}
In each case below, express $gcd(a,b)$ as a linear combination of $a$ and $b$.
\begin{enumerate}
\item $a = 8, b= 12$
\item $a = 20, b= 12$
\item $a=10, b=7$
\item $a=20, b=14$
\end{enumerate}
\end{homework}

\begin{definition}
Two integers $a$ and $b$ are called \emph{relatively prime} if
$\gcd(a,b) = 1$.
\end{definition}

\begin{lemma}
Let $a$, $b$ be integers. Then $a$, $b$ are relatively prime iff
their only positive common divisor is $1$.
\end{lemma}
\begin{proof} $a$ and $b$ are relatively prime iff
$\gcd(a,b)=1$ iff all common divisors of $a$ and $b$ divide 1
iff the only positive common divisor is 1.
\end{proof}

\begin{in_class_example}
$\gcd(6,25) = 1$ so $6$ and $25$ are relatively prime.
\end{in_class_example}

\begin{theorem}
Let $a$, $b$ be integers. Then $a$ and $b$ are relatively prime iff
there is an integer linear combination of $a$ and $b$ that equals 1.
\end{theorem}
\begin{proof}
We already saw that $\gcd(a,b)$ can be expressed as a linear combination
of $a$ and $b$ so if $1=\gcd(a,b)$ then 1 can be expressed as a linear combination
of $a$ and $b$.

Conversely, suppose 1 can be expressed as a linear combination of $a$ and $b$.
This means that $1\in (a,b)$ so $(a,b) = \Z = (1)$ so $1=\gcd(a,b)$.
\end{proof}

\begin{in_class_example}
$\gcd(6,25) = 1$. Find an integer linear combination of $6$ and $25$
that equals 1.
\end{in_class_example}

This last idea is the key ingredient we will need in the next section to
prove the Fundamental Theorem of Arithmetic.

\begin{in_class_example}
True or false? If $c \divides ab$ then $c\divides a$ or $c \divides b$.
Consider this question with two examples:
\begin{itemize}
\item $a = 9,  b = 50, c = 10$
\item $a = 9,  b = 50, c = 15$
\end{itemize}
\end{in_class_example}

\begin{theorem}
\label{PrePrimeProperty}
Let $a,b,c \in \Z$ such that $c\divides ab$ and $\gcd(c,a) = 1$.
Then $c \divides b$.
\end{theorem}
\begin{proof}
Since $\gcd(c,a) = 1$ there are integers $x$ and $y$ such that
$xc + ya = 1$. Multiply both sides of this equation by $b$ to
get $xcb + yab = b$.  Now $c\divides xcb$ and by hypothesis
$c\divides ab$. So $c\divides b$.
\end{proof}

\bigskip

\textbf{Textbook Reading} Read Section 1.2
\emph{Ideals and greatest common divisors}, pages 5-8 of \emph{Shoup}.

\bigskip

\textbf{Textbook Exercises:} Do the following exercises on page 9 of
\emph{Shoup}: 1.8, 1.9, 1.10, 1.11

\newpage

\section{Primes}

A \emph{prime number} is an integer $p>1$ such that the only positive divisors
of $p$ are $1$ and $p$. The first few prime numbers are $2,3,5,7,11,13,17,19,23$.

If $n>1$ is not prime it is called \emph{composite}.

Notice that $1$ is not considered either prime or composite.

Also we will only use the terms prime and composite for positive numbers.

Notice that $n$ is composite iff  $n=ab$ for some integers $a$ and $b$ with
$1<a,b<n$.

Let $p$ be prime and let $a$ be any integer. What are the possibilities
for $\gcd(p,a)$?

\begin{homework}
Let $p$ be prime and let $a$ be any integer. Then either
\begin{itemize}
  \item $p\divides a$ and $\gcd(p,a) = p$, or
  \item $p\ndivides a$ and $\gcd(p,a) = 1$.
\end{itemize}
\end{homework}

\begin{theorem} Let $p$ be prime and let $a,b\in\Z$.
\label{PrimeProperty}
If $p\divides ab$ then $p\divides a$ or $p\divides b$.
\end{theorem}
\begin{proof}
Suppose $p\divides ab$ and $p\ndivides a$. Then $\gcd(p,a) = 1$.
By Theorem \ref{PrePrimeProperty} $p\divides b$.
\end{proof}

\begin{definition}
\label{PrimePropertyDef}
Let $x>1$ be an integer. We say that $x$ has the \emph{prime property}
iff for all $a,b\in\Z$, if $x\divides ab$ then $x\divides a$ or $x\divides b$.
\end{definition}

The previous theorem says that primes have the prime property.

\begin{theorem}
\label{PrimesAreIrreducible}
Let $x>1$ be an integer with the prime property. Then $x$ is prime.
\end{theorem}
\begin{proof}
Suppose $x$ is not prime. Then $x=ab$ with $1<a,b<x$. But then $x\divides ab$
and $x\ndivides a$ and $x\ndivides b$ so $x$ does not have the prime property.
\end{proof}

So for positive integers, having the prime property is equivalent to being
prime. In fact in an arbitrary commutative ring, the prime property is
used as the definition of prime and our definition of prime is what in an
arbitrary integral domain is called \emph{irreducible}.

More precisely, in a commutative ring $R$, an element $x\in R$ is called
\emph{prime} iff $x\not=0$ and $x$ is not a unit and for all $a,b\in R$, if $x\divides ab$ then
$x\divides a$ or $x\divides b$.

In an integral domain $R$, an element $x\in R$ is called \emph{irreducible}
iff $x\not=0$ and $x$ is not a unit and whenever $x = ab$ then either $a$ or $b$ is a unit.

In an arbitrary integral domain, all primes are irreducible. A proof similar to
our proof of Theorem \ref{PrimesAreIrreducible} works. But it is not necessarily
true that every prime is irreducible. This is because
for the proof of Theorem \ref{PrimeProperty} we needed to use GCDs and not
every integral domain has GCDs.

\begin{corollary}
\label{GeneralizedPrimeProperty}
Let $p$ be prime and let $a_1,\cdots,a_k\in\Z$. If $p$ divides the product
$a_1\cdots a_k$ then $p$ divides one of the $a_i$.
\end{corollary}
\begin{proof}
By induction on $k$. For $k=1$ there is nothing to prove and $k=2$ is
Theorem \ref{PrimeProperty}. Suppose the corollary is true for $k$ and
we will show it is true for $k+1$. Suppose $p$ divides $a_1\cdots a_k a_{k+1}$.
Let $b=a_1\cdots a_k$. Then $p\divides b a_{k+1}$. By the case $k=2$,
$p\divides b$ or $p\divides a_{k+1}$. If $p\divides a_{k+1}$ we are done,
so suppose $p\divides b$. Then $p$ divides the product $a_1\cdots a_k$, so
by induction, $p$ divides one of the $a_i$.
\end{proof}

\begin{theorem}[Fundamental theorem of arithmetic]
Every non-zero integer $n$ can be expressed as
$$n=\pm p_1^{e_1} \cdots p_r^{e_r}$$
where $p_1,\cdots,p_r$ are distinct primes and $e_1,\cdots,e_r$ are positive integers.
Moreover this expression is unique up to a reordering of the primes.
\end{theorem}

If $n=\pm 1$ then $r=0$ and the empty product is interpreted as 1.

\begin{proof}
 First, it suffices to prove the theorem for $n > 0$ because then
if $n < 0$ then the unique prime factorization of $n$ is just $-1$ times the
unique prime factorization of $-n$. So assume that $n > 0$.
First we will prove existence and then we will prove uniqueness.

We prove existence by induction on $n$. For $n = 1$, we have that 1
has a prime factorization given by the empty product.

Now assume $n > 1$ and that for all positive $m < n$, $m$ has a prime
factorization. If $n$ is prime then $n$ is already a prime factorization.
So suppose $n$ is composite. Then there exists $a, b$ with $1 < a, b < n$
such that $n = ab$. By induction, $a$ and $b$ have a prime factorization, so
$n = ab$ does.

That completes the proof of existence of prime factorizations. Now
we turn to uniqueness.


Suppose $p_1 p_2 \cdots p_r = q_1 q_2 \cdots q_r$ where each $p_i$ and each $q_j$ are 
prime.
But we are no longer assuming that the $p_i$ are distinct. That is, $p_1$
might equal $p_2$. That is what allows us to not write the exponents $e_i$
from the statement of the theorem.

We want to prove that the $p_i$ are a rearrangement of the $q_j$.
In other
words, we want to prove that $r = s$ and that the $p_i$ and the $q_j$ are the
same, except possibly in a different order.


We prove this fact by induction on $r$. If $r = 0$ then the left-hand-side
equals 1 so the right hand side equals 1 so $s = 0$ and we are done. So
now suppose that $r > 0$. So then we must also have $s > 0$.

Assume the uniqueness claim is true for $r - 1$ and we will show it
is true for $r$. Since $p_1$ divides the left-hand-side, $p_1$ divides the 
righthand-side. In other words

$$p_1 \divides q_1 q_2 \cdots q_s.$$

Because $p_1$ is prime, by Corollary \ref{GeneralizedPrimeProperty},
$p_1$ divides one of the $q_j$.
But $q_j$
is prime so its only positive divisors are $q_j$ and 1, so we must have
that $p_1=q_j$.
This allows us to cancel $p_1$ from the left-hand-side and $q_j$
from the right-hand-side and get 
$p_1\cdots p_r = q_1 \cdots q_{j-i} \cdot q_{j+1} \cdots q_s$. By
induction, $r - 1 = s - 1$ and the remaining $p_i$ are a rearrangement of
the remaining $q_k$. Therefore the original $p_i$ are a rearrangement of the
original $q_j$.
\end{proof}


\textbf{Textbook Exercises:} Do exercise 1.2 on page 4 of Shoup. Do
exercises 1.12 and 1.15 on page 9 of Shoup.

\newpage

\section{Consequences of Unique Factorization}

\begin{theorem}
There are infinitely many primes.
\end{theorem}
\begin{proof}
Suppose towards a contradiction that there were finitely many primes,
$p_1,p_2,\cdots,p_n$. Let $a$ be the product of all of these primes:
$$a=p_1 p_2 \cdots p_n$$

Consider the number $a+1$. Since $a+1$ is greater than each of the $p_i$, it is not
one of the $p_i$, so it is not prime, so it is composite. Since it is composite
it is divisible by a prime and so there is some $i$ such that $p_i \divides (a+1)$.

But also $p_i \divides a$ so then $p_i \divides 1$ which is impossible.

This contradiction implies that there were not finitely many primes.
\end{proof}

\begin{definition}
Let $p$ be a prime and $n$ a non-zero integer. Then $\nu_p(n)$ is the greatest
non-negative integer $\nu$ such that $p^{\nu}\divides n$. 

In other words,
Let $n = p_1^{e_1} \cdots p_r^{e_r}$ be the unique prime factorization of $n$ with each
fo the $p_i$ distinct. If $p$ is not one of the $p_i$ then $p\ndivides n$ and
$\nu_p(n) = 0$. Otherwise, $p\divides n$ and $p=p_i$ for some $i$. In this case
$\nu_p(n) = e_i$.
\end{definition}

With this definition, each non-zero integer $n$ can be represented as

$$n = \pm \prod_p p^{\nu_p(n)}$$

where the product is over all primes $p$.For all but finitely many primes $p$,
$\nu_p(n)=0$.

\begin{in_class_example}
Examples of this expression for some $n$.
\end{in_class_example}

\begin{numbered_fact}
$\nu_p(a\cdot b) = \nu_p(a) + \nu_p(b)$
\end{numbered_fact}

\begin{in_class_example}
Examples of this fact.
\end{in_class_example}

\begin{numbered_fact}
$a\divides b$ iff $\nu_p(a) \leq \nu_p(b)$ for all primes $p$.
\end{numbered_fact}

\begin{in_class_example}
Examples of this fact.
\end{in_class_example}

\begin{numbered_fact}
$$\gcd(a,b)=\prod_p p^{\min(\nu_p(a), \nu_p(b))}$$
\end{numbered_fact}

\begin{in_class_example}
Examples of this fact.
\end{in_class_example}

\begin{homework} Find
\begin{enumerate}
\item[(a)] $\nu_5(100)$
\item[(b)] $\nu_5(-100)$
\item[(c)] the smallest positive integers $a,b,c$ such that $\nu_2(a)=b$ and 
$\nu_3(b)=c$ and $\nu_2(c) > \nu_3(c)$.
\end{enumerate}
\end{homework}

\begin{definition}
Let $a,b\in\Z$. Then a \emph{common multiple} of $a$ and $b$ is an integer $m$
that is a multiple of both $a$ and $b$. 
\end{definition}

Notice that 0 is a common multiple of every pair $a,b$.

Let $a,b$ be two non-zero integers. Then they have at least one positive common multiple,
namely $\av{a} \av{b}$.

\begin{definition}
Let $a,b\in \Z$. Then the \textbf{least common multiple} of $a$ and $b$,
$\lcm(a,b)$ is the least non-negative common multiple of $a$ and $b$.
\end{definition}

\begin{in_class_example}
Examples of $\lcm$s.
\end{in_class_example}

\begin{numbered_fact}
$$\lcm(a,b) =\prod_p p^{\max(\nu_p(a), \nu_p(b))}$$
\end{numbered_fact}

\begin{numbered_fact}
$\lcm(a,b)$ divides every common multiple of $a$ and $b$.
\end{numbered_fact}

\begin{in_class_example}
Examples of these facts.
\end{in_class_example}

\begin{fact}
$\lcm(a,b)$ is the unique non-negative integer $m$ such that $(m) = (a) \intersect (b)$.
\end{fact}
You will prove this fact for homework.

\begin{fact}
$\gcd(a,b) \cdot \lcm(a,b) = \av{a,b}$.
\end{fact}
You will prove this fact for homework.

\begin{fact}
Let $a$ and $b$ be positive integers. Then
$\lcm(a,b) = ab$ iff $a$ and $b$ are relatively prime.
\end{fact}
You will prove this fact for homework.

\textbf{Textbook Reading} Read Section 1.3 of Shoup.

\textbf{Textbook Exercises:} Do exercise 1.20, 1.21, 1.27, 1.28, 1.29, 1.34
of \emph{Shoup}.



\bibliographystyle{amsalpha}

\bibliography{math}

\end{document}

