% $Header$

%\documentclass{beamer}
\documentclass[handout]{beamer}
\usepackage{amsmath,amssymb,latexsym,eucal,amsthm,graphicx,hyperref,changepage}
%%%%%%%%%%%%%%%%%%%%%%%%%%%%%%%%%%%%%%%%%%%%%
% Common Set Theory Constructs
%%%%%%%%%%%%%%%%%%%%%%%%%%%%%%%%%%%%%%%%%%%%%

\newcommand{\setof}[2]{\left\{ \, #1 \, \left| \, #2 \, \right.\right\}}
\newcommand{\lsetof}[2]{\left\{\left. \, #1 \, \right| \, #2 \,  \right\}}
\newcommand{\bigsetof}[2]{\bigl\{ \, #1 \, \bigm | \, #2 \,\bigr\}}
\newcommand{\Bigsetof}[2]{\Bigl\{ \, #1 \, \Bigm | \, #2 \,\Bigr\}}
\newcommand{\biggsetof}[2]{\biggl\{ \, #1 \, \biggm | \, #2 \,\biggr\}}
\newcommand{\Biggsetof}[2]{\Biggl\{ \, #1 \, \Biggm | \, #2 \,\Biggr\}}
\newcommand{\dotsetof}[2]{\left\{ \, #1 \, : \, #2 \, \right\}}
\newcommand{\bigdotsetof}[2]{\bigl\{ \, #1 \, : \, #2 \,\bigr\}}
\newcommand{\Bigdotsetof}[2]{\Bigl\{ \, #1 \, \Bigm : \, #2 \,\Bigr\}}
\newcommand{\biggdotsetof}[2]{\biggl\{ \, #1 \, \biggm : \, #2 \,\biggr\}}
\newcommand{\Biggdotsetof}[2]{\Biggl\{ \, #1 \, \Biggm : \, #2 \,\Biggr\}}
\newcommand{\sequence}[2]{\left\langle \, #1 \,\left| \, #2 \, \right. \right\rangle}
\newcommand{\lsequence}[2]{\left\langle\left. \, #1 \, \right| \,#2 \,  \right\rangle}
\newcommand{\bigsequence}[2]{\bigl\langle \,#1 \, \bigm | \, #2 \, \bigr\rangle}
\newcommand{\Bigsequence}[2]{\Bigl\langle \,#1 \, \Bigm | \, #2 \, \Bigr\rangle}
\newcommand{\biggsequence}[2]{\biggl\langle \,#1 \, \biggm | \, #2 \, \biggr\rangle}
\newcommand{\Biggsequence}[2]{\Biggl\langle \,#1 \, \Biggm | \, #2 \, \Biggr\rangle}
\newcommand{\singleton}[1]{\left\{#1\right\}}
\newcommand{\angles}[1]{\left\langle #1 \right\rangle}
\newcommand{\bigangles}[1]{\bigl\langle #1 \bigr\rangle}
\newcommand{\Bigangles}[1]{\Bigl\langle #1 \Bigr\rangle}
\newcommand{\biggangles}[1]{\biggl\langle #1 \biggr\rangle}
\newcommand{\Biggangles}[1]{\Biggl\langle #1 \Biggr\rangle}


\newcommand{\force}[1]{\Vert\!\underset{\!\!\!\!\!#1}{\!\!\!\relbar\!\!\!%
\relbar\!\!\relbar\!\!\relbar\!\!\!\relbar\!\!\relbar\!\!\relbar\!\!\!%
\relbar\!\!\relbar\!\!\relbar}}
\newcommand{\nforce}[1]{\Vert\!\underset{\!\!\!\!\!#1}{\!\!\!\relbar\!\!\!%
\relbar\!\!\relbar\!\!\relbar\!\!\!\relbar\!\!\relbar\!\!\relbar\!\!\!%
\relbar\!\!\not\relbar\!\!\relbar}}
\newcommand{\forcein}[2]{\overset{#2}{\Vert\underset{\!\!\!\!\!#1}%
{\!\!\!\relbar\!\!\!\relbar\!\!\relbar\!\!\relbar\!\!\!\relbar\!\!\relbar\!%
\!\relbar\!\!\!\relbar\!\!\relbar\!\!\relbar\!\!\relbar\!\!\!\relbar\!\!%
\relbar\!\!\relbar}}}

\newcommand{\pre}[2]{{}^{#2}\!{#1}}

\newcommand{\restr}{\!\!\upharpoonright\!}

%%%%%%%%%%%%%%%%%%%%%%%%%%%%%%%%%%%%%%%%%%%%%
% Set-Theoretic Connectives
%%%%%%%%%%%%%%%%%%%%%%%%%%%%%%%%%%%%%%%%%%%%%

\newcommand{\intersect}{\cap}
\newcommand{\union}{\cup}
\newcommand{\Intersection}[1]{\bigcap\limits_{#1}}
\newcommand{\Union}[1]{\bigcup\limits_{#1}}
\newcommand{\adjoin}{{}^\frown}
\newcommand{\forces}{\Vdash}

%%%%%%%%%%%%%%%%%%%%%%%%%%%%%%%%%%%%%%%%%%%%%
% Miscellaneous
%%%%%%%%%%%%%%%%%%%%%%%%%%%%%%%%%%%%%%%%%%%%%
\newcommand{\defeq}{=_{\text{def}}}
\newcommand{\Turingleq}{\leq_{\text{T}}}
\newcommand{\Turingless}{<_{\text{T}}}
\newcommand{\lexleq}{\leq_{\text{lex}}}
\newcommand{\lexless}{<_{\text{lex}}}
\newcommand{\Turingequiv}{\equiv_{\text{T}}}

%%%%%%%%%%%%%%%%%%%%%%%%%%%%%%%%%%%%%%%%%%%%%
% Constants
%%%%%%%%%%%%%%%%%%%%%%%%%%%%%%%%%%%%%%%%%%%%%
\newcommand{\R}{\mathbb{R}}
\renewcommand{\P}{\mathbb{P}}
\newcommand{\Q}{\mathbb{Q}}
\newcommand{\Z}{\mathbb{Z}}
\newcommand{\C}{\mathbb{C}}
\newcommand{\N}{\mathbb{N}}
\newcommand{\B}{\mathbb{B}}
\newcommand{\LofR}{L(\R)}
\newcommand{\JofR}[1]{J_{#1}(\R)}
\newcommand{\SofR}[1]{S_{#1}(\R)}
\newcommand{\JalphaR}{\JofR{\alpha}}
\newcommand{\JbetaR}{\JofR{\beta}}
\newcommand{\JlambdaR}{\JofR{\lambda}}
\newcommand{\SalphaR}{\SofR{\alpha}}
\newcommand{\SbetaR}{\SofR{\beta}}
\newcommand{\Pkl}{\mathcal{P}_{\kappa}(\lambda)}
\DeclareMathOperator{\con}{con}
\DeclareMathOperator{\ORD}{OR}
\DeclareMathOperator{\Ord}{OR}
\DeclareMathOperator{\WO}{WO}
\DeclareMathOperator{\OD}{OD}
\DeclareMathOperator{\HOD}{HOD}
\DeclareMathOperator{\HC}{HC}
\DeclareMathOperator{\WF}{WF}
\DeclareMathOperator{\HF}{HF}
\newcommand{\One}{I}
\newcommand{\ONE}{I}
\newcommand{\Two}{II}
\newcommand{\TWO}{II}

%%%%%%%%%%%%%%%%%%%%%%%%%%%%%%%%%%%%%%%%%%%%%
% Commutative Algebra Constants
%%%%%%%%%%%%%%%%%%%%%%%%%%%%%%%%%%%%%%%%%%%%%
\DeclareMathOperator{\dottimes}{\dot{\times}}

%%%%%%%%%%%%%%%%%%%%%%%%%%%%%%%%%%%%%%%%%%%%%
% Theories
%%%%%%%%%%%%%%%%%%%%%%%%%%%%%%%%%%%%%%%%%%%%%
\DeclareMathOperator{\ZFC}{ZFC}
\DeclareMathOperator{\ZF}{ZF}
\DeclareMathOperator{\AD}{AD}
\DeclareMathOperator{\ADR}{AD_{\R}}
\DeclareMathOperator{\KP}{KP}
\DeclareMathOperator{\PD}{PD}
\DeclareMathOperator{\CH}{CH}
\DeclareMathOperator{\HPC}{HPC} % HOD pair capturing
%%%%%%%%%%%%%%%%%%%%%%%%%%%%%%%%%%%%%%%%%%%%%
% Iteration Trees
%%%%%%%%%%%%%%%%%%%%%%%%%%%%%%%%%%%%%%%%%%%%%

\newcommand{\pred}{\text{-pred}}

%%%%%%%%%%%%%%%%%%%%%%%%%%%%%%%%%%%%%%%%%%%%%%%%
% Operator Names
%%%%%%%%%%%%%%%%%%%%%%%%%%%%%%%%%%%%%%%%%%%%%%%%
\DeclareMathOperator{\Det}{Det}
\DeclareMathOperator{\dom}{dom}
\DeclareMathOperator{\ran}{ran}
\DeclareMathOperator{\range}{ran}
\DeclareMathOperator{\image}{image}
\DeclareMathOperator{\crit}{crit}
\DeclareMathOperator{\card}{card}
\DeclareMathOperator{\cf}{cf}
\DeclareMathOperator{\cof}{cof}
\DeclareMathOperator{\rank}{rank}
\DeclareMathOperator{\ot}{o.t.}
\DeclareMathOperator{\ords}{o}
\DeclareMathOperator{\ro}{r.o.}
\DeclareMathOperator{\rud}{rud}
\DeclareMathOperator{\Powerset}{\mathcal{P}}
\DeclareMathOperator{\length}{lh}
\DeclareMathOperator{\lh}{lh}
\DeclareMathOperator{\limit}{lim}
\DeclareMathOperator{\fld}{fld}
\DeclareMathOperator{\projection}{p}
\DeclareMathOperator{\Ult}{Ult}
\DeclareMathOperator{\ULT}{Ult}
\DeclareMathOperator{\Coll}{Coll}
\DeclareMathOperator{\Col}{Col}
\DeclareMathOperator{\Hull}{Hull}
\DeclareMathOperator*{\dirlim}{dir lim}
\DeclareMathOperator{\Scale}{Scale}
\DeclareMathOperator{\supp}{supp}
\DeclareMathOperator{\trancl}{tran.cl.}
\DeclareMathOperator{\trace}{Tr}
\DeclareMathOperator{\diag}{diag}
\DeclareMathOperator{\spn}{span}
\DeclareMathOperator{\sgn}{sgn}
%%%%%%%%%%%%%%%%%%%%%%%%%%%%%%%%%%%%%%%%%%%%%
% Logical Connectives
%%%%%%%%%%%%%%%%%%%%%%%%%%%%%%%%%%%%%%%%%%%%%
\newcommand{\IImplies}{\Longrightarrow}
\newcommand{\SkipImplies}{\quad\Longrightarrow\quad}
\newcommand{\Ifff}{\Longleftrightarrow}
\newcommand{\iimplies}{\longrightarrow}
\newcommand{\ifff}{\longleftrightarrow}
\newcommand{\Implies}{\Rightarrow}
\newcommand{\Iff}{\Leftrightarrow}
\renewcommand{\implies}{\rightarrow}
\renewcommand{\iff}{\leftrightarrow}
\newcommand{\AND}{\wedge}
\newcommand{\OR}{\vee}
\newcommand{\st}{\text{ s.t. }}
\newcommand{\Or}{\text{ or }}

%%%%%%%%%%%%%%%%%%%%%%%%%%%%%%%%%%%%%%%%%%%%%
% Function Arrows
%%%%%%%%%%%%%%%%%%%%%%%%%%%%%%%%%%%%%%%%%%%%%

\newcommand{\injection}{\xrightarrow{\text{1-1}}}
\newcommand{\surjection}{\xrightarrow{\text{onto}}}
\newcommand{\bijection}{\xrightarrow[\text{onto}]{\text{1-1}}}
\newcommand{\cofmap}{\xrightarrow{\text{cof}}}
\newcommand{\map}{\rightarrow}

%%%%%%%%%%%%%%%%%%%%%%%%%%%%%%%%%%%%%%%%%%%%%
% Mouse Comparison Operators
%%%%%%%%%%%%%%%%%%%%%%%%%%%%%%%%%%%%%%%%%%%%%
\newcommand{\initseg}{\trianglelefteq}
\newcommand{\properseg}{\lhd}
\newcommand{\notinitseg}{\ntrianglelefteq}
\newcommand{\notproperseg}{\ntriangleleft}

%%%%%%%%%%%%%%%%%%%%%%%%%%%%%%%%%%%%%%%%%%%%%
%           calligraphic letters
%%%%%%%%%%%%%%%%%%%%%%%%%%%%%%%%%%%%%%%%%%%%%
\newcommand{\cA}{\mathcal{A}}
\newcommand{\cB}{\mathcal{B}}
\newcommand{\cC}{\mathcal{C}}
\newcommand{\cD}{\mathcal{D}}
\newcommand{\cE}{\mathcal{E}}
\newcommand{\cF}{\mathcal{F}}
\newcommand{\cG}{\mathcal{G}}
\newcommand{\cH}{\mathcal{H}}
\newcommand{\cI}{\mathcal{I}}
\newcommand{\cJ}{\mathcal{J}}
\newcommand{\cK}{\mathcal{K}}
\newcommand{\cL}{\mathcal{L}}
\newcommand{\cM}{\mathcal{M}}
\newcommand{\cN}{\mathcal{N}}
\newcommand{\cO}{\mathcal{O}}
\newcommand{\cP}{\mathcal{P}}
\newcommand{\cQ}{\mathcal{Q}}
\newcommand{\cR}{\mathcal{R}}
\newcommand{\cS}{\mathcal{S}}
\newcommand{\cT}{\mathcal{T}}
\newcommand{\cU}{\mathcal{U}}
\newcommand{\cV}{\mathcal{V}}
\newcommand{\cW}{\mathcal{W}}
\newcommand{\cX}{\mathcal{X}}
\newcommand{\cY}{\mathcal{Y}}
\newcommand{\cZ}{\mathcal{Z}}


%%%%%%%%%%%%%%%%%%%%%%%%%%%%%%%%%%%%%%%%%%%%%
%          Primed Letters
%%%%%%%%%%%%%%%%%%%%%%%%%%%%%%%%%%%%%%%%%%%%%
\newcommand{\aprime}{a^{\prime}}
\newcommand{\bprime}{b^{\prime}}
\newcommand{\cprime}{c^{\prime}}
\newcommand{\dprime}{d^{\prime}}
\newcommand{\eprime}{e^{\prime}}
\newcommand{\fprime}{f^{\prime}}
\newcommand{\gprime}{g^{\prime}}
\newcommand{\hprime}{h^{\prime}}
\newcommand{\iprime}{i^{\prime}}
\newcommand{\jprime}{j^{\prime}}
\newcommand{\kprime}{k^{\prime}}
\newcommand{\lprime}{l^{\prime}}
\newcommand{\mprime}{m^{\prime}}
\newcommand{\nprime}{n^{\prime}}
\newcommand{\oprime}{o^{\prime}}
\newcommand{\pprime}{p^{\prime}}
\newcommand{\qprime}{q^{\prime}}
\newcommand{\rprime}{r^{\prime}}
\newcommand{\sprime}{s^{\prime}}
\newcommand{\tprime}{t^{\prime}}
\newcommand{\uprime}{u^{\prime}}
\newcommand{\vprime}{v^{\prime}}
\newcommand{\wprime}{w^{\prime}}
\newcommand{\xprime}{x^{\prime}}
\newcommand{\yprime}{y^{\prime}}
\newcommand{\zprime}{z^{\prime}}
\newcommand{\Aprime}{A^{\prime}}
\newcommand{\Bprime}{B^{\prime}}
\newcommand{\Cprime}{C^{\prime}}
\newcommand{\Dprime}{D^{\prime}}
\newcommand{\Eprime}{E^{\prime}}
\newcommand{\Fprime}{F^{\prime}}
\newcommand{\Gprime}{G^{\prime}}
\newcommand{\Hprime}{H^{\prime}}
\newcommand{\Iprime}{I^{\prime}}
\newcommand{\Jprime}{J^{\prime}}
\newcommand{\Kprime}{K^{\prime}}
\newcommand{\Lprime}{L^{\prime}}
\newcommand{\Mprime}{M^{\prime}}
\newcommand{\Nprime}{N^{\prime}}
\newcommand{\Oprime}{O^{\prime}}
\newcommand{\Pprime}{P^{\prime}}
\newcommand{\Qprime}{Q^{\prime}}
\newcommand{\Rprime}{R^{\prime}}
\newcommand{\Sprime}{S^{\prime}}
\newcommand{\Tprime}{T^{\prime}}
\newcommand{\Uprime}{U^{\prime}}
\newcommand{\Vprime}{V^{\prime}}
\newcommand{\Wprime}{W^{\prime}}
\newcommand{\Xprime}{X^{\prime}}
\newcommand{\Yprime}{Y^{\prime}}
\newcommand{\Zprime}{Z^{\prime}}
\newcommand{\alphaprime}{\alpha^{\prime}}
\newcommand{\betaprime}{\beta^{\prime}}
\newcommand{\gammaprime}{\gamma^{\prime}}
\newcommand{\Gammaprime}{\Gamma^{\prime}}
\newcommand{\deltaprime}{\delta^{\prime}}
\newcommand{\epsilonprime}{\epsilon^{\prime}}
\newcommand{\kappaprime}{\kappa^{\prime}}
\newcommand{\lambdaprime}{\lambda^{\prime}}
\newcommand{\rhoprime}{\rho^{\prime}}
\newcommand{\Sigmaprime}{\Sigma^{\prime}}
\newcommand{\tauprime}{\tau^{\prime}}
\newcommand{\xiprime}{\xi^{\prime}}
\newcommand{\thetaprime}{\theta^{\prime}}
\newcommand{\Omegaprime}{\Omega^{\prime}}
\newcommand{\cMprime}{\cM^{\prime}}
\newcommand{\cNprime}{\cN^{\prime}}
\newcommand{\cPprime}{\cP^{\prime}}
\newcommand{\cQprime}{\cQ^{\prime}}
\newcommand{\cRprime}{\cR^{\prime}}
\newcommand{\cSprime}{\cS^{\prime}}
\newcommand{\cTprime}{\cT^{\prime}}

%%%%%%%%%%%%%%%%%%%%%%%%%%%%%%%%%%%%%%%%%%%%%
%          bar Letters
%%%%%%%%%%%%%%%%%%%%%%%%%%%%%%%%%%%%%%%%%%%%%
\newcommand{\abar}{\bar{a}}
\newcommand{\bbar}{\bar{b}}
\newcommand{\zbar}{\bar{z}}
\newcommand{\phibar}{\bar{\varphi}}
\newcommand{\psibar}{\bar{\psi}}
\newcommand{\thetabar}{\bar{\theta}}
\newcommand{\nubar}{\bar{\nu}}

%%%%%%%%%%%%%%%%%%%%%%%%%%%%%%%%%%%%%%%%%%%%%
%          star Letters
%%%%%%%%%%%%%%%%%%%%%%%%%%%%%%%%%%%%%%%%%%%%%
\newcommand{\phistar}{\phi^{*}}


%%%%%%%%%%%%%%%%%%%%%%%%%%%%%%%%%%%%%%%%%%%%%
%          Formulas
%%%%%%%%%%%%%%%%%%%%%%%%%%%%%%%%%%%%%%%%%%%%%

\newcommand{\formulaphi}{\text{\large $\varphi$}}
\newcommand{\Formulaphi}{\text{\Large $\varphi$}}


%%%%%%%%%%%%%%%%%%%%%%%%%%%%%%%%%%%%%%%%%%%%%
%          Fraktur Letters
%%%%%%%%%%%%%%%%%%%%%%%%%%%%%%%%%%%%%%%%%%%%%

\newcommand{\fa}{\mathfrak{a}}
\newcommand{\fb}{\mathfrak{b}}
\newcommand{\fc}{\mathfrak{c}}
\newcommand{\fk}{\mathfrak{k}}
\newcommand{\fp}{\mathfrak{p}}
\newcommand{\fq}{\mathfrak{q}}
\newcommand{\fr}{\mathfrak{r}}
\newcommand{\fA}{\mathfrak{A}}
\newcommand{\fB}{\mathfrak{B}}
\newcommand{\fC}{\mathfrak{C}}
\newcommand{\fD}{\mathfrak{D}}

%%%%%%%%%%%%%%%%%%%%%%%%%%%%%%%%%%%%%%%%%%%%%
%          Bold Letters
%%%%%%%%%%%%%%%%%%%%%%%%%%%%%%%%%%%%%%%%%%%%%
\newcommand{\ba}{\mathbf{a}}
\newcommand{\bb}{\mathbf{b}}
\newcommand{\bc}{\mathbf{c}}
\newcommand{\bd}{\mathbf{d}}
\newcommand{\be}{\mathbf{e}}
\newcommand{\bu}{\mathbf{u}}
\newcommand{\bv}{\mathbf{v}}
\newcommand{\bw}{\mathbf{w}}
\newcommand{\bx}{\mathbf{x}}
\newcommand{\by}{\mathbf{y}}
\newcommand{\bz}{\mathbf{z}}
\newcommand{\bSigma}{\boldsymbol{\Sigma}}
\newcommand{\bPi}{\boldsymbol{\Pi}}
\newcommand{\bDelta}{\boldsymbol{\Delta}}
\newcommand{\bdelta}{\boldsymbol{\delta}}
\newcommand{\bgamma}{\boldsymbol{\gamma}}
\newcommand{\bGamma}{\boldsymbol{\Gamma}}

%%%%%%%%%%%%%%%%%%%%%%%%%%%%%%%%%%%%%%%%%%%%%
%         Bold numbers
%%%%%%%%%%%%%%%%%%%%%%%%%%%%%%%%%%%%%%%%%%%%%
\newcommand{\bzero}{\mathbf{0}}

%%%%%%%%%%%%%%%%%%%%%%%%%%%%%%%%%%%%%%%%%%%%%
% Projective-Like Pointclasses
%%%%%%%%%%%%%%%%%%%%%%%%%%%%%%%%%%%%%%%%%%%%%
\newcommand{\Sa}[2][\alpha]{\Sigma_{(#1,#2)}}
\newcommand{\Pa}[2][\alpha]{\Pi_{(#1,#2)}}
\newcommand{\Da}[2][\alpha]{\Delta_{(#1,#2)}}
\newcommand{\Aa}[2][\alpha]{A_{(#1,#2)}}
\newcommand{\Ca}[2][\alpha]{C_{(#1,#2)}}
\newcommand{\Qa}[2][\alpha]{Q_{(#1,#2)}}
\newcommand{\da}[2][\alpha]{\delta_{(#1,#2)}}
\newcommand{\leqa}[2][\alpha]{\leq_{(#1,#2)}}
\newcommand{\lessa}[2][\alpha]{<_{(#1,#2)}}
\newcommand{\equiva}[2][\alpha]{\equiv_{(#1,#2)}}


\newcommand{\Sl}[1]{\Sa[\lambda]{#1}}
\newcommand{\Pl}[1]{\Pa[\lambda]{#1}}
\newcommand{\Dl}[1]{\Da[\lambda]{#1}}
\newcommand{\Al}[1]{\Aa[\lambda]{#1}}
\newcommand{\Cl}[1]{\Ca[\lambda]{#1}}
\newcommand{\Ql}[1]{\Qa[\lambda]{#1}}

\newcommand{\San}{\Sa{n}}
\newcommand{\Pan}{\Pa{n}}
\newcommand{\Dan}{\Da{n}}
\newcommand{\Can}{\Ca{n}}
\newcommand{\Qan}{\Qa{n}}
\newcommand{\Aan}{\Aa{n}}
\newcommand{\dan}{\da{n}}
\newcommand{\leqan}{\leqa{n}}
\newcommand{\lessan}{\lessa{n}}
\newcommand{\equivan}{\equiva{n}}

%%%%%%%%%%%%%%%%%%%%%%%%%%%%%%%%%%%%%%%%%%%%%
% Linear Algebra
%%%%%%%%%%%%%%%%%%%%%%%%%%%%%%%%%%%%%%%%%%%%%
\newcommand{\transpose}[1]{{#1}^{\text{T}}}
\newcommand{\norm}[1]{\lVert{#1}\rVert}
\newcommand\aug{\fboxsep=-\fboxrule\!\!\!\fbox{\strut}\!\!\!}

%%%%%%%%%%%%%%%%%%%%%%%%%%%%%%%%%%%%%%%%%%%%%
% Number Theory
%%%%%%%%%%%%%%%%%%%%%%%%%%%%%%%%%%%%%%%%%%%%%
\DeclareMathOperator{\Spec}{Spec}
\newcommand{\av}[1]{\lvert#1\rvert}
\DeclareMathOperator{\divides}{\mid}
\DeclareMathOperator{\ndivides}{\nmid}


\graphicspath{{images/}}

\newtheorem*{claim}{claim}
\newtheorem*{observation}{Observation}
\newtheorem*{warning}{Warning}
\newtheorem*{question}{Question}
\newtheorem{remark}[theorem]{Remark}

\newenvironment*{subproof}[1][Proof]
{\begin{proof}[#1]}{\renewcommand{\qedsymbol}{$\diamondsuit$} \end{proof}}

\mode<presentation>
{
  \usetheme{Singapore}
  % or ...

  \setbeamercovered{invisible}
  % or whatever (possibly just delete it)
}


\usepackage[english]{babel}
% or whatever

\usepackage[latin1]{inputenc}
% or whatever

\usepackage{times}
\usepackage[T1]{fontenc}
% Or whatever. Note that the encoding and the font should match. If T1
% does not look nice, try deleting the line with the fontenc.

\title{Lesson 21 \\ Quadratic Residues}
\subtitle{Math 310, Elementary Number Theory \\ Fall 2021 \\ SFSU}
\author{Mitch Rudominer}
\date{}



% If you wish to uncover everything in a step-wise fashion, uncomment
% the following command:

\beamerdefaultoverlayspecification{<+->}

\begin{document}

\begin{frame}
  \titlepage
\end{frame}

\begin{frame}{Quadratic Residues}

\begin{itemize}
  \item \textbf{Definition.} Let $n>1$ be an integer. Let $x$ be relatively prime to $n$.
  \item Then $x$ is a \emph{quadratic residue} of $n$ iff
  \item there is an integer $y$ such that $y^2 \equiv x \pmod n$.
  \item i.e. $x$ is a perfect square mod $n$.
  \item Otherwise $x$ is a \emph{quadratic nonresidue} of $n$.
  \item We will study quadratic residues mod $p$, for $p$ prime.
  \item \textbf{Example} Work in $\Z^*_5$.
  \item $[1]^2 = [1], [2]^2 = [4], [3]^2 = [4], [4]^2 = [1]$.
  \item The perfect squares in $\Z^*_5$ are $[1], [4]$.
  \item $[2]$ and $[3]$ are not perfect squares.
  \item So 1 and 4 are quadratic residues of 5. So are 6,9,11,14,-1,-4
  \item 2 and 3 are quadratic nonresidues of 5. So are 7,8,12,13,-2,-3
\end{itemize}

\end{frame}

\begin{frame}{Quadratic Residues and Primitive Roots}

\begin{itemize}
  \item \textbf{Theorem.} Let $p$ be an odd prime and $a$ a primitive root of $p$.
  \item Then $a^k$ is a quadratic residue of $p$ iff $k$ is even.
  \item \textbf{proof.} First suppose $a^k$ is a quadratic residue.
  \item Then there is some $s$ such that $a^k \equiv (a^s)^2 = a^{2s} \pmod p$.
  \item So $k\equiv 2s \pmod {p-1}$. So $p-1 \divides (k-2s)$. Since $p-1$ and $2s$ are even, $k$ is even.
  \item Conversely, suppose $k=2s$. Then $a^k = (a^s)^2$. $\qed$.
  \item \textbf{Example.} $2$ is a primitive root of $11$. So $2^2, 2^4, 2^6, 2^8, 2^{10}$ are quadratic
  residues of $11$, and $2, 2^3, 2^5, 2^7, 2^9$ are quadratic nonresidues of 11.
  \item \textbf{Corollary.} Let $p$ be an odd prime and $a$ a primitive root of $p$.
  \item Then $a$ is a quadratic nonresidue of $p$.
\end{itemize}

\end{frame}

\begin{frame}{Half of the elements are quadratic residues}

\begin{itemize}
  \item \textbf{Corollary.} Let $p$ be an odd prime.
  \item Then exactly half of the integers $1,2,\cdots p-1$ are quadratic residues of $p$.
  \item \textbf{proof.} Let $a$ be a primitive root of $p$. The integers $1,2,\cdots p-1$ are congruent mod $p$
  \item to the integers $a^k$, for $k=1,2,\cdots p-1$. Exactly half of these $k$ are even. $\qed$.
  \item \textbf{Example.} Earlier we saw that 1,4 are quadratic residues and 2,3 are quadratic nonresidues of 5.
  \item \textbf{Example.} Work in $\Z^*_7$.
  \item $[1]^2=[1], [2]^2=[4], [3]^2=[2], [4]^2 = [2], [5]^2=[4], [6]^2=[1]$.
  \item $[1],[2],[4]$ are perfect squares. $[3],[5],[6]$ are not.
  \item $1,2,4$ are quadratic residues and $3,5,6$ are quadratic nonresidues of $7$.
\end{itemize}

\end{frame}

\begin{frame}{Legendre Symbol}

\begin{itemize}
  \item \textbf{Definition.} Let $p$ be an odd prime and $a$ an integer not divisible by $p$.
  \item The \emph{Legendrea symbol} $(\frac{a}{p})$ is defined by
  \item $(\frac{a}{p}) = 1$ if $a$ is a quadratic residue of $p$;
  \item $(\frac{a}{p}) = -1$ if $a$ is a quadratic nonresidue of $p$.
  \item \textbf{Examples} $(\frac{1}{7}) = 1$,$(\frac{2}{7}) = 1$,$(\frac{4}{7}) = 1$
  \item \textbf{Examples} $(\frac{3}{7}) = -1$,$(\frac{5}{7}) = -1$,$(\frac{6}{7}) = -1$
\end{itemize}

\end{frame}

\begin{frame}{Euler's Criterion}

\begin{itemize}
  \item \textbf{Theorem.} Let $p$ be an odd prime and $a$ an integer not divisible by $p$. Then
  \item $(\frac{a}{p}) \equiv a^{(p-1)/2} \pmod p$.
  \item Notice that the theorem implies that $a^{(p-1)/2} \bmod p$ is always equal to 1 or -1.
  \item \textbf{Example.} Let $p=11$. Let $1\leq a \leq 10$.
  \item If $a$ is a quadratic residue of 11 then $a^5 \bmod 11 = 1$.
  \item If $a$ is a quadratic nonresidue of 11 then $a^5 \bmod 11 = -1$.
  \item $9$ is a quadratic residue of 11. So $9^5 \bmod 11 = 1$.
  \item $8$ is a quadratic nonresidue of 11. So $8^5 \bmod 11 = -1$.
\end{itemize}

\end{frame}


\begin{frame}{Proof of Euler's Criterion}

\begin{itemize}
  \item \textbf{Proof.} Let $p$ be an odd prime and $[a]\in\Z^*_p$. Work in $\Z^*_p$.
  \item Let $[b]=[a]^{(p-1)/2}$.
  \item Notice that $[b]^2 = [a]^{p-1} = [1]$ by Fermat's Little Theorem.
  \item So $[b]$ is a square-root of $[1]$.
  \item So $[b] = [1]$ or $[b] = -[1]$. We want to see that $[b]=[1]$ iff $[a]$ is a perfect square.
  \item First suppose $[a]$ is a perfect square. Say $[a] = [c]^2$.
  \item Then $[b]= ([c]^2)^{(p-1)/2}=[c]^{p-1}=[1]$ by Fermat's Little Theorem.
  \item Now suppose $[a]$ is not a perfect square.
  \item Notice that for each $[c]\in\Z^*_p$ there is a unique $[\cprime]\in\Z^*_p$ such that $[c][\cprime] = [a]$.
  \item Namely, $[\cprime] = [a][c]^{-1}$.
  \item Also, $[\cprime]\not= [c]$ because $[a]$ is not a perfect square.
\end{itemize}

\end{frame}

\begin{frame}{Proof of Euler's Criterion, page 2}

\begin{itemize}
  \item Now consider the product of all of the elements of $\Z^*_p$.
  \item By Wilson's Theorem we have
  \item $[1][2][3]\cdots[c]\cdots[\cprime]\cdots[p-1] = -[1]$.
  \item But each $[c]$ in the product has a partner $[\cprime]$ in the product
  so that $[c][\cprime] = [a]$.
  \item There are $(p-1)/2$ such pairs of $[c], [\cprime]$.
  \item So the product may be seen to be equal to $[a]^{(p-1)/2}$.
  \item So $[a]^{(p-1)/2}= -[1]$. $\qed$
\end{itemize}

\end{frame}

\begin{frame}{When is -1 a quadratic residue?}

\begin{itemize}
  \item \textbf{Theorem.}  Let $p$ be an odd prime. Then
  \item $(\frac{-1}{p}) = 1$ iff $p\bmod 4 = 1$.
  \item Notice that if $p$ is any odd number then $p \bmod 4$ is 1 or 3.
  \item \textbf{Example.} $11\bmod 4 = 3$. So $-1$ is not a quadratic residue of 11.
  \item \textbf{Example.} $13\bmod 4 = 1$. So $-1$ is a quadratic residue of 13. Find a square-root of -1 mod 13.
  \item $5^2 = 25 \equiv -1 \pmod {13}$.
  \item \textbf{proof.} By Euler's Criterion, $(\frac{-1}{p}) \equiv (-1)^{(p-1)/2} \pmod p$.
  \item So $(\frac{-1}{p}) = 1$ iff $(p-1)/2$ is even iff $p-1$ is divisible by 4. $\qed$.
\end{itemize}

\end{frame}


\begin{frame}{Guass's Lemma}

\begin{itemize}
  \item \textbf{Theorem.}  Let $p$ be an odd prime and $a$ an integer not divisible by $p$.
  \item Consider the least positive residues mod $p$ of $a, 2a, 3a, \cdots \frac{p-1}{2} a$.
  \item Let $s$ be the number of these that are greater than $(p-1)/2$. Then
  \item $(\frac{a}{p}) = (-1)^s$.
  \item \textbf{Example.} Let $p=11$ and $a=3$. Consider the least positive residues mod 11 of:
  $3, 6, 9, 12, 15$.
  \item These are: $1, 3, 4, 6, 9$.
  \item $s=2$ because 6 and 9 are greater than 5.
  \item So  $(\frac{3}{11}) = (-1)^2 = 1$. So 3 is a quadratic residue mod 11.
  \item Find a square-root of 3 mod 11.
  \item $5^2 \equiv 3 \pmod {11}$.
\end{itemize}

\end{frame}

\begin{frame}{Proof of Guass's Lemma}

\begin{itemize}
  \item \textbf{Proof.} Let $p$ be an odd prime and $[a]\in\Z^*_p$. Work in $\Z^*_p$.
  \item Consider the multiples $[a],[2][a],[3][a],\cdots[\frac{p-1}{2}][a]$.
  \item Notice that no two of them are equal because if $[x][a]=[y][a]$ then $[x]=[y]$.
  \item But also notice that if $x\not=y$ then $[x][a] \not= -[y][a]$, because
  $[x][a] = -[y][a] \Implies [x]=-[y]$ but $1\leq x,y\leq \frac{p-1}{2}$.
  \item For $[x]\in \Z^*_p$ let $|[x]| = [x]$ if $1\leq x \leq \frac{p-1}{2}$ and
  let $|[x]| = -[x] = [p-x]$ if $x>\frac{p-1}{2}$.
  \item So for all $[x]\in\Z^*_p$, $|[x]|\in \singleton{[1],[2],\cdots [\frac{p-1}{2}]}$.
  \item $\singleton{|[a]|,|[2a]|,\cdots |[\frac{p-1}{2}a]|}$ contains $(p-1)/2$ distinct values.
  \item So $\singleton{|[a]|,|[2a]|,\cdots |[\frac{p-1}{2}a]|} = \singleton{[1],[2],\cdots [\frac{p-1}{2}]}.$
\end{itemize}

\end{frame}

\begin{frame}{Proof of Gauss's Lemma, page 2}

\begin{itemize}
  \item Let $s$ be the number of $[xa]$ with $1\leq x \leq \frac{p-1}{2}$ such that $|[xa]| = - [xa]$.
  \item Let $u=[1]\times[2]\cdots \times [\frac{p-1}{2}]$.
  \item Then $[a]\times[2][a]\times[3][a]\cdots \times [\frac{p-1}{2}][a] = [a]^{(p-1)/2} u$.
  \item But also $[a]\times[2][a]\times[3][a]\cdots \times [\frac{p-1}{2}][a] = (-1)^s u$.
  \item So $[a]^{(p-1)/2}=[(-1)^s]$.
  \item By Euler's Criterion, $(\frac{a}{p}) = 1$ iff $[a]^{(p-1)/2} = [1]$.
  \item So $(\frac{a}{p}) = 1$ iff $(-1)^s = 1$. $\qed$
\end{itemize}

\end{frame}

\begin{frame}{When is 2 a quadratic residue?}

\begin{itemize}
  \item \textbf{Theorem.}  Let $p$ be an odd prime. Then
  \item $(\frac{2}{p}) = 1$ iff $p\bmod 8 = 1$ or $p\bmod 8 = 7$.
  \item Notice that if $p$ is any odd number then $p \bmod 8$ is 1,3,5 or 7.
  \item \textbf{Example.} $11\bmod 8 = 3$. So $2$ is not a quadratic residue of 11.
  \item \textbf{Example.} $13\bmod 8 = 5$. So $2$ is not a quadratic residue of 13.
  \item \textbf{Example.} $17\bmod 8 = 1$. So $2$ is a quadratic residue of 17. Find a square-root of 2 mod 17.
  \item $6^2 = 36 \equiv 2 \pmod {17}$.
\end{itemize}

\end{frame}

\begin{frame}{Proof of when 2 is a quadratic residue}

\begin{itemize}
  \item \textbf{proof.}  Let $p$ be an odd prime.
  \item Consider the the first $(p-1)/2$ multiples of $2$:
  \item  $2, 4, 6, 8, \cdots p-1$.
  \item Let $s$ be the number of such multiples that are greater than $(p-1)/2$.
  \item By Gauss's Lemma, $2$ is a quadratic residue of $p$ just in $s$ is even.
  \item We claim that $s$ is even just in case $p\bmod 8 = 1$ or $p\bmod 8 = 7$.
  \item Notice the following facts:
\end{itemize}

\end{frame}

\begin{frame}{Proof of when is 2 is a quadratic residue, page 2}

\begin{enumerate}
\item[(i)] Every sequence of 4 consecutive integers contains an even number of even integers.
\item[(ii)] A sequence of 3 consecutive integers contains an even number of even integers iff the first element of the sequence is even.
\item[(iii)] A sequence of 2 consecutive integers contains an odd number of even integers
\item[(iv)] A sequence of 1 integer contains an even number of even integers iff the single integer in the sequence is odd.
\end{enumerate}

\end{frame}

\begin{frame}{Proof of when is 2 is a quadratic residue, page 3}

Now let $I=\setof{x\in\Z}{(p-1)/2 < x < p}$. So $|I|=(p-1)/2$ and the first element of $I$ is $(p-1)/2 + 1$.
We want to know if there are an even or odd number of even numbers in $I$.
Consider four cases:

\begin{itemize}
\item $p\bmod 8 = 1$. Then $(p-1)/2$ is a multiple of 4 and so there are an even number of evens in $I$ by (i) above.
\item $p\bmod 8 = 3$. Then $(p-1)/2 \bmod 4 = 1$ and the first element of $I$ is even so $I$ contains an odd number of evens by (i) and (iv) above.
\item $p\bmod 8 = 5$. Then $(p-1)/2 \bmod 4 = 2$ so $I$ contains an odd number of evens by (i) and (iii) above.
\item $p\bmod 8 = 7$. Then $(p-1)/2 \bmod 4 = 3$ and the first element of $I$ is even so $I$ contains an even number of evens by (i) and (ii) above. $\qed$.
\end{itemize}

\end{frame}









\end{document}
