% $Header$

\documentclass{beamer}
%\documentclass[handout]{beamer}
\usepackage{amsmath,amssymb,latexsym,eucal,amsthm,graphicx,hyperref}
%%%%%%%%%%%%%%%%%%%%%%%%%%%%%%%%%%%%%%%%%%%%%
% Common Set Theory Constructs
%%%%%%%%%%%%%%%%%%%%%%%%%%%%%%%%%%%%%%%%%%%%%

\newcommand{\setof}[2]{\left\{ \, #1 \, \left| \, #2 \, \right.\right\}}
\newcommand{\lsetof}[2]{\left\{\left. \, #1 \, \right| \, #2 \,  \right\}}
\newcommand{\bigsetof}[2]{\bigl\{ \, #1 \, \bigm | \, #2 \,\bigr\}}
\newcommand{\Bigsetof}[2]{\Bigl\{ \, #1 \, \Bigm | \, #2 \,\Bigr\}}
\newcommand{\biggsetof}[2]{\biggl\{ \, #1 \, \biggm | \, #2 \,\biggr\}}
\newcommand{\Biggsetof}[2]{\Biggl\{ \, #1 \, \Biggm | \, #2 \,\Biggr\}}
\newcommand{\dotsetof}[2]{\left\{ \, #1 \, : \, #2 \, \right\}}
\newcommand{\bigdotsetof}[2]{\bigl\{ \, #1 \, : \, #2 \,\bigr\}}
\newcommand{\Bigdotsetof}[2]{\Bigl\{ \, #1 \, \Bigm : \, #2 \,\Bigr\}}
\newcommand{\biggdotsetof}[2]{\biggl\{ \, #1 \, \biggm : \, #2 \,\biggr\}}
\newcommand{\Biggdotsetof}[2]{\Biggl\{ \, #1 \, \Biggm : \, #2 \,\Biggr\}}
\newcommand{\sequence}[2]{\left\langle \, #1 \,\left| \, #2 \, \right. \right\rangle}
\newcommand{\lsequence}[2]{\left\langle\left. \, #1 \, \right| \,#2 \,  \right\rangle}
\newcommand{\bigsequence}[2]{\bigl\langle \,#1 \, \bigm | \, #2 \, \bigr\rangle}
\newcommand{\Bigsequence}[2]{\Bigl\langle \,#1 \, \Bigm | \, #2 \, \Bigr\rangle}
\newcommand{\biggsequence}[2]{\biggl\langle \,#1 \, \biggm | \, #2 \, \biggr\rangle}
\newcommand{\Biggsequence}[2]{\Biggl\langle \,#1 \, \Biggm | \, #2 \, \Biggr\rangle}
\newcommand{\singleton}[1]{\left\{#1\right\}}
\newcommand{\angles}[1]{\left\langle #1 \right\rangle}
\newcommand{\bigangles}[1]{\bigl\langle #1 \bigr\rangle}
\newcommand{\Bigangles}[1]{\Bigl\langle #1 \Bigr\rangle}
\newcommand{\biggangles}[1]{\biggl\langle #1 \biggr\rangle}
\newcommand{\Biggangles}[1]{\Biggl\langle #1 \Biggr\rangle}


\newcommand{\force}[1]{\Vert\!\underset{\!\!\!\!\!#1}{\!\!\!\relbar\!\!\!%
\relbar\!\!\relbar\!\!\relbar\!\!\!\relbar\!\!\relbar\!\!\relbar\!\!\!%
\relbar\!\!\relbar\!\!\relbar}}
\newcommand{\longforce}[1]{\Vert\!\underset{\!\!\!\!\!#1}{\!\!\!\relbar\!\!\!%
\relbar\!\!\relbar\!\!\relbar\!\!\!\relbar\!\!\relbar\!\!\relbar\!\!\!%
\relbar\!\!\relbar\!\!\relbar\!\!\relbar\!\!\relbar\!\!\relbar\!\!\relbar\!\!\relbar}}
\newcommand{\nforce}[1]{\Vert\!\underset{\!\!\!\!\!#1}{\!\!\!\relbar\!\!\!%
\relbar\!\!\relbar\!\!\relbar\!\!\!\relbar\!\!\relbar\!\!\relbar\!\!\!%
\relbar\!\!\not\relbar\!\!\relbar}}
\newcommand{\forcein}[2]{\overset{#2}{\Vert\underset{\!\!\!\!\!#1}%
{\!\!\!\relbar\!\!\!\relbar\!\!\relbar\!\!\relbar\!\!\!\relbar\!\!\relbar\!%
\!\relbar\!\!\!\relbar\!\!\relbar\!\!\relbar\!\!\relbar\!\!\!\relbar\!\!%
\relbar\!\!\relbar}}}

\newcommand{\pre}[2]{{}^{#2}{#1}}

\newcommand{\restr}{\!\!\upharpoonright\!}

%%%%%%%%%%%%%%%%%%%%%%%%%%%%%%%%%%%%%%%%%%%%%
% Set-Theoretic Connectives
%%%%%%%%%%%%%%%%%%%%%%%%%%%%%%%%%%%%%%%%%%%%%

\newcommand{\intersect}{\cap}
\newcommand{\union}{\cup}
\newcommand{\Intersection}[1]{\bigcap\limits_{#1}}
\newcommand{\Union}[1]{\bigcup\limits_{#1}}
\newcommand{\adjoin}{{}^\frown}
\newcommand{\forces}{\Vdash}

%%%%%%%%%%%%%%%%%%%%%%%%%%%%%%%%%%%%%%%%%%%%%
% Miscellaneous
%%%%%%%%%%%%%%%%%%%%%%%%%%%%%%%%%%%%%%%%%%%%%
\newcommand{\defeq}{=_{\text{def}}}
\newcommand{\Turingleq}{\leq_{\text{T}}}
\newcommand{\Turingless}{<_{\text{T}}}
\newcommand{\lexleq}{\leq_{\text{lex}}}
\newcommand{\lexless}{<_{\text{lex}}}
\newcommand{\Turingequiv}{\equiv_{\text{T}}}
\newcommand{\isomorphic}{\cong}

%%%%%%%%%%%%%%%%%%%%%%%%%%%%%%%%%%%%%%%%%%%%%
% Constants
%%%%%%%%%%%%%%%%%%%%%%%%%%%%%%%%%%%%%%%%%%%%%
\newcommand{\R}{\mathbb{R}}
\renewcommand{\P}{\mathbb{P}}
\newcommand{\Q}{\mathbb{Q}}
\newcommand{\Z}{\mathbb{Z}}
\newcommand{\Zpos}{\Z^{+}}
\newcommand{\Znonneg}{\Z^{\geq 0}}
\newcommand{\C}{\mathbb{C}}
\newcommand{\N}{\mathbb{N}}
\newcommand{\B}{\mathbb{B}}
\newcommand{\Bairespace}{\pre{\omega}{\omega}}
\newcommand{\LofR}{L(\R)}
\newcommand{\JofR}[1]{J_{#1}(\R)}
\newcommand{\SofR}[1]{S_{#1}(\R)}
\newcommand{\JalphaR}{\JofR{\alpha}}
\newcommand{\JbetaR}{\JofR{\beta}}
\newcommand{\JlambdaR}{\JofR{\lambda}}
\newcommand{\SalphaR}{\SofR{\alpha}}
\newcommand{\SbetaR}{\SofR{\beta}}
\newcommand{\Pkl}{\mathcal{P}_{\kappa}(\lambda)}
\DeclareMathOperator{\con}{con}
\DeclareMathOperator{\ORD}{OR}
\DeclareMathOperator{\Ord}{OR}
\DeclareMathOperator{\WO}{WO}
\DeclareMathOperator{\OD}{OD}
\DeclareMathOperator{\HOD}{HOD}
\DeclareMathOperator{\HC}{HC}
\DeclareMathOperator{\WF}{WF}
\DeclareMathOperator{\wfp}{wfp}
\DeclareMathOperator{\HF}{HF}
\newcommand{\One}{I}
\newcommand{\ONE}{I}
\newcommand{\Two}{II}
\newcommand{\TWO}{II}
\newcommand{\Mladder}{M^{\text{ld}}}

%%%%%%%%%%%%%%%%%%%%%%%%%%%%%%%%%%%%%%%%%%%%%
% Commutative Algebra Constants
%%%%%%%%%%%%%%%%%%%%%%%%%%%%%%%%%%%%%%%%%%%%%
\DeclareMathOperator{\dottimes}{\dot{\times}}
\DeclareMathOperator{\dotminus}{\dot{-}}
\DeclareMathOperator{\Spec}{Spec}

%%%%%%%%%%%%%%%%%%%%%%%%%%%%%%%%%%%%%%%%%%%%%
% Theories
%%%%%%%%%%%%%%%%%%%%%%%%%%%%%%%%%%%%%%%%%%%%%
\DeclareMathOperator{\ZFC}{ZFC}
\DeclareMathOperator{\ZF}{ZF}
\DeclareMathOperator{\AD}{AD}
\DeclareMathOperator{\ADR}{AD_{\R}}
\DeclareMathOperator{\KP}{KP}
\DeclareMathOperator{\PD}{PD}
\DeclareMathOperator{\CH}{CH}
\DeclareMathOperator{\GCH}{GCH}
\DeclareMathOperator{\HPC}{HPC} % HOD pair capturing
%%%%%%%%%%%%%%%%%%%%%%%%%%%%%%%%%%%%%%%%%%%%%
% Iteration Trees
%%%%%%%%%%%%%%%%%%%%%%%%%%%%%%%%%%%%%%%%%%%%%

\newcommand{\pred}{\text{-pred}}

%%%%%%%%%%%%%%%%%%%%%%%%%%%%%%%%%%%%%%%%%%%%%%%%
% Operator Names
%%%%%%%%%%%%%%%%%%%%%%%%%%%%%%%%%%%%%%%%%%%%%%%%
\DeclareMathOperator{\Det}{Det}
\DeclareMathOperator{\dom}{dom}
\DeclareMathOperator{\ran}{ran}
\DeclareMathOperator{\range}{ran}
\DeclareMathOperator{\image}{image}
\DeclareMathOperator{\crit}{crit}
\DeclareMathOperator{\card}{card}
\DeclareMathOperator{\cf}{cf}
\DeclareMathOperator{\cof}{cof}
\DeclareMathOperator{\rank}{rank}
\DeclareMathOperator{\ot}{o.t.}
\DeclareMathOperator{\ords}{o}
\DeclareMathOperator{\ro}{r.o.}
\DeclareMathOperator{\rud}{rud}
\DeclareMathOperator{\Powerset}{\mathcal{P}}
\DeclareMathOperator{\length}{lh}
\DeclareMathOperator{\lh}{lh}
\DeclareMathOperator{\limit}{lim}
\DeclareMathOperator{\fld}{fld}
\DeclareMathOperator{\projection}{p}
\DeclareMathOperator{\Ult}{Ult}
\DeclareMathOperator{\ULT}{Ult}
\DeclareMathOperator{\Coll}{Coll}
\DeclareMathOperator{\Col}{Col}
\DeclareMathOperator{\Hull}{Hull}
\DeclareMathOperator*{\dirlim}{dir lim}
\DeclareMathOperator{\Scale}{Scale}
\DeclareMathOperator{\supp}{supp}
\DeclareMathOperator{\trancl}{tran.cl.}
\DeclareMathOperator{\trace}{Tr}
\DeclareMathOperator{\diag}{diag}
\DeclareMathOperator{\spn}{span}
\DeclareMathOperator{\sgn}{sgn}
%%%%%%%%%%%%%%%%%%%%%%%%%%%%%%%%%%%%%%%%%%%%%
% Logical Connectives
%%%%%%%%%%%%%%%%%%%%%%%%%%%%%%%%%%%%%%%%%%%%%
\newcommand{\IImplies}{\Longrightarrow}
\newcommand{\SkipImplies}{\quad\Longrightarrow\quad}
\newcommand{\Ifff}{\Longleftrightarrow}
\newcommand{\iimplies}{\longrightarrow}
\newcommand{\ifff}{\longleftrightarrow}
\newcommand{\Implies}{\Rightarrow}
\newcommand{\Iff}{\Leftrightarrow}
\renewcommand{\implies}{\rightarrow}
\renewcommand{\iff}{\leftrightarrow}
\newcommand{\AND}{\wedge}
\newcommand{\OR}{\vee}
\newcommand{\st}{\text{ s.t. }}
\newcommand{\Or}{\text{ or }}

%%%%%%%%%%%%%%%%%%%%%%%%%%%%%%%%%%%%%%%%%%%%%
% Function Arrows
%%%%%%%%%%%%%%%%%%%%%%%%%%%%%%%%%%%%%%%%%%%%%

\newcommand{\injection}{\xrightarrow{\text{1-1}}}
\newcommand{\surjection}{\xrightarrow{\text{onto}}}
\newcommand{\bijection}{\xrightarrow[\text{onto}]{\text{1-1}}}
\newcommand{\cofmap}{\xrightarrow{\text{cof}}}
\newcommand{\map}{\rightarrow}

%%%%%%%%%%%%%%%%%%%%%%%%%%%%%%%%%%%%%%%%%%%%%
% Mouse Comparison Operators
%%%%%%%%%%%%%%%%%%%%%%%%%%%%%%%%%%%%%%%%%%%%%
\newcommand{\initseg}{\trianglelefteq}
\newcommand{\properseg}{\lhd}
\newcommand{\notinitseg}{\ntrianglelefteq}
\newcommand{\notproperseg}{\ntriangleleft}

%%%%%%%%%%%%%%%%%%%%%%%%%%%%%%%%%%%%%%%%%%%%%
%           calligraphic letters
%%%%%%%%%%%%%%%%%%%%%%%%%%%%%%%%%%%%%%%%%%%%%
\newcommand{\cA}{\mathcal{A}}
\newcommand{\cB}{\mathcal{B}}
\newcommand{\cC}{\mathcal{C}}
\newcommand{\cD}{\mathcal{D}}
\newcommand{\cE}{\mathcal{E}}
\newcommand{\cF}{\mathcal{F}}
\newcommand{\cG}{\mathcal{G}}
\newcommand{\cH}{\mathcal{H}}
\newcommand{\cI}{\mathcal{I}}
\newcommand{\cJ}{\mathcal{J}}
\newcommand{\cK}{\mathcal{K}}
\newcommand{\cL}{\mathcal{L}}
\newcommand{\cM}{\mathcal{M}}
\newcommand{\cN}{\mathcal{N}}
\newcommand{\cO}{\mathcal{O}}
\newcommand{\cP}{\mathcal{P}}
\newcommand{\cQ}{\mathcal{Q}}
\newcommand{\cR}{\mathcal{R}}
\newcommand{\cS}{\mathcal{S}}
\newcommand{\cT}{\mathcal{T}}
\newcommand{\cU}{\mathcal{U}}
\newcommand{\cV}{\mathcal{V}}
\newcommand{\cW}{\mathcal{W}}
\newcommand{\cX}{\mathcal{X}}
\newcommand{\cY}{\mathcal{Y}}
\newcommand{\cZ}{\mathcal{Z}}


%%%%%%%%%%%%%%%%%%%%%%%%%%%%%%%%%%%%%%%%%%%%%
%          Primed Letters
%%%%%%%%%%%%%%%%%%%%%%%%%%%%%%%%%%%%%%%%%%%%%
\newcommand{\aprime}{a^{\prime}}
\newcommand{\bprime}{b^{\prime}}
\newcommand{\cprime}{c^{\prime}}
\newcommand{\dprime}{d^{\prime}}
\newcommand{\eprime}{e^{\prime}}
\newcommand{\fprime}{f^{\prime}}
\newcommand{\gprime}{g^{\prime}}
\newcommand{\hprime}{h^{\prime}}
\newcommand{\iprime}{i^{\prime}}
\newcommand{\jprime}{j^{\prime}}
\newcommand{\kprime}{k^{\prime}}
\newcommand{\lprime}{l^{\prime}}
\newcommand{\mprime}{m^{\prime}}
\newcommand{\nprime}{n^{\prime}}
\newcommand{\oprime}{o^{\prime}}
\newcommand{\pprime}{p^{\prime}}
\newcommand{\qprime}{q^{\prime}}
\newcommand{\rprime}{r^{\prime}}
\newcommand{\sprime}{s^{\prime}}
\newcommand{\tprime}{t^{\prime}}
\newcommand{\uprime}{u^{\prime}}
\newcommand{\vprime}{v^{\prime}}
\newcommand{\wprime}{w^{\prime}}
\newcommand{\xprime}{x^{\prime}}
\newcommand{\yprime}{y^{\prime}}
\newcommand{\zprime}{z^{\prime}}
\newcommand{\Aprime}{A^{\prime}}
\newcommand{\Bprime}{B^{\prime}}
\newcommand{\Cprime}{C^{\prime}}
\newcommand{\Dprime}{D^{\prime}}
\newcommand{\Eprime}{E^{\prime}}
\newcommand{\Fprime}{F^{\prime}}
\newcommand{\Gprime}{G^{\prime}}
\newcommand{\Hprime}{H^{\prime}}
\newcommand{\Iprime}{I^{\prime}}
\newcommand{\Jprime}{J^{\prime}}
\newcommand{\Kprime}{K^{\prime}}
\newcommand{\Lprime}{L^{\prime}}
\newcommand{\Mprime}{M^{\prime}}
\newcommand{\Nprime}{N^{\prime}}
\newcommand{\Oprime}{O^{\prime}}
\newcommand{\Pprime}{P^{\prime}}
\newcommand{\Qprime}{Q^{\prime}}
\newcommand{\Rprime}{R^{\prime}}
\newcommand{\Sprime}{S^{\prime}}
\newcommand{\Tprime}{T^{\prime}}
\newcommand{\Uprime}{U^{\prime}}
\newcommand{\Vprime}{V^{\prime}}
\newcommand{\Wprime}{W^{\prime}}
\newcommand{\Xprime}{X^{\prime}}
\newcommand{\Yprime}{Y^{\prime}}
\newcommand{\Zprime}{Z^{\prime}}
\newcommand{\alphaprime}{\alpha^{\prime}}
\newcommand{\betaprime}{\beta^{\prime}}
\newcommand{\gammaprime}{\gamma^{\prime}}
\newcommand{\Gammaprime}{\Gamma^{\prime}}
\newcommand{\deltaprime}{\delta^{\prime}}
\newcommand{\epsilonprime}{\epsilon^{\prime}}
\newcommand{\kappaprime}{\kappa^{\prime}}
\newcommand{\lambdaprime}{\lambda^{\prime}}
\newcommand{\rhoprime}{\rho^{\prime}}
\newcommand{\Sigmaprime}{\Sigma^{\prime}}
\newcommand{\tauprime}{\tau^{\prime}}
\newcommand{\xiprime}{\xi^{\prime}}
\newcommand{\thetaprime}{\theta^{\prime}}
\newcommand{\Omegaprime}{\Omega^{\prime}}
\newcommand{\cMprime}{\cM^{\prime}}
\newcommand{\cNprime}{\cN^{\prime}}
\newcommand{\cPprime}{\cP^{\prime}}
\newcommand{\cQprime}{\cQ^{\prime}}
\newcommand{\cRprime}{\cR^{\prime}}
\newcommand{\cSprime}{\cS^{\prime}}
\newcommand{\cTprime}{\cT^{\prime}}

%%%%%%%%%%%%%%%%%%%%%%%%%%%%%%%%%%%%%%%%%%%%%
%          bar Letters
%%%%%%%%%%%%%%%%%%%%%%%%%%%%%%%%%%%%%%%%%%%%%
\newcommand{\abar}{\bar{a}}
\newcommand{\bbar}{\bar{b}}
\newcommand{\cbar}{\bar{c}}
\newcommand{\ibar}{\bar{i}}
\newcommand{\jbar}{\bar{j}}
\newcommand{\nbar}{\bar{n}}
\newcommand{\xbar}{\bar{x}}
\newcommand{\ybar}{\bar{y}}
\newcommand{\zbar}{\bar{z}}
\newcommand{\pibar}{\bar{\pi}}
\newcommand{\phibar}{\bar{\varphi}}
\newcommand{\psibar}{\bar{\psi}}
\newcommand{\thetabar}{\bar{\theta}}
\newcommand{\nubar}{\bar{\nu}}

%%%%%%%%%%%%%%%%%%%%%%%%%%%%%%%%%%%%%%%%%%%%%
%          star Letters
%%%%%%%%%%%%%%%%%%%%%%%%%%%%%%%%%%%%%%%%%%%%%
\newcommand{\phistar}{\phi^{*}}
\newcommand{\Mstar}{M^{*}}

%%%%%%%%%%%%%%%%%%%%%%%%%%%%%%%%%%%%%%%%%%%%%
%          dotletters Letters
%%%%%%%%%%%%%%%%%%%%%%%%%%%%%%%%%%%%%%%%%%%%%
\newcommand{\Gdot}{\dot{G}}

%%%%%%%%%%%%%%%%%%%%%%%%%%%%%%%%%%%%%%%%%%%%%
%         check Letters
%%%%%%%%%%%%%%%%%%%%%%%%%%%%%%%%%%%%%%%%%%%%%
\newcommand{\deltacheck}{\check{\delta}}
\newcommand{\gammacheck}{\check{\gamma}}


%%%%%%%%%%%%%%%%%%%%%%%%%%%%%%%%%%%%%%%%%%%%%
%          Formulas
%%%%%%%%%%%%%%%%%%%%%%%%%%%%%%%%%%%%%%%%%%%%%

\newcommand{\formulaphi}{\text{\large $\varphi$}}
\newcommand{\Formulaphi}{\text{\Large $\varphi$}}


%%%%%%%%%%%%%%%%%%%%%%%%%%%%%%%%%%%%%%%%%%%%%
%          Fraktur Letters
%%%%%%%%%%%%%%%%%%%%%%%%%%%%%%%%%%%%%%%%%%%%%

\newcommand{\fa}{\mathfrak{a}}
\newcommand{\fb}{\mathfrak{b}}
\newcommand{\fc}{\mathfrak{c}}
\newcommand{\fk}{\mathfrak{k}}
\newcommand{\fp}{\mathfrak{p}}
\newcommand{\fq}{\mathfrak{q}}
\newcommand{\fr}{\mathfrak{r}}
\newcommand{\fA}{\mathfrak{A}}
\newcommand{\fB}{\mathfrak{B}}
\newcommand{\fC}{\mathfrak{C}}
\newcommand{\fD}{\mathfrak{D}}

%%%%%%%%%%%%%%%%%%%%%%%%%%%%%%%%%%%%%%%%%%%%%
%          Bold Letters
%%%%%%%%%%%%%%%%%%%%%%%%%%%%%%%%%%%%%%%%%%%%%
\newcommand{\ba}{\mathbf{a}}
\newcommand{\bb}{\mathbf{b}}
\newcommand{\bc}{\mathbf{c}}
\newcommand{\bd}{\mathbf{d}}
\newcommand{\be}{\mathbf{e}}
\newcommand{\bu}{\mathbf{u}}
\newcommand{\bv}{\mathbf{v}}
\newcommand{\bw}{\mathbf{w}}
\newcommand{\bx}{\mathbf{x}}
\newcommand{\by}{\mathbf{y}}
\newcommand{\bz}{\mathbf{z}}
\newcommand{\bSigma}{\boldsymbol{\Sigma}}
\newcommand{\bPi}{\boldsymbol{\Pi}}
\newcommand{\bDelta}{\boldsymbol{\Delta}}
\newcommand{\bdelta}{\boldsymbol{\delta}}
\newcommand{\bgamma}{\boldsymbol{\gamma}}
\newcommand{\bGamma}{\boldsymbol{\Gamma}}

%%%%%%%%%%%%%%%%%%%%%%%%%%%%%%%%%%%%%%%%%%%%%
%         Bold numbers
%%%%%%%%%%%%%%%%%%%%%%%%%%%%%%%%%%%%%%%%%%%%%
\newcommand{\bzero}{\mathbf{0}}

%%%%%%%%%%%%%%%%%%%%%%%%%%%%%%%%%%%%%%%%%%%%%
% Projective-Like Pointclasses
%%%%%%%%%%%%%%%%%%%%%%%%%%%%%%%%%%%%%%%%%%%%%
\newcommand{\Sa}[2][\alpha]{\Sigma_{(#1,#2)}}
\newcommand{\Pa}[2][\alpha]{\Pi_{(#1,#2)}}
\newcommand{\Da}[2][\alpha]{\Delta_{(#1,#2)}}
\newcommand{\Aa}[2][\alpha]{A_{(#1,#2)}}
\newcommand{\Ca}[2][\alpha]{C_{(#1,#2)}}
\newcommand{\Qa}[2][\alpha]{Q_{(#1,#2)}}
\newcommand{\da}[2][\alpha]{\delta_{(#1,#2)}}
\newcommand{\leqa}[2][\alpha]{\leq_{(#1,#2)}}
\newcommand{\lessa}[2][\alpha]{<_{(#1,#2)}}
\newcommand{\equiva}[2][\alpha]{\equiv_{(#1,#2)}}


\newcommand{\Sl}[1]{\Sa[\lambda]{#1}}
\newcommand{\Pl}[1]{\Pa[\lambda]{#1}}
\newcommand{\Dl}[1]{\Da[\lambda]{#1}}
\newcommand{\Al}[1]{\Aa[\lambda]{#1}}
\newcommand{\Cl}[1]{\Ca[\lambda]{#1}}
\newcommand{\Ql}[1]{\Qa[\lambda]{#1}}

\newcommand{\San}{\Sa{n}}
\newcommand{\Pan}{\Pa{n}}
\newcommand{\Dan}{\Da{n}}
\newcommand{\Can}{\Ca{n}}
\newcommand{\Qan}{\Qa{n}}
\newcommand{\Aan}{\Aa{n}}
\newcommand{\dan}{\da{n}}
\newcommand{\leqan}{\leqa{n}}
\newcommand{\lessan}{\lessa{n}}
\newcommand{\equivan}{\equiva{n}}

\newcommand{\SigmaOneOmega}{\Sigma^1_{\omega}}
\newcommand{\SigmaOneOmegaPlusOne}{\Sigma^1_{\omega+1}}
\newcommand{\PiOneOmega}{\Pi^1_{\omega}}
\newcommand{\PiOneOmegaPlusOne}{\Pi^1_{\omega+1}}
\newcommand{\DeltaOneOmegaPlusOne}{\Delta^1_{\omega+1}}

%%%%%%%%%%%%%%%%%%%%%%%%%%%%%%%%%%%%%%%%%%%%%
% Linear Algebra
%%%%%%%%%%%%%%%%%%%%%%%%%%%%%%%%%%%%%%%%%%%%%
\newcommand{\transpose}[1]{{#1}^{\text{T}}}
\newcommand{\norm}[1]{\lVert{#1}\rVert}
\newcommand\aug{\fboxsep=-\fboxrule\!\!\!\fbox{\strut}\!\!\!}

%%%%%%%%%%%%%%%%%%%%%%%%%%%%%%%%%%%%%%%%%%%%%
% Number Theory
%%%%%%%%%%%%%%%%%%%%%%%%%%%%%%%%%%%%%%%%%%%%%
\newcommand{\av}[1]{\lvert#1\rvert}
\DeclareMathOperator{\divides}{\mid}
\DeclareMathOperator{\ndivides}{\nmid}
\DeclareMathOperator{\lcm}{lcm}
\DeclareMathOperator{\sign}{sign}
\newcommand{\floor}[1]{\left\lfloor{#1}\right\rfloor}
\DeclareMathOperator{\ConCl}{CC}
\DeclareMathOperator{\ord}{ord}



\graphicspath{{images/}}

\newtheorem*{claim}{claim}
\newtheorem*{observation}{Observation}
\newtheorem*{warning}{Warning}
\newtheorem*{question}{Question}
\newtheorem{remark}[theorem]{Remark}

\newenvironment*{subproof}[1][Proof]
{\begin{proof}[#1]}{\renewcommand{\qedsymbol}{$\diamondsuit$} \end{proof}}

\mode<presentation>
{
  \usetheme{Singapore}
  % or ...

  \setbeamercovered{invisible}
  % or whatever (possibly just delete it)
}


\usepackage[english]{babel}
% or whatever

\usepackage[latin1]{inputenc}
% or whatever

\usepackage{times}
\usepackage[T1]{fontenc}
% Or whatever. Note that the encoding and the font should match. If T1
% does not look nice, try deleting the line with the fontenc.

\title{Lesson 8 \\ Linear Diophantine Equations}
\subtitle{Math 310, Elementary Number Theory \\ Fall 2021 \\ SFSU}
\author{Mitch Rudominer}
\date{}



% If you wish to uncover everything in a step-wise fashion, uncomment
% the following command:

\beamerdefaultoverlayspecification{<+->}

\begin{document}

\begin{frame}
  \titlepage
\end{frame}

\begin{frame}{Linear Diophantine Equations}

\begin{itemize}
  \item A linear Diophantine equation is an equation of the following form:
  \item $ax + by = c$
  \item where $a,b,c$ are integer constants, and $x,y$ are integer variables.
  \item To \emph{solve} the above equation means
  \item to find all integers $x,y$ that make the equation true.
  \item \textbf{Example.} $12x + 18y = 30$.
  \item This is a linear Diophantine equation.
  \item The goal is to find all integers $x,y$ that make the equation true.
\end{itemize}

\end{frame}

%%%%%%%%%%%%%%%%%%%%%%%%%%%%%%%%%%%%%%%%%%%%%%%%%%%%%%%%%%%%%%%%%%%%%

\begin{frame}{Solution Sets}

\begin{itemize}
  \item The \emph{solution set} to the linear Diophantine equation $ax + by = c$
  \item is the set of all ordered pairs $(x,y)$ such that
  \item $x$ and $y$ are integers and the equation holds.
  \item The solution set might be empty. There may be no solutions.
  \item If there are any solutions, we will see that the solution set is infinite. i.e. there are
  infinitely many solutions.
\end{itemize}

\end{frame}

%%%%%%%%%%%%%%%%%%%%%%%%%%%%%%%%%%%%%%%%%%%%%%%%%%%%%%%%%%%%%%%%%%%%%

\begin{frame}{When do solutions exist?}

\begin{itemize}
  \item \textbf{Theorem.} The linear Diophantine equation $ax+by=c$ has a solution
  \item iff $d\divides c$, where $d=\gcd(a,b)$.
  \item \textbf{Example.} $12x + 18y = 30$ has a solution.
  \item Because $\gcd(12,18) = 6$ and $6\divides 30$.
  \item \textbf{Example.} $12x + 18y = 20$ has no solutions.
  \item Because $\gcd(12,18) = 6$ and $6\ndivides 20$.
  \item \textbf{proof.} Let $d=\gcd(a,b)$.
  \item Recall that the set of multiples of $d$ is the same as the set of
  integer linear combinations of $a$ and $b$.
  \item $ax + by = c$ has a solution iff $c$ is an integer linear combination
  of $a$ and $b$.
  \item So $ax + by = c$ has a solution iff $c$ is a multiple of $d$. $\qed$
\end{itemize}

\end{frame}

%%%%%%%%%%%%%%%%%%%%%%%%%%%%%%%%%%%%%%%%%%%%%%%%%%%%%%%%%%%%%%%%%%%%%

\begin{frame}{Homogeneous Equations}

\begin{itemize}
  \item The linear Diophantine equation $ax+by=c$ is called \emph{homogeneous}
  \item iff $c=0$.
  \item Otherwise it is called \emph{inhomogeneous.}
  \item \textbf{Example.} $12x + 18y = 0$ is a homogeneous linear Diophantine equation.
  \item \textbf{Example.} $12x + 18y = 30$ is an inhomogeneous linear Diophantine equation.
  \item \textbf{Fact.} Homogeneous linear Diophantine equations always have a solution.
  \item $x=y=0$ is always one solution.
  \item We can also see this from the previous theorem because $d\divides 0$.
\end{itemize}

\end{frame}

%%%%%%%%%%%%%%%%%%%%%%%%%%%%%%%%%%%%%%%%%%%%%%%%%%%%%%%%%%%%%%%%%%%%%

\begin{frame}{Solving Homogeneous Equations, Example}

\begin{itemize}
  \item \textbf{Example.} Solve $12x + 18y = 0$.
  \item $2x + 3y = 0$  (cancelled the gcd of 12 and 18)
  \item One solution: $x=3, y=-2$.
  \item General solution $x=3k,y=-2k$, $k\in\Z$.
  \item Proof that these are the only solutions: Since $2x = -3y$, we know $3\divides 2x$.
  \item Since $\gcd(3,2)=1$ we know $3\divides x$ i.e. $x=3k$ for some $k\in Z$.
  \item Plug in $x=3k$ into $2x + 3y = 0$. We get $6k + 3y = 0$.
  \item Solving for $y$ we get $y=-2k$. $\qed$
  \item Solution set = $\setof{(3k, -2k)}{k\in\Z}$.
  \item Some of the solutions: $(0,0), (3,-2), (-3, 2), (6, -4), (-6, 4), (12, -8)$.
\end{itemize}

\end{frame}

%%%%%%%%%%%%%%%%%%%%%%%%%%%%%%%%%%%%%%%%%%%%%%%%%%%%%%%%%%%%%%%%%%%%%

\begin{frame}{Solving Homogeneous Equations}

\begin{itemize}
  \item \textbf{Theorem.} The solution set for the homogeneous linear Diophantine equation
  $ax + by =0$ is
  \item $S = \setof{(\bbar k, -\abar k)}{k\in \Z}$
  \item where $d=\gcd(a,b)$, $a=\abar d$, $b = \bbar d$.
  \item (Recall that $\gcd(\abar,\bbar)=1$.)
  \item \textbf{proof of Theorem.} First let $(x,y)\in S$ and let's check that $ax+by=0$.
  \item $a \bbar k + b(-\abar k) = d\abar\bbar k  - d\bbar\abar k = 0$.
  \item So every element of $S$ is a solution.
  \item Conversely if $(x,y)$ is a solution then $ax+by=0$
  \item so $d\abar x + d \bbar y = 0$ so $\abar x + \bbar y = 0$.
  \item Since $\abar x = -\bbar y$ we know that $\bbar \divides \abar x$.
  \item Since $\gcd(\abar,\bbar) = 1$ we know  $\bbar\divides x$, so $x=\bbar k$ for some $k\in\Z$.
  \item Plugging in $x=\bbar k$ into $\abar x + \bbar y = 0$ we get
  \item $\abar \bbar k + \bbar y = 0$ so $y=-\abar k$. $\qed$
\end{itemize}

\end{frame}

%%%%%%%%%%%%%%%%%%%%%%%%%%%%%%%%%%%%%%%%%%%%%%%%%%%%%%%%%%%%%%%%%%%%%

\begin{frame}{Homogeneous Linear Diophantine Equation, Example}

\begin{itemize}
  \item Find the solution set of $10x - 25y = 0$.
  \item $d = \gcd(10, -25) = 5$.
  \item $\abar = 2, \bbar = -5$.
  \item $2x - 5y = 0$.
  \item General solution: $x=-5k, y=-2k$.
  \item Or equivalently $x=5k, y=2k$.
  \item Solution set = $\setof{(5k, 2k)}{k\in\Z}$.
  \item Some example solutions: $(0,0), (5,2),(-5,-2), (10,4), (15, 6)$.
\end{itemize}

\end{frame}

%%%%%%%%%%%%%%%%%%%%%%%%%%%%%%%%%%%%%%%%%%%%%%%%%%%%%%%%%%%%%%%%%%%%%

\begin{frame}{Particular and General Solution}

\begin{itemize}
  \item Given any \emph{particular} solution $(x_0,y_0)$ to the inhomogeneous linear Diophantine equation $ax+by=c$
  \item we can obtain the \emph{general} solution by adding to $(x_0,y_0)$ the general solution to the
  corresponding \emph{homogeneous} equation.
  \item \textbf{Theorem.} If $(x_0, y_0)$ is a particular solution to the linear Diophantine
  equation $ax+by=c$,
  \item and if $T$ is the solution set for
  $ax + by = 0$,
  \item then the general solution to $ax+by=c$ is given by
  \item $S=\setof{(x_0 + u, y_0 + v)}{(u,v) \in T}$.
\end{itemize}

\end{frame}

%%%%%%%%%%%%%%%%%%%%%%%%%%%%%%%%%%%%%%%%%%%%%%%%%%%%%%%%%%%%%%%%%%%%%

\begin{frame}{Particular and General Solution, Proof}

\begin{itemize}
  \item \textbf{proof} If $(u,v)\in T$ then $au+bv = 0$ so $a(x_0 +u) + b(y_0 +v) = (ax_0 + by_0) + (au + bv)
  =c +0 =c$ so $(x_0+u, y_0+v)$ is a solution.
  \item Conversely suppose $(x,y)$ is any solution so $ax + by = c$ and let $u=x-x_0, v=y-y_0$ so that $x=x_0+u, y=y_0+v$.
  \item Then $au + bv = a(x-x_0) + b(y-y_0) = (ax + by) - (ax_0 + by_0) = c-c = 0$ so $(u,v)\in T$. $\qed$
\end{itemize}
\end{frame}

%%%%%%%%%%%%%%%%%%%%%%%%%%%%%%%%%%%%%%%%%%%%%%%%%%%%%%%%%%%%%%%%%%%%%

\begin{frame}{Particular and General Solution, Example}

\begin{itemize}
  \item \textbf{Example.} Find the general solution to the inhomogeneous linear Diophantine equation
  $10x - 25y = 5$.
  \item Solution: A particular solution is $x_0=3, y_0 = 1$.
  \item The general solution to the homogeneous equation $10x - 25y = 0$ is
  \item $T=\setof{(5k, 2k)}{k\in\Z}$.
  \item So the general solution to the inhomogeneous equation is
  \item $S = \setof{(3 + 5k, 1 + 2k)}{k\in\Z}$.
  \item Some example solutions are: $(3,1), (8,3), (-2, -1), (13, 5)$.
\end{itemize}
\end{frame}

%%%%%%%%%%%%%%%%%%%%%%%%%%%%%%%%%%%%%%%%%%%%%%%%%%%%%%%%%%%%%%%%%%%%%

\begin{frame}{Finding a particular solution}

\begin{itemize}
  \item Three techniques for finding a particular solution to an inhomogeneous linear Diophantine equation
  $ax+by=c$.
  \item Technique 1: Trial and error.
  \item Example: Find a particular solution to $10x - 25y = 5$.
  \item Solution: I can see that 30 - 25 = 5 so letting $x_0=3, y_0 = 1$ works.
  \item Technique 2: Use the extended Euclidean algorithm to find $d=\gcd(a,b)$
  and $s,t$ such that $as + bt = d$.
  \item If $d\ndivides c$ there is no solution.
  \item Suppose $d\divides c$ and write $c=d w$ for some $w\in Z$.
  \item Then $asw + btw = dw = c$. So $x_0 = sw, y_0 = tw$ is a particular solution.
\end{itemize}

\end{frame}

%%%%%%%%%%%%%%%%%%%%%%%%%%%%%%%%%%%%%%%%%%%%%%%%%%%%%%%%%%%%%%%%%%%%%

\begin{frame}{Finding a particular solution: Extended Euclidean Algorithm}

\begin{itemize}
  \item Example: Find a particular solution to $87x + 51y = 6$.
  \item Solution: Last class we used the extended Euclidean to compute
  \item $d=\gcd(87, 51) = 3$ and
  \item $87\cdot (-7) + 51 \cdot 12 = 3$. \quad $(\dagger)$
  \item $3\divides 6$ so there are solutions to $87x + 51y = 6$.
  \item Multiplying equation $(\dagger)$ above by 2 we get
  \item $87\cdot (-14) + 51 \cdot 24 = 6$.
  \item So one particular solution is $(-14, 24)$.
\end{itemize}

\end{frame}
%%%%%%%%%%%%%%%%%%%%%%%%%%%%%%%%%%%%%%%%%%%%%%%%%%%%%%%%%%%%%%%%%%%%%


\begin{frame}{Finding a particular solution: Variant technique}
\begin{itemize}
  \item Technique 3: Variant of technique 2.
  \item Suppose you can easily find $d=\gcd(a,b)$ but you don't yet know $s,t$
  such tht $as + bt = d$.
  \item Let $\abar=\frac{a}{d}$, $\bbar=\frac{b}{d}$.
  \item $\gcd(\abar,\bbar)=1$. It may be easier to find $s,t$ such that $\abar s + \bbar t = 1$.
  \item Then multiply by $d$ to get $a s + b t = d$.
  \item Then proceed as in technique 2.
\end{itemize}

\end{frame}

%%%%%%%%%%%%%%%%%%%%%%%%%%%%%%%%%%%%%%%%%%%%%%%%%%%%%%%%%%%%%%%%%%%%


\begin{frame}{Example of Variant technique}
\begin{itemize}
  \item Example: Find a particular solution to $27x + 51y = 6$.
  \item $27=3^3$, $51 =17 \cdot 3$, so $d = \gcd(27, 51) = 3$.
  \item $3\divides 6$ so there are solutions.
  \item Let $\abar=\frac{a}{d} = \frac{27}{3} = 9$, $\bbar=\frac{b}{d} = \frac{51}{3} = 17$.
  \item $\gcd(\abar,\bbar)=1$. \item Solve $9x + 17y = 1$.
  \item Solution: $9 \cdot (2) + 17 \cdot (-1) = 1$.
  \item Multiply by $d$: $27\cdot (2) + 51 \cdot (-1) = 3$.
  \item Proceed as in technique 2: $27\cdot (4) + 51 \cdot (-2) = 6$.
  \item So one particular solution is $(4,-2)$.
\end{itemize}

\end{frame}



%%%%%%%%%%%%%%%%%%%%%%%%%%%%%%%%%%%%%%%%%%%%%%%%%%%%%%%%%%%%%%%%%%%%%

\begin{frame}{Geometric Representation}

\begin{itemize}
  \item In pre-Calculus we learn that the graph of the equation $ax+by = c$
  \item is a line in the plane. (Here $a,b,c,x,y$ are allowed to be real numbers, not only integers.)
  \item If $c=0$, the graph of $ax+by=0$ is a line through the origin.
  \item If $a,b,c$ are integers, then the solution to the linear Diophantine equation $ax+by=c$
  \item are the points $(x,y)$ on the line with integer coordinates.
\end{itemize}

\end{frame}


\end{document}
