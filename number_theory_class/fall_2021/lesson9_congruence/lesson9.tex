% $Header$

\documentclass{beamer}
%\documentclass[handout]{beamer}
\usepackage{amsmath,amssymb,latexsym,eucal,amsthm,graphicx,hyperref}
%%%%%%%%%%%%%%%%%%%%%%%%%%%%%%%%%%%%%%%%%%%%%
% Common Set Theory Constructs
%%%%%%%%%%%%%%%%%%%%%%%%%%%%%%%%%%%%%%%%%%%%%

\newcommand{\setof}[2]{\left\{ \, #1 \, \left| \, #2 \, \right.\right\}}
\newcommand{\lsetof}[2]{\left\{\left. \, #1 \, \right| \, #2 \,  \right\}}
\newcommand{\bigsetof}[2]{\bigl\{ \, #1 \, \bigm | \, #2 \,\bigr\}}
\newcommand{\Bigsetof}[2]{\Bigl\{ \, #1 \, \Bigm | \, #2 \,\Bigr\}}
\newcommand{\biggsetof}[2]{\biggl\{ \, #1 \, \biggm | \, #2 \,\biggr\}}
\newcommand{\Biggsetof}[2]{\Biggl\{ \, #1 \, \Biggm | \, #2 \,\Biggr\}}
\newcommand{\dotsetof}[2]{\left\{ \, #1 \, : \, #2 \, \right\}}
\newcommand{\bigdotsetof}[2]{\bigl\{ \, #1 \, : \, #2 \,\bigr\}}
\newcommand{\Bigdotsetof}[2]{\Bigl\{ \, #1 \, \Bigm : \, #2 \,\Bigr\}}
\newcommand{\biggdotsetof}[2]{\biggl\{ \, #1 \, \biggm : \, #2 \,\biggr\}}
\newcommand{\Biggdotsetof}[2]{\Biggl\{ \, #1 \, \Biggm : \, #2 \,\Biggr\}}
\newcommand{\sequence}[2]{\left\langle \, #1 \,\left| \, #2 \, \right. \right\rangle}
\newcommand{\lsequence}[2]{\left\langle\left. \, #1 \, \right| \,#2 \,  \right\rangle}
\newcommand{\bigsequence}[2]{\bigl\langle \,#1 \, \bigm | \, #2 \, \bigr\rangle}
\newcommand{\Bigsequence}[2]{\Bigl\langle \,#1 \, \Bigm | \, #2 \, \Bigr\rangle}
\newcommand{\biggsequence}[2]{\biggl\langle \,#1 \, \biggm | \, #2 \, \biggr\rangle}
\newcommand{\Biggsequence}[2]{\Biggl\langle \,#1 \, \Biggm | \, #2 \, \Biggr\rangle}
\newcommand{\singleton}[1]{\left\{#1\right\}}
\newcommand{\angles}[1]{\left\langle #1 \right\rangle}
\newcommand{\bigangles}[1]{\bigl\langle #1 \bigr\rangle}
\newcommand{\Bigangles}[1]{\Bigl\langle #1 \Bigr\rangle}
\newcommand{\biggangles}[1]{\biggl\langle #1 \biggr\rangle}
\newcommand{\Biggangles}[1]{\Biggl\langle #1 \Biggr\rangle}


\newcommand{\force}[1]{\Vert\!\underset{\!\!\!\!\!#1}{\!\!\!\relbar\!\!\!%
\relbar\!\!\relbar\!\!\relbar\!\!\!\relbar\!\!\relbar\!\!\relbar\!\!\!%
\relbar\!\!\relbar\!\!\relbar}}
\newcommand{\nforce}[1]{\Vert\!\underset{\!\!\!\!\!#1}{\!\!\!\relbar\!\!\!%
\relbar\!\!\relbar\!\!\relbar\!\!\!\relbar\!\!\relbar\!\!\relbar\!\!\!%
\relbar\!\!\not\relbar\!\!\relbar}}
\newcommand{\forcein}[2]{\overset{#2}{\Vert\underset{\!\!\!\!\!#1}%
{\!\!\!\relbar\!\!\!\relbar\!\!\relbar\!\!\relbar\!\!\!\relbar\!\!\relbar\!%
\!\relbar\!\!\!\relbar\!\!\relbar\!\!\relbar\!\!\relbar\!\!\!\relbar\!\!%
\relbar\!\!\relbar}}}

\newcommand{\pre}[2]{{}^{#2}\!{#1}}

\newcommand{\restr}{\!\!\upharpoonright\!}

%%%%%%%%%%%%%%%%%%%%%%%%%%%%%%%%%%%%%%%%%%%%%
% Set-Theoretic Connectives
%%%%%%%%%%%%%%%%%%%%%%%%%%%%%%%%%%%%%%%%%%%%%

\newcommand{\intersect}{\cap}
\newcommand{\union}{\cup}
\newcommand{\Intersection}[1]{\bigcap\limits_{#1}}
\newcommand{\Union}[1]{\bigcup\limits_{#1}}
\newcommand{\adjoin}{{}^\frown}
\newcommand{\forces}{\Vdash}

%%%%%%%%%%%%%%%%%%%%%%%%%%%%%%%%%%%%%%%%%%%%%
% Miscellaneous
%%%%%%%%%%%%%%%%%%%%%%%%%%%%%%%%%%%%%%%%%%%%%
\newcommand{\defeq}{=_{\text{def}}}
\newcommand{\Turingleq}{\leq_{\text{T}}}
\newcommand{\Turingless}{<_{\text{T}}}
\newcommand{\lexleq}{\leq_{\text{lex}}}
\newcommand{\lexless}{<_{\text{lex}}}
\newcommand{\Turingequiv}{\equiv_{\text{T}}}

%%%%%%%%%%%%%%%%%%%%%%%%%%%%%%%%%%%%%%%%%%%%%
% Constants
%%%%%%%%%%%%%%%%%%%%%%%%%%%%%%%%%%%%%%%%%%%%%
\newcommand{\R}{\mathbb{R}}
\renewcommand{\P}{\mathbb{P}}
\newcommand{\Q}{\mathbb{Q}}
\newcommand{\Z}{\mathbb{Z}}
\newcommand{\C}{\mathbb{C}}
\newcommand{\N}{\mathbb{N}}
\newcommand{\B}{\mathbb{B}}
\newcommand{\LofR}{L(\R)}
\newcommand{\JofR}[1]{J_{#1}(\R)}
\newcommand{\SofR}[1]{S_{#1}(\R)}
\newcommand{\JalphaR}{\JofR{\alpha}}
\newcommand{\JbetaR}{\JofR{\beta}}
\newcommand{\JlambdaR}{\JofR{\lambda}}
\newcommand{\SalphaR}{\SofR{\alpha}}
\newcommand{\SbetaR}{\SofR{\beta}}
\newcommand{\Pkl}{\mathcal{P}_{\kappa}(\lambda)}
\DeclareMathOperator{\con}{con}
\DeclareMathOperator{\ORD}{OR}
\DeclareMathOperator{\Ord}{OR}
\DeclareMathOperator{\WO}{WO}
\DeclareMathOperator{\OD}{OD}
\DeclareMathOperator{\HOD}{HOD}
\DeclareMathOperator{\HC}{HC}
\DeclareMathOperator{\WF}{WF}
\DeclareMathOperator{\HF}{HF}
\newcommand{\One}{I}
\newcommand{\ONE}{I}
\newcommand{\Two}{II}
\newcommand{\TWO}{II}

%%%%%%%%%%%%%%%%%%%%%%%%%%%%%%%%%%%%%%%%%%%%%
% Commutative Algebra Constants
%%%%%%%%%%%%%%%%%%%%%%%%%%%%%%%%%%%%%%%%%%%%%
\DeclareMathOperator{\dottimes}{\dot{\times}}

%%%%%%%%%%%%%%%%%%%%%%%%%%%%%%%%%%%%%%%%%%%%%
% Theories
%%%%%%%%%%%%%%%%%%%%%%%%%%%%%%%%%%%%%%%%%%%%%
\DeclareMathOperator{\ZFC}{ZFC}
\DeclareMathOperator{\ZF}{ZF}
\DeclareMathOperator{\AD}{AD}
\DeclareMathOperator{\ADR}{AD_{\R}}
\DeclareMathOperator{\KP}{KP}
\DeclareMathOperator{\PD}{PD}
\DeclareMathOperator{\CH}{CH}
\DeclareMathOperator{\HPC}{HPC} % HOD pair capturing
%%%%%%%%%%%%%%%%%%%%%%%%%%%%%%%%%%%%%%%%%%%%%
% Iteration Trees
%%%%%%%%%%%%%%%%%%%%%%%%%%%%%%%%%%%%%%%%%%%%%

\newcommand{\pred}{\text{-pred}}

%%%%%%%%%%%%%%%%%%%%%%%%%%%%%%%%%%%%%%%%%%%%%%%%
% Operator Names
%%%%%%%%%%%%%%%%%%%%%%%%%%%%%%%%%%%%%%%%%%%%%%%%
\DeclareMathOperator{\Det}{Det}
\DeclareMathOperator{\dom}{dom}
\DeclareMathOperator{\ran}{ran}
\DeclareMathOperator{\range}{ran}
\DeclareMathOperator{\image}{image}
\DeclareMathOperator{\crit}{crit}
\DeclareMathOperator{\card}{card}
\DeclareMathOperator{\cf}{cf}
\DeclareMathOperator{\cof}{cof}
\DeclareMathOperator{\rank}{rank}
\DeclareMathOperator{\ot}{o.t.}
\DeclareMathOperator{\ords}{o}
\DeclareMathOperator{\ro}{r.o.}
\DeclareMathOperator{\rud}{rud}
\DeclareMathOperator{\Powerset}{\mathcal{P}}
\DeclareMathOperator{\length}{lh}
\DeclareMathOperator{\lh}{lh}
\DeclareMathOperator{\limit}{lim}
\DeclareMathOperator{\fld}{fld}
\DeclareMathOperator{\projection}{p}
\DeclareMathOperator{\Ult}{Ult}
\DeclareMathOperator{\ULT}{Ult}
\DeclareMathOperator{\Coll}{Coll}
\DeclareMathOperator{\Col}{Col}
\DeclareMathOperator{\Hull}{Hull}
\DeclareMathOperator*{\dirlim}{dir lim}
\DeclareMathOperator{\Scale}{Scale}
\DeclareMathOperator{\supp}{supp}
\DeclareMathOperator{\trancl}{tran.cl.}
\DeclareMathOperator{\trace}{Tr}
\DeclareMathOperator{\diag}{diag}
\DeclareMathOperator{\spn}{span}
\DeclareMathOperator{\sgn}{sgn}
%%%%%%%%%%%%%%%%%%%%%%%%%%%%%%%%%%%%%%%%%%%%%
% Logical Connectives
%%%%%%%%%%%%%%%%%%%%%%%%%%%%%%%%%%%%%%%%%%%%%
\newcommand{\IImplies}{\Longrightarrow}
\newcommand{\SkipImplies}{\quad\Longrightarrow\quad}
\newcommand{\Ifff}{\Longleftrightarrow}
\newcommand{\iimplies}{\longrightarrow}
\newcommand{\ifff}{\longleftrightarrow}
\newcommand{\Implies}{\Rightarrow}
\newcommand{\Iff}{\Leftrightarrow}
\renewcommand{\implies}{\rightarrow}
\renewcommand{\iff}{\leftrightarrow}
\newcommand{\AND}{\wedge}
\newcommand{\OR}{\vee}
\newcommand{\st}{\text{ s.t. }}
\newcommand{\Or}{\text{ or }}

%%%%%%%%%%%%%%%%%%%%%%%%%%%%%%%%%%%%%%%%%%%%%
% Function Arrows
%%%%%%%%%%%%%%%%%%%%%%%%%%%%%%%%%%%%%%%%%%%%%

\newcommand{\injection}{\xrightarrow{\text{1-1}}}
\newcommand{\surjection}{\xrightarrow{\text{onto}}}
\newcommand{\bijection}{\xrightarrow[\text{onto}]{\text{1-1}}}
\newcommand{\cofmap}{\xrightarrow{\text{cof}}}
\newcommand{\map}{\rightarrow}

%%%%%%%%%%%%%%%%%%%%%%%%%%%%%%%%%%%%%%%%%%%%%
% Mouse Comparison Operators
%%%%%%%%%%%%%%%%%%%%%%%%%%%%%%%%%%%%%%%%%%%%%
\newcommand{\initseg}{\trianglelefteq}
\newcommand{\properseg}{\lhd}
\newcommand{\notinitseg}{\ntrianglelefteq}
\newcommand{\notproperseg}{\ntriangleleft}

%%%%%%%%%%%%%%%%%%%%%%%%%%%%%%%%%%%%%%%%%%%%%
%           calligraphic letters
%%%%%%%%%%%%%%%%%%%%%%%%%%%%%%%%%%%%%%%%%%%%%
\newcommand{\cA}{\mathcal{A}}
\newcommand{\cB}{\mathcal{B}}
\newcommand{\cC}{\mathcal{C}}
\newcommand{\cD}{\mathcal{D}}
\newcommand{\cE}{\mathcal{E}}
\newcommand{\cF}{\mathcal{F}}
\newcommand{\cG}{\mathcal{G}}
\newcommand{\cH}{\mathcal{H}}
\newcommand{\cI}{\mathcal{I}}
\newcommand{\cJ}{\mathcal{J}}
\newcommand{\cK}{\mathcal{K}}
\newcommand{\cL}{\mathcal{L}}
\newcommand{\cM}{\mathcal{M}}
\newcommand{\cN}{\mathcal{N}}
\newcommand{\cO}{\mathcal{O}}
\newcommand{\cP}{\mathcal{P}}
\newcommand{\cQ}{\mathcal{Q}}
\newcommand{\cR}{\mathcal{R}}
\newcommand{\cS}{\mathcal{S}}
\newcommand{\cT}{\mathcal{T}}
\newcommand{\cU}{\mathcal{U}}
\newcommand{\cV}{\mathcal{V}}
\newcommand{\cW}{\mathcal{W}}
\newcommand{\cX}{\mathcal{X}}
\newcommand{\cY}{\mathcal{Y}}
\newcommand{\cZ}{\mathcal{Z}}


%%%%%%%%%%%%%%%%%%%%%%%%%%%%%%%%%%%%%%%%%%%%%
%          Primed Letters
%%%%%%%%%%%%%%%%%%%%%%%%%%%%%%%%%%%%%%%%%%%%%
\newcommand{\aprime}{a^{\prime}}
\newcommand{\bprime}{b^{\prime}}
\newcommand{\cprime}{c^{\prime}}
\newcommand{\dprime}{d^{\prime}}
\newcommand{\eprime}{e^{\prime}}
\newcommand{\fprime}{f^{\prime}}
\newcommand{\gprime}{g^{\prime}}
\newcommand{\hprime}{h^{\prime}}
\newcommand{\iprime}{i^{\prime}}
\newcommand{\jprime}{j^{\prime}}
\newcommand{\kprime}{k^{\prime}}
\newcommand{\lprime}{l^{\prime}}
\newcommand{\mprime}{m^{\prime}}
\newcommand{\nprime}{n^{\prime}}
\newcommand{\oprime}{o^{\prime}}
\newcommand{\pprime}{p^{\prime}}
\newcommand{\qprime}{q^{\prime}}
\newcommand{\rprime}{r^{\prime}}
\newcommand{\sprime}{s^{\prime}}
\newcommand{\tprime}{t^{\prime}}
\newcommand{\uprime}{u^{\prime}}
\newcommand{\vprime}{v^{\prime}}
\newcommand{\wprime}{w^{\prime}}
\newcommand{\xprime}{x^{\prime}}
\newcommand{\yprime}{y^{\prime}}
\newcommand{\zprime}{z^{\prime}}
\newcommand{\Aprime}{A^{\prime}}
\newcommand{\Bprime}{B^{\prime}}
\newcommand{\Cprime}{C^{\prime}}
\newcommand{\Dprime}{D^{\prime}}
\newcommand{\Eprime}{E^{\prime}}
\newcommand{\Fprime}{F^{\prime}}
\newcommand{\Gprime}{G^{\prime}}
\newcommand{\Hprime}{H^{\prime}}
\newcommand{\Iprime}{I^{\prime}}
\newcommand{\Jprime}{J^{\prime}}
\newcommand{\Kprime}{K^{\prime}}
\newcommand{\Lprime}{L^{\prime}}
\newcommand{\Mprime}{M^{\prime}}
\newcommand{\Nprime}{N^{\prime}}
\newcommand{\Oprime}{O^{\prime}}
\newcommand{\Pprime}{P^{\prime}}
\newcommand{\Qprime}{Q^{\prime}}
\newcommand{\Rprime}{R^{\prime}}
\newcommand{\Sprime}{S^{\prime}}
\newcommand{\Tprime}{T^{\prime}}
\newcommand{\Uprime}{U^{\prime}}
\newcommand{\Vprime}{V^{\prime}}
\newcommand{\Wprime}{W^{\prime}}
\newcommand{\Xprime}{X^{\prime}}
\newcommand{\Yprime}{Y^{\prime}}
\newcommand{\Zprime}{Z^{\prime}}
\newcommand{\alphaprime}{\alpha^{\prime}}
\newcommand{\betaprime}{\beta^{\prime}}
\newcommand{\gammaprime}{\gamma^{\prime}}
\newcommand{\Gammaprime}{\Gamma^{\prime}}
\newcommand{\deltaprime}{\delta^{\prime}}
\newcommand{\epsilonprime}{\epsilon^{\prime}}
\newcommand{\kappaprime}{\kappa^{\prime}}
\newcommand{\lambdaprime}{\lambda^{\prime}}
\newcommand{\rhoprime}{\rho^{\prime}}
\newcommand{\Sigmaprime}{\Sigma^{\prime}}
\newcommand{\tauprime}{\tau^{\prime}}
\newcommand{\xiprime}{\xi^{\prime}}
\newcommand{\thetaprime}{\theta^{\prime}}
\newcommand{\Omegaprime}{\Omega^{\prime}}
\newcommand{\cMprime}{\cM^{\prime}}
\newcommand{\cNprime}{\cN^{\prime}}
\newcommand{\cPprime}{\cP^{\prime}}
\newcommand{\cQprime}{\cQ^{\prime}}
\newcommand{\cRprime}{\cR^{\prime}}
\newcommand{\cSprime}{\cS^{\prime}}
\newcommand{\cTprime}{\cT^{\prime}}

%%%%%%%%%%%%%%%%%%%%%%%%%%%%%%%%%%%%%%%%%%%%%
%          bar Letters
%%%%%%%%%%%%%%%%%%%%%%%%%%%%%%%%%%%%%%%%%%%%%
\newcommand{\abar}{\bar{a}}
\newcommand{\bbar}{\bar{b}}
\newcommand{\zbar}{\bar{z}}
\newcommand{\phibar}{\bar{\varphi}}
\newcommand{\psibar}{\bar{\psi}}
\newcommand{\thetabar}{\bar{\theta}}
\newcommand{\nubar}{\bar{\nu}}

%%%%%%%%%%%%%%%%%%%%%%%%%%%%%%%%%%%%%%%%%%%%%
%          star Letters
%%%%%%%%%%%%%%%%%%%%%%%%%%%%%%%%%%%%%%%%%%%%%
\newcommand{\phistar}{\phi^{*}}


%%%%%%%%%%%%%%%%%%%%%%%%%%%%%%%%%%%%%%%%%%%%%
%          Formulas
%%%%%%%%%%%%%%%%%%%%%%%%%%%%%%%%%%%%%%%%%%%%%

\newcommand{\formulaphi}{\text{\large $\varphi$}}
\newcommand{\Formulaphi}{\text{\Large $\varphi$}}


%%%%%%%%%%%%%%%%%%%%%%%%%%%%%%%%%%%%%%%%%%%%%
%          Fraktur Letters
%%%%%%%%%%%%%%%%%%%%%%%%%%%%%%%%%%%%%%%%%%%%%

\newcommand{\fa}{\mathfrak{a}}
\newcommand{\fb}{\mathfrak{b}}
\newcommand{\fc}{\mathfrak{c}}
\newcommand{\fk}{\mathfrak{k}}
\newcommand{\fp}{\mathfrak{p}}
\newcommand{\fq}{\mathfrak{q}}
\newcommand{\fr}{\mathfrak{r}}
\newcommand{\fA}{\mathfrak{A}}
\newcommand{\fB}{\mathfrak{B}}
\newcommand{\fC}{\mathfrak{C}}
\newcommand{\fD}{\mathfrak{D}}

%%%%%%%%%%%%%%%%%%%%%%%%%%%%%%%%%%%%%%%%%%%%%
%          Bold Letters
%%%%%%%%%%%%%%%%%%%%%%%%%%%%%%%%%%%%%%%%%%%%%
\newcommand{\ba}{\mathbf{a}}
\newcommand{\bb}{\mathbf{b}}
\newcommand{\bc}{\mathbf{c}}
\newcommand{\bd}{\mathbf{d}}
\newcommand{\be}{\mathbf{e}}
\newcommand{\bu}{\mathbf{u}}
\newcommand{\bv}{\mathbf{v}}
\newcommand{\bw}{\mathbf{w}}
\newcommand{\bx}{\mathbf{x}}
\newcommand{\by}{\mathbf{y}}
\newcommand{\bz}{\mathbf{z}}
\newcommand{\bSigma}{\boldsymbol{\Sigma}}
\newcommand{\bPi}{\boldsymbol{\Pi}}
\newcommand{\bDelta}{\boldsymbol{\Delta}}
\newcommand{\bdelta}{\boldsymbol{\delta}}
\newcommand{\bgamma}{\boldsymbol{\gamma}}
\newcommand{\bGamma}{\boldsymbol{\Gamma}}

%%%%%%%%%%%%%%%%%%%%%%%%%%%%%%%%%%%%%%%%%%%%%
%         Bold numbers
%%%%%%%%%%%%%%%%%%%%%%%%%%%%%%%%%%%%%%%%%%%%%
\newcommand{\bzero}{\mathbf{0}}

%%%%%%%%%%%%%%%%%%%%%%%%%%%%%%%%%%%%%%%%%%%%%
% Projective-Like Pointclasses
%%%%%%%%%%%%%%%%%%%%%%%%%%%%%%%%%%%%%%%%%%%%%
\newcommand{\Sa}[2][\alpha]{\Sigma_{(#1,#2)}}
\newcommand{\Pa}[2][\alpha]{\Pi_{(#1,#2)}}
\newcommand{\Da}[2][\alpha]{\Delta_{(#1,#2)}}
\newcommand{\Aa}[2][\alpha]{A_{(#1,#2)}}
\newcommand{\Ca}[2][\alpha]{C_{(#1,#2)}}
\newcommand{\Qa}[2][\alpha]{Q_{(#1,#2)}}
\newcommand{\da}[2][\alpha]{\delta_{(#1,#2)}}
\newcommand{\leqa}[2][\alpha]{\leq_{(#1,#2)}}
\newcommand{\lessa}[2][\alpha]{<_{(#1,#2)}}
\newcommand{\equiva}[2][\alpha]{\equiv_{(#1,#2)}}


\newcommand{\Sl}[1]{\Sa[\lambda]{#1}}
\newcommand{\Pl}[1]{\Pa[\lambda]{#1}}
\newcommand{\Dl}[1]{\Da[\lambda]{#1}}
\newcommand{\Al}[1]{\Aa[\lambda]{#1}}
\newcommand{\Cl}[1]{\Ca[\lambda]{#1}}
\newcommand{\Ql}[1]{\Qa[\lambda]{#1}}

\newcommand{\San}{\Sa{n}}
\newcommand{\Pan}{\Pa{n}}
\newcommand{\Dan}{\Da{n}}
\newcommand{\Can}{\Ca{n}}
\newcommand{\Qan}{\Qa{n}}
\newcommand{\Aan}{\Aa{n}}
\newcommand{\dan}{\da{n}}
\newcommand{\leqan}{\leqa{n}}
\newcommand{\lessan}{\lessa{n}}
\newcommand{\equivan}{\equiva{n}}

%%%%%%%%%%%%%%%%%%%%%%%%%%%%%%%%%%%%%%%%%%%%%
% Linear Algebra
%%%%%%%%%%%%%%%%%%%%%%%%%%%%%%%%%%%%%%%%%%%%%
\newcommand{\transpose}[1]{{#1}^{\text{T}}}
\newcommand{\norm}[1]{\lVert{#1}\rVert}
\newcommand\aug{\fboxsep=-\fboxrule\!\!\!\fbox{\strut}\!\!\!}

%%%%%%%%%%%%%%%%%%%%%%%%%%%%%%%%%%%%%%%%%%%%%
% Number Theory
%%%%%%%%%%%%%%%%%%%%%%%%%%%%%%%%%%%%%%%%%%%%%
\DeclareMathOperator{\Spec}{Spec}
\newcommand{\av}[1]{\lvert#1\rvert}
\DeclareMathOperator{\divides}{\mid}
\DeclareMathOperator{\ndivides}{\nmid}


\graphicspath{{images/}}

\newtheorem*{claim}{claim}
\newtheorem*{observation}{Observation}
\newtheorem*{warning}{Warning}
\newtheorem*{question}{Question}
\newtheorem{remark}[theorem]{Remark}

\newenvironment*{subproof}[1][Proof]
{\begin{proof}[#1]}{\renewcommand{\qedsymbol}{$\diamondsuit$} \end{proof}}

\mode<presentation>
{
  \usetheme{Singapore}
  % or ...

  \setbeamercovered{invisible}
  % or whatever (possibly just delete it)
}


\usepackage[english]{babel}
% or whatever

\usepackage[latin1]{inputenc}
% or whatever

\usepackage{times}
\usepackage[T1]{fontenc}
% Or whatever. Note that the encoding and the font should match. If T1
% does not look nice, try deleting the line with the fontenc.

\title{Lesson 9 \\ Congruence}
\subtitle{Math 310, Elementary Number Theory \\ Fall 2021 \\ SFSU}
\author{Mitch Rudominer}
\date{}



% If you wish to uncover everything in a step-wise fashion, uncomment
% the following command:

\beamerdefaultoverlayspecification{<+->}

\begin{document}

\begin{frame}
  \titlepage
\end{frame}

\begin{frame}{Congruence}

\begin{itemize}
  \item \textbf{Definition.} Let $a,b,m$ be integers with $m>1$.
  \item Then we say that $a$ is congruent to $b$ modulo $m$ iff $a\bmod m = b\bmod m$.
  \item We write this as $a\equiv b \pmod m$.
  \item If $a$ is not congruent to $b$ modulo $m$ we write $a\not\equiv b \pmod m$.
  \item \textbf{Example.} $13\bmod 5 = 3$.
  \item $33 \bmod 5 = 3$.
  \item So $13 \equiv 33 \pmod 5$.
  \item \textbf{Example.} $13 \bmod 7 = 6$
  \item $33 \bmod 7 = 5$.
  \item So $13 \not\equiv 33 \bmod 7$.
\end{itemize}

\end{frame}

\begin{frame}{Two different uses of ``mod''}

\begin{itemize}
  \item Notice the two different ways of using the symbol ``mod''.
  \item When we write $13\bmod 5 = 3$ the symbol ``mod'' is being used as a function.
  \item The expression ``$13\bmod 5$'' has a value that is an integer. (The value is 3.)
  \item When we write $13 \equiv 33 \pmod 5$ the pair of symbols $\equiv$ and ``mod'' are being
  used together to represent a relation.
  \item The expression ``$13 \equiv 33 \pmod 5$'' has a value that is either true or false. (The value is true.)
  \item You can always tell which meaning of ``mod'' is intended by checking for the $\equiv$ and
  also checking for parentheses around  ``mod'' next to a number.
\end{itemize}

\end{frame}

\begin{frame}{More examples ``mod''}

\begin{itemize}
  \item Give the value of each expression:
  \item \textbf{(a)} $16 \bmod 6$.
  \item \textbf{Answer:} 4
  \item \textbf{(b)} $26 \bmod 6$
  \item \textbf{Answer:} 2
  \item \textbf{(c)} $8 \bmod 6$.
  \item \textbf{Answer:} 2
  \item \textbf{(d)} $16 \equiv 26 \pmod 6$.
  \item \textbf{Answer:} False
  \item \textbf{(e)} $8 \equiv 26 \pmod 6$.
  \item \textbf{Answer:} True
\end{itemize}

\end{frame}

\begin{frame}{Slots in Blocks}

\begin{itemize}
  \item  You can visualize congruence in terms of ``slots in blocks.''
  \item Take congruence mod 3 for example.
  \item We can divide the integers into infinitely many blocks of size 3.
  \item $\singleton{0,1,2}$, $\singleton{3,4,5}$, $\singleton{6,7,8}$, $\cdots$
  \item Every integer is in exactly one block.
  \item For example 7 is in the block $\singleton{6,7,8}$.
  \item Each block has three slots: slot 0, slot 1, slot 2.
  \item 7 is in slot 1 of it's block. (i.e. $7\bmod 3 = 1$)
  \item 4 is also in slot 1 of it's block.
  \item So $4\equiv 7 \pmod 3$.
\end{itemize}

\end{frame}


\begin{frame}{Another characterization of congruence}

\begin{itemize}
  \item  Let $a,b,m$ be integers with $m>1$.
  \item Then $a\equiv b \pmod m$ iff $m \divides (a-b)$.
  \item Example: Suppose we want to know whether or not $26 \equiv 8 \pmod 6$.
  \item $26 - 8 = 18$.
  \item $6\divides 18$.
  \item So $26 \equiv 8 \pmod 6$.
  \item Proof of the new characterization:
  \item Write $a=q_1 m + r_1$, $b=q_2 m + r_2$, with $0\leq r_1,r_2 < m$.
  \item $a -b = (q_1-q_2) m + (r_1 - r_2)$.
  \item $a\equiv b \pmod m$ iff $r_1 = r_2$ iff $a-b$ is a multiple of $m$.
\end{itemize}

\end{frame}

\begin{frame}{Finding other things congruent to $a$.}

\begin{itemize}
  \item  Question: Suppose you have an integer $a$ and you want
  a different integer $b$ such that $a\equiv b \pmod m$.
  \item How would you go about finding $b$?
  \item Answer: You would add a multiple of $m$.
  \item Example: Find several integers $b$ such that $7\equiv b \pmod 3$.
  \item Solution: Try 10, 13, 16, 19.
  \item Why? Because $a\equiv b \pmod m$ iff $b=a + qm$ for some $q$.
\end{itemize}

\end{frame}

%%%%%%%%%%%%%%%%%%%%%%%%%%%%%%%%%%%%%%%%%%%%%%%%%%%%%%%%%%%%%%%%%%%%%%%%%%%%%%%%%%%


\begin{frame}{Congruence classes}

\begin{itemize}
  \item  Let $a,m$ be integers with $m>1$.
  \item The congruence class of $a$ mod $m$ is the set of all integers $b$ such that $a\equiv b \pmod m$.
  \item $[a]_m = \setof{b\in\Z}{a\equiv b \pmod m}$.
  \item Example. Find $[4]_7$.
  \item $[4]_7 = \singleton{\cdots -24, -17, -10, -3, 4, 11, 18, 25, 32, \cdots}$.
  \item Example. Find $[11]_7$.
  \item $[11]_7 = \singleton{\cdots -24, -17, -10, -3, 4, 11, 18, 25, 32, \cdots}$.
\end{itemize}

\end{frame}

\begin{frame}{Congruence is transitive}

\begin{itemize}
  \item  Let $a,b,c,m$ be integers with $m>1$.
  \item Suppose $a\equiv b \pmod m$ and $b \equiv c \pmod m$.
  \item Then $a\equiv c \pmod m$.
  \item Example: $1 \equiv 4 \pmod 3$.
  \item $4 \equiv 7 \pmod 3$.
  \item So $1 \equiv 7 \pmod 3$.
  \item Proof: If $a \bmod m = b \bmod m$
  \item and $b \bmod m = c \bmod m$,
  \item then $a\bmod m = c \bmod m$.
\end{itemize}

\end{frame}


\begin{frame}{Congruence is symmetric}

\begin{itemize}
  \item  Let $a,b,m$ be integers with $m>1$.
  \item Suppose $a\equiv b \pmod m$.
  \item Then $b\equiv a \pmod m$.
  \item Proof: If $a \bmod m = b \bmod m$,
  \item then $b\bmod m = a \bmod m$.
\end{itemize}

\end{frame}

\begin{frame}{Congruence is reflexive}

\begin{itemize}
  \item  Let $a, m$ be integers with $m>1$.
  \item Then $a\equiv a \pmod m$.
  \item Proof: $a \bmod m = a \bmod m$.
\end{itemize}

\end{frame}


\begin{frame}{Congruence is an equivalence relation}

\begin{itemize}
  \item  Congruence is transitive, symmetric and reflexive.
  \item This makes it an equivalence relation.
  \item This means that the congruence classes mod $m$ partition the integers into $m$ pieces.
  \item Example: There are three congruence classes mod 3.
  \item They partition the integers into three disjoint infinite sets.
  \item $[0]_3 = \singleton{\cdots -9, -6, -3, 0, 3, 6, 9,\cdots}$
  \item $[1]_3 = \singleton{\cdots -8, -5, -2, 1, 4, 7, 10,\cdots}$
  \item $[2]_3 = \singleton{\cdots -7, -4, -1, 2, 5, 8, 11,\cdots}$
  \item Every integer is in exactly one of those sets.
  \item In terms of slots in blocks, we have partitioned the integers into slot 0 integers,
  slot 1 integers, and slot 2 integers.
\end{itemize}

\end{frame}


\begin{frame}{Congruence vs Congruence Classes}

\begin{itemize}
  \item  $a\equiv b \bmod m$ iff $[a]_m = [b]_m$
  \item iff $a\in[b]_m$ iff $b\in[a]_m$.
  \item Example:
  \item $[0]_3 = \singleton{\cdots -9, -6, -3, 0, 3, 6, 9,\cdots} = [9]_3$
  \item $[1]_3 = \singleton{\cdots -8, -5, -2, 1, 4, 7, 10,\cdots} =[10]_3$
  \item $[2]_3 = \singleton{\cdots -7, -4, -1, 2, 5, 8, 11,\cdots} = [11]_3$
  \item $1 \equiv 10 \bmod 3$
  \item $[1]_3 = [10]_3$
  \item $10\in [1]_3$
  \item $1\in[10]_3$.
\end{itemize}

\end{frame}



%%%%%%%%%%%%%%%%%%%%%%%%%%%%%%%%%%%%%%%%%%%%%%%%%%%%%%%%%%%%%%%%%%%%%

\begin{frame}{A complete system of residues}

\begin{itemize}
  \item  A \emph{complete system of residues}  mod $m$ is a set of integers that
  chooses exactly one integer from each of the congruence classes mod $m$.
  \item Example. Give a complete system of residues mod $3$.
  \item $[0]_3 = \singleton{\cdots -9, -6, -3, 0, 3, 6, 9,\cdots}$
  \item $[1]_3 = \singleton{\cdots -8, -5, -2, 1, 4, 7, 10,\cdots}$
  \item $[2]_3 = \singleton{\cdots -7, -4, -1, 2, 5, 8, 11,\cdots}$
  \item Therefore each of the following are a complete system of residues mod 3.
  \item (a) $\singleton{0,1,2}$.
  \item (b) $\singleton{-3,-2,-1}$.
  \item (c) $\singleton{0,1,-1}$.
  \item (d) $\singleton{-9,7,5}$.
  \item In all cases the system consists of 3 mutually non-congruent integers.
\end{itemize}

\end{frame}

\begin{frame}{Least non-negative residues}

\begin{itemize}
  \item  The \emph{least non-negative residues} modulo $m$ is the complete set of residues
  consisting of the least non-negative integer from each congruence class.
  \item Example. Find the least non-negative residues mod $3$.
  \item $[0]_3 = \singleton{\cdots -9, -6, -3, 0, 3, 6, 9,\cdots}$
  \item $[1]_3 = \singleton{\cdots -8, -5, -2, 1, 4, 7, 10,\cdots}$
  \item $[2]_3 = \singleton{\cdots -7, -4, -1, 2, 5, 8, 11,\cdots}$
  \item Therefore the least non-negative residues modulo 3 are:
  \item $\singleton{0,1,2}$.
\end{itemize}

\end{frame}

\begin{frame}{Absolute least residues}

\begin{itemize}
  \item  The \emph{absolute least residues} modulo $m$ is the complete set of residues
  consisting of the element of each congruence class of smallest absolute value.
  \item Example. Find the absolute least residues mod $3$.
 \item $[0]_3 = \singleton{\cdots -9, -6, -3, 0, 3, 6, 9,\cdots}$
  \item $[1]_3 = \singleton{\cdots -8, -5, -2, 1, 4, 7, 10,\cdots}$
  \item $[2](2)_3 = \singleton{\cdots -7, -4, -1, 2, 5, 8, 11,\cdots}$
  \item Therefore the absolute residues modulo 3 are:
  \item $\singleton{0,1,-1}$.
  \item Usually we will write this as $\singleton{-1,0,1}$.
\end{itemize}

\end{frame}

\begin{frame}{Example}

\begin{itemize}
  \item  Give the set of least non-negative residues mod 5.
  \item Answer: $\singleton{0,1,2,3,4}$.
  \item Every integer is congruent mod 5 to exactly one of these.
  \item Give the set of absolute least residues mod 5.
  \item Answer: $\singleton{-2,-1,0,1,2}$.
  \item Every integer is congruent mod 5 to exactly one of these.
\end{itemize}

\end{frame}

%%%%%%%%%%%%%%%%%%%%%%%%%%%%%%%%%%%%%%%%%%%%%%%%%%%%%%%%%%%%%%%%%%%%%

\begin{frame}{Characterization of a complete system of residues}

\begin{itemize}
  \item  \textbf{Lemma} Let $m>1$ be an integer. Let $S$ be a finite set of integers.
  \item Then $S$ is a complete system of residues modulo $m$ iff
  \item (a) $S$ contains $m$ elements.
  \item (b) The elements of $S$ are mutually incongruent.
  \item \textbf{proof} If $S$ is a complete system of residues then it contains exactly
  one element from each of the $m$ congruence classes.
  \item So it contains $m$ mutually incongruent elements.
  \item Conversely, suppose $S$ contains $m$ mutually incongruent elements.
  \item Since the elements are mutually incongruent, they come from different congruence classes.
  \item Since there are $m$ of them they must represent all $m$ of the congruence classes. $\qed$.
\end{itemize}

\end{frame}

%%%%%%%%%%%%%%%%%%%%%%%%%%%%%%%%%%%%%%%%%%%%%%%%%%%%%%%%%%%%%%%%%%%%%

\begin{frame}{Example problem: Is a complete system of residues?}

\begin{itemize}
  \item  Example: Is $S=\singleton{0, -10, 31, 15}$ a complete system of residues modulo 6?
  \item Answer: No, it contains only 4 elements.
  \item  Example: Is $S=\singleton{0,-10,31,  15, 3, 11}$ a complete system of residues modulo 6?
  \item Answer: No, 15 and 3 are congruent modulo 6.
  \item Example: Is $S=\singleton{0, -10,31, 15, 22, 11}$ a complete system of residues modulo 6?
  \item Answer: Yes
\end{itemize}

\end{frame}

%%%%%%%%%%%%%%%%%%%%%%%%%%%%%%%%%%%%%%%%%%%%%%%%%%%%%%%%%%%%%%%%%%%%%

\begin{frame}{Consecutive Complete Systems}

\begin{itemize}
  \item  \textbf{Lemma} Let $m>1$ be an integer. Then every set of $m$ consecutive integers is
  a complete system of residues modulo $m$.
  \item \textbf{proof} Let $S=\singleton{x, x+1, x+2, \cdots ,x+(m-1)}$.
  \item Then every two elements of $S$ are non-congruent mod $m$ because their difference is less than $m$. $\qed$.
  \item Example: The following are all complete systems of residues mod 7
  \item $\singleton{14,15,16,17,18,19, 20}$
  \item $\singleton{-103, -102, -101, -100, -99, -98, -97}$.
  \item $\singleton{0,1,2,3,4,5, 6}$. (The least non-negative residues.)
  \item $\singleton{-3,-2,-1,0,1,2,3}$. (The absolute least residues.)
  \item It is useful to understand this in terms of slots and blocks.
  \item The \emph{standard} blocks are lined up so the slots yield the least non-negative residues.
  \item But we can consider non-standard blocks that are shifted relative to the standard blocks.
\end{itemize}

\end{frame}

%%%%%%%%%%%%%%%%%%%%%%%%%%%%%%%%%%%%%%%%%%%%%%%%%%%%%%%%%%%%%%%%%%%%%

\begin{frame}{The Residue Of an Integer}

\begin{itemize}
  \item  \textbf{Definition} Let $a, m \in\Z$ with $m>1$.
  \item The least non-negative residue of $a$ mod $m$ is just $a\mod m$.
  \item The absolute least residue of $a$ mod $m$ is the unique element $x$ of
  the set of absolute least residues mod $m$ such that $x\equiv a \pmod m$.
  \item Example: Find the least non-negative residue of 39 mod 7.
  \item Answer: 4
  \item Example: Find the absolute least residue of 39 mod 7.
  \item Answer: $-3$.
  \item Why: Because $-3\equiv 4 \equiv 39 \pmod 7$ and $-3$ is an absolute least residue mod 7.
  Note that 4 is not an absolute least residue mod 7.
\end{itemize}

\end{frame}



\end{document}
