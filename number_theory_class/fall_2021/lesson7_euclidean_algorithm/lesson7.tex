% $Header$

%\documentclass{beamer}
\documentclass[handout]{beamer}
\usepackage{amsmath,amssymb,latexsym,eucal,amsthm,graphicx,hyperref}
%%%%%%%%%%%%%%%%%%%%%%%%%%%%%%%%%%%%%%%%%%%%%
% Common Set Theory Constructs
%%%%%%%%%%%%%%%%%%%%%%%%%%%%%%%%%%%%%%%%%%%%%

\newcommand{\setof}[2]{\left\{ \, #1 \, \left| \, #2 \, \right.\right\}}
\newcommand{\lsetof}[2]{\left\{\left. \, #1 \, \right| \, #2 \,  \right\}}
\newcommand{\bigsetof}[2]{\bigl\{ \, #1 \, \bigm | \, #2 \,\bigr\}}
\newcommand{\Bigsetof}[2]{\Bigl\{ \, #1 \, \Bigm | \, #2 \,\Bigr\}}
\newcommand{\biggsetof}[2]{\biggl\{ \, #1 \, \biggm | \, #2 \,\biggr\}}
\newcommand{\Biggsetof}[2]{\Biggl\{ \, #1 \, \Biggm | \, #2 \,\Biggr\}}
\newcommand{\dotsetof}[2]{\left\{ \, #1 \, : \, #2 \, \right\}}
\newcommand{\bigdotsetof}[2]{\bigl\{ \, #1 \, : \, #2 \,\bigr\}}
\newcommand{\Bigdotsetof}[2]{\Bigl\{ \, #1 \, \Bigm : \, #2 \,\Bigr\}}
\newcommand{\biggdotsetof}[2]{\biggl\{ \, #1 \, \biggm : \, #2 \,\biggr\}}
\newcommand{\Biggdotsetof}[2]{\Biggl\{ \, #1 \, \Biggm : \, #2 \,\Biggr\}}
\newcommand{\sequence}[2]{\left\langle \, #1 \,\left| \, #2 \, \right. \right\rangle}
\newcommand{\lsequence}[2]{\left\langle\left. \, #1 \, \right| \,#2 \,  \right\rangle}
\newcommand{\bigsequence}[2]{\bigl\langle \,#1 \, \bigm | \, #2 \, \bigr\rangle}
\newcommand{\Bigsequence}[2]{\Bigl\langle \,#1 \, \Bigm | \, #2 \, \Bigr\rangle}
\newcommand{\biggsequence}[2]{\biggl\langle \,#1 \, \biggm | \, #2 \, \biggr\rangle}
\newcommand{\Biggsequence}[2]{\Biggl\langle \,#1 \, \Biggm | \, #2 \, \Biggr\rangle}
\newcommand{\singleton}[1]{\left\{#1\right\}}
\newcommand{\angles}[1]{\left\langle #1 \right\rangle}
\newcommand{\bigangles}[1]{\bigl\langle #1 \bigr\rangle}
\newcommand{\Bigangles}[1]{\Bigl\langle #1 \Bigr\rangle}
\newcommand{\biggangles}[1]{\biggl\langle #1 \biggr\rangle}
\newcommand{\Biggangles}[1]{\Biggl\langle #1 \Biggr\rangle}


\newcommand{\force}[1]{\Vert\!\underset{\!\!\!\!\!#1}{\!\!\!\relbar\!\!\!%
\relbar\!\!\relbar\!\!\relbar\!\!\!\relbar\!\!\relbar\!\!\relbar\!\!\!%
\relbar\!\!\relbar\!\!\relbar}}
\newcommand{\longforce}[1]{\Vert\!\underset{\!\!\!\!\!#1}{\!\!\!\relbar\!\!\!%
\relbar\!\!\relbar\!\!\relbar\!\!\!\relbar\!\!\relbar\!\!\relbar\!\!\!%
\relbar\!\!\relbar\!\!\relbar\!\!\relbar\!\!\relbar\!\!\relbar\!\!\relbar\!\!\relbar}}
\newcommand{\nforce}[1]{\Vert\!\underset{\!\!\!\!\!#1}{\!\!\!\relbar\!\!\!%
\relbar\!\!\relbar\!\!\relbar\!\!\!\relbar\!\!\relbar\!\!\relbar\!\!\!%
\relbar\!\!\not\relbar\!\!\relbar}}
\newcommand{\forcein}[2]{\overset{#2}{\Vert\underset{\!\!\!\!\!#1}%
{\!\!\!\relbar\!\!\!\relbar\!\!\relbar\!\!\relbar\!\!\!\relbar\!\!\relbar\!%
\!\relbar\!\!\!\relbar\!\!\relbar\!\!\relbar\!\!\relbar\!\!\!\relbar\!\!%
\relbar\!\!\relbar}}}

\newcommand{\pre}[2]{{}^{#2}{#1}}

\newcommand{\restr}{\!\!\upharpoonright\!}

%%%%%%%%%%%%%%%%%%%%%%%%%%%%%%%%%%%%%%%%%%%%%
% Set-Theoretic Connectives
%%%%%%%%%%%%%%%%%%%%%%%%%%%%%%%%%%%%%%%%%%%%%

\newcommand{\intersect}{\cap}
\newcommand{\union}{\cup}
\newcommand{\Intersection}[1]{\bigcap\limits_{#1}}
\newcommand{\Union}[1]{\bigcup\limits_{#1}}
\newcommand{\adjoin}{{}^\frown}
\newcommand{\forces}{\Vdash}

%%%%%%%%%%%%%%%%%%%%%%%%%%%%%%%%%%%%%%%%%%%%%
% Miscellaneous
%%%%%%%%%%%%%%%%%%%%%%%%%%%%%%%%%%%%%%%%%%%%%
\newcommand{\defeq}{=_{\text{def}}}
\newcommand{\Turingleq}{\leq_{\text{T}}}
\newcommand{\Turingless}{<_{\text{T}}}
\newcommand{\lexleq}{\leq_{\text{lex}}}
\newcommand{\lexless}{<_{\text{lex}}}
\newcommand{\Turingequiv}{\equiv_{\text{T}}}
\newcommand{\isomorphic}{\cong}

%%%%%%%%%%%%%%%%%%%%%%%%%%%%%%%%%%%%%%%%%%%%%
% Constants
%%%%%%%%%%%%%%%%%%%%%%%%%%%%%%%%%%%%%%%%%%%%%
\newcommand{\R}{\mathbb{R}}
\renewcommand{\P}{\mathbb{P}}
\newcommand{\Q}{\mathbb{Q}}
\newcommand{\Z}{\mathbb{Z}}
\newcommand{\Zpos}{\Z^{+}}
\newcommand{\Znonneg}{\Z^{\geq 0}}
\newcommand{\C}{\mathbb{C}}
\newcommand{\N}{\mathbb{N}}
\newcommand{\B}{\mathbb{B}}
\newcommand{\Bairespace}{\pre{\omega}{\omega}}
\newcommand{\LofR}{L(\R)}
\newcommand{\JofR}[1]{J_{#1}(\R)}
\newcommand{\SofR}[1]{S_{#1}(\R)}
\newcommand{\JalphaR}{\JofR{\alpha}}
\newcommand{\JbetaR}{\JofR{\beta}}
\newcommand{\JlambdaR}{\JofR{\lambda}}
\newcommand{\SalphaR}{\SofR{\alpha}}
\newcommand{\SbetaR}{\SofR{\beta}}
\newcommand{\Pkl}{\mathcal{P}_{\kappa}(\lambda)}
\DeclareMathOperator{\con}{con}
\DeclareMathOperator{\ORD}{OR}
\DeclareMathOperator{\Ord}{OR}
\DeclareMathOperator{\WO}{WO}
\DeclareMathOperator{\OD}{OD}
\DeclareMathOperator{\HOD}{HOD}
\DeclareMathOperator{\HC}{HC}
\DeclareMathOperator{\WF}{WF}
\DeclareMathOperator{\wfp}{wfp}
\DeclareMathOperator{\HF}{HF}
\newcommand{\One}{I}
\newcommand{\ONE}{I}
\newcommand{\Two}{II}
\newcommand{\TWO}{II}
\newcommand{\Mladder}{M^{\text{ld}}}

%%%%%%%%%%%%%%%%%%%%%%%%%%%%%%%%%%%%%%%%%%%%%
% Commutative Algebra Constants
%%%%%%%%%%%%%%%%%%%%%%%%%%%%%%%%%%%%%%%%%%%%%
\DeclareMathOperator{\dottimes}{\dot{\times}}
\DeclareMathOperator{\dotminus}{\dot{-}}
\DeclareMathOperator{\Spec}{Spec}

%%%%%%%%%%%%%%%%%%%%%%%%%%%%%%%%%%%%%%%%%%%%%
% Theories
%%%%%%%%%%%%%%%%%%%%%%%%%%%%%%%%%%%%%%%%%%%%%
\DeclareMathOperator{\ZFC}{ZFC}
\DeclareMathOperator{\ZF}{ZF}
\DeclareMathOperator{\AD}{AD}
\DeclareMathOperator{\ADR}{AD_{\R}}
\DeclareMathOperator{\KP}{KP}
\DeclareMathOperator{\PD}{PD}
\DeclareMathOperator{\CH}{CH}
\DeclareMathOperator{\GCH}{GCH}
\DeclareMathOperator{\HPC}{HPC} % HOD pair capturing
%%%%%%%%%%%%%%%%%%%%%%%%%%%%%%%%%%%%%%%%%%%%%
% Iteration Trees
%%%%%%%%%%%%%%%%%%%%%%%%%%%%%%%%%%%%%%%%%%%%%

\newcommand{\pred}{\text{-pred}}

%%%%%%%%%%%%%%%%%%%%%%%%%%%%%%%%%%%%%%%%%%%%%%%%
% Operator Names
%%%%%%%%%%%%%%%%%%%%%%%%%%%%%%%%%%%%%%%%%%%%%%%%
\DeclareMathOperator{\Det}{Det}
\DeclareMathOperator{\dom}{dom}
\DeclareMathOperator{\ran}{ran}
\DeclareMathOperator{\range}{ran}
\DeclareMathOperator{\image}{image}
\DeclareMathOperator{\crit}{crit}
\DeclareMathOperator{\card}{card}
\DeclareMathOperator{\cf}{cf}
\DeclareMathOperator{\cof}{cof}
\DeclareMathOperator{\rank}{rank}
\DeclareMathOperator{\ot}{o.t.}
\DeclareMathOperator{\ords}{o}
\DeclareMathOperator{\ro}{r.o.}
\DeclareMathOperator{\rud}{rud}
\DeclareMathOperator{\Powerset}{\mathcal{P}}
\DeclareMathOperator{\length}{lh}
\DeclareMathOperator{\lh}{lh}
\DeclareMathOperator{\limit}{lim}
\DeclareMathOperator{\fld}{fld}
\DeclareMathOperator{\projection}{p}
\DeclareMathOperator{\Ult}{Ult}
\DeclareMathOperator{\ULT}{Ult}
\DeclareMathOperator{\Coll}{Coll}
\DeclareMathOperator{\Col}{Col}
\DeclareMathOperator{\Hull}{Hull}
\DeclareMathOperator*{\dirlim}{dir lim}
\DeclareMathOperator{\Scale}{Scale}
\DeclareMathOperator{\supp}{supp}
\DeclareMathOperator{\trancl}{tran.cl.}
\DeclareMathOperator{\trace}{Tr}
\DeclareMathOperator{\diag}{diag}
\DeclareMathOperator{\spn}{span}
\DeclareMathOperator{\sgn}{sgn}
%%%%%%%%%%%%%%%%%%%%%%%%%%%%%%%%%%%%%%%%%%%%%
% Logical Connectives
%%%%%%%%%%%%%%%%%%%%%%%%%%%%%%%%%%%%%%%%%%%%%
\newcommand{\IImplies}{\Longrightarrow}
\newcommand{\SkipImplies}{\quad\Longrightarrow\quad}
\newcommand{\Ifff}{\Longleftrightarrow}
\newcommand{\iimplies}{\longrightarrow}
\newcommand{\ifff}{\longleftrightarrow}
\newcommand{\Implies}{\Rightarrow}
\newcommand{\Iff}{\Leftrightarrow}
\renewcommand{\implies}{\rightarrow}
\renewcommand{\iff}{\leftrightarrow}
\newcommand{\AND}{\wedge}
\newcommand{\OR}{\vee}
\newcommand{\st}{\text{ s.t. }}
\newcommand{\Or}{\text{ or }}

%%%%%%%%%%%%%%%%%%%%%%%%%%%%%%%%%%%%%%%%%%%%%
% Function Arrows
%%%%%%%%%%%%%%%%%%%%%%%%%%%%%%%%%%%%%%%%%%%%%

\newcommand{\injection}{\xrightarrow{\text{1-1}}}
\newcommand{\surjection}{\xrightarrow{\text{onto}}}
\newcommand{\bijection}{\xrightarrow[\text{onto}]{\text{1-1}}}
\newcommand{\cofmap}{\xrightarrow{\text{cof}}}
\newcommand{\map}{\rightarrow}

%%%%%%%%%%%%%%%%%%%%%%%%%%%%%%%%%%%%%%%%%%%%%
% Mouse Comparison Operators
%%%%%%%%%%%%%%%%%%%%%%%%%%%%%%%%%%%%%%%%%%%%%
\newcommand{\initseg}{\trianglelefteq}
\newcommand{\properseg}{\lhd}
\newcommand{\notinitseg}{\ntrianglelefteq}
\newcommand{\notproperseg}{\ntriangleleft}

%%%%%%%%%%%%%%%%%%%%%%%%%%%%%%%%%%%%%%%%%%%%%
%           calligraphic letters
%%%%%%%%%%%%%%%%%%%%%%%%%%%%%%%%%%%%%%%%%%%%%
\newcommand{\cA}{\mathcal{A}}
\newcommand{\cB}{\mathcal{B}}
\newcommand{\cC}{\mathcal{C}}
\newcommand{\cD}{\mathcal{D}}
\newcommand{\cE}{\mathcal{E}}
\newcommand{\cF}{\mathcal{F}}
\newcommand{\cG}{\mathcal{G}}
\newcommand{\cH}{\mathcal{H}}
\newcommand{\cI}{\mathcal{I}}
\newcommand{\cJ}{\mathcal{J}}
\newcommand{\cK}{\mathcal{K}}
\newcommand{\cL}{\mathcal{L}}
\newcommand{\cM}{\mathcal{M}}
\newcommand{\cN}{\mathcal{N}}
\newcommand{\cO}{\mathcal{O}}
\newcommand{\cP}{\mathcal{P}}
\newcommand{\cQ}{\mathcal{Q}}
\newcommand{\cR}{\mathcal{R}}
\newcommand{\cS}{\mathcal{S}}
\newcommand{\cT}{\mathcal{T}}
\newcommand{\cU}{\mathcal{U}}
\newcommand{\cV}{\mathcal{V}}
\newcommand{\cW}{\mathcal{W}}
\newcommand{\cX}{\mathcal{X}}
\newcommand{\cY}{\mathcal{Y}}
\newcommand{\cZ}{\mathcal{Z}}


%%%%%%%%%%%%%%%%%%%%%%%%%%%%%%%%%%%%%%%%%%%%%
%          Primed Letters
%%%%%%%%%%%%%%%%%%%%%%%%%%%%%%%%%%%%%%%%%%%%%
\newcommand{\aprime}{a^{\prime}}
\newcommand{\bprime}{b^{\prime}}
\newcommand{\cprime}{c^{\prime}}
\newcommand{\dprime}{d^{\prime}}
\newcommand{\eprime}{e^{\prime}}
\newcommand{\fprime}{f^{\prime}}
\newcommand{\gprime}{g^{\prime}}
\newcommand{\hprime}{h^{\prime}}
\newcommand{\iprime}{i^{\prime}}
\newcommand{\jprime}{j^{\prime}}
\newcommand{\kprime}{k^{\prime}}
\newcommand{\lprime}{l^{\prime}}
\newcommand{\mprime}{m^{\prime}}
\newcommand{\nprime}{n^{\prime}}
\newcommand{\oprime}{o^{\prime}}
\newcommand{\pprime}{p^{\prime}}
\newcommand{\qprime}{q^{\prime}}
\newcommand{\rprime}{r^{\prime}}
\newcommand{\sprime}{s^{\prime}}
\newcommand{\tprime}{t^{\prime}}
\newcommand{\uprime}{u^{\prime}}
\newcommand{\vprime}{v^{\prime}}
\newcommand{\wprime}{w^{\prime}}
\newcommand{\xprime}{x^{\prime}}
\newcommand{\yprime}{y^{\prime}}
\newcommand{\zprime}{z^{\prime}}
\newcommand{\Aprime}{A^{\prime}}
\newcommand{\Bprime}{B^{\prime}}
\newcommand{\Cprime}{C^{\prime}}
\newcommand{\Dprime}{D^{\prime}}
\newcommand{\Eprime}{E^{\prime}}
\newcommand{\Fprime}{F^{\prime}}
\newcommand{\Gprime}{G^{\prime}}
\newcommand{\Hprime}{H^{\prime}}
\newcommand{\Iprime}{I^{\prime}}
\newcommand{\Jprime}{J^{\prime}}
\newcommand{\Kprime}{K^{\prime}}
\newcommand{\Lprime}{L^{\prime}}
\newcommand{\Mprime}{M^{\prime}}
\newcommand{\Nprime}{N^{\prime}}
\newcommand{\Oprime}{O^{\prime}}
\newcommand{\Pprime}{P^{\prime}}
\newcommand{\Qprime}{Q^{\prime}}
\newcommand{\Rprime}{R^{\prime}}
\newcommand{\Sprime}{S^{\prime}}
\newcommand{\Tprime}{T^{\prime}}
\newcommand{\Uprime}{U^{\prime}}
\newcommand{\Vprime}{V^{\prime}}
\newcommand{\Wprime}{W^{\prime}}
\newcommand{\Xprime}{X^{\prime}}
\newcommand{\Yprime}{Y^{\prime}}
\newcommand{\Zprime}{Z^{\prime}}
\newcommand{\alphaprime}{\alpha^{\prime}}
\newcommand{\betaprime}{\beta^{\prime}}
\newcommand{\gammaprime}{\gamma^{\prime}}
\newcommand{\Gammaprime}{\Gamma^{\prime}}
\newcommand{\deltaprime}{\delta^{\prime}}
\newcommand{\epsilonprime}{\epsilon^{\prime}}
\newcommand{\kappaprime}{\kappa^{\prime}}
\newcommand{\lambdaprime}{\lambda^{\prime}}
\newcommand{\rhoprime}{\rho^{\prime}}
\newcommand{\Sigmaprime}{\Sigma^{\prime}}
\newcommand{\tauprime}{\tau^{\prime}}
\newcommand{\xiprime}{\xi^{\prime}}
\newcommand{\thetaprime}{\theta^{\prime}}
\newcommand{\Omegaprime}{\Omega^{\prime}}
\newcommand{\cMprime}{\cM^{\prime}}
\newcommand{\cNprime}{\cN^{\prime}}
\newcommand{\cPprime}{\cP^{\prime}}
\newcommand{\cQprime}{\cQ^{\prime}}
\newcommand{\cRprime}{\cR^{\prime}}
\newcommand{\cSprime}{\cS^{\prime}}
\newcommand{\cTprime}{\cT^{\prime}}

%%%%%%%%%%%%%%%%%%%%%%%%%%%%%%%%%%%%%%%%%%%%%
%          bar Letters
%%%%%%%%%%%%%%%%%%%%%%%%%%%%%%%%%%%%%%%%%%%%%
\newcommand{\abar}{\bar{a}}
\newcommand{\bbar}{\bar{b}}
\newcommand{\cbar}{\bar{c}}
\newcommand{\ibar}{\bar{i}}
\newcommand{\jbar}{\bar{j}}
\newcommand{\nbar}{\bar{n}}
\newcommand{\xbar}{\bar{x}}
\newcommand{\ybar}{\bar{y}}
\newcommand{\zbar}{\bar{z}}
\newcommand{\pibar}{\bar{\pi}}
\newcommand{\phibar}{\bar{\varphi}}
\newcommand{\psibar}{\bar{\psi}}
\newcommand{\thetabar}{\bar{\theta}}
\newcommand{\nubar}{\bar{\nu}}

%%%%%%%%%%%%%%%%%%%%%%%%%%%%%%%%%%%%%%%%%%%%%
%          star Letters
%%%%%%%%%%%%%%%%%%%%%%%%%%%%%%%%%%%%%%%%%%%%%
\newcommand{\phistar}{\phi^{*}}
\newcommand{\Mstar}{M^{*}}

%%%%%%%%%%%%%%%%%%%%%%%%%%%%%%%%%%%%%%%%%%%%%
%          dotletters Letters
%%%%%%%%%%%%%%%%%%%%%%%%%%%%%%%%%%%%%%%%%%%%%
\newcommand{\Gdot}{\dot{G}}

%%%%%%%%%%%%%%%%%%%%%%%%%%%%%%%%%%%%%%%%%%%%%
%         check Letters
%%%%%%%%%%%%%%%%%%%%%%%%%%%%%%%%%%%%%%%%%%%%%
\newcommand{\deltacheck}{\check{\delta}}
\newcommand{\gammacheck}{\check{\gamma}}


%%%%%%%%%%%%%%%%%%%%%%%%%%%%%%%%%%%%%%%%%%%%%
%          Formulas
%%%%%%%%%%%%%%%%%%%%%%%%%%%%%%%%%%%%%%%%%%%%%

\newcommand{\formulaphi}{\text{\large $\varphi$}}
\newcommand{\Formulaphi}{\text{\Large $\varphi$}}


%%%%%%%%%%%%%%%%%%%%%%%%%%%%%%%%%%%%%%%%%%%%%
%          Fraktur Letters
%%%%%%%%%%%%%%%%%%%%%%%%%%%%%%%%%%%%%%%%%%%%%

\newcommand{\fa}{\mathfrak{a}}
\newcommand{\fb}{\mathfrak{b}}
\newcommand{\fc}{\mathfrak{c}}
\newcommand{\fk}{\mathfrak{k}}
\newcommand{\fp}{\mathfrak{p}}
\newcommand{\fq}{\mathfrak{q}}
\newcommand{\fr}{\mathfrak{r}}
\newcommand{\fA}{\mathfrak{A}}
\newcommand{\fB}{\mathfrak{B}}
\newcommand{\fC}{\mathfrak{C}}
\newcommand{\fD}{\mathfrak{D}}

%%%%%%%%%%%%%%%%%%%%%%%%%%%%%%%%%%%%%%%%%%%%%
%          Bold Letters
%%%%%%%%%%%%%%%%%%%%%%%%%%%%%%%%%%%%%%%%%%%%%
\newcommand{\ba}{\mathbf{a}}
\newcommand{\bb}{\mathbf{b}}
\newcommand{\bc}{\mathbf{c}}
\newcommand{\bd}{\mathbf{d}}
\newcommand{\be}{\mathbf{e}}
\newcommand{\bu}{\mathbf{u}}
\newcommand{\bv}{\mathbf{v}}
\newcommand{\bw}{\mathbf{w}}
\newcommand{\bx}{\mathbf{x}}
\newcommand{\by}{\mathbf{y}}
\newcommand{\bz}{\mathbf{z}}
\newcommand{\bSigma}{\boldsymbol{\Sigma}}
\newcommand{\bPi}{\boldsymbol{\Pi}}
\newcommand{\bDelta}{\boldsymbol{\Delta}}
\newcommand{\bdelta}{\boldsymbol{\delta}}
\newcommand{\bgamma}{\boldsymbol{\gamma}}
\newcommand{\bGamma}{\boldsymbol{\Gamma}}

%%%%%%%%%%%%%%%%%%%%%%%%%%%%%%%%%%%%%%%%%%%%%
%         Bold numbers
%%%%%%%%%%%%%%%%%%%%%%%%%%%%%%%%%%%%%%%%%%%%%
\newcommand{\bzero}{\mathbf{0}}

%%%%%%%%%%%%%%%%%%%%%%%%%%%%%%%%%%%%%%%%%%%%%
% Projective-Like Pointclasses
%%%%%%%%%%%%%%%%%%%%%%%%%%%%%%%%%%%%%%%%%%%%%
\newcommand{\Sa}[2][\alpha]{\Sigma_{(#1,#2)}}
\newcommand{\Pa}[2][\alpha]{\Pi_{(#1,#2)}}
\newcommand{\Da}[2][\alpha]{\Delta_{(#1,#2)}}
\newcommand{\Aa}[2][\alpha]{A_{(#1,#2)}}
\newcommand{\Ca}[2][\alpha]{C_{(#1,#2)}}
\newcommand{\Qa}[2][\alpha]{Q_{(#1,#2)}}
\newcommand{\da}[2][\alpha]{\delta_{(#1,#2)}}
\newcommand{\leqa}[2][\alpha]{\leq_{(#1,#2)}}
\newcommand{\lessa}[2][\alpha]{<_{(#1,#2)}}
\newcommand{\equiva}[2][\alpha]{\equiv_{(#1,#2)}}


\newcommand{\Sl}[1]{\Sa[\lambda]{#1}}
\newcommand{\Pl}[1]{\Pa[\lambda]{#1}}
\newcommand{\Dl}[1]{\Da[\lambda]{#1}}
\newcommand{\Al}[1]{\Aa[\lambda]{#1}}
\newcommand{\Cl}[1]{\Ca[\lambda]{#1}}
\newcommand{\Ql}[1]{\Qa[\lambda]{#1}}

\newcommand{\San}{\Sa{n}}
\newcommand{\Pan}{\Pa{n}}
\newcommand{\Dan}{\Da{n}}
\newcommand{\Can}{\Ca{n}}
\newcommand{\Qan}{\Qa{n}}
\newcommand{\Aan}{\Aa{n}}
\newcommand{\dan}{\da{n}}
\newcommand{\leqan}{\leqa{n}}
\newcommand{\lessan}{\lessa{n}}
\newcommand{\equivan}{\equiva{n}}

\newcommand{\SigmaOneOmega}{\Sigma^1_{\omega}}
\newcommand{\SigmaOneOmegaPlusOne}{\Sigma^1_{\omega+1}}
\newcommand{\PiOneOmega}{\Pi^1_{\omega}}
\newcommand{\PiOneOmegaPlusOne}{\Pi^1_{\omega+1}}
\newcommand{\DeltaOneOmegaPlusOne}{\Delta^1_{\omega+1}}

%%%%%%%%%%%%%%%%%%%%%%%%%%%%%%%%%%%%%%%%%%%%%
% Linear Algebra
%%%%%%%%%%%%%%%%%%%%%%%%%%%%%%%%%%%%%%%%%%%%%
\newcommand{\transpose}[1]{{#1}^{\text{T}}}
\newcommand{\norm}[1]{\lVert{#1}\rVert}
\newcommand\aug{\fboxsep=-\fboxrule\!\!\!\fbox{\strut}\!\!\!}

%%%%%%%%%%%%%%%%%%%%%%%%%%%%%%%%%%%%%%%%%%%%%
% Number Theory
%%%%%%%%%%%%%%%%%%%%%%%%%%%%%%%%%%%%%%%%%%%%%
\newcommand{\av}[1]{\lvert#1\rvert}
\DeclareMathOperator{\divides}{\mid}
\DeclareMathOperator{\ndivides}{\nmid}
\DeclareMathOperator{\lcm}{lcm}
\DeclareMathOperator{\sign}{sign}
\newcommand{\floor}[1]{\left\lfloor{#1}\right\rfloor}
\DeclareMathOperator{\ConCl}{CC}
\DeclareMathOperator{\ord}{ord}



\graphicspath{{images/}}

\newtheorem*{claim}{claim}
\newtheorem*{observation}{Observation}
\newtheorem*{warning}{Warning}
\newtheorem*{question}{Question}
\newtheorem{remark}[theorem]{Remark}

\newenvironment*{subproof}[1][Proof]
{\begin{proof}[#1]}{\renewcommand{\qedsymbol}{$\diamondsuit$} \end{proof}}

\mode<presentation>
{
  \usetheme{Singapore}
  % or ...

  \setbeamercovered{invisible}
  % or whatever (possibly just delete it)
}


\usepackage[english]{babel}
% or whatever

\usepackage[latin1]{inputenc}
% or whatever

\usepackage{times}
\usepackage[T1]{fontenc}
% Or whatever. Note that the encoding and the font should match. If T1
% does not look nice, try deleting the line with the fontenc.

\title{Lesson 7 \\ The Euclidean Algorithm}
\subtitle{Math 310, Elementary Number Theory \\ Fall 2021 \\ SFSU}
\author{Mitch Rudominer}
\date{}



% If you wish to uncover everything in a step-wise fashion, uncomment
% the following command:

\beamerdefaultoverlayspecification{<+->}

\begin{document}

\begin{frame}
  \titlepage
\end{frame}

\begin{frame}{An Efficient Algorithm for GCD}

\begin{itemize}
  \item Given two positive integers $a,b$ how do you compute $\gcd(a,b)$? How could a computer do it?
  \item One way: First compute the prime factorizations of $a$ and $b$ and then use the minimum exponent rule.
  \item Example: Find $\gcd(72, 108)$.
  \item $72=2^3\cdot 3^2$, $108=2^2\cdot 3^3$.
  \item $\gcd(72,108) = 2^2\cdot 3^2 = 36$.
  \item The problem is that finding the prime factorization of a large integer is intractable.
  \item That algorithm will not work to find the GCD of two 200-digit integers.
  \item But there is a different algorithm that allows us to efficiently compute GCD of two large integers.
  \item Example: \href{https://colab.research.google.com/drive/1itrO4ePMS6PxUmBhbTBumrgoafyIglD7}{Use Colab to find the GCD of two 200-digit integers.}
\end{itemize}

\end{frame}

\begin{frame}{Key Reduction Step}

\begin{itemize}
  \item The key ingredient of the efficient algorithm is the following
  \item \textbf{Lemma} Suppose $a,b$ are integers with $a>b>1$.
  \item Write $a = bq + r$ with $0\leq r < b$.
  \item Then $\gcd(a,b) = \gcd(b,r)$.
  \item Example: $75 = 20\cdot 3 + 15$.
  \item So $\gcd(75, 20) = \gcd(20, 15)$.
  \item \textbf{proof.} Notice that $r$ is an integer linear combination of
  $a$ and $b$. So any common divisor of $a$ and $b$ is a common divisor of $b$ and $r$.
  \item Also $a$ is an integer linear combination of $b$ and $r$ so any common divisor
  of $b$ and $r$ is a common divisor of $a$ and $b$.
  \item So the pair $a,b$ has the same set of common divisors as the pair $b,r$.
  \item So $\gcd(a,b)=\gcd(b,r)$. $\qed$.
\end{itemize}

\end{frame}

\begin{frame}{The Euclidean Algorithm}

\begin{itemize}
  \item We turn the key reduction step into an algorithm by repeating it.
  \item Example: Find $\gcd(75, 20)$.
  \item $75 = 20\cdot 3 + 15$ so $\gcd(75, 20) = \gcd(20, 15)$.
  \item $20 = 15 \cdot 1 + 5$ so $\gcd(20, 15) = \gcd(15, 5)$.
  \item $15 = 5 \cdot 3 + 0$ so $\gcd(15, 5) = \gcd(5, 0) = 5$.
  \item So $\gcd(75, 20) = 5$.
\end{itemize}

\end{frame}

\begin{frame}{The Euclidean Algorithm: General Procedure}

\begin{itemize}
  \item Let $a,b$ be positive integers with $a>b$.
  \item To find $\gcd(a,b)$ we do the following.
  \item Let $r_0 = a, r_1 = b$.
  \item At stage $j$ of the algorithm, with $j\geq 0$, we will
  have two integers $r_j, r_{j+1}$ with $r_j > r_{j+1} > 0$.
  \item Use Quotient-Remainder Theorem to write $r_j=r_{j+1} \cdot q_{j+1} + r_{j+2}$.
  \item $0\leq r_{j+2} < r_{j+1}$.
  \item For example with $j=0$, $r_0 = r_{1} \cdot q_{1} + r_2$, and $0\leq r_2 < r_1$.
  \item $\gcd(r_j, r_{j+1}) = \gcd(r_{j+1}, r_{j+2})$.
  \item Because the $r_j$ are decreasing, eventually we must reach a stage $j$ such that $r_{j+2} = 0$.
  \item Then $\gcd(r_{j+1}, r_{j+2}) = r_{j+1}$ and we are done.
\end{itemize}

\end{frame}

\begin{frame}{The Euclidean Algorithm: Example General Procedure}

\begin{itemize}
  \item Example: Find $\gcd(a, b)$ with $a=75, b=20$.
  \item $r_0 = 75, r_1 = 20$.
  \item $75 = 20\cdot 3 + 15$.
  \item $q_1 = 3, r_2 = 15$.
  \item $r_0 = r_1 \cdot q_1 + r_2$. $\gcd(r_0, r_1) = \gcd(r_1, r_2)$.
  \item $20 = 15 \cdot 1 + 5$.
  \item $q_2 = 1, r_3 = 5$.
  \item $r_1 = r_2 \cdot q_2 + r_3$. $\gcd(r_1, r_2) = \gcd(r_2, r_3)$.
  \item $15 = 5 \cdot 3 + 0$.
  \item $r_2 = r_3 \cdot q_3 + r_4$. $r_4 = 0$.
  \item So $\gcd(a,b) = r_3 = 5$.
\end{itemize}

\end{frame}


\begin{frame}{Euclidean Algorithm in table form}

Example: Find $\gcd(a, b)$ with $a=75, b=20$.

\begin{tabular}{|c|c|c|c|c|}\hline
$j$   &  $r_{j}$    & $r_{j+1}$ & $q_{j+1}$ & $r_{j+2}$ \\ \hline\hline
0     &  75         &  20       &    3      &   15      \\ \hline
1     &             &           &           &           \\ \hline
2     &             &           &           &           \\ \hline
\end{tabular}

\end{frame}

\begin{frame}{Euclidean Algorithm in table form, $j=1$}

\begin{tabular}{|c|c|c|c|c|}\hline
$j$   &  $r_{j}$    & $r_{j+1}$ & $q_{j+1}$ & $r_{j+2}$ \\ \hline\hline
0     &  75         &  20       &    3      &   15      \\ \hline
1     &  20         &  15       &    1      &   5       \\ \hline
2     &             &           &           &           \\ \hline
\end{tabular}

\end{frame}

\begin{frame}{Euclidean Algorithm in table form, $j=2$}

\begin{tabular}{|c|c|c|c|c|}\hline
$j$   &  $r_{j}$    & $r_{j+1}$ & $q_{j+1}$ & $r_{j+2}$ \\ \hline\hline
0     &  75         &  20       &    3      &   15      \\ \hline
1     &  20         &  15       &    1      &   5       \\ \hline
2     &  15         &  5        &    3      &   0       \\ \hline
\end{tabular}

\vspace{0.5in}

So $\gcd(75, 20) = 5$.

\end{frame}

\begin{frame}{Another example}

\begin{itemize}
\item Find $\gcd(51, 87)$.
\item Let $a=87, b=51$. We want $a>b$.
\end{itemize}

\pause

\begin{tabular}{|c|c|c|c|c|}\hline
$j$   &  $r_{j}$    & $r_{j+1}$ & $q_{j+1}$ & $r_{j+2}$ \\ \hline\hline
0     &  87         &  51       &    1      &   36      \\ \hline
1     &             &           &           &           \\ \hline
2     &             &           &           &           \\ \hline
3     &             &           &           &           \\ \hline
4     &             &           &           &           \\ \hline
\end{tabular}

\end{frame}

\begin{frame}{Another example, $j=1$}

\begin{tabular}{|c|c|c|c|c|}\hline
$j$   &  $r_{j}$    & $r_{j+1}$ & $q_{j+1}$ & $r_{j+2}$ \\ \hline\hline
0     &  87         &  51       &    1      &   36      \\ \hline
1     &  51         &  36       &    1      &   15      \\ \hline
2     &             &           &           &           \\ \hline
3     &             &           &           &           \\ \hline
4     &             &           &           &           \\ \hline
\end{tabular}

\end{frame}

\begin{frame}{Another example, $j=2$}

\begin{tabular}{|c|c|c|c|c|}\hline
$j$   &  $r_{j}$    & $r_{j+1}$ & $q_{j+1}$ & $r_{j+2}$ \\ \hline\hline
0     &  87         &  51       &    1      &   36      \\ \hline
1     &  51         &  36       &    1      &   15      \\ \hline
2     &  36         &  15       &    2      &   6       \\ \hline
3     &             &           &           &           \\ \hline
4     &             &           &           &           \\ \hline
\end{tabular}

\end{frame}

\begin{frame}{Another example, $j=3$}

\begin{tabular}{|c|c|c|c|c|}\hline
$j$   &  $r_{j}$    & $r_{j+1}$ & $q_{j+1}$ & $r_{j+2}$ \\ \hline\hline
0     &  87         &  51       &    1      &   36      \\ \hline
1     &  51         &  36       &    1      &   15      \\ \hline
2     &  36         &  15       &    2      &   6       \\ \hline
3     &  15         &   6       &    2      &   3       \\ \hline
4     &             &           &           &           \\ \hline
\end{tabular}

\end{frame}

\begin{frame}{Another example, $j=4$}

\begin{tabular}{|c|c|c|c|c|}\hline
$j$   &  $r_{j}$    & $r_{j+1}$ & $q_{j+1}$ & $r_{j+2}$ \\ \hline\hline
0     &  87         &  51       &    1      &   36      \\ \hline
1     &  51         &  36       &    1      &   15      \\ \hline
2     &  36         &  15       &    2      &   6       \\ \hline
3     &  15         &   6       &    2      &   3       \\ \hline
4     &   6         &   3       &    2      &   0       \\ \hline
\end{tabular}

\vspace{1em}

So $\gcd(87,51) = 3$.

\end{frame}

\begin{frame}{Algorithm For Bezout's Theorem}

\begin{itemize}
\item Recall Bezout's Theorem.
\item Let $a$ and $b$ be integers. Let $d=\gcd(a,b)$.
\item Then there are integers $s,t$ such that $as +bt = d$.
\item Suppose you are given two positive integers $a, b$ with $a>b$.
\item The Euclidean algorithm let's you find $d=\gcd(a,b)$.
\item But how can you find $s,t$ such that $as +bt = d$?
\item The Euclidean Algorithm can be extended to find $s$ and $t$ also.
\end{itemize}

\end{frame}

\begin{frame}{Back-substitution in the Euclidean Algorithm}

\begin{itemize}
  \item Example: Find $s$ and $t$ so that $75s+20t = \gcd(75, 20)$.
  \item $75 = 20\cdot 3 + 15$.
  \item $20 = 15 \cdot 1 + 5$.
  \item $15 = 5 \cdot 3 + 0$.
  \item So $\gcd(75, 20) = 5$.
  \item $5= 20 - 15 \cdot 1$.
  \item $15 = 75 -20 \cdot 3$.
  \item $5=20 - (75 - 20\cdot 3) \cdot 1$.
  \item $5 = 75 \cdot (-1) + 20 \cdot 4$.
  \item $s=-1, t=4$.
\end{itemize}

\end{frame}

\begin{frame}{The extended Euclidean Algorithm}

\begin{itemize}
  \item The previous technique is called back-substitution.
  \item It works but there is a more efficient technique that doesn't require
  working backwards through the whole table of values after you are done.
  \item Instead we compute $s$ and $t$ while working forwards through the table.
  \item We add two more columns to our table labeled $s_j$ and $t_j$.
  \item We will also add one more row to the table.
\end{itemize}

\end{frame}

\begin{frame}{The extended Euclidean Algorithm, continued}

\begin{itemize}
  \item For the first two rows we always use:
  \item  $s_0=1, t_0=0$
  \item  $s_1=0, t_1=1$.
  \item For $j\geq 2$, $s_j=s_{j-2} - q_{j-1}s_{j-1}$ \quad $t_j=t_{j-2} - q_{j-1}t_{j-1}$
  \item For each row $j$ we will have $a \cdot s_j + b \cdot t_j = r_j$.
  \item Eventually we will reach a row $j$ so that $r_j = d$.
  \item For that $j$, $s=s_j$ and $t=t_j$.
  \item Example: Find $s$ and $t$ so that $75s+20t = \gcd(75, 20)$.
\end{itemize}

\end{frame}

\beamerdefaultoverlayspecification{}

\begin{frame}{Extended Euclidean Algorithm}

\begin{itemize}
  \item Example: Find $s$ and $t$ so that $75s+20t = \gcd(75, 20)$.
  \item Notes:
  \begin{itemize}
    \item For $j\geq0$, $a \cdot s_j + b \cdot t_j = r_j$.
    \item For $j\geq 2$, $s_j=s_{j-2} - q_{j-1}s_{j-1}$ \quad $t_j=t_{j-2} - q_{j-1}t_{j-1}$
  \end{itemize}
\end{itemize}

\vspace{1em}

\begin{tabular}{|c|c|c|c|c|c|c|}\hline
$j$   &  $r_{j}$    & $r_{j+1}$ & $q_{j+1}$ & $r_{j+2}$ & $s_j$ & $t_j$ \\ \hline\hline
0     &  75         &  20       &    3      &   15      &  1    &   0   \\ \hline
1     &             &           &           &           &  0    &   1   \\ \hline
2     &             &           &           &           &       &       \\ \hline
3     &             &           &           &           &       &       \\ \hline
\end{tabular}

\vspace{1em}

\begin{itemize}
  \item $r_0 = 75, s_0=1, t_0=0$.
  \item $75\cdot 1 + 20 \cdot 0 = 75$.
\end{itemize}

\end{frame}

\begin{frame}{Extended Euclidean Algorithm, $j=1$}

\begin{itemize}
  \item Example: Find $s$ and $t$ so that $75s+20t = \gcd(75, 20)$.
  \item Notes:
  \begin{itemize}
    \item For $j\geq0$, $a \cdot s_j + b \cdot t_j = r_j$.
    \item For $j\geq 2$, $s_j=s_{j-2} - q_{j-1}s_{j-1}$ \quad $t_j=t_{j-2} - q_{j-1}t_{j-1}$
  \end{itemize}
\end{itemize}

\vspace{1em}

\begin{tabular}{|c|c|c|c|c|c|c|}\hline
$j$   &  $r_{j}$    & $r_{j+1}$ & $q_{j+1}$ & $r_{j+2}$ & $s_j$ & $t_j$ \\ \hline\hline
0     &  75         &  20       &    3      &   15      &  1    &   0   \\ \hline
1     &  20         &  15       &    1      &    5      &  0    &   1   \\ \hline
2     &             &           &           &           &       &       \\ \hline
3     &             &           &           &           &       &       \\ \hline
\end{tabular}

\vspace{1em}

\begin{itemize}
  \item $r_1 = 20, s_1=0, t_1=1$.
  \item $75\cdot 0 + 20 \cdot 1 = 20$.
\end{itemize}

\end{frame}

\begin{frame}{Extended Euclidean Algorithm, $j=2$}

\begin{itemize}
  \item Example: Find $s$ and $t$ so that $75s+20t = \gcd(75, 20)$.
  \item Notes:
  \begin{itemize}
    \item For $j\geq0$, $a \cdot s_j + b \cdot t_j = r_j$.
    \item For $j\geq 2$, $s_j=s_{j-2} - q_{j-1}s_{j-1}$ \quad $t_j=t_{j-2} - q_{j-1}t_{j-1}$
  \end{itemize}
\end{itemize}

\vspace{1em}

\begin{tabular}{|c|c|c|c|c|c|c|}\hline
$j$   &  $r_{j}$    & $r_{j+1}$ & $q_{j+1}$ & $r_{j+2}$ & $s_j$ & $t_j$ \\ \hline\hline
0     &  75         &  20       &    3      &   15      &  1    &   0   \\ \hline
1     &  20         &  15       &    1      &    5      &  0    &   1   \\ \hline
2     &  15         &   5       &    3      &    0      &  1    &  -3   \\ \hline
3     &             &           &           &           &       &       \\ \hline
\end{tabular}

\vspace{1em}

\begin{itemize}
  \item $s_2 = s_0 - q_1 \cdot s_1, t_2 = t_0 - q_1 \cdot t_1$.
  \item $s_2 =1 - 3 \cdot 0 = 1, t_2 = 0 - 3 \cdot 1 = -3$.
  \item $75\cdot 1 + 20 \cdot (-3) = 15 = r_2$.
\end{itemize}

\end{frame}

\begin{frame}{Extended Euclidean Algorithm, $j=3$}

\begin{itemize}
  \item Example: Find $s$ and $t$ so that $75s+20t = \gcd(75, 20)$.
  \item Notes:
  \begin{itemize}
    \item For $j\geq0$, $a \cdot s_j + b \cdot t_j = r_j$.
    \item For $j\geq 2$, $s_j=s_{j-2} - q_{j-1}s_{j-1}$ \quad $t_j=t_{j-2} - q_{j-1}t_{j-1}$
  \end{itemize}
\end{itemize}

\vspace{1em}

\begin{tabular}{|c|c|c|c|c|c|c|}\hline
$j$   &  $r_{j}$    & $r_{j+1}$ & $q_{j+1}$ & $r_{j+2}$ & $s_j$ & $t_j$ \\ \hline\hline
0     &  75         &  20       &    3      &   15      &  1    &   0   \\ \hline
1     &  20         &  15       &    1      &    5      &  0    &   1   \\ \hline
2     &  15         &   5       &    3      &    0      &  1    &  -3   \\ \hline
3     &   5         &           &           &           & -1    &   4   \\ \hline
\end{tabular}

\vspace{1em}

\begin{itemize}
  \item $s_3 = s_1 - q_2 \cdot s_2, t_3 = t_1 - q_2 \cdot t_2$.
  \item $s_3 =0 - 1 \cdot 1 = -1, t_3 = 1 - 1 \cdot (-3) = 4$.
  \item $75\cdot (-1) + 20 \cdot (4) = 5 = r_3$.
\end{itemize}

\end{frame}

\begin{frame}{Extended Euclidean Algorithm, Conclusion}

\begin{itemize}
  \item Example: Find $s$ and $t$ so that $75s+20t = \gcd(75, 20)$.
  \item Notes:
  \begin{itemize}
    \item For $j\geq0$, $a \cdot s_j + b \cdot t_j = r_j$.
    \item For $j\geq 2$, $s_j=s_{j-2} - q_{j-1}s_{j-1}$ \quad $t_j=t_{j-2} - q_{j-1}t_{j-1}$
  \end{itemize}
\end{itemize}

\vspace{1em}

\begin{tabular}{|c|c|c|c|c|c|c|}\hline
$j$   &  $r_{j}$    & $r_{j+1}$ & $q_{j+1}$ & $r_{j+2}$ & $s_j$ & $t_j$ \\ \hline\hline
0     &  75         &  20       &    3      &   15      &  1    &   0   \\ \hline
1     &  20         &  15       &    1      &    5      &  0    &   1   \\ \hline
2     &  15         &   5       &    3      &    0      &  1    &  -3   \\ \hline
3     &   5         &           &           &           & -1    &   4   \\ \hline
\end{tabular}

\begin{itemize}
  \item $d = \gcd(75, 20) = 5$.
  \item $s=-1, t=4$.
  \item $75\cdot(-1) + 20 \cdot 4 = 5$.
\end{itemize}

\end{frame}

\begin{frame}{Another example}

Example: Find $s$ and $t$ so that $87s+51t = \gcd(87, 51)$.

\vspace{1em}

\begin{tabular}{|c|c|c|c|c|c|c|}\hline
$j$   &  $r_{j}$    & $r_{j+1}$ & $q_{j+1}$ & $r_{j+2}$ & $s_j$ & $t_j$ \\ \hline\hline
0     &  87         &  51       &    1      &   36      &  1    &   0   \\ \hline
1     &             &           &           &           &  0    &   1   \\ \hline
2     &             &           &           &           &       &       \\ \hline
3     &             &           &           &           &       &       \\ \hline
4     &             &           &           &           &       &       \\ \hline
5     &             &           &           &           &       &       \\ \hline
\end{tabular}

\vspace{1em}

\begin{itemize}
  \item $s_0 = 1, t_0 = 0$.
  \item Check: $87\cdot 1 + 51\cdot 0 = 87$
\end{itemize}

\end{frame}

\begin{frame}{Another example, $j=1$}

Example: Find $s$ and $t$ so that $87s+51t = \gcd(87, 51)$.

\vspace{1em}

\begin{tabular}{|c|c|c|c|c|c|c|}\hline
$j$   &  $r_{j}$    & $r_{j+1}$ & $q_{j+1}$ & $r_{j+2}$ & $s_j$ & $t_j$ \\ \hline\hline
0     &  87         &  51       &    1      &   36      &  1    &   0   \\ \hline
1     &  51         &  36       &    1      &   15      &  0    &   1   \\ \hline
2     &             &           &           &           &       &       \\ \hline
3     &             &           &           &           &       &       \\ \hline
4     &             &           &           &           &       &       \\ \hline
5     &             &           &           &           &       &       \\ \hline
\end{tabular}

\vspace{1em}

\begin{itemize}
  \item $s_1 = 0, t_1 = 1$.
  \item Check: $87\cdot 0 + 51\cdot 1 = 51$
\end{itemize}

\end{frame}

\begin{frame}{Another example, $j=2$}

Example: Find $s$ and $t$ so that $87s+51t = \gcd(87, 51)$.

\vspace{1em}

\begin{tabular}{|c|c|c|c|c|c|c|}\hline
$j$   &  $r_{j}$    & $r_{j+1}$ & $q_{j+1}$ & $r_{j+2}$ & $s_j$ & $t_j$ \\ \hline\hline
0     &  87         &  51       &    1      &   36      &  1    &   0   \\ \hline
1     &  51         &  36       &    1      &   15      &  0    &   1   \\ \hline
2     &  36         &  15       &    2      &    6      &  1    &  -1   \\ \hline
3     &             &           &           &           &       &       \\ \hline
4     &             &           &           &           &       &       \\ \hline
5     &             &           &           &           &       &       \\ \hline
\end{tabular}

\vspace{1em}

\begin{itemize}
  \item $s_2 = 1 - 1\cdot 0 = 1, t_2 = 0 - 1\cdot 1 = -1$.
  \item Check: $87\cdot 1 + 51\cdot (-1) = 36$
\end{itemize}

\end{frame}

\begin{frame}{Another example, $j=3$}

Example: Find $s$ and $t$ so that $87s+51t = \gcd(87, 51)$.

\vspace{1em}

\begin{tabular}{|c|c|c|c|c|c|c|}\hline
$j$   &  $r_{j}$    & $r_{j+1}$ & $q_{j+1}$ & $r_{j+2}$ & $s_j$ & $t_j$ \\ \hline\hline
0     &  87         &  51       &    1      &   36      &  1    &   0   \\ \hline
1     &  51         &  36       &    1      &   15      &  0    &   1   \\ \hline
2     &  36         &  15       &    2      &    6      &  1    &  -1   \\ \hline
3     &  15         &   6       &    2      &    3      & -1    &   2   \\ \hline
4     &             &           &           &           &       &       \\ \hline
5     &             &           &           &           &       &       \\ \hline
\end{tabular}

\vspace{1em}

\begin{itemize}
  \item $s_3 = 0 - 1\cdot 1 = -1, t_3 = 1 - 1\cdot (-1) = 2$.
  \item Check: $87\cdot (-1) + 51\cdot 2 = 15$
\end{itemize}

\end{frame}

\begin{frame}{Another example, $j=4$}

Example: Find $s$ and $t$ so that $87s+51t = \gcd(87, 51)$.

\vspace{1em}

\begin{tabular}{|c|c|c|c|c|c|c|}\hline
$j$   &  $r_{j}$    & $r_{j+1}$ & $q_{j+1}$ & $r_{j+2}$ & $s_j$ & $t_j$ \\ \hline\hline
0     &  87         &  51       &    1      &   36      &  1    &   0   \\ \hline
1     &  51         &  36       &    1      &   15      &  0    &   1   \\ \hline
2     &  36         &  15       &    2      &    6      &  1    &  -1   \\ \hline
3     &  15         &   6       &    2      &    3      & -1    &   2   \\ \hline
4     &   6         &   3       &    2      &    0      &  3    &  -5   \\ \hline
5     &             &           &           &           &       &       \\ \hline
\end{tabular}

\vspace{1em}

\begin{itemize}
  \item $\gcd(87,51) = 3$
  \item $s_4 = 1 - 2\cdot (-1) = 3, t_4 = -1 - 2\cdot 2 = -5$.
  \item Check: $87\cdot 3 + 51\cdot (-5) = 6$
\end{itemize}

\end{frame}

\begin{frame}{Another example, $j=5$}

Example: Find $s$ and $t$ so that $87s+51t = \gcd(87, 51)$.

\vspace{1em}

\begin{tabular}{|c|c|c|c|c|c|c|}\hline
$j$   &  $r_{j}$    & $r_{j+1}$ & $q_{j+1}$ & $r_{j+2}$ & $s_j$ & $t_j$ \\ \hline\hline
0     &  87         &  51       &    1      &   36      &  1    &   0   \\ \hline
1     &  51         &  36       &    1      &   15      &  0    &   1   \\ \hline
2     &  36         &  15       &    2      &    6      &  1    &  -1   \\ \hline
3     &  15         &   6       &    2      &    3      & -1    &   2   \\ \hline
4     &   6         &   3       &    2      &    0      &  3    &  -5   \\ \hline
5     &   3         &           &           &           & -7    &  12   \\ \hline
\end{tabular}

\vspace{1em}

\begin{itemize}
  \item $\gcd(87,51) = 3$
  \item $s_5 = -1 - 2\cdot 3 = -7, t_5 = 2 - 2\cdot (-5) = 12$.
  \item Check: $87\cdot (-7) + 51\cdot 12 = 3$
  \item $s=-7, t=12$.
\end{itemize}


\end{frame}


\beamerdefaultoverlayspecification{<+->}

\begin{frame}{Why does it work?}
\begin{itemize}
\item Why do the formulas  $s_j=s_{j-2} - q_{j-1}s_{j-1}$ \quad $t_j=t_{j-2} - q_{j-1}t_{j-1}$ work?
\item Let's consider the case $j=2$. $s_2=s_0 - q_1s_1$ \quad $t_2=t_0 - q_1t_1$.
\item We know that $a s_0+bt_0 = r_0$ and $a s_1 + b t_1 = r_1$ and we want to see that $as_2+bt_2=r_2$.
\item $r_0 = r_1 q_1 + r_2$.
\item So $r_2 = r_0 - r_1 q_1$.
\item So $r_2 = (a s_0+bt_0) - (a s_1 + b t_1)q_1 = a(s_0 - q_1 s_1) + b(t_0 - q_1 t_1)$
\item = $as_2 + b t_2$.
\item Formally, the proof is by induction on $j$.

\end{itemize}
\end{frame}

\begin{frame}{Is it more efficient?}
\begin{itemize}
\item Remember that the reason why the Euclidean algorithm is interesting to us is that it is much more
efficient than factoring.
\item Suppose $a$ and $b$ are two 200-digit positive integers and you want to compute $\gcd(a,b)$.
\item To attempt to do this by factoring $a$ and $b$ using brute-force trial division would
take on the order of $10^{100}$ steps.
\item This would take longer than the age of the universe, as we discussed previously.
\item So how many steps does it take to compute $\gcd(a,b)$ using the Euclidean algorithm?
\item Answer: In the worst case, fewer than 300 million basic computer steps.
\item A fast modern computer can compute this in less than one second.

\end{itemize}
\end{frame}

\end{document}
