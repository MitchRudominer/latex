% $Header$

%\documentclass{beamer}
\documentclass[handout]{beamer}
\usepackage{amsmath,amssymb,latexsym,eucal,amsthm,graphicx,hyperref}
%%%%%%%%%%%%%%%%%%%%%%%%%%%%%%%%%%%%%%%%%%%%%
% Common Set Theory Constructs
%%%%%%%%%%%%%%%%%%%%%%%%%%%%%%%%%%%%%%%%%%%%%

\newcommand{\setof}[2]{\left\{ \, #1 \, \left| \, #2 \, \right.\right\}}
\newcommand{\lsetof}[2]{\left\{\left. \, #1 \, \right| \, #2 \,  \right\}}
\newcommand{\bigsetof}[2]{\bigl\{ \, #1 \, \bigm | \, #2 \,\bigr\}}
\newcommand{\Bigsetof}[2]{\Bigl\{ \, #1 \, \Bigm | \, #2 \,\Bigr\}}
\newcommand{\biggsetof}[2]{\biggl\{ \, #1 \, \biggm | \, #2 \,\biggr\}}
\newcommand{\Biggsetof}[2]{\Biggl\{ \, #1 \, \Biggm | \, #2 \,\Biggr\}}
\newcommand{\dotsetof}[2]{\left\{ \, #1 \, : \, #2 \, \right\}}
\newcommand{\bigdotsetof}[2]{\bigl\{ \, #1 \, : \, #2 \,\bigr\}}
\newcommand{\Bigdotsetof}[2]{\Bigl\{ \, #1 \, \Bigm : \, #2 \,\Bigr\}}
\newcommand{\biggdotsetof}[2]{\biggl\{ \, #1 \, \biggm : \, #2 \,\biggr\}}
\newcommand{\Biggdotsetof}[2]{\Biggl\{ \, #1 \, \Biggm : \, #2 \,\Biggr\}}
\newcommand{\sequence}[2]{\left\langle \, #1 \,\left| \, #2 \, \right. \right\rangle}
\newcommand{\lsequence}[2]{\left\langle\left. \, #1 \, \right| \,#2 \,  \right\rangle}
\newcommand{\bigsequence}[2]{\bigl\langle \,#1 \, \bigm | \, #2 \, \bigr\rangle}
\newcommand{\Bigsequence}[2]{\Bigl\langle \,#1 \, \Bigm | \, #2 \, \Bigr\rangle}
\newcommand{\biggsequence}[2]{\biggl\langle \,#1 \, \biggm | \, #2 \, \biggr\rangle}
\newcommand{\Biggsequence}[2]{\Biggl\langle \,#1 \, \Biggm | \, #2 \, \Biggr\rangle}
\newcommand{\singleton}[1]{\left\{#1\right\}}
\newcommand{\angles}[1]{\left\langle #1 \right\rangle}
\newcommand{\bigangles}[1]{\bigl\langle #1 \bigr\rangle}
\newcommand{\Bigangles}[1]{\Bigl\langle #1 \Bigr\rangle}
\newcommand{\biggangles}[1]{\biggl\langle #1 \biggr\rangle}
\newcommand{\Biggangles}[1]{\Biggl\langle #1 \Biggr\rangle}


\newcommand{\force}[1]{\Vert\!\underset{\!\!\!\!\!#1}{\!\!\!\relbar\!\!\!%
\relbar\!\!\relbar\!\!\relbar\!\!\!\relbar\!\!\relbar\!\!\relbar\!\!\!%
\relbar\!\!\relbar\!\!\relbar}}
\newcommand{\nforce}[1]{\Vert\!\underset{\!\!\!\!\!#1}{\!\!\!\relbar\!\!\!%
\relbar\!\!\relbar\!\!\relbar\!\!\!\relbar\!\!\relbar\!\!\relbar\!\!\!%
\relbar\!\!\not\relbar\!\!\relbar}}
\newcommand{\forcein}[2]{\overset{#2}{\Vert\underset{\!\!\!\!\!#1}%
{\!\!\!\relbar\!\!\!\relbar\!\!\relbar\!\!\relbar\!\!\!\relbar\!\!\relbar\!%
\!\relbar\!\!\!\relbar\!\!\relbar\!\!\relbar\!\!\relbar\!\!\!\relbar\!\!%
\relbar\!\!\relbar}}}

\newcommand{\pre}[2]{{}^{#2}\!{#1}}

\newcommand{\restr}{\!\!\upharpoonright\!}

%%%%%%%%%%%%%%%%%%%%%%%%%%%%%%%%%%%%%%%%%%%%%
% Set-Theoretic Connectives
%%%%%%%%%%%%%%%%%%%%%%%%%%%%%%%%%%%%%%%%%%%%%

\newcommand{\intersect}{\cap}
\newcommand{\union}{\cup}
\newcommand{\Intersection}[1]{\bigcap\limits_{#1}}
\newcommand{\Union}[1]{\bigcup\limits_{#1}}
\newcommand{\adjoin}{{}^\frown}
\newcommand{\forces}{\Vdash}

%%%%%%%%%%%%%%%%%%%%%%%%%%%%%%%%%%%%%%%%%%%%%
% Miscellaneous
%%%%%%%%%%%%%%%%%%%%%%%%%%%%%%%%%%%%%%%%%%%%%
\newcommand{\defeq}{=_{\text{def}}}
\newcommand{\Turingleq}{\leq_{\text{T}}}
\newcommand{\Turingless}{<_{\text{T}}}
\newcommand{\lexleq}{\leq_{\text{lex}}}
\newcommand{\lexless}{<_{\text{lex}}}
\newcommand{\Turingequiv}{\equiv_{\text{T}}}

%%%%%%%%%%%%%%%%%%%%%%%%%%%%%%%%%%%%%%%%%%%%%
% Constants
%%%%%%%%%%%%%%%%%%%%%%%%%%%%%%%%%%%%%%%%%%%%%
\newcommand{\R}{\mathbb{R}}
\renewcommand{\P}{\mathbb{P}}
\newcommand{\Q}{\mathbb{Q}}
\newcommand{\Z}{\mathbb{Z}}
\newcommand{\C}{\mathbb{C}}
\newcommand{\N}{\mathbb{N}}
\newcommand{\B}{\mathbb{B}}
\newcommand{\LofR}{L(\R)}
\newcommand{\JofR}[1]{J_{#1}(\R)}
\newcommand{\SofR}[1]{S_{#1}(\R)}
\newcommand{\JalphaR}{\JofR{\alpha}}
\newcommand{\JbetaR}{\JofR{\beta}}
\newcommand{\JlambdaR}{\JofR{\lambda}}
\newcommand{\SalphaR}{\SofR{\alpha}}
\newcommand{\SbetaR}{\SofR{\beta}}
\newcommand{\Pkl}{\mathcal{P}_{\kappa}(\lambda)}
\DeclareMathOperator{\con}{con}
\DeclareMathOperator{\ORD}{OR}
\DeclareMathOperator{\Ord}{OR}
\DeclareMathOperator{\WO}{WO}
\DeclareMathOperator{\OD}{OD}
\DeclareMathOperator{\HOD}{HOD}
\DeclareMathOperator{\HC}{HC}
\DeclareMathOperator{\WF}{WF}
\DeclareMathOperator{\HF}{HF}
\newcommand{\One}{I}
\newcommand{\ONE}{I}
\newcommand{\Two}{II}
\newcommand{\TWO}{II}

%%%%%%%%%%%%%%%%%%%%%%%%%%%%%%%%%%%%%%%%%%%%%
% Commutative Algebra Constants
%%%%%%%%%%%%%%%%%%%%%%%%%%%%%%%%%%%%%%%%%%%%%
\DeclareMathOperator{\dottimes}{\dot{\times}}

%%%%%%%%%%%%%%%%%%%%%%%%%%%%%%%%%%%%%%%%%%%%%
% Theories
%%%%%%%%%%%%%%%%%%%%%%%%%%%%%%%%%%%%%%%%%%%%%
\DeclareMathOperator{\ZFC}{ZFC}
\DeclareMathOperator{\ZF}{ZF}
\DeclareMathOperator{\AD}{AD}
\DeclareMathOperator{\ADR}{AD_{\R}}
\DeclareMathOperator{\KP}{KP}
\DeclareMathOperator{\PD}{PD}
\DeclareMathOperator{\CH}{CH}
\DeclareMathOperator{\HPC}{HPC} % HOD pair capturing
%%%%%%%%%%%%%%%%%%%%%%%%%%%%%%%%%%%%%%%%%%%%%
% Iteration Trees
%%%%%%%%%%%%%%%%%%%%%%%%%%%%%%%%%%%%%%%%%%%%%

\newcommand{\pred}{\text{-pred}}

%%%%%%%%%%%%%%%%%%%%%%%%%%%%%%%%%%%%%%%%%%%%%%%%
% Operator Names
%%%%%%%%%%%%%%%%%%%%%%%%%%%%%%%%%%%%%%%%%%%%%%%%
\DeclareMathOperator{\Det}{Det}
\DeclareMathOperator{\dom}{dom}
\DeclareMathOperator{\ran}{ran}
\DeclareMathOperator{\range}{ran}
\DeclareMathOperator{\image}{image}
\DeclareMathOperator{\crit}{crit}
\DeclareMathOperator{\card}{card}
\DeclareMathOperator{\cf}{cf}
\DeclareMathOperator{\cof}{cof}
\DeclareMathOperator{\rank}{rank}
\DeclareMathOperator{\ot}{o.t.}
\DeclareMathOperator{\ords}{o}
\DeclareMathOperator{\ro}{r.o.}
\DeclareMathOperator{\rud}{rud}
\DeclareMathOperator{\Powerset}{\mathcal{P}}
\DeclareMathOperator{\length}{lh}
\DeclareMathOperator{\lh}{lh}
\DeclareMathOperator{\limit}{lim}
\DeclareMathOperator{\fld}{fld}
\DeclareMathOperator{\projection}{p}
\DeclareMathOperator{\Ult}{Ult}
\DeclareMathOperator{\ULT}{Ult}
\DeclareMathOperator{\Coll}{Coll}
\DeclareMathOperator{\Col}{Col}
\DeclareMathOperator{\Hull}{Hull}
\DeclareMathOperator*{\dirlim}{dir lim}
\DeclareMathOperator{\Scale}{Scale}
\DeclareMathOperator{\supp}{supp}
\DeclareMathOperator{\trancl}{tran.cl.}
\DeclareMathOperator{\trace}{Tr}
\DeclareMathOperator{\diag}{diag}
\DeclareMathOperator{\spn}{span}
\DeclareMathOperator{\sgn}{sgn}
%%%%%%%%%%%%%%%%%%%%%%%%%%%%%%%%%%%%%%%%%%%%%
% Logical Connectives
%%%%%%%%%%%%%%%%%%%%%%%%%%%%%%%%%%%%%%%%%%%%%
\newcommand{\IImplies}{\Longrightarrow}
\newcommand{\SkipImplies}{\quad\Longrightarrow\quad}
\newcommand{\Ifff}{\Longleftrightarrow}
\newcommand{\iimplies}{\longrightarrow}
\newcommand{\ifff}{\longleftrightarrow}
\newcommand{\Implies}{\Rightarrow}
\newcommand{\Iff}{\Leftrightarrow}
\renewcommand{\implies}{\rightarrow}
\renewcommand{\iff}{\leftrightarrow}
\newcommand{\AND}{\wedge}
\newcommand{\OR}{\vee}
\newcommand{\st}{\text{ s.t. }}
\newcommand{\Or}{\text{ or }}

%%%%%%%%%%%%%%%%%%%%%%%%%%%%%%%%%%%%%%%%%%%%%
% Function Arrows
%%%%%%%%%%%%%%%%%%%%%%%%%%%%%%%%%%%%%%%%%%%%%

\newcommand{\injection}{\xrightarrow{\text{1-1}}}
\newcommand{\surjection}{\xrightarrow{\text{onto}}}
\newcommand{\bijection}{\xrightarrow[\text{onto}]{\text{1-1}}}
\newcommand{\cofmap}{\xrightarrow{\text{cof}}}
\newcommand{\map}{\rightarrow}

%%%%%%%%%%%%%%%%%%%%%%%%%%%%%%%%%%%%%%%%%%%%%
% Mouse Comparison Operators
%%%%%%%%%%%%%%%%%%%%%%%%%%%%%%%%%%%%%%%%%%%%%
\newcommand{\initseg}{\trianglelefteq}
\newcommand{\properseg}{\lhd}
\newcommand{\notinitseg}{\ntrianglelefteq}
\newcommand{\notproperseg}{\ntriangleleft}

%%%%%%%%%%%%%%%%%%%%%%%%%%%%%%%%%%%%%%%%%%%%%
%           calligraphic letters
%%%%%%%%%%%%%%%%%%%%%%%%%%%%%%%%%%%%%%%%%%%%%
\newcommand{\cA}{\mathcal{A}}
\newcommand{\cB}{\mathcal{B}}
\newcommand{\cC}{\mathcal{C}}
\newcommand{\cD}{\mathcal{D}}
\newcommand{\cE}{\mathcal{E}}
\newcommand{\cF}{\mathcal{F}}
\newcommand{\cG}{\mathcal{G}}
\newcommand{\cH}{\mathcal{H}}
\newcommand{\cI}{\mathcal{I}}
\newcommand{\cJ}{\mathcal{J}}
\newcommand{\cK}{\mathcal{K}}
\newcommand{\cL}{\mathcal{L}}
\newcommand{\cM}{\mathcal{M}}
\newcommand{\cN}{\mathcal{N}}
\newcommand{\cO}{\mathcal{O}}
\newcommand{\cP}{\mathcal{P}}
\newcommand{\cQ}{\mathcal{Q}}
\newcommand{\cR}{\mathcal{R}}
\newcommand{\cS}{\mathcal{S}}
\newcommand{\cT}{\mathcal{T}}
\newcommand{\cU}{\mathcal{U}}
\newcommand{\cV}{\mathcal{V}}
\newcommand{\cW}{\mathcal{W}}
\newcommand{\cX}{\mathcal{X}}
\newcommand{\cY}{\mathcal{Y}}
\newcommand{\cZ}{\mathcal{Z}}


%%%%%%%%%%%%%%%%%%%%%%%%%%%%%%%%%%%%%%%%%%%%%
%          Primed Letters
%%%%%%%%%%%%%%%%%%%%%%%%%%%%%%%%%%%%%%%%%%%%%
\newcommand{\aprime}{a^{\prime}}
\newcommand{\bprime}{b^{\prime}}
\newcommand{\cprime}{c^{\prime}}
\newcommand{\dprime}{d^{\prime}}
\newcommand{\eprime}{e^{\prime}}
\newcommand{\fprime}{f^{\prime}}
\newcommand{\gprime}{g^{\prime}}
\newcommand{\hprime}{h^{\prime}}
\newcommand{\iprime}{i^{\prime}}
\newcommand{\jprime}{j^{\prime}}
\newcommand{\kprime}{k^{\prime}}
\newcommand{\lprime}{l^{\prime}}
\newcommand{\mprime}{m^{\prime}}
\newcommand{\nprime}{n^{\prime}}
\newcommand{\oprime}{o^{\prime}}
\newcommand{\pprime}{p^{\prime}}
\newcommand{\qprime}{q^{\prime}}
\newcommand{\rprime}{r^{\prime}}
\newcommand{\sprime}{s^{\prime}}
\newcommand{\tprime}{t^{\prime}}
\newcommand{\uprime}{u^{\prime}}
\newcommand{\vprime}{v^{\prime}}
\newcommand{\wprime}{w^{\prime}}
\newcommand{\xprime}{x^{\prime}}
\newcommand{\yprime}{y^{\prime}}
\newcommand{\zprime}{z^{\prime}}
\newcommand{\Aprime}{A^{\prime}}
\newcommand{\Bprime}{B^{\prime}}
\newcommand{\Cprime}{C^{\prime}}
\newcommand{\Dprime}{D^{\prime}}
\newcommand{\Eprime}{E^{\prime}}
\newcommand{\Fprime}{F^{\prime}}
\newcommand{\Gprime}{G^{\prime}}
\newcommand{\Hprime}{H^{\prime}}
\newcommand{\Iprime}{I^{\prime}}
\newcommand{\Jprime}{J^{\prime}}
\newcommand{\Kprime}{K^{\prime}}
\newcommand{\Lprime}{L^{\prime}}
\newcommand{\Mprime}{M^{\prime}}
\newcommand{\Nprime}{N^{\prime}}
\newcommand{\Oprime}{O^{\prime}}
\newcommand{\Pprime}{P^{\prime}}
\newcommand{\Qprime}{Q^{\prime}}
\newcommand{\Rprime}{R^{\prime}}
\newcommand{\Sprime}{S^{\prime}}
\newcommand{\Tprime}{T^{\prime}}
\newcommand{\Uprime}{U^{\prime}}
\newcommand{\Vprime}{V^{\prime}}
\newcommand{\Wprime}{W^{\prime}}
\newcommand{\Xprime}{X^{\prime}}
\newcommand{\Yprime}{Y^{\prime}}
\newcommand{\Zprime}{Z^{\prime}}
\newcommand{\alphaprime}{\alpha^{\prime}}
\newcommand{\betaprime}{\beta^{\prime}}
\newcommand{\gammaprime}{\gamma^{\prime}}
\newcommand{\Gammaprime}{\Gamma^{\prime}}
\newcommand{\deltaprime}{\delta^{\prime}}
\newcommand{\epsilonprime}{\epsilon^{\prime}}
\newcommand{\kappaprime}{\kappa^{\prime}}
\newcommand{\lambdaprime}{\lambda^{\prime}}
\newcommand{\rhoprime}{\rho^{\prime}}
\newcommand{\Sigmaprime}{\Sigma^{\prime}}
\newcommand{\tauprime}{\tau^{\prime}}
\newcommand{\xiprime}{\xi^{\prime}}
\newcommand{\thetaprime}{\theta^{\prime}}
\newcommand{\Omegaprime}{\Omega^{\prime}}
\newcommand{\cMprime}{\cM^{\prime}}
\newcommand{\cNprime}{\cN^{\prime}}
\newcommand{\cPprime}{\cP^{\prime}}
\newcommand{\cQprime}{\cQ^{\prime}}
\newcommand{\cRprime}{\cR^{\prime}}
\newcommand{\cSprime}{\cS^{\prime}}
\newcommand{\cTprime}{\cT^{\prime}}

%%%%%%%%%%%%%%%%%%%%%%%%%%%%%%%%%%%%%%%%%%%%%
%          bar Letters
%%%%%%%%%%%%%%%%%%%%%%%%%%%%%%%%%%%%%%%%%%%%%
\newcommand{\abar}{\bar{a}}
\newcommand{\bbar}{\bar{b}}
\newcommand{\zbar}{\bar{z}}
\newcommand{\phibar}{\bar{\varphi}}
\newcommand{\psibar}{\bar{\psi}}
\newcommand{\thetabar}{\bar{\theta}}
\newcommand{\nubar}{\bar{\nu}}

%%%%%%%%%%%%%%%%%%%%%%%%%%%%%%%%%%%%%%%%%%%%%
%          star Letters
%%%%%%%%%%%%%%%%%%%%%%%%%%%%%%%%%%%%%%%%%%%%%
\newcommand{\phistar}{\phi^{*}}


%%%%%%%%%%%%%%%%%%%%%%%%%%%%%%%%%%%%%%%%%%%%%
%          Formulas
%%%%%%%%%%%%%%%%%%%%%%%%%%%%%%%%%%%%%%%%%%%%%

\newcommand{\formulaphi}{\text{\large $\varphi$}}
\newcommand{\Formulaphi}{\text{\Large $\varphi$}}


%%%%%%%%%%%%%%%%%%%%%%%%%%%%%%%%%%%%%%%%%%%%%
%          Fraktur Letters
%%%%%%%%%%%%%%%%%%%%%%%%%%%%%%%%%%%%%%%%%%%%%

\newcommand{\fa}{\mathfrak{a}}
\newcommand{\fb}{\mathfrak{b}}
\newcommand{\fc}{\mathfrak{c}}
\newcommand{\fk}{\mathfrak{k}}
\newcommand{\fp}{\mathfrak{p}}
\newcommand{\fq}{\mathfrak{q}}
\newcommand{\fr}{\mathfrak{r}}
\newcommand{\fA}{\mathfrak{A}}
\newcommand{\fB}{\mathfrak{B}}
\newcommand{\fC}{\mathfrak{C}}
\newcommand{\fD}{\mathfrak{D}}

%%%%%%%%%%%%%%%%%%%%%%%%%%%%%%%%%%%%%%%%%%%%%
%          Bold Letters
%%%%%%%%%%%%%%%%%%%%%%%%%%%%%%%%%%%%%%%%%%%%%
\newcommand{\ba}{\mathbf{a}}
\newcommand{\bb}{\mathbf{b}}
\newcommand{\bc}{\mathbf{c}}
\newcommand{\bd}{\mathbf{d}}
\newcommand{\be}{\mathbf{e}}
\newcommand{\bu}{\mathbf{u}}
\newcommand{\bv}{\mathbf{v}}
\newcommand{\bw}{\mathbf{w}}
\newcommand{\bx}{\mathbf{x}}
\newcommand{\by}{\mathbf{y}}
\newcommand{\bz}{\mathbf{z}}
\newcommand{\bSigma}{\boldsymbol{\Sigma}}
\newcommand{\bPi}{\boldsymbol{\Pi}}
\newcommand{\bDelta}{\boldsymbol{\Delta}}
\newcommand{\bdelta}{\boldsymbol{\delta}}
\newcommand{\bgamma}{\boldsymbol{\gamma}}
\newcommand{\bGamma}{\boldsymbol{\Gamma}}

%%%%%%%%%%%%%%%%%%%%%%%%%%%%%%%%%%%%%%%%%%%%%
%         Bold numbers
%%%%%%%%%%%%%%%%%%%%%%%%%%%%%%%%%%%%%%%%%%%%%
\newcommand{\bzero}{\mathbf{0}}

%%%%%%%%%%%%%%%%%%%%%%%%%%%%%%%%%%%%%%%%%%%%%
% Projective-Like Pointclasses
%%%%%%%%%%%%%%%%%%%%%%%%%%%%%%%%%%%%%%%%%%%%%
\newcommand{\Sa}[2][\alpha]{\Sigma_{(#1,#2)}}
\newcommand{\Pa}[2][\alpha]{\Pi_{(#1,#2)}}
\newcommand{\Da}[2][\alpha]{\Delta_{(#1,#2)}}
\newcommand{\Aa}[2][\alpha]{A_{(#1,#2)}}
\newcommand{\Ca}[2][\alpha]{C_{(#1,#2)}}
\newcommand{\Qa}[2][\alpha]{Q_{(#1,#2)}}
\newcommand{\da}[2][\alpha]{\delta_{(#1,#2)}}
\newcommand{\leqa}[2][\alpha]{\leq_{(#1,#2)}}
\newcommand{\lessa}[2][\alpha]{<_{(#1,#2)}}
\newcommand{\equiva}[2][\alpha]{\equiv_{(#1,#2)}}


\newcommand{\Sl}[1]{\Sa[\lambda]{#1}}
\newcommand{\Pl}[1]{\Pa[\lambda]{#1}}
\newcommand{\Dl}[1]{\Da[\lambda]{#1}}
\newcommand{\Al}[1]{\Aa[\lambda]{#1}}
\newcommand{\Cl}[1]{\Ca[\lambda]{#1}}
\newcommand{\Ql}[1]{\Qa[\lambda]{#1}}

\newcommand{\San}{\Sa{n}}
\newcommand{\Pan}{\Pa{n}}
\newcommand{\Dan}{\Da{n}}
\newcommand{\Can}{\Ca{n}}
\newcommand{\Qan}{\Qa{n}}
\newcommand{\Aan}{\Aa{n}}
\newcommand{\dan}{\da{n}}
\newcommand{\leqan}{\leqa{n}}
\newcommand{\lessan}{\lessa{n}}
\newcommand{\equivan}{\equiva{n}}

%%%%%%%%%%%%%%%%%%%%%%%%%%%%%%%%%%%%%%%%%%%%%
% Linear Algebra
%%%%%%%%%%%%%%%%%%%%%%%%%%%%%%%%%%%%%%%%%%%%%
\newcommand{\transpose}[1]{{#1}^{\text{T}}}
\newcommand{\norm}[1]{\lVert{#1}\rVert}
\newcommand\aug{\fboxsep=-\fboxrule\!\!\!\fbox{\strut}\!\!\!}

%%%%%%%%%%%%%%%%%%%%%%%%%%%%%%%%%%%%%%%%%%%%%
% Number Theory
%%%%%%%%%%%%%%%%%%%%%%%%%%%%%%%%%%%%%%%%%%%%%
\DeclareMathOperator{\Spec}{Spec}
\newcommand{\av}[1]{\lvert#1\rvert}
\DeclareMathOperator{\divides}{\mid}
\DeclareMathOperator{\ndivides}{\nmid}


\graphicspath{{images/}}

\newtheorem*{claim}{claim}
\newtheorem*{observation}{Observation}
\newtheorem*{warning}{Warning}
\newtheorem*{question}{Question}
\newtheorem{remark}[theorem]{Remark}

\newenvironment*{subproof}[1][Proof]
{\begin{proof}[#1]}{\renewcommand{\qedsymbol}{$\diamondsuit$} \end{proof}}

\mode<presentation>
{
  \usetheme{Singapore}
  % or ...

  \setbeamercovered{invisible}
  % or whatever (possibly just delete it)
}


\usepackage[english]{babel}
% or whatever

\usepackage[latin1]{inputenc}
% or whatever

\usepackage{times}
\usepackage[T1]{fontenc}
% Or whatever. Note that the encoding and the font should match. If T1
% does not look nice, try deleting the line with the fontenc.

\title{Lesson 12 \\ Chinese Remainder Theorem}
\subtitle{Math 310, Elementary Number Theory \\ Fall 2021 \\ SFSU}
\author{Mitch Rudominer}
\date{}



% If you wish to uncover everything in a step-wise fashion, uncomment
% the following command:

\beamerdefaultoverlayspecification{<+->}

\begin{document}

\begin{frame}
  \titlepage
\end{frame}

%%%%%%%%%%%%%%%%%%%%%%%%%%%%%%%%%%%%%%%%%%%%%%%%%%%%%%%%%%%%%%%%%%%%%%%%%%%%%%
\begin{frame}{Pairwise relatively prime}

\begin{itemize}
  \item \textbf{Definition.} Let $m_1, m_2, \cdots m_r$  be integers.
  \item We say that the $m_i$ are \emph{pairwise relatively prime}
  \item iff $\gcd(m_i,m_j)=1$ for all $i,j$ with $1\leq i < j \leq r$.
  \item Example: 3, 10, 77 are pairwise relatively prime.
  \item Because $\gcd(3,10) = \gcd(3,77)=\gcd(10,77) = 1$.
  \item Example 6, 10, 77 are not pairwise relatively prime.
  \item Because $\gcd(6,10)\not=1$.
\end{itemize}

\end{frame}

%%%%%%%%%%%%%%%%%%%%%%%%%%%%%%%%%%%%%%%%%%%%%%%%%%%%%%%%%%%%%%%%%%%%%%%%%%%%%%

\begin{frame}{The Chinese Remainder Theorem}

\begin{itemize}
  \item \textbf{Theorem.} Let $m_1, m_2, \cdots m_r$  be pairwise relatively prime.
  \item Consider the following problem. Solve for $x$:
  \item $x \equiv a_1 \pmod {m_1}$
  \item $x \equiv a_2 \pmod {m_2}$
  \item $\vdots$
  \item $x \equiv a_r \pmod {m_r}$
  \item Then there is a unique solution $x$ with $0\leq x < M$,
  \item where $M=m_1 m_2 \cdots m_r$.
  \item Every $y$ congruent to $x$ mod $M$ is also a solution.
\end{itemize}

\end{frame}

\begin{frame}{Example}

\begin{itemize}
  \item Consider the following problem:
  \item $x \equiv 1 \pmod 3$
  \item $x \equiv 3 \pmod 5$
  \item $x \equiv 5 \pmod 7$
  \item $3\cdot 5 \cdot 7 = 105$.
  \item The Chinese Remainder Theorem says the problem has a unique solution $x$
  \item such that $0\leq x < 105$.
  \item If $x$ is the solution and $y\equiv x \pmod {105}$ then $y$ is also a solution.
\end{itemize}

\end{frame}

%%%%%%%%%%%%%%%%%%%%%%%%%%%%%%%%%%%%%%%%%%%%%%%%%%%%%%%%%%%%%%%%%%%%%%%%%%%%%%

\begin{frame}{Relatively prime moduli}

\begin{itemize}
  \item \textbf{Lemma.} Let $m_1, m_2, \cdots m_r$  be pairwise relatively prime.
  \item Let $M=m_1 m_2 \cdots m_r$.
  \item For any integers $x,y$ the following two things are equivalent:
  \item (a) $x \equiv y \pmod M$
  \item (b) $x \equiv y \pmod {m_1}$, $x\equiv y\pmod {m_2}, \cdots x\equiv y \pmod {m_r}$.
  \item \textbf{proof.} (a) $\Implies$ (b) is easy:
  If $x \equiv y \pmod M$ then $x\equiv y \pmod {m_i}$ because $m_i\divides M$.
  \item Conversely assume (b).
  \item Then $m_1 \divides (x-y)$ and $m_2 \divides (x-y)$ etc.
  \item We have $(x-y) = m_1 w$ for some integer $w$.
  \item Since $\gcd(m_1,m_2) = 1$ and $m_2\divides m_1 w$, we have $m_2 \divides w$
  so $w=m_2s$ for some integer $s$.
  \item So $(x-y) = m_1 m_2 s$ and we have that $m_1 m_2 \divides (x-y)$.
  \item Continuing in this way we get $m_1 m_2 \cdots m_r \divides (x-y)$
  \item So $x\equiv y \pmod M$. $\qed$
\end{itemize}

\end{frame}

%%%%%%%%%%%%%%%%%%%%%%%%%%%%%%%%%%%%%%%%%%%%%%%%%%%%%%%%%%%%%%%%%%%%%%%%%%%%%%

\begin{frame}{Uniqueness Proof}

\begin{itemize}
  \item The Lemma gives us two parts of the proof of the Theorem.
  \item Uniqueness: If $x$ and $y$ are two solutions to the system of equations,
  then by the Lemma they are congruent mod $M$.
  \item Using our example problem, if $x$ and $y$ are both solutions, then
  $x\equiv y \pmod 3, x\equiv y \pmod 5, x\equiv y \pmod 7$.
  \item By the Lemma, $x\equiv y \pmod {105}$.
  \item So there is a unique solution in $[0,104]$.
  \item Also: If $x$ is a solution to the system of equations and $y\equiv x \pmod {105}$ then,
   by the Lemma, $y$ is also a solution.
\end{itemize}

\end{frame}

%%%%%%%%%%%%%%%%%%%%%%%%%%%%%%%%%%%%%%%%%%%%%%%%%%%%%%%%%%%%%%%%%%%%%%%%%%%%%%


\begin{frame}{Existence Proof}

\begin{itemize}
  \item We will describe a procedure for finding a solution and that will prove existence.
  \item Let $M_i = M/{m_i}$.
  \item Then for each $i$,
  \item $\gcd(M_i, m_i) = 1$
  \item and $M_i \equiv 0 \pmod {m_j}$, for $j\not= i$.

\end{itemize}

\end{frame}

\begin{frame}{Proof, page 2}

\begin{itemize}
  \item Using our example problem:
  \item $x \equiv 1 \pmod 3$
  \item $x \equiv 3 \pmod 5$
  \item $x \equiv 5 \pmod 7$
  \item Then $M_1 = 5\cdot 7 = 35, M_2 = 3\cdot 7 = 21, M_3 = 3\cdot 5 = 15$.
  \item $\gcd(M_1, m_1) = 1$ and $M_1 \equiv 0 \pmod {m_2}$ and $M_1 \equiv 0 \pmod {m_3}$.
  \item $\gcd(35,3) = 1$ and $35\equiv 0 \pmod 5$ and $35 \equiv 0 \pmod 7$.
  \item Also $\gcd(21,5) = 1$ and $21\equiv 0 \pmod 3$ and $21 \equiv 0 \pmod 7$.
  \item Also $\gcd(15,7) = 1$ and $15\equiv 0 \pmod 3$ and $15 \equiv 0 \pmod 5$.

\end{itemize}

\end{frame}


\begin{frame}{Proof, page 3}

\begin{itemize}
  \item For each $i$ because $\gcd(M_i, m_i)=1$, $M_i$ has a unique multiplicative inverse mod $m_i$.
  \item Recall this means, there is a unique solution to
  \item $M_i x \equiv 1 \pmod {m_i}$.
  \item Let us call this unique multiplicative inverse $y_i$.
  \item So $M_i y_i \equiv 1 \pmod {m_i}$.
  \item To find $y_i$ we can try every value from 1 to $m_i-1$, or we can use the Euclidean algorithm.
\end{itemize}

\end{frame}

\begin{frame}{Proof, page 4}

\begin{itemize}
  \item Using our example problem:
  \item $x \equiv 1 \pmod 3$
  \item $x \equiv 3 \pmod 5$
  \item $x \equiv 5 \pmod 7$
  \item we have $M_1 = 5\cdot 7 = 35, M_2 = 3\cdot 7 = 21, M_3 = 3\cdot 5 = 15$.
  \item Then $y_1$ is the multiplicative inverse of $35$ mod $3$,
  \item $y_2$ is the multiplicative inverse of $21$ mod $5$,
  \item and $y_3$ is the multiplicative inverse of $15$ mod $7$.
  \item $35 \bmod 3 =2$ and $2\cdot 2 \equiv 1 \pmod 3$. So $y_1 = 2$.
  \item $21 \bmod 5 = 1$ and  $1\cdot 1 \equiv 1 \pmod 5$. So $y_2 = 1$.
  \item $15 \bmod 7 = 1$ and $1\cdot 1 \equiv 1 \pmod 7$. So $y_3 = 1$.
\end{itemize}

\end{frame}

\begin{frame}{Proof, page 5}

\begin{itemize}
  \item Let $x = a_1 M_1 y_1 + a_2 M_2 y_2 + \cdots + a_r M_r y_r$.
  \item Let us focus on one fixed $i$.
  \item For $j\not=i$, $a_j M_j y_j \equiv 0 \pmod {m_i}$,
  \item while $a_i M_i y_i \equiv a_i \pmod {m_i}$.
  \item So $x \equiv a_i \pmod {m_i}$.
  \item So $x$ is a solution.
\end{itemize}

\end{frame}

\begin{frame}{Proof, page 6}

\begin{itemize}
  \item Using our example problem:
  \item $x \equiv 1 \pmod 3$
  \item $x \equiv 3 \pmod 5$
  \item $x \equiv 5 \pmod 7$
  \item we computed $y_1 = 2, y_2 = 1, y_3 = 1$.
  \item Let $x = a_1 M_1 y_1 + a_2 M_2 y_2 + a_3 M_3 y_3$,
  \item $x = 1 \cdot 35 \cdot 2 + 3 \cdot 21 \cdot 1 + 5 \cdot 15 \cdot 1$.
  \item $1 \cdot 35 \cdot 2 \equiv 1 \pmod 3$,
  \item $3 \cdot 21 \cdot 1 \equiv 0 \pmod 3$,
  \item $5 \cdot 15 \cdot 1 \equiv 0 \pmod 3$,
  \item So $x \equiv 1 \pmod 3$.
  \item So $x$ satisfies the first congruence.
\end{itemize}

\end{frame}

\begin{frame}{Proof, page 7}

\begin{itemize}
  \item $x \equiv 1 \pmod 3$
  \item $x \equiv 3 \pmod 5$
  \item $x \equiv 5 \pmod 7$
  \item Let $x = 1 \cdot 35 \cdot 2 + 3 \cdot 21 \cdot 1 + 5 \cdot 15 \cdot 1$.
  \item $1 \cdot 35 \cdot 2 \equiv 0 \pmod 5$,
  \item $3 \cdot 21 \cdot 1 \equiv 3 \pmod 5$,
  \item $5 \cdot 15 \cdot 1 \equiv 0 \pmod 5$,
  \item So $x \equiv 3 \pmod 5$.
  \item So $x$ satisfies the second congruence.
\end{itemize}

\end{frame}

\begin{frame}{Proof, page 8}

\begin{itemize}
  \item $x \equiv 1 \pmod 3$
  \item $x \equiv 3 \pmod 5$
  \item $x \equiv 5 \pmod 7$
  \item $x = 1 \cdot 35 \cdot 2 + 3 \cdot 21 \cdot 1 + 5 \cdot 15 \cdot 1$.
  \item $1 \cdot 35 \cdot 2 \equiv 0 \pmod 7$,
  \item $3 \cdot 21 \cdot 1 \equiv 0 \pmod 7$,
  \item $5 \cdot 15 \cdot 1 \equiv 5 \pmod 7$,
  \item So $x \equiv 5 \pmod 7$.
  \item So $x$ satisfies all three congruences.
  \item So $x$ is a solution to our problem.
\end{itemize}

\end{frame}

\begin{frame}{Proof, page 9}

\begin{itemize}
  \item $x \equiv 1 \pmod 3$
  \item $x \equiv 3 \pmod 5$
  \item $x \equiv 5 \pmod 7$
  \item $x = 1 \cdot 35 \cdot 2 + 3 \cdot 21 \cdot 1 + 5 \cdot 15 \cdot 1$.
  \item $x = 70 + 63 + 75 = 208$.
  \item So $208$ is a solution to our problem.
  \item But we want to express the answer mod 105.
  \item $208 \bmod 105 = 103$.
  \item So $x=103$ is the unique solution with $0\leq x < 105$.
  \item Check: $103 \bmod 3 = 1$, $103 \bmod  5 = 3$, $103\bmod 7 = 5$.
  \item $\qed$.
\end{itemize}

\end{frame}

\begin{frame}{Example}

\begin{itemize}
  \item Find the unique $x$ with $0\leq x < 315$ such that
  \item $x \equiv 1 \pmod 5$
  \item $x \equiv 3 \pmod 7$
  \item $x \equiv 5 \pmod 9$
  \item $M_1 = 63$. Find $y_1$.
  \item $63 \bmod 5 = 3$ and $3\cdot 2 \equiv 1 \pmod 5$ so $y_1 = 2$.
  \item $M_2 = 45$. Find $y_2$.
  \item $45 \bmod 7 = 3$ and  $3 \cdot 5 \equiv 1 \pmod 7$ so  $y_2 = 5$.
  \item $M_3 = 35$. Find $y_3$.
  \item $35 \bmod 9 = 8$ and $8 \cdot 8 \equiv 1 \pmod 9$ so $y_3 = 8$.
\end{itemize}

\end{frame}

\begin{frame}{Example, continued}

\begin{itemize}
  \item $x \equiv 1 \pmod 5$
  \item $x \equiv 3 \pmod 7$
  \item $x \equiv 5 \pmod 9$
  \item $M_1 = 63$. $y_1 = 2$
  \item $M_2 = 45$. $y_2 = 5$.
  \item $M_3 = 35$. $y_3 = 8$.
  \item $x = 1 \cdot 63 \cdot 2 + 3 \cdot 45 \cdot 5 + 5 \cdot 35 \cdot 8 $
  \item $= 126 + 675 + 1400 = 2201$.
  \item The problem said we were supposed to find $x$ with $0\leq x < 315$.
  \item $2201\bmod 315 = 311$. $x=311$ is the final answer.
  \item Check: $311 \bmod 5 = 1$ ; $311 \bmod 7 = 3$ ; $311 \bmod 9 = 5$.
\end{itemize}

\end{frame}







\end{document}
