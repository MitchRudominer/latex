% $Header$

\documentclass{beamer}
%\documentclass[handout]{beamer}
\usepackage{amsmath,amssymb,latexsym,eucal,amsthm,graphicx,hyperref}
%%%%%%%%%%%%%%%%%%%%%%%%%%%%%%%%%%%%%%%%%%%%%
% Common Set Theory Constructs
%%%%%%%%%%%%%%%%%%%%%%%%%%%%%%%%%%%%%%%%%%%%%

\newcommand{\setof}[2]{\left\{ \, #1 \, \left| \, #2 \, \right.\right\}}
\newcommand{\lsetof}[2]{\left\{\left. \, #1 \, \right| \, #2 \,  \right\}}
\newcommand{\bigsetof}[2]{\bigl\{ \, #1 \, \bigm | \, #2 \,\bigr\}}
\newcommand{\Bigsetof}[2]{\Bigl\{ \, #1 \, \Bigm | \, #2 \,\Bigr\}}
\newcommand{\biggsetof}[2]{\biggl\{ \, #1 \, \biggm | \, #2 \,\biggr\}}
\newcommand{\Biggsetof}[2]{\Biggl\{ \, #1 \, \Biggm | \, #2 \,\Biggr\}}
\newcommand{\dotsetof}[2]{\left\{ \, #1 \, : \, #2 \, \right\}}
\newcommand{\bigdotsetof}[2]{\bigl\{ \, #1 \, : \, #2 \,\bigr\}}
\newcommand{\Bigdotsetof}[2]{\Bigl\{ \, #1 \, \Bigm : \, #2 \,\Bigr\}}
\newcommand{\biggdotsetof}[2]{\biggl\{ \, #1 \, \biggm : \, #2 \,\biggr\}}
\newcommand{\Biggdotsetof}[2]{\Biggl\{ \, #1 \, \Biggm : \, #2 \,\Biggr\}}
\newcommand{\sequence}[2]{\left\langle \, #1 \,\left| \, #2 \, \right. \right\rangle}
\newcommand{\lsequence}[2]{\left\langle\left. \, #1 \, \right| \,#2 \,  \right\rangle}
\newcommand{\bigsequence}[2]{\bigl\langle \,#1 \, \bigm | \, #2 \, \bigr\rangle}
\newcommand{\Bigsequence}[2]{\Bigl\langle \,#1 \, \Bigm | \, #2 \, \Bigr\rangle}
\newcommand{\biggsequence}[2]{\biggl\langle \,#1 \, \biggm | \, #2 \, \biggr\rangle}
\newcommand{\Biggsequence}[2]{\Biggl\langle \,#1 \, \Biggm | \, #2 \, \Biggr\rangle}
\newcommand{\singleton}[1]{\left\{#1\right\}}
\newcommand{\angles}[1]{\left\langle #1 \right\rangle}
\newcommand{\bigangles}[1]{\bigl\langle #1 \bigr\rangle}
\newcommand{\Bigangles}[1]{\Bigl\langle #1 \Bigr\rangle}
\newcommand{\biggangles}[1]{\biggl\langle #1 \biggr\rangle}
\newcommand{\Biggangles}[1]{\Biggl\langle #1 \Biggr\rangle}


\newcommand{\force}[1]{\Vert\!\underset{\!\!\!\!\!#1}{\!\!\!\relbar\!\!\!%
\relbar\!\!\relbar\!\!\relbar\!\!\!\relbar\!\!\relbar\!\!\relbar\!\!\!%
\relbar\!\!\relbar\!\!\relbar}}
\newcommand{\longforce}[1]{\Vert\!\underset{\!\!\!\!\!#1}{\!\!\!\relbar\!\!\!%
\relbar\!\!\relbar\!\!\relbar\!\!\!\relbar\!\!\relbar\!\!\relbar\!\!\!%
\relbar\!\!\relbar\!\!\relbar\!\!\relbar\!\!\relbar\!\!\relbar\!\!\relbar\!\!\relbar}}
\newcommand{\nforce}[1]{\Vert\!\underset{\!\!\!\!\!#1}{\!\!\!\relbar\!\!\!%
\relbar\!\!\relbar\!\!\relbar\!\!\!\relbar\!\!\relbar\!\!\relbar\!\!\!%
\relbar\!\!\not\relbar\!\!\relbar}}
\newcommand{\forcein}[2]{\overset{#2}{\Vert\underset{\!\!\!\!\!#1}%
{\!\!\!\relbar\!\!\!\relbar\!\!\relbar\!\!\relbar\!\!\!\relbar\!\!\relbar\!%
\!\relbar\!\!\!\relbar\!\!\relbar\!\!\relbar\!\!\relbar\!\!\!\relbar\!\!%
\relbar\!\!\relbar}}}

\newcommand{\pre}[2]{{}^{#2}{#1}}

\newcommand{\restr}{\!\!\upharpoonright\!}

%%%%%%%%%%%%%%%%%%%%%%%%%%%%%%%%%%%%%%%%%%%%%
% Set-Theoretic Connectives
%%%%%%%%%%%%%%%%%%%%%%%%%%%%%%%%%%%%%%%%%%%%%

\newcommand{\intersect}{\cap}
\newcommand{\union}{\cup}
\newcommand{\Intersection}[1]{\bigcap\limits_{#1}}
\newcommand{\Union}[1]{\bigcup\limits_{#1}}
\newcommand{\adjoin}{{}^\frown}
\newcommand{\forces}{\Vdash}

%%%%%%%%%%%%%%%%%%%%%%%%%%%%%%%%%%%%%%%%%%%%%
% Miscellaneous
%%%%%%%%%%%%%%%%%%%%%%%%%%%%%%%%%%%%%%%%%%%%%
\newcommand{\defeq}{=_{\text{def}}}
\newcommand{\Turingleq}{\leq_{\text{T}}}
\newcommand{\Turingless}{<_{\text{T}}}
\newcommand{\lexleq}{\leq_{\text{lex}}}
\newcommand{\lexless}{<_{\text{lex}}}
\newcommand{\Turingequiv}{\equiv_{\text{T}}}
\newcommand{\isomorphic}{\cong}

%%%%%%%%%%%%%%%%%%%%%%%%%%%%%%%%%%%%%%%%%%%%%
% Constants
%%%%%%%%%%%%%%%%%%%%%%%%%%%%%%%%%%%%%%%%%%%%%
\newcommand{\R}{\mathbb{R}}
\renewcommand{\P}{\mathbb{P}}
\newcommand{\Q}{\mathbb{Q}}
\newcommand{\Z}{\mathbb{Z}}
\newcommand{\Zpos}{\Z^{+}}
\newcommand{\Znonneg}{\Z^{\geq 0}}
\newcommand{\C}{\mathbb{C}}
\newcommand{\N}{\mathbb{N}}
\newcommand{\B}{\mathbb{B}}
\newcommand{\Bairespace}{\pre{\omega}{\omega}}
\newcommand{\LofR}{L(\R)}
\newcommand{\JofR}[1]{J_{#1}(\R)}
\newcommand{\SofR}[1]{S_{#1}(\R)}
\newcommand{\JalphaR}{\JofR{\alpha}}
\newcommand{\JbetaR}{\JofR{\beta}}
\newcommand{\JlambdaR}{\JofR{\lambda}}
\newcommand{\SalphaR}{\SofR{\alpha}}
\newcommand{\SbetaR}{\SofR{\beta}}
\newcommand{\Pkl}{\mathcal{P}_{\kappa}(\lambda)}
\DeclareMathOperator{\con}{con}
\DeclareMathOperator{\ORD}{OR}
\DeclareMathOperator{\Ord}{OR}
\DeclareMathOperator{\WO}{WO}
\DeclareMathOperator{\OD}{OD}
\DeclareMathOperator{\HOD}{HOD}
\DeclareMathOperator{\HC}{HC}
\DeclareMathOperator{\WF}{WF}
\DeclareMathOperator{\wfp}{wfp}
\DeclareMathOperator{\HF}{HF}
\newcommand{\One}{I}
\newcommand{\ONE}{I}
\newcommand{\Two}{II}
\newcommand{\TWO}{II}
\newcommand{\Mladder}{M^{\text{ld}}}

%%%%%%%%%%%%%%%%%%%%%%%%%%%%%%%%%%%%%%%%%%%%%
% Commutative Algebra Constants
%%%%%%%%%%%%%%%%%%%%%%%%%%%%%%%%%%%%%%%%%%%%%
\DeclareMathOperator{\dottimes}{\dot{\times}}
\DeclareMathOperator{\dotminus}{\dot{-}}
\DeclareMathOperator{\Spec}{Spec}

%%%%%%%%%%%%%%%%%%%%%%%%%%%%%%%%%%%%%%%%%%%%%
% Theories
%%%%%%%%%%%%%%%%%%%%%%%%%%%%%%%%%%%%%%%%%%%%%
\DeclareMathOperator{\ZFC}{ZFC}
\DeclareMathOperator{\ZF}{ZF}
\DeclareMathOperator{\AD}{AD}
\DeclareMathOperator{\ADR}{AD_{\R}}
\DeclareMathOperator{\KP}{KP}
\DeclareMathOperator{\PD}{PD}
\DeclareMathOperator{\CH}{CH}
\DeclareMathOperator{\GCH}{GCH}
\DeclareMathOperator{\HPC}{HPC} % HOD pair capturing
%%%%%%%%%%%%%%%%%%%%%%%%%%%%%%%%%%%%%%%%%%%%%
% Iteration Trees
%%%%%%%%%%%%%%%%%%%%%%%%%%%%%%%%%%%%%%%%%%%%%

\newcommand{\pred}{\text{-pred}}

%%%%%%%%%%%%%%%%%%%%%%%%%%%%%%%%%%%%%%%%%%%%%%%%
% Operator Names
%%%%%%%%%%%%%%%%%%%%%%%%%%%%%%%%%%%%%%%%%%%%%%%%
\DeclareMathOperator{\Det}{Det}
\DeclareMathOperator{\dom}{dom}
\DeclareMathOperator{\ran}{ran}
\DeclareMathOperator{\range}{ran}
\DeclareMathOperator{\image}{image}
\DeclareMathOperator{\crit}{crit}
\DeclareMathOperator{\card}{card}
\DeclareMathOperator{\cf}{cf}
\DeclareMathOperator{\cof}{cof}
\DeclareMathOperator{\rank}{rank}
\DeclareMathOperator{\ot}{o.t.}
\DeclareMathOperator{\ords}{o}
\DeclareMathOperator{\ro}{r.o.}
\DeclareMathOperator{\rud}{rud}
\DeclareMathOperator{\Powerset}{\mathcal{P}}
\DeclareMathOperator{\length}{lh}
\DeclareMathOperator{\lh}{lh}
\DeclareMathOperator{\limit}{lim}
\DeclareMathOperator{\fld}{fld}
\DeclareMathOperator{\projection}{p}
\DeclareMathOperator{\Ult}{Ult}
\DeclareMathOperator{\ULT}{Ult}
\DeclareMathOperator{\Coll}{Coll}
\DeclareMathOperator{\Col}{Col}
\DeclareMathOperator{\Hull}{Hull}
\DeclareMathOperator*{\dirlim}{dir lim}
\DeclareMathOperator{\Scale}{Scale}
\DeclareMathOperator{\supp}{supp}
\DeclareMathOperator{\trancl}{tran.cl.}
\DeclareMathOperator{\trace}{Tr}
\DeclareMathOperator{\diag}{diag}
\DeclareMathOperator{\spn}{span}
\DeclareMathOperator{\sgn}{sgn}
%%%%%%%%%%%%%%%%%%%%%%%%%%%%%%%%%%%%%%%%%%%%%
% Logical Connectives
%%%%%%%%%%%%%%%%%%%%%%%%%%%%%%%%%%%%%%%%%%%%%
\newcommand{\IImplies}{\Longrightarrow}
\newcommand{\SkipImplies}{\quad\Longrightarrow\quad}
\newcommand{\Ifff}{\Longleftrightarrow}
\newcommand{\iimplies}{\longrightarrow}
\newcommand{\ifff}{\longleftrightarrow}
\newcommand{\Implies}{\Rightarrow}
\newcommand{\Iff}{\Leftrightarrow}
\renewcommand{\implies}{\rightarrow}
\renewcommand{\iff}{\leftrightarrow}
\newcommand{\AND}{\wedge}
\newcommand{\OR}{\vee}
\newcommand{\st}{\text{ s.t. }}
\newcommand{\Or}{\text{ or }}

%%%%%%%%%%%%%%%%%%%%%%%%%%%%%%%%%%%%%%%%%%%%%
% Function Arrows
%%%%%%%%%%%%%%%%%%%%%%%%%%%%%%%%%%%%%%%%%%%%%

\newcommand{\injection}{\xrightarrow{\text{1-1}}}
\newcommand{\surjection}{\xrightarrow{\text{onto}}}
\newcommand{\bijection}{\xrightarrow[\text{onto}]{\text{1-1}}}
\newcommand{\cofmap}{\xrightarrow{\text{cof}}}
\newcommand{\map}{\rightarrow}

%%%%%%%%%%%%%%%%%%%%%%%%%%%%%%%%%%%%%%%%%%%%%
% Mouse Comparison Operators
%%%%%%%%%%%%%%%%%%%%%%%%%%%%%%%%%%%%%%%%%%%%%
\newcommand{\initseg}{\trianglelefteq}
\newcommand{\properseg}{\lhd}
\newcommand{\notinitseg}{\ntrianglelefteq}
\newcommand{\notproperseg}{\ntriangleleft}

%%%%%%%%%%%%%%%%%%%%%%%%%%%%%%%%%%%%%%%%%%%%%
%           calligraphic letters
%%%%%%%%%%%%%%%%%%%%%%%%%%%%%%%%%%%%%%%%%%%%%
\newcommand{\cA}{\mathcal{A}}
\newcommand{\cB}{\mathcal{B}}
\newcommand{\cC}{\mathcal{C}}
\newcommand{\cD}{\mathcal{D}}
\newcommand{\cE}{\mathcal{E}}
\newcommand{\cF}{\mathcal{F}}
\newcommand{\cG}{\mathcal{G}}
\newcommand{\cH}{\mathcal{H}}
\newcommand{\cI}{\mathcal{I}}
\newcommand{\cJ}{\mathcal{J}}
\newcommand{\cK}{\mathcal{K}}
\newcommand{\cL}{\mathcal{L}}
\newcommand{\cM}{\mathcal{M}}
\newcommand{\cN}{\mathcal{N}}
\newcommand{\cO}{\mathcal{O}}
\newcommand{\cP}{\mathcal{P}}
\newcommand{\cQ}{\mathcal{Q}}
\newcommand{\cR}{\mathcal{R}}
\newcommand{\cS}{\mathcal{S}}
\newcommand{\cT}{\mathcal{T}}
\newcommand{\cU}{\mathcal{U}}
\newcommand{\cV}{\mathcal{V}}
\newcommand{\cW}{\mathcal{W}}
\newcommand{\cX}{\mathcal{X}}
\newcommand{\cY}{\mathcal{Y}}
\newcommand{\cZ}{\mathcal{Z}}


%%%%%%%%%%%%%%%%%%%%%%%%%%%%%%%%%%%%%%%%%%%%%
%          Primed Letters
%%%%%%%%%%%%%%%%%%%%%%%%%%%%%%%%%%%%%%%%%%%%%
\newcommand{\aprime}{a^{\prime}}
\newcommand{\bprime}{b^{\prime}}
\newcommand{\cprime}{c^{\prime}}
\newcommand{\dprime}{d^{\prime}}
\newcommand{\eprime}{e^{\prime}}
\newcommand{\fprime}{f^{\prime}}
\newcommand{\gprime}{g^{\prime}}
\newcommand{\hprime}{h^{\prime}}
\newcommand{\iprime}{i^{\prime}}
\newcommand{\jprime}{j^{\prime}}
\newcommand{\kprime}{k^{\prime}}
\newcommand{\lprime}{l^{\prime}}
\newcommand{\mprime}{m^{\prime}}
\newcommand{\nprime}{n^{\prime}}
\newcommand{\oprime}{o^{\prime}}
\newcommand{\pprime}{p^{\prime}}
\newcommand{\qprime}{q^{\prime}}
\newcommand{\rprime}{r^{\prime}}
\newcommand{\sprime}{s^{\prime}}
\newcommand{\tprime}{t^{\prime}}
\newcommand{\uprime}{u^{\prime}}
\newcommand{\vprime}{v^{\prime}}
\newcommand{\wprime}{w^{\prime}}
\newcommand{\xprime}{x^{\prime}}
\newcommand{\yprime}{y^{\prime}}
\newcommand{\zprime}{z^{\prime}}
\newcommand{\Aprime}{A^{\prime}}
\newcommand{\Bprime}{B^{\prime}}
\newcommand{\Cprime}{C^{\prime}}
\newcommand{\Dprime}{D^{\prime}}
\newcommand{\Eprime}{E^{\prime}}
\newcommand{\Fprime}{F^{\prime}}
\newcommand{\Gprime}{G^{\prime}}
\newcommand{\Hprime}{H^{\prime}}
\newcommand{\Iprime}{I^{\prime}}
\newcommand{\Jprime}{J^{\prime}}
\newcommand{\Kprime}{K^{\prime}}
\newcommand{\Lprime}{L^{\prime}}
\newcommand{\Mprime}{M^{\prime}}
\newcommand{\Nprime}{N^{\prime}}
\newcommand{\Oprime}{O^{\prime}}
\newcommand{\Pprime}{P^{\prime}}
\newcommand{\Qprime}{Q^{\prime}}
\newcommand{\Rprime}{R^{\prime}}
\newcommand{\Sprime}{S^{\prime}}
\newcommand{\Tprime}{T^{\prime}}
\newcommand{\Uprime}{U^{\prime}}
\newcommand{\Vprime}{V^{\prime}}
\newcommand{\Wprime}{W^{\prime}}
\newcommand{\Xprime}{X^{\prime}}
\newcommand{\Yprime}{Y^{\prime}}
\newcommand{\Zprime}{Z^{\prime}}
\newcommand{\alphaprime}{\alpha^{\prime}}
\newcommand{\betaprime}{\beta^{\prime}}
\newcommand{\gammaprime}{\gamma^{\prime}}
\newcommand{\Gammaprime}{\Gamma^{\prime}}
\newcommand{\deltaprime}{\delta^{\prime}}
\newcommand{\epsilonprime}{\epsilon^{\prime}}
\newcommand{\kappaprime}{\kappa^{\prime}}
\newcommand{\lambdaprime}{\lambda^{\prime}}
\newcommand{\rhoprime}{\rho^{\prime}}
\newcommand{\Sigmaprime}{\Sigma^{\prime}}
\newcommand{\tauprime}{\tau^{\prime}}
\newcommand{\xiprime}{\xi^{\prime}}
\newcommand{\thetaprime}{\theta^{\prime}}
\newcommand{\Omegaprime}{\Omega^{\prime}}
\newcommand{\cMprime}{\cM^{\prime}}
\newcommand{\cNprime}{\cN^{\prime}}
\newcommand{\cPprime}{\cP^{\prime}}
\newcommand{\cQprime}{\cQ^{\prime}}
\newcommand{\cRprime}{\cR^{\prime}}
\newcommand{\cSprime}{\cS^{\prime}}
\newcommand{\cTprime}{\cT^{\prime}}

%%%%%%%%%%%%%%%%%%%%%%%%%%%%%%%%%%%%%%%%%%%%%
%          bar Letters
%%%%%%%%%%%%%%%%%%%%%%%%%%%%%%%%%%%%%%%%%%%%%
\newcommand{\abar}{\bar{a}}
\newcommand{\bbar}{\bar{b}}
\newcommand{\cbar}{\bar{c}}
\newcommand{\ibar}{\bar{i}}
\newcommand{\jbar}{\bar{j}}
\newcommand{\nbar}{\bar{n}}
\newcommand{\xbar}{\bar{x}}
\newcommand{\ybar}{\bar{y}}
\newcommand{\zbar}{\bar{z}}
\newcommand{\pibar}{\bar{\pi}}
\newcommand{\phibar}{\bar{\varphi}}
\newcommand{\psibar}{\bar{\psi}}
\newcommand{\thetabar}{\bar{\theta}}
\newcommand{\nubar}{\bar{\nu}}

%%%%%%%%%%%%%%%%%%%%%%%%%%%%%%%%%%%%%%%%%%%%%
%          star Letters
%%%%%%%%%%%%%%%%%%%%%%%%%%%%%%%%%%%%%%%%%%%%%
\newcommand{\phistar}{\phi^{*}}
\newcommand{\Mstar}{M^{*}}

%%%%%%%%%%%%%%%%%%%%%%%%%%%%%%%%%%%%%%%%%%%%%
%          dotletters Letters
%%%%%%%%%%%%%%%%%%%%%%%%%%%%%%%%%%%%%%%%%%%%%
\newcommand{\Gdot}{\dot{G}}

%%%%%%%%%%%%%%%%%%%%%%%%%%%%%%%%%%%%%%%%%%%%%
%         check Letters
%%%%%%%%%%%%%%%%%%%%%%%%%%%%%%%%%%%%%%%%%%%%%
\newcommand{\deltacheck}{\check{\delta}}
\newcommand{\gammacheck}{\check{\gamma}}


%%%%%%%%%%%%%%%%%%%%%%%%%%%%%%%%%%%%%%%%%%%%%
%          Formulas
%%%%%%%%%%%%%%%%%%%%%%%%%%%%%%%%%%%%%%%%%%%%%

\newcommand{\formulaphi}{\text{\large $\varphi$}}
\newcommand{\Formulaphi}{\text{\Large $\varphi$}}


%%%%%%%%%%%%%%%%%%%%%%%%%%%%%%%%%%%%%%%%%%%%%
%          Fraktur Letters
%%%%%%%%%%%%%%%%%%%%%%%%%%%%%%%%%%%%%%%%%%%%%

\newcommand{\fa}{\mathfrak{a}}
\newcommand{\fb}{\mathfrak{b}}
\newcommand{\fc}{\mathfrak{c}}
\newcommand{\fk}{\mathfrak{k}}
\newcommand{\fp}{\mathfrak{p}}
\newcommand{\fq}{\mathfrak{q}}
\newcommand{\fr}{\mathfrak{r}}
\newcommand{\fA}{\mathfrak{A}}
\newcommand{\fB}{\mathfrak{B}}
\newcommand{\fC}{\mathfrak{C}}
\newcommand{\fD}{\mathfrak{D}}

%%%%%%%%%%%%%%%%%%%%%%%%%%%%%%%%%%%%%%%%%%%%%
%          Bold Letters
%%%%%%%%%%%%%%%%%%%%%%%%%%%%%%%%%%%%%%%%%%%%%
\newcommand{\ba}{\mathbf{a}}
\newcommand{\bb}{\mathbf{b}}
\newcommand{\bc}{\mathbf{c}}
\newcommand{\bd}{\mathbf{d}}
\newcommand{\be}{\mathbf{e}}
\newcommand{\bu}{\mathbf{u}}
\newcommand{\bv}{\mathbf{v}}
\newcommand{\bw}{\mathbf{w}}
\newcommand{\bx}{\mathbf{x}}
\newcommand{\by}{\mathbf{y}}
\newcommand{\bz}{\mathbf{z}}
\newcommand{\bSigma}{\boldsymbol{\Sigma}}
\newcommand{\bPi}{\boldsymbol{\Pi}}
\newcommand{\bDelta}{\boldsymbol{\Delta}}
\newcommand{\bdelta}{\boldsymbol{\delta}}
\newcommand{\bgamma}{\boldsymbol{\gamma}}
\newcommand{\bGamma}{\boldsymbol{\Gamma}}

%%%%%%%%%%%%%%%%%%%%%%%%%%%%%%%%%%%%%%%%%%%%%
%         Bold numbers
%%%%%%%%%%%%%%%%%%%%%%%%%%%%%%%%%%%%%%%%%%%%%
\newcommand{\bzero}{\mathbf{0}}

%%%%%%%%%%%%%%%%%%%%%%%%%%%%%%%%%%%%%%%%%%%%%
% Projective-Like Pointclasses
%%%%%%%%%%%%%%%%%%%%%%%%%%%%%%%%%%%%%%%%%%%%%
\newcommand{\Sa}[2][\alpha]{\Sigma_{(#1,#2)}}
\newcommand{\Pa}[2][\alpha]{\Pi_{(#1,#2)}}
\newcommand{\Da}[2][\alpha]{\Delta_{(#1,#2)}}
\newcommand{\Aa}[2][\alpha]{A_{(#1,#2)}}
\newcommand{\Ca}[2][\alpha]{C_{(#1,#2)}}
\newcommand{\Qa}[2][\alpha]{Q_{(#1,#2)}}
\newcommand{\da}[2][\alpha]{\delta_{(#1,#2)}}
\newcommand{\leqa}[2][\alpha]{\leq_{(#1,#2)}}
\newcommand{\lessa}[2][\alpha]{<_{(#1,#2)}}
\newcommand{\equiva}[2][\alpha]{\equiv_{(#1,#2)}}


\newcommand{\Sl}[1]{\Sa[\lambda]{#1}}
\newcommand{\Pl}[1]{\Pa[\lambda]{#1}}
\newcommand{\Dl}[1]{\Da[\lambda]{#1}}
\newcommand{\Al}[1]{\Aa[\lambda]{#1}}
\newcommand{\Cl}[1]{\Ca[\lambda]{#1}}
\newcommand{\Ql}[1]{\Qa[\lambda]{#1}}

\newcommand{\San}{\Sa{n}}
\newcommand{\Pan}{\Pa{n}}
\newcommand{\Dan}{\Da{n}}
\newcommand{\Can}{\Ca{n}}
\newcommand{\Qan}{\Qa{n}}
\newcommand{\Aan}{\Aa{n}}
\newcommand{\dan}{\da{n}}
\newcommand{\leqan}{\leqa{n}}
\newcommand{\lessan}{\lessa{n}}
\newcommand{\equivan}{\equiva{n}}

\newcommand{\SigmaOneOmega}{\Sigma^1_{\omega}}
\newcommand{\SigmaOneOmegaPlusOne}{\Sigma^1_{\omega+1}}
\newcommand{\PiOneOmega}{\Pi^1_{\omega}}
\newcommand{\PiOneOmegaPlusOne}{\Pi^1_{\omega+1}}
\newcommand{\DeltaOneOmegaPlusOne}{\Delta^1_{\omega+1}}

%%%%%%%%%%%%%%%%%%%%%%%%%%%%%%%%%%%%%%%%%%%%%
% Linear Algebra
%%%%%%%%%%%%%%%%%%%%%%%%%%%%%%%%%%%%%%%%%%%%%
\newcommand{\transpose}[1]{{#1}^{\text{T}}}
\newcommand{\norm}[1]{\lVert{#1}\rVert}
\newcommand\aug{\fboxsep=-\fboxrule\!\!\!\fbox{\strut}\!\!\!}

%%%%%%%%%%%%%%%%%%%%%%%%%%%%%%%%%%%%%%%%%%%%%
% Number Theory
%%%%%%%%%%%%%%%%%%%%%%%%%%%%%%%%%%%%%%%%%%%%%
\newcommand{\av}[1]{\lvert#1\rvert}
\DeclareMathOperator{\divides}{\mid}
\DeclareMathOperator{\ndivides}{\nmid}
\DeclareMathOperator{\lcm}{lcm}
\DeclareMathOperator{\sign}{sign}
\newcommand{\floor}[1]{\left\lfloor{#1}\right\rfloor}
\DeclareMathOperator{\ConCl}{CC}
\DeclareMathOperator{\ord}{ord}



\graphicspath{{images/}}

\newtheorem*{claim}{claim}
\newtheorem*{observation}{Observation}
\newtheorem*{warning}{Warning}
\newtheorem*{question}{Question}
\newtheorem{remark}[theorem]{Remark}

\newenvironment*{subproof}[1][Proof]
{\begin{proof}[#1]}{\renewcommand{\qedsymbol}{$\diamondsuit$} \end{proof}}

\mode<presentation>
{
  \usetheme{Singapore}
  % or ...

  \setbeamercovered{invisible}
  % or whatever (possibly just delete it)
}


\usepackage[english]{babel}
% or whatever

\usepackage[latin1]{inputenc}
% or whatever

\usepackage{times}
\usepackage[T1]{fontenc}
% Or whatever. Note that the encoding and the font should match. If T1
% does not look nice, try deleting the line with the fontenc.

\title{Lesson 15 \\ $\Z_n$}
\subtitle{Math 310, Elementary Number Theory \\ Fall 2020 \\ SFSU}
\author{Mitch Rudominer}
\date{}



% If you wish to uncover everything in a step-wise fashion, uncomment
% the following command:

\beamerdefaultoverlayspecification{<+->}

\begin{document}

\begin{frame}
  \titlepage
\end{frame}

\begin{frame}{$\Z_n$}

\begin{itemize}
  \item Let $n>1$ be an integer.
  \item $\Z_n$ is pronounced $\Z$-mod-n.
  \item It is also written $\Z/(n)$.
  \item It consists of the  set of integers from $0$ to $n-1$.
  \item But in their role as the least non-negative residues.
  \item In other words in their role as ``slots'' in ``blocks'' of size $n$.
  \item To indicate this, will write square brackets around each integer.
  \item  $\Z_n=\singleton{[0],[1],[2],\cdots,[n-1]}$.
  \item For example:
  \item  $\Z_6=\singleton{[0],[1],[2],[3],[4],[5]}$.
\end{itemize}

\end{frame}

\begin{frame}{Addition in $\Z_n$}

\begin{itemize}
  \item We define addition in $\Z_n$ as follows:
  \item $[a] + [b] = [(a+b) \bmod n]$.
  \item For example, letting $n=6$, addition in $\Z_6$ is defined as:
  \item $[a] + [b] = [(a+b) \bmod 6]$.
  \item $[3] + [0] = [3]$.
  \item $[3] + [1] = [4]$.
  \item $[3] + [2] = [5]$.
  \item $[3] + [3] = [0]$.
  \item $[3] + [4] = [1]$.
  \item $[3] + [5] = [2]$.
\end{itemize}

\end{frame}

\begin{frame}{Multiplication in $\Z_n$}

\begin{itemize}
  \item We define multiplication in $\Z_n$ as follows:
  \item $[a] \cdot [b] = [(a\cdot b) \bmod n]$.
  \item For example, letting $n=6$, multplication in $\Z_6$ is defined as:
  \item $[a] \cdot [b] = [(a\cdot b) \bmod 6]$.
  \item $[3] \cdot [0] = [0]$.
  \item $[3] \cdot [1] = [3]$.
  \item $[3] \cdot [2] = [0]$.
  \item $[3] \cdot [3] = [3]$.
  \item $[3] \cdot [4] = [0]$.
  \item $[3] \cdot [5] = [3]$.
\end{itemize}

\end{frame}

\begin{frame}{Self-contained arithmetic universe}

\begin{itemize}
  \item This is all just another way of thinking about ideas we already learned.
  \item Let's continue to work in $\Z_6$.
  \item $[3] + [5] = [2]$.
  \item This is just another way of looking at: $3+5 \equiv 2 \pmod 6$.
  \item $[3] \cdot [5] = [3]$.
  \item This is just another way of looking at: $3\cdot 5 \equiv 3 \pmod 6$.
  \item What's new is that we are now thinking of $\Z_6$ as a closed, self-contained universe.
  \item It is a set of six elements. When you add or multiply two elements of the
  set you get another element of the set.
  \item We don't need the rest of the integers when we think about this.
  \item The arithmetic is just that of ``wrapping around.''
  \item i.e. $[5] + [1] = [0]$.
\end{itemize}

\end{frame}

\begin{frame}{How to indicate the modulus}
\begin{itemize}
  \item If there is any chance of confusion about which modulus $n$ we mean, we will write it as a subscript.
  \item For example:
  \item  $\Z_6=\singleton{[0]_6,[1]_6,[2]_6,[3]_6,[4]_6,[5]_6}$.
  \item  $\Z_3=\singleton{[0]_4,[1]_4,[2]_4,[3]_4}$.
  \item So $[3]_6 \cdot [3]_6 = [3]_6$, whereas $[3]_4 \cdot [3]_4 = [1]_4$.
   \item We will never try to add or multiply one element from $\Z_n$ and one element from $\Z_m$ when $n\not=m$.
  \item This does not make sense: $[3]_6 + [3]_4$.
  \item Because $[3]_6\in\Z_6$ and $[3]_4\in\Z_4$. They are in different ``universes.''
  \item Usually, instead of the subscripts we will make it clear from the context what the modulus $n$ is.
  \item We will say, for example, ``We are working in $\Z_6$''.
  \item or ``Using arithmetic in $\Z_6$...''
\end{itemize}
\end{frame}

\begin{frame}{Exponents}
\begin{itemize}
  \item If $[a]\in\Z_n$ and $k$ is a positive integer, then $[a]^k$ means
  \item $[a]\cdot[a]\cdots[a]$, where there are $k$ terms.
  \item Working in $\Z_5$...
  \item $[4]^6=[4]\cdot[4]\cdot[4]\cdot[4]\cdot[4]\cdot[4]$
  \item $=[1]\cdot[4]\cdot[4]\cdot[4]\cdot[4]$
  \item $=[1]\cdot[1]\cdot[4]\cdot[4]$
  \item $=[1]\cdot[1]\cdot[1]$
  \item $=[1]$
  \item Notice the exponent 6 is not in $\Z_5$. It is just a regular integer.
\end{itemize}
\end{frame}

\begin{frame}{Additive Inverses}
\begin{itemize}
  \item If $[a]$ and $[b]$ are in $\Z_n$
  \item and $[a]+[b] = [0]$
  \item then we say that $[a]$ and $[b]$ are additive inverses.
  \item We write $[a] = -[b]$ and $[b] = -[a]$.
  \item For example, working in $\Z_7$...
  \item $[3] = -[4]$ and $[4] = -[3]$.
  \item In general, it is easy to see that, working in $\Z_n$,
  \item $-[a] = [n-a]$.
  \item Notice that $-(-[a]) = [a]$.
  \item Important: $-[a]$ is not an additional element of $\Z_n$. It is another name for one of the elements.
  \item For example, in $\Z_7$, $-[4]$ is another name for $[3]$. It's not an additional element.
\end{itemize}
\end{frame}

\begin{frame}{Subtraction}
\begin{itemize}
  \item We can also define subtraction in $\Z_n$.
  \item $[a] - [b] = [(a-b) \bmod n]$.
  \item This is equivalent to
  \item $[a] - [b] = [a] + (-[b])$.
  \item For example, working in $\Z_9$...
  \item $[5] - [8] = [6]$.
  \item We can see this a few different ways:
  \item Explanation 1: $[5]-[8] = [5-8 \bmod 9] = [-3 \bmod 9] = [6]$
  \item Explanation 2: $[5] - [8] = [5] + (-[8]) = [5] + [1] = [6]$.
  \item Explanation 3: $[5] - [8] = -([8] - [5]) = -[3] = [6]$.
  \item Notice that arithmetic in $\Z_n$ ``wraps around'' the bottom as well as the top.
  \item i.e. working in $\Z_9$, $[0] - [1] = [8]$.
\end{itemize}
\end{frame}

\begin{frame}{$\Z_n$ is a commutative ring}
\begin{itemize}
  \item $[0]$ is an additive identity
  \item Every element has an additive inverse.
  \item Addition is commutative: $[a]+[b]=[b]+[a]$.
  \item Addition is associative: $([a]+[b])+[c]=[a]+([b]+[c])$.
  \item $[1]$ is a multiplicative identity.
  \item Multiplication is commutative: $[a]\cdot[b]=[b]\cdot[a]$.
  \item Multiplication is associative: $([a]\cdot[b])\cdot[c]=[a]\cdot([b]\cdot[c])$.
  \item Distributive law: $[a]\cdot([b]+[c]) = ([a]\cdot[b])+([a]\cdot [c])$.
\end{itemize}
\end{frame}

\begin{frame}{Multiplying with negatives}
\begin{itemize}
  \item Multiplying with negatives works the same way it does with integers.
  \item $(-[a])\cdot [b] = -([a]\cdot [b])$
  \item For example, working in $\Z_5$...
  \item Compute $-[2] \cdot [3]$.
  \item $-[2] \cdot [3] = -([2]\cdot[3]) = -[1] = [4]$
  \item Another way to see it: $-[2] = [3]$, $[3]\cdot[3] = [4]$.
  \item Compute $-[2]\cdot-[3]$.
  \item Solution: $-[2]\cdot-[3] = -([2]\cdot (-[3])) = --([2]\cdot[3]) = [1]$.
  \item Another way to see it: $-[2] = [3]$ and $-[3]=[2]$. So $-[2]\cdot-[3] = [3]\cdot[2] = [1]$.
\end{itemize}
\end{frame}

\begin{frame}{Negative one}
\begin{itemize}
  \item In any $\Z_n$, $-[1] = [n-1]$.
  \item For example, in $\Z_{100}$, $-[1] = [99]$.
  \item We can use arithmetic in $\Z_n$ to help us understand modular arithmetic in the integers.
  \item \textbf{Theorem.} Let $n>1$. Then $(n-1)^2 \equiv 1 \pmod n$.
  \item Example: $99^2 \equiv 1 \pmod {100}$.
  \item We can see this must be true using $\Z_n$ because $-[1]\cdot-[1] = [1]$.
  \item We could also prove it directly using congruence on the integers, without resorting to $\Z_n$:
  \item $(n-1)^2 = n^2 -2n + 1 \equiv 1 \pmod n$.
\end{itemize}
\end{frame}

\begin{frame}{Multiplicative Inverses}
\begin{itemize}
  \item If $[a]$ and $[b]$ are in $\Z_n$
  \item and $[a]\cdot [b] = [1]$
  \item then $[a]$ and $[b]$ are called multiplicative inverses.
  \item We write $[a] = [b]^{-1}$ and $[b] = [a]^{-1}$.
  \item For example, working in $\Z_7$...
  \item $[2]\cdot[4] = [1]$.
  \item $[2]^{-1} = [4]$ and $[4]^{-1} = [2]$.
  \item It's always true in any $\Z_n$ that $[1]\cdot[1] = [1]$,
  \item and that $-[1]\cdot-[1] = [1]$.
  \item So $[1]^{-1} = [1]$ and $(-[1])^{-1} = -[1]$.
  \item Working in $\Z_{100}$, $[99]^{-1}=[99]$.
\end{itemize}
\end{frame}

\begin{frame}{Units}
\begin{itemize}
  \item Sometimes $[a]^{-1}$ doesn't exist.
  \item If $[a]$ has a multiplicative inverse it is called a \emph{unit}.
  \item For example, working in $\Z_9$...
  \item $[4]\cdot [7] = [1]$ so $[4]^{-1} = [7]$ and $[7]^{-1} = [4]$.
  \item So in $\Z_9$, $[4]$ and $[7]$ are units.
  \item But $[3]$ doesn't have a multiplicative inverse in $\Z_9$.
  \item So $[3]$ is not a unit.
\end{itemize}
\end{frame}

\begin{frame}{Which are the units?}
\begin{itemize}
  \item Question: Which elements of $\Z_9$ are units?
  \item Notice that $[a]\cdot[x] = [1]$ in $\Z_9$ is the same as
  \item $ax \equiv 1 \pmod 9$.
  \item We know when there is a solution to this congruence.
  \item Answer: When $a$ is relatively prime to $9$.
  \item So the units of $\Z_9$ are: $[1],[2],[4],[5],[7],[8]$.
  \item In general:
  \item \textbf{Theorem.} $[a]$ is a unit of $\Z_n$ iff $\gcd(a, n) = 1$.
\end{itemize}
\end{frame}

\begin{frame}{Zero Divisors}
\begin{itemize}
  \item We work in $\Z_9$...
  \item Notice that $[3] \cdot [6] = [0]$.
  \item \textbf{Definition.} In any $\Z_n$ suppose $[a]$ and $[b]$ are non-zero
  \item and $[a]\cdot [b] = [0]$. Then $[a]$ and $[b]$ are called \emph{zero-divisors}.
  \item Which are the zero-divisors of $\Z_n$?
  \item $[a]$ is a zero-divisor of $\Z_n$ iff $a\not=0$ and
  \item there is a non-zero solution to $ax \equiv 0 \pmod n$.
  \item Let $d=\gcd(a,n)$. $d\divides 0$. So there are solutions.
  \item $x=0$ is one solution. $[a]$ is a zero-divisor iff there is a second solution.
  \item There are $d$ solutions. $[a]$ is a zero-divisor iff $d\not=1$.
  \item \textbf{Theorem.} $[a]$ is a zero-divisor of $\Z_n$ iff $a\not=0$ and $\gcd(a, n) \not= 1$.
\end{itemize}
\end{frame}

\begin{frame}{Zero Divisors and Units}
\begin{itemize}
  \item So every non-zero element of $\Z_n$ is either a unit or a zero-divisor.
  \item We work in $\Z_9$...
  \item $[1]$ is a unit. ($[1]$ is a unit in $\Z_n$ for every $n$.)
  \item $[2]$ is a unit. $[2]\cdot [5] = [1]$.
  \item $[3]$ is a zero-divisor. $[3]\cdot[6] = [0]$.
  \item $[4]$ is a unit. $[4]\cdot [7] = [1]$
  \item $[5]$ is a unit. $[5]\cdot [2] = [1]$
  \item $[6]$ is a zero-divisor. $[6]\cdot [3] = [0]$
  \item $[7]$ is a unit. $[7]\cdot [4] = [1]$
  \item $[8]$ is a unit. $[8]\cdot [8] = [1]$
  \item $[8] = -[1]$ and $-[1]$ is a unit in every $\Z_n$.
\end{itemize}
\end{frame}

\begin{frame}{Units and Zero-Divisors of $\Z$?}
\begin{itemize}
  \item The idea of units also makes sense for the integers.
  \item An integer $x$ is a \emph{unit} iff there is another integer $y$ such
  that $x\cdot y = 1$.
  \item Here we are using multiplication in $\Z$ (i.e. ordinary integer multiplication.)
  \item Which are the units of $\Z$?
  \item An integer $x$ is a zero-divisor of $\Z$ iff $x\not=0$ and there is a $y\not=0$
  such that $x\cdot y = 0$.
  \item Which are the zero-divisors of $\Z$?
\end{itemize}
\end{frame}







\end{document}
