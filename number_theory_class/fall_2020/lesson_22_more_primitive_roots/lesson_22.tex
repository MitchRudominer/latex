% $Header$

\documentclass{beamer}
%\documentclass[handout]{beamer}
\usepackage{amsmath,amssymb,latexsym,eucal,amsthm,graphicx,hyperref,changepage}
%%%%%%%%%%%%%%%%%%%%%%%%%%%%%%%%%%%%%%%%%%%%%
% Common Set Theory Constructs
%%%%%%%%%%%%%%%%%%%%%%%%%%%%%%%%%%%%%%%%%%%%%

\newcommand{\setof}[2]{\left\{ \, #1 \, \left| \, #2 \, \right.\right\}}
\newcommand{\lsetof}[2]{\left\{\left. \, #1 \, \right| \, #2 \,  \right\}}
\newcommand{\bigsetof}[2]{\bigl\{ \, #1 \, \bigm | \, #2 \,\bigr\}}
\newcommand{\Bigsetof}[2]{\Bigl\{ \, #1 \, \Bigm | \, #2 \,\Bigr\}}
\newcommand{\biggsetof}[2]{\biggl\{ \, #1 \, \biggm | \, #2 \,\biggr\}}
\newcommand{\Biggsetof}[2]{\Biggl\{ \, #1 \, \Biggm | \, #2 \,\Biggr\}}
\newcommand{\dotsetof}[2]{\left\{ \, #1 \, : \, #2 \, \right\}}
\newcommand{\bigdotsetof}[2]{\bigl\{ \, #1 \, : \, #2 \,\bigr\}}
\newcommand{\Bigdotsetof}[2]{\Bigl\{ \, #1 \, \Bigm : \, #2 \,\Bigr\}}
\newcommand{\biggdotsetof}[2]{\biggl\{ \, #1 \, \biggm : \, #2 \,\biggr\}}
\newcommand{\Biggdotsetof}[2]{\Biggl\{ \, #1 \, \Biggm : \, #2 \,\Biggr\}}
\newcommand{\sequence}[2]{\left\langle \, #1 \,\left| \, #2 \, \right. \right\rangle}
\newcommand{\lsequence}[2]{\left\langle\left. \, #1 \, \right| \,#2 \,  \right\rangle}
\newcommand{\bigsequence}[2]{\bigl\langle \,#1 \, \bigm | \, #2 \, \bigr\rangle}
\newcommand{\Bigsequence}[2]{\Bigl\langle \,#1 \, \Bigm | \, #2 \, \Bigr\rangle}
\newcommand{\biggsequence}[2]{\biggl\langle \,#1 \, \biggm | \, #2 \, \biggr\rangle}
\newcommand{\Biggsequence}[2]{\Biggl\langle \,#1 \, \Biggm | \, #2 \, \Biggr\rangle}
\newcommand{\singleton}[1]{\left\{#1\right\}}
\newcommand{\angles}[1]{\left\langle #1 \right\rangle}
\newcommand{\bigangles}[1]{\bigl\langle #1 \bigr\rangle}
\newcommand{\Bigangles}[1]{\Bigl\langle #1 \Bigr\rangle}
\newcommand{\biggangles}[1]{\biggl\langle #1 \biggr\rangle}
\newcommand{\Biggangles}[1]{\Biggl\langle #1 \Biggr\rangle}


\newcommand{\force}[1]{\Vert\!\underset{\!\!\!\!\!#1}{\!\!\!\relbar\!\!\!%
\relbar\!\!\relbar\!\!\relbar\!\!\!\relbar\!\!\relbar\!\!\relbar\!\!\!%
\relbar\!\!\relbar\!\!\relbar}}
\newcommand{\nforce}[1]{\Vert\!\underset{\!\!\!\!\!#1}{\!\!\!\relbar\!\!\!%
\relbar\!\!\relbar\!\!\relbar\!\!\!\relbar\!\!\relbar\!\!\relbar\!\!\!%
\relbar\!\!\not\relbar\!\!\relbar}}
\newcommand{\forcein}[2]{\overset{#2}{\Vert\underset{\!\!\!\!\!#1}%
{\!\!\!\relbar\!\!\!\relbar\!\!\relbar\!\!\relbar\!\!\!\relbar\!\!\relbar\!%
\!\relbar\!\!\!\relbar\!\!\relbar\!\!\relbar\!\!\relbar\!\!\!\relbar\!\!%
\relbar\!\!\relbar}}}

\newcommand{\pre}[2]{{}^{#2}\!{#1}}

\newcommand{\restr}{\!\!\upharpoonright\!}

%%%%%%%%%%%%%%%%%%%%%%%%%%%%%%%%%%%%%%%%%%%%%
% Set-Theoretic Connectives
%%%%%%%%%%%%%%%%%%%%%%%%%%%%%%%%%%%%%%%%%%%%%

\newcommand{\intersect}{\cap}
\newcommand{\union}{\cup}
\newcommand{\Intersection}[1]{\bigcap\limits_{#1}}
\newcommand{\Union}[1]{\bigcup\limits_{#1}}
\newcommand{\adjoin}{{}^\frown}
\newcommand{\forces}{\Vdash}

%%%%%%%%%%%%%%%%%%%%%%%%%%%%%%%%%%%%%%%%%%%%%
% Miscellaneous
%%%%%%%%%%%%%%%%%%%%%%%%%%%%%%%%%%%%%%%%%%%%%
\newcommand{\defeq}{=_{\text{def}}}
\newcommand{\Turingleq}{\leq_{\text{T}}}
\newcommand{\Turingless}{<_{\text{T}}}
\newcommand{\lexleq}{\leq_{\text{lex}}}
\newcommand{\lexless}{<_{\text{lex}}}
\newcommand{\Turingequiv}{\equiv_{\text{T}}}

%%%%%%%%%%%%%%%%%%%%%%%%%%%%%%%%%%%%%%%%%%%%%
% Constants
%%%%%%%%%%%%%%%%%%%%%%%%%%%%%%%%%%%%%%%%%%%%%
\newcommand{\R}{\mathbb{R}}
\renewcommand{\P}{\mathbb{P}}
\newcommand{\Q}{\mathbb{Q}}
\newcommand{\Z}{\mathbb{Z}}
\newcommand{\C}{\mathbb{C}}
\newcommand{\N}{\mathbb{N}}
\newcommand{\B}{\mathbb{B}}
\newcommand{\LofR}{L(\R)}
\newcommand{\JofR}[1]{J_{#1}(\R)}
\newcommand{\SofR}[1]{S_{#1}(\R)}
\newcommand{\JalphaR}{\JofR{\alpha}}
\newcommand{\JbetaR}{\JofR{\beta}}
\newcommand{\JlambdaR}{\JofR{\lambda}}
\newcommand{\SalphaR}{\SofR{\alpha}}
\newcommand{\SbetaR}{\SofR{\beta}}
\newcommand{\Pkl}{\mathcal{P}_{\kappa}(\lambda)}
\DeclareMathOperator{\con}{con}
\DeclareMathOperator{\ORD}{OR}
\DeclareMathOperator{\Ord}{OR}
\DeclareMathOperator{\WO}{WO}
\DeclareMathOperator{\OD}{OD}
\DeclareMathOperator{\HOD}{HOD}
\DeclareMathOperator{\HC}{HC}
\DeclareMathOperator{\WF}{WF}
\DeclareMathOperator{\HF}{HF}
\newcommand{\One}{I}
\newcommand{\ONE}{I}
\newcommand{\Two}{II}
\newcommand{\TWO}{II}

%%%%%%%%%%%%%%%%%%%%%%%%%%%%%%%%%%%%%%%%%%%%%
% Commutative Algebra Constants
%%%%%%%%%%%%%%%%%%%%%%%%%%%%%%%%%%%%%%%%%%%%%
\DeclareMathOperator{\dottimes}{\dot{\times}}

%%%%%%%%%%%%%%%%%%%%%%%%%%%%%%%%%%%%%%%%%%%%%
% Theories
%%%%%%%%%%%%%%%%%%%%%%%%%%%%%%%%%%%%%%%%%%%%%
\DeclareMathOperator{\ZFC}{ZFC}
\DeclareMathOperator{\ZF}{ZF}
\DeclareMathOperator{\AD}{AD}
\DeclareMathOperator{\ADR}{AD_{\R}}
\DeclareMathOperator{\KP}{KP}
\DeclareMathOperator{\PD}{PD}
\DeclareMathOperator{\CH}{CH}
\DeclareMathOperator{\HPC}{HPC} % HOD pair capturing
%%%%%%%%%%%%%%%%%%%%%%%%%%%%%%%%%%%%%%%%%%%%%
% Iteration Trees
%%%%%%%%%%%%%%%%%%%%%%%%%%%%%%%%%%%%%%%%%%%%%

\newcommand{\pred}{\text{-pred}}

%%%%%%%%%%%%%%%%%%%%%%%%%%%%%%%%%%%%%%%%%%%%%%%%
% Operator Names
%%%%%%%%%%%%%%%%%%%%%%%%%%%%%%%%%%%%%%%%%%%%%%%%
\DeclareMathOperator{\Det}{Det}
\DeclareMathOperator{\dom}{dom}
\DeclareMathOperator{\ran}{ran}
\DeclareMathOperator{\range}{ran}
\DeclareMathOperator{\image}{image}
\DeclareMathOperator{\crit}{crit}
\DeclareMathOperator{\card}{card}
\DeclareMathOperator{\cf}{cf}
\DeclareMathOperator{\cof}{cof}
\DeclareMathOperator{\rank}{rank}
\DeclareMathOperator{\ot}{o.t.}
\DeclareMathOperator{\ords}{o}
\DeclareMathOperator{\ro}{r.o.}
\DeclareMathOperator{\rud}{rud}
\DeclareMathOperator{\Powerset}{\mathcal{P}}
\DeclareMathOperator{\length}{lh}
\DeclareMathOperator{\lh}{lh}
\DeclareMathOperator{\limit}{lim}
\DeclareMathOperator{\fld}{fld}
\DeclareMathOperator{\projection}{p}
\DeclareMathOperator{\Ult}{Ult}
\DeclareMathOperator{\ULT}{Ult}
\DeclareMathOperator{\Coll}{Coll}
\DeclareMathOperator{\Col}{Col}
\DeclareMathOperator{\Hull}{Hull}
\DeclareMathOperator*{\dirlim}{dir lim}
\DeclareMathOperator{\Scale}{Scale}
\DeclareMathOperator{\supp}{supp}
\DeclareMathOperator{\trancl}{tran.cl.}
\DeclareMathOperator{\trace}{Tr}
\DeclareMathOperator{\diag}{diag}
\DeclareMathOperator{\spn}{span}
\DeclareMathOperator{\sgn}{sgn}
%%%%%%%%%%%%%%%%%%%%%%%%%%%%%%%%%%%%%%%%%%%%%
% Logical Connectives
%%%%%%%%%%%%%%%%%%%%%%%%%%%%%%%%%%%%%%%%%%%%%
\newcommand{\IImplies}{\Longrightarrow}
\newcommand{\SkipImplies}{\quad\Longrightarrow\quad}
\newcommand{\Ifff}{\Longleftrightarrow}
\newcommand{\iimplies}{\longrightarrow}
\newcommand{\ifff}{\longleftrightarrow}
\newcommand{\Implies}{\Rightarrow}
\newcommand{\Iff}{\Leftrightarrow}
\renewcommand{\implies}{\rightarrow}
\renewcommand{\iff}{\leftrightarrow}
\newcommand{\AND}{\wedge}
\newcommand{\OR}{\vee}
\newcommand{\st}{\text{ s.t. }}
\newcommand{\Or}{\text{ or }}

%%%%%%%%%%%%%%%%%%%%%%%%%%%%%%%%%%%%%%%%%%%%%
% Function Arrows
%%%%%%%%%%%%%%%%%%%%%%%%%%%%%%%%%%%%%%%%%%%%%

\newcommand{\injection}{\xrightarrow{\text{1-1}}}
\newcommand{\surjection}{\xrightarrow{\text{onto}}}
\newcommand{\bijection}{\xrightarrow[\text{onto}]{\text{1-1}}}
\newcommand{\cofmap}{\xrightarrow{\text{cof}}}
\newcommand{\map}{\rightarrow}

%%%%%%%%%%%%%%%%%%%%%%%%%%%%%%%%%%%%%%%%%%%%%
% Mouse Comparison Operators
%%%%%%%%%%%%%%%%%%%%%%%%%%%%%%%%%%%%%%%%%%%%%
\newcommand{\initseg}{\trianglelefteq}
\newcommand{\properseg}{\lhd}
\newcommand{\notinitseg}{\ntrianglelefteq}
\newcommand{\notproperseg}{\ntriangleleft}

%%%%%%%%%%%%%%%%%%%%%%%%%%%%%%%%%%%%%%%%%%%%%
%           calligraphic letters
%%%%%%%%%%%%%%%%%%%%%%%%%%%%%%%%%%%%%%%%%%%%%
\newcommand{\cA}{\mathcal{A}}
\newcommand{\cB}{\mathcal{B}}
\newcommand{\cC}{\mathcal{C}}
\newcommand{\cD}{\mathcal{D}}
\newcommand{\cE}{\mathcal{E}}
\newcommand{\cF}{\mathcal{F}}
\newcommand{\cG}{\mathcal{G}}
\newcommand{\cH}{\mathcal{H}}
\newcommand{\cI}{\mathcal{I}}
\newcommand{\cJ}{\mathcal{J}}
\newcommand{\cK}{\mathcal{K}}
\newcommand{\cL}{\mathcal{L}}
\newcommand{\cM}{\mathcal{M}}
\newcommand{\cN}{\mathcal{N}}
\newcommand{\cO}{\mathcal{O}}
\newcommand{\cP}{\mathcal{P}}
\newcommand{\cQ}{\mathcal{Q}}
\newcommand{\cR}{\mathcal{R}}
\newcommand{\cS}{\mathcal{S}}
\newcommand{\cT}{\mathcal{T}}
\newcommand{\cU}{\mathcal{U}}
\newcommand{\cV}{\mathcal{V}}
\newcommand{\cW}{\mathcal{W}}
\newcommand{\cX}{\mathcal{X}}
\newcommand{\cY}{\mathcal{Y}}
\newcommand{\cZ}{\mathcal{Z}}


%%%%%%%%%%%%%%%%%%%%%%%%%%%%%%%%%%%%%%%%%%%%%
%          Primed Letters
%%%%%%%%%%%%%%%%%%%%%%%%%%%%%%%%%%%%%%%%%%%%%
\newcommand{\aprime}{a^{\prime}}
\newcommand{\bprime}{b^{\prime}}
\newcommand{\cprime}{c^{\prime}}
\newcommand{\dprime}{d^{\prime}}
\newcommand{\eprime}{e^{\prime}}
\newcommand{\fprime}{f^{\prime}}
\newcommand{\gprime}{g^{\prime}}
\newcommand{\hprime}{h^{\prime}}
\newcommand{\iprime}{i^{\prime}}
\newcommand{\jprime}{j^{\prime}}
\newcommand{\kprime}{k^{\prime}}
\newcommand{\lprime}{l^{\prime}}
\newcommand{\mprime}{m^{\prime}}
\newcommand{\nprime}{n^{\prime}}
\newcommand{\oprime}{o^{\prime}}
\newcommand{\pprime}{p^{\prime}}
\newcommand{\qprime}{q^{\prime}}
\newcommand{\rprime}{r^{\prime}}
\newcommand{\sprime}{s^{\prime}}
\newcommand{\tprime}{t^{\prime}}
\newcommand{\uprime}{u^{\prime}}
\newcommand{\vprime}{v^{\prime}}
\newcommand{\wprime}{w^{\prime}}
\newcommand{\xprime}{x^{\prime}}
\newcommand{\yprime}{y^{\prime}}
\newcommand{\zprime}{z^{\prime}}
\newcommand{\Aprime}{A^{\prime}}
\newcommand{\Bprime}{B^{\prime}}
\newcommand{\Cprime}{C^{\prime}}
\newcommand{\Dprime}{D^{\prime}}
\newcommand{\Eprime}{E^{\prime}}
\newcommand{\Fprime}{F^{\prime}}
\newcommand{\Gprime}{G^{\prime}}
\newcommand{\Hprime}{H^{\prime}}
\newcommand{\Iprime}{I^{\prime}}
\newcommand{\Jprime}{J^{\prime}}
\newcommand{\Kprime}{K^{\prime}}
\newcommand{\Lprime}{L^{\prime}}
\newcommand{\Mprime}{M^{\prime}}
\newcommand{\Nprime}{N^{\prime}}
\newcommand{\Oprime}{O^{\prime}}
\newcommand{\Pprime}{P^{\prime}}
\newcommand{\Qprime}{Q^{\prime}}
\newcommand{\Rprime}{R^{\prime}}
\newcommand{\Sprime}{S^{\prime}}
\newcommand{\Tprime}{T^{\prime}}
\newcommand{\Uprime}{U^{\prime}}
\newcommand{\Vprime}{V^{\prime}}
\newcommand{\Wprime}{W^{\prime}}
\newcommand{\Xprime}{X^{\prime}}
\newcommand{\Yprime}{Y^{\prime}}
\newcommand{\Zprime}{Z^{\prime}}
\newcommand{\alphaprime}{\alpha^{\prime}}
\newcommand{\betaprime}{\beta^{\prime}}
\newcommand{\gammaprime}{\gamma^{\prime}}
\newcommand{\Gammaprime}{\Gamma^{\prime}}
\newcommand{\deltaprime}{\delta^{\prime}}
\newcommand{\epsilonprime}{\epsilon^{\prime}}
\newcommand{\kappaprime}{\kappa^{\prime}}
\newcommand{\lambdaprime}{\lambda^{\prime}}
\newcommand{\rhoprime}{\rho^{\prime}}
\newcommand{\Sigmaprime}{\Sigma^{\prime}}
\newcommand{\tauprime}{\tau^{\prime}}
\newcommand{\xiprime}{\xi^{\prime}}
\newcommand{\thetaprime}{\theta^{\prime}}
\newcommand{\Omegaprime}{\Omega^{\prime}}
\newcommand{\cMprime}{\cM^{\prime}}
\newcommand{\cNprime}{\cN^{\prime}}
\newcommand{\cPprime}{\cP^{\prime}}
\newcommand{\cQprime}{\cQ^{\prime}}
\newcommand{\cRprime}{\cR^{\prime}}
\newcommand{\cSprime}{\cS^{\prime}}
\newcommand{\cTprime}{\cT^{\prime}}

%%%%%%%%%%%%%%%%%%%%%%%%%%%%%%%%%%%%%%%%%%%%%
%          bar Letters
%%%%%%%%%%%%%%%%%%%%%%%%%%%%%%%%%%%%%%%%%%%%%
\newcommand{\abar}{\bar{a}}
\newcommand{\bbar}{\bar{b}}
\newcommand{\zbar}{\bar{z}}
\newcommand{\phibar}{\bar{\varphi}}
\newcommand{\psibar}{\bar{\psi}}
\newcommand{\thetabar}{\bar{\theta}}
\newcommand{\nubar}{\bar{\nu}}

%%%%%%%%%%%%%%%%%%%%%%%%%%%%%%%%%%%%%%%%%%%%%
%          star Letters
%%%%%%%%%%%%%%%%%%%%%%%%%%%%%%%%%%%%%%%%%%%%%
\newcommand{\phistar}{\phi^{*}}


%%%%%%%%%%%%%%%%%%%%%%%%%%%%%%%%%%%%%%%%%%%%%
%          Formulas
%%%%%%%%%%%%%%%%%%%%%%%%%%%%%%%%%%%%%%%%%%%%%

\newcommand{\formulaphi}{\text{\large $\varphi$}}
\newcommand{\Formulaphi}{\text{\Large $\varphi$}}


%%%%%%%%%%%%%%%%%%%%%%%%%%%%%%%%%%%%%%%%%%%%%
%          Fraktur Letters
%%%%%%%%%%%%%%%%%%%%%%%%%%%%%%%%%%%%%%%%%%%%%

\newcommand{\fa}{\mathfrak{a}}
\newcommand{\fb}{\mathfrak{b}}
\newcommand{\fc}{\mathfrak{c}}
\newcommand{\fk}{\mathfrak{k}}
\newcommand{\fp}{\mathfrak{p}}
\newcommand{\fq}{\mathfrak{q}}
\newcommand{\fr}{\mathfrak{r}}
\newcommand{\fA}{\mathfrak{A}}
\newcommand{\fB}{\mathfrak{B}}
\newcommand{\fC}{\mathfrak{C}}
\newcommand{\fD}{\mathfrak{D}}

%%%%%%%%%%%%%%%%%%%%%%%%%%%%%%%%%%%%%%%%%%%%%
%          Bold Letters
%%%%%%%%%%%%%%%%%%%%%%%%%%%%%%%%%%%%%%%%%%%%%
\newcommand{\ba}{\mathbf{a}}
\newcommand{\bb}{\mathbf{b}}
\newcommand{\bc}{\mathbf{c}}
\newcommand{\bd}{\mathbf{d}}
\newcommand{\be}{\mathbf{e}}
\newcommand{\bu}{\mathbf{u}}
\newcommand{\bv}{\mathbf{v}}
\newcommand{\bw}{\mathbf{w}}
\newcommand{\bx}{\mathbf{x}}
\newcommand{\by}{\mathbf{y}}
\newcommand{\bz}{\mathbf{z}}
\newcommand{\bSigma}{\boldsymbol{\Sigma}}
\newcommand{\bPi}{\boldsymbol{\Pi}}
\newcommand{\bDelta}{\boldsymbol{\Delta}}
\newcommand{\bdelta}{\boldsymbol{\delta}}
\newcommand{\bgamma}{\boldsymbol{\gamma}}
\newcommand{\bGamma}{\boldsymbol{\Gamma}}

%%%%%%%%%%%%%%%%%%%%%%%%%%%%%%%%%%%%%%%%%%%%%
%         Bold numbers
%%%%%%%%%%%%%%%%%%%%%%%%%%%%%%%%%%%%%%%%%%%%%
\newcommand{\bzero}{\mathbf{0}}

%%%%%%%%%%%%%%%%%%%%%%%%%%%%%%%%%%%%%%%%%%%%%
% Projective-Like Pointclasses
%%%%%%%%%%%%%%%%%%%%%%%%%%%%%%%%%%%%%%%%%%%%%
\newcommand{\Sa}[2][\alpha]{\Sigma_{(#1,#2)}}
\newcommand{\Pa}[2][\alpha]{\Pi_{(#1,#2)}}
\newcommand{\Da}[2][\alpha]{\Delta_{(#1,#2)}}
\newcommand{\Aa}[2][\alpha]{A_{(#1,#2)}}
\newcommand{\Ca}[2][\alpha]{C_{(#1,#2)}}
\newcommand{\Qa}[2][\alpha]{Q_{(#1,#2)}}
\newcommand{\da}[2][\alpha]{\delta_{(#1,#2)}}
\newcommand{\leqa}[2][\alpha]{\leq_{(#1,#2)}}
\newcommand{\lessa}[2][\alpha]{<_{(#1,#2)}}
\newcommand{\equiva}[2][\alpha]{\equiv_{(#1,#2)}}


\newcommand{\Sl}[1]{\Sa[\lambda]{#1}}
\newcommand{\Pl}[1]{\Pa[\lambda]{#1}}
\newcommand{\Dl}[1]{\Da[\lambda]{#1}}
\newcommand{\Al}[1]{\Aa[\lambda]{#1}}
\newcommand{\Cl}[1]{\Ca[\lambda]{#1}}
\newcommand{\Ql}[1]{\Qa[\lambda]{#1}}

\newcommand{\San}{\Sa{n}}
\newcommand{\Pan}{\Pa{n}}
\newcommand{\Dan}{\Da{n}}
\newcommand{\Can}{\Ca{n}}
\newcommand{\Qan}{\Qa{n}}
\newcommand{\Aan}{\Aa{n}}
\newcommand{\dan}{\da{n}}
\newcommand{\leqan}{\leqa{n}}
\newcommand{\lessan}{\lessa{n}}
\newcommand{\equivan}{\equiva{n}}

%%%%%%%%%%%%%%%%%%%%%%%%%%%%%%%%%%%%%%%%%%%%%
% Linear Algebra
%%%%%%%%%%%%%%%%%%%%%%%%%%%%%%%%%%%%%%%%%%%%%
\newcommand{\transpose}[1]{{#1}^{\text{T}}}
\newcommand{\norm}[1]{\lVert{#1}\rVert}
\newcommand\aug{\fboxsep=-\fboxrule\!\!\!\fbox{\strut}\!\!\!}

%%%%%%%%%%%%%%%%%%%%%%%%%%%%%%%%%%%%%%%%%%%%%
% Number Theory
%%%%%%%%%%%%%%%%%%%%%%%%%%%%%%%%%%%%%%%%%%%%%
\DeclareMathOperator{\Spec}{Spec}
\newcommand{\av}[1]{\lvert#1\rvert}
\DeclareMathOperator{\divides}{\mid}
\DeclareMathOperator{\ndivides}{\nmid}


\graphicspath{{images/}}

\newtheorem*{claim}{claim}
\newtheorem*{observation}{Observation}
\newtheorem*{warning}{Warning}
\newtheorem*{question}{Question}
\newtheorem{remark}[theorem]{Remark}

\newenvironment*{subproof}[1][Proof]
{\begin{proof}[#1]}{\renewcommand{\qedsymbol}{$\diamondsuit$} \end{proof}}

\mode<presentation>
{
  \usetheme{Singapore}
  % or ...

  \setbeamercovered{invisible}
  % or whatever (possibly just delete it)
}


\usepackage[english]{babel}
% or whatever

\usepackage[latin1]{inputenc}
% or whatever

\usepackage{times}
\usepackage[T1]{fontenc}
% Or whatever. Note that the encoding and the font should match. If T1
% does not look nice, try deleting the line with the fontenc.

\title{Lesson 22 \\ More about Primitive Roots}
\subtitle{Math 310, Elementary Number Theory \\ Fall 2020 \\ SFSU}
\author{Mitch Rudominer}
\date{}



% If you wish to uncover everything in a step-wise fashion, uncomment
% the following command:

\beamerdefaultoverlayspecification{<+->}

\begin{document}

\begin{frame}
  \titlepage
\end{frame}

\begin{frame}{The Order of a power of an element.}

\begin{itemize}
  \item Work in $\Z_7^*$...
  \item $[3]^1=[3]$, $[3]^2=[2]$, $[3]^3=[6]$, $[3]^4=[4]$
  \item $[3]^5=[5]$, $[3]^6=[1]$.
  \item The order of $[3]$ is $6$. ($\ord_7 3 = 6$.)
  \item $[3]$ is a generator of $\Z_7^*$. (3 is a primitive root of 7.)
  \item What is the order $[3]^2$ in $\Z_7^*$. (What is $\ord_7 3^2$?)
  \item $([3]^2)([3]^2)([3]^2)=[1]$.
  \item $([3]^2)^3 = [1]$ and $([3]^2)^k\not=[1]$ for $k<3$.
  \item So $\ord_7 3^2 = 3$.
  \item What can we say in general about the order of a power of an element?
\end{itemize}

\end{frame}

\begin{frame}{The Order of a power of an element.}

\begin{itemize}
  \item Work in $\Z_n^*$...
  \item Suppose the order of $[a]$ is $t$. What is the order of $[a]^u$?
  \item We want the least $r$ such that $([a]^u)^r = [1]$.
  \item We want the least $r$ such that $[a]^{ru} = [1]$.
  \item We want the least $r$ such that $ru$ is a multiple of $t$.
  \item We want $r$ such that $ru = \lcm(u,t) = \frac{ut}{\gcd(u,t)}$.
  \item Solution: $r=\frac{t}{\gcd(u,t)}$.
  \item \textbf{Theorem.} If $[a]$ is in $\Z_n^*$ and $t$ is the order of $[a]$.
  \item Then the order of $[a]^u$ in $\Z_n^*$ is $\frac{t}{\gcd(u,t)}$.
  \item \textbf{Theorem.} If $n>1$ and $\gcd(a,n)=1$ and $\ord_n a = t$,
  \item then $\ord_n a^u = \frac{t}{\gcd(u,t)}$.
\end{itemize}

\end{frame}

\begin{frame}{Examples}

\begin{itemize}
  \item Example: Suppose $\ord_n a = 36$.
  \item Find $\ord_n a^{24}$.
  \item $\ord_n a^{24} = \frac{36}{\gcd(24,36)} = \frac{36}{12} = 3$.
  \item Check: $a^{24} \not\equiv 1 \pmod{36}$, $(a^{24})^2=a^{48} \not\equiv 1 \pmod{36}$
  \item $(a^{24})^3=a^{72} \equiv 1 \pmod{36}$.
  \item Find $\ord_n a^{6}$.
  \item $\ord_n a^{6} = \frac{36}{\gcd(6,36)} = \frac{36}{6} = 6$.
  \item Find $\ord_n a^{7}$.
  \item $\ord_n a^{7} = \frac{36}{\gcd(7,36)} = \frac{36}{1} = 36$.
  \item Notice: If $u$ is relatively prime to $\ord_n a$ then $\ord_n a^u = \ord_n a$.
\end{itemize}

\end{frame}

\begin{frame}{All the primitive roots}

\begin{itemize}
  \item \textbf{Theorem.} Suppose $[a]$ is a generator of $\Z_n^*$. Then
  \item $[a]^u$ is also a generator iff $\gcd(u,\phi(n)) = 1$.
  \item \textbf{Theorem.} Suppose $a$ is a primitive root of $n$. Then
  \item $a^u$ is also a primitive root of $n$ iff $\gcd(u,\phi(n))=1$.
  \item Example $[3]$ is a generator of $\Z_7^*$. $\phi(7) = 6$.
  \item $[3]^2 = [2], [3]^3 = [6], [3]^4=[4], [3]^6=[1]$ are not generators.
  \item $[3]^5=[5]$ is a generator.
  \item In the language of congruence: $3$ is a primitive root of $7$.
  \item $3^2, 3^3, 3^4, 3^6$ are not primitive roots of $7$.
  \item $3^5$ is also a primitive root of $7$.
  \item $3^5 \equiv 5 \pmod 7$.
  \item $5$ is also a primitive root of $7$.
\end{itemize}

\end{frame}

\begin{frame}{Number of primitive roots}

\begin{itemize}
  \item \textbf{Theorem.} If $\Z_n^*$ is cyclic then
  \item the number of generators of $\Z_n^*$ is $\phi(\phi(n))$.
  \item Example: $\Z_7^*$ is cyclic with generator $[3]$.
  \item $\phi(\phi(7)) = \phi(6) = 2$.
  \item $[3]$ and $[3]^5 = [5]$ are the two generators.
  \item In the language of congruence:
  \item \textbf{Theorem.} Let $n>1$ be an integer. Suppose $n$ has a primitive root.
  \item Then $n$ has $\phi(\phi(n))$ incongruent primitive roots.
  \item Example: $3$ is a primitive root of $7$.
  \item $3^5$ is also a primitive root of $7$.
  \item $3^5 \equiv 5 \pmod 7$
  \item $7$ has two incongruent primitive roots: $3$ and $5$.
\end{itemize}

\end{frame}

\begin{frame}{Number of primitive roots}

\begin{itemize}
  \item Example: $\Z_{11}^*$ is cyclic with generator $[2]$.
  \item $\phi(\phi(11)) = \phi(10) = 4$.
  \item So $\Z_{11}^*$ must have 4 generators.
  \item $[2],[2]^3=[8], [2]^7=[7], [2]^9 = [6]$ are the four generators.
  \item In the language of congruence: $2$ is a primitive root of $11$.
  \item $2^3,2^7,2^9$ are also primitive roots of $11$.
  \item $2^3 = 8, 2^7 \equiv 7 \pmod{11}, 2^9 \equiv 6 \pmod {11}$.
  \item $11$ has four incongruent primitive roots: 2,6,7,8
\end{itemize}

\end{frame}

\begin{frame}{Which integers have primitive roots?}
\begin{adjustwidth}{-2em}{-2em}
\small
\begin{tabular}{|c|c|c|c|c|c|c|c|c|c|}\hline
$2$           &  $3$         & $4$           &   $5$         & $6$            & $7$   & $8$          & $9$         & $10$            & $11$       \\ \hline
$2$           &  $3$         & $2\cdot2$     &   $5$         & $2\cdot3$      & $7$   & $2^3$        & $3^2$       & $2\cdot 5$      & $11$       \\ \hline
yes           &  yes         &  yes          &   yes         & yes            & yes   & no           & yes         & yes             & yes        \\ \hline\hline
$12$          &  $13$        & $14$          &   $15$        & $16$           & $17$  & $18$         & $19$        & $20$            & $21$       \\ \hline
$2^2\cdot3$   &  $13$        & $2\cdot7$     &   $3\cdot 5$  & $2^4$          & $17$  & $2\cdot 3^2$ & $19$        & $2^2\cdot 5$    & $3\cdot 7$ \\ \hline
no            &  yes         &  yes          &   no          & no             & yes   & yes          & yes         & no              & no         \\ \hline\hline
$22$          &  $23$        & $24$          &   $25$        & $26$           & $27$  & $28$         & $29$        & $30$            & $31$       \\ \hline
$2\cdot11$    &  $23$        & $2^3\cdot 3$  &   $5^2$       & $2\cdot 13$    & $3^3$ & $2^2\cdot 7$ & $29$        & $2\cdot3\cdot5$ & $31$       \\ \hline
yes           &  yes         &  no           &   yes         & yes            & yes   & no           & yes         & no              & yes        \\ \hline\hline
$32$          &  $33$        & $34$          &   $35$        & $36$           & $37$  & $38$         & $39$        & $40$            & $41$       \\ \hline
$2^5$         &  $3\cdot 11$ & $2\cdot 17$   &   $5\cdot 7$  & $2^2\cdot 3^2$ & $37$  & $2\cdot 19$  & $3\cdot 13$ & $2^3\cdot 5$    & $41$       \\ \hline
no            &  no          &  yes          &   no          & no             & yes   & yes          & no          & no              & yes        \\ \hline
\end{tabular}
\normalsize
\end{adjustwidth}

\end{frame}

\begin{frame}{Primitive Root Theorem}

\begin{itemize}
  \item \textbf{Theorem.} Let $n>1$ be an integer. Then
  \item $n$ has a primitive root iff one of the following is true:
  \item (a) $n$ is a power of an odd prime. $n=p^k$, $k\geq 1$.
  \item (b) $n$ is twice a power of an odd prime. $n=2p^k$, $k\geq 1$.
  \item (c) $n=2$ or $n=4$.
  \item In other words, $n$ does \emph{not} have a primitive root iff one of the following is true:
  \item (d) $n$ is divisible by two distinct odd primes.
  \item (e) $n$ is divisible by $4$ but not equal to $4$.
  \item In other words, $n$ does \emph{not} have a primitive root iff one of the following is true:
  \item (f) $n=ab$ where $\gcd(a,b)=1$ and $a,b>2$.
  \item (g) $n=2^k$ with $k\geq 3$.
\end{itemize}

\end{frame}

\begin{frame}{Example}

\begin{itemize}
  \item Example: How many incongruent primitive roots does 56 have?
  \item Answer: Zero because $8\divides 56$.
  \item Example: How many incongruent primitive roots does 77 have?
  \item Answer: Zero because 77 is divisible by two odd primes.
  \item Example: How many incongruent primitive roots does 50 have?
  \item Answer: $50=2\cdot 5^2$. So 50 has a primitive root.
  \item The number of incongruent primitive roots is $\phi(\phi(50))=$
  \item $\phi(50)= 1\cdot4\cdot 5 = 20$.
  \item $\phi(\phi(50))=\phi(20)=\phi(2^2\cdot 5)=1\cdot 2\cdot 4 = 8$.
  \item 50 has 8 incongruent primitive roots.
\end{itemize}

\end{frame}

\begin{frame}{Example}

\begin{itemize}
  \item Example: Given that 3 is a primitive root of 50, find 7 other incongruent primitive roots expressed as a power of 3.
  \item Answer: We want $3^k$ with $1\leq k < \phi(50)=20$, and $\gcd(k,20)=1$.
  \item $3^1,3^3,3^7,3^9,3^{11},3^{13},3^{17},3^{19}$.
  \item The above is the solution. We were not asked to do this, but if we want we could also reduce the expressions mod 50 and get
  the eight incongruent primitive roots of 50 are:
  \item 3, 13, 17, 23, 27, 33, 37, 47
  \item Note that in the line above we listed the primitive roots in increasing numerical order. They are not in the same order as their
  equivalent expressions in the form $3^k$. For example $3^3 = 27$.
\end{itemize}

\end{frame}

\begin{frame}{Applications of Primitive Roots}

\begin{itemize}
  \item Why do we care about primitive roots?
  \item One reason is that the subject is interesting mathematically in its own right.
  \item Another reason is that if $n$ has a primitive root, it is much easier for us to
  analyze congruence mod n.
  \item In other words, if $\Z_n^*$ is cyclic it is much easier for us to analyze $\Z_n^*$.
  \item This is because if $[a]$ is a generator of $\Z_n^*$ then the elements of
  $\Z_n^*$ are $[a]$, $[a]^2$, $[a]^3\cdots$ so it is easier to answer questions about these elements. For example...
  \item \textbf{Theorem.} If $\Z_n^*$ is cyclic then it has only two square-roots of $[1]$.
  \item (The two square roots of $[1]$ are $[1]$ and $-[1] = [n-1]$.)
  \item In the language of congruence...
  \item \textbf{Theorem.} If $n$ has a primitive root, then there are only two integers, up to congruence,
  whose squares are congruent to $1$ modulo $n$. (They are $1$ and $n-1$.)
\end{itemize}

\end{frame}

\begin{frame}{Proof}

\begin{itemize}
  \item \textbf{Theorem.} If $\Z_n^*$ is cyclic then it has only two square-roots of $[1]$.
  \item \textbf{proof.} Let $[a]$ be a generator of $\Z_n^*$.
  \item The elements of $\Z_n^*$ are $[a],[a]^2,\cdots [a]^{\phi(n)} = [1]$.
  \item The question is, for which values of $k$  is $([a]^k)^2=[1]$?
  \item $[a]^{2k}=[1]$ iff $\phi(n)\divides 2k$.
  \item So, for which values of $k$, with $1\leq k \leq \phi(n)$ does $\phi(n)\divides 2k$?
  \item If $1\leq k \leq \phi(n)$, then $2\leq 2k \leq 2\phi(n)$.
  \item So $\phi(n)\divides 2k$ iff $2k = \phi(n)$ or $2k = 2\phi(n)$.
  \item So there are exactly two values, namely $k=\phi(n)$ and $k=\frac{\phi(n)}{2}$.
  \item So there are exactly two elements of $\Z_n^*$ that are square-roots of $[1]$:
  \item $[a]^{\phi(n)}=[1]$ and $[a]^{\frac{\phi(n)}{2}}.\qed$.
  \item Note that in the proof above, $[a]^{\frac{\phi(n)}{2}}=-[1]$
\end{itemize}

\end{frame}


\begin{frame}{Example}

\begin{itemize}
  \item Example: Suppose you are told that $11$ is a primitive root of $118$.
  \item Problem: Express $117$ as a power of $11$ mod $118$.
  \item i.e. find $k$ so that $11^k \equiv 117 \pmod {118}$.
  \item Solution: $118=2\cdot 59$. So $\phi(118) = 58$.
  \item The elements of $\Z_{118}^*$ are $[11],[11]^2,\cdots[11]^{58} = [1]$.
  \item $\Z_{118}^*$ has two square roots of $[1]$:
  \item $[11]^{58}$ and $[11]^{29}$.
  \item $[11]^{58} = [1]$ so $[11]^{29} = -[1] = [117]$.
  \item So $117 \equiv 11^{29} \pmod {118}$.
\end{itemize}

\end{frame}



\end{document}
