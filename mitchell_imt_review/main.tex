\documentclass[oneside,12pt]{amsart}

\usepackage{amsmath,amssymb,latexsym,eucal,amsthm}

%%%%%%%%%%%%%%%%%%%%%%%%%%%%%%%%%%%%%%%%%%%%%
% Common Set Theory Constructs
%%%%%%%%%%%%%%%%%%%%%%%%%%%%%%%%%%%%%%%%%%%%%

\newcommand{\setof}[2]{\left\{ \, #1 \, \left| \, #2 \, \right.\right\}}
\newcommand{\lsetof}[2]{\left\{\left. \, #1 \, \right| \, #2 \,  \right\}}
\newcommand{\bigsetof}[2]{\bigl\{ \, #1 \, \bigm | \, #2 \,\bigr\}}
\newcommand{\Bigsetof}[2]{\Bigl\{ \, #1 \, \Bigm | \, #2 \,\Bigr\}}
\newcommand{\biggsetof}[2]{\biggl\{ \, #1 \, \biggm | \, #2 \,\biggr\}}
\newcommand{\Biggsetof}[2]{\Biggl\{ \, #1 \, \Biggm | \, #2 \,\Biggr\}}
\newcommand{\dotsetof}[2]{\left\{ \, #1 \, : \, #2 \, \right\}}
\newcommand{\bigdotsetof}[2]{\bigl\{ \, #1 \, : \, #2 \,\bigr\}}
\newcommand{\Bigdotsetof}[2]{\Bigl\{ \, #1 \, \Bigm : \, #2 \,\Bigr\}}
\newcommand{\biggdotsetof}[2]{\biggl\{ \, #1 \, \biggm : \, #2 \,\biggr\}}
\newcommand{\Biggdotsetof}[2]{\Biggl\{ \, #1 \, \Biggm : \, #2 \,\Biggr\}}
\newcommand{\sequence}[2]{\left\langle \, #1 \,\left| \, #2 \, \right. \right\rangle}
\newcommand{\lsequence}[2]{\left\langle\left. \, #1 \, \right| \,#2 \,  \right\rangle}
\newcommand{\bigsequence}[2]{\bigl\langle \,#1 \, \bigm | \, #2 \, \bigr\rangle}
\newcommand{\Bigsequence}[2]{\Bigl\langle \,#1 \, \Bigm | \, #2 \, \Bigr\rangle}
\newcommand{\biggsequence}[2]{\biggl\langle \,#1 \, \biggm | \, #2 \, \biggr\rangle}
\newcommand{\Biggsequence}[2]{\Biggl\langle \,#1 \, \Biggm | \, #2 \, \Biggr\rangle}
\newcommand{\singleton}[1]{\left\{#1\right\}}
\newcommand{\angles}[1]{\left\langle #1 \right\rangle}
\newcommand{\bigangles}[1]{\bigl\langle #1 \bigr\rangle}
\newcommand{\Bigangles}[1]{\Bigl\langle #1 \Bigr\rangle}
\newcommand{\biggangles}[1]{\biggl\langle #1 \biggr\rangle}
\newcommand{\Biggangles}[1]{\Biggl\langle #1 \Biggr\rangle}


\newcommand{\force}[1]{\Vert\!\underset{\!\!\!\!\!#1}{\!\!\!\relbar\!\!\!%
\relbar\!\!\relbar\!\!\relbar\!\!\!\relbar\!\!\relbar\!\!\relbar\!\!\!%
\relbar\!\!\relbar\!\!\relbar}}
\newcommand{\longforce}[1]{\Vert\!\underset{\!\!\!\!\!#1}{\!\!\!\relbar\!\!\!%
\relbar\!\!\relbar\!\!\relbar\!\!\!\relbar\!\!\relbar\!\!\relbar\!\!\!%
\relbar\!\!\relbar\!\!\relbar\!\!\relbar\!\!\relbar\!\!\relbar\!\!\relbar\!\!\relbar}}
\newcommand{\nforce}[1]{\Vert\!\underset{\!\!\!\!\!#1}{\!\!\!\relbar\!\!\!%
\relbar\!\!\relbar\!\!\relbar\!\!\!\relbar\!\!\relbar\!\!\relbar\!\!\!%
\relbar\!\!\not\relbar\!\!\relbar}}
\newcommand{\forcein}[2]{\overset{#2}{\Vert\underset{\!\!\!\!\!#1}%
{\!\!\!\relbar\!\!\!\relbar\!\!\relbar\!\!\relbar\!\!\!\relbar\!\!\relbar\!%
\!\relbar\!\!\!\relbar\!\!\relbar\!\!\relbar\!\!\relbar\!\!\!\relbar\!\!%
\relbar\!\!\relbar}}}

\newcommand{\pre}[2]{{}^{#2}{#1}}

\newcommand{\restr}{\!\!\upharpoonright\!}

%%%%%%%%%%%%%%%%%%%%%%%%%%%%%%%%%%%%%%%%%%%%%
% Set-Theoretic Connectives
%%%%%%%%%%%%%%%%%%%%%%%%%%%%%%%%%%%%%%%%%%%%%

\newcommand{\intersect}{\cap}
\newcommand{\union}{\cup}
\newcommand{\Intersection}[1]{\bigcap\limits_{#1}}
\newcommand{\Union}[1]{\bigcup\limits_{#1}}
\newcommand{\adjoin}{{}^\frown}
\newcommand{\forces}{\Vdash}

%%%%%%%%%%%%%%%%%%%%%%%%%%%%%%%%%%%%%%%%%%%%%
% Miscellaneous
%%%%%%%%%%%%%%%%%%%%%%%%%%%%%%%%%%%%%%%%%%%%%
\newcommand{\defeq}{=_{\text{def}}}
\newcommand{\Turingleq}{\leq_{\text{T}}}
\newcommand{\Turingless}{<_{\text{T}}}
\newcommand{\lexleq}{\leq_{\text{lex}}}
\newcommand{\lexless}{<_{\text{lex}}}
\newcommand{\Turingequiv}{\equiv_{\text{T}}}
\newcommand{\isomorphic}{\cong}

%%%%%%%%%%%%%%%%%%%%%%%%%%%%%%%%%%%%%%%%%%%%%
% Constants
%%%%%%%%%%%%%%%%%%%%%%%%%%%%%%%%%%%%%%%%%%%%%
\newcommand{\R}{\mathbb{R}}
\renewcommand{\P}{\mathbb{P}}
\newcommand{\Q}{\mathbb{Q}}
\newcommand{\Z}{\mathbb{Z}}
\newcommand{\Zpos}{\Z^{+}}
\newcommand{\Znonneg}{\Z^{\geq 0}}
\newcommand{\C}{\mathbb{C}}
\newcommand{\N}{\mathbb{N}}
\newcommand{\B}{\mathbb{B}}
\newcommand{\Bairespace}{\pre{\omega}{\omega}}
\newcommand{\LofR}{L(\R)}
\newcommand{\JofR}[1]{J_{#1}(\R)}
\newcommand{\SofR}[1]{S_{#1}(\R)}
\newcommand{\JalphaR}{\JofR{\alpha}}
\newcommand{\JbetaR}{\JofR{\beta}}
\newcommand{\JlambdaR}{\JofR{\lambda}}
\newcommand{\SalphaR}{\SofR{\alpha}}
\newcommand{\SbetaR}{\SofR{\beta}}
\newcommand{\Pkl}{\mathcal{P}_{\kappa}(\lambda)}
\DeclareMathOperator{\con}{con}
\DeclareMathOperator{\ORD}{OR}
\DeclareMathOperator{\Ord}{OR}
\DeclareMathOperator{\WO}{WO}
\DeclareMathOperator{\OD}{OD}
\DeclareMathOperator{\HOD}{HOD}
\DeclareMathOperator{\HC}{HC}
\DeclareMathOperator{\WF}{WF}
\DeclareMathOperator{\wfp}{wfp}
\DeclareMathOperator{\HF}{HF}
\newcommand{\One}{I}
\newcommand{\ONE}{I}
\newcommand{\Two}{II}
\newcommand{\TWO}{II}
\newcommand{\Mladder}{M^{\text{ld}}}

%%%%%%%%%%%%%%%%%%%%%%%%%%%%%%%%%%%%%%%%%%%%%
% Commutative Algebra Constants
%%%%%%%%%%%%%%%%%%%%%%%%%%%%%%%%%%%%%%%%%%%%%
\DeclareMathOperator{\dottimes}{\dot{\times}}
\DeclareMathOperator{\dotminus}{\dot{-}}
\DeclareMathOperator{\Spec}{Spec}

%%%%%%%%%%%%%%%%%%%%%%%%%%%%%%%%%%%%%%%%%%%%%
% Theories
%%%%%%%%%%%%%%%%%%%%%%%%%%%%%%%%%%%%%%%%%%%%%
\DeclareMathOperator{\ZFC}{ZFC}
\DeclareMathOperator{\ZF}{ZF}
\DeclareMathOperator{\AD}{AD}
\DeclareMathOperator{\ADR}{AD_{\R}}
\DeclareMathOperator{\KP}{KP}
\DeclareMathOperator{\PD}{PD}
\DeclareMathOperator{\CH}{CH}
\DeclareMathOperator{\GCH}{GCH}
\DeclareMathOperator{\HPC}{HPC} % HOD pair capturing
%%%%%%%%%%%%%%%%%%%%%%%%%%%%%%%%%%%%%%%%%%%%%
% Iteration Trees
%%%%%%%%%%%%%%%%%%%%%%%%%%%%%%%%%%%%%%%%%%%%%

\newcommand{\pred}{\text{-pred}}

%%%%%%%%%%%%%%%%%%%%%%%%%%%%%%%%%%%%%%%%%%%%%%%%
% Operator Names
%%%%%%%%%%%%%%%%%%%%%%%%%%%%%%%%%%%%%%%%%%%%%%%%
\DeclareMathOperator{\Det}{Det}
\DeclareMathOperator{\dom}{dom}
\DeclareMathOperator{\ran}{ran}
\DeclareMathOperator{\range}{ran}
\DeclareMathOperator{\image}{image}
\DeclareMathOperator{\crit}{crit}
\DeclareMathOperator{\card}{card}
\DeclareMathOperator{\cf}{cf}
\DeclareMathOperator{\cof}{cof}
\DeclareMathOperator{\rank}{rank}
\DeclareMathOperator{\ot}{o.t.}
\DeclareMathOperator{\ords}{o}
\DeclareMathOperator{\ro}{r.o.}
\DeclareMathOperator{\rud}{rud}
\DeclareMathOperator{\Powerset}{\mathcal{P}}
\DeclareMathOperator{\length}{lh}
\DeclareMathOperator{\lh}{lh}
\DeclareMathOperator{\limit}{lim}
\DeclareMathOperator{\fld}{fld}
\DeclareMathOperator{\projection}{p}
\DeclareMathOperator{\Ult}{Ult}
\DeclareMathOperator{\ULT}{Ult}
\DeclareMathOperator{\Coll}{Coll}
\DeclareMathOperator{\Col}{Col}
\DeclareMathOperator{\Hull}{Hull}
\DeclareMathOperator*{\dirlim}{dir lim}
\DeclareMathOperator{\Scale}{Scale}
\DeclareMathOperator{\supp}{supp}
\DeclareMathOperator{\trancl}{tran.cl.}
\DeclareMathOperator{\trace}{Tr}
\DeclareMathOperator{\diag}{diag}
\DeclareMathOperator{\spn}{span}
\DeclareMathOperator{\sgn}{sgn}
%%%%%%%%%%%%%%%%%%%%%%%%%%%%%%%%%%%%%%%%%%%%%
% Logical Connectives
%%%%%%%%%%%%%%%%%%%%%%%%%%%%%%%%%%%%%%%%%%%%%
\newcommand{\IImplies}{\Longrightarrow}
\newcommand{\SkipImplies}{\quad\Longrightarrow\quad}
\newcommand{\Ifff}{\Longleftrightarrow}
\newcommand{\iimplies}{\longrightarrow}
\newcommand{\ifff}{\longleftrightarrow}
\newcommand{\Implies}{\Rightarrow}
\newcommand{\Iff}{\Leftrightarrow}
\renewcommand{\implies}{\rightarrow}
\renewcommand{\iff}{\leftrightarrow}
\newcommand{\AND}{\wedge}
\newcommand{\OR}{\vee}
\newcommand{\st}{\text{ s.t. }}
\newcommand{\Or}{\text{ or }}

%%%%%%%%%%%%%%%%%%%%%%%%%%%%%%%%%%%%%%%%%%%%%
% Function Arrows
%%%%%%%%%%%%%%%%%%%%%%%%%%%%%%%%%%%%%%%%%%%%%

\newcommand{\injection}{\xrightarrow{\text{1-1}}}
\newcommand{\surjection}{\xrightarrow{\text{onto}}}
\newcommand{\bijection}{\xrightarrow[\text{onto}]{\text{1-1}}}
\newcommand{\cofmap}{\xrightarrow{\text{cof}}}
\newcommand{\map}{\rightarrow}

%%%%%%%%%%%%%%%%%%%%%%%%%%%%%%%%%%%%%%%%%%%%%
% Mouse Comparison Operators
%%%%%%%%%%%%%%%%%%%%%%%%%%%%%%%%%%%%%%%%%%%%%
\newcommand{\initseg}{\trianglelefteq}
\newcommand{\properseg}{\lhd}
\newcommand{\notinitseg}{\ntrianglelefteq}
\newcommand{\notproperseg}{\ntriangleleft}

%%%%%%%%%%%%%%%%%%%%%%%%%%%%%%%%%%%%%%%%%%%%%
%           calligraphic letters
%%%%%%%%%%%%%%%%%%%%%%%%%%%%%%%%%%%%%%%%%%%%%
\newcommand{\cA}{\mathcal{A}}
\newcommand{\cB}{\mathcal{B}}
\newcommand{\cC}{\mathcal{C}}
\newcommand{\cD}{\mathcal{D}}
\newcommand{\cE}{\mathcal{E}}
\newcommand{\cF}{\mathcal{F}}
\newcommand{\cG}{\mathcal{G}}
\newcommand{\cH}{\mathcal{H}}
\newcommand{\cI}{\mathcal{I}}
\newcommand{\cJ}{\mathcal{J}}
\newcommand{\cK}{\mathcal{K}}
\newcommand{\cL}{\mathcal{L}}
\newcommand{\cM}{\mathcal{M}}
\newcommand{\cN}{\mathcal{N}}
\newcommand{\cO}{\mathcal{O}}
\newcommand{\cP}{\mathcal{P}}
\newcommand{\cQ}{\mathcal{Q}}
\newcommand{\cR}{\mathcal{R}}
\newcommand{\cS}{\mathcal{S}}
\newcommand{\cT}{\mathcal{T}}
\newcommand{\cU}{\mathcal{U}}
\newcommand{\cV}{\mathcal{V}}
\newcommand{\cW}{\mathcal{W}}
\newcommand{\cX}{\mathcal{X}}
\newcommand{\cY}{\mathcal{Y}}
\newcommand{\cZ}{\mathcal{Z}}


%%%%%%%%%%%%%%%%%%%%%%%%%%%%%%%%%%%%%%%%%%%%%
%          Primed Letters
%%%%%%%%%%%%%%%%%%%%%%%%%%%%%%%%%%%%%%%%%%%%%
\newcommand{\aprime}{a^{\prime}}
\newcommand{\bprime}{b^{\prime}}
\newcommand{\cprime}{c^{\prime}}
\newcommand{\dprime}{d^{\prime}}
\newcommand{\eprime}{e^{\prime}}
\newcommand{\fprime}{f^{\prime}}
\newcommand{\gprime}{g^{\prime}}
\newcommand{\hprime}{h^{\prime}}
\newcommand{\iprime}{i^{\prime}}
\newcommand{\jprime}{j^{\prime}}
\newcommand{\kprime}{k^{\prime}}
\newcommand{\lprime}{l^{\prime}}
\newcommand{\mprime}{m^{\prime}}
\newcommand{\nprime}{n^{\prime}}
\newcommand{\oprime}{o^{\prime}}
\newcommand{\pprime}{p^{\prime}}
\newcommand{\qprime}{q^{\prime}}
\newcommand{\rprime}{r^{\prime}}
\newcommand{\sprime}{s^{\prime}}
\newcommand{\tprime}{t^{\prime}}
\newcommand{\uprime}{u^{\prime}}
\newcommand{\vprime}{v^{\prime}}
\newcommand{\wprime}{w^{\prime}}
\newcommand{\xprime}{x^{\prime}}
\newcommand{\yprime}{y^{\prime}}
\newcommand{\zprime}{z^{\prime}}
\newcommand{\Aprime}{A^{\prime}}
\newcommand{\Bprime}{B^{\prime}}
\newcommand{\Cprime}{C^{\prime}}
\newcommand{\Dprime}{D^{\prime}}
\newcommand{\Eprime}{E^{\prime}}
\newcommand{\Fprime}{F^{\prime}}
\newcommand{\Gprime}{G^{\prime}}
\newcommand{\Hprime}{H^{\prime}}
\newcommand{\Iprime}{I^{\prime}}
\newcommand{\Jprime}{J^{\prime}}
\newcommand{\Kprime}{K^{\prime}}
\newcommand{\Lprime}{L^{\prime}}
\newcommand{\Mprime}{M^{\prime}}
\newcommand{\Nprime}{N^{\prime}}
\newcommand{\Oprime}{O^{\prime}}
\newcommand{\Pprime}{P^{\prime}}
\newcommand{\Qprime}{Q^{\prime}}
\newcommand{\Rprime}{R^{\prime}}
\newcommand{\Sprime}{S^{\prime}}
\newcommand{\Tprime}{T^{\prime}}
\newcommand{\Uprime}{U^{\prime}}
\newcommand{\Vprime}{V^{\prime}}
\newcommand{\Wprime}{W^{\prime}}
\newcommand{\Xprime}{X^{\prime}}
\newcommand{\Yprime}{Y^{\prime}}
\newcommand{\Zprime}{Z^{\prime}}
\newcommand{\alphaprime}{\alpha^{\prime}}
\newcommand{\betaprime}{\beta^{\prime}}
\newcommand{\gammaprime}{\gamma^{\prime}}
\newcommand{\Gammaprime}{\Gamma^{\prime}}
\newcommand{\deltaprime}{\delta^{\prime}}
\newcommand{\epsilonprime}{\epsilon^{\prime}}
\newcommand{\kappaprime}{\kappa^{\prime}}
\newcommand{\lambdaprime}{\lambda^{\prime}}
\newcommand{\rhoprime}{\rho^{\prime}}
\newcommand{\Sigmaprime}{\Sigma^{\prime}}
\newcommand{\tauprime}{\tau^{\prime}}
\newcommand{\xiprime}{\xi^{\prime}}
\newcommand{\thetaprime}{\theta^{\prime}}
\newcommand{\Omegaprime}{\Omega^{\prime}}
\newcommand{\cMprime}{\cM^{\prime}}
\newcommand{\cNprime}{\cN^{\prime}}
\newcommand{\cPprime}{\cP^{\prime}}
\newcommand{\cQprime}{\cQ^{\prime}}
\newcommand{\cRprime}{\cR^{\prime}}
\newcommand{\cSprime}{\cS^{\prime}}
\newcommand{\cTprime}{\cT^{\prime}}

%%%%%%%%%%%%%%%%%%%%%%%%%%%%%%%%%%%%%%%%%%%%%
%          bar Letters
%%%%%%%%%%%%%%%%%%%%%%%%%%%%%%%%%%%%%%%%%%%%%
\newcommand{\abar}{\bar{a}}
\newcommand{\bbar}{\bar{b}}
\newcommand{\cbar}{\bar{c}}
\newcommand{\ibar}{\bar{i}}
\newcommand{\jbar}{\bar{j}}
\newcommand{\nbar}{\bar{n}}
\newcommand{\xbar}{\bar{x}}
\newcommand{\ybar}{\bar{y}}
\newcommand{\zbar}{\bar{z}}
\newcommand{\pibar}{\bar{\pi}}
\newcommand{\phibar}{\bar{\varphi}}
\newcommand{\psibar}{\bar{\psi}}
\newcommand{\thetabar}{\bar{\theta}}
\newcommand{\nubar}{\bar{\nu}}

%%%%%%%%%%%%%%%%%%%%%%%%%%%%%%%%%%%%%%%%%%%%%
%          star Letters
%%%%%%%%%%%%%%%%%%%%%%%%%%%%%%%%%%%%%%%%%%%%%
\newcommand{\phistar}{\phi^{*}}
\newcommand{\Mstar}{M^{*}}

%%%%%%%%%%%%%%%%%%%%%%%%%%%%%%%%%%%%%%%%%%%%%
%          dotletters Letters
%%%%%%%%%%%%%%%%%%%%%%%%%%%%%%%%%%%%%%%%%%%%%
\newcommand{\Gdot}{\dot{G}}

%%%%%%%%%%%%%%%%%%%%%%%%%%%%%%%%%%%%%%%%%%%%%
%         check Letters
%%%%%%%%%%%%%%%%%%%%%%%%%%%%%%%%%%%%%%%%%%%%%
\newcommand{\deltacheck}{\check{\delta}}
\newcommand{\gammacheck}{\check{\gamma}}


%%%%%%%%%%%%%%%%%%%%%%%%%%%%%%%%%%%%%%%%%%%%%
%          Formulas
%%%%%%%%%%%%%%%%%%%%%%%%%%%%%%%%%%%%%%%%%%%%%

\newcommand{\formulaphi}{\text{\large $\varphi$}}
\newcommand{\Formulaphi}{\text{\Large $\varphi$}}


%%%%%%%%%%%%%%%%%%%%%%%%%%%%%%%%%%%%%%%%%%%%%
%          Fraktur Letters
%%%%%%%%%%%%%%%%%%%%%%%%%%%%%%%%%%%%%%%%%%%%%

\newcommand{\fa}{\mathfrak{a}}
\newcommand{\fb}{\mathfrak{b}}
\newcommand{\fc}{\mathfrak{c}}
\newcommand{\fk}{\mathfrak{k}}
\newcommand{\fp}{\mathfrak{p}}
\newcommand{\fq}{\mathfrak{q}}
\newcommand{\fr}{\mathfrak{r}}
\newcommand{\fA}{\mathfrak{A}}
\newcommand{\fB}{\mathfrak{B}}
\newcommand{\fC}{\mathfrak{C}}
\newcommand{\fD}{\mathfrak{D}}

%%%%%%%%%%%%%%%%%%%%%%%%%%%%%%%%%%%%%%%%%%%%%
%          Bold Letters
%%%%%%%%%%%%%%%%%%%%%%%%%%%%%%%%%%%%%%%%%%%%%
\newcommand{\ba}{\mathbf{a}}
\newcommand{\bb}{\mathbf{b}}
\newcommand{\bc}{\mathbf{c}}
\newcommand{\bd}{\mathbf{d}}
\newcommand{\be}{\mathbf{e}}
\newcommand{\bu}{\mathbf{u}}
\newcommand{\bv}{\mathbf{v}}
\newcommand{\bw}{\mathbf{w}}
\newcommand{\bx}{\mathbf{x}}
\newcommand{\by}{\mathbf{y}}
\newcommand{\bz}{\mathbf{z}}
\newcommand{\bSigma}{\boldsymbol{\Sigma}}
\newcommand{\bPi}{\boldsymbol{\Pi}}
\newcommand{\bDelta}{\boldsymbol{\Delta}}
\newcommand{\bdelta}{\boldsymbol{\delta}}
\newcommand{\bgamma}{\boldsymbol{\gamma}}
\newcommand{\bGamma}{\boldsymbol{\Gamma}}

%%%%%%%%%%%%%%%%%%%%%%%%%%%%%%%%%%%%%%%%%%%%%
%         Bold numbers
%%%%%%%%%%%%%%%%%%%%%%%%%%%%%%%%%%%%%%%%%%%%%
\newcommand{\bzero}{\mathbf{0}}

%%%%%%%%%%%%%%%%%%%%%%%%%%%%%%%%%%%%%%%%%%%%%
% Projective-Like Pointclasses
%%%%%%%%%%%%%%%%%%%%%%%%%%%%%%%%%%%%%%%%%%%%%
\newcommand{\Sa}[2][\alpha]{\Sigma_{(#1,#2)}}
\newcommand{\Pa}[2][\alpha]{\Pi_{(#1,#2)}}
\newcommand{\Da}[2][\alpha]{\Delta_{(#1,#2)}}
\newcommand{\Aa}[2][\alpha]{A_{(#1,#2)}}
\newcommand{\Ca}[2][\alpha]{C_{(#1,#2)}}
\newcommand{\Qa}[2][\alpha]{Q_{(#1,#2)}}
\newcommand{\da}[2][\alpha]{\delta_{(#1,#2)}}
\newcommand{\leqa}[2][\alpha]{\leq_{(#1,#2)}}
\newcommand{\lessa}[2][\alpha]{<_{(#1,#2)}}
\newcommand{\equiva}[2][\alpha]{\equiv_{(#1,#2)}}


\newcommand{\Sl}[1]{\Sa[\lambda]{#1}}
\newcommand{\Pl}[1]{\Pa[\lambda]{#1}}
\newcommand{\Dl}[1]{\Da[\lambda]{#1}}
\newcommand{\Al}[1]{\Aa[\lambda]{#1}}
\newcommand{\Cl}[1]{\Ca[\lambda]{#1}}
\newcommand{\Ql}[1]{\Qa[\lambda]{#1}}

\newcommand{\San}{\Sa{n}}
\newcommand{\Pan}{\Pa{n}}
\newcommand{\Dan}{\Da{n}}
\newcommand{\Can}{\Ca{n}}
\newcommand{\Qan}{\Qa{n}}
\newcommand{\Aan}{\Aa{n}}
\newcommand{\dan}{\da{n}}
\newcommand{\leqan}{\leqa{n}}
\newcommand{\lessan}{\lessa{n}}
\newcommand{\equivan}{\equiva{n}}

\newcommand{\SigmaOneOmega}{\Sigma^1_{\omega}}
\newcommand{\SigmaOneOmegaPlusOne}{\Sigma^1_{\omega+1}}
\newcommand{\PiOneOmega}{\Pi^1_{\omega}}
\newcommand{\PiOneOmegaPlusOne}{\Pi^1_{\omega+1}}
\newcommand{\DeltaOneOmegaPlusOne}{\Delta^1_{\omega+1}}

%%%%%%%%%%%%%%%%%%%%%%%%%%%%%%%%%%%%%%%%%%%%%
% Linear Algebra
%%%%%%%%%%%%%%%%%%%%%%%%%%%%%%%%%%%%%%%%%%%%%
\newcommand{\transpose}[1]{{#1}^{\text{T}}}
\newcommand{\norm}[1]{\lVert{#1}\rVert}
\newcommand\aug{\fboxsep=-\fboxrule\!\!\!\fbox{\strut}\!\!\!}

%%%%%%%%%%%%%%%%%%%%%%%%%%%%%%%%%%%%%%%%%%%%%
% Number Theory
%%%%%%%%%%%%%%%%%%%%%%%%%%%%%%%%%%%%%%%%%%%%%
\newcommand{\av}[1]{\lvert#1\rvert}
\DeclareMathOperator{\divides}{\mid}
\DeclareMathOperator{\ndivides}{\nmid}
\DeclareMathOperator{\lcm}{lcm}
\DeclareMathOperator{\sign}{sign}
\newcommand{\floor}[1]{\left\lfloor{#1}\right\rfloor}
\DeclareMathOperator{\ConCl}{CC}
\DeclareMathOperator{\ord}{ord}


%%%%%%%%%%%%%%%%%%%%%%%%%%%%%%%%%%%%%%%%%%%%%%%%%%%%%%%%%%%%%%%%%%%%%%%%%%%
%%  Theorem-Like Declarations
%%%%%%%%%%%%%%%%%%%%%%%%%%%%%%%%%%%%%%%%%%%%%%%%%%%%%%%%%%%%%%%%%%%%%%%%%%

\newtheorem{theorem}{Theorem}[section]
\newtheorem{lemma}[theorem]{Lemma}
\newtheorem{corollary}[theorem]{Corollary}
\newtheorem{proposition}[theorem]{Proposition}


\theoremstyle{definition}

\newtheorem{definition}[theorem]{Definition}
\newtheorem{conjecture}[theorem]{Conjecture}
\newtheorem{remark}[theorem]{Remark}
\newtheorem{remarks}[theorem]{Remarks}
\newtheorem{notation}[theorem]{Notation}

\theoremstyle{remark}

\newtheorem*{note}{Note}
\newtheorem*{warning}{Warning}
\newtheorem*{question}{Question}
\newtheorem*{example}{Example}
\newtheorem*{fact}{Fact}


\newenvironment*{subproof}[1][Proof]
{\begin{proof}[#1]}{\renewcommand{\qedsymbol}{$\diamondsuit$} \end{proof}}

\newenvironment*{case}[1]
{\textbf{Case #1.  }\itshape }{}

\newenvironment*{claim}[1][Claim]
{\textbf{#1.  }\itshape }{}


\pagestyle{plain}

\begin{document}

\title{Review of Beginning Inner Model Theory by Bill Mitchell}

\maketitle

The program of inner model theory is to define and study the properties of
canonical models of large cardinal axioms. The minimal and historically first
inner model is $L$, G\"{o}del's constructible universe, and this model serves
as the touchstone for what a ``canonical" model should be like. The smallest
large cardinals, for example weakly compact cardinals, can consistently exist
in $L$ and so $L$ may be considered the canonical inner model for these small
large cardinal axioms. But stronger large cardinal axioms are inconsistent
with ``$V=L$" and hence the program to find larger inner models is pursued.

We mention some of the properties of $L$ that contribute to it being considered
canonical. $L$ is a model of ZFC + GCH and the reason that AC and GCH are true
in $L$ can be explained by \emph{structural} properties of $L$. The axiom of choice
is true in $L$ because $L$ has a natural simply-definable global wellordering.
The GCH is true in $L$ because, working in $L$, if $\kappa\geq\omega$ is a cardinal
and $A\subseteq\kappa$  then $\exists \alpha<\kappa^{+}$ s.t.
$A\in L_{\alpha}$ (and $|L_{\alpha}| \leq \kappa$).
We will refer to this property as \emph{condensation} (though usually that term refers
to a different structural property that implies this property.)
$L$ is unique: If $M$ is a transitive proper class and $M\models$``$V=L$" then $M=L$.

The chapter under review begins in section 1.1 with a review of the theory of $L$
and goes on in later sections to describe, in varying levels of details, inner
models for progressively stronger large cardinal axioms: one measurable cardinal,
many measurable cardinals of higher Mitchell order, a strong cardinal. The theory
of the the last two models is given in some detail. Besides this main thread,
other important themes in inner model theory are intertwined in the narrative
of the chapter but in less detail: sharps, the covering lemma and the core model.
Besides describing the ideas mentioned above another stated goal of the chapter
is to prepare the reader for the next three chapters
in the Handbook which extend the conversation to include more
modern variants of the theory with stronger large cardinal axioms (Woodin cardinals
and beyond) and which describe the core model and covering theorems in more detail.
Bill Mitchell, the author of the chapter, is recognized as one of the founding
fathers of inner model theory. For this reason alone it is a pleasure to read
this chapter.

In the remainder of this review we will give some more details about the type of
material described in the chapter and we will point out a small error in one
of the definitions and give a correction.

Historically the next inner model to be studied after $L$, and the subject
of section 1.2 of the chapter, was $L[U]$, the canonical inner model
for a measurable cardinal. Here $U$ is a normal measure in $L[U]$ on some
$\kappa$ that is a measurable cardinal in $L[U]$. All of the canonical
properties of $L$ mentioned above generalize to $L[U]$ with some small provisos.
If $U^{\prime}$ is a normal measure in $L[U^{\prime}]$ on \emph{the same} $\kappa$
then $U^{\prime}=U$. Also $U$ is the only normal measure in $L[U]$ and $\kappa$ is the
only measurable cardinal in $L[U]$. $L[U]$ is not quite as unique as $L$ in that there
are other $L[U^{\prime}]$ models at other $\kappa^{\prime}$'s. But given any two such models one is
elementarily embeddable into the other. $L[U]$ is a model of GCH but the condensation
property is not true below $\kappa$ if we use the $L_{\alpha}[U]$ as levels.
This is becuase for $\alpha\leq\kappa$, $U\intersect L_{\alpha}[U]=\emptyset$ and
so $L_{\alpha}[U]=L_{\alpha}$ and so it is not reasonable to expect that
$L_{\alpha}[U]$ contains enough sets for condensation to hold. But there is a more
modern organization of the model in which partial information about $U$ is
used as a predicate in constructing the earlier levels of the model and using
this presentation we recover condensation. $L[U]$ has a definable wellorder but its
definition is more complex than the one for $L$. Restricted to reals the wellorder
for $L$ is $\Delta^1_2$ and the one for $L[U]$ is $\Delta^1_3$.

Historically the next step in the progression we are outlining, and the subject
of section 2, is the models of the form
$L[\vec{U}]$ where $\vec{U}$ is a coherent sequence of normal measures in
$L[\vec{U}]$. These models were introduced and studied by Bill Mitchell, the
author of the chapter under review. These models capture a class of large cardinal properties
stronger than a measurable limit of measurable cardinals, namely the property that
$\kappa$ is a measurable cardinal of higher Mitchell order.

If $U_1$ and $U_2$ are normal measures on $\kappa$ then $U_1 \vartriangleleft U_2$ iff
$U_1 \in \text{Ult}(V, U_2)$. $\vartriangleleft$ is wellfounded and $o(U)$ denotes the
rank of $U$ under this partial order. $o(\kappa) = \text{sup}
\setof{o(U) + 1}{U \text{ is a normal measure on } \kappa}$  is called
the Mitchell order of $\kappa$. So  $o(U) = 0$ if $U$
concentrates on non-measurable cardinals, $o(\kappa) = 1$ if $\kappa$ is measurable
but doesn't have a $U$ that concentrates on measurables, $o(U) = 1$ if $U$
concentrates on cardinals $\alpha$ with $o(\alpha) = 1$ and $o(\kappa) = 2$ if
$\kappa$ admits a $U$ with $o(U) = 1$ but not a $U$ that concentrates on cardinals
$\alpha$ that admit a measure of order 1. A sequence of measures $\vec{U}$ is
called \emph{coherent} if the measures are organized in a certain precise way
implying, among other things, that the measures on a fixed $\kappa$ are ordered
increasingly under $\vartriangleleft$.

Models of the form $L[\vec{U}]$ can satisfy
that there exists many measurable cardinals with many normal measures and
with Mitchell order up to $o(\kappa)=\kappa^{++}$. All of the canonical properties
of $L$ generalize to $L[\vec{U}]$ with some more provisos.
$L[\vec{U}]$ is a model of GCH but the situation with condensation is similar
to what was discussed above for $L[U]$. $L[\vec{U}]$ has a definable wellorder
which, when restricted to reals, is $\Delta^1_3$. (Many measurable cardinals is
not enough to increase the complexity of the wellorder beyond $\Delta^1_3$. That
does not occur until we get past the model with one Woodin cardinal.) As for uniqueness,
the only normal measures and the only measurable cardinals in $L[\vec{U}]$ are
those on the sequence $\vec{U}$. If $\vec{U_1}$ and $\vec{U_2}$ have the same
domain (meaning the same set of measurable cardinals and the same number
of measures on each cardinal), we would like to conclude that
$L[\vec{U_1}] = L[\vec{U_2}]$. This is true if the lengths of the sequences
are definable but it turns out that the hypothesis
$L[\vec{U_i}]\models \text{``}\vec{U_i} \text{ is coherent"}$
is not sufficient for the result in general. It is sufficient to assume that the $\vec{U_i}$ consist
of measures in $V$ (not just in the inner model) and that in $V$ the sequences satisfy a property called
\emph{weak coherence}. With this assumption it is true that $L[\vec{U_1}] = L[\vec{U_2}]$
and the restrictions of the two sequences to the inner model are equal.

But the definition of weakly coherent given in the chapter is not correct.
In Definition 2.6 on page 1466 the definition of a \emph{coherent} sequence $\vec{U}$
is given as
\begin{enumerate}
\item $\dom(\vec{U}) = \setof{(\kappa, \beta)}{\kappa < \length(\vec{U}) \AND \beta < o^{\vec{U}}(\kappa)}$,
where $\length(\vec{U})$ and  $o^{\vec{U}}(\kappa)$ are some ordinals.
\item If $(\kappa, \beta) \in \dom(\vec{U})$ then $\vec{U}(\kappa, \beta)$ is a normal measure on $\kappa$.
\item If $U=\vec{U}(\kappa, \beta)$ then $o^{i^{U}(\vec{U})}(\kappa) = \beta$ and $i^{U}(\vec{U})(\kappa, \betaprime)=
\vec{U}(\kappa, \betaprime)$ for all $\betaprime<\beta$, where $i^{U}$ is the ultrapower embedding by $U$.
\end{enumerate}


At the bottom of page 1469 a sequence $\vec{U}$ is defined to be \emph{weakly coherent}
if condition 3 above is replaced by the condition: If $U=\vec{U}(\kappa, \beta)$  then $o^{V}(U) = \beta$.

This condition is not sufficient for the purposes that the author will use it. For example Lemma 2.10 on
page 1470 states: Suppose that $\vec{U}$ and $\vec{W}$ are weakly coherent sequences of measures in $V$ with
the same domain. Then $L[\vec{U}] = L[\vec{W}]$ and $\vec{U}(\kappa,\beta)\intersect L[\vec{U}] =
\vec{W}(\kappa,\beta)\intersect L[\vec{W}]$ for every $(\kappa,\beta)$ in their common domain.

A simple example suffices to see that this fails using the definition of weakly coherent given above. Suppose
$o(\kappa) = 2$ and there exists two different normal measures on $\kappa$ of order 1,
$U_1\not=W_1$. Let $A\subset \kappa$ be in $U_1$ but
not in $W_1$ and consist only of measurable cardinals. There is such an $A$ since both $U_1$ and $W_1$ concentrate
on measurable cardinals.
 Let $\vec{U}$ and $\vec{W}$ be sequences of measures with common domain
$\setof{(\alpha, 0)}{\alpha \in A} \union \singleton{(\kappa, 0), (\kappa, 1)}$. Let
$\vec{U}(\alpha, 0) = \vec{W}(\alpha, 0)$ be any measure of order zero on $\alpha$ for $\alpha \in A$.
Let $\vec{U}(\kappa, 0) = \vec{W}(\kappa, 0)$ be any measure of order zero on $\kappa$. Finally
let $\vec{U}(\kappa, 1) = U_1$ and $\vec{W}(\kappa, 1) = W_1$. Then $\vec{U}$ and $\vec{W}$ satisfy
the definition of weakly coherent given in the chapter but, since $A\in L[\vec{U}]\intersect L[\vec{W}]$,
$U_1\intersect L[\vec{U}] \not= W_1\intersect L[\vec{W}]$.

A correct definition of weakly coherent may be found in Mitchell's own earlier \cite{Mitchell-Revisited}.
There a sequence $\vec{U}$ is defined to be weakly coherent
if condition 3 above is replaced by just the first half of condition 3:
If $U=\vec{U}(\kappa, \beta)$ then $o^{i^{U}(\vec{U})}(\kappa) = \beta$.
It is shown in \cite{Mitchell-Revisited} that this
condition suffices for the proof of Lemma 2.10.
Some authors use the notation
$o^{\vec{U}}(U) = \beta$ for this condition (but the chapter under review does not introduce this notation.)
Thus the correction may be construed as consisting of changing a single symbol:
For weak coherence, instead of $o^V(U) = \beta$ we want $o^{\vec{U}}(U) = \beta$. The reason our
counterexample above fails to satisfy this correct definition of weakly coherent is that although
$o^{V}(W_1) = 1$, $o^{\vec{W}}(W_1) = 0$.

In \cite{Mitchell-Revisited} it is pointed out that an example of a weakly coherent sequence
$\vec{U}$ (in the correct sense) is given by letting, for every measurable cardinal $\alpha$ and every $\beta<o(\alpha)$,
$\vec{U}(\alpha,\beta)$ be any normal measure on $\alpha$ of order $\beta$. Notice that in
this example $o^{\vec{U}}(\alpha) = o^V(\alpha)$ for every $\alpha$ and so this example does
actually satisfy the definition of weak coherence given in the chapter. The reason that our
counterexample above differs from this example is that the set $A$ did not contain all measurable
cardinals below $\kappa$.

The motivation for considering weakly coherent sequences is the following problem:
Given a measurable cardinal $\kappa$ of high Mitchell order in $V$, construct a sequence $\vec{U}$ such that
in $L[\vec{U}]$ $U$ is coherent and $\kappa$ has high Mitchell order. In the case of
a single measurable cardinal the analogous problem was easy. If $\kappa$ is measureable let $U$ be a normal
measure on $\kappa$ in $V$. Then $U\intersect L[U]$ is a normal measure on $\kappa$ in
$L[U]$. But for $\beta>=\omega$ it is unknown whether the hypothesis $o(\kappa) = \beta$
implies that there is a coherent $\vec{U}$ such $o^{\vec{U}}(\kappa) = \beta$ in $V$.
But the example from the previous paragraph shows that it is possible to construct a \emph{weakly} coherent
sequence $\vec{U}$ with $o^{\vec{U}}(\kappa) = \beta$.
This explains the motivation behind Corollary 2.11 of the chapter which says that either
this $\vec{U}$ is coherent in $L[\vec{U}]$ or else there is an inner model
of $\exists \kappa (o(\kappa) = \kappa^{++})$.

The subject of section 3 of the chapter, and approximately the next step historically, is the
class of models of the form $L[\vec{E}]$ where $\vec{E}$ is a coherent sequence of
extenders. With the particular organization described in the chapter an $L[\vec{E}]$
model can satisfy large cardinal hypotheses up to and including ``there exists
one strong cardinal." A cardinal $\kappa$ is $\beta$-\emph{strong} if there exists
an elementary embedding $j:V\map M$ with critical point $\kappa$ such that
$\beta<j(\kappa)$ and $V_{\kappa+\beta}\subset M$. $\kappa$ is \emph{strong} iff it is
$\beta$-strong for all $\beta$. So $\kappa$ is 1-strong iff $\kappa$ is
measurable, but the previously considered models of the form $L[\vec{U}]$ cannot have
even a 2-strong cardinal. An \emph{extender} is a generalization of a normal measure
that is able to represent these stronger embeddings. The $L[\vec{E}]$ models described
in the chapter are organized in a way very similar to the the $L[\vec{U}]$ models
but using extenders instead of measures. Whereas in $L[\vec{U}]$
$o(\kappa) <= \kappa^{++}$ was necessary, in $L[\vec{E}]$ there is no necessary
bound on $o(\kappa)$. Mitchell shows that if there is a strong cardinal in V then
there is an $L[\vec{E}]$ with a $\kappa$ with $o(\kappa) = \text{ON}$ and this $\kappa$
is a strong cardinal in $L[\vec{E}]$.

All of the canonical properties
of $L$ generalize to $L[\vec{E}]$ with more provisos.
$L[\vec{E}]$ is a model of GCH but the situation with condensation is similar
to what was discussed above. $L[\vec{E}]$ has a definable wellorder
which, when restricted to reals, is $\Delta^1_3$. (We are still below the
level of a Woodin cardinal here.) The situation with uniqueness is similar
to what was described for $L[\vec{U}]$.

Modern inner model theory studies inner models for large cardinal axioms that
are stronger than those described in the chapter: Woodin cardinals and beyond.
Today's models are also referred to as being of the form $L[\vec{E}]$ where $\vec{E}$
is a sequence of extenders. But the models are organized in a different way that
allow them to contain cardinals beyond one strong cardinal. One of the most significant
differences between the models described in the chapter and modern models is that
modern models require fine-structure theory even to define them whereas the theory
that Mitchell describes in the chapter avoids fine-structure entirely and is therefore
easier to access. One advantage to the modern fine-structural technique is
that modern models do in fact satisfy condensation. Section 4 of the chapter is
a brief discussion of modern inner model theory.

There is another important thread in inner model theory that gets woven together with
the one we have been describing: sharps, the core model and the covering theorem.
Mitchell repeatedly refers to these themes in the chapter and gives small hints about
the ideas involved but never gives much detail. A couple of the chapters in the
Handbook that follow the chapter under review do explore these concepts in depth.
The final section of the chapter, section 5, is entitled ``What is the Core Model?''
and offers a short philisophical future-looking discussion of the topic.
We close this review with a short discussion about these ideas.

Jensen's covering theorem for $L$ states that if $0^{\#}$ doesn't exist then every
uncountable set $X$ of ordinals can be \emph{covered} by a set in $L$. This means
that there is a $Y\in L$ with $|Y| = |X|$. An important consequence is called
\emph{weak covering}: If $\lambda$ is a singular cardinal then $(\lambda^{+})^L=\lambda^{+}$.

The most important application of covering is as a tool to derive large cardinal
strength from propositions that on their face do not involve large cardinals.
For example we can use covering to derive large cardinal strength from the failure of the
Singular Cardinal Hypothesis. This is because given the GCH in $L$, covering
for $L$ implies the SCH in $V$. We can turn this around: Failure of SCH implies
that $0^{\#}$ exists (because covering for $L$ must fail.) Thus we were able to
obtain large cardinal strength (the existence of $0^{\#}$) from a proposition that
does not literally entail large cardinals (the failure of the SCH.)

It is natural to ask whether more large cardinal strength can be obtained from
the failure of SCH using a similar tecnhique. A natural strategy would be
to try to extend the covering theorem to a larger inner model like $L[U]$.
The way to extend the notion of $0^{\#}$ is straightforward: We say the sharp for
$L[U]$ exists iff there is a proper class of indiscernibles for $L[U]$. A natural
target might be to try to show that if $L[U]$ exists but
its sharp does not exist then covering for $L[U]$ holds. This turns out to be
true with a weakened version of covering. This gives us that failure of SCH
implies that if $L[U]$ exists then its sharp exists.

But how do we know that $L[U]$ exists? In the previous
discussion we knew that $L[U]$ existed because there was a measurable cardinal
in $V$. But the whole point of our current endeavor is that we are starting
with a hypothesis that does not outright imply large cardinals in $V$.

The idea of the core model is, without assuming large cardinals in $V$, to
construct inner models exhibiting large cardinal strength. Furthermore, the
idea is to construct an inner model with ``all of the large cardinal strength
that exists in $V$'' and then show that this model is close enough to $V$ that
a version of the covering theorem holds for it. One way of making sense
of ``all of the large cardinal strength that exists in $V$'' is the following:
If you have constructed a model $M$ and its sharp exists then you are not done:
adjoin the sharp and keep going. So the core model should be the weakest inner
model whose sharp does not exist.

The Dodd-Jensen core model $K^{\text{DJ}}$ was designed to fill the gap between $L$ and
$L[U]$. The idea is to assume that $L[U]$ does not exist and then let $K^{\text{DJ}}$
be the model that contains all the approximations to $L[U]$ which do exist.
Dodd and Jensen show that assuming $L[U]$ does not exist that full covering
holds for $K^{\text{DJ}}$. This gives us that failure of SCH first implies
that $L[U]$ exists and then by the previous idea that its sharp exists.

These ideas can be extended to the higher models $L[\vec{U}]$, $L[\vec{E}]$
and beyond. In each case we define the appropriate notions of sharps and
the core model and prove a version of covering. As the models get stronger
the version of covering that can be proved gets weaker. For our specific
running example of the failure of SCH the game runs out at the level of
$L[\vec{U}]$ and does not make it to $L[\vec{E}]$. The failure of SCH turns
out to be equiconsistent with the existense of a measurable cardinal $\kappa$
such that $o(\kappa) = \kappa^{++}$. But these same techniques may be used
with other propositions besides failure of SCH to get large cardinal strength in the realm of
$L[\vec{E}]$ and beyond.

\bibliographystyle{amsalpha}

\bibliography{math}

\end{document}
